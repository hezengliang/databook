
% Default to the notebook output style

    


% Inherit from the specified cell style.




    
\documentclass[11pt]{article}

    
    
    \usepackage[T1]{fontenc}
    % Nicer default font (+ math font) than Computer Modern for most use cases
    \usepackage{mathpazo}

    % Basic figure setup, for now with no caption control since it's done
    % automatically by Pandoc (which extracts ![](path) syntax from Markdown).
    \usepackage{graphicx}
    % We will generate all images so they have a width \maxwidth. This means
    % that they will get their normal width if they fit onto the page, but
    % are scaled down if they would overflow the margins.
    \makeatletter
    \def\maxwidth{\ifdim\Gin@nat@width>\linewidth\linewidth
    \else\Gin@nat@width\fi}
    \makeatother
    \let\Oldincludegraphics\includegraphics
    % Set max figure width to be 80% of text width, for now hardcoded.
    \renewcommand{\includegraphics}[1]{\Oldincludegraphics[width=.8\maxwidth]{#1}}
    % Ensure that by default, figures have no caption (until we provide a
    % proper Figure object with a Caption API and a way to capture that
    % in the conversion process - todo).
    \usepackage{caption}
    \DeclareCaptionLabelFormat{nolabel}{}
    \captionsetup{labelformat=nolabel}

    \usepackage{adjustbox} % Used to constrain images to a maximum size 
    \usepackage{xcolor} % Allow colors to be defined
    \usepackage{enumerate} % Needed for markdown enumerations to work
    \usepackage{geometry} % Used to adjust the document margins
    \usepackage{amsmath} % Equations
    \usepackage{amssymb} % Equations
    \usepackage{textcomp} % defines textquotesingle
    % Hack from http://tex.stackexchange.com/a/47451/13684:
    \AtBeginDocument{%
        \def\PYZsq{\textquotesingle}% Upright quotes in Pygmentized code
    }
    \usepackage{upquote} % Upright quotes for verbatim code
    \usepackage{eurosym} % defines \euro
    \usepackage[mathletters]{ucs} % Extended unicode (utf-8) support
    \usepackage[utf8x]{inputenc} % Allow utf-8 characters in the tex document
    \usepackage{fancyvrb} % verbatim replacement that allows latex
    \usepackage{grffile} % extends the file name processing of package graphics 
                         % to support a larger range 
    % The hyperref package gives us a pdf with properly built
    % internal navigation ('pdf bookmarks' for the table of contents,
    % internal cross-reference links, web links for URLs, etc.)
    \usepackage{hyperref}
    \usepackage{longtable} % longtable support required by pandoc >1.10
    \usepackage{booktabs}  % table support for pandoc > 1.12.2
    \usepackage[inline]{enumitem} % IRkernel/repr support (it uses the enumerate* environment)
    \usepackage[normalem]{ulem} % ulem is needed to support strikethroughs (\sout)
                                % normalem makes italics be italics, not underlines
    

    
    
    % Colors for the hyperref package
    \definecolor{urlcolor}{rgb}{0,.145,.698}
    \definecolor{linkcolor}{rgb}{.71,0.21,0.01}
    \definecolor{citecolor}{rgb}{.12,.54,.11}

    % ANSI colors
    \definecolor{ansi-black}{HTML}{3E424D}
    \definecolor{ansi-black-intense}{HTML}{282C36}
    \definecolor{ansi-red}{HTML}{E75C58}
    \definecolor{ansi-red-intense}{HTML}{B22B31}
    \definecolor{ansi-green}{HTML}{00A250}
    \definecolor{ansi-green-intense}{HTML}{007427}
    \definecolor{ansi-yellow}{HTML}{DDB62B}
    \definecolor{ansi-yellow-intense}{HTML}{B27D12}
    \definecolor{ansi-blue}{HTML}{208FFB}
    \definecolor{ansi-blue-intense}{HTML}{0065CA}
    \definecolor{ansi-magenta}{HTML}{D160C4}
    \definecolor{ansi-magenta-intense}{HTML}{A03196}
    \definecolor{ansi-cyan}{HTML}{60C6C8}
    \definecolor{ansi-cyan-intense}{HTML}{258F8F}
    \definecolor{ansi-white}{HTML}{C5C1B4}
    \definecolor{ansi-white-intense}{HTML}{A1A6B2}

    % commands and environments needed by pandoc snippets
    % extracted from the output of `pandoc -s`
    \providecommand{\tightlist}{%
      \setlength{\itemsep}{0pt}\setlength{\parskip}{0pt}}
    \DefineVerbatimEnvironment{Highlighting}{Verbatim}{commandchars=\\\{\}}
    % Add ',fontsize=\small' for more characters per line
    \newenvironment{Shaded}{}{}
    \newcommand{\KeywordTok}[1]{\textcolor[rgb]{0.00,0.44,0.13}{\textbf{{#1}}}}
    \newcommand{\DataTypeTok}[1]{\textcolor[rgb]{0.56,0.13,0.00}{{#1}}}
    \newcommand{\DecValTok}[1]{\textcolor[rgb]{0.25,0.63,0.44}{{#1}}}
    \newcommand{\BaseNTok}[1]{\textcolor[rgb]{0.25,0.63,0.44}{{#1}}}
    \newcommand{\FloatTok}[1]{\textcolor[rgb]{0.25,0.63,0.44}{{#1}}}
    \newcommand{\CharTok}[1]{\textcolor[rgb]{0.25,0.44,0.63}{{#1}}}
    \newcommand{\StringTok}[1]{\textcolor[rgb]{0.25,0.44,0.63}{{#1}}}
    \newcommand{\CommentTok}[1]{\textcolor[rgb]{0.38,0.63,0.69}{\textit{{#1}}}}
    \newcommand{\OtherTok}[1]{\textcolor[rgb]{0.00,0.44,0.13}{{#1}}}
    \newcommand{\AlertTok}[1]{\textcolor[rgb]{1.00,0.00,0.00}{\textbf{{#1}}}}
    \newcommand{\FunctionTok}[1]{\textcolor[rgb]{0.02,0.16,0.49}{{#1}}}
    \newcommand{\RegionMarkerTok}[1]{{#1}}
    \newcommand{\ErrorTok}[1]{\textcolor[rgb]{1.00,0.00,0.00}{\textbf{{#1}}}}
    \newcommand{\NormalTok}[1]{{#1}}
    
    % Additional commands for more recent versions of Pandoc
    \newcommand{\ConstantTok}[1]{\textcolor[rgb]{0.53,0.00,0.00}{{#1}}}
    \newcommand{\SpecialCharTok}[1]{\textcolor[rgb]{0.25,0.44,0.63}{{#1}}}
    \newcommand{\VerbatimStringTok}[1]{\textcolor[rgb]{0.25,0.44,0.63}{{#1}}}
    \newcommand{\SpecialStringTok}[1]{\textcolor[rgb]{0.73,0.40,0.53}{{#1}}}
    \newcommand{\ImportTok}[1]{{#1}}
    \newcommand{\DocumentationTok}[1]{\textcolor[rgb]{0.73,0.13,0.13}{\textit{{#1}}}}
    \newcommand{\AnnotationTok}[1]{\textcolor[rgb]{0.38,0.63,0.69}{\textbf{\textit{{#1}}}}}
    \newcommand{\CommentVarTok}[1]{\textcolor[rgb]{0.38,0.63,0.69}{\textbf{\textit{{#1}}}}}
    \newcommand{\VariableTok}[1]{\textcolor[rgb]{0.10,0.09,0.49}{{#1}}}
    \newcommand{\ControlFlowTok}[1]{\textcolor[rgb]{0.00,0.44,0.13}{\textbf{{#1}}}}
    \newcommand{\OperatorTok}[1]{\textcolor[rgb]{0.40,0.40,0.40}{{#1}}}
    \newcommand{\BuiltInTok}[1]{{#1}}
    \newcommand{\ExtensionTok}[1]{{#1}}
    \newcommand{\PreprocessorTok}[1]{\textcolor[rgb]{0.74,0.48,0.00}{{#1}}}
    \newcommand{\AttributeTok}[1]{\textcolor[rgb]{0.49,0.56,0.16}{{#1}}}
    \newcommand{\InformationTok}[1]{\textcolor[rgb]{0.38,0.63,0.69}{\textbf{\textit{{#1}}}}}
    \newcommand{\WarningTok}[1]{\textcolor[rgb]{0.38,0.63,0.69}{\textbf{\textit{{#1}}}}}
    
    
    % Define a nice break command that doesn't care if a line doesn't already
    % exist.
    \def\br{\hspace*{\fill} \\* }
    % Math Jax compatability definitions
    \def\gt{>}
    \def\lt{<}
    % Document parameters
    \title{smopy\_about}
    
    
    

    % Pygments definitions
    
\makeatletter
\def\PY@reset{\let\PY@it=\relax \let\PY@bf=\relax%
    \let\PY@ul=\relax \let\PY@tc=\relax%
    \let\PY@bc=\relax \let\PY@ff=\relax}
\def\PY@tok#1{\csname PY@tok@#1\endcsname}
\def\PY@toks#1+{\ifx\relax#1\empty\else%
    \PY@tok{#1}\expandafter\PY@toks\fi}
\def\PY@do#1{\PY@bc{\PY@tc{\PY@ul{%
    \PY@it{\PY@bf{\PY@ff{#1}}}}}}}
\def\PY#1#2{\PY@reset\PY@toks#1+\relax+\PY@do{#2}}

\expandafter\def\csname PY@tok@w\endcsname{\def\PY@tc##1{\textcolor[rgb]{0.73,0.73,0.73}{##1}}}
\expandafter\def\csname PY@tok@c\endcsname{\let\PY@it=\textit\def\PY@tc##1{\textcolor[rgb]{0.25,0.50,0.50}{##1}}}
\expandafter\def\csname PY@tok@cp\endcsname{\def\PY@tc##1{\textcolor[rgb]{0.74,0.48,0.00}{##1}}}
\expandafter\def\csname PY@tok@k\endcsname{\let\PY@bf=\textbf\def\PY@tc##1{\textcolor[rgb]{0.00,0.50,0.00}{##1}}}
\expandafter\def\csname PY@tok@kp\endcsname{\def\PY@tc##1{\textcolor[rgb]{0.00,0.50,0.00}{##1}}}
\expandafter\def\csname PY@tok@kt\endcsname{\def\PY@tc##1{\textcolor[rgb]{0.69,0.00,0.25}{##1}}}
\expandafter\def\csname PY@tok@o\endcsname{\def\PY@tc##1{\textcolor[rgb]{0.40,0.40,0.40}{##1}}}
\expandafter\def\csname PY@tok@ow\endcsname{\let\PY@bf=\textbf\def\PY@tc##1{\textcolor[rgb]{0.67,0.13,1.00}{##1}}}
\expandafter\def\csname PY@tok@nb\endcsname{\def\PY@tc##1{\textcolor[rgb]{0.00,0.50,0.00}{##1}}}
\expandafter\def\csname PY@tok@nf\endcsname{\def\PY@tc##1{\textcolor[rgb]{0.00,0.00,1.00}{##1}}}
\expandafter\def\csname PY@tok@nc\endcsname{\let\PY@bf=\textbf\def\PY@tc##1{\textcolor[rgb]{0.00,0.00,1.00}{##1}}}
\expandafter\def\csname PY@tok@nn\endcsname{\let\PY@bf=\textbf\def\PY@tc##1{\textcolor[rgb]{0.00,0.00,1.00}{##1}}}
\expandafter\def\csname PY@tok@ne\endcsname{\let\PY@bf=\textbf\def\PY@tc##1{\textcolor[rgb]{0.82,0.25,0.23}{##1}}}
\expandafter\def\csname PY@tok@nv\endcsname{\def\PY@tc##1{\textcolor[rgb]{0.10,0.09,0.49}{##1}}}
\expandafter\def\csname PY@tok@no\endcsname{\def\PY@tc##1{\textcolor[rgb]{0.53,0.00,0.00}{##1}}}
\expandafter\def\csname PY@tok@nl\endcsname{\def\PY@tc##1{\textcolor[rgb]{0.63,0.63,0.00}{##1}}}
\expandafter\def\csname PY@tok@ni\endcsname{\let\PY@bf=\textbf\def\PY@tc##1{\textcolor[rgb]{0.60,0.60,0.60}{##1}}}
\expandafter\def\csname PY@tok@na\endcsname{\def\PY@tc##1{\textcolor[rgb]{0.49,0.56,0.16}{##1}}}
\expandafter\def\csname PY@tok@nt\endcsname{\let\PY@bf=\textbf\def\PY@tc##1{\textcolor[rgb]{0.00,0.50,0.00}{##1}}}
\expandafter\def\csname PY@tok@nd\endcsname{\def\PY@tc##1{\textcolor[rgb]{0.67,0.13,1.00}{##1}}}
\expandafter\def\csname PY@tok@s\endcsname{\def\PY@tc##1{\textcolor[rgb]{0.73,0.13,0.13}{##1}}}
\expandafter\def\csname PY@tok@sd\endcsname{\let\PY@it=\textit\def\PY@tc##1{\textcolor[rgb]{0.73,0.13,0.13}{##1}}}
\expandafter\def\csname PY@tok@si\endcsname{\let\PY@bf=\textbf\def\PY@tc##1{\textcolor[rgb]{0.73,0.40,0.53}{##1}}}
\expandafter\def\csname PY@tok@se\endcsname{\let\PY@bf=\textbf\def\PY@tc##1{\textcolor[rgb]{0.73,0.40,0.13}{##1}}}
\expandafter\def\csname PY@tok@sr\endcsname{\def\PY@tc##1{\textcolor[rgb]{0.73,0.40,0.53}{##1}}}
\expandafter\def\csname PY@tok@ss\endcsname{\def\PY@tc##1{\textcolor[rgb]{0.10,0.09,0.49}{##1}}}
\expandafter\def\csname PY@tok@sx\endcsname{\def\PY@tc##1{\textcolor[rgb]{0.00,0.50,0.00}{##1}}}
\expandafter\def\csname PY@tok@m\endcsname{\def\PY@tc##1{\textcolor[rgb]{0.40,0.40,0.40}{##1}}}
\expandafter\def\csname PY@tok@gh\endcsname{\let\PY@bf=\textbf\def\PY@tc##1{\textcolor[rgb]{0.00,0.00,0.50}{##1}}}
\expandafter\def\csname PY@tok@gu\endcsname{\let\PY@bf=\textbf\def\PY@tc##1{\textcolor[rgb]{0.50,0.00,0.50}{##1}}}
\expandafter\def\csname PY@tok@gd\endcsname{\def\PY@tc##1{\textcolor[rgb]{0.63,0.00,0.00}{##1}}}
\expandafter\def\csname PY@tok@gi\endcsname{\def\PY@tc##1{\textcolor[rgb]{0.00,0.63,0.00}{##1}}}
\expandafter\def\csname PY@tok@gr\endcsname{\def\PY@tc##1{\textcolor[rgb]{1.00,0.00,0.00}{##1}}}
\expandafter\def\csname PY@tok@ge\endcsname{\let\PY@it=\textit}
\expandafter\def\csname PY@tok@gs\endcsname{\let\PY@bf=\textbf}
\expandafter\def\csname PY@tok@gp\endcsname{\let\PY@bf=\textbf\def\PY@tc##1{\textcolor[rgb]{0.00,0.00,0.50}{##1}}}
\expandafter\def\csname PY@tok@go\endcsname{\def\PY@tc##1{\textcolor[rgb]{0.53,0.53,0.53}{##1}}}
\expandafter\def\csname PY@tok@gt\endcsname{\def\PY@tc##1{\textcolor[rgb]{0.00,0.27,0.87}{##1}}}
\expandafter\def\csname PY@tok@err\endcsname{\def\PY@bc##1{\setlength{\fboxsep}{0pt}\fcolorbox[rgb]{1.00,0.00,0.00}{1,1,1}{\strut ##1}}}
\expandafter\def\csname PY@tok@kc\endcsname{\let\PY@bf=\textbf\def\PY@tc##1{\textcolor[rgb]{0.00,0.50,0.00}{##1}}}
\expandafter\def\csname PY@tok@kd\endcsname{\let\PY@bf=\textbf\def\PY@tc##1{\textcolor[rgb]{0.00,0.50,0.00}{##1}}}
\expandafter\def\csname PY@tok@kn\endcsname{\let\PY@bf=\textbf\def\PY@tc##1{\textcolor[rgb]{0.00,0.50,0.00}{##1}}}
\expandafter\def\csname PY@tok@kr\endcsname{\let\PY@bf=\textbf\def\PY@tc##1{\textcolor[rgb]{0.00,0.50,0.00}{##1}}}
\expandafter\def\csname PY@tok@bp\endcsname{\def\PY@tc##1{\textcolor[rgb]{0.00,0.50,0.00}{##1}}}
\expandafter\def\csname PY@tok@fm\endcsname{\def\PY@tc##1{\textcolor[rgb]{0.00,0.00,1.00}{##1}}}
\expandafter\def\csname PY@tok@vc\endcsname{\def\PY@tc##1{\textcolor[rgb]{0.10,0.09,0.49}{##1}}}
\expandafter\def\csname PY@tok@vg\endcsname{\def\PY@tc##1{\textcolor[rgb]{0.10,0.09,0.49}{##1}}}
\expandafter\def\csname PY@tok@vi\endcsname{\def\PY@tc##1{\textcolor[rgb]{0.10,0.09,0.49}{##1}}}
\expandafter\def\csname PY@tok@vm\endcsname{\def\PY@tc##1{\textcolor[rgb]{0.10,0.09,0.49}{##1}}}
\expandafter\def\csname PY@tok@sa\endcsname{\def\PY@tc##1{\textcolor[rgb]{0.73,0.13,0.13}{##1}}}
\expandafter\def\csname PY@tok@sb\endcsname{\def\PY@tc##1{\textcolor[rgb]{0.73,0.13,0.13}{##1}}}
\expandafter\def\csname PY@tok@sc\endcsname{\def\PY@tc##1{\textcolor[rgb]{0.73,0.13,0.13}{##1}}}
\expandafter\def\csname PY@tok@dl\endcsname{\def\PY@tc##1{\textcolor[rgb]{0.73,0.13,0.13}{##1}}}
\expandafter\def\csname PY@tok@s2\endcsname{\def\PY@tc##1{\textcolor[rgb]{0.73,0.13,0.13}{##1}}}
\expandafter\def\csname PY@tok@sh\endcsname{\def\PY@tc##1{\textcolor[rgb]{0.73,0.13,0.13}{##1}}}
\expandafter\def\csname PY@tok@s1\endcsname{\def\PY@tc##1{\textcolor[rgb]{0.73,0.13,0.13}{##1}}}
\expandafter\def\csname PY@tok@mb\endcsname{\def\PY@tc##1{\textcolor[rgb]{0.40,0.40,0.40}{##1}}}
\expandafter\def\csname PY@tok@mf\endcsname{\def\PY@tc##1{\textcolor[rgb]{0.40,0.40,0.40}{##1}}}
\expandafter\def\csname PY@tok@mh\endcsname{\def\PY@tc##1{\textcolor[rgb]{0.40,0.40,0.40}{##1}}}
\expandafter\def\csname PY@tok@mi\endcsname{\def\PY@tc##1{\textcolor[rgb]{0.40,0.40,0.40}{##1}}}
\expandafter\def\csname PY@tok@il\endcsname{\def\PY@tc##1{\textcolor[rgb]{0.40,0.40,0.40}{##1}}}
\expandafter\def\csname PY@tok@mo\endcsname{\def\PY@tc##1{\textcolor[rgb]{0.40,0.40,0.40}{##1}}}
\expandafter\def\csname PY@tok@ch\endcsname{\let\PY@it=\textit\def\PY@tc##1{\textcolor[rgb]{0.25,0.50,0.50}{##1}}}
\expandafter\def\csname PY@tok@cm\endcsname{\let\PY@it=\textit\def\PY@tc##1{\textcolor[rgb]{0.25,0.50,0.50}{##1}}}
\expandafter\def\csname PY@tok@cpf\endcsname{\let\PY@it=\textit\def\PY@tc##1{\textcolor[rgb]{0.25,0.50,0.50}{##1}}}
\expandafter\def\csname PY@tok@c1\endcsname{\let\PY@it=\textit\def\PY@tc##1{\textcolor[rgb]{0.25,0.50,0.50}{##1}}}
\expandafter\def\csname PY@tok@cs\endcsname{\let\PY@it=\textit\def\PY@tc##1{\textcolor[rgb]{0.25,0.50,0.50}{##1}}}

\def\PYZbs{\char`\\}
\def\PYZus{\char`\_}
\def\PYZob{\char`\{}
\def\PYZcb{\char`\}}
\def\PYZca{\char`\^}
\def\PYZam{\char`\&}
\def\PYZlt{\char`\<}
\def\PYZgt{\char`\>}
\def\PYZsh{\char`\#}
\def\PYZpc{\char`\%}
\def\PYZdl{\char`\$}
\def\PYZhy{\char`\-}
\def\PYZsq{\char`\'}
\def\PYZdq{\char`\"}
\def\PYZti{\char`\~}
% for compatibility with earlier versions
\def\PYZat{@}
\def\PYZlb{[}
\def\PYZrb{]}
\makeatother


    % Exact colors from NB
    \definecolor{incolor}{rgb}{0.0, 0.0, 0.5}
    \definecolor{outcolor}{rgb}{0.545, 0.0, 0.0}



    
    % Prevent overflowing lines due to hard-to-break entities
    \sloppy 
    % Setup hyperref package
    \hypersetup{
      breaklinks=true,  % so long urls are correctly broken across lines
      colorlinks=true,
      urlcolor=urlcolor,
      linkcolor=linkcolor,
      citecolor=citecolor,
      }
    % Slightly bigger margins than the latex defaults
    
    \geometry{verbose,tmargin=1in,bmargin=1in,lmargin=1in,rmargin=1in}
    
    

    \begin{document}
    
    
    \maketitle
    
    

    
    \hypertarget{about-supermap-iobjects-for-python}{%
\section{About SuperMap iObjects for
python}\label{about-supermap-iobjects-for-python}}

\begin{itemize}
\tightlist
\item
  安装:https://pypi.org/project/iobjectspy/

  \begin{itemize}
  \tightlist
  \item
    pip install iobjectspy
  \end{itemize}
\item
  文档:http://iobjectspy.supermap.io/guide.html
\item
  下载:

  \begin{itemize}
  \tightlist
  \item
    iObjects
    Java,各版本,http://support.supermap.com.cn/product/iObjects.aspx
  \item
    iObjects
    Java,bin包,http://support.supermap.com.cn/DownloadCenter/DownloadPage.aspx?id=1112
  \item
    iObjects
    Java,绿色精简版,http://support.supermap.com.cn/DownloadCenter/DownloadPage.aspx?id=1104
  \end{itemize}
\end{itemize}

    \begin{Verbatim}[commandchars=\\\{\}]
{\color{incolor}In [{\color{incolor}1}]:} \PY{k+kn}{import} \PY{n+nn}{iobjectspy} \PY{k}{as} \PY{n+nn}{smo}
\end{Verbatim}


    \begin{Verbatim}[commandchars=\\\{\}]
{\color{incolor}In [{\color{incolor}2}]:} \PY{n}{help}\PY{p}{(}\PY{n}{smo}\PY{p}{)}
\end{Verbatim}


    \begin{Verbatim}[commandchars=\\\{\}]
Help on package iobjectspy:

NAME
    iobjectspy

PACKAGE CONTENTS
    \_jsuperpy (package)
    \_logger
    \_numpy
    \_pandas
    \_version
    ai (package)
    analyst
    conversion
    data
    enums
    env
    threeddesigner

SUBMODULES
    supermap

VERSION
    9.1.1.0

FILE
    /opt/conda/lib/python3.6/site-packages/iobjectspy/\_\_init\_\_.py



    \end{Verbatim}

    \begin{Verbatim}[commandchars=\\\{\}]
{\color{incolor}In [{\color{incolor}3}]:} \PY{n}{help}\PY{p}{(}\PY{n}{smo}\PY{o}{.}\PY{n}{ai}\PY{p}{)}
\end{Verbatim}


    \begin{Verbatim}[commandchars=\\\{\}]
Help on package iobjectspy.ai in iobjectspy:

NAME
    iobjectspy.ai

PACKAGE CONTENTS
    \_detection
    \_segmentation
    \_toolkit
    recognition
    utils

FILE
    /opt/conda/lib/python3.6/site-packages/iobjectspy/ai/\_\_init\_\_.py



    \end{Verbatim}

    \begin{Verbatim}[commandchars=\\\{\}]
{\color{incolor}In [{\color{incolor}4}]:} \PY{n}{help}\PY{p}{(}\PY{n}{smo}\PY{o}{.}\PY{n}{env}\PY{p}{)}
\end{Verbatim}


    \begin{Verbatim}[commandchars=\\\{\}]
Help on module iobjectspy.env in iobjectspy:

NAME
    iobjectspy.env

FUNCTIONS
    get\_omp\_num\_threads()
        获取并行计算所使用的线程数
        
        :rtype: int
    
    is\_auto\_close\_output\_datasource()
        是否自动关闭结果数据源对象。在处理数据或分析时,设置的结果数据源信息如果是程序自动打开的(即当前工作空间下不存在此数据源),默认情形下程序在
        完成单个功能后会自动关闭。用户可以通过设置 :py:meth:`set\_auto\_close\_output\_datasource` 使结果数据源不被自动关闭,这样,结果数据源将存在于当前的工作空间中。
        
        :rtype: bool
    
    is\_use\_analyst\_memory\_mode()
        空间分析是否使用内存模式
        
        :rtype: bool
    
    set\_analyst\_memory\_mode(is\_use\_memory)
        设置空间分析是否启用内存模式。
        
        :param bool is\_use\_memory: 启用内存模式设置 True , 否则设置为 False
    
    set\_auto\_close\_output\_datasource(auto\_close)
        设置是否关闭结果数据源对象。在处理数据或分析时,设置的结果数据源信息如果是程序自动打开的(不是用户调用打开数据源接口打开,即当前工作空间下不存在此数据源),默认情形下程序在
        完成单个功能后会自动关闭。用户可以通过此接口设置 auto\_close 为 False 使结果数据源不被自动关闭,这样,结果数据源将存在于当前的工作空间中。
        
        :param bool auto\_close: 是否自动关闭程序内部打开的数据源对象。
    
    set\_omp\_num\_threads(num\_threads)
        设置并行计算所使用的线程数
        
        :param int num\_threads: 并行计算所使用的线程数

DATA
    \_\_all\_\_ = ['is\_auto\_close\_output\_datasource', 'set\_auto\_close\_output\_d{\ldots}

FILE
    /opt/conda/lib/python3.6/site-packages/iobjectspy/env.py



    \end{Verbatim}

    \begin{Verbatim}[commandchars=\\\{\}]
{\color{incolor}In [{\color{incolor}5}]:} \PY{n}{help}\PY{p}{(}\PY{n}{smo}\PY{o}{.}\PY{n}{enums}\PY{p}{)}
\end{Verbatim}


    \begin{Verbatim}[commandchars=\\\{\}]
Help on module iobjectspy.enums in iobjectspy:

NAME
    iobjectspy.enums

CLASSES
    iobjectspy.\_jsuperpy.enums.JEnum(enum.IntEnum)
        iobjectspy.\_jsuperpy.enums.AggregationMethod
        iobjectspy.\_jsuperpy.enums.AggregationType
        iobjectspy.\_jsuperpy.enums.ArcAndVertexFilterMode
        iobjectspy.\_jsuperpy.enums.AreaUnit
        iobjectspy.\_jsuperpy.enums.AttributeStatisticsMode
        iobjectspy.\_jsuperpy.enums.BandWidthType
        iobjectspy.\_jsuperpy.enums.BlockSizeOption
        iobjectspy.\_jsuperpy.enums.BufferEndType
        iobjectspy.\_jsuperpy.enums.BufferRadiusUnit
        iobjectspy.\_jsuperpy.enums.CADVersion
        iobjectspy.\_jsuperpy.enums.Charset
        iobjectspy.\_jsuperpy.enums.ColorGradientType
        iobjectspy.\_jsuperpy.enums.ComputeType
        iobjectspy.\_jsuperpy.enums.ConceptualizationModel
        iobjectspy.\_jsuperpy.enums.CoordSysTransMethod
        iobjectspy.\_jsuperpy.enums.CursorType
        iobjectspy.\_jsuperpy.enums.DatasetType
        iobjectspy.\_jsuperpy.enums.DissolveType
        iobjectspy.\_jsuperpy.enums.DistanceMethod
        iobjectspy.\_jsuperpy.enums.EdgeMatchMode
        iobjectspy.\_jsuperpy.enums.EllipseSize
        iobjectspy.\_jsuperpy.enums.EncodeType
        iobjectspy.\_jsuperpy.enums.EngineType
        iobjectspy.\_jsuperpy.enums.Exponent
        iobjectspy.\_jsuperpy.enums.FieldType
        iobjectspy.\_jsuperpy.enums.FunctionType
        iobjectspy.\_jsuperpy.enums.GeoCoordSysType
        iobjectspy.\_jsuperpy.enums.GeoDatumType
        iobjectspy.\_jsuperpy.enums.GeoPrimeMeridianType
        iobjectspy.\_jsuperpy.enums.GeoSpatialRefType
        iobjectspy.\_jsuperpy.enums.GeoSpheroidType
        iobjectspy.\_jsuperpy.enums.GeometryType
        iobjectspy.\_jsuperpy.enums.GridStatisticsMode
        iobjectspy.\_jsuperpy.enums.GriddingLevel
        iobjectspy.\_jsuperpy.enums.IgnoreMode
        iobjectspy.\_jsuperpy.enums.ImportMode
        iobjectspy.\_jsuperpy.enums.InterpolationAlgorithmType
        iobjectspy.\_jsuperpy.enums.JoinType
        iobjectspy.\_jsuperpy.enums.KernelFunction
        iobjectspy.\_jsuperpy.enums.KernelType
        iobjectspy.\_jsuperpy.enums.LineToPointMode
        iobjectspy.\_jsuperpy.enums.MultiBandImportMode
        iobjectspy.\_jsuperpy.enums.NeighbourShapeType
        iobjectspy.\_jsuperpy.enums.NeighbourUnitType
        iobjectspy.\_jsuperpy.enums.OverlayMode
        iobjectspy.\_jsuperpy.enums.PixelFormat
        iobjectspy.\_jsuperpy.enums.PrjCoordSysType
        iobjectspy.\_jsuperpy.enums.ProjectionType
        iobjectspy.\_jsuperpy.enums.RasterJoinPixelFormat
        iobjectspy.\_jsuperpy.enums.RasterJoinType
        iobjectspy.\_jsuperpy.enums.RasterResampleMode
        iobjectspy.\_jsuperpy.enums.ReclassPixelFormat
        iobjectspy.\_jsuperpy.enums.ReclassSegmentType
        iobjectspy.\_jsuperpy.enums.ReclassType
        iobjectspy.\_jsuperpy.enums.RegionToPointMode
        iobjectspy.\_jsuperpy.enums.ResamplingMethod
        iobjectspy.\_jsuperpy.enums.SearchMode
        iobjectspy.\_jsuperpy.enums.ShadowMode
        iobjectspy.\_jsuperpy.enums.SlopeType
        iobjectspy.\_jsuperpy.enums.SmoothMethod
        iobjectspy.\_jsuperpy.enums.SpatialIndexType
        iobjectspy.\_jsuperpy.enums.SpatialQueryMode
        iobjectspy.\_jsuperpy.enums.SpatialStatisticsType
        iobjectspy.\_jsuperpy.enums.StatisticMode
        iobjectspy.\_jsuperpy.enums.StatisticsCompareType
        iobjectspy.\_jsuperpy.enums.StatisticsFieldType
        iobjectspy.\_jsuperpy.enums.StatisticsType
        iobjectspy.\_jsuperpy.enums.StreamOrderType
        iobjectspy.\_jsuperpy.enums.StringAlignment
        iobjectspy.\_jsuperpy.enums.TerrainInterpolateType
        iobjectspy.\_jsuperpy.enums.TerrainStatisticType
        iobjectspy.\_jsuperpy.enums.TextAlignment
        iobjectspy.\_jsuperpy.enums.TopologyRule
        iobjectspy.\_jsuperpy.enums.Unit
        iobjectspy.\_jsuperpy.enums.VCTVersion
        iobjectspy.\_jsuperpy.enums.VariogramMode
        iobjectspy.\_jsuperpy.enums.VectorResampleType
        iobjectspy.\_jsuperpy.enums.WorkspaceType
        iobjectspy.\_jsuperpy.enums.WorkspaceVersion
    
    class AggregationMethod(JEnum)
     |  用于通过事件点创建数据集进行分析的聚合方法常量
     |  
     |  :var AggregationMethod.NETWORKPOLYGONS: 计算合适的网格大小,创建网格面数据集,生成的网格面数据集以面网格单元的点计数将作
     |                                          为分析字段执行热点分析。网格会覆盖在输入事件点的上方,并将计算每个面网格单元内的
     |                                          点数目。如果未提供事件点发生区域的边界面数据(参阅 :py:func:`optimized\_hot\_spot\_analyst` 的 bounding\_polygons 参数),
     |                                          则会利用输入事件点数据集范围划分网格,并且会删除不含点的面网格单元,仅会分析剩下的
     |                                          面网格单元;如果提供了边界面数据,则只会保留并分析在边界面数据集范围内的面网格单元。
     |  
     |  :var AggregationMethod.AGGREGATIONPOLYGONS: 需要提供聚合事件点以获得事件计数的面数据集(参阅 参阅 :py:func:`optimized\_hot\_spot\_analyst` 的 aggregating\_polygons 参数),
     |                                              将计算每个面对象内的点事件数目,然后对面数据集以点事件数目作为分析字段执行热点分析。
     |  
     |  :var AggregationMethod.SNAPNEARBYPOINTS: 为输入事件点数据集计算捕捉距离并使用该距离聚合附近的事件点,为每个聚合点提供一个
     |                                           点计数,代表聚合到一起的事件点数目,然后对生成聚合点数据集以聚合在一起的点事件数
     |                                           目作为分析字段执行热点分析
     |  
     |  Method resolution order:
     |      AggregationMethod
     |      JEnum
     |      enum.IntEnum
     |      builtins.int
     |      enum.Enum
     |      builtins.object
     |  
     |  Data and other attributes defined here:
     |  
     |  AGGREGATIONPOLYGONS = AggregationMethod.AGGREGATIONPOLYGONS
     |  
     |  NETWORKPOLYGONS = AggregationMethod.NETWORKPOLYGONS
     |  
     |  SNAPNEARBYPOINTS = AggregationMethod.SNAPNEARBYPOINTS
     |  
     |  ----------------------------------------------------------------------
     |  Data descriptors inherited from enum.Enum:
     |  
     |  name
     |      The name of the Enum member.
     |  
     |  value
     |      The value of the Enum member.
     |  
     |  ----------------------------------------------------------------------
     |  Data descriptors inherited from enum.EnumMeta:
     |  
     |  \_\_members\_\_
     |      Returns a mapping of member name->value.
     |      
     |      This mapping lists all enum members, including aliases. Note that this
     |      is a read-only view of the internal mapping.
    
    class AggregationType(JEnum)
     |  定义了聚合操作时结果栅格的计算方式类型常量
     |  
     |  :var AggregationType.SUM: 一个聚合栅格内包含的所有栅格值之和
     |  :var AggregationType.MIN: 一个聚合栅格内包含的所有栅格值中的最小值
     |  :var AggregationType.MAX: 一个聚合栅格内包含的所有栅格值中的最大值
     |  :var AggregationType.AVERRAGE: 一个聚合栅格内包含的所有栅格值中的平均值
     |  :var AggregationType.MEDIAN: 一个聚合栅格内包含的所有栅格值中的中值
     |  
     |  Method resolution order:
     |      AggregationType
     |      JEnum
     |      enum.IntEnum
     |      builtins.int
     |      enum.Enum
     |      builtins.object
     |  
     |  Data and other attributes defined here:
     |  
     |  AVERRAGE = AggregationType.AVERRAGE
     |  
     |  MAX = AggregationType.MAX
     |  
     |  MEDIAN = AggregationType.MEDIAN
     |  
     |  MIN = AggregationType.MIN
     |  
     |  SUM = AggregationType.SUM
     |  
     |  ----------------------------------------------------------------------
     |  Data descriptors inherited from enum.Enum:
     |  
     |  name
     |      The name of the Enum member.
     |  
     |  value
     |      The value of the Enum member.
     |  
     |  ----------------------------------------------------------------------
     |  Data descriptors inherited from enum.EnumMeta:
     |  
     |  \_\_members\_\_
     |      Returns a mapping of member name->value.
     |      
     |      This mapping lists all enum members, including aliases. Note that this
     |      is a read-only view of the internal mapping.
    
    class ArcAndVertexFilterMode(JEnum)
     |  该类定义了弧段求交过滤模式常量。
     |  
     |  弧段求交用于将线对象在相交处打断,通常是对线数据建立拓扑关系时的首要步骤。
     |  
     |  :var ArcAndVertexFilterMode.NONE: 不过滤,即在所有交点处打断线对象。该模式下设置过滤线表达式或过滤点数据集均无效。
     |                                    如下图所示,线对象 A、B、C、D 在它们的相交处分别打断,即 A、B 在它们相交处分别被打断,C 在与 A、D 的相交处被打断。
     |  
     |                                    .. image:: ../image/FilterMode\_None.png
     |  
     |  :var ArcAndVertexFilterMode.ARC: 仅由过滤线表达式过滤,即过滤线表达式查询出的线对象不打断。该模式下设置过滤点记录集无效。
     |                                   如下图所示,线对象 C 是满足过滤线表达式的对象,则线对象 C 整条线不会在任何位置被打断。
     |  
     |                                   .. image:: ../image/FilterMode\_Arc.png
     |  
     |  :var ArcAndVertexFilterMode.VERTEX:  仅由过滤点记录集过滤,即线对象在过滤点所在位置(或与过滤点的距离在容限范围内)处不打断。该模式下设置过滤线表达式无效。
     |                                       如下图所示,某个过滤点位于线对象 A 和 C 在相交处,则在该处 C 不会被打断,其他相交位置仍会打断。
     |  
     |                                       .. image:: ../image/FilterMode\_Vertex.png
     |  
     |  :var ArcAndVertexFilterMode.ARC\_AND\_VERTEX: 由过滤线表达式和过滤点记录集共同决定哪些位置不打断,二者为且的关系,即只有过滤线表达式查询出的线对象在过滤点位置处(或二者在容限范围内)不打断。
     |                                              如下图所示,线对象 C 是满足过滤线表达式的对象,A、B 相交处,C、D 相交处分别有一个过滤点,根据该模式规则,过滤线上过滤点所在的位置不会被打断,即 C 在与 D 的相交处不打断。
     |  
     |                                              .. image:: ../image/FilterMode\_ArcAndVertex.png
     |  
     |  :var ArcAndVertexFilterMode.ARC\_OR\_VERTEX: 过滤线表达式查询出的线对象以及过滤点位置处(或与过滤点距离在容限范围内)的线对象不打断,二者为并的关系。
     |                                             如下图所示,线对象 C 是满足过滤线表达式的对象,A、B 相交处,C、D 相交处分别有一个过滤点,根据该模式规则,结果如右图所示,C 整体不被打断,A、B 相交处也不打断。
     |  
     |                                             .. image:: ../image/FilterMode\_ArcOrVertex.png
     |  
     |  Method resolution order:
     |      ArcAndVertexFilterMode
     |      JEnum
     |      enum.IntEnum
     |      builtins.int
     |      enum.Enum
     |      builtins.object
     |  
     |  Data and other attributes defined here:
     |  
     |  ARC = ArcAndVertexFilterMode.ARC
     |  
     |  ARC\_AND\_VERTEX = ArcAndVertexFilterMode.ARC\_AND\_VERTEX
     |  
     |  ARC\_OR\_VERTEX = ArcAndVertexFilterMode.ARC\_OR\_VERTEX
     |  
     |  NONE = ArcAndVertexFilterMode.NONE
     |  
     |  VERTEX = ArcAndVertexFilterMode.VERTEX
     |  
     |  ----------------------------------------------------------------------
     |  Data descriptors inherited from enum.Enum:
     |  
     |  name
     |      The name of the Enum member.
     |  
     |  value
     |      The value of the Enum member.
     |  
     |  ----------------------------------------------------------------------
     |  Data descriptors inherited from enum.EnumMeta:
     |  
     |  \_\_members\_\_
     |      Returns a mapping of member name->value.
     |      
     |      This mapping lists all enum members, including aliases. Note that this
     |      is a read-only view of the internal mapping.
    
    class AreaUnit(JEnum)
     |  面积单位类型:
     |  
     |  :var AreaUnit.SQUAREMILLIMETER: 公制单位,平方毫米。
     |  :var AreaUnit.SQUARECENTIMETER: 公制单位,平方厘米。
     |  :var AreaUnit.SQUAREDECIMETER: 公制单位,平方分米。
     |  :var AreaUnit.SQUAREMETER: 公制单位,平方米。
     |  :var AreaUnit.SQUAREKILOMETER: 公制单位,平方千米。
     |  :var AreaUnit.HECTARE: 公制单位,公顷。
     |  :var AreaUnit.ARE: 公制单位,公亩。
     |  :var AreaUnit.QING: 市制单位,顷。
     |  :var AreaUnit.MU: 市制单位,亩。
     |  :var AreaUnit.SQUAREINCH: 英制单位,平方英寸。
     |  :var AreaUnit.SQUAREFOOT: 英制单位,平方尺。
     |  :var AreaUnit.SQUAREYARD: 英制单位,平方码。
     |  :var AreaUnit.SQUAREMILE: 英制单位,平方英里。
     |  :var AreaUnit.ACRE: 英制单位,英亩。
     |  
     |  Method resolution order:
     |      AreaUnit
     |      JEnum
     |      enum.IntEnum
     |      builtins.int
     |      enum.Enum
     |      builtins.object
     |  
     |  Data and other attributes defined here:
     |  
     |  ACRE = AreaUnit.ACRE
     |  
     |  ARE = AreaUnit.ARE
     |  
     |  HECTARE = AreaUnit.HECTARE
     |  
     |  MU = AreaUnit.MU
     |  
     |  QING = AreaUnit.QING
     |  
     |  SQUARECENTIMETER = AreaUnit.SQUARECENTIMETER
     |  
     |  SQUAREDECIMETER = AreaUnit.SQUAREDECIMETER
     |  
     |  SQUAREFOOT = AreaUnit.SQUAREFOOT
     |  
     |  SQUAREINCH = AreaUnit.SQUAREINCH
     |  
     |  SQUAREKILOMETER = AreaUnit.SQUAREKILOMETER
     |  
     |  SQUAREMETER = AreaUnit.SQUAREMETER
     |  
     |  SQUAREMILE = AreaUnit.SQUAREMILE
     |  
     |  SQUAREMILLIMETER = AreaUnit.SQUAREMILLIMETER
     |  
     |  SQUAREYARD = AreaUnit.SQUAREYARD
     |  
     |  ----------------------------------------------------------------------
     |  Data descriptors inherited from enum.Enum:
     |  
     |  name
     |      The name of the Enum member.
     |  
     |  value
     |      The value of the Enum member.
     |  
     |  ----------------------------------------------------------------------
     |  Data descriptors inherited from enum.EnumMeta:
     |  
     |  \_\_members\_\_
     |      Returns a mapping of member name->value.
     |      
     |      This mapping lists all enum members, including aliases. Note that this
     |      is a read-only view of the internal mapping.
    
    class AttributeStatisticsMode(JEnum)
     |  在进行点连接成线时和矢量数据集属性更新时,进行属性统计的模式。
     |  
     |  :var AttributeStatisticsMode.MAX: 统计最大值,可以对数值型、文本型和时间类型的字段进行统计。
     |  :var AttributeStatisticsMode.MIN: 统计最小值,可以对数值型、文本型和时间类型的字段进行统计。
     |  :var AttributeStatisticsMode.SUM: 统计一组数的和,只对数值型字段有效
     |  :var AttributeStatisticsMode.MEAN: 统计一组数的平均值,只对数值型字段有效
     |  :var AttributeStatisticsMode.STDEV: 统计一组数的标准差,只对数值型字段有效
     |  :var AttributeStatisticsMode.VAR: 统计一组数的方差,只对数值型字段有效
     |  :var AttributeStatisticsMode.MODALVALUE: 取众数,众数是出现频率最高的的值,可以是任何类型字段
     |  :var AttributeStatisticsMode.RECORDCOUNT: 统计一组数的记录数。统计记录数不针对特定的字段,只针对一个分组。
     |  :var AttributeStatisticsMode.MAXINTERSECTAREA: 取相交面积最大。如果面对象与提供属性的多个面对象相交,则取与原面对象相交面积最大的对象属性值用于更新。对任意类型的字段有效。
     |                                                 只对矢量数据集属性更新( :py:func:`update\_attributes` )有效
     |  
     |  Method resolution order:
     |      AttributeStatisticsMode
     |      JEnum
     |      enum.IntEnum
     |      builtins.int
     |      enum.Enum
     |      builtins.object
     |  
     |  Data and other attributes defined here:
     |  
     |  COUNT = AttributeStatisticsMode.COUNT
     |  
     |  MAX = AttributeStatisticsMode.MAX
     |  
     |  MAXINTERSECTAREA = AttributeStatisticsMode.MAXINTERSECTAREA
     |  
     |  MEAN = AttributeStatisticsMode.MEAN
     |  
     |  MIN = AttributeStatisticsMode.MIN
     |  
     |  MODALVALUE = AttributeStatisticsMode.MODALVALUE
     |  
     |  STDEV = AttributeStatisticsMode.STDEV
     |  
     |  SUM = AttributeStatisticsMode.SUM
     |  
     |  VAR = AttributeStatisticsMode.VAR
     |  
     |  ----------------------------------------------------------------------
     |  Data descriptors inherited from enum.Enum:
     |  
     |  name
     |      The name of the Enum member.
     |  
     |  value
     |      The value of the Enum member.
     |  
     |  ----------------------------------------------------------------------
     |  Data descriptors inherited from enum.EnumMeta:
     |  
     |  \_\_members\_\_
     |      Returns a mapping of member name->value.
     |      
     |      This mapping lists all enum members, including aliases. Note that this
     |      is a read-only view of the internal mapping.
    
    class BandWidthType(JEnum)
     |  地理加权回归分析带宽确定方式常量。
     |  
     |  :var BandWidthType.AICC: 使用" Akaike 信息准则(AICc)"确定带宽范围。
     |  :var BandWidthType.CV: 使用"交叉验证"确定带宽范围。
     |  :var BandWidthType.BANDWIDTH: 根据给定的固定距离或固定相邻数确定带宽范围。
     |  
     |  Method resolution order:
     |      BandWidthType
     |      JEnum
     |      enum.IntEnum
     |      builtins.int
     |      enum.Enum
     |      builtins.object
     |  
     |  Data and other attributes defined here:
     |  
     |  AICC = BandWidthType.AICC
     |  
     |  BANDWIDTH = BandWidthType.BANDWIDTH
     |  
     |  CV = BandWidthType.CV
     |  
     |  ----------------------------------------------------------------------
     |  Data descriptors inherited from enum.Enum:
     |  
     |  name
     |      The name of the Enum member.
     |  
     |  value
     |      The value of the Enum member.
     |  
     |  ----------------------------------------------------------------------
     |  Data descriptors inherited from enum.EnumMeta:
     |  
     |  \_\_members\_\_
     |      Returns a mapping of member name->value.
     |      
     |      This mapping lists all enum members, including aliases. Note that this
     |      is a read-only view of the internal mapping.
    
    class BlockSizeOption(JEnum)
     |  该枚举定义了像素分块的类型常量。用于栅格数据集或影像数据:
     |  
     |  :var BlockSizeOption.BS\_64: 表示64像素*64像素的分块
     |  :var BlockSizeOption.BS\_128: 表示128像素*128像素的分块
     |  :var BlockSizeOption.BS\_256: 表示256像素*256像素的分块
     |  :var BlockSizeOption.BS\_512: 表示512像素*512像素的分块。
     |  :var BlockSizeOption.BS\_1024: 表示1024像素*1024像素的分块。
     |  
     |  Method resolution order:
     |      BlockSizeOption
     |      JEnum
     |      enum.IntEnum
     |      builtins.int
     |      enum.Enum
     |      builtins.object
     |  
     |  Data and other attributes defined here:
     |  
     |  BS\_1024 = BlockSizeOption.BS\_1024
     |  
     |  BS\_128 = BlockSizeOption.BS\_128
     |  
     |  BS\_256 = BlockSizeOption.BS\_256
     |  
     |  BS\_512 = BlockSizeOption.BS\_512
     |  
     |  BS\_64 = BlockSizeOption.BS\_64
     |  
     |  ----------------------------------------------------------------------
     |  Data descriptors inherited from enum.Enum:
     |  
     |  name
     |      The name of the Enum member.
     |  
     |  value
     |      The value of the Enum member.
     |  
     |  ----------------------------------------------------------------------
     |  Data descriptors inherited from enum.EnumMeta:
     |  
     |  \_\_members\_\_
     |      Returns a mapping of member name->value.
     |      
     |      This mapping lists all enum members, including aliases. Note that this
     |      is a read-only view of the internal mapping.
    
    class BufferEndType(JEnum)
     |  该类定义了缓冲区端点类型常量。
     |  
     |  用以区分线对象缓冲区分析时的端点是圆头缓冲还是平头缓冲。
     |  
     |  :var BufferEndType.ROUND: 圆头缓冲。圆头缓冲区是在生成缓冲区时,在线段的端点处做半圆弧处理
     |  :var BufferEndType.FLAT: 平头缓冲。平头缓冲区是在生成缓冲区时,在线段的端点处做圆弧的垂线。
     |  
     |  Method resolution order:
     |      BufferEndType
     |      JEnum
     |      enum.IntEnum
     |      builtins.int
     |      enum.Enum
     |      builtins.object
     |  
     |  Data and other attributes defined here:
     |  
     |  FLAT = BufferEndType.FLAT
     |  
     |  ROUND = BufferEndType.ROUND
     |  
     |  ----------------------------------------------------------------------
     |  Data descriptors inherited from enum.Enum:
     |  
     |  name
     |      The name of the Enum member.
     |  
     |  value
     |      The value of the Enum member.
     |  
     |  ----------------------------------------------------------------------
     |  Data descriptors inherited from enum.EnumMeta:
     |  
     |  \_\_members\_\_
     |      Returns a mapping of member name->value.
     |      
     |      This mapping lists all enum members, including aliases. Note that this
     |      is a read-only view of the internal mapping.
    
    class BufferRadiusUnit(JEnum)
     |  该枚举定义了缓冲区分析半径单位类型常量
     |  
     |  :var BufferRadiusUnit.MILIMETER: 毫米
     |  :var BufferRadiusUnit.CENTIMETER: 厘米
     |  :var BufferRadiusUnit.DECIMETER: 分米
     |  :var BufferRadiusUnit.METER: 米
     |  :var BufferRadiusUnit.KILOMETER: 千米
     |  :var BufferRadiusUnit.INCH: 英寸
     |  :var BufferRadiusUnit.FOOT: 英尺
     |  :var BufferRadiusUnit.YARD: 码
     |  :var BufferRadiusUnit.MILE: 英里
     |  
     |  Method resolution order:
     |      BufferRadiusUnit
     |      JEnum
     |      enum.IntEnum
     |      builtins.int
     |      enum.Enum
     |      builtins.object
     |  
     |  Data and other attributes defined here:
     |  
     |  CENTIMETER = BufferRadiusUnit.CENTIMETER
     |  
     |  DECIMETER = BufferRadiusUnit.DECIMETER
     |  
     |  FOOT = BufferRadiusUnit.FOOT
     |  
     |  INCH = BufferRadiusUnit.INCH
     |  
     |  KILOMETER = BufferRadiusUnit.KILOMETER
     |  
     |  METER = BufferRadiusUnit.METER
     |  
     |  MILE = BufferRadiusUnit.MILE
     |  
     |  MILIMETER = BufferRadiusUnit.MILIMETER
     |  
     |  YARD = BufferRadiusUnit.YARD
     |  
     |  ----------------------------------------------------------------------
     |  Data descriptors inherited from enum.Enum:
     |  
     |  name
     |      The name of the Enum member.
     |  
     |  value
     |      The value of the Enum member.
     |  
     |  ----------------------------------------------------------------------
     |  Data descriptors inherited from enum.EnumMeta:
     |  
     |  \_\_members\_\_
     |      Returns a mapping of member name->value.
     |      
     |      This mapping lists all enum members, including aliases. Note that this
     |      is a read-only view of the internal mapping.
    
    class CADVersion(JEnum)
     |  该类定义了 AutoCAD 版本类型常量。提供了 AutoCAD 的不同版本类型及说明。
     |  
     |  :var CADVersion.CAD12: OdDb::vAC12 R11-12
     |  :var CADVersion.CAD13: OdDb::vAC13 R13
     |  :var CADVersion.CAD14: OdDb::vAC14 R14
     |  :var CADVersion.CAD2000: OdDb::vAC15 2000-2002
     |  :var CADVersion.CAD2004: OdDb::vAC18 2004-2006
     |  :var CADVersion.CAD2007: OdDb::vAC21 2007
     |  
     |  Method resolution order:
     |      CADVersion
     |      JEnum
     |      enum.IntEnum
     |      builtins.int
     |      enum.Enum
     |      builtins.object
     |  
     |  Data and other attributes defined here:
     |  
     |  CAD12 = CADVersion.CAD12
     |  
     |  CAD13 = CADVersion.CAD13
     |  
     |  CAD14 = CADVersion.CAD14
     |  
     |  CAD2000 = CADVersion.CAD2000
     |  
     |  CAD2004 = CADVersion.CAD2004
     |  
     |  CAD2007 = CADVersion.CAD2007
     |  
     |  ----------------------------------------------------------------------
     |  Data descriptors inherited from enum.Enum:
     |  
     |  name
     |      The name of the Enum member.
     |  
     |  value
     |      The value of the Enum member.
     |  
     |  ----------------------------------------------------------------------
     |  Data descriptors inherited from enum.EnumMeta:
     |  
     |  \_\_members\_\_
     |      Returns a mapping of member name->value.
     |      
     |      This mapping lists all enum members, including aliases. Note that this
     |      is a read-only view of the internal mapping.
    
    class Charset(JEnum)
     |  该类定义了矢量数据集的字符集类型常量。
     |  
     |  :var Charset.ANSI:  ASCII 字符集
     |  :var Charset.DEFAULT: 扩展的 ASCII 字符集。
     |  :var Charset.SYMBOL: 符号字符集。
     |  :var Charset.MAC: Macintosh 使用的字符
     |  :var Charset.SHIFTJIS: 日语字符集
     |  :var Charset.HANGEUL: 朝鲜字符集的其它常用拼写
     |  :var Charset.JOHAB: 朝鲜字符集
     |  :var Charset.GB18030: 在中国大陆使用的中文字符集
     |  :var Charset.CHINESEBIG5: 在中国香港特别行政区和台湾最常用的中文字符集
     |  :var Charset.GREEK: 希腊字符集
     |  :var Charset.TURKISH: 土耳其语字符集
     |  :var Charset.VIETNAMESE: 越南语字符集
     |  :var Charset.HEBREW: 希伯来字符集
     |  :var Charset.ARABIC: 阿拉伯字符集
     |  :var Charset.BALTIC: 波罗的海字符集
     |  :var Charset.RUSSIAN: 俄语字符集
     |  :var Charset.THAI: 泰语字符集
     |  :var Charset.EASTEUROPE: 东欧字符集
     |  :var Charset.OEM: 扩展的 ASCII 字符集
     |  :var Charset.UTF8: UTF-8(8 位元 Universal Character Set/Unicode Transformation Format)是针对Unicode 的一种可变长度字符编码。它可以用来表示 Unicode 标准中的任何字符,而且其编码中的第一个字节仍与 ASCII 相容,使得原来处理 ASCII 字符的软件无需或只作少部份修改后,便可继续使用。
     |  :var Charset.UTF7: UTF-7 (7-位元 Unicode 转换格式(Unicode Transformation Format,简写成 UTF)) 是一种可变长度字符编码方式,用以将 Unicode 字符以 ASCII 编码的字符串来呈现。
     |  :var Charset.WINDOWS1252: 英文常用的编码。Windows1252(Window 9x标准for西欧语言)。
     |  :var Charset.KOREAN: 韩语字符集
     |  :var Charset.UNICODE: 在计算机科学领域中,Unicode(统一码、万国码、单一码、标准万国码)是业界的一种标准。
     |  :var Charset.CYRILLIC: Cyrillic (Windows)
     |  :var Charset.XIA5: IA5
     |  :var Charset.XIA5GERMAN: IA5 (German)
     |  :var Charset.XIA5SWEDISH: IA5 (Swedish)
     |  :var Charset.XIA5NORWEGIAN: IA5 (Norwegian)
     |  
     |  Method resolution order:
     |      Charset
     |      JEnum
     |      enum.IntEnum
     |      builtins.int
     |      enum.Enum
     |      builtins.object
     |  
     |  Data and other attributes defined here:
     |  
     |  ANSI = Charset.ANSI
     |  
     |  ARABIC = Charset.ARABIC
     |  
     |  BALTIC = Charset.BALTIC
     |  
     |  CHINESEBIG5 = Charset.CHINESEBIG5
     |  
     |  CYRILLIC = Charset.CYRILLIC
     |  
     |  DEFAULT = Charset.DEFAULT
     |  
     |  EASTEUROPE = Charset.EASTEUROPE
     |  
     |  GB18030 = Charset.GB18030
     |  
     |  GREEK = Charset.GREEK
     |  
     |  HANGEUL = Charset.HANGEUL
     |  
     |  HEBREW = Charset.HEBREW
     |  
     |  JOHAB = Charset.JOHAB
     |  
     |  KOREAN = Charset.KOREAN
     |  
     |  MAC = Charset.MAC
     |  
     |  OEM = Charset.OEM
     |  
     |  RUSSIAN = Charset.RUSSIAN
     |  
     |  SHIFTJIS = Charset.SHIFTJIS
     |  
     |  SYMBOL = Charset.SYMBOL
     |  
     |  THAI = Charset.THAI
     |  
     |  TURKISH = Charset.TURKISH
     |  
     |  UNICODE = Charset.UNICODE
     |  
     |  UTF7 = Charset.UTF7
     |  
     |  UTF8 = Charset.UTF8
     |  
     |  VIETNAMESE = Charset.VIETNAMESE
     |  
     |  WINDOWS1252 = Charset.WINDOWS1252
     |  
     |  XIA5 = Charset.XIA5
     |  
     |  XIA5GERMAN = Charset.XIA5GERMAN
     |  
     |  XIA5NORWEGIAN = Charset.XIA5NORWEGIAN
     |  
     |  XIA5SWEDISH = Charset.XIA5SWEDISH
     |  
     |  ----------------------------------------------------------------------
     |  Data descriptors inherited from enum.Enum:
     |  
     |  name
     |      The name of the Enum member.
     |  
     |  value
     |      The value of the Enum member.
     |  
     |  ----------------------------------------------------------------------
     |  Data descriptors inherited from enum.EnumMeta:
     |  
     |  \_\_members\_\_
     |      Returns a mapping of member name->value.
     |      
     |      This mapping lists all enum members, including aliases. Note that this
     |      is a read-only view of the internal mapping.
    
    class ColorGradientType(JEnum)
     |  该类定义了颜色渐变类型常量。
     |  
     |  颜色渐变是多种颜色间的逐渐混合,可以是从起始色到终止色两种颜色的渐变,或者在起始色到终止色之间具有多种中间颜色进行渐变。该颜色渐变类型可应用于专题图对象的颜色方案设置中如:单值专题图、 分段专题图、 统计专题图、标签专题图、栅格分段专题图和栅格单值专题图。
     |  
     |  :var ColorGradientType.BLACKWHITE: 黑白渐变色
     |  :var ColorGradientType.REDWHITE: 红白渐变色
     |  :var ColorGradientType.GREENWHITE: 绿白渐变色
     |  :var ColorGradientType.BLUEWHITE: 蓝白渐变色
     |  :var ColorGradientType.YELLOWWHITE: 黄白渐变色
     |  :var ColorGradientType.PINKWHITE: 粉红白渐变色
     |  :var ColorGradientType.CYANWHITE: 青白渐变色
     |  :var ColorGradientType.REDBLACK: 红黑渐变色
     |  :var ColorGradientType.GREENBLACK: 绿黑渐变色
     |  :var ColorGradientType.BLUEBLACK: 蓝黑渐变色
     |  :var ColorGradientType.YELLOWBLACK: 黄黑渐变色
     |  :var ColorGradientType.PINKBLACK: 粉红黑渐变色
     |  :var ColorGradientType.CYANBLACK: 青黑渐变色
     |  :var ColorGradientType.YELLOWRED: 黄红渐变色
     |  :var ColorGradientType.YELLOWGREEN: 黄绿渐变色
     |  :var ColorGradientType.YELLOWBLUE: 黄蓝渐变色
     |  :var ColorGradientType.GREENBLUE: 绿蓝渐变色
     |  :var ColorGradientType.GREENRED: 绿红渐变色
     |  :var ColorGradientType.BLUERED: 蓝红渐变色
     |  :var ColorGradientType.PINKRED: 粉红红渐变色
     |  :var ColorGradientType.PINKBLUE: 粉红蓝渐变色
     |  :var ColorGradientType.CYANBLUE: 青蓝渐变色
     |  :var ColorGradientType.CYANGREEN: 青绿渐变色
     |  :var ColorGradientType.RAINBOW: 彩虹色
     |  :var ColorGradientType.GREENORANGEVIOLET: 绿橙紫渐变色
     |  :var ColorGradientType.TERRAIN: 地形渐变
     |  :var ColorGradientType.SPECTRUM: 光谱渐变
     |  
     |  Method resolution order:
     |      ColorGradientType
     |      JEnum
     |      enum.IntEnum
     |      builtins.int
     |      enum.Enum
     |      builtins.object
     |  
     |  Data and other attributes defined here:
     |  
     |  BLACKWHITE = ColorGradientType.BLACKWHITE
     |  
     |  BLUEBLACK = ColorGradientType.BLUEBLACK
     |  
     |  BLUERED = ColorGradientType.BLUERED
     |  
     |  BLUEWHITE = ColorGradientType.BLUEWHITE
     |  
     |  CYANBLACK = ColorGradientType.CYANBLACK
     |  
     |  CYANBLUE = ColorGradientType.CYANBLUE
     |  
     |  CYANGREEN = ColorGradientType.CYANGREEN
     |  
     |  CYANWHITE = ColorGradientType.CYANWHITE
     |  
     |  GREENBLACK = ColorGradientType.GREENBLACK
     |  
     |  GREENBLUE = ColorGradientType.GREENBLUE
     |  
     |  GREENORANGEVIOLET = ColorGradientType.GREENORANGEVIOLET
     |  
     |  GREENRED = ColorGradientType.GREENRED
     |  
     |  GREENWHITE = ColorGradientType.GREENWHITE
     |  
     |  PINKBLACK = ColorGradientType.PINKBLACK
     |  
     |  PINKBLUE = ColorGradientType.PINKBLUE
     |  
     |  PINKRED = ColorGradientType.PINKRED
     |  
     |  PINKWHITE = ColorGradientType.PINKWHITE
     |  
     |  RAINBOW = ColorGradientType.RAINBOW
     |  
     |  REDBLACK = ColorGradientType.REDBLACK
     |  
     |  REDWHITE = ColorGradientType.REDWHITE
     |  
     |  SPECTRUM = ColorGradientType.SPECTRUM
     |  
     |  TERRAIN = ColorGradientType.TERRAIN
     |  
     |  YELLOWBLACK = ColorGradientType.YELLOWBLACK
     |  
     |  YELLOWBLUE = ColorGradientType.YELLOWBLUE
     |  
     |  YELLOWGREEN = ColorGradientType.YELLOWGREEN
     |  
     |  YELLOWRED = ColorGradientType.YELLOWRED
     |  
     |  YELLOWWHITE = ColorGradientType.YELLOWWHITE
     |  
     |  ----------------------------------------------------------------------
     |  Data descriptors inherited from enum.Enum:
     |  
     |  name
     |      The name of the Enum member.
     |  
     |  value
     |      The value of the Enum member.
     |  
     |  ----------------------------------------------------------------------
     |  Data descriptors inherited from enum.EnumMeta:
     |  
     |  \_\_members\_\_
     |      Returns a mapping of member name->value.
     |      
     |      This mapping lists all enum members, including aliases. Note that this
     |      is a read-only view of the internal mapping.
    
    class ComputeType(JEnum)
     |  该类定义了距离栅格最短路径分析的计算方式类型常量
     |  
     |  :var ComputeType.CELL: 像元路径,目标对象对应的每一个栅格单元都生成一条最短路径。如下图所示,红色点作为源,黑线框多边形作为目标,采用该方式
     |                         进行栅格最短路径分析,得到蓝色单元格表示的最短路径。
     |  
     |                         .. image:: ../image/ComputeType\_CELL.png
     |  
     |  :var ComputeType.ZONE: 区域路径,每个目标对象对应的栅格区域都只生成一条最短路径。如下图所示,红色点作为源,黑线框多边形作为目标,采用该方
     |                         式进行栅格最短路径分析,得到蓝色单元格表示的最短路径。
     |  
     |                         .. image:: ../image/ComputeType\_ZONE.png
     |  
     |  :var ComputeType.ALL: 单一路径,所有目标对象对应的单元格只生成一条最短路径,即对于整个目标区域数据集来说所有路径中最短的那一条。如下图所示,
     |                        红色点作为源,黑线框多边形作为目标,采用该方式进行栅格最短路径分析,得到蓝色单元格表示的最短路径。
     |  
     |                        .. image:: ../image/ComputeType\_ALL.png
     |  
     |  Method resolution order:
     |      ComputeType
     |      JEnum
     |      enum.IntEnum
     |      builtins.int
     |      enum.Enum
     |      builtins.object
     |  
     |  Data and other attributes defined here:
     |  
     |  ALL = ComputeType.ALL
     |  
     |  CELL = ComputeType.CELL
     |  
     |  ZONE = ComputeType.ZONE
     |  
     |  ----------------------------------------------------------------------
     |  Data descriptors inherited from enum.Enum:
     |  
     |  name
     |      The name of the Enum member.
     |  
     |  value
     |      The value of the Enum member.
     |  
     |  ----------------------------------------------------------------------
     |  Data descriptors inherited from enum.EnumMeta:
     |  
     |  \_\_members\_\_
     |      Returns a mapping of member name->value.
     |      
     |      This mapping lists all enum members, including aliases. Note that this
     |      is a read-only view of the internal mapping.
    
    class ConceptualizationModel(JEnum)
     |  空间关系概念化模型常量
     |  
     |  :var ConceptualizationModel.INVERSEDISTANCE: 反距离模型。任何要素都会影响目标要素,但是随着距离的增加,影响会越小。要素之间的权重为距离分之一。
     |  :var ConceptualizationModel.INVERSEDISTANCESQUARED: 反距离平方模型。与"反距离模型"相似,随着距离的增加,影响下降的更快。要素之间的权重为距离的平方分之一。
     |  :var ConceptualizationModel.FIXEDDISTANCEBAND: 固定距离模型。在指定的固定距离范围内的要素具有相等的权重(权重为1),在指定的固定距离范围之外的要素不会影响计算(权重为0)。
     |  :var ConceptualizationModel.ZONEOFINDIFFERENCE: 无差别区域模型。 该模型是"反距离模型"和"固定距离模型"的结合。在指定的固定距离范围内的要素具有相等的权重(权重为1);在指定的固定距离范围之外的要素,随着距离的增加,影响会越小。
     |  :var ConceptualizationModel.CONTIGUITYEDGESONLY: 面邻接模型。只有面面在有共享边界、重叠、包含、被包含的情况才会影响目标要素(权重为1),否则,将会排除在目标要素计算之外(权重为0)。
     |  :var ConceptualizationModel.CONTIGUITYEDGESNODE: 面邻接模型。只有面面在有接触的情况才会影响目标要素(权重为1),否则,将会排除在目标要素计算之外(权重为0)。
     |  :var ConceptualizationModel.KNEARESTNEIGHBORS: K最邻近模型。 距目标要素最近的K个要素包含在目标要素的计算中(权重为1),其余的要素将会排除在目标要素计算之外(权重为0)。
     |  :var ConceptualizationModel.SPATIALWEIGHTMATRIXFILE: 提供空间权重矩阵文件。
     |  
     |  Method resolution order:
     |      ConceptualizationModel
     |      JEnum
     |      enum.IntEnum
     |      builtins.int
     |      enum.Enum
     |      builtins.object
     |  
     |  Data and other attributes defined here:
     |  
     |  CONTIGUITYEDGESNODE = ConceptualizationModel.CONTIGUITYEDGESNODE
     |  
     |  CONTIGUITYEDGESONLY = ConceptualizationModel.CONTIGUITYEDGESONLY
     |  
     |  FIXEDDISTANCEBAND = ConceptualizationModel.FIXEDDISTANCEBAND
     |  
     |  INVERSEDISTANCE = ConceptualizationModel.INVERSEDISTANCE
     |  
     |  INVERSEDISTANCESQUARED = ConceptualizationModel.INVERSEDISTANCESQUARED
     |  
     |  KNEARESTNEIGHBORS = ConceptualizationModel.KNEARESTNEIGHBORS
     |  
     |  SPATIALWEIGHTMATRIXFILE = ConceptualizationModel.SPATIALWEIGHTMATRIXFI{\ldots}
     |  
     |  ZONEOFINDIFFERENCE = ConceptualizationModel.ZONEOFINDIFFERENCE
     |  
     |  ----------------------------------------------------------------------
     |  Data descriptors inherited from enum.Enum:
     |  
     |  name
     |      The name of the Enum member.
     |  
     |  value
     |      The value of the Enum member.
     |  
     |  ----------------------------------------------------------------------
     |  Data descriptors inherited from enum.EnumMeta:
     |  
     |  \_\_members\_\_
     |      Returns a mapping of member name->value.
     |      
     |      This mapping lists all enum members, including aliases. Note that this
     |      is a read-only view of the internal mapping.
    
    class CoordSysTransMethod(JEnum)
     |  该类定义了投影转换方法类型常量。
     |  
     |  在投影转换中,如果源投影和目标投影的地理坐标系不同,则需要进行参照系的转换。
     |  
     |  参照系的转换有两种,基于网格的转换和基于公式的转换。本类所提供的转换方法均为基于公式的转换。依据转换参数的不同可以分为三参数法和七参数法。目前使
     |  用最广泛的是七参数法。参数信息参见 :py:class:`CoordSysTransParameter`;如果源投影和目标投影的地理坐标系相同,用户无需进行参照系的转换,即可以不进行
     |  :py:class:`CoordSysTransParameter` 参数信息的设置。本版本中的 GeocentricTranslation、Molodensky、MolodenskyAbridged 是基于地心的三参数转换
     |  法;PositionVector、CoordinateFrame、BursaWolf都是七参数法。
     |  
     |  :var CoordSysTransMethod.MTH\_GEOCENTRIC\_TRANSLATION: 基于地心的三参数转换法
     |  :var CoordSysTransMethod.MTH\_MOLODENSKY: 莫洛金斯基(Molodensky)转换法
     |  :var CoordSysTransMethod.MTH\_MOLODENSKY\_ABRIDGED: 简化的莫洛金斯基转换法
     |  :var CoordSysTransMethod.MTH\_POSITION\_VECTOR: 位置矢量法
     |  :var CoordSysTransMethod.MTH\_COORDINATE\_FRAME: 基于地心的七参数转换法
     |  :var CoordSysTransMethod.MTH\_BURSA\_WOLF: Bursa-Wolf 方法
     |  :var CoordSysTransMethod.MolodenskyBadekas: 莫洛金斯基—巴待卡斯投影转换方法,一种十参数的空间坐标转换模型。
     |  
     |  Method resolution order:
     |      CoordSysTransMethod
     |      JEnum
     |      enum.IntEnum
     |      builtins.int
     |      enum.Enum
     |      builtins.object
     |  
     |  Data and other attributes defined here:
     |  
     |  MTH\_BURSA\_WOLF = CoordSysTransMethod.MTH\_BURSA\_WOLF
     |  
     |  MTH\_COORDINATE\_FRAME = CoordSysTransMethod.MTH\_COORDINATE\_FRAME
     |  
     |  MTH\_GEOCENTRIC\_TRANSLATION = CoordSysTransMethod.MTH\_GEOCENTRIC\_TRANSL{\ldots}
     |  
     |  MTH\_MOLODENSKY = CoordSysTransMethod.MTH\_MOLODENSKY
     |  
     |  MTH\_MOLODENSKY\_ABRIDGED = CoordSysTransMethod.MTH\_MOLODENSKY\_ABRIDGED
     |  
     |  MTH\_POSITION\_VECTOR = CoordSysTransMethod.MTH\_POSITION\_VECTOR
     |  
     |  MolodenskyBadekas = CoordSysTransMethod.MolodenskyBadekas
     |  
     |  ----------------------------------------------------------------------
     |  Data descriptors inherited from enum.Enum:
     |  
     |  name
     |      The name of the Enum member.
     |  
     |  value
     |      The value of the Enum member.
     |  
     |  ----------------------------------------------------------------------
     |  Data descriptors inherited from enum.EnumMeta:
     |  
     |  \_\_members\_\_
     |      Returns a mapping of member name->value.
     |      
     |      This mapping lists all enum members, including aliases. Note that this
     |      is a read-only view of the internal mapping.
    
    class CursorType(JEnum)
     |  游标类型:
     |  
     |  :var CursorType.DYNAMIC: 动态游标类型。支持各种编辑操作,速度慢。动态游标含义:可以看见其他用户所作的添加、更改和删除。允许在记录集中进行前后移动, (但不包括
     |    数据提供者不支持的书签操作,书签操作主要针对于 ADO 而言)。此类型的游标功能很强大,但同时也是耗费系统资源最多的游标。动态游标可以知道记录
     |    集(Recordset)的所有变化。使用动态游标的用户可以看到其他用户对数据集所做的编辑、增加、删除等操作。如果数据提供者允许这种类型的游标,那么它
     |    是通过每隔一段时间从数据源重取数据来动态刷新查询的记录集的。毫无疑问这会需要很多的资源。
     |  
     |  :var CursorType.STATIC: 静态游标类型。静态游标含义:可以用来查找数据或生成报告的记录集合的静态副本。另外,对其他用户所作的添加、更改或删除不可见。静态游标只
     |    是数据的一幅快照。也就是说,它无法看到自从被创建以后其他用户对记录集(Recordset)所做的编辑操作。
     |    采用这类游标你可以向前和向后回溯。由于其功能简单,资源的耗费比动态游标(DYNAMIC)要小!)
     |  
     |  Method resolution order:
     |      CursorType
     |      JEnum
     |      enum.IntEnum
     |      builtins.int
     |      enum.Enum
     |      builtins.object
     |  
     |  Data and other attributes defined here:
     |  
     |  DYNAMIC = CursorType.DYNAMIC
     |  
     |  STATIC = CursorType.STATIC
     |  
     |  ----------------------------------------------------------------------
     |  Data descriptors inherited from enum.Enum:
     |  
     |  name
     |      The name of the Enum member.
     |  
     |  value
     |      The value of the Enum member.
     |  
     |  ----------------------------------------------------------------------
     |  Data descriptors inherited from enum.EnumMeta:
     |  
     |  \_\_members\_\_
     |      Returns a mapping of member name->value.
     |      
     |      This mapping lists all enum members, including aliases. Note that this
     |      is a read-only view of the internal mapping.
    
    class DatasetType(JEnum)
     |  该类定义了数据集类型常量。数据集一般为存储在一起的相关数据的集合;根据数据类型的不同,分为矢量数据集、栅格数据集和影像数据集,以及为了处理特定问题而设计的如拓扑数据集,网络
     |  数据集等。根据要素的空间特征的不同,矢量数据集又分为点数据集,线数据集,面数据集,复合数据集,文本数据集,纯属性数据集等。
     |  
     |  :var DatasetType.UNKNOWN: 未知类型数据集
     |  :var DatasetType.TABULAR: 纯属性数据集。用于存储和管理纯属性数据,纯属性数据用来描述地形地物特征、形状等信息,如河流的长度、宽度等。该数据
     |                            集没有空间图形数据。即纯属性数据集不能作为图层被添加到地图窗口中显示。
     |  :var DatasetType.POINT: 点数据集。用于存储点对象的数据集类,例如离散点的分布。
     |  :var DatasetType.LINE: 线数据集。用于存储线对象的数据集,例如河流、道路、国家边界线的分布。
     |  :var DatasetType.REGION: 多边形数据集。用于存储面对象的数据集,例如表示房屋的分布、行政区域等。
     |  :var DatasetType.TEXT: 文本数据集。用于存储文本对象的数据集,那么文本数据集中只能存储文本对象,而不能存储其他几何对象。例如表示注记的文本对象。
     |  :var DatasetType.CAD: 复合数据集。指可以存储多种几何对象的数据集,即用来存储点、线、面、文本等不同类型的对象的集合。CAD 数据集中各对象可以
     |                        有不同的风格,CAD 数据集为每个对象存储风格。
     |  :var DatasetType.LINKTABLE: 数据库表。即外挂属性表,不包含系统字段(以 SM 开头的字段)。与一般的属性数据集一样使用,但该数据集只具有只读功能。
     |  :var DatasetType.NETWORK: 网络数据集。网络数据集是用于存储具有网络拓扑关系的数据集。如道路交通网络等。网络数据集和点数据集、线数据集不同,
     |                            它既包含了网络线对象,也包含了网络结点对象,还包含了两种对象之间的空间拓扑关系。基于网络数据集,可以进行路径分析、
     |                            服务区分析、最近设施查找、选址分区、公交换乘以及邻接点、通达点分析等多种网络分析。
     |  :var DatasetType.NETWORK3D: 三维网络数据集,用于存储三维网络对象的数据集。
     |  :var DatasetType.LINEM: 路由数据集。是由一系列空间信息中带有刻度值Measure的线对象构成。通常可应用于线性参考模型或者作为网络分析的结果数据。
     |  :var DatasetType.PARAMETRICLINE: 复合参数化线数据集,用于存储复合参数化线几何对象的数据集。
     |  :var DatasetType.PARAMETRICREGION: 复合参数化面数据集,用于存储复合参数化面几何对象的数据集。
     |  :var DatasetType.GRIDCOLLECTION: 存储栅格数据集集合对象的数据集。对栅格数据集集合对象的详细描述请参考 :py:class:`DatasetGridCollection` 。
     |  :var DatasetType.IMAGECOLLECTION: 存储影像数据集集合对象的数据集。对影像数据集集合对象的详细描述请参考 :py:class:`DatasetImageCollection` 。
     |  :var DatasetType.MODEL: 模型数据集。
     |  :var DatasetType.TEXTURE:  纹理数据集,模型数据集的子数据集。
     |  :var DatasetType.IMAGE: 影像数据集。不具备属性信息,例如影像地图、多波段影像和实物地图等。其中每一个栅格存储的是一个颜色值或颜色的索引值(RGB 值)。
     |  :var DatasetType.WMS: WMS 数据集,是 DatasetImage 的一种类型。WMS (Web Map Service),即 Web 地图服务。WMS 利用具有地理空间位置
     |                        信息的数据制作地图。Web 地图服务返回的是图层级的地图影像。其中将地图定义为地理数据可视的表现。
     |  :var DatasetType.WCS: WCS 数据集,是 DatasetImage 的一种类型。 WCS( Web Coverage Service),即 Web 覆盖服务,面向空间影像数据,
     |                        它将包含地理位置值的地理空间数据作为“覆盖(Coverage)”在网上相互交换。
     |  :var DatasetType.GRID: 栅格数据集。例如高程数据集和土地利用图。其中每一个栅格存储的是表示地物的属性值(例如高程值)。
     |  :var DatasetType.VOLUME: 栅格体数据集合,以切片采样方式对三维体数据进行表达,例如指定空间范围的手机信号强度、雾霾污染指数等。
     |  :var DatasetType.TOPOLOGY: 拓扑数据集。拓扑数据集实际上是一个对拓扑错误提供综合管理能力的容器。它覆盖了拓扑关联数据集、拓扑规则、拓扑预处理、
     |                             拓扑错误生成以及定位修改、脏区自动维护等拓扑错误检查的关键要素,为拓扑错误检查提供了一套完整的解决方案。脏区指的
     |                             是未进行拓扑检查的区域,就已经进行了拓扑检查的区域,若用户在局部对数据进行了部分编辑时,则在此局部区域又将生成新的脏区。
     |  :var DatasetType.POINT3D: 三维点数据集,用于存储三维点对象的数据集。
     |  :var DatasetType.LINE3D: 三维线数据集,用于存储三维线对象的数据集。
     |  :var DatasetType.REGION3D: 三维面数据集,用于存储三维面对象的数据集。
     |  :var DatasetType.POINTEPS: 清华山维点数据集,用于存储清华山维点对象的数据集。
     |  :var DatasetType.LINEEPS: 清华山维线数据集,用于存储清华山维线对象的数据集。
     |  :var DatasetType.REGIONEPS: 清华山维面数据集,用于存储清华山维面对象的数据集。
     |  :var DatasetType.TEXTEPS: 清华山维文本数据集,用于存储清华山维文本对象的数据集。
     |  :var DatasetType.VECTORCOLLECTION: 矢量数据集集合,用于存储多个矢量数据集,仅支持 PostgreSQL 引擎。
     |  :var DatasetType.MOSAIC: 镶嵌数据集
     |  
     |  Method resolution order:
     |      DatasetType
     |      JEnum
     |      enum.IntEnum
     |      builtins.int
     |      enum.Enum
     |      builtins.object
     |  
     |  Data and other attributes defined here:
     |  
     |  CAD = DatasetType.CAD
     |  
     |  GRID = DatasetType.GRID
     |  
     |  GRIDCOLLECTION = DatasetType.GRIDCOLLECTION
     |  
     |  IMAGE = DatasetType.IMAGE
     |  
     |  IMAGECOLLECTION = DatasetType.IMAGECOLLECTION
     |  
     |  LINE = DatasetType.LINE
     |  
     |  LINE3D = DatasetType.LINE3D
     |  
     |  LINEEPS = DatasetType.LINEEPS
     |  
     |  LINEM = DatasetType.LINEM
     |  
     |  LINKTABLE = DatasetType.LINKTABLE
     |  
     |  MODEL = DatasetType.MODEL
     |  
     |  MOSAIC = DatasetType.MOSAIC
     |  
     |  NETWORK = DatasetType.NETWORK
     |  
     |  NETWORK3D = DatasetType.NETWORK3D
     |  
     |  PARAMETRICLINE = DatasetType.PARAMETRICLINE
     |  
     |  PARAMETRICREGION = DatasetType.PARAMETRICREGION
     |  
     |  POINT = DatasetType.POINT
     |  
     |  POINT3D = DatasetType.POINT3D
     |  
     |  POINTEPS = DatasetType.POINTEPS
     |  
     |  REGION = DatasetType.REGION
     |  
     |  REGION3D = DatasetType.REGION3D
     |  
     |  REGIONEPS = DatasetType.REGIONEPS
     |  
     |  TABULAR = DatasetType.TABULAR
     |  
     |  TEXT = DatasetType.TEXT
     |  
     |  TEXTEPS = DatasetType.TEXTEPS
     |  
     |  TEXTURE = DatasetType.TEXTURE
     |  
     |  TOPOLOGY = DatasetType.TOPOLOGY
     |  
     |  UNKNOWN = DatasetType.UNKNOWN
     |  
     |  VECTORCOLLECTION = DatasetType.VECTORCOLLECTION
     |  
     |  VOLUME = DatasetType.VOLUME
     |  
     |  WCS = DatasetType.WCS
     |  
     |  WMS = DatasetType.WMS
     |  
     |  ----------------------------------------------------------------------
     |  Data descriptors inherited from enum.Enum:
     |  
     |  name
     |      The name of the Enum member.
     |  
     |  value
     |      The value of the Enum member.
     |  
     |  ----------------------------------------------------------------------
     |  Data descriptors inherited from enum.EnumMeta:
     |  
     |  \_\_members\_\_
     |      Returns a mapping of member name->value.
     |      
     |      This mapping lists all enum members, including aliases. Note that this
     |      is a read-only view of the internal mapping.
    
    class DissolveType(JEnum)
     |  融合类型常量
     |  
     |  :var DissolveType.ONLYMULTIPART: 组合。将融合字段值相同的对象合并成一个复杂对象。
     |  :var DissolveType.SINGLE: 融合。将融合字段值相同且拓扑邻近的对象合并成一个简单对象。
     |  :var DissolveType.MULTIPART: 融合后组合。将融合字段值相同且拓扑邻近的对象合并成一个简单对象,然后将融合字段值相同的非邻近对象组合成一个复杂对象。
     |  
     |  Method resolution order:
     |      DissolveType
     |      JEnum
     |      enum.IntEnum
     |      builtins.int
     |      enum.Enum
     |      builtins.object
     |  
     |  Data and other attributes defined here:
     |  
     |  MULTIPART = DissolveType.MULTIPART
     |  
     |  ONLYMULTIPART = DissolveType.ONLYMULTIPART
     |  
     |  SINGLE = DissolveType.SINGLE
     |  
     |  ----------------------------------------------------------------------
     |  Data descriptors inherited from enum.Enum:
     |  
     |  name
     |      The name of the Enum member.
     |  
     |  value
     |      The value of the Enum member.
     |  
     |  ----------------------------------------------------------------------
     |  Data descriptors inherited from enum.EnumMeta:
     |  
     |  \_\_members\_\_
     |      Returns a mapping of member name->value.
     |      
     |      This mapping lists all enum members, including aliases. Note that this
     |      is a read-only view of the internal mapping.
    
    class DistanceMethod(JEnum)
     |  距离计算方法常量
     |  
     |  :var DistanceMethod.EUCLIDEAN: 欧式距离。计算两点间的直线距离。
     |  
     |                                 DistanceMethod\_EUCLIDEAN.png
     |  
     |  :var DistanceMethod.MANHATTAN: 曼哈顿距离。计算两点的x和y坐标的差值绝对值求和。该类型暂时不可用,仅作为测试,使用结果未知。
     |  
     |                                 DistanceMethod\_MANHATTAN.png
     |  
     |  Method resolution order:
     |      DistanceMethod
     |      JEnum
     |      enum.IntEnum
     |      builtins.int
     |      enum.Enum
     |      builtins.object
     |  
     |  Data and other attributes defined here:
     |  
     |  EUCLIDEAN = DistanceMethod.EUCLIDEAN
     |  
     |  MANHATTAN = DistanceMethod.MANHATTAN
     |  
     |  ----------------------------------------------------------------------
     |  Data descriptors inherited from enum.Enum:
     |  
     |  name
     |      The name of the Enum member.
     |  
     |  value
     |      The value of the Enum member.
     |  
     |  ----------------------------------------------------------------------
     |  Data descriptors inherited from enum.EnumMeta:
     |  
     |  \_\_members\_\_
     |      Returns a mapping of member name->value.
     |      
     |      This mapping lists all enum members, including aliases. Note that this
     |      is a read-only view of the internal mapping.
    
    class EdgeMatchMode(JEnum)
     |  该枚举定义了图幅接边的方式常量。
     |  
     |  :var EdgeMatchMode.THEOTHEREDGE: 向一边接边。接边连接点为接边目标数据集中发生接边关联的记录的端点,源数据集中接边关联到的记录的端点将移动到该连接点。
     |  :var EdgeMatchMode.THEMIDPOINT: 在中点位置接边。 接边连接点为接边目标数据集和源数据集中发生接边关联记录端点的中点,源和目标数据集中发生接边关联的记录的端点将移动到该连接点。
     |  :var EdgeMatchMode.THEINTERSECTION: 在交点位置接边。接边连接点为接边目标数据集和源数据集中发生接边关联记录端点的连线和接边线的交点,源和目标数据集中发生接边关联的记录的端点将移动到该连接点。
     |  
     |  Method resolution order:
     |      EdgeMatchMode
     |      JEnum
     |      enum.IntEnum
     |      builtins.int
     |      enum.Enum
     |      builtins.object
     |  
     |  Data and other attributes defined here:
     |  
     |  THEINTERSECTION = EdgeMatchMode.THEINTERSECTION
     |  
     |  THEMIDPOINT = EdgeMatchMode.THEMIDPOINT
     |  
     |  THEOTHEREDGE = EdgeMatchMode.THEOTHEREDGE
     |  
     |  ----------------------------------------------------------------------
     |  Data descriptors inherited from enum.Enum:
     |  
     |  name
     |      The name of the Enum member.
     |  
     |  value
     |      The value of the Enum member.
     |  
     |  ----------------------------------------------------------------------
     |  Data descriptors inherited from enum.EnumMeta:
     |  
     |  \_\_members\_\_
     |      Returns a mapping of member name->value.
     |      
     |      This mapping lists all enum members, including aliases. Note that this
     |      is a read-only view of the internal mapping.
    
    class EllipseSize(JEnum)
     |  输出椭圆的大小常量
     |  
     |  :var EllipseSize.SINGLE: 一个标准差。输出椭圆的长半轴和短半轴是对应的标准差的一倍。当几何对象具有空间正态分布时,即这些几
     |                           何对象在中心处集中而朝向外围时较少,则生成的椭圆将会包含约占总数68\%的几何对象在内。
     |  
     |                           .. image:: ../image/EllipseSize\_SINGLE.png
     |  
     |  
     |  :var EllipseSize.TWICE: 二个标准差。输出椭圆的长半轴和短半轴是对应的标准差的二倍。当几何对象具有空间正态分布时,即这些几
     |                          何对象在中心处集中而朝向外围时较少,则生成的椭圆将会包含约占总数95\%的几何对象在内。
     |  
     |                          .. image:: ../image/EllipseSize\_TWICE.png
     |  
     |  :var EllipseSize.TRIPLE: 三个标准差。输出椭圆的长半轴和短半轴是对应的标准差的三倍。当几何对象具有空间正态分布时,即这些几
     |                           何对象在中心处集中而朝向外围时较少,则生成的椭圆将会包含约占总数99\%的几何对象在内。
     |  
     |                           .. image:: ../image/EllipseSize\_TRIPLE.png
     |  
     |  Method resolution order:
     |      EllipseSize
     |      JEnum
     |      enum.IntEnum
     |      builtins.int
     |      enum.Enum
     |      builtins.object
     |  
     |  Data and other attributes defined here:
     |  
     |  SINGLE = EllipseSize.SINGLE
     |  
     |  TRIPLE = EllipseSize.TRIPLE
     |  
     |  TWICE = EllipseSize.TWICE
     |  
     |  ----------------------------------------------------------------------
     |  Data descriptors inherited from enum.Enum:
     |  
     |  name
     |      The name of the Enum member.
     |  
     |  value
     |      The value of the Enum member.
     |  
     |  ----------------------------------------------------------------------
     |  Data descriptors inherited from enum.EnumMeta:
     |  
     |  \_\_members\_\_
     |      Returns a mapping of member name->value.
     |      
     |      This mapping lists all enum members, including aliases. Note that this
     |      is a read-only view of the internal mapping.
    
    class EncodeType(JEnum)
     |  该类定义了数据集存储时的压缩编码方式类型常量。
     |  
     |  对矢量数据集,支持四种压缩编码方式,即单字节,双字节,三字节和四字节编码方式,这四种压缩编码方式采用相同的压缩编码机制,但是压缩的比率不同。
     |  其均为有损压缩。需要注意的是点数据集、纯属性数据集以及 CAD 数据集不可压缩编码。对光栅数据,可以采用四种压缩编码方式,即 DCT、SGL、LZW 和
     |  COMPOUND。其中 DCT 和 COMPOUND 为有损压缩编码方式,SGL 和 LZW 为无损压缩编码方式。
     |  
     |  对于影像和栅格数据集,根据其像素格式(PixelFormat)选择合适的压缩编码方式,对提高系统运行的效率,节省存储空间非常有利。下表列出了影像和栅格数
     |  据集不同像素格式对应的合理的编码方式:
     |  
     |  .. image:: ../image/EncodeTypeRec.png
     |  
     |  :var EncodeType.NONE: 不使用编码方式
     |  :var EncodeType.BYTE: 单字节编码方式。使用1个字节存储一个坐标值。(只适用于线和面数据集)
     |  :var EncodeType.INT16: 双字节编码方式。使用2个字节存储一个坐标值。(只适用于线和面数据集)
     |  :var EncodeType.INT24: 三字节编码方式。使用3个字节存储一个坐标值。(只适用于线和面数据集)
     |  :var EncodeType.INT32: 四字节编码方式。使用4个字节存储一个坐标值。(只适用于线和面数据集)
     |  :var EncodeType.DCT: DCT(Discrete Cosine Transform),离散余弦编码。是一种广泛应用于图像压缩中的变换编码方法,这种变换方法在信息的压
     |                       缩能力、重构图像质量、适应范围和算法复杂性等方面之间提供了一种很好的平衡,成为目前应用最广泛的图像压缩技术。其原理是通
     |                       过变换降低图像原始空间域表示中存在的非常强的相关性,使信号更紧凑地表达。该方法有很高的压缩率和性能,但编码是有失真的。
     |                       由于影像数据集一般不用来进行精确的分析,所以 DCT 编码方式是影像数据集存储的压缩编码方式。(适用于影像数据集)
     |  :var EncodeType.SGL: SGL(SuperMap Grid LZW),SuperMap 自定义的一种压缩存储格式。其实质是改进的 LZW 编码方式。SGL 对 LZW 进行了改
     |                       进,是一种更高效的压缩存储方式。目前 SuperMap 中的对 Grid 数据集和 DEM 数据集压缩存储采用的就是 SGL 的压缩编码方
     |                       式,这是一种无损压缩。(适用于栅格数据集)
     |  :var EncodeType.LZW: LZW 是一种广泛采用的字典压缩方法,其最早是用在文字数据的压缩方面。LZW的编码的原理是用代号来取代一段字符串,后续的相同
     |                       的字符串就使用相同代号,所以该编码方式不仅可以对重复数据起到压缩作用,还可以对不重复数据进行压缩操作。适用于索引色影像
     |                       的压缩方式,这是一种无损压缩编码方式。(适用于栅格和影像数据集)
     |  :var EncodeType.PNG: PNG 压缩编码方式,支持多种位深的图像,是一种无损压缩方式。(适用于影像数据集)
     |  :var EncodeType.COMPOUND: 数据集复合编码方式,其压缩比接近于 DCT 编码方式,主要针对 DCT 压缩导致的边界影像块失真的问题。(适用于 RGB 格式的影像数据集)
     |  
     |  Method resolution order:
     |      EncodeType
     |      JEnum
     |      enum.IntEnum
     |      builtins.int
     |      enum.Enum
     |      builtins.object
     |  
     |  Data and other attributes defined here:
     |  
     |  BYTE = EncodeType.BYTE
     |  
     |  COMPOUND = EncodeType.COMPOUND
     |  
     |  DCT = EncodeType.DCT
     |  
     |  INT16 = EncodeType.INT16
     |  
     |  INT24 = EncodeType.INT24
     |  
     |  INT32 = EncodeType.INT32
     |  
     |  LZW = EncodeType.LZW
     |  
     |  NONE = EncodeType.NONE
     |  
     |  PNG = EncodeType.PNG
     |  
     |  SGL = EncodeType.SGL
     |  
     |  ----------------------------------------------------------------------
     |  Data descriptors inherited from enum.Enum:
     |  
     |  name
     |      The name of the Enum member.
     |  
     |  value
     |      The value of the Enum member.
     |  
     |  ----------------------------------------------------------------------
     |  Data descriptors inherited from enum.EnumMeta:
     |  
     |  \_\_members\_\_
     |      Returns a mapping of member name->value.
     |      
     |      This mapping lists all enum members, including aliases. Note that this
     |      is a read-only view of the internal mapping.
    
    class EngineType(JEnum)
     |  该类定义了空间数据库引擎类型常量。
     |  空间数据库引擎是在常规数据库管理系统之上的,除具备常规数据库管理系统所必备的功能之外,还提供特定的针对空间数据的存储和管理能力。
     |  SuperMap SDX+ 是 supermap 的空间数据库技术,也是 SuperMap GIS 软件数据模型的重要组成部分。各种空间几何对象和影像数据都可以通过 SDX+
     |  引擎,存放到关系型数据库中,形成空间数据和属性数据一体化的空间数据库
     |  
     |  :var EngineType.IMAGEPLUGINS: 影像只读引擎类型,对应的枚举值为 5。针对通用影像格式如 BMP,JPG,TIFF 以及超图自定义影像格式 SIT,二维地图缓存配置文件格式SCI等。用户在进行二维地图缓存加载的时候,需要设置为此引擎类型,另外还需要使用 :py:meth:`DatasourceConnectionInfo.set\_server` 方法,将参数设置为二维地图缓存配置文件(SCI)。对于MrSID和ECW,只读打开为了快速原则,以合成波段的方式打开,非灰度数据会默认为RGB或者RGBA的方式显示,灰度数据按原始方式显示。
     |  :var EngineType.ORACLEPLUS:  Oracle 引擎类型
     |  :var EngineType.SQLPLUS: SQL Server 引擎类型,仅在 Windows 平台版本中支持
     |  :var EngineType.DB2: DB2 引擎类型
     |  :var EngineType.KINGBASE:  Kingbase 引擎类型,针对 Kingbase 数据源,不支持多波段数据
     |  :var EngineType.MEMORY:  内存数据源。
     |  :var EngineType.OGC: OGC 引擎类型,针对于 Web 数据源,对应的枚举值为 23。目前支持的类型有 WMS,WFS,WCS 和 WMTS。WMTS服务中默认BoundingBox和TopLeftCorner标签读取方式为(经度,纬度)。而一部分服务提供商提供的坐标格式为(纬度,经度),当你遇到这个情况时,为了保证坐标数据读取的正确性,请对SuperMap.xml文件(该文件位于Bin目录下)中相应的内容进行正确的修改。通常出现该情况的表现是本地矢量数据与发布的WMTS服务数据无法叠加到一起。
     |  :var EngineType.MYSQL:  MYSQL 引擎类型,支持 MySQL 5.6.16以上版本
     |  :var EngineType.MONGODB:  MongoDB 引擎类型,目前支持的认证方式为Mongodb-cr
     |  :var EngineType.BEYONDB:  BeyonDB 引擎类型
     |  :var EngineType.GBASE:  GBase 引擎类型
     |  :var EngineType.HIGHGODB:  HighGoDB 引擎类型
     |  :var EngineType.UDB: UDB 引擎类型
     |  :var EngineType.POSTGRESQL: PostgreSQL 引擎类型
     |  :var EngineType.GOOGLEMAPS: GoogleMaps 引擎类型,该引擎为只读引擎,且不能创建。该常量仅在 Windows 32 位平台版本中支持,在 Linux版本中不提供
     |  :var EngineType.SUPERMAPCLOUD: 超图云服务引擎类型,该引擎为只读引擎,且不能创建。该常量仅在 Windows 32 位平台版本中支持,在 Linux版本中不提供。
     |  :var EngineType.ISERVERREST: REST 地图服务引擎类型,该引擎为只读引擎,且不能创建。针对基于 REST 协议发布的地图服务。该常量仅在 Windows 32 位平台版本中支持,在 Linux版本中不提供。
     |  :var EngineType.BAIDUMAPS: 百度地图服务引擎类型
     |  :var EngineType.BINGMAPS: 必应地图服务引擎类型
     |  :var EngineType.OPENSTREETMAPS: OpenStreetMap 引擎类型。该常量仅在 Windows 32 位平台版本中支持,在 Linux版本中不提供
     |  :var EngineType.SCV: 矢量缓存引擎类型
     |  :var EngineType.DM: 第三代DM引擎类型
     |  :var EngineType.ORACLESPATIAL: Oracle Spatial 引擎类型
     |  :var EngineType.SDE: ArcSDE 引擎类型:
     |  
     |                       - 支持ArcSDE 9.2.0 及以上版本
     |                       - 支持ArcSDE 9.2.0 及以上版本的点、线、面、文本和栅格数据集5种数据类型的读取,不支持写。
     |                       - 不支持读取ArcSDE文本的风格,ArcSDE默认存放文本的字段“TEXTSTRING”不能删,否则我们读取不到文本。
     |                       - 不支持ArcSDE 2bit位深的栅格的读取,其它位深均支持,并可拉伸显示。
     |                       - 不支持多线程。
     |                       - 使用SDE引擎,需要ArcInfo的许可,并把ArcSDE安装目录bin下的 sde.dll 、sg.dll 和 pe.dll这三个dll拷贝到SuperMap产品下的Bin目录(即SuSDECI.dll 和 SuEngineSDE.sdx 同级目录)
     |                       - 支持平台:Windows 32位 ,Windows 64位。
     |  
     |  :var EngineType.ALTIBASE: Altibase 引擎类型
     |  :var EngineType.KDB: KDB 引擎类型
     |  :var EngineType.SRDB: 上容关系数据库引擎类型
     |  :var EngineType.MYSQLPlus: MySQLPlus数据库引擎类型,实质上为MySQL+Mongo
     |  :var EngineType.VECTORFILE: 矢量文件引擎类型。针对通用矢量格式如 shp,tab,Acad等,支持矢量文件的编辑和保存,如果是FME支持的类型则需要对应的FME许可,目前没有FME许可不支持FileGDBVector格式。
     |  :var EngineType.PGGIS: PostgreSQL的空间数据扩展PostGIS 引擎类型
     |  :var EngineType.ES: Elasticsearch 引擎类型
     |  :var EngineType.SQLSPATIAL: SQLSpatial 引擎类型
     |  :var EngineType.UDBX: UDBX 引擎类型
     |  
     |  Method resolution order:
     |      EngineType
     |      JEnum
     |      enum.IntEnum
     |      builtins.int
     |      enum.Enum
     |      builtins.object
     |  
     |  Data and other attributes defined here:
     |  
     |  ALTIBASE = EngineType.ALTIBASE
     |  
     |  BAIDUMAPS = EngineType.BAIDUMAPS
     |  
     |  BEYONDB = EngineType.BEYONDB
     |  
     |  BINGMAPS = EngineType.BINGMAPS
     |  
     |  DB2 = EngineType.DB2
     |  
     |  DM = EngineType.DM
     |  
     |  ES = EngineType.ES
     |  
     |  GBASE = EngineType.GBASE
     |  
     |  GOOGLEMAPS = EngineType.GOOGLEMAPS
     |  
     |  HIGHGODB = EngineType.HIGHGODB
     |  
     |  IMAGEPLUGINS = EngineType.IMAGEPLUGINS
     |  
     |  ISERVERREST = EngineType.ISERVERREST
     |  
     |  KDB = EngineType.KDB
     |  
     |  KINGBASE = EngineType.KINGBASE
     |  
     |  MEMORY = EngineType.MEMORY
     |  
     |  MONGODB = EngineType.MONGODB
     |  
     |  MYSQL = EngineType.MYSQL
     |  
     |  MYSQLPlus = EngineType.MYSQLPlus
     |  
     |  OGC = EngineType.OGC
     |  
     |  OPENSTREETMAPS = EngineType.OPENSTREETMAPS
     |  
     |  ORACLEPLUS = EngineType.ORACLEPLUS
     |  
     |  ORACLESPATIAL = EngineType.ORACLESPATIAL
     |  
     |  PGGIS = EngineType.PGGIS
     |  
     |  POSTGRESQL = EngineType.POSTGRESQL
     |  
     |  SCV = EngineType.SCV
     |  
     |  SDE = EngineType.SDE
     |  
     |  SQLPLUS = EngineType.SQLPLUS
     |  
     |  SQLSPATIAL = EngineType.SQLSPATIAL
     |  
     |  SRDB = EngineType.SRDB
     |  
     |  SUPERMAPCLOUD = EngineType.SUPERMAPCLOUD
     |  
     |  UDB = EngineType.UDB
     |  
     |  UDBX = EngineType.UDBX
     |  
     |  VECTORFILE = EngineType.VECTORFILE
     |  
     |  ----------------------------------------------------------------------
     |  Data descriptors inherited from enum.Enum:
     |  
     |  name
     |      The name of the Enum member.
     |  
     |  value
     |      The value of the Enum member.
     |  
     |  ----------------------------------------------------------------------
     |  Data descriptors inherited from enum.EnumMeta:
     |  
     |  \_\_members\_\_
     |      Returns a mapping of member name->value.
     |      
     |      This mapping lists all enum members, including aliases. Note that this
     |      is a read-only view of the internal mapping.
    
    class Exponent(JEnum)
     |  该类定义了泛克吕金(UniversalKriging)插值时样点数据中趋势面方程的阶数的类型常量。样点数据集中样点之间固有的某种趋势,可以通过函数或者多项式的拟合呈现。
     |  
     |  :var SearchMode.EXP1: 阶数为1,表示样点数据集中趋势面呈一阶趋势。
     |  :var SearchMode.EXP2: 阶数为2,表示样点数据集中趋势面呈二阶趋势。
     |  
     |  Method resolution order:
     |      Exponent
     |      JEnum
     |      enum.IntEnum
     |      builtins.int
     |      enum.Enum
     |      builtins.object
     |  
     |  Data and other attributes defined here:
     |  
     |  EXP1 = Exponent.EXP1
     |  
     |  EXP2 = Exponent.EXP2
     |  
     |  ----------------------------------------------------------------------
     |  Data descriptors inherited from enum.Enum:
     |  
     |  name
     |      The name of the Enum member.
     |  
     |  value
     |      The value of the Enum member.
     |  
     |  ----------------------------------------------------------------------
     |  Data descriptors inherited from enum.EnumMeta:
     |  
     |  \_\_members\_\_
     |      Returns a mapping of member name->value.
     |      
     |      This mapping lists all enum members, including aliases. Note that this
     |      is a read-only view of the internal mapping.
    
    class FieldType(JEnum)
     |  该类定义了字段类型常量。 定义一系列常量表示存储不同类型值的字段。
     |  
     |  :var FieldType.BOOLEAN: 布尔类型
     |  :var FieldType.BYTE: 字节型字段
     |  :var FieldType.INT16: 16位整型字段
     |  :var FieldType.INT32: 32位整型字段
     |  :var FieldType.INT64: 64位整型字段
     |  :var FieldType.SINGLE: 32位精度浮点型字段
     |  :var FieldType.DOUBLE: 64位精度浮点型字段
     |  :var FieldType.DATETIME: 日期型字段
     |  :var FieldType.LONGBINARY: 二进制型字段
     |  :var FieldType.TEXT: 变长的文本型字段
     |  :var FieldType.CHAR: 长的文本类型字段,例如指定的字符串长度为10,那么输入的字符串只有3个字符,则其他都用0来占位
     |  :var FieldType.WTEXT: 宽字符类型字段
     |  :var FieldType.JSONB: JSONB 类型字段(PostgreSQL独有字段)
     |  
     |  Method resolution order:
     |      FieldType
     |      JEnum
     |      enum.IntEnum
     |      builtins.int
     |      enum.Enum
     |      builtins.object
     |  
     |  Data and other attributes defined here:
     |  
     |  BOOLEAN = FieldType.BOOLEAN
     |  
     |  BYTE = FieldType.BYTE
     |  
     |  CHAR = FieldType.CHAR
     |  
     |  DATETIME = FieldType.DATETIME
     |  
     |  DOUBLE = FieldType.DOUBLE
     |  
     |  INT16 = FieldType.INT16
     |  
     |  INT32 = FieldType.INT32
     |  
     |  INT64 = FieldType.INT64
     |  
     |  JSONB = FieldType.JSONB
     |  
     |  LONGBINARY = FieldType.LONGBINARY
     |  
     |  SINGLE = FieldType.SINGLE
     |  
     |  TEXT = FieldType.TEXT
     |  
     |  WTEXT = FieldType.WTEXT
     |  
     |  ----------------------------------------------------------------------
     |  Data descriptors inherited from enum.Enum:
     |  
     |  name
     |      The name of the Enum member.
     |  
     |  value
     |      The value of the Enum member.
     |  
     |  ----------------------------------------------------------------------
     |  Data descriptors inherited from enum.EnumMeta:
     |  
     |  \_\_members\_\_
     |      Returns a mapping of member name->value.
     |      
     |      This mapping lists all enum members, including aliases. Note that this
     |      is a read-only view of the internal mapping.
    
    class FunctionType(JEnum)
     |  变换函数类型常量
     |  
     |  :var FunctionType.NONE: 不使用变换函数。
     |  :var FunctionType.LOG: 变换函数为log,要求原值大于0。
     |  :var FunctionType.ARCSIN: 变换函数为 arcsin,要求原值在范围[-1,1]内。
     |  
     |  Method resolution order:
     |      FunctionType
     |      JEnum
     |      enum.IntEnum
     |      builtins.int
     |      enum.Enum
     |      builtins.object
     |  
     |  Data and other attributes defined here:
     |  
     |  ARCSIN = FunctionType.ARCSIN
     |  
     |  LOG = FunctionType.LOG
     |  
     |  NONE = FunctionType.NONE
     |  
     |  ----------------------------------------------------------------------
     |  Data descriptors inherited from enum.Enum:
     |  
     |  name
     |      The name of the Enum member.
     |  
     |  value
     |      The value of the Enum member.
     |  
     |  ----------------------------------------------------------------------
     |  Data descriptors inherited from enum.EnumMeta:
     |  
     |  \_\_members\_\_
     |      Returns a mapping of member name->value.
     |      
     |      This mapping lists all enum members, including aliases. Note that this
     |      is a read-only view of the internal mapping.
    
    class GeoCoordSysType(JEnum)
     |  An enumeration.
     |  
     |  Method resolution order:
     |      GeoCoordSysType
     |      JEnum
     |      enum.IntEnum
     |      builtins.int
     |      enum.Enum
     |      builtins.object
     |  
     |  Data and other attributes defined here:
     |  
     |  GCS\_ADINDAN = GeoCoordSysType.GCS\_ADINDAN
     |  
     |  GCS\_AFGOOYE = GeoCoordSysType.GCS\_AFGOOYE
     |  
     |  GCS\_AGADEZ = GeoCoordSysType.GCS\_AGADEZ
     |  
     |  GCS\_AGD\_1966 = GeoCoordSysType.GCS\_AGD\_1966
     |  
     |  GCS\_AGD\_1984 = GeoCoordSysType.GCS\_AGD\_1984
     |  
     |  GCS\_AIN\_EL\_ABD\_1970 = GeoCoordSysType.GCS\_AIN\_EL\_ABD\_1970
     |  
     |  GCS\_AIRY\_1830 = GeoCoordSysType.GCS\_AIRY\_1830
     |  
     |  GCS\_AIRY\_MOD = GeoCoordSysType.GCS\_AIRY\_MOD
     |  
     |  GCS\_ALASKAN\_ISLANDS = GeoCoordSysType.GCS\_ALASKAN\_ISLANDS
     |  
     |  GCS\_AMERSFOORT = GeoCoordSysType.GCS\_AMERSFOORT
     |  
     |  GCS\_ANNA\_1\_1965 = GeoCoordSysType.GCS\_ANNA\_1\_1965
     |  
     |  GCS\_ANTIGUA\_ISLAND\_1943 = GeoCoordSysType.GCS\_ANTIGUA\_ISLAND\_1943
     |  
     |  GCS\_ARATU = GeoCoordSysType.GCS\_ARATU
     |  
     |  GCS\_ARC\_1950 = GeoCoordSysType.GCS\_ARC\_1950
     |  
     |  GCS\_ARC\_1960 = GeoCoordSysType.GCS\_ARC\_1960
     |  
     |  GCS\_ASCENSION\_ISLAND\_1958 = GeoCoordSysType.GCS\_ASCENSION\_ISLAND\_1958
     |  
     |  GCS\_ASTRO\_1952 = GeoCoordSysType.GCS\_ASTRO\_1952
     |  
     |  GCS\_ATF\_PARIS = GeoCoordSysType.GCS\_ATF\_PARIS
     |  
     |  GCS\_ATS\_1977 = GeoCoordSysType.GCS\_ATS\_1977
     |  
     |  GCS\_AUSTRALIAN = GeoCoordSysType.GCS\_AUSTRALIAN
     |  
     |  GCS\_AYABELLE = GeoCoordSysType.GCS\_AYABELLE
     |  
     |  GCS\_AZORES\_CENTRAL\_1948 = GeoCoordSysType.GCS\_AZORES\_CENTRAL\_1948
     |  
     |  GCS\_AZORES\_OCCIDENTAL\_1939 = GeoCoordSysType.GCS\_AZORES\_OCCIDENTAL\_193{\ldots}
     |  
     |  GCS\_AZORES\_ORIENTAL\_1940 = GeoCoordSysType.GCS\_AZORES\_ORIENTAL\_1940
     |  
     |  GCS\_BARBADOS = GeoCoordSysType.GCS\_BARBADOS
     |  
     |  GCS\_BATAVIA = GeoCoordSysType.GCS\_BATAVIA
     |  
     |  GCS\_BATAVIA\_JAKARTA = GeoCoordSysType.GCS\_BATAVIA\_JAKARTA
     |  
     |  GCS\_BEACON\_E\_1945 = GeoCoordSysType.GCS\_BEACON\_E\_1945
     |  
     |  GCS\_BEDUARAM = GeoCoordSysType.GCS\_BEDUARAM
     |  
     |  GCS\_BEIJING\_1954 = GeoCoordSysType.GCS\_BEIJING\_1954
     |  
     |  GCS\_BELGE\_1950 = GeoCoordSysType.GCS\_BELGE\_1950
     |  
     |  GCS\_BELGE\_1950\_BRUSSELS = GeoCoordSysType.GCS\_BELGE\_1950\_BRUSSELS
     |  
     |  GCS\_BELGE\_1972 = GeoCoordSysType.GCS\_BELGE\_1972
     |  
     |  GCS\_BELLEVUE = GeoCoordSysType.GCS\_BELLEVUE
     |  
     |  GCS\_BERMUDA\_1957 = GeoCoordSysType.GCS\_BERMUDA\_1957
     |  
     |  GCS\_BERN\_1898 = GeoCoordSysType.GCS\_BERN\_1898
     |  
     |  GCS\_BERN\_1898\_BERN = GeoCoordSysType.GCS\_BERN\_1898\_BERN
     |  
     |  GCS\_BERN\_1938 = GeoCoordSysType.GCS\_BERN\_1938
     |  
     |  GCS\_BESSEL\_1841 = GeoCoordSysType.GCS\_BESSEL\_1841
     |  
     |  GCS\_BESSEL\_MOD = GeoCoordSysType.GCS\_BESSEL\_MOD
     |  
     |  GCS\_BESSEL\_NAMIBIA = GeoCoordSysType.GCS\_BESSEL\_NAMIBIA
     |  
     |  GCS\_BISSAU = GeoCoordSysType.GCS\_BISSAU
     |  
     |  GCS\_BOGOTA = GeoCoordSysType.GCS\_BOGOTA
     |  
     |  GCS\_BOGOTA\_BOGOTA = GeoCoordSysType.GCS\_BOGOTA\_BOGOTA
     |  
     |  GCS\_BUKIT\_RIMPAH = GeoCoordSysType.GCS\_BUKIT\_RIMPAH
     |  
     |  GCS\_CACANAVERAL = GeoCoordSysType.GCS\_CACANAVERAL
     |  
     |  GCS\_CAMACUPA = GeoCoordSysType.GCS\_CAMACUPA
     |  
     |  GCS\_CAMPO\_INCHAUSPE = GeoCoordSysType.GCS\_CAMPO\_INCHAUSPE
     |  
     |  GCS\_CAMP\_AREA = GeoCoordSysType.GCS\_CAMP\_AREA
     |  
     |  GCS\_CANTON\_1966 = GeoCoordSysType.GCS\_CANTON\_1966
     |  
     |  GCS\_CAPE = GeoCoordSysType.GCS\_CAPE
     |  
     |  GCS\_CARTHAGE = GeoCoordSysType.GCS\_CARTHAGE
     |  
     |  GCS\_CARTHAGE\_DEGREE = GeoCoordSysType.GCS\_CARTHAGE\_DEGREE
     |  
     |  GCS\_CHATHAM\_ISLAND\_1971 = GeoCoordSysType.GCS\_CHATHAM\_ISLAND\_1971
     |  
     |  GCS\_CHINA\_2000 = GeoCoordSysType.GCS\_CHINA\_2000
     |  
     |  GCS\_CHUA = GeoCoordSysType.GCS\_CHUA
     |  
     |  GCS\_CLARKE\_1858 = GeoCoordSysType.GCS\_CLARKE\_1858
     |  
     |  GCS\_CLARKE\_1866 = GeoCoordSysType.GCS\_CLARKE\_1866
     |  
     |  GCS\_CLARKE\_1866\_MICH = GeoCoordSysType.GCS\_CLARKE\_1866\_MICH
     |  
     |  GCS\_CLARKE\_1880 = GeoCoordSysType.GCS\_CLARKE\_1880
     |  
     |  GCS\_CLARKE\_1880\_ARC = GeoCoordSysType.GCS\_CLARKE\_1880\_ARC
     |  
     |  GCS\_CLARKE\_1880\_BENOIT = GeoCoordSysType.GCS\_CLARKE\_1880\_BENOIT
     |  
     |  GCS\_CLARKE\_1880\_IGN = GeoCoordSysType.GCS\_CLARKE\_1880\_IGN
     |  
     |  GCS\_CLARKE\_1880\_RGS = GeoCoordSysType.GCS\_CLARKE\_1880\_RGS
     |  
     |  GCS\_CLARKE\_1880\_SGA = GeoCoordSysType.GCS\_CLARKE\_1880\_SGA
     |  
     |  GCS\_CONAKRY\_1905 = GeoCoordSysType.GCS\_CONAKRY\_1905
     |  
     |  GCS\_CORREGO\_ALEGRE = GeoCoordSysType.GCS\_CORREGO\_ALEGRE
     |  
     |  GCS\_COTE\_D\_IVOIRE = GeoCoordSysType.GCS\_COTE\_D\_IVOIRE
     |  
     |  GCS\_DABOLA = GeoCoordSysType.GCS\_DABOLA
     |  
     |  GCS\_DATUM\_73 = GeoCoordSysType.GCS\_DATUM\_73
     |  
     |  GCS\_DEALUL\_PISCULUI\_1933 = GeoCoordSysType.GCS\_DEALUL\_PISCULUI\_1933
     |  
     |  GCS\_DEALUL\_PISCULUI\_1970 = GeoCoordSysType.GCS\_DEALUL\_PISCULUI\_1970
     |  
     |  GCS\_DECEPTION\_ISLAND = GeoCoordSysType.GCS\_DECEPTION\_ISLAND
     |  
     |  GCS\_DEIR\_EZ\_ZOR = GeoCoordSysType.GCS\_DEIR\_EZ\_ZOR
     |  
     |  GCS\_DHDNB = GeoCoordSysType.GCS\_DHDNB
     |  
     |  GCS\_DOS\_1968 = GeoCoordSysType.GCS\_DOS\_1968
     |  
     |  GCS\_DOS\_71\_4 = GeoCoordSysType.GCS\_DOS\_71\_4
     |  
     |  GCS\_DOUALA = GeoCoordSysType.GCS\_DOUALA
     |  
     |  GCS\_EASTER\_ISLAND\_1967 = GeoCoordSysType.GCS\_EASTER\_ISLAND\_1967
     |  
     |  GCS\_ED\_1950 = GeoCoordSysType.GCS\_ED\_1950
     |  
     |  GCS\_ED\_1987 = GeoCoordSysType.GCS\_ED\_1987
     |  
     |  GCS\_EGYPT\_1907 = GeoCoordSysType.GCS\_EGYPT\_1907
     |  
     |  GCS\_ETRS\_1989 = GeoCoordSysType.GCS\_ETRS\_1989
     |  
     |  GCS\_EUROPEAN\_1979 = GeoCoordSysType.GCS\_EUROPEAN\_1979
     |  
     |  GCS\_EVEREST\_1830 = GeoCoordSysType.GCS\_EVEREST\_1830
     |  
     |  GCS\_EVEREST\_BANGLADESH = GeoCoordSysType.GCS\_EVEREST\_BANGLADESH
     |  
     |  GCS\_EVEREST\_DEF\_1967 = GeoCoordSysType.GCS\_EVEREST\_DEF\_1967
     |  
     |  GCS\_EVEREST\_DEF\_1975 = GeoCoordSysType.GCS\_EVEREST\_DEF\_1975
     |  
     |  GCS\_EVEREST\_INDIA\_NEPAL = GeoCoordSysType.GCS\_EVEREST\_INDIA\_NEPAL
     |  
     |  GCS\_EVEREST\_MOD = GeoCoordSysType.GCS\_EVEREST\_MOD
     |  
     |  GCS\_EVEREST\_MOD\_1969 = GeoCoordSysType.GCS\_EVEREST\_MOD\_1969
     |  
     |  GCS\_FAHUD = GeoCoordSysType.GCS\_FAHUD
     |  
     |  GCS\_FISCHER\_1960 = GeoCoordSysType.GCS\_FISCHER\_1960
     |  
     |  GCS\_FISCHER\_1968 = GeoCoordSysType.GCS\_FISCHER\_1968
     |  
     |  GCS\_FISCHER\_MOD = GeoCoordSysType.GCS\_FISCHER\_MOD
     |  
     |  GCS\_FORT\_THOMAS\_1955 = GeoCoordSysType.GCS\_FORT\_THOMAS\_1955
     |  
     |  GCS\_GANDAJIKA\_1970 = GeoCoordSysType.GCS\_GANDAJIKA\_1970
     |  
     |  GCS\_GAN\_1970 = GeoCoordSysType.GCS\_GAN\_1970
     |  
     |  GCS\_GAROUA = GeoCoordSysType.GCS\_GAROUA
     |  
     |  GCS\_GDA\_1994 = GeoCoordSysType.GCS\_GDA\_1994
     |  
     |  GCS\_GEM\_10C = GeoCoordSysType.GCS\_GEM\_10C
     |  
     |  GCS\_GGRS\_1987 = GeoCoordSysType.GCS\_GGRS\_1987
     |  
     |  GCS\_GRACIOSA\_1948 = GeoCoordSysType.GCS\_GRACIOSA\_1948
     |  
     |  GCS\_GREEK = GeoCoordSysType.GCS\_GREEK
     |  
     |  GCS\_GREEK\_ATHENS = GeoCoordSysType.GCS\_GREEK\_ATHENS
     |  
     |  GCS\_GRS\_1967 = GeoCoordSysType.GCS\_GRS\_1967
     |  
     |  GCS\_GRS\_1980 = GeoCoordSysType.GCS\_GRS\_1980
     |  
     |  GCS\_GUAM\_1963 = GeoCoordSysType.GCS\_GUAM\_1963
     |  
     |  GCS\_GUNUNG\_SEGARA = GeoCoordSysType.GCS\_GUNUNG\_SEGARA
     |  
     |  GCS\_GUX\_1 = GeoCoordSysType.GCS\_GUX\_1
     |  
     |  GCS\_GUYANE\_FRANCAISE = GeoCoordSysType.GCS\_GUYANE\_FRANCAISE
     |  
     |  GCS\_HELMERT\_1906 = GeoCoordSysType.GCS\_HELMERT\_1906
     |  
     |  GCS\_HERAT\_NORTH = GeoCoordSysType.GCS\_HERAT\_NORTH
     |  
     |  GCS\_HITO\_XVIII\_1963 = GeoCoordSysType.GCS\_HITO\_XVIII\_1963
     |  
     |  GCS\_HJORSEY\_1955 = GeoCoordSysType.GCS\_HJORSEY\_1955
     |  
     |  GCS\_HONG\_KONG\_1963 = GeoCoordSysType.GCS\_HONG\_KONG\_1963
     |  
     |  GCS\_HOUGH\_1960 = GeoCoordSysType.GCS\_HOUGH\_1960
     |  
     |  GCS\_HUNGARIAN\_1972 = GeoCoordSysType.GCS\_HUNGARIAN\_1972
     |  
     |  GCS\_HU\_TZU\_SHAN = GeoCoordSysType.GCS\_HU\_TZU\_SHAN
     |  
     |  GCS\_INDIAN\_1954 = GeoCoordSysType.GCS\_INDIAN\_1954
     |  
     |  GCS\_INDIAN\_1960 = GeoCoordSysType.GCS\_INDIAN\_1960
     |  
     |  GCS\_INDIAN\_1975 = GeoCoordSysType.GCS\_INDIAN\_1975
     |  
     |  GCS\_INDONESIAN = GeoCoordSysType.GCS\_INDONESIAN
     |  
     |  GCS\_INDONESIAN\_1974 = GeoCoordSysType.GCS\_INDONESIAN\_1974
     |  
     |  GCS\_INTERNATIONAL\_1924 = GeoCoordSysType.GCS\_INTERNATIONAL\_1924
     |  
     |  GCS\_INTERNATIONAL\_1967 = GeoCoordSysType.GCS\_INTERNATIONAL\_1967
     |  
     |  GCS\_ISTS\_061\_1968 = GeoCoordSysType.GCS\_ISTS\_061\_1968
     |  
     |  GCS\_ISTS\_073\_1969 = GeoCoordSysType.GCS\_ISTS\_073\_1969
     |  
     |  GCS\_ITRF\_1993 = GeoCoordSysType.GCS\_ITRF\_1993
     |  
     |  GCS\_JAMAICA\_1875 = GeoCoordSysType.GCS\_JAMAICA\_1875
     |  
     |  GCS\_JAMAICA\_1969 = GeoCoordSysType.GCS\_JAMAICA\_1969
     |  
     |  GCS\_JAPAN\_2000 = GeoCoordSysType.GCS\_JAPAN\_2000
     |  
     |  GCS\_JOHNSTON\_ISLAND\_1961 = GeoCoordSysType.GCS\_JOHNSTON\_ISLAND\_1961
     |  
     |  GCS\_KALIANPUR = GeoCoordSysType.GCS\_KALIANPUR
     |  
     |  GCS\_KANDAWALA = GeoCoordSysType.GCS\_KANDAWALA
     |  
     |  GCS\_KERGUELEN\_ISLAND\_1949 = GeoCoordSysType.GCS\_KERGUELEN\_ISLAND\_1949
     |  
     |  GCS\_KERTAU = GeoCoordSysType.GCS\_KERTAU
     |  
     |  GCS\_KKJ = GeoCoordSysType.GCS\_KKJ
     |  
     |  GCS\_KOC\_ = GeoCoordSysType.GCS\_KOC\_
     |  
     |  GCS\_KRASOVSKY\_1940 = GeoCoordSysType.GCS\_KRASOVSKY\_1940
     |  
     |  GCS\_KUDAMS = GeoCoordSysType.GCS\_KUDAMS
     |  
     |  GCS\_KUSAIE\_1951 = GeoCoordSysType.GCS\_KUSAIE\_1951
     |  
     |  GCS\_LAKE = GeoCoordSysType.GCS\_LAKE
     |  
     |  GCS\_LA\_CANOA = GeoCoordSysType.GCS\_LA\_CANOA
     |  
     |  GCS\_LC5\_1961 = GeoCoordSysType.GCS\_LC5\_1961
     |  
     |  GCS\_LEIGON = GeoCoordSysType.GCS\_LEIGON
     |  
     |  GCS\_LIBERIA\_1964 = GeoCoordSysType.GCS\_LIBERIA\_1964
     |  
     |  GCS\_LISBON = GeoCoordSysType.GCS\_LISBON
     |  
     |  GCS\_LISBON\_1890 = GeoCoordSysType.GCS\_LISBON\_1890
     |  
     |  GCS\_LISBON\_LISBON = GeoCoordSysType.GCS\_LISBON\_LISBON
     |  
     |  GCS\_LOMA\_QUINTANA = GeoCoordSysType.GCS\_LOMA\_QUINTANA
     |  
     |  GCS\_LOME = GeoCoordSysType.GCS\_LOME
     |  
     |  GCS\_LUZON\_1911 = GeoCoordSysType.GCS\_LUZON\_1911
     |  
     |  GCS\_MADEIRA\_1936 = GeoCoordSysType.GCS\_MADEIRA\_1936
     |  
     |  GCS\_MAHE\_1971 = GeoCoordSysType.GCS\_MAHE\_1971
     |  
     |  GCS\_MAKASSAR = GeoCoordSysType.GCS\_MAKASSAR
     |  
     |  GCS\_MAKASSAR\_JAKARTA = GeoCoordSysType.GCS\_MAKASSAR\_JAKARTA
     |  
     |  GCS\_MALONGO\_1987 = GeoCoordSysType.GCS\_MALONGO\_1987
     |  
     |  GCS\_MANOCA = GeoCoordSysType.GCS\_MANOCA
     |  
     |  GCS\_MASSAWA = GeoCoordSysType.GCS\_MASSAWA
     |  
     |  GCS\_MERCHICH = GeoCoordSysType.GCS\_MERCHICH
     |  
     |  GCS\_MGI\_ = GeoCoordSysType.GCS\_MGI\_
     |  
     |  GCS\_MGI\_FERRO = GeoCoordSysType.GCS\_MGI\_FERRO
     |  
     |  GCS\_MHAST = GeoCoordSysType.GCS\_MHAST
     |  
     |  GCS\_MIDWAY\_1961 = GeoCoordSysType.GCS\_MIDWAY\_1961
     |  
     |  GCS\_MINNA = GeoCoordSysType.GCS\_MINNA
     |  
     |  GCS\_MONTE\_MARIO = GeoCoordSysType.GCS\_MONTE\_MARIO
     |  
     |  GCS\_MONTE\_MARIO\_ROME = GeoCoordSysType.GCS\_MONTE\_MARIO\_ROME
     |  
     |  GCS\_MONTSERRAT\_ISLAND\_1958 = GeoCoordSysType.GCS\_MONTSERRAT\_ISLAND\_195{\ldots}
     |  
     |  GCS\_MPORALOKO = GeoCoordSysType.GCS\_MPORALOKO
     |  
     |  GCS\_NAD\_1927 = GeoCoordSysType.GCS\_NAD\_1927
     |  
     |  GCS\_NAD\_1983 = GeoCoordSysType.GCS\_NAD\_1983
     |  
     |  GCS\_NAD\_MICH = GeoCoordSysType.GCS\_NAD\_MICH
     |  
     |  GCS\_NAHRWAN\_1967 = GeoCoordSysType.GCS\_NAHRWAN\_1967
     |  
     |  GCS\_NAPARIMA\_1972 = GeoCoordSysType.GCS\_NAPARIMA\_1972
     |  
     |  GCS\_NDG\_PARIS = GeoCoordSysType.GCS\_NDG\_PARIS
     |  
     |  GCS\_NGN = GeoCoordSysType.GCS\_NGN
     |  
     |  GCS\_NGO\_1948\_ = GeoCoordSysType.GCS\_NGO\_1948\_
     |  
     |  GCS\_NORD\_SAHARA\_1959 = GeoCoordSysType.GCS\_NORD\_SAHARA\_1959
     |  
     |  GCS\_NSWC\_9Z\_2\_ = GeoCoordSysType.GCS\_NSWC\_9Z\_2\_
     |  
     |  GCS\_NTF\_ = GeoCoordSysType.GCS\_NTF\_
     |  
     |  GCS\_NTF\_PARIS = GeoCoordSysType.GCS\_NTF\_PARIS
     |  
     |  GCS\_NWL\_9D = GeoCoordSysType.GCS\_NWL\_9D
     |  
     |  GCS\_NZGD\_1949 = GeoCoordSysType.GCS\_NZGD\_1949
     |  
     |  GCS\_OBSERV\_METEOR\_1939 = GeoCoordSysType.GCS\_OBSERV\_METEOR\_1939
     |  
     |  GCS\_OLD\_HAWAIIAN = GeoCoordSysType.GCS\_OLD\_HAWAIIAN
     |  
     |  GCS\_OMAN = GeoCoordSysType.GCS\_OMAN
     |  
     |  GCS\_OSGB\_1936 = GeoCoordSysType.GCS\_OSGB\_1936
     |  
     |  GCS\_OSGB\_1970\_SN = GeoCoordSysType.GCS\_OSGB\_1970\_SN
     |  
     |  GCS\_OSU\_86F = GeoCoordSysType.GCS\_OSU\_86F
     |  
     |  GCS\_OSU\_91A = GeoCoordSysType.GCS\_OSU\_91A
     |  
     |  GCS\_OS\_SN\_1980 = GeoCoordSysType.GCS\_OS\_SN\_1980
     |  
     |  GCS\_PADANG\_1884 = GeoCoordSysType.GCS\_PADANG\_1884
     |  
     |  GCS\_PADANG\_1884\_JAKARTA = GeoCoordSysType.GCS\_PADANG\_1884\_JAKARTA
     |  
     |  GCS\_PALESTINE\_1923 = GeoCoordSysType.GCS\_PALESTINE\_1923
     |  
     |  GCS\_PICO\_DE\_LAS\_NIEVES = GeoCoordSysType.GCS\_PICO\_DE\_LAS\_NIEVES
     |  
     |  GCS\_PITCAIRN\_1967 = GeoCoordSysType.GCS\_PITCAIRN\_1967
     |  
     |  GCS\_PLESSIS\_1817 = GeoCoordSysType.GCS\_PLESSIS\_1817
     |  
     |  GCS\_POINT58 = GeoCoordSysType.GCS\_POINT58
     |  
     |  GCS\_POINTE\_NOIRE = GeoCoordSysType.GCS\_POINTE\_NOIRE
     |  
     |  GCS\_PORTO\_SANTO\_1936 = GeoCoordSysType.GCS\_PORTO\_SANTO\_1936
     |  
     |  GCS\_PSAD\_1956 = GeoCoordSysType.GCS\_PSAD\_1956
     |  
     |  GCS\_PUERTO\_RICO = GeoCoordSysType.GCS\_PUERTO\_RICO
     |  
     |  GCS\_PULKOVO\_1942 = GeoCoordSysType.GCS\_PULKOVO\_1942
     |  
     |  GCS\_PULKOVO\_1995 = GeoCoordSysType.GCS\_PULKOVO\_1995
     |  
     |  GCS\_QATAR = GeoCoordSysType.GCS\_QATAR
     |  
     |  GCS\_QATAR\_1948 = GeoCoordSysType.GCS\_QATAR\_1948
     |  
     |  GCS\_QORNOQ = GeoCoordSysType.GCS\_QORNOQ
     |  
     |  GCS\_REUNION = GeoCoordSysType.GCS\_REUNION
     |  
     |  GCS\_RT38\_ = GeoCoordSysType.GCS\_RT38\_
     |  
     |  GCS\_RT38\_STOCKHOLM = GeoCoordSysType.GCS\_RT38\_STOCKHOLM
     |  
     |  GCS\_S42\_HUNGARY = GeoCoordSysType.GCS\_S42\_HUNGARY
     |  
     |  GCS\_SAD\_1969 = GeoCoordSysType.GCS\_SAD\_1969
     |  
     |  GCS\_SAMOA\_1962 = GeoCoordSysType.GCS\_SAMOA\_1962
     |  
     |  GCS\_SANTO\_DOS\_1965 = GeoCoordSysType.GCS\_SANTO\_DOS\_1965
     |  
     |  GCS\_SAO\_BRAZ = GeoCoordSysType.GCS\_SAO\_BRAZ
     |  
     |  GCS\_SAPPER\_HILL\_1943 = GeoCoordSysType.GCS\_SAPPER\_HILL\_1943
     |  
     |  GCS\_SCHWARZECK = GeoCoordSysType.GCS\_SCHWARZECK
     |  
     |  GCS\_SEGORA = GeoCoordSysType.GCS\_SEGORA
     |  
     |  GCS\_SELVAGEM\_GRANDE\_1938 = GeoCoordSysType.GCS\_SELVAGEM\_GRANDE\_1938
     |  
     |  GCS\_SERINDUNG = GeoCoordSysType.GCS\_SERINDUNG
     |  
     |  GCS\_SPHERE = GeoCoordSysType.GCS\_SPHERE
     |  
     |  GCS\_SPHERE\_AI = GeoCoordSysType.GCS\_SPHERE\_AI
     |  
     |  GCS\_STRUVE\_1860 = GeoCoordSysType.GCS\_STRUVE\_1860
     |  
     |  GCS\_SUDAN = GeoCoordSysType.GCS\_SUDAN
     |  
     |  GCS\_S\_ASIA\_SINGAPORE = GeoCoordSysType.GCS\_S\_ASIA\_SINGAPORE
     |  
     |  GCS\_S\_JTSK = GeoCoordSysType.GCS\_S\_JTSK
     |  
     |  GCS\_TANANARIVE\_1925 = GeoCoordSysType.GCS\_TANANARIVE\_1925
     |  
     |  GCS\_TANANARIVE\_1925\_PARIS = GeoCoordSysType.GCS\_TANANARIVE\_1925\_PARIS
     |  
     |  GCS\_TERN\_ISLAND\_1961 = GeoCoordSysType.GCS\_TERN\_ISLAND\_1961
     |  
     |  GCS\_TIMBALAI\_1948 = GeoCoordSysType.GCS\_TIMBALAI\_1948
     |  
     |  GCS\_TM65 = GeoCoordSysType.GCS\_TM65
     |  
     |  GCS\_TM75 = GeoCoordSysType.GCS\_TM75
     |  
     |  GCS\_TOKYO = GeoCoordSysType.GCS\_TOKYO
     |  
     |  GCS\_TRINIDAD\_1903 = GeoCoordSysType.GCS\_TRINIDAD\_1903
     |  
     |  GCS\_TRISTAN\_1968 = GeoCoordSysType.GCS\_TRISTAN\_1968
     |  
     |  GCS\_TRUCIAL\_COAST\_1948 = GeoCoordSysType.GCS\_TRUCIAL\_COAST\_1948
     |  
     |  GCS\_USER\_DEFINE = GeoCoordSysType.GCS\_USER\_DEFINE
     |  
     |  GCS\_VITI\_LEVU\_1916 = GeoCoordSysType.GCS\_VITI\_LEVU\_1916
     |  
     |  GCS\_VOIROL\_1875 = GeoCoordSysType.GCS\_VOIROL\_1875
     |  
     |  GCS\_VOIROL\_1875\_PARIS = GeoCoordSysType.GCS\_VOIROL\_1875\_PARIS
     |  
     |  GCS\_VOIROL\_UNIFIE\_1960 = GeoCoordSysType.GCS\_VOIROL\_UNIFIE\_1960
     |  
     |  GCS\_VOIROL\_UNIFIE\_1960\_PARIS = GeoCoordSysType.GCS\_VOIROL\_UNIFIE\_1960\_{\ldots}
     |  
     |  GCS\_WAKE\_ENIWETOK\_1960 = GeoCoordSysType.GCS\_WAKE\_ENIWETOK\_1960
     |  
     |  GCS\_WAKE\_ISLAND\_1952 = GeoCoordSysType.GCS\_WAKE\_ISLAND\_1952
     |  
     |  GCS\_WALBECK = GeoCoordSysType.GCS\_WALBECK
     |  
     |  GCS\_WAR\_OFFICE = GeoCoordSysType.GCS\_WAR\_OFFICE
     |  
     |  GCS\_WGS\_1966 = GeoCoordSysType.GCS\_WGS\_1966
     |  
     |  GCS\_WGS\_1972 = GeoCoordSysType.GCS\_WGS\_1972
     |  
     |  GCS\_WGS\_1972\_BE = GeoCoordSysType.GCS\_WGS\_1972\_BE
     |  
     |  GCS\_WGS\_1984 = GeoCoordSysType.GCS\_WGS\_1984
     |  
     |  GCS\_XIAN\_1980 = GeoCoordSysType.GCS\_XIAN\_1980
     |  
     |  GCS\_YACARE = GeoCoordSysType.GCS\_YACARE
     |  
     |  GCS\_YOFF = GeoCoordSysType.GCS\_YOFF
     |  
     |  GCS\_ZANDERIJ = GeoCoordSysType.GCS\_ZANDERIJ
     |  
     |  ----------------------------------------------------------------------
     |  Data descriptors inherited from enum.Enum:
     |  
     |  name
     |      The name of the Enum member.
     |  
     |  value
     |      The value of the Enum member.
     |  
     |  ----------------------------------------------------------------------
     |  Data descriptors inherited from enum.EnumMeta:
     |  
     |  \_\_members\_\_
     |      Returns a mapping of member name->value.
     |      
     |      This mapping lists all enum members, including aliases. Note that this
     |      is a read-only view of the internal mapping.
    
    class GeoDatumType(JEnum)
     |  An enumeration.
     |  
     |  Method resolution order:
     |      GeoDatumType
     |      JEnum
     |      enum.IntEnum
     |      builtins.int
     |      enum.Enum
     |      builtins.object
     |  
     |  Data and other attributes defined here:
     |  
     |  DATUM\_ADINDAN = GeoDatumType.DATUM\_ADINDAN
     |  
     |  DATUM\_AFGOOYE = GeoDatumType.DATUM\_AFGOOYE
     |  
     |  DATUM\_AGADEZ = GeoDatumType.DATUM\_AGADEZ
     |  
     |  DATUM\_AGD\_1966 = GeoDatumType.DATUM\_AGD\_1966
     |  
     |  DATUM\_AGD\_1984 = GeoDatumType.DATUM\_AGD\_1984
     |  
     |  DATUM\_AIN\_EL\_ABD\_1970 = GeoDatumType.DATUM\_AIN\_EL\_ABD\_1970
     |  
     |  DATUM\_AIRY\_1830 = GeoDatumType.DATUM\_AIRY\_1830
     |  
     |  DATUM\_AIRY\_MOD = GeoDatumType.DATUM\_AIRY\_MOD
     |  
     |  DATUM\_ALASKAN\_ISLANDS = GeoDatumType.DATUM\_ALASKAN\_ISLANDS
     |  
     |  DATUM\_AMERSFOORT = GeoDatumType.DATUM\_AMERSFOORT
     |  
     |  DATUM\_ANNA\_1\_1965 = GeoDatumType.DATUM\_ANNA\_1\_1965
     |  
     |  DATUM\_ANTIGUA\_ISLAND\_1943 = GeoDatumType.DATUM\_ANTIGUA\_ISLAND\_1943
     |  
     |  DATUM\_ARATU = GeoDatumType.DATUM\_ARATU
     |  
     |  DATUM\_ARC\_1950 = GeoDatumType.DATUM\_ARC\_1950
     |  
     |  DATUM\_ARC\_1960 = GeoDatumType.DATUM\_ARC\_1960
     |  
     |  DATUM\_ASCENSION\_ISLAND\_1958 = GeoDatumType.DATUM\_ASCENSION\_ISLAND\_1958
     |  
     |  DATUM\_ASTRO\_1952 = GeoDatumType.DATUM\_ASTRO\_1952
     |  
     |  DATUM\_ATF = GeoDatumType.DATUM\_ATF
     |  
     |  DATUM\_ATS\_1977 = GeoDatumType.DATUM\_ATS\_1977
     |  
     |  DATUM\_AUSTRALIAN = GeoDatumType.DATUM\_AUSTRALIAN
     |  
     |  DATUM\_AYABELLE = GeoDatumType.DATUM\_AYABELLE
     |  
     |  DATUM\_BARBADOS = GeoDatumType.DATUM\_BARBADOS
     |  
     |  DATUM\_BATAVIA = GeoDatumType.DATUM\_BATAVIA
     |  
     |  DATUM\_BEACON\_E\_1945 = GeoDatumType.DATUM\_BEACON\_E\_1945
     |  
     |  DATUM\_BEDUARAM = GeoDatumType.DATUM\_BEDUARAM
     |  
     |  DATUM\_BEIJING\_1954 = GeoDatumType.DATUM\_BEIJING\_1954
     |  
     |  DATUM\_BELGE\_1950 = GeoDatumType.DATUM\_BELGE\_1950
     |  
     |  DATUM\_BELGE\_1972 = GeoDatumType.DATUM\_BELGE\_1972
     |  
     |  DATUM\_BELLEVUE = GeoDatumType.DATUM\_BELLEVUE
     |  
     |  DATUM\_BERMUDA\_1957 = GeoDatumType.DATUM\_BERMUDA\_1957
     |  
     |  DATUM\_BERN\_1898 = GeoDatumType.DATUM\_BERN\_1898
     |  
     |  DATUM\_BERN\_1938 = GeoDatumType.DATUM\_BERN\_1938
     |  
     |  DATUM\_BESSEL\_1841 = GeoDatumType.DATUM\_BESSEL\_1841
     |  
     |  DATUM\_BESSEL\_MOD = GeoDatumType.DATUM\_BESSEL\_MOD
     |  
     |  DATUM\_BESSEL\_NAMIBIA = GeoDatumType.DATUM\_BESSEL\_NAMIBIA
     |  
     |  DATUM\_BISSAU = GeoDatumType.DATUM\_BISSAU
     |  
     |  DATUM\_BOGOTA = GeoDatumType.DATUM\_BOGOTA
     |  
     |  DATUM\_BUKIT\_RIMPAH = GeoDatumType.DATUM\_BUKIT\_RIMPAH
     |  
     |  DATUM\_CACANAVERAL = GeoDatumType.DATUM\_CACANAVERAL
     |  
     |  DATUM\_CAMACUPA = GeoDatumType.DATUM\_CAMACUPA
     |  
     |  DATUM\_CAMPO\_INCHAUSPE = GeoDatumType.DATUM\_CAMPO\_INCHAUSPE
     |  
     |  DATUM\_CAMP\_AREA = GeoDatumType.DATUM\_CAMP\_AREA
     |  
     |  DATUM\_CANTON\_1966 = GeoDatumType.DATUM\_CANTON\_1966
     |  
     |  DATUM\_CAPE = GeoDatumType.DATUM\_CAPE
     |  
     |  DATUM\_CARTHAGE = GeoDatumType.DATUM\_CARTHAGE
     |  
     |  DATUM\_CHATHAM\_ISLAND\_1971 = GeoDatumType.DATUM\_CHATHAM\_ISLAND\_1971
     |  
     |  DATUM\_CHINA\_2000 = GeoDatumType.DATUM\_CHINA\_2000
     |  
     |  DATUM\_CHUA = GeoDatumType.DATUM\_CHUA
     |  
     |  DATUM\_CLARKE\_1858 = GeoDatumType.DATUM\_CLARKE\_1858
     |  
     |  DATUM\_CLARKE\_1866 = GeoDatumType.DATUM\_CLARKE\_1866
     |  
     |  DATUM\_CLARKE\_1866\_MICH = GeoDatumType.DATUM\_CLARKE\_1866\_MICH
     |  
     |  DATUM\_CLARKE\_1880 = GeoDatumType.DATUM\_CLARKE\_1880
     |  
     |  DATUM\_CLARKE\_1880\_ARC = GeoDatumType.DATUM\_CLARKE\_1880\_ARC
     |  
     |  DATUM\_CLARKE\_1880\_BENOIT = GeoDatumType.DATUM\_CLARKE\_1880\_BENOIT
     |  
     |  DATUM\_CLARKE\_1880\_IGN = GeoDatumType.DATUM\_CLARKE\_1880\_IGN
     |  
     |  DATUM\_CLARKE\_1880\_RGS = GeoDatumType.DATUM\_CLARKE\_1880\_RGS
     |  
     |  DATUM\_CLARKE\_1880\_SGA = GeoDatumType.DATUM\_CLARKE\_1880\_SGA
     |  
     |  DATUM\_CONAKRY\_1905 = GeoDatumType.DATUM\_CONAKRY\_1905
     |  
     |  DATUM\_CORREGO\_ALEGRE = GeoDatumType.DATUM\_CORREGO\_ALEGRE
     |  
     |  DATUM\_COTE\_D\_IVOIRE = GeoDatumType.DATUM\_COTE\_D\_IVOIRE
     |  
     |  DATUM\_DABOLA = GeoDatumType.DATUM\_DABOLA
     |  
     |  DATUM\_DATUM\_73 = GeoDatumType.DATUM\_DATUM\_73
     |  
     |  DATUM\_DEALUL\_PISCULUI\_1933 = GeoDatumType.DATUM\_DEALUL\_PISCULUI\_1933
     |  
     |  DATUM\_DEALUL\_PISCULUI\_1970 = GeoDatumType.DATUM\_DEALUL\_PISCULUI\_1970
     |  
     |  DATUM\_DECEPTION\_ISLAND = GeoDatumType.DATUM\_DECEPTION\_ISLAND
     |  
     |  DATUM\_DEIR\_EZ\_ZOR = GeoDatumType.DATUM\_DEIR\_EZ\_ZOR
     |  
     |  DATUM\_DHDN = GeoDatumType.DATUM\_DHDN
     |  
     |  DATUM\_DOS\_1968 = GeoDatumType.DATUM\_DOS\_1968
     |  
     |  DATUM\_DOS\_71\_4 = GeoDatumType.DATUM\_DOS\_71\_4
     |  
     |  DATUM\_DOUALA = GeoDatumType.DATUM\_DOUALA
     |  
     |  DATUM\_EASTER\_ISLAND\_1967 = GeoDatumType.DATUM\_EASTER\_ISLAND\_1967
     |  
     |  DATUM\_ED\_1950 = GeoDatumType.DATUM\_ED\_1950
     |  
     |  DATUM\_ED\_1987 = GeoDatumType.DATUM\_ED\_1987
     |  
     |  DATUM\_EGYPT\_1907 = GeoDatumType.DATUM\_EGYPT\_1907
     |  
     |  DATUM\_ETRS\_1989 = GeoDatumType.DATUM\_ETRS\_1989
     |  
     |  DATUM\_EUROPEAN\_1979 = GeoDatumType.DATUM\_EUROPEAN\_1979
     |  
     |  DATUM\_EVEREST\_1830 = GeoDatumType.DATUM\_EVEREST\_1830
     |  
     |  DATUM\_EVEREST\_BANGLADESH = GeoDatumType.DATUM\_EVEREST\_BANGLADESH
     |  
     |  DATUM\_EVEREST\_DEF\_1967 = GeoDatumType.DATUM\_EVEREST\_DEF\_1967
     |  
     |  DATUM\_EVEREST\_DEF\_1975 = GeoDatumType.DATUM\_EVEREST\_DEF\_1975
     |  
     |  DATUM\_EVEREST\_INDIA\_NEPAL = GeoDatumType.DATUM\_EVEREST\_INDIA\_NEPAL
     |  
     |  DATUM\_EVEREST\_MOD = GeoDatumType.DATUM\_EVEREST\_MOD
     |  
     |  DATUM\_EVEREST\_MOD\_1969 = GeoDatumType.DATUM\_EVEREST\_MOD\_1969
     |  
     |  DATUM\_FAHUD = GeoDatumType.DATUM\_FAHUD
     |  
     |  DATUM\_FISCHER\_1960 = GeoDatumType.DATUM\_FISCHER\_1960
     |  
     |  DATUM\_FISCHER\_1968 = GeoDatumType.DATUM\_FISCHER\_1968
     |  
     |  DATUM\_FISCHER\_MOD = GeoDatumType.DATUM\_FISCHER\_MOD
     |  
     |  DATUM\_FORT\_THOMAS\_1955 = GeoDatumType.DATUM\_FORT\_THOMAS\_1955
     |  
     |  DATUM\_GANDAJIKA\_1970 = GeoDatumType.DATUM\_GANDAJIKA\_1970
     |  
     |  DATUM\_GAN\_1970 = GeoDatumType.DATUM\_GAN\_1970
     |  
     |  DATUM\_GAROUA = GeoDatumType.DATUM\_GAROUA
     |  
     |  DATUM\_GDA\_1994 = GeoDatumType.DATUM\_GDA\_1994
     |  
     |  DATUM\_GEM\_10C = GeoDatumType.DATUM\_GEM\_10C
     |  
     |  DATUM\_GGRS\_1987 = GeoDatumType.DATUM\_GGRS\_1987
     |  
     |  DATUM\_GRACIOSA\_1948 = GeoDatumType.DATUM\_GRACIOSA\_1948
     |  
     |  DATUM\_GREEK = GeoDatumType.DATUM\_GREEK
     |  
     |  DATUM\_GRS\_1967 = GeoDatumType.DATUM\_GRS\_1967
     |  
     |  DATUM\_GRS\_1980 = GeoDatumType.DATUM\_GRS\_1980
     |  
     |  DATUM\_GUAM\_1963 = GeoDatumType.DATUM\_GUAM\_1963
     |  
     |  DATUM\_GUNUNG\_SEGARA = GeoDatumType.DATUM\_GUNUNG\_SEGARA
     |  
     |  DATUM\_GUX\_1 = GeoDatumType.DATUM\_GUX\_1
     |  
     |  DATUM\_GUYANE\_FRANCAISE = GeoDatumType.DATUM\_GUYANE\_FRANCAISE
     |  
     |  DATUM\_HELMERT\_1906 = GeoDatumType.DATUM\_HELMERT\_1906
     |  
     |  DATUM\_HERAT\_NORTH = GeoDatumType.DATUM\_HERAT\_NORTH
     |  
     |  DATUM\_HITO\_XVIII\_1963 = GeoDatumType.DATUM\_HITO\_XVIII\_1963
     |  
     |  DATUM\_HJORSEY\_1955 = GeoDatumType.DATUM\_HJORSEY\_1955
     |  
     |  DATUM\_HONG\_KONG\_1963 = GeoDatumType.DATUM\_HONG\_KONG\_1963
     |  
     |  DATUM\_HOUGH\_1960 = GeoDatumType.DATUM\_HOUGH\_1960
     |  
     |  DATUM\_HUNGARIAN\_1972 = GeoDatumType.DATUM\_HUNGARIAN\_1972
     |  
     |  DATUM\_HU\_TZU\_SHAN = GeoDatumType.DATUM\_HU\_TZU\_SHAN
     |  
     |  DATUM\_INDIAN\_1954 = GeoDatumType.DATUM\_INDIAN\_1954
     |  
     |  DATUM\_INDIAN\_1960 = GeoDatumType.DATUM\_INDIAN\_1960
     |  
     |  DATUM\_INDIAN\_1975 = GeoDatumType.DATUM\_INDIAN\_1975
     |  
     |  DATUM\_INDONESIAN = GeoDatumType.DATUM\_INDONESIAN
     |  
     |  DATUM\_INDONESIAN\_1974 = GeoDatumType.DATUM\_INDONESIAN\_1974
     |  
     |  DATUM\_INTERNATIONAL\_1924 = GeoDatumType.DATUM\_INTERNATIONAL\_1924
     |  
     |  DATUM\_INTERNATIONAL\_1967 = GeoDatumType.DATUM\_INTERNATIONAL\_1967
     |  
     |  DATUM\_ISTS\_061\_1968 = GeoDatumType.DATUM\_ISTS\_061\_1968
     |  
     |  DATUM\_ISTS\_073\_1969 = GeoDatumType.DATUM\_ISTS\_073\_1969
     |  
     |  DATUM\_JAMAICA\_1875 = GeoDatumType.DATUM\_JAMAICA\_1875
     |  
     |  DATUM\_JAMAICA\_1969 = GeoDatumType.DATUM\_JAMAICA\_1969
     |  
     |  DATUM\_JAPAN\_2000 = GeoDatumType.DATUM\_JAPAN\_2000
     |  
     |  DATUM\_JOHNSTON\_ISLAND\_1961 = GeoDatumType.DATUM\_JOHNSTON\_ISLAND\_1961
     |  
     |  DATUM\_KALIANPUR = GeoDatumType.DATUM\_KALIANPUR
     |  
     |  DATUM\_KANDAWALA = GeoDatumType.DATUM\_KANDAWALA
     |  
     |  DATUM\_KERGUELEN\_ISLAND\_1949 = GeoDatumType.DATUM\_KERGUELEN\_ISLAND\_1949
     |  
     |  DATUM\_KERTAU = GeoDatumType.DATUM\_KERTAU
     |  
     |  DATUM\_KKJ = GeoDatumType.DATUM\_KKJ
     |  
     |  DATUM\_KOC = GeoDatumType.DATUM\_KOC
     |  
     |  DATUM\_KRASOVSKY\_1940 = GeoDatumType.DATUM\_KRASOVSKY\_1940
     |  
     |  DATUM\_KUDAMS = GeoDatumType.DATUM\_KUDAMS
     |  
     |  DATUM\_KUSAIE\_1951 = GeoDatumType.DATUM\_KUSAIE\_1951
     |  
     |  DATUM\_LAKE = GeoDatumType.DATUM\_LAKE
     |  
     |  DATUM\_LA\_CANOA = GeoDatumType.DATUM\_LA\_CANOA
     |  
     |  DATUM\_LC5\_1961 = GeoDatumType.DATUM\_LC5\_1961
     |  
     |  DATUM\_LEIGON = GeoDatumType.DATUM\_LEIGON
     |  
     |  DATUM\_LIBERIA\_1964 = GeoDatumType.DATUM\_LIBERIA\_1964
     |  
     |  DATUM\_LISBON = GeoDatumType.DATUM\_LISBON
     |  
     |  DATUM\_LOMA\_QUINTANA = GeoDatumType.DATUM\_LOMA\_QUINTANA
     |  
     |  DATUM\_LOME = GeoDatumType.DATUM\_LOME
     |  
     |  DATUM\_LUZON\_1911 = GeoDatumType.DATUM\_LUZON\_1911
     |  
     |  DATUM\_MAHE\_1971 = GeoDatumType.DATUM\_MAHE\_1971
     |  
     |  DATUM\_MAKASSAR = GeoDatumType.DATUM\_MAKASSAR
     |  
     |  DATUM\_MALONGO\_1987 = GeoDatumType.DATUM\_MALONGO\_1987
     |  
     |  DATUM\_MANOCA = GeoDatumType.DATUM\_MANOCA
     |  
     |  DATUM\_MASSAWA = GeoDatumType.DATUM\_MASSAWA
     |  
     |  DATUM\_MERCHICH = GeoDatumType.DATUM\_MERCHICH
     |  
     |  DATUM\_MGI = GeoDatumType.DATUM\_MGI
     |  
     |  DATUM\_MHAST = GeoDatumType.DATUM\_MHAST
     |  
     |  DATUM\_MIDWAY\_1961 = GeoDatumType.DATUM\_MIDWAY\_1961
     |  
     |  DATUM\_MINNA = GeoDatumType.DATUM\_MINNA
     |  
     |  DATUM\_MONTE\_MARIO = GeoDatumType.DATUM\_MONTE\_MARIO
     |  
     |  DATUM\_MONTSERRAT\_ISLAND\_1958 = GeoDatumType.DATUM\_MONTSERRAT\_ISLAND\_19{\ldots}
     |  
     |  DATUM\_MPORALOKO = GeoDatumType.DATUM\_MPORALOKO
     |  
     |  DATUM\_NAD\_1927 = GeoDatumType.DATUM\_NAD\_1927
     |  
     |  DATUM\_NAD\_1983 = GeoDatumType.DATUM\_NAD\_1983
     |  
     |  DATUM\_NAD\_MICH = GeoDatumType.DATUM\_NAD\_MICH
     |  
     |  DATUM\_NAHRWAN\_1967 = GeoDatumType.DATUM\_NAHRWAN\_1967
     |  
     |  DATUM\_NAPARIMA\_1972 = GeoDatumType.DATUM\_NAPARIMA\_1972
     |  
     |  DATUM\_NDG = GeoDatumType.DATUM\_NDG
     |  
     |  DATUM\_NGN = GeoDatumType.DATUM\_NGN
     |  
     |  DATUM\_NGO\_1948 = GeoDatumType.DATUM\_NGO\_1948
     |  
     |  DATUM\_NORD\_SAHARA\_1959 = GeoDatumType.DATUM\_NORD\_SAHARA\_1959
     |  
     |  DATUM\_NSWC\_9Z\_2 = GeoDatumType.DATUM\_NSWC\_9Z\_2
     |  
     |  DATUM\_NTF = GeoDatumType.DATUM\_NTF
     |  
     |  DATUM\_NWL\_9D = GeoDatumType.DATUM\_NWL\_9D
     |  
     |  DATUM\_NZGD\_1949 = GeoDatumType.DATUM\_NZGD\_1949
     |  
     |  DATUM\_OBSERV\_METEOR\_1939 = GeoDatumType.DATUM\_OBSERV\_METEOR\_1939
     |  
     |  DATUM\_OLD\_HAWAIIAN = GeoDatumType.DATUM\_OLD\_HAWAIIAN
     |  
     |  DATUM\_OMAN = GeoDatumType.DATUM\_OMAN
     |  
     |  DATUM\_OSGB\_1936 = GeoDatumType.DATUM\_OSGB\_1936
     |  
     |  DATUM\_OSGB\_1970\_SN = GeoDatumType.DATUM\_OSGB\_1970\_SN
     |  
     |  DATUM\_OSU\_86F = GeoDatumType.DATUM\_OSU\_86F
     |  
     |  DATUM\_OSU\_91A = GeoDatumType.DATUM\_OSU\_91A
     |  
     |  DATUM\_OS\_SN\_1980 = GeoDatumType.DATUM\_OS\_SN\_1980
     |  
     |  DATUM\_PADANG\_1884 = GeoDatumType.DATUM\_PADANG\_1884
     |  
     |  DATUM\_PALESTINE\_1923 = GeoDatumType.DATUM\_PALESTINE\_1923
     |  
     |  DATUM\_PICO\_DE\_LAS\_NIEVES = GeoDatumType.DATUM\_PICO\_DE\_LAS\_NIEVES
     |  
     |  DATUM\_PITCAIRN\_1967 = GeoDatumType.DATUM\_PITCAIRN\_1967
     |  
     |  DATUM\_PLESSIS\_1817 = GeoDatumType.DATUM\_PLESSIS\_1817
     |  
     |  DATUM\_POINT58 = GeoDatumType.DATUM\_POINT58
     |  
     |  DATUM\_POINTE\_NOIRE = GeoDatumType.DATUM\_POINTE\_NOIRE
     |  
     |  DATUM\_POPULAR\_VISUALISATION = GeoDatumType.DATUM\_POPULAR\_VISUALISATION
     |  
     |  DATUM\_PORTO\_SANTO\_1936 = GeoDatumType.DATUM\_PORTO\_SANTO\_1936
     |  
     |  DATUM\_PSAD\_1956 = GeoDatumType.DATUM\_PSAD\_1956
     |  
     |  DATUM\_PUERTO\_RICO = GeoDatumType.DATUM\_PUERTO\_RICO
     |  
     |  DATUM\_PULKOVO\_1942 = GeoDatumType.DATUM\_PULKOVO\_1942
     |  
     |  DATUM\_PULKOVO\_1995 = GeoDatumType.DATUM\_PULKOVO\_1995
     |  
     |  DATUM\_QATAR = GeoDatumType.DATUM\_QATAR
     |  
     |  DATUM\_QATAR\_1948 = GeoDatumType.DATUM\_QATAR\_1948
     |  
     |  DATUM\_QORNOQ = GeoDatumType.DATUM\_QORNOQ
     |  
     |  DATUM\_REUNION = GeoDatumType.DATUM\_REUNION
     |  
     |  DATUM\_S42\_HUNGARY = GeoDatumType.DATUM\_S42\_HUNGARY
     |  
     |  DATUM\_SAD\_1969 = GeoDatumType.DATUM\_SAD\_1969
     |  
     |  DATUM\_SAMOA\_1962 = GeoDatumType.DATUM\_SAMOA\_1962
     |  
     |  DATUM\_SANTO\_DOS\_1965 = GeoDatumType.DATUM\_SANTO\_DOS\_1965
     |  
     |  DATUM\_SAO\_BRAZ = GeoDatumType.DATUM\_SAO\_BRAZ
     |  
     |  DATUM\_SAPPER\_HILL\_1943 = GeoDatumType.DATUM\_SAPPER\_HILL\_1943
     |  
     |  DATUM\_SCHWARZECK = GeoDatumType.DATUM\_SCHWARZECK
     |  
     |  DATUM\_SEGORA = GeoDatumType.DATUM\_SEGORA
     |  
     |  DATUM\_SELVAGEM\_GRANDE\_1938 = GeoDatumType.DATUM\_SELVAGEM\_GRANDE\_1938
     |  
     |  DATUM\_SERINDUNG = GeoDatumType.DATUM\_SERINDUNG
     |  
     |  DATUM\_SPHERE = GeoDatumType.DATUM\_SPHERE
     |  
     |  DATUM\_SPHERE\_AI = GeoDatumType.DATUM\_SPHERE\_AI
     |  
     |  DATUM\_STOCKHOLM\_1938 = GeoDatumType.DATUM\_STOCKHOLM\_1938
     |  
     |  DATUM\_STRUVE\_1860 = GeoDatumType.DATUM\_STRUVE\_1860
     |  
     |  DATUM\_SUDAN = GeoDatumType.DATUM\_SUDAN
     |  
     |  DATUM\_S\_ASIA\_SINGAPORE = GeoDatumType.DATUM\_S\_ASIA\_SINGAPORE
     |  
     |  DATUM\_S\_JTSK = GeoDatumType.DATUM\_S\_JTSK
     |  
     |  DATUM\_TANANARIVE\_1925 = GeoDatumType.DATUM\_TANANARIVE\_1925
     |  
     |  DATUM\_TERN\_ISLAND\_1961 = GeoDatumType.DATUM\_TERN\_ISLAND\_1961
     |  
     |  DATUM\_TIMBALAI\_1948 = GeoDatumType.DATUM\_TIMBALAI\_1948
     |  
     |  DATUM\_TM65 = GeoDatumType.DATUM\_TM65
     |  
     |  DATUM\_TM75 = GeoDatumType.DATUM\_TM75
     |  
     |  DATUM\_TOKYO = GeoDatumType.DATUM\_TOKYO
     |  
     |  DATUM\_TRINIDAD\_1903 = GeoDatumType.DATUM\_TRINIDAD\_1903
     |  
     |  DATUM\_TRISTAN\_1968 = GeoDatumType.DATUM\_TRISTAN\_1968
     |  
     |  DATUM\_TRUCIAL\_COAST\_1948 = GeoDatumType.DATUM\_TRUCIAL\_COAST\_1948
     |  
     |  DATUM\_USER\_DEFINED = GeoDatumType.DATUM\_USER\_DEFINED
     |  
     |  DATUM\_VITI\_LEVU\_1916 = GeoDatumType.DATUM\_VITI\_LEVU\_1916
     |  
     |  DATUM\_VOIROL\_1875 = GeoDatumType.DATUM\_VOIROL\_1875
     |  
     |  DATUM\_VOIROL\_UNIFIE\_1960 = GeoDatumType.DATUM\_VOIROL\_UNIFIE\_1960
     |  
     |  DATUM\_WAKE\_ENIWETOK\_1960 = GeoDatumType.DATUM\_WAKE\_ENIWETOK\_1960
     |  
     |  DATUM\_WAKE\_ISLAND\_1952 = GeoDatumType.DATUM\_WAKE\_ISLAND\_1952
     |  
     |  DATUM\_WALBECK = GeoDatumType.DATUM\_WALBECK
     |  
     |  DATUM\_WAR\_OFFICE = GeoDatumType.DATUM\_WAR\_OFFICE
     |  
     |  DATUM\_WGS\_1966 = GeoDatumType.DATUM\_WGS\_1966
     |  
     |  DATUM\_WGS\_1972 = GeoDatumType.DATUM\_WGS\_1972
     |  
     |  DATUM\_WGS\_1972\_BE = GeoDatumType.DATUM\_WGS\_1972\_BE
     |  
     |  DATUM\_WGS\_1984 = GeoDatumType.DATUM\_WGS\_1984
     |  
     |  DATUM\_XIAN\_1980 = GeoDatumType.DATUM\_XIAN\_1980
     |  
     |  DATUM\_YACARE = GeoDatumType.DATUM\_YACARE
     |  
     |  DATUM\_YOFF = GeoDatumType.DATUM\_YOFF
     |  
     |  DATUM\_ZANDERIJ = GeoDatumType.DATUM\_ZANDERIJ
     |  
     |  ----------------------------------------------------------------------
     |  Data descriptors inherited from enum.Enum:
     |  
     |  name
     |      The name of the Enum member.
     |  
     |  value
     |      The value of the Enum member.
     |  
     |  ----------------------------------------------------------------------
     |  Data descriptors inherited from enum.EnumMeta:
     |  
     |  \_\_members\_\_
     |      Returns a mapping of member name->value.
     |      
     |      This mapping lists all enum members, including aliases. Note that this
     |      is a read-only view of the internal mapping.
    
    class GeoPrimeMeridianType(JEnum)
     |  该类定义了中央经线类型常量。
     |  
     |  :var GeoPrimeMeridianType.PRIMEMERIDIAN\_USER\_DEFINED: 用户自定义
     |  :var GeoPrimeMeridianType.PRIMEMERIDIAN\_GREENWICH: 格林威治本初子午线,即0°经线
     |  :var GeoPrimeMeridianType.PRIMEMERIDIAN\_LISBON: 9°07'54".862 W
     |  :var GeoPrimeMeridianType.PRIMEMERIDIAN\_PARIS: 2°20'14".025 E
     |  :var GeoPrimeMeridianType.PRIMEMERIDIAN\_BOGOTA: 74°04'51".3 W
     |  :var GeoPrimeMeridianType.PRIMEMERIDIAN\_MADRID: 3°41'16".58 W
     |  :var GeoPrimeMeridianType.PRIMEMERIDIAN\_ROME: 12°27'08".4 E
     |  :var GeoPrimeMeridianType.PRIMEMERIDIAN\_BERN: 7°26'22".5 E
     |  :var GeoPrimeMeridianType.PRIMEMERIDIAN\_JAKARTA: 106°48'27".79 E
     |  :var GeoPrimeMeridianType.PRIMEMERIDIAN\_FERRO: 17°40'00" W
     |  :var GeoPrimeMeridianType.PRIMEMERIDIAN\_BRUSSELS: 4°22'04".71 E
     |  :var GeoPrimeMeridianType.PRIMEMERIDIAN\_STOCKHOLM: 18°03'29".8 E
     |  :var GeoPrimeMeridianType.PRIMEMERIDIAN\_ATHENS: 23°42'58".815 E
     |  
     |  Method resolution order:
     |      GeoPrimeMeridianType
     |      JEnum
     |      enum.IntEnum
     |      builtins.int
     |      enum.Enum
     |      builtins.object
     |  
     |  Data and other attributes defined here:
     |  
     |  PRIMEMERIDIAN\_ATHENS = GeoPrimeMeridianType.PRIMEMERIDIAN\_ATHENS
     |  
     |  PRIMEMERIDIAN\_BERN = GeoPrimeMeridianType.PRIMEMERIDIAN\_BERN
     |  
     |  PRIMEMERIDIAN\_BOGOTA = GeoPrimeMeridianType.PRIMEMERIDIAN\_BOGOTA
     |  
     |  PRIMEMERIDIAN\_BRUSSELS = GeoPrimeMeridianType.PRIMEMERIDIAN\_BRUSSELS
     |  
     |  PRIMEMERIDIAN\_FERRO = GeoPrimeMeridianType.PRIMEMERIDIAN\_FERRO
     |  
     |  PRIMEMERIDIAN\_GREENWICH = GeoPrimeMeridianType.PRIMEMERIDIAN\_GREENWICH
     |  
     |  PRIMEMERIDIAN\_JAKARTA = GeoPrimeMeridianType.PRIMEMERIDIAN\_JAKARTA
     |  
     |  PRIMEMERIDIAN\_LISBON = GeoPrimeMeridianType.PRIMEMERIDIAN\_LISBON
     |  
     |  PRIMEMERIDIAN\_MADRID = GeoPrimeMeridianType.PRIMEMERIDIAN\_MADRID
     |  
     |  PRIMEMERIDIAN\_PARIS = GeoPrimeMeridianType.PRIMEMERIDIAN\_PARIS
     |  
     |  PRIMEMERIDIAN\_ROME = GeoPrimeMeridianType.PRIMEMERIDIAN\_ROME
     |  
     |  PRIMEMERIDIAN\_STOCKHOLM = GeoPrimeMeridianType.PRIMEMERIDIAN\_STOCKHOLM
     |  
     |  PRIMEMERIDIAN\_USER\_DEFINED = GeoPrimeMeridianType.PRIMEMERIDIAN\_USER\_D{\ldots}
     |  
     |  ----------------------------------------------------------------------
     |  Data descriptors inherited from enum.Enum:
     |  
     |  name
     |      The name of the Enum member.
     |  
     |  value
     |      The value of the Enum member.
     |  
     |  ----------------------------------------------------------------------
     |  Data descriptors inherited from enum.EnumMeta:
     |  
     |  \_\_members\_\_
     |      Returns a mapping of member name->value.
     |      
     |      This mapping lists all enum members, including aliases. Note that this
     |      is a read-only view of the internal mapping.
    
    class GeoSpatialRefType(JEnum)
     |  该类定义了空间坐标系类型常量。
     |  
     |  空间坐标系类型,用以区分平面坐标系、地理坐标系、投影坐标系,其中地理坐标系又称为经纬度坐标系。
     |  
     |  :var GeoSpatialRefType.SPATIALREF\_NONEARTH: 平面坐标系。当坐标系为平面坐标系时,不能进行投影转换。
     |  :var GeoSpatialRefType.SPATIALREF\_EARTH\_LONGITUDE\_LATITUDE: 地理坐标系。地理坐标系由大地参照系、中央经线、坐标单位组成。在地理坐标系中,单位可以是度,分,秒。东西向(水平方向)的范围为-180度至180度。南北向(垂直方向)的范围为-90度至90度。
     |  :var GeoSpatialRefType.SPATIALREF\_EARTH\_PROJECTION: 投影坐标系。投影坐标系统由地图投影方式、投影参数、坐标单位和地理坐标系组成。SuperMap Objects Java 中提供了很多预定义的投影系统,用户可以直接使用,此外,用户还可以定制自己的投影系统。
     |  
     |  Method resolution order:
     |      GeoSpatialRefType
     |      JEnum
     |      enum.IntEnum
     |      builtins.int
     |      enum.Enum
     |      builtins.object
     |  
     |  Data and other attributes defined here:
     |  
     |  SPATIALREF\_EARTH\_LONGITUDE\_LATITUDE = GeoSpatialRefType.SPATIALREF\_EAR{\ldots}
     |  
     |  SPATIALREF\_EARTH\_PROJECTION = GeoSpatialRefType.SPATIALREF\_EARTH\_PROJE{\ldots}
     |  
     |  SPATIALREF\_NONEARTH = GeoSpatialRefType.SPATIALREF\_NONEARTH
     |  
     |  ----------------------------------------------------------------------
     |  Data descriptors inherited from enum.Enum:
     |  
     |  name
     |      The name of the Enum member.
     |  
     |  value
     |      The value of the Enum member.
     |  
     |  ----------------------------------------------------------------------
     |  Data descriptors inherited from enum.EnumMeta:
     |  
     |  \_\_members\_\_
     |      Returns a mapping of member name->value.
     |      
     |      This mapping lists all enum members, including aliases. Note that this
     |      is a read-only view of the internal mapping.
    
    class GeoSpheroidType(JEnum)
     |  An enumeration.
     |  
     |  Method resolution order:
     |      GeoSpheroidType
     |      JEnum
     |      enum.IntEnum
     |      builtins.int
     |      enum.Enum
     |      builtins.object
     |  
     |  Data and other attributes defined here:
     |  
     |  SPHEROID\_AIRY\_1830 = GeoSpheroidType.SPHEROID\_AIRY\_1830
     |  
     |  SPHEROID\_AIRY\_MOD = GeoSpheroidType.SPHEROID\_AIRY\_MOD
     |  
     |  SPHEROID\_ATS\_1977 = GeoSpheroidType.SPHEROID\_ATS\_1977
     |  
     |  SPHEROID\_AUSTRALIAN = GeoSpheroidType.SPHEROID\_AUSTRALIAN
     |  
     |  SPHEROID\_BESSEL\_1841 = GeoSpheroidType.SPHEROID\_BESSEL\_1841
     |  
     |  SPHEROID\_BESSEL\_MOD = GeoSpheroidType.SPHEROID\_BESSEL\_MOD
     |  
     |  SPHEROID\_BESSEL\_NAMIBIA = GeoSpheroidType.SPHEROID\_BESSEL\_NAMIBIA
     |  
     |  SPHEROID\_CHINA\_2000 = GeoSpheroidType.SPHEROID\_CHINA\_2000
     |  
     |  SPHEROID\_CLARKE\_1858 = GeoSpheroidType.SPHEROID\_CLARKE\_1858
     |  
     |  SPHEROID\_CLARKE\_1866 = GeoSpheroidType.SPHEROID\_CLARKE\_1866
     |  
     |  SPHEROID\_CLARKE\_1866\_MICH = GeoSpheroidType.SPHEROID\_CLARKE\_1866\_MICH
     |  
     |  SPHEROID\_CLARKE\_1880 = GeoSpheroidType.SPHEROID\_CLARKE\_1880
     |  
     |  SPHEROID\_CLARKE\_1880\_ARC = GeoSpheroidType.SPHEROID\_CLARKE\_1880\_ARC
     |  
     |  SPHEROID\_CLARKE\_1880\_BENOIT = GeoSpheroidType.SPHEROID\_CLARKE\_1880\_BEN{\ldots}
     |  
     |  SPHEROID\_CLARKE\_1880\_IGN = GeoSpheroidType.SPHEROID\_CLARKE\_1880\_IGN
     |  
     |  SPHEROID\_CLARKE\_1880\_RGS = GeoSpheroidType.SPHEROID\_CLARKE\_1880\_RGS
     |  
     |  SPHEROID\_CLARKE\_1880\_SGA = GeoSpheroidType.SPHEROID\_CLARKE\_1880\_SGA
     |  
     |  SPHEROID\_EVEREST\_1830 = GeoSpheroidType.SPHEROID\_EVEREST\_1830
     |  
     |  SPHEROID\_EVEREST\_DEF\_1967 = GeoSpheroidType.SPHEROID\_EVEREST\_DEF\_1967
     |  
     |  SPHEROID\_EVEREST\_DEF\_1975 = GeoSpheroidType.SPHEROID\_EVEREST\_DEF\_1975
     |  
     |  SPHEROID\_EVEREST\_MOD = GeoSpheroidType.SPHEROID\_EVEREST\_MOD
     |  
     |  SPHEROID\_EVEREST\_MOD\_1969 = GeoSpheroidType.SPHEROID\_EVEREST\_MOD\_1969
     |  
     |  SPHEROID\_FISCHER\_1960 = GeoSpheroidType.SPHEROID\_FISCHER\_1960
     |  
     |  SPHEROID\_FISCHER\_1968 = GeoSpheroidType.SPHEROID\_FISCHER\_1968
     |  
     |  SPHEROID\_FISCHER\_MOD = GeoSpheroidType.SPHEROID\_FISCHER\_MOD
     |  
     |  SPHEROID\_GEM\_10C = GeoSpheroidType.SPHEROID\_GEM\_10C
     |  
     |  SPHEROID\_GRS\_1967 = GeoSpheroidType.SPHEROID\_GRS\_1967
     |  
     |  SPHEROID\_GRS\_1980 = GeoSpheroidType.SPHEROID\_GRS\_1980
     |  
     |  SPHEROID\_HELMERT\_1906 = GeoSpheroidType.SPHEROID\_HELMERT\_1906
     |  
     |  SPHEROID\_HOUGH\_1960 = GeoSpheroidType.SPHEROID\_HOUGH\_1960
     |  
     |  SPHEROID\_INDONESIAN = GeoSpheroidType.SPHEROID\_INDONESIAN
     |  
     |  SPHEROID\_INTERNATIONAL\_1924 = GeoSpheroidType.SPHEROID\_INTERNATIONAL\_1{\ldots}
     |  
     |  SPHEROID\_INTERNATIONAL\_1967 = GeoSpheroidType.SPHEROID\_INTERNATIONAL\_1{\ldots}
     |  
     |  SPHEROID\_INTERNATIONAL\_1975 = GeoSpheroidType.SPHEROID\_INTERNATIONAL\_1{\ldots}
     |  
     |  SPHEROID\_KRASOVSKY\_1940 = GeoSpheroidType.SPHEROID\_KRASOVSKY\_1940
     |  
     |  SPHEROID\_NWL\_10D = GeoSpheroidType.SPHEROID\_NWL\_10D
     |  
     |  SPHEROID\_NWL\_9D = GeoSpheroidType.SPHEROID\_NWL\_9D
     |  
     |  SPHEROID\_OSU\_86F = GeoSpheroidType.SPHEROID\_OSU\_86F
     |  
     |  SPHEROID\_OSU\_91A = GeoSpheroidType.SPHEROID\_OSU\_91A
     |  
     |  SPHEROID\_PLESSIS\_1817 = GeoSpheroidType.SPHEROID\_PLESSIS\_1817
     |  
     |  SPHEROID\_POPULAR\_VISUALISATON = GeoSpheroidType.SPHEROID\_POPULAR\_VISUA{\ldots}
     |  
     |  SPHEROID\_SPHERE = GeoSpheroidType.SPHEROID\_SPHERE
     |  
     |  SPHEROID\_SPHERE\_AI = GeoSpheroidType.SPHEROID\_SPHERE\_AI
     |  
     |  SPHEROID\_STRUVE\_1860 = GeoSpheroidType.SPHEROID\_STRUVE\_1860
     |  
     |  SPHEROID\_USER\_DEFINED = GeoSpheroidType.SPHEROID\_USER\_DEFINED
     |  
     |  SPHEROID\_WALBECK = GeoSpheroidType.SPHEROID\_WALBECK
     |  
     |  SPHEROID\_WAR\_OFFICE = GeoSpheroidType.SPHEROID\_WAR\_OFFICE
     |  
     |  SPHEROID\_WGS\_1966 = GeoSpheroidType.SPHEROID\_WGS\_1966
     |  
     |  SPHEROID\_WGS\_1972 = GeoSpheroidType.SPHEROID\_WGS\_1972
     |  
     |  SPHEROID\_WGS\_1984 = GeoSpheroidType.SPHEROID\_WGS\_1984
     |  
     |  ----------------------------------------------------------------------
     |  Data descriptors inherited from enum.Enum:
     |  
     |  name
     |      The name of the Enum member.
     |  
     |  value
     |      The value of the Enum member.
     |  
     |  ----------------------------------------------------------------------
     |  Data descriptors inherited from enum.EnumMeta:
     |  
     |  \_\_members\_\_
     |      Returns a mapping of member name->value.
     |      
     |      This mapping lists all enum members, including aliases. Note that this
     |      is a read-only view of the internal mapping.
    
    class GeometryType(JEnum)
     |  该类定义了一系列几何对象的类型常量。
     |  
     |  :var GeometryType.GEOPOINT: 点几何对象
     |  :var GeometryType.GEOLINE: 线几何对象。
     |  :var GeometryType.GEOREGION: 面几何对象
     |  :var GeometryType.GEOTEXT: 文本几何对象
     |  :var GeometryType.GEOLINEM: 路由对象,是一组具有 X,Y 坐标与线性度量值的点组成的线性地物对象。
     |  :var GeometryType.GEOCOMPOUND: 复合几何对象。复合几何对象由多个子对象构成,每一个子对象可以是任何一种类型的几何对象。
     |  :var GeometryType.GEOPARAMETRICLINECOMPOUND: 复合参数化线几何对象。
     |  :var GeometryType.GEOPARAMETRICREGIONCOMPOUND:  复合参数化面几何对象。
     |  :var GeometryType.GEOPARAMETRICLINE: 参数化线几何对象。
     |  :var GeometryType.GEOPARAMETRICREGION: 参数化面几何对象。
     |  :var GeometryType.GEOMULTIPOINT: 多点对象,参数化的几何对象类型。
     |  :var GeometryType.GEOROUNDRECTANGLE: 圆角矩形几何对象,参数化的几何对象类型。
     |  :var GeometryType.GEOCIRCLE: 圆几何对象,参数化的几何对象类型。
     |  :var GeometryType.GEOELLIPSE: 椭圆几何对象,参数化的几何对象类型。
     |  :var GeometryType.GEOPIE: 扇面几何对象,参数化的几何对象类型。
     |  :var GeometryType.GEOARC: 圆弧几何对象,参数化的几何对象类型。
     |  :var GeometryType.GEOELLIPTICARC: 椭圆弧几何对象,参数化的几何对象类型。
     |  :var GeometryType.GEOCARDINAL: 二维 Cardinal 样条曲线几何对象,参数化的几何对象类型。
     |  :var GeometryType.GEOCURVE: 二维曲线几何对象,参数化的几何对象类型。
     |  :var GeometryType.GEOBSPLINE: 二维 B 样条曲线几何对象,参数化的几何对象类型。
     |  :var GeometryType.GEOPOINT3D: 三维点几何对象。
     |  :var GeometryType.GEOLINE3D: 三维线几何对象。
     |  :var GeometryType.GEOREGION3D: 三维面几何对象。
     |  :var GeometryType.GEOCHORD:  弓形几何对象,参数化的几何对象类型。
     |  :var GeometryType.GEOCYLINDER: 圆台几何对象。
     |  :var GeometryType.GEOPYRAMID:  四棱锥几何对象。
     |  :var GeometryType.GEORECTANGLE: 矩形几何对象,参数化的几何对象类型。
     |  :var GeometryType.GEOBOX: 长方体几何对象。
     |  :var GeometryType.GEOPICTURE: 二维图片几何对象。
     |  :var GeometryType.GEOCONE: 圆锥体几何对象。
     |  :var GeometryType.GEOPLACEMARK: 三维地标几何对象。
     |  :var GeometryType.GEOCIRCLE3D: 三维圆面几何对象。
     |  :var GeometryType.GEOSPHERE:  球体几何对象
     |  :var GeometryType.GEOHEMISPHERE: 半球体几何对象。
     |  :var GeometryType.GEOPIECYLINDER: 饼台几何对象。
     |  :var GeometryType.GEOPIE3D: 三维扇面几何对象。
     |  :var GeometryType.GEOELLIPSOID:  椭球体几何对象。
     |  :var GeometryType.GEOPARTICLE: 三维粒子几何对象。
     |  :var GeometryType.GEOTEXT3D: 三维文本几何对象。
     |  :var GeometryType.GEOMODEL: 三维模型几何对象。
     |  :var GeometryType.GEOMAP: 地图几何对象,用于在布局中添加地图。
     |  :var GeometryType.GEOMAPSCALE: 地图比例尺几何对象。
     |  :var GeometryType.GEONORTHARROW: 指北针几何对象。
     |  :var GeometryType.GEOMAPBORDER: 地图几何对象边框。
     |  :var GeometryType.GEOPICTURE3D: 三维图片几何对象。
     |  :var GeometryType.GEOLEGEND: 图例对象。
     |  :var GeometryType.GEOUSERDEFINED: 用户自定义的几何对象。
     |  :var GeometryType.GEOPOINTEPS: EPS 点几何对象
     |  :var GeometryType.GEOLINEEPS: EPS 线几何对象
     |  :var GeometryType.GEOREGIONEPS: EPS 面几何对象
     |  :var GeometryType.GEOTEXTEPS: EPS 文本几何对象
     |  
     |  Method resolution order:
     |      GeometryType
     |      JEnum
     |      enum.IntEnum
     |      builtins.int
     |      enum.Enum
     |      builtins.object
     |  
     |  Data and other attributes defined here:
     |  
     |  GEOARC = GeometryType.GEOARC
     |  
     |  GEOBOX = GeometryType.GEOBOX
     |  
     |  GEOBSPLINE = GeometryType.GEOBSPLINE
     |  
     |  GEOCARDINAL = GeometryType.GEOCARDINAL
     |  
     |  GEOCHORD = GeometryType.GEOCHORD
     |  
     |  GEOCIRCLE = GeometryType.GEOCIRCLE
     |  
     |  GEOCIRCLE3D = GeometryType.GEOCIRCLE3D
     |  
     |  GEOCOMPOUND = GeometryType.GEOCOMPOUND
     |  
     |  GEOCONE = GeometryType.GEOCONE
     |  
     |  GEOCURVE = GeometryType.GEOCURVE
     |  
     |  GEOCYLINDER = GeometryType.GEOCYLINDER
     |  
     |  GEOELLIPSE = GeometryType.GEOELLIPSE
     |  
     |  GEOELLIPSOID = GeometryType.GEOELLIPSOID
     |  
     |  GEOELLIPTICARC = GeometryType.GEOELLIPTICARC
     |  
     |  GEOHEMISPHERE = GeometryType.GEOHEMISPHERE
     |  
     |  GEOLEGEND = GeometryType.GEOLEGEND
     |  
     |  GEOLINE = GeometryType.GEOLINE
     |  
     |  GEOLINE3D = GeometryType.GEOLINE3D
     |  
     |  GEOLINEEPS = GeometryType.GEOLINEEPS
     |  
     |  GEOLINEM = GeometryType.GEOLINEM
     |  
     |  GEOMAP = GeometryType.GEOMAP
     |  
     |  GEOMAPBORDER = GeometryType.GEOMAPBORDER
     |  
     |  GEOMAPSCALE = GeometryType.GEOMAPSCALE
     |  
     |  GEOMODEL = GeometryType.GEOMODEL
     |  
     |  GEOMODEL3D = GeometryType.GEOMODEL3D
     |  
     |  GEOMULTIPOINT = GeometryType.GEOMULTIPOINT
     |  
     |  GEONORTHARROW = GeometryType.GEONORTHARROW
     |  
     |  GEOPARAMETRICLINE = GeometryType.GEOPARAMETRICLINE
     |  
     |  GEOPARAMETRICLINECOMPOUND = GeometryType.GEOPARAMETRICLINECOMPOUND
     |  
     |  GEOPARAMETRICREGION = GeometryType.GEOPARAMETRICREGION
     |  
     |  GEOPARAMETRICREGIONCOMPOUND = GeometryType.GEOPARAMETRICREGIONCOMPOUND
     |  
     |  GEOPARTICLE = GeometryType.GEOPARTICLE
     |  
     |  GEOPICTURE = GeometryType.GEOPICTURE
     |  
     |  GEOPICTURE3D = GeometryType.GEOPICTURE3D
     |  
     |  GEOPIE = GeometryType.GEOPIE
     |  
     |  GEOPIE3D = GeometryType.GEOPIE3D
     |  
     |  GEOPIECYLINDER = GeometryType.GEOPIECYLINDER
     |  
     |  GEOPLACEMARK = GeometryType.GEOPLACEMARK
     |  
     |  GEOPOINT = GeometryType.GEOPOINT
     |  
     |  GEOPOINT3D = GeometryType.GEOPOINT3D
     |  
     |  GEOPOINTEPS = GeometryType.GEOPOINTEPS
     |  
     |  GEOPYRAMID = GeometryType.GEOPYRAMID
     |  
     |  GEORECTANGLE = GeometryType.GEORECTANGLE
     |  
     |  GEOREGION = GeometryType.GEOREGION
     |  
     |  GEOREGION3D = GeometryType.GEOREGION3D
     |  
     |  GEOREGIONEPS = GeometryType.GEOREGIONEPS
     |  
     |  GEOROUNDRECTANGLE = GeometryType.GEOROUNDRECTANGLE
     |  
     |  GEOSPHERE = GeometryType.GEOSPHERE
     |  
     |  GEOTEXT = GeometryType.GEOTEXT
     |  
     |  GEOTEXT3D = GeometryType.GEOTEXT3D
     |  
     |  GEOTEXTEPS = GeometryType.GEOTEXTEPS
     |  
     |  GEOUSERDEFINED = GeometryType.GEOUSERDEFINED
     |  
     |  GRAPHICOBJECT = GeometryType.GRAPHICOBJECT
     |  
     |  ----------------------------------------------------------------------
     |  Data descriptors inherited from enum.Enum:
     |  
     |  name
     |      The name of the Enum member.
     |  
     |  value
     |      The value of the Enum member.
     |  
     |  ----------------------------------------------------------------------
     |  Data descriptors inherited from enum.EnumMeta:
     |  
     |  \_\_members\_\_
     |      Returns a mapping of member name->value.
     |      
     |      This mapping lists all enum members, including aliases. Note that this
     |      is a read-only view of the internal mapping.
    
    class GridStatisticsMode(JEnum)
     |  栅格统计类型常量
     |  
     |  :var GridStatisticsMode.MIN: 最小值
     |  :var GridStatisticsMode.MAX: 最大值
     |  :var GridStatisticsMode.MEAN: 平均值
     |  :var GridStatisticsMode.STDEV: 标准差
     |  :var GridStatisticsMode.SUM: 总和
     |  :var GridStatisticsMode.VARIETY: 种类
     |  :var GridStatisticsMode.RANGE: 值域,即最大值与最小值的差
     |  :var GridStatisticsMode.MAJORITY: 众数(出现频率最高的栅格值)
     |  :var GridStatisticsMode.MINORITY: 最少数(出现频率最低的栅格值)
     |  :var GridStatisticsMode.MEDIAN: 中位数(将所有栅格的值从小到大排列,取位于中间位置的栅格值)
     |  
     |  Method resolution order:
     |      GridStatisticsMode
     |      JEnum
     |      enum.IntEnum
     |      builtins.int
     |      enum.Enum
     |      builtins.object
     |  
     |  Data and other attributes defined here:
     |  
     |  MAJORITY = GridStatisticsMode.MAJORITY
     |  
     |  MAX = GridStatisticsMode.MAX
     |  
     |  MEAN = GridStatisticsMode.MEAN
     |  
     |  MEDIAN = GridStatisticsMode.MEDIAN
     |  
     |  MIN = GridStatisticsMode.MIN
     |  
     |  MINORITY = GridStatisticsMode.MINORITY
     |  
     |  RANGE = GridStatisticsMode.RANGE
     |  
     |  STDEV = GridStatisticsMode.STDEV
     |  
     |  SUM = GridStatisticsMode.SUM
     |  
     |  VARIETY = GridStatisticsMode.VARIETY
     |  
     |  ----------------------------------------------------------------------
     |  Data descriptors inherited from enum.Enum:
     |  
     |  name
     |      The name of the Enum member.
     |  
     |  value
     |      The value of the Enum member.
     |  
     |  ----------------------------------------------------------------------
     |  Data descriptors inherited from enum.EnumMeta:
     |  
     |  \_\_members\_\_
     |      Returns a mapping of member name->value.
     |      
     |      This mapping lists all enum members, including aliases. Note that this
     |      is a read-only view of the internal mapping.
    
    class GriddingLevel(JEnum)
     |  对于几何面对象的查询(GeometriesRelation),通过设置面对象的格网化,可以加快判断速度,比如面包含点判断。 单个面对象的格网化
     |  等级越高,所需的内存也越多,一般适用于面对象少但单个面对象比较大的情形。
     |  
     |  :var GriddingLevel.NONE: 无格网化
     |  :var GriddingLevel.LOWER: 低等级格网化,对每个面使用 32*32 个方格进行格网化
     |  :var GriddingLevel.MIDDLE: 中等级格网化,对每个面使用 64*64 个方格进行格网化
     |  :var GriddingLevel.NORMAL: 一般等级格网化,对每个面使用 128*128 个方格进行格网化
     |  :var GriddingLevel.HIGHER: 高等级格网化,对每个面使用 256*256 个方格进行格网化
     |  
     |  Method resolution order:
     |      GriddingLevel
     |      JEnum
     |      enum.IntEnum
     |      builtins.int
     |      enum.Enum
     |      builtins.object
     |  
     |  Data and other attributes defined here:
     |  
     |  HIGHER = GriddingLevel.HIGHER
     |  
     |  LOWER = GriddingLevel.LOWER
     |  
     |  MIDDLE = GriddingLevel.MIDDLE
     |  
     |  NONE = GriddingLevel.NONE
     |  
     |  NORMAL = GriddingLevel.NORMAL
     |  
     |  ----------------------------------------------------------------------
     |  Data descriptors inherited from enum.Enum:
     |  
     |  name
     |      The name of the Enum member.
     |  
     |  value
     |      The value of the Enum member.
     |  
     |  ----------------------------------------------------------------------
     |  Data descriptors inherited from enum.EnumMeta:
     |  
     |  \_\_members\_\_
     |      Returns a mapping of member name->value.
     |      
     |      This mapping lists all enum members, including aliases. Note that this
     |      is a read-only view of the internal mapping.
    
    class IgnoreMode(JEnum)
     |  该类定义了忽略颜色值模式的类型常量。
     |  
     |  :var IgnoreMode.IGNORENONE: 不忽略颜色值。
     |  :var IgnoreMode.IGNORESIGNAL: 按值忽略,忽略某个或某几个颜色值。
     |  :var IgnoreMode.IGNOREBORDER: 按照扫描线的方式忽略颜色值。
     |  
     |  Method resolution order:
     |      IgnoreMode
     |      JEnum
     |      enum.IntEnum
     |      builtins.int
     |      enum.Enum
     |      builtins.object
     |  
     |  Data and other attributes defined here:
     |  
     |  IGNOREBORDER = IgnoreMode.IGNOREBORDER
     |  
     |  IGNORENONE = IgnoreMode.IGNORENONE
     |  
     |  IGNORESIGNAL = IgnoreMode.IGNORESIGNAL
     |  
     |  ----------------------------------------------------------------------
     |  Data descriptors inherited from enum.Enum:
     |  
     |  name
     |      The name of the Enum member.
     |  
     |  value
     |      The value of the Enum member.
     |  
     |  ----------------------------------------------------------------------
     |  Data descriptors inherited from enum.EnumMeta:
     |  
     |  \_\_members\_\_
     |      Returns a mapping of member name->value.
     |      
     |      This mapping lists all enum members, including aliases. Note that this
     |      is a read-only view of the internal mapping.
    
    class ImportMode(JEnum)
     |  该类定义了导入模式类型常量。用于控制在数据导入时出现的设置的目标对象(数据集等)名称已存在情况下,即设置的名称已有名称冲突时的操作模式。
     |  
     |  :var ImportMode.NONE: 如存在名称冲突,则自动修改目标对象的名称后进行导入。
     |  :var ImportMode.OVERWRITE: 如存在名称冲突,则进行强制覆盖。
     |  :var ImportMode.APPEND: 如存在名称冲突,则进行数据集的追加。
     |  
     |  Method resolution order:
     |      ImportMode
     |      JEnum
     |      enum.IntEnum
     |      builtins.int
     |      enum.Enum
     |      builtins.object
     |  
     |  Data and other attributes defined here:
     |  
     |  APPEND = ImportMode.APPEND
     |  
     |  NONE = ImportMode.NONE
     |  
     |  OVERWRITE = ImportMode.OVERWRITE
     |  
     |  ----------------------------------------------------------------------
     |  Data descriptors inherited from enum.Enum:
     |  
     |  name
     |      The name of the Enum member.
     |  
     |  value
     |      The value of the Enum member.
     |  
     |  ----------------------------------------------------------------------
     |  Data descriptors inherited from enum.EnumMeta:
     |  
     |  \_\_members\_\_
     |      Returns a mapping of member name->value.
     |      
     |      This mapping lists all enum members, including aliases. Note that this
     |      is a read-only view of the internal mapping.
    
    class InterpolationAlgorithmType(JEnum)
     |  该类定义了插值分析所支持的算法的类型常量。
     |  
     |  对于一个区域,如果只有部分离散点数据已知,要想创建或者模拟一个表面或者场,需要对未知点的值进行估计,通常采用的是内插表面的方法。SuperMap 中提供
     |  三种内插方法,用于模拟或者创建一个表面,分别是:距离反比权重法(IDW)、克吕金插值方法(Kriging)、径向基函数插值法(RBF)。选用何种方法进行内
     |  插,通常取决于样点数据的分布和要创建表面的类型。
     |  
     |  :var InterpolationAlgorithmType.IDW: 距离反比权值(Inverse Distance Weighted)插值法。该方法通过计算附近区域离散点群的平均值来估算
     |                                       单元格的值,生成栅格数据集。这是一种简单有效的数据内插方法,运算速度相对较快。距离离散中心越近的点,其估算值越受影响。
     |  :var InterpolationAlgorithmType.SIMPLEKRIGING: 简单克吕金(Simple Kriging)插值法。简单克吕金是常用的克吕金插值方法之一,该方法假
     |                                                 定用于插值的字段值的期望(平均值)已知的某一常数。
     |  :var InterpolationAlgorithmType.KRIGING: 普通克吕金(Kriging)插值法。最常用的克吕金插值方法之一。该方法假定用于插值的字段值的期望(平
     |                                           均值)未知且恒定。它利用一定的数学函数,通过对给定的空间点进行拟合来估算单元格的值,生
     |                                           成格网数据集。它不仅可以生成一个表面,还可以给出预测结果的精度或者确定性的度量。因此,此方法计
     |                                           算精度较高,常用于社会科学及地质学。
     |  :var InterpolationAlgorithmType.UNIVERSALKRIGING: 泛克吕金(Universal Kriging)插值法。泛克吕金也是常用的克吕金插值方法之一,该
     |                                                    方法假定用于插值的字段值的期望(平均值)未知的变量。在样点数据中存在某种主导趋势,并且该趋势可以通过某一个确定
     |                                                    的函数或者多项式进行拟合的情况下适用泛克吕金插值法。
     |  :var InterpolationAlgorithmType.RBF: 径向基函数(Radial Basis Function)插值法。该方法假设变化是平滑的,它有两个特点:
     |  
     |                                       - 表面必须精确通过数据点;
     |                                       - 表面必须有最小曲率。
     |  
     |                                       该插值在创建有视觉要求的曲线和等高线方面有优势。
     |  :var InterpolationAlgorithmType.DENSITY: 点密度(Density)插值法
     |  
     |  Method resolution order:
     |      InterpolationAlgorithmType
     |      JEnum
     |      enum.IntEnum
     |      builtins.int
     |      enum.Enum
     |      builtins.object
     |  
     |  Data and other attributes defined here:
     |  
     |  DENSITY = InterpolationAlgorithmType.DENSITY
     |  
     |  IDW = InterpolationAlgorithmType.IDW
     |  
     |  KRIGING = InterpolationAlgorithmType.KRIGING
     |  
     |  RBF = InterpolationAlgorithmType.RBF
     |  
     |  SIMPLEKRIGING = InterpolationAlgorithmType.SIMPLEKRIGING
     |  
     |  UNIVERSALKRIGING = InterpolationAlgorithmType.UNIVERSALKRIGING
     |  
     |  ----------------------------------------------------------------------
     |  Data descriptors inherited from enum.Enum:
     |  
     |  name
     |      The name of the Enum member.
     |  
     |  value
     |      The value of the Enum member.
     |  
     |  ----------------------------------------------------------------------
     |  Data descriptors inherited from enum.EnumMeta:
     |  
     |  \_\_members\_\_
     |      Returns a mapping of member name->value.
     |      
     |      This mapping lists all enum members, including aliases. Note that this
     |      is a read-only view of the internal mapping.
    
    class JoinType(JEnum)
     |  该类定义了定义两个表之间连接类型常量。
     |  
     |  该类用于对相连接的两个表之间进行查询时,决定了查询结果中得到的记录的情况
     |  
     |  :var JoinType.INNERJOIN: 完全内连接,只有两个表中都有相关的记录才加入查询结果集。
     |  :var JoinType.LEFTJOIN: 左连接,左边表中所有相关记录进入查询结果集,右边表中无相关的记录则其对应的字段值显示为空。
     |  
     |  Method resolution order:
     |      JoinType
     |      JEnum
     |      enum.IntEnum
     |      builtins.int
     |      enum.Enum
     |      builtins.object
     |  
     |  Data and other attributes defined here:
     |  
     |  INNERJOIN = JoinType.INNERJOIN
     |  
     |  LEFTJOIN = JoinType.LEFTJOIN
     |  
     |  ----------------------------------------------------------------------
     |  Data descriptors inherited from enum.Enum:
     |  
     |  name
     |      The name of the Enum member.
     |  
     |  value
     |      The value of the Enum member.
     |  
     |  ----------------------------------------------------------------------
     |  Data descriptors inherited from enum.EnumMeta:
     |  
     |  \_\_members\_\_
     |      Returns a mapping of member name->value.
     |      
     |      This mapping lists all enum members, including aliases. Note that this
     |      is a read-only view of the internal mapping.
    
    class KernelFunction(JEnum)
     |  地理加权回归分析核函数类型常量。
     |  
     |  :var KernelFunction.GAUSSIAN: 高斯核函数。
     |  
     |                                高斯核函数计算公式:
     |  
     |                                W\_ij=e\^{}(-((d\_ij/b)\^{}2)/2)。
     |  
     |                                其中W\_ij为点i和点j之间的权重,d\_ij为点i和点j之间的距离,b为带宽范围。
     |  
     |  :var KernelFunction.BISQUARE: 二次核函数。
     |                                二次核函数计算公式:
     |  
     |                                如果d\_ij≤b, W\_ij=(1-(d\_ij/b)\^{}2))\^{}2;否则,W\_ij=0。
     |  
     |                                其中W\_ij为点i和点j之间的权重,d\_ij为点i和点j之间的距离,b为带宽范围。
     |  
     |  :var KernelFunction.BOXCAR: 盒状核函数。
     |  
     |                              盒状核函数计算公式:
     |  
     |                              如果d\_ij≤b, W\_ij=1;否则,W\_ij=0。
     |  
     |                              其中W\_ij为点i和点j之间的权重,d\_ij为点i和点j之间的距离,b为带宽范围。
     |  
     |  :var KernelFunction.TRICUBE: 立方体核函数。
     |  
     |                               立方体核函数计算公式:
     |  
     |                               如果d\_ij≤b, W\_ij=(1-(d\_ij/b)\^{}3))\^{}3;否则,W\_ij=0。
     |  
     |                               其中W\_ij为点i和点j之间的权重,d\_ij为点i和点j之间的距离,b为带宽范围。
     |  
     |  Method resolution order:
     |      KernelFunction
     |      JEnum
     |      enum.IntEnum
     |      builtins.int
     |      enum.Enum
     |      builtins.object
     |  
     |  Data and other attributes defined here:
     |  
     |  BISQUARE = KernelFunction.BISQUARE
     |  
     |  BOXCAR = KernelFunction.BOXCAR
     |  
     |  GAUSSIAN = KernelFunction.GAUSSIAN
     |  
     |  TRICUBE = KernelFunction.TRICUBE
     |  
     |  ----------------------------------------------------------------------
     |  Data descriptors inherited from enum.Enum:
     |  
     |  name
     |      The name of the Enum member.
     |  
     |  value
     |      The value of the Enum member.
     |  
     |  ----------------------------------------------------------------------
     |  Data descriptors inherited from enum.EnumMeta:
     |  
     |  \_\_members\_\_
     |      Returns a mapping of member name->value.
     |      
     |      This mapping lists all enum members, including aliases. Note that this
     |      is a read-only view of the internal mapping.
    
    class KernelType(JEnum)
     |  地理加权回归分析带宽类型常量
     |  
     |  :var KernelType.FIXED: 固定型带宽。针对每个回归分析点,使用一个固定的值作为带宽范围。
     |  :var KernelType.ADAPTIVE: 可变型带宽。针对每个回归分析点,使用回归点与第K个最近相邻点之间的距离作为带宽范围。其中,K为相邻数目。
     |  
     |  Method resolution order:
     |      KernelType
     |      JEnum
     |      enum.IntEnum
     |      builtins.int
     |      enum.Enum
     |      builtins.object
     |  
     |  Data and other attributes defined here:
     |  
     |  ADAPTIVE = KernelType.ADAPTIVE
     |  
     |  FIXED = KernelType.FIXED
     |  
     |  ----------------------------------------------------------------------
     |  Data descriptors inherited from enum.Enum:
     |  
     |  name
     |      The name of the Enum member.
     |  
     |  value
     |      The value of the Enum member.
     |  
     |  ----------------------------------------------------------------------
     |  Data descriptors inherited from enum.EnumMeta:
     |  
     |  \_\_members\_\_
     |      Returns a mapping of member name->value.
     |      
     |      This mapping lists all enum members, including aliases. Note that this
     |      is a read-only view of the internal mapping.
    
    class LineToPointMode(JEnum)
     |  线转点的方式
     |  
     |  :var LineToPointMode.VERTEX: 节点模式,将线对象的每个节点都转换为一个点对象
     |  :var LineToPointMode.INNER\_POINT: 内点模式,将线对象的内点转换为一个点对象
     |  :var LineToPointMode.SUB\_INNER\_POINT: 子对象内点模式,将线对象的每个子对象的内点分别转换为一个点对象,如果线的子对象数目为1,将与 INNER\_POINT 的结果相同。
     |  :var LineToPointMode.START\_NODE: 起始点模式,将线对象的第一个节点,即起点,转换为一个点对象
     |  :var LineToPointMode.END\_NODE: 终止点模式,将线对象的最后一个节点,即终点,转换为一个点对象
     |  :var LineToPointMode.START\_END\_NODE: 起始终止点模式,将线对象的起点和终点分别转换为一个点对象
     |  :var LineToPointMode.SEGMENT\_INNER\_POINT: 线段内点模式,将线对象的每个线段的内点,分别转换为一个点对象,线段指的是相邻两个节点构成的线。
     |  :var LineToPointMode.SUB\_START\_NODE: 子对象起始点模式,将线对象的每个子对象的第一个点,分别转换为一个点对象
     |  :var LineToPointMode.SUB\_END\_NODE: 子对象终止点模式,将线对象的每个子对象的对后一个点,分别转换为一个点对象
     |  :var LineToPointMode.SUB\_START\_END\_NODE: 子对象起始终止点模式,将线对象的每个子对象的第一个点和最后一个点,分别转换为一个点对象。
     |  
     |  Method resolution order:
     |      LineToPointMode
     |      JEnum
     |      enum.IntEnum
     |      builtins.int
     |      enum.Enum
     |      builtins.object
     |  
     |  Data and other attributes defined here:
     |  
     |  END\_NODE = LineToPointMode.END\_NODE
     |  
     |  INNER\_POINT = LineToPointMode.INNER\_POINT
     |  
     |  SEGMENT\_INNER\_POINT = LineToPointMode.SEGMENT\_INNER\_POINT
     |  
     |  START\_END\_NODE = LineToPointMode.START\_END\_NODE
     |  
     |  START\_NODE = LineToPointMode.START\_NODE
     |  
     |  SUB\_END\_NODE = LineToPointMode.SUB\_END\_NODE
     |  
     |  SUB\_INNER\_POINT = LineToPointMode.SUB\_INNER\_POINT
     |  
     |  SUB\_START\_END\_NODE = LineToPointMode.SUB\_START\_END\_NODE
     |  
     |  SUB\_START\_NODE = LineToPointMode.SUB\_START\_NODE
     |  
     |  VERTEX = LineToPointMode.VERTEX
     |  
     |  ----------------------------------------------------------------------
     |  Data descriptors inherited from enum.Enum:
     |  
     |  name
     |      The name of the Enum member.
     |  
     |  value
     |      The value of the Enum member.
     |  
     |  ----------------------------------------------------------------------
     |  Data descriptors inherited from enum.EnumMeta:
     |  
     |  \_\_members\_\_
     |      Returns a mapping of member name->value.
     |      
     |      This mapping lists all enum members, including aliases. Note that this
     |      is a read-only view of the internal mapping.
    
    class MultiBandImportMode(JEnum)
     |  该类定义了多波段导入模式类型常量,提供了导入多波段数据所采用的模式。
     |  
     |  :var MultiBandImportMode.SINGLEBAND: 将多波段数据导入为多个单波段数据集
     |  :var MultiBandImportMode.MULTIBAND: 将多波段数据导入为一个多波段数据集
     |  :var MultiBandImportMode.COMPOSITE: 将多波段数据导入为一个单波段数据集,目前此模式适用于以下两种情况:
     |  
     |                                      - 三波段 8 位的数据导入为一个 RGB 单波段 24 位的数据集;
     |                                      - 四波段 8 位的数据导入为一个 RGBA 单波段 32 位的数据集。
     |  
     |  Method resolution order:
     |      MultiBandImportMode
     |      JEnum
     |      enum.IntEnum
     |      builtins.int
     |      enum.Enum
     |      builtins.object
     |  
     |  Data and other attributes defined here:
     |  
     |  COMPOSITE = MultiBandImportMode.COMPOSITE
     |  
     |  MULTIBAND = MultiBandImportMode.MULTIBAND
     |  
     |  SINGLEBAND = MultiBandImportMode.SINGLEBAND
     |  
     |  ----------------------------------------------------------------------
     |  Data descriptors inherited from enum.Enum:
     |  
     |  name
     |      The name of the Enum member.
     |  
     |  value
     |      The value of the Enum member.
     |  
     |  ----------------------------------------------------------------------
     |  Data descriptors inherited from enum.EnumMeta:
     |  
     |  \_\_members\_\_
     |      Returns a mapping of member name->value.
     |      
     |      This mapping lists all enum members, including aliases. Note that this
     |      is a read-only view of the internal mapping.
    
    class NeighbourShapeType(JEnum)
     |  :var NeighbourShapeType.RECTANGLE: 矩形邻域,矩形的大小由指定的宽和高来确定,矩形范围内的单元格参与邻域统计的计算。矩形邻域的默认宽和高均
     |                                     为 0(单位为地理单位或栅格单位)。
     |  
     |                                     .. image:: ../image/Rectangle.png
     |  
     |  :var NeighbourShapeType.CIRCLE: 圆形邻域,圆形邻域的大小根据指定的半径来确定,圆形范围内的所有单元格都参与邻域处理,只要单元格有部分包含在
     |                                  圆形范围内都将参与邻域统计。圆形邻域的默认半径为 0(单位为地理单位或栅格单位)。
     |  
     |                                  .. image:: ../image/Circle.png
     |  
     |  :var NeighbourShapeType.ANNULUS: 圆环邻域。环形邻域的大小根据指定的外圆半径和内圆半径来确定,环形区域内的单元格都参与邻域处理。环行邻域的
     |                                  默认外圆半径和内圆半径均为 0(单位为地理单位或栅格单位)。
     |  
     |                                  .. image:: ../image/Annulus.png
     |  
     |  :var NeighbourShapeType.WEDGE: 扇形邻域。扇形邻域的大小根据指定的圆半径、起始角度和终止角度来确定。在扇形区内的所有单元格都参与邻域处理。
     |                                 扇形邻域的默认半径为 0(单位为地理单位或栅格单位),起始角度和终止角度的默认值均为 0 度。
     |  
     |                                 .. image:: ../image/Wedge.png
     |  
     |  Method resolution order:
     |      NeighbourShapeType
     |      JEnum
     |      enum.IntEnum
     |      builtins.int
     |      enum.Enum
     |      builtins.object
     |  
     |  Data and other attributes defined here:
     |  
     |  ANNULUS = NeighbourShapeType.ANNULUS
     |  
     |  CIRCLE = NeighbourShapeType.CIRCLE
     |  
     |  RECTANGLE = NeighbourShapeType.RECTANGLE
     |  
     |  WEDGE = NeighbourShapeType.WEDGE
     |  
     |  ----------------------------------------------------------------------
     |  Data descriptors inherited from enum.Enum:
     |  
     |  name
     |      The name of the Enum member.
     |  
     |  value
     |      The value of the Enum member.
     |  
     |  ----------------------------------------------------------------------
     |  Data descriptors inherited from enum.EnumMeta:
     |  
     |  \_\_members\_\_
     |      Returns a mapping of member name->value.
     |      
     |      This mapping lists all enum members, including aliases. Note that this
     |      is a read-only view of the internal mapping.
    
    class NeighbourUnitType(JEnum)
     |  该类定义了邻域分析的单位类型常量。
     |  
     |  :var NeighbourUnitType.CELL: 栅格坐标,即使用栅格数作为邻域单位。
     |  :var NeighbourUnitType.MAP: 地理坐标,即使用地图的长度单位作为邻域单位。
     |  
     |  Method resolution order:
     |      NeighbourUnitType
     |      JEnum
     |      enum.IntEnum
     |      builtins.int
     |      enum.Enum
     |      builtins.object
     |  
     |  Data and other attributes defined here:
     |  
     |  CELL = NeighbourUnitType.CELL
     |  
     |  MAP = NeighbourUnitType.MAP
     |  
     |  ----------------------------------------------------------------------
     |  Data descriptors inherited from enum.Enum:
     |  
     |  name
     |      The name of the Enum member.
     |  
     |  value
     |      The value of the Enum member.
     |  
     |  ----------------------------------------------------------------------
     |  Data descriptors inherited from enum.EnumMeta:
     |  
     |  \_\_members\_\_
     |      Returns a mapping of member name->value.
     |      
     |      This mapping lists all enum members, including aliases. Note that this
     |      is a read-only view of the internal mapping.
    
    class OverlayMode(JEnum)
     |  叠加分析模式类型
     |  
     |  * 裁剪(CLIP)
     |  
     |      用于对数据集进行擦除方式的叠加分析,将第一个数据集中包含在第二个数据集内的对象裁剪并删除。
     |  
     |      * 裁剪数据集(第二数据集)的类型必须是面,被剪裁的数据集(第一数据集)可以是点、线、面。
     |      * 在被裁剪数据集中,只有落在裁剪数据集多边形内的对象才会被输出到结果数据集中。
     |      * 裁剪数据集、被裁剪数据集以及结果数据集的地理坐标系必须一致。
     |      * clip 与 intersect 在空间处理上是一致的,不同在于对结果记录集属性的处理,clip 分析只是用来做裁剪,结果记录集与第一个记录集的属性表结构相同,此处叠加分析参数对象设置无效。而 intersect 求交分析的结果则可以根据字段设置情况来保留两个记录集的字段。
     |      * 所有叠加分析的结果都不考虑数据集的系统字段。
     |  
     |      .. image:: ../image/OverlayClip.png
     |  
     |  * 擦除(ERASE)
     |  
     |      用于对数据集进行同一方式的叠加分析,结果数据集中保留被同一运算的数据集的全部对象和被同一运算的数据集与用来进行同一运算的数据集相交的对象。
     |  
     |      * 擦除数据集(第二数据集)的类型必须是面,被擦除的数据集(第一数据集)可以是点、线、面数据集。
     |      * 擦除数据集中的多边形集合定义了擦除区域,被擦除数据集中凡是落在这些多边形区域内的特征都将被去除,而落在多边形区域外的特征要素都将被输出到结果数据集中,与 clip 运算相反。
     |      * 擦除数据集、被擦除数据集以及结果数据集的地理坐标系必须一致。
     |  
     |      .. image:: ../image/OverlayErase.png
     |  
     |  * 同一(IDENTITY)
     |  
     |      用于对数据集进行同一方式的叠加分析,结果数据集中保留被同一运算的数据集的全部对象和被同一运算的数据集与用来进行同一运算的数据集相交的对象。
     |  
     |      * 同一运算就是第一数据集与第二数据集先求交,然后求交结果再与第一数据集求并的一个运算。其中,第二数据集的类型必须是面,第一数据集的类型可以是点、线、面数据集。如果第一个数据集为点数集,则新生成的数据集中保留第一个数据集的所有对象;如果第一个数据集为线数据集,则新生成的数据集中保留第一个数据集的所有对象,但是把与第二个数据集相交的对象在相交的地方打断;如果第一个数据集为面数据集,则结果数据集保留以第一数据集为控制边界之内的所有多边形,并且把与第二个数据集相交的对象在相交的地方分割成多个对象。
     |      * identiy 运算与 union 运算有相似之处,所不同之处在于 union 运算保留了两个数据集的所有部分,而 identity 运算是把第一个数据集中与第二个数据集不相交的部分进行保留。identity 运算的结果属性表来自于两个数据集的属性表。
     |      * 用于进行同一运算的数据集、被同一运算的数据集以及结果数据集的地理坐标系必须一致。
     |  
     |      .. image:: ../image/OverlayIdentity.png
     |  
     |  * 相交(INTERSECT)
     |  
     |      进行相交方式的叠加分析,将被相交叠加分析的数据集中不包含在用来相交叠加分析的数据集中的对象切割并删除。即两个数据集中重叠的部分将被输出到结果数据集中,其余部分将被排除。
     |  
     |      * 被相交叠加分析的数据集可以是点类型、线类型和面类型,用来相交叠加分析的数据集必须是面类型。第一数据集的特征对象(点、线和面)在与第二数据集中的多边形相交处被分裂(点对象除外),分裂结果被输出到结果数据集中。
     |      * 求交运算与裁剪运算得到的结果数据集的空间几何信息相同的,但是裁剪运算不对属性表做任何处理,而求交运算可以让用户选择需要保留的属性字段。
     |      * 用于相交叠加分析的数据集、被相交叠加分析的数据集以及结果数据集的地理坐标系必须一致。
     |  
     |      .. image:: ../image/OverlayIntersect.png
     |  
     |  * 对称差(XOR)
     |  
     |      对两个面数据集进行对称差分析运算。即交集取反运算。
     |  
     |      * 用于对称差分析的数据集、被对称差分析的数据集以及结果数据集的地理坐标系必须一致。
     |      * 对称差运算是两个数据集的异或运算。操作的结果是,对于每一个面对象,去掉其与另一个数据集中的几何对象相交的部分,而保留剩下的部分。对称差运算的输出结果的属性表包含两个输入数据集的非系统属性字段。
     |  
     |      .. image:: ../image/OverlayXOR.png
     |  
     |  * 合并(UNION)
     |  
     |      用于对两个面数据集进行合并方式的叠加分析,结果数据集中保存被合并叠加分析的数据集和用于合并叠加分析的数据集中的全部对象,并且对相交部分进行求交和分割运算。 注意:
     |  
     |      * 合并是求两个数据集并的运算,合并后的图层保留两个数据集所有图层要素,只限于两个面数据集之间进行。
     |      * 进行 union 运算后,两个面数据集在相交处多边形被分割,且两个数据集的几何和属性信息都被输出到结果数据集中。
     |      * 用于合并叠加分析的数据集、被合并叠加分析的数据集以及结果数据集的地理坐标系必须一致。
     |  
     |      .. image:: ../image/OverlayUnion.png
     |  
     |  * 更新(UPDATE)
     |  
     |      用于对两个面数据集进行更新方式的叠加分析, 更新运算是用用于更新的数据集替换与被更新数据集的重合部分,是一个先擦除后粘贴的过程。
     |  
     |      * 用于更新叠加分析的数据集、被更新叠加分析的数据集以及结果数据集的地理坐标系必须一致。
     |      * 第一数据集与第二数据集的类型都必须是面数据集。结果数据集中保留了更新数据集的几何形状和属性信息。
     |  
     |      .. image:: ../image/OverlayUpdate.png
     |  
     |  :var OverlayMode.CLIP: 裁剪
     |  :var OverlayMode.ERASE: 擦除
     |  :var OverlayMode.IDENTITY: 同一
     |  :var OverlayMode.INTERSECT: 相交
     |  :var OverlayMode.XOR: 对称差
     |  :var OverlayMode.UNION: 合并
     |  :var OverlayMode.UPDATE: 更新
     |  
     |  Method resolution order:
     |      OverlayMode
     |      JEnum
     |      enum.IntEnum
     |      builtins.int
     |      enum.Enum
     |      builtins.object
     |  
     |  Data and other attributes defined here:
     |  
     |  CLIP = OverlayMode.CLIP
     |  
     |  ERASE = OverlayMode.ERASE
     |  
     |  IDENTITY = OverlayMode.IDENTITY
     |  
     |  INTERSECT = OverlayMode.INTERSECT
     |  
     |  UNION = OverlayMode.UNION
     |  
     |  UPDATE = OverlayMode.UPDATE
     |  
     |  XOR = OverlayMode.XOR
     |  
     |  ----------------------------------------------------------------------
     |  Data descriptors inherited from enum.Enum:
     |  
     |  name
     |      The name of the Enum member.
     |  
     |  value
     |      The value of the Enum member.
     |  
     |  ----------------------------------------------------------------------
     |  Data descriptors inherited from enum.EnumMeta:
     |  
     |  \_\_members\_\_
     |      Returns a mapping of member name->value.
     |      
     |      This mapping lists all enum members, including aliases. Note that this
     |      is a read-only view of the internal mapping.
    
    class PixelFormat(JEnum)
     |  该类定义了栅格与影像数据存储的像素格式类型常量。
     |  
     |  光栅数据结构实际上就是像元的阵列,像元(或像素)是光栅数据的最基本信息存储单位。在 SuperMap 中有两种类型的光栅数据:栅格数据集(DatasetGrid)
     |  和影像数据集(DatasetImage),栅格数据集多用来进行栅格分析,因而其像元值为地物的属性值,如高程,降水量等;而影像数据集一般用来进行显示或作为底
     |  图,因而其像元值为颜色值或颜色的索引值。
     |  
     |  :var PixelFormat.UNKONOWN: 未知的像素格式
     |  :var PixelFormat.UBIT1: 每个像元用 1 个比特表示。对栅格数据集来说,可表示 0 和 1 两种值;对影像数据集来说,可表示黑白两种颜色,对应单色影像数据。
     |  :var PixelFormat.UBIT4: 每个像元用 4 个比特表示。对栅格数据集来说,可表示 0 到 15 共 16 个整数值;对影像数据集来说,可表示 16 种颜色,这 16 种颜色为索引色,在其颜色表中定义,对应 16 色的影像数据。
     |  :var PixelFormat.UBIT8: 每个像元用 8 个比特,即 1 个字节表示。对栅格数据集来说,可表示 0 到 255 共 256 个整数值;对影像数据集来说,可表示 256 种渐变的颜色,这 256 种颜色为索引色,在其颜色表中定义,对应 256 色的影像数据。
     |  :var PixelFormat.BIT8: 每个像元用 8 个比特,即 1 个字节来表示。对栅格数据集来说,可表示 -128 到 127 共 256 个整数值。每个像元用 8 个比特,即 1 个字节来表示。对栅格数据集来说,可表示 -128 到 127 共 256 个整数值。
     |  :var PixelFormat.BIT16: 每个像元用 16 个比特,即 2 个字节表示。对栅格数据集来说,可表示 -32768 到 32767 共 65536 个整数值;对影像数据集来说,16 个比特中,红,绿,蓝各用 5 比特来表示,剩余 1 比特未使用,对应彩色的影像数据。
     |  :var PixelFormat.UBIT16: 每个像元用 16 个比特,即 2 个字节来表示。对栅格数据集来说,可表示 0 到 65535 共 65536 个整数值
     |  :var PixelFormat.RGB: 每个像元用 24 个比特,即 3 个字节来表示。仅提供给影像数据集使用,24 比特中红、绿、蓝各用 8 比特来表示,对应真彩色的影像数据。
     |  :var PixelFormat.RGBA: 每个像元用 32 个比特,即 4 个字节来表示。仅提供给影像数据集使用,32 比特中红、绿、蓝和 alpha 各用 8 比特来表示,对应增强真彩色的影像数据。
     |  :var PixelFormat.BIT32: 每个像元用 32 个比特,即 4 个字节来表示。对栅格数据集来说,可表示 -231 到 (231-1) 共 4294967296 个整数值;对影像数据集来说,32 比特中,红,绿,蓝和 alpha 各用 8 比特来表示,对应增强真彩色的影像数据。该格式支持 DatasetGrid,DatasetImage(仅支持多波段)。
     |  :var PixelFormat.UBIT32: 每个像元用 32 个比特,即 4 个字节来表示,可表示 0 到 4294967295 共 4294967296 个整数值。
     |  :var PixelFormat.BIT64: 每个像元用 64 个比特,即 8 个字节来表示。可表示 -263 到 (263-1) 共 18446744073709551616 个整数值。。
     |  :var PixelFormat.SINGLE: 每个像元用 4 个字节来表示。可表示 -3.402823E+38 到 3.402823E+38 范围内的单精度浮点数。
     |  :var PixelFormat.DOUBLE: 每个像元用 8 个字节来表示。可表示 -1.79769313486232E+308 到 1.79769313486232E+308 范围内的双精度浮点数。
     |  
     |  Method resolution order:
     |      PixelFormat
     |      JEnum
     |      enum.IntEnum
     |      builtins.int
     |      enum.Enum
     |      builtins.object
     |  
     |  Data and other attributes defined here:
     |  
     |  BIT16 = PixelFormat.BIT16
     |  
     |  BIT32 = PixelFormat.BIT32
     |  
     |  BIT64 = PixelFormat.BIT64
     |  
     |  BIT8 = PixelFormat.BIT8
     |  
     |  DOUBLE = PixelFormat.DOUBLE
     |  
     |  RGB = PixelFormat.RGB
     |  
     |  RGBA = PixelFormat.RGBA
     |  
     |  SINGLE = PixelFormat.SINGLE
     |  
     |  UBIT1 = PixelFormat.UBIT1
     |  
     |  UBIT16 = PixelFormat.UBIT16
     |  
     |  UBIT32 = PixelFormat.UBIT32
     |  
     |  UBIT4 = PixelFormat.UBIT4
     |  
     |  UBIT8 = PixelFormat.UBIT8
     |  
     |  UNKONOWN = PixelFormat.UNKONOWN
     |  
     |  ----------------------------------------------------------------------
     |  Data descriptors inherited from enum.Enum:
     |  
     |  name
     |      The name of the Enum member.
     |  
     |  value
     |      The value of the Enum member.
     |  
     |  ----------------------------------------------------------------------
     |  Data descriptors inherited from enum.EnumMeta:
     |  
     |  \_\_members\_\_
     |      Returns a mapping of member name->value.
     |      
     |      This mapping lists all enum members, including aliases. Note that this
     |      is a read-only view of the internal mapping.
    
    class PrjCoordSysType(JEnum)
     |  An enumeration.
     |  
     |  Method resolution order:
     |      PrjCoordSysType
     |      JEnum
     |      enum.IntEnum
     |      builtins.int
     |      enum.Enum
     |      builtins.object
     |  
     |  Data and other attributes defined here:
     |  
     |  PCS\_ADINDAN\_UTM\_37N = PrjCoordSysType.PCS\_ADINDAN\_UTM\_37N
     |  
     |  PCS\_ADINDAN\_UTM\_38N = PrjCoordSysType.PCS\_ADINDAN\_UTM\_38N
     |  
     |  PCS\_AFGOOYE\_UTM\_38N = PrjCoordSysType.PCS\_AFGOOYE\_UTM\_38N
     |  
     |  PCS\_AFGOOYE\_UTM\_39N = PrjCoordSysType.PCS\_AFGOOYE\_UTM\_39N
     |  
     |  PCS\_AGD\_1966\_AMG\_48 = PrjCoordSysType.PCS\_AGD\_1966\_AMG\_48
     |  
     |  PCS\_AGD\_1966\_AMG\_49 = PrjCoordSysType.PCS\_AGD\_1966\_AMG\_49
     |  
     |  PCS\_AGD\_1966\_AMG\_50 = PrjCoordSysType.PCS\_AGD\_1966\_AMG\_50
     |  
     |  PCS\_AGD\_1966\_AMG\_51 = PrjCoordSysType.PCS\_AGD\_1966\_AMG\_51
     |  
     |  PCS\_AGD\_1966\_AMG\_52 = PrjCoordSysType.PCS\_AGD\_1966\_AMG\_52
     |  
     |  PCS\_AGD\_1966\_AMG\_53 = PrjCoordSysType.PCS\_AGD\_1966\_AMG\_53
     |  
     |  PCS\_AGD\_1966\_AMG\_54 = PrjCoordSysType.PCS\_AGD\_1966\_AMG\_54
     |  
     |  PCS\_AGD\_1966\_AMG\_55 = PrjCoordSysType.PCS\_AGD\_1966\_AMG\_55
     |  
     |  PCS\_AGD\_1966\_AMG\_56 = PrjCoordSysType.PCS\_AGD\_1966\_AMG\_56
     |  
     |  PCS\_AGD\_1966\_AMG\_57 = PrjCoordSysType.PCS\_AGD\_1966\_AMG\_57
     |  
     |  PCS\_AGD\_1966\_AMG\_58 = PrjCoordSysType.PCS\_AGD\_1966\_AMG\_58
     |  
     |  PCS\_AGD\_1984\_AMG\_48 = PrjCoordSysType.PCS\_AGD\_1984\_AMG\_48
     |  
     |  PCS\_AGD\_1984\_AMG\_49 = PrjCoordSysType.PCS\_AGD\_1984\_AMG\_49
     |  
     |  PCS\_AGD\_1984\_AMG\_50 = PrjCoordSysType.PCS\_AGD\_1984\_AMG\_50
     |  
     |  PCS\_AGD\_1984\_AMG\_51 = PrjCoordSysType.PCS\_AGD\_1984\_AMG\_51
     |  
     |  PCS\_AGD\_1984\_AMG\_52 = PrjCoordSysType.PCS\_AGD\_1984\_AMG\_52
     |  
     |  PCS\_AGD\_1984\_AMG\_53 = PrjCoordSysType.PCS\_AGD\_1984\_AMG\_53
     |  
     |  PCS\_AGD\_1984\_AMG\_54 = PrjCoordSysType.PCS\_AGD\_1984\_AMG\_54
     |  
     |  PCS\_AGD\_1984\_AMG\_55 = PrjCoordSysType.PCS\_AGD\_1984\_AMG\_55
     |  
     |  PCS\_AGD\_1984\_AMG\_56 = PrjCoordSysType.PCS\_AGD\_1984\_AMG\_56
     |  
     |  PCS\_AGD\_1984\_AMG\_57 = PrjCoordSysType.PCS\_AGD\_1984\_AMG\_57
     |  
     |  PCS\_AGD\_1984\_AMG\_58 = PrjCoordSysType.PCS\_AGD\_1984\_AMG\_58
     |  
     |  PCS\_AIN\_EL\_ABD\_BAHRAIN\_GRID = PrjCoordSysType.PCS\_AIN\_EL\_ABD\_BAHRAIN\_G{\ldots}
     |  
     |  PCS\_AIN\_EL\_ABD\_UTM\_37N = PrjCoordSysType.PCS\_AIN\_EL\_ABD\_UTM\_37N
     |  
     |  PCS\_AIN\_EL\_ABD\_UTM\_38N = PrjCoordSysType.PCS\_AIN\_EL\_ABD\_UTM\_38N
     |  
     |  PCS\_AIN\_EL\_ABD\_UTM\_39N = PrjCoordSysType.PCS\_AIN\_EL\_ABD\_UTM\_39N
     |  
     |  PCS\_AMERSFOORT\_RD\_NEW = PrjCoordSysType.PCS\_AMERSFOORT\_RD\_NEW
     |  
     |  PCS\_ARATU\_UTM\_22S = PrjCoordSysType.PCS\_ARATU\_UTM\_22S
     |  
     |  PCS\_ARATU\_UTM\_23S = PrjCoordSysType.PCS\_ARATU\_UTM\_23S
     |  
     |  PCS\_ARATU\_UTM\_24S = PrjCoordSysType.PCS\_ARATU\_UTM\_24S
     |  
     |  PCS\_ATF\_NORD\_DE\_GUERRE = PrjCoordSysType.PCS\_ATF\_NORD\_DE\_GUERRE
     |  
     |  PCS\_ATS\_1977\_UTM\_19N = PrjCoordSysType.PCS\_ATS\_1977\_UTM\_19N
     |  
     |  PCS\_ATS\_1977\_UTM\_20N = PrjCoordSysType.PCS\_ATS\_1977\_UTM\_20N
     |  
     |  PCS\_AZORES\_CENTRAL\_1948\_UTM\_ZONE\_26N = PrjCoordSysType.PCS\_AZORES\_CENT{\ldots}
     |  
     |  PCS\_AZORES\_OCCIDENTAL\_1939\_UTM\_ZONE\_25N = PrjCoordSysType.PCS\_AZORES\_O{\ldots}
     |  
     |  PCS\_AZORES\_ORIENTAL\_1940\_UTM\_ZONE\_26N = PrjCoordSysType.PCS\_AZORES\_ORI{\ldots}
     |  
     |  PCS\_BATAVIA\_UTM\_48S = PrjCoordSysType.PCS\_BATAVIA\_UTM\_48S
     |  
     |  PCS\_BATAVIA\_UTM\_49S = PrjCoordSysType.PCS\_BATAVIA\_UTM\_49S
     |  
     |  PCS\_BATAVIA\_UTM\_50S = PrjCoordSysType.PCS\_BATAVIA\_UTM\_50S
     |  
     |  PCS\_BEIJING\_1954\_3\_DEGREE\_GK\_25 = PrjCoordSysType.PCS\_BEIJING\_1954\_3\_D{\ldots}
     |  
     |  PCS\_BEIJING\_1954\_3\_DEGREE\_GK\_25N = PrjCoordSysType.PCS\_BEIJING\_1954\_3\_{\ldots}
     |  
     |  PCS\_BEIJING\_1954\_3\_DEGREE\_GK\_26 = PrjCoordSysType.PCS\_BEIJING\_1954\_3\_D{\ldots}
     |  
     |  PCS\_BEIJING\_1954\_3\_DEGREE\_GK\_26N = PrjCoordSysType.PCS\_BEIJING\_1954\_3\_{\ldots}
     |  
     |  PCS\_BEIJING\_1954\_3\_DEGREE\_GK\_27 = PrjCoordSysType.PCS\_BEIJING\_1954\_3\_D{\ldots}
     |  
     |  PCS\_BEIJING\_1954\_3\_DEGREE\_GK\_27N = PrjCoordSysType.PCS\_BEIJING\_1954\_3\_{\ldots}
     |  
     |  PCS\_BEIJING\_1954\_3\_DEGREE\_GK\_28 = PrjCoordSysType.PCS\_BEIJING\_1954\_3\_D{\ldots}
     |  
     |  PCS\_BEIJING\_1954\_3\_DEGREE\_GK\_28N = PrjCoordSysType.PCS\_BEIJING\_1954\_3\_{\ldots}
     |  
     |  PCS\_BEIJING\_1954\_3\_DEGREE\_GK\_29 = PrjCoordSysType.PCS\_BEIJING\_1954\_3\_D{\ldots}
     |  
     |  PCS\_BEIJING\_1954\_3\_DEGREE\_GK\_29N = PrjCoordSysType.PCS\_BEIJING\_1954\_3\_{\ldots}
     |  
     |  PCS\_BEIJING\_1954\_3\_DEGREE\_GK\_30 = PrjCoordSysType.PCS\_BEIJING\_1954\_3\_D{\ldots}
     |  
     |  PCS\_BEIJING\_1954\_3\_DEGREE\_GK\_30N = PrjCoordSysType.PCS\_BEIJING\_1954\_3\_{\ldots}
     |  
     |  PCS\_BEIJING\_1954\_3\_DEGREE\_GK\_31 = PrjCoordSysType.PCS\_BEIJING\_1954\_3\_D{\ldots}
     |  
     |  PCS\_BEIJING\_1954\_3\_DEGREE\_GK\_31N = PrjCoordSysType.PCS\_BEIJING\_1954\_3\_{\ldots}
     |  
     |  PCS\_BEIJING\_1954\_3\_DEGREE\_GK\_32 = PrjCoordSysType.PCS\_BEIJING\_1954\_3\_D{\ldots}
     |  
     |  PCS\_BEIJING\_1954\_3\_DEGREE\_GK\_32N = PrjCoordSysType.PCS\_BEIJING\_1954\_3\_{\ldots}
     |  
     |  PCS\_BEIJING\_1954\_3\_DEGREE\_GK\_33 = PrjCoordSysType.PCS\_BEIJING\_1954\_3\_D{\ldots}
     |  
     |  PCS\_BEIJING\_1954\_3\_DEGREE\_GK\_33N = PrjCoordSysType.PCS\_BEIJING\_1954\_3\_{\ldots}
     |  
     |  PCS\_BEIJING\_1954\_3\_DEGREE\_GK\_34 = PrjCoordSysType.PCS\_BEIJING\_1954\_3\_D{\ldots}
     |  
     |  PCS\_BEIJING\_1954\_3\_DEGREE\_GK\_34N = PrjCoordSysType.PCS\_BEIJING\_1954\_3\_{\ldots}
     |  
     |  PCS\_BEIJING\_1954\_3\_DEGREE\_GK\_35 = PrjCoordSysType.PCS\_BEIJING\_1954\_3\_D{\ldots}
     |  
     |  PCS\_BEIJING\_1954\_3\_DEGREE\_GK\_35N = PrjCoordSysType.PCS\_BEIJING\_1954\_3\_{\ldots}
     |  
     |  PCS\_BEIJING\_1954\_3\_DEGREE\_GK\_36 = PrjCoordSysType.PCS\_BEIJING\_1954\_3\_D{\ldots}
     |  
     |  PCS\_BEIJING\_1954\_3\_DEGREE\_GK\_36N = PrjCoordSysType.PCS\_BEIJING\_1954\_3\_{\ldots}
     |  
     |  PCS\_BEIJING\_1954\_3\_DEGREE\_GK\_37 = PrjCoordSysType.PCS\_BEIJING\_1954\_3\_D{\ldots}
     |  
     |  PCS\_BEIJING\_1954\_3\_DEGREE\_GK\_37N = PrjCoordSysType.PCS\_BEIJING\_1954\_3\_{\ldots}
     |  
     |  PCS\_BEIJING\_1954\_3\_DEGREE\_GK\_38 = PrjCoordSysType.PCS\_BEIJING\_1954\_3\_D{\ldots}
     |  
     |  PCS\_BEIJING\_1954\_3\_DEGREE\_GK\_38N = PrjCoordSysType.PCS\_BEIJING\_1954\_3\_{\ldots}
     |  
     |  PCS\_BEIJING\_1954\_3\_DEGREE\_GK\_39 = PrjCoordSysType.PCS\_BEIJING\_1954\_3\_D{\ldots}
     |  
     |  PCS\_BEIJING\_1954\_3\_DEGREE\_GK\_39N = PrjCoordSysType.PCS\_BEIJING\_1954\_3\_{\ldots}
     |  
     |  PCS\_BEIJING\_1954\_3\_DEGREE\_GK\_40 = PrjCoordSysType.PCS\_BEIJING\_1954\_3\_D{\ldots}
     |  
     |  PCS\_BEIJING\_1954\_3\_DEGREE\_GK\_40N = PrjCoordSysType.PCS\_BEIJING\_1954\_3\_{\ldots}
     |  
     |  PCS\_BEIJING\_1954\_3\_DEGREE\_GK\_41 = PrjCoordSysType.PCS\_BEIJING\_1954\_3\_D{\ldots}
     |  
     |  PCS\_BEIJING\_1954\_3\_DEGREE\_GK\_41N = PrjCoordSysType.PCS\_BEIJING\_1954\_3\_{\ldots}
     |  
     |  PCS\_BEIJING\_1954\_3\_DEGREE\_GK\_42 = PrjCoordSysType.PCS\_BEIJING\_1954\_3\_D{\ldots}
     |  
     |  PCS\_BEIJING\_1954\_3\_DEGREE\_GK\_42N = PrjCoordSysType.PCS\_BEIJING\_1954\_3\_{\ldots}
     |  
     |  PCS\_BEIJING\_1954\_3\_DEGREE\_GK\_43 = PrjCoordSysType.PCS\_BEIJING\_1954\_3\_D{\ldots}
     |  
     |  PCS\_BEIJING\_1954\_3\_DEGREE\_GK\_43N = PrjCoordSysType.PCS\_BEIJING\_1954\_3\_{\ldots}
     |  
     |  PCS\_BEIJING\_1954\_3\_DEGREE\_GK\_44 = PrjCoordSysType.PCS\_BEIJING\_1954\_3\_D{\ldots}
     |  
     |  PCS\_BEIJING\_1954\_3\_DEGREE\_GK\_44N = PrjCoordSysType.PCS\_BEIJING\_1954\_3\_{\ldots}
     |  
     |  PCS\_BEIJING\_1954\_3\_DEGREE\_GK\_45 = PrjCoordSysType.PCS\_BEIJING\_1954\_3\_D{\ldots}
     |  
     |  PCS\_BEIJING\_1954\_3\_DEGREE\_GK\_45N = PrjCoordSysType.PCS\_BEIJING\_1954\_3\_{\ldots}
     |  
     |  PCS\_BEIJING\_1954\_GK\_13 = PrjCoordSysType.PCS\_BEIJING\_1954\_GK\_13
     |  
     |  PCS\_BEIJING\_1954\_GK\_13N = PrjCoordSysType.PCS\_BEIJING\_1954\_GK\_13N
     |  
     |  PCS\_BEIJING\_1954\_GK\_14 = PrjCoordSysType.PCS\_BEIJING\_1954\_GK\_14
     |  
     |  PCS\_BEIJING\_1954\_GK\_14N = PrjCoordSysType.PCS\_BEIJING\_1954\_GK\_14N
     |  
     |  PCS\_BEIJING\_1954\_GK\_15 = PrjCoordSysType.PCS\_BEIJING\_1954\_GK\_15
     |  
     |  PCS\_BEIJING\_1954\_GK\_15N = PrjCoordSysType.PCS\_BEIJING\_1954\_GK\_15N
     |  
     |  PCS\_BEIJING\_1954\_GK\_16 = PrjCoordSysType.PCS\_BEIJING\_1954\_GK\_16
     |  
     |  PCS\_BEIJING\_1954\_GK\_16N = PrjCoordSysType.PCS\_BEIJING\_1954\_GK\_16N
     |  
     |  PCS\_BEIJING\_1954\_GK\_17 = PrjCoordSysType.PCS\_BEIJING\_1954\_GK\_17
     |  
     |  PCS\_BEIJING\_1954\_GK\_17N = PrjCoordSysType.PCS\_BEIJING\_1954\_GK\_17N
     |  
     |  PCS\_BEIJING\_1954\_GK\_18 = PrjCoordSysType.PCS\_BEIJING\_1954\_GK\_18
     |  
     |  PCS\_BEIJING\_1954\_GK\_18N = PrjCoordSysType.PCS\_BEIJING\_1954\_GK\_18N
     |  
     |  PCS\_BEIJING\_1954\_GK\_19 = PrjCoordSysType.PCS\_BEIJING\_1954\_GK\_19
     |  
     |  PCS\_BEIJING\_1954\_GK\_19N = PrjCoordSysType.PCS\_BEIJING\_1954\_GK\_19N
     |  
     |  PCS\_BEIJING\_1954\_GK\_20 = PrjCoordSysType.PCS\_BEIJING\_1954\_GK\_20
     |  
     |  PCS\_BEIJING\_1954\_GK\_20N = PrjCoordSysType.PCS\_BEIJING\_1954\_GK\_20N
     |  
     |  PCS\_BEIJING\_1954\_GK\_21 = PrjCoordSysType.PCS\_BEIJING\_1954\_GK\_21
     |  
     |  PCS\_BEIJING\_1954\_GK\_21N = PrjCoordSysType.PCS\_BEIJING\_1954\_GK\_21N
     |  
     |  PCS\_BEIJING\_1954\_GK\_22 = PrjCoordSysType.PCS\_BEIJING\_1954\_GK\_22
     |  
     |  PCS\_BEIJING\_1954\_GK\_22N = PrjCoordSysType.PCS\_BEIJING\_1954\_GK\_22N
     |  
     |  PCS\_BEIJING\_1954\_GK\_23 = PrjCoordSysType.PCS\_BEIJING\_1954\_GK\_23
     |  
     |  PCS\_BEIJING\_1954\_GK\_23N = PrjCoordSysType.PCS\_BEIJING\_1954\_GK\_23N
     |  
     |  PCS\_BELGE\_LAMBERT\_1950 = PrjCoordSysType.PCS\_BELGE\_LAMBERT\_1950
     |  
     |  PCS\_BOGOTA\_COLOMBIA\_BOGOTA = PrjCoordSysType.PCS\_BOGOTA\_COLOMBIA\_BOGOT{\ldots}
     |  
     |  PCS\_BOGOTA\_COLOMBIA\_EAST = PrjCoordSysType.PCS\_BOGOTA\_COLOMBIA\_EAST
     |  
     |  PCS\_BOGOTA\_COLOMBIA\_E\_CENTRAL = PrjCoordSysType.PCS\_BOGOTA\_COLOMBIA\_E\_{\ldots}
     |  
     |  PCS\_BOGOTA\_COLOMBIA\_WEST = PrjCoordSysType.PCS\_BOGOTA\_COLOMBIA\_WEST
     |  
     |  PCS\_BOGOTA\_UTM\_17N = PrjCoordSysType.PCS\_BOGOTA\_UTM\_17N
     |  
     |  PCS\_BOGOTA\_UTM\_18N = PrjCoordSysType.PCS\_BOGOTA\_UTM\_18N
     |  
     |  PCS\_CAMACUPA\_UTM\_32S = PrjCoordSysType.PCS\_CAMACUPA\_UTM\_32S
     |  
     |  PCS\_CAMACUPA\_UTM\_33S = PrjCoordSysType.PCS\_CAMACUPA\_UTM\_33S
     |  
     |  PCS\_CARTHAGE\_NORD\_TUNISIE = PrjCoordSysType.PCS\_CARTHAGE\_NORD\_TUNISIE
     |  
     |  PCS\_CARTHAGE\_SUD\_TUNISIE = PrjCoordSysType.PCS\_CARTHAGE\_SUD\_TUNISIE
     |  
     |  PCS\_CARTHAGE\_UTM\_32N = PrjCoordSysType.PCS\_CARTHAGE\_UTM\_32N
     |  
     |  PCS\_CHINA\_2000\_3\_DEGREE\_GK\_25 = PrjCoordSysType.PCS\_CHINA\_2000\_3\_DEGRE{\ldots}
     |  
     |  PCS\_CHINA\_2000\_3\_DEGREE\_GK\_25N = PrjCoordSysType.PCS\_CHINA\_2000\_3\_DEGR{\ldots}
     |  
     |  PCS\_CHINA\_2000\_3\_DEGREE\_GK\_26 = PrjCoordSysType.PCS\_CHINA\_2000\_3\_DEGRE{\ldots}
     |  
     |  PCS\_CHINA\_2000\_3\_DEGREE\_GK\_26N = PrjCoordSysType.PCS\_CHINA\_2000\_3\_DEGR{\ldots}
     |  
     |  PCS\_CHINA\_2000\_3\_DEGREE\_GK\_27 = PrjCoordSysType.PCS\_CHINA\_2000\_3\_DEGRE{\ldots}
     |  
     |  PCS\_CHINA\_2000\_3\_DEGREE\_GK\_27N = PrjCoordSysType.PCS\_CHINA\_2000\_3\_DEGR{\ldots}
     |  
     |  PCS\_CHINA\_2000\_3\_DEGREE\_GK\_28 = PrjCoordSysType.PCS\_CHINA\_2000\_3\_DEGRE{\ldots}
     |  
     |  PCS\_CHINA\_2000\_3\_DEGREE\_GK\_28N = PrjCoordSysType.PCS\_CHINA\_2000\_3\_DEGR{\ldots}
     |  
     |  PCS\_CHINA\_2000\_3\_DEGREE\_GK\_29 = PrjCoordSysType.PCS\_CHINA\_2000\_3\_DEGRE{\ldots}
     |  
     |  PCS\_CHINA\_2000\_3\_DEGREE\_GK\_29N = PrjCoordSysType.PCS\_CHINA\_2000\_3\_DEGR{\ldots}
     |  
     |  PCS\_CHINA\_2000\_3\_DEGREE\_GK\_30 = PrjCoordSysType.PCS\_CHINA\_2000\_3\_DEGRE{\ldots}
     |  
     |  PCS\_CHINA\_2000\_3\_DEGREE\_GK\_30N = PrjCoordSysType.PCS\_CHINA\_2000\_3\_DEGR{\ldots}
     |  
     |  PCS\_CHINA\_2000\_3\_DEGREE\_GK\_31 = PrjCoordSysType.PCS\_CHINA\_2000\_3\_DEGRE{\ldots}
     |  
     |  PCS\_CHINA\_2000\_3\_DEGREE\_GK\_31N = PrjCoordSysType.PCS\_CHINA\_2000\_3\_DEGR{\ldots}
     |  
     |  PCS\_CHINA\_2000\_3\_DEGREE\_GK\_32 = PrjCoordSysType.PCS\_CHINA\_2000\_3\_DEGRE{\ldots}
     |  
     |  PCS\_CHINA\_2000\_3\_DEGREE\_GK\_32N = PrjCoordSysType.PCS\_CHINA\_2000\_3\_DEGR{\ldots}
     |  
     |  PCS\_CHINA\_2000\_3\_DEGREE\_GK\_33 = PrjCoordSysType.PCS\_CHINA\_2000\_3\_DEGRE{\ldots}
     |  
     |  PCS\_CHINA\_2000\_3\_DEGREE\_GK\_33N = PrjCoordSysType.PCS\_CHINA\_2000\_3\_DEGR{\ldots}
     |  
     |  PCS\_CHINA\_2000\_3\_DEGREE\_GK\_34 = PrjCoordSysType.PCS\_CHINA\_2000\_3\_DEGRE{\ldots}
     |  
     |  PCS\_CHINA\_2000\_3\_DEGREE\_GK\_34N = PrjCoordSysType.PCS\_CHINA\_2000\_3\_DEGR{\ldots}
     |  
     |  PCS\_CHINA\_2000\_3\_DEGREE\_GK\_35 = PrjCoordSysType.PCS\_CHINA\_2000\_3\_DEGRE{\ldots}
     |  
     |  PCS\_CHINA\_2000\_3\_DEGREE\_GK\_35N = PrjCoordSysType.PCS\_CHINA\_2000\_3\_DEGR{\ldots}
     |  
     |  PCS\_CHINA\_2000\_3\_DEGREE\_GK\_36 = PrjCoordSysType.PCS\_CHINA\_2000\_3\_DEGRE{\ldots}
     |  
     |  PCS\_CHINA\_2000\_3\_DEGREE\_GK\_36N = PrjCoordSysType.PCS\_CHINA\_2000\_3\_DEGR{\ldots}
     |  
     |  PCS\_CHINA\_2000\_3\_DEGREE\_GK\_37 = PrjCoordSysType.PCS\_CHINA\_2000\_3\_DEGRE{\ldots}
     |  
     |  PCS\_CHINA\_2000\_3\_DEGREE\_GK\_37N = PrjCoordSysType.PCS\_CHINA\_2000\_3\_DEGR{\ldots}
     |  
     |  PCS\_CHINA\_2000\_3\_DEGREE\_GK\_38 = PrjCoordSysType.PCS\_CHINA\_2000\_3\_DEGRE{\ldots}
     |  
     |  PCS\_CHINA\_2000\_3\_DEGREE\_GK\_38N = PrjCoordSysType.PCS\_CHINA\_2000\_3\_DEGR{\ldots}
     |  
     |  PCS\_CHINA\_2000\_3\_DEGREE\_GK\_39 = PrjCoordSysType.PCS\_CHINA\_2000\_3\_DEGRE{\ldots}
     |  
     |  PCS\_CHINA\_2000\_3\_DEGREE\_GK\_39N = PrjCoordSysType.PCS\_CHINA\_2000\_3\_DEGR{\ldots}
     |  
     |  PCS\_CHINA\_2000\_3\_DEGREE\_GK\_40 = PrjCoordSysType.PCS\_CHINA\_2000\_3\_DEGRE{\ldots}
     |  
     |  PCS\_CHINA\_2000\_3\_DEGREE\_GK\_40N = PrjCoordSysType.PCS\_CHINA\_2000\_3\_DEGR{\ldots}
     |  
     |  PCS\_CHINA\_2000\_3\_DEGREE\_GK\_41 = PrjCoordSysType.PCS\_CHINA\_2000\_3\_DEGRE{\ldots}
     |  
     |  PCS\_CHINA\_2000\_3\_DEGREE\_GK\_41N = PrjCoordSysType.PCS\_CHINA\_2000\_3\_DEGR{\ldots}
     |  
     |  PCS\_CHINA\_2000\_3\_DEGREE\_GK\_42 = PrjCoordSysType.PCS\_CHINA\_2000\_3\_DEGRE{\ldots}
     |  
     |  PCS\_CHINA\_2000\_3\_DEGREE\_GK\_42N = PrjCoordSysType.PCS\_CHINA\_2000\_3\_DEGR{\ldots}
     |  
     |  PCS\_CHINA\_2000\_3\_DEGREE\_GK\_43 = PrjCoordSysType.PCS\_CHINA\_2000\_3\_DEGRE{\ldots}
     |  
     |  PCS\_CHINA\_2000\_3\_DEGREE\_GK\_43N = PrjCoordSysType.PCS\_CHINA\_2000\_3\_DEGR{\ldots}
     |  
     |  PCS\_CHINA\_2000\_3\_DEGREE\_GK\_44 = PrjCoordSysType.PCS\_CHINA\_2000\_3\_DEGRE{\ldots}
     |  
     |  PCS\_CHINA\_2000\_3\_DEGREE\_GK\_44N = PrjCoordSysType.PCS\_CHINA\_2000\_3\_DEGR{\ldots}
     |  
     |  PCS\_CHINA\_2000\_3\_DEGREE\_GK\_45 = PrjCoordSysType.PCS\_CHINA\_2000\_3\_DEGRE{\ldots}
     |  
     |  PCS\_CHINA\_2000\_3\_DEGREE\_GK\_45N = PrjCoordSysType.PCS\_CHINA\_2000\_3\_DEGR{\ldots}
     |  
     |  PCS\_CHINA\_2000\_GK\_13 = PrjCoordSysType.PCS\_CHINA\_2000\_GK\_13
     |  
     |  PCS\_CHINA\_2000\_GK\_13N = PrjCoordSysType.PCS\_CHINA\_2000\_GK\_13N
     |  
     |  PCS\_CHINA\_2000\_GK\_14 = PrjCoordSysType.PCS\_CHINA\_2000\_GK\_14
     |  
     |  PCS\_CHINA\_2000\_GK\_14N = PrjCoordSysType.PCS\_CHINA\_2000\_GK\_14N
     |  
     |  PCS\_CHINA\_2000\_GK\_15 = PrjCoordSysType.PCS\_CHINA\_2000\_GK\_15
     |  
     |  PCS\_CHINA\_2000\_GK\_15N = PrjCoordSysType.PCS\_CHINA\_2000\_GK\_15N
     |  
     |  PCS\_CHINA\_2000\_GK\_16 = PrjCoordSysType.PCS\_CHINA\_2000\_GK\_16
     |  
     |  PCS\_CHINA\_2000\_GK\_16N = PrjCoordSysType.PCS\_CHINA\_2000\_GK\_16N
     |  
     |  PCS\_CHINA\_2000\_GK\_17 = PrjCoordSysType.PCS\_CHINA\_2000\_GK\_17
     |  
     |  PCS\_CHINA\_2000\_GK\_17N = PrjCoordSysType.PCS\_CHINA\_2000\_GK\_17N
     |  
     |  PCS\_CHINA\_2000\_GK\_18 = PrjCoordSysType.PCS\_CHINA\_2000\_GK\_18
     |  
     |  PCS\_CHINA\_2000\_GK\_18N = PrjCoordSysType.PCS\_CHINA\_2000\_GK\_18N
     |  
     |  PCS\_CHINA\_2000\_GK\_19 = PrjCoordSysType.PCS\_CHINA\_2000\_GK\_19
     |  
     |  PCS\_CHINA\_2000\_GK\_19N = PrjCoordSysType.PCS\_CHINA\_2000\_GK\_19N
     |  
     |  PCS\_CHINA\_2000\_GK\_20 = PrjCoordSysType.PCS\_CHINA\_2000\_GK\_20
     |  
     |  PCS\_CHINA\_2000\_GK\_20N = PrjCoordSysType.PCS\_CHINA\_2000\_GK\_20N
     |  
     |  PCS\_CHINA\_2000\_GK\_21 = PrjCoordSysType.PCS\_CHINA\_2000\_GK\_21
     |  
     |  PCS\_CHINA\_2000\_GK\_21N = PrjCoordSysType.PCS\_CHINA\_2000\_GK\_21N
     |  
     |  PCS\_CHINA\_2000\_GK\_22 = PrjCoordSysType.PCS\_CHINA\_2000\_GK\_22
     |  
     |  PCS\_CHINA\_2000\_GK\_22N = PrjCoordSysType.PCS\_CHINA\_2000\_GK\_22N
     |  
     |  PCS\_CHINA\_2000\_GK\_23 = PrjCoordSysType.PCS\_CHINA\_2000\_GK\_23
     |  
     |  PCS\_CHINA\_2000\_GK\_23N = PrjCoordSysType.PCS\_CHINA\_2000\_GK\_23N
     |  
     |  PCS\_CORREGO\_ALEGRE\_UTM\_23S = PrjCoordSysType.PCS\_CORREGO\_ALEGRE\_UTM\_23{\ldots}
     |  
     |  PCS\_CORREGO\_ALEGRE\_UTM\_24S = PrjCoordSysType.PCS\_CORREGO\_ALEGRE\_UTM\_24{\ldots}
     |  
     |  PCS\_C\_INCHAUSARGENTINA\_1 = PrjCoordSysType.PCS\_C\_INCHAUSARGENTINA\_1
     |  
     |  PCS\_C\_INCHAUSARGENTINA\_2 = PrjCoordSysType.PCS\_C\_INCHAUSARGENTINA\_2
     |  
     |  PCS\_C\_INCHAUSARGENTINA\_3 = PrjCoordSysType.PCS\_C\_INCHAUSARGENTINA\_3
     |  
     |  PCS\_C\_INCHAUSARGENTINA\_4 = PrjCoordSysType.PCS\_C\_INCHAUSARGENTINA\_4
     |  
     |  PCS\_C\_INCHAUSARGENTINA\_5 = PrjCoordSysType.PCS\_C\_INCHAUSARGENTINA\_5
     |  
     |  PCS\_C\_INCHAUSARGENTINA\_6 = PrjCoordSysType.PCS\_C\_INCHAUSARGENTINA\_6
     |  
     |  PCS\_C\_INCHAUSARGENTINA\_7 = PrjCoordSysType.PCS\_C\_INCHAUSARGENTINA\_7
     |  
     |  PCS\_DATUM73\_MODIFIED\_PORTUGUESE\_GRID = PrjCoordSysType.PCS\_DATUM73\_MOD{\ldots}
     |  
     |  PCS\_DATUM73\_MODIFIED\_PORTUGUESE\_NATIONAL\_GRID = PrjCoordSysType.PCS\_DA{\ldots}
     |  
     |  PCS\_DATUM\_73\_UTM\_ZONE\_29N = PrjCoordSysType.PCS\_DATUM\_73\_UTM\_ZONE\_29N
     |  
     |  PCS\_DEALUL\_PISCULUI\_1933\_STEREO\_33 = PrjCoordSysType.PCS\_DEALUL\_PISCUL{\ldots}
     |  
     |  PCS\_DEALUL\_PISCULUI\_1970\_STEREO\_EALUL\_PISCULUI\_1970\_STEREO\_70 = PrjCoo{\ldots}
     |  
     |  PCS\_DHDN\_GERMANY\_1 = PrjCoordSysType.PCS\_DHDN\_GERMANY\_1
     |  
     |  PCS\_DHDN\_GERMANY\_2 = PrjCoordSysType.PCS\_DHDN\_GERMANY\_2
     |  
     |  PCS\_DHDN\_GERMANY\_3 = PrjCoordSysType.PCS\_DHDN\_GERMANY\_3
     |  
     |  PCS\_DHDN\_GERMANY\_4 = PrjCoordSysType.PCS\_DHDN\_GERMANY\_4
     |  
     |  PCS\_DHDN\_GERMANY\_5 = PrjCoordSysType.PCS\_DHDN\_GERMANY\_5
     |  
     |  PCS\_DOUALA\_UTM\_32N = PrjCoordSysType.PCS\_DOUALA\_UTM\_32N
     |  
     |  PCS\_EARTH\_LONGITUDE\_LATITUDE = PrjCoordSysType.PCS\_EARTH\_LONGITUDE\_LAT{\ldots}
     |  
     |  PCS\_ED50\_CENTRAL\_GROUP = PrjCoordSysType.PCS\_ED50\_CENTRAL\_GROUP
     |  
     |  PCS\_ED50\_OCCIDENTAL\_GROUP = PrjCoordSysType.PCS\_ED50\_OCCIDENTAL\_GROUP
     |  
     |  PCS\_ED50\_ORIENTAL\_GROUP = PrjCoordSysType.PCS\_ED50\_ORIENTAL\_GROUP
     |  
     |  PCS\_ED\_1950\_UTM\_28N = PrjCoordSysType.PCS\_ED\_1950\_UTM\_28N
     |  
     |  PCS\_ED\_1950\_UTM\_29N = PrjCoordSysType.PCS\_ED\_1950\_UTM\_29N
     |  
     |  PCS\_ED\_1950\_UTM\_30N = PrjCoordSysType.PCS\_ED\_1950\_UTM\_30N
     |  
     |  PCS\_ED\_1950\_UTM\_31N = PrjCoordSysType.PCS\_ED\_1950\_UTM\_31N
     |  
     |  PCS\_ED\_1950\_UTM\_32N = PrjCoordSysType.PCS\_ED\_1950\_UTM\_32N
     |  
     |  PCS\_ED\_1950\_UTM\_33N = PrjCoordSysType.PCS\_ED\_1950\_UTM\_33N
     |  
     |  PCS\_ED\_1950\_UTM\_34N = PrjCoordSysType.PCS\_ED\_1950\_UTM\_34N
     |  
     |  PCS\_ED\_1950\_UTM\_35N = PrjCoordSysType.PCS\_ED\_1950\_UTM\_35N
     |  
     |  PCS\_ED\_1950\_UTM\_36N = PrjCoordSysType.PCS\_ED\_1950\_UTM\_36N
     |  
     |  PCS\_ED\_1950\_UTM\_37N = PrjCoordSysType.PCS\_ED\_1950\_UTM\_37N
     |  
     |  PCS\_ED\_1950\_UTM\_38N = PrjCoordSysType.PCS\_ED\_1950\_UTM\_38N
     |  
     |  PCS\_EGYPT\_EXT\_PURPLE\_BELT = PrjCoordSysType.PCS\_EGYPT\_EXT\_PURPLE\_BELT
     |  
     |  PCS\_EGYPT\_PURPLE\_BELT = PrjCoordSysType.PCS\_EGYPT\_PURPLE\_BELT
     |  
     |  PCS\_EGYPT\_RED\_BELT = PrjCoordSysType.PCS\_EGYPT\_RED\_BELT
     |  
     |  PCS\_ETRS89\_PORTUGAL\_TM06 = PrjCoordSysType.PCS\_ETRS89\_PORTUGAL\_TM06
     |  
     |  PCS\_ETRS\_1989\_UTM\_28N = PrjCoordSysType.PCS\_ETRS\_1989\_UTM\_28N
     |  
     |  PCS\_ETRS\_1989\_UTM\_29N = PrjCoordSysType.PCS\_ETRS\_1989\_UTM\_29N
     |  
     |  PCS\_ETRS\_1989\_UTM\_30N = PrjCoordSysType.PCS\_ETRS\_1989\_UTM\_30N
     |  
     |  PCS\_ETRS\_1989\_UTM\_31N = PrjCoordSysType.PCS\_ETRS\_1989\_UTM\_31N
     |  
     |  PCS\_ETRS\_1989\_UTM\_32N = PrjCoordSysType.PCS\_ETRS\_1989\_UTM\_32N
     |  
     |  PCS\_ETRS\_1989\_UTM\_33N = PrjCoordSysType.PCS\_ETRS\_1989\_UTM\_33N
     |  
     |  PCS\_ETRS\_1989\_UTM\_34N = PrjCoordSysType.PCS\_ETRS\_1989\_UTM\_34N
     |  
     |  PCS\_ETRS\_1989\_UTM\_35N = PrjCoordSysType.PCS\_ETRS\_1989\_UTM\_35N
     |  
     |  PCS\_ETRS\_1989\_UTM\_36N = PrjCoordSysType.PCS\_ETRS\_1989\_UTM\_36N
     |  
     |  PCS\_ETRS\_1989\_UTM\_37N = PrjCoordSysType.PCS\_ETRS\_1989\_UTM\_37N
     |  
     |  PCS\_ETRS\_1989\_UTM\_38N = PrjCoordSysType.PCS\_ETRS\_1989\_UTM\_38N
     |  
     |  PCS\_FAHUD\_UTM\_39N = PrjCoordSysType.PCS\_FAHUD\_UTM\_39N
     |  
     |  PCS\_FAHUD\_UTM\_40N = PrjCoordSysType.PCS\_FAHUD\_UTM\_40N
     |  
     |  PCS\_GAROUA\_UTM\_33N = PrjCoordSysType.PCS\_GAROUA\_UTM\_33N
     |  
     |  PCS\_GDA\_1994\_MGA\_48 = PrjCoordSysType.PCS\_GDA\_1994\_MGA\_48
     |  
     |  PCS\_GDA\_1994\_MGA\_49 = PrjCoordSysType.PCS\_GDA\_1994\_MGA\_49
     |  
     |  PCS\_GDA\_1994\_MGA\_50 = PrjCoordSysType.PCS\_GDA\_1994\_MGA\_50
     |  
     |  PCS\_GDA\_1994\_MGA\_51 = PrjCoordSysType.PCS\_GDA\_1994\_MGA\_51
     |  
     |  PCS\_GDA\_1994\_MGA\_52 = PrjCoordSysType.PCS\_GDA\_1994\_MGA\_52
     |  
     |  PCS\_GDA\_1994\_MGA\_53 = PrjCoordSysType.PCS\_GDA\_1994\_MGA\_53
     |  
     |  PCS\_GDA\_1994\_MGA\_54 = PrjCoordSysType.PCS\_GDA\_1994\_MGA\_54
     |  
     |  PCS\_GDA\_1994\_MGA\_55 = PrjCoordSysType.PCS\_GDA\_1994\_MGA\_55
     |  
     |  PCS\_GDA\_1994\_MGA\_56 = PrjCoordSysType.PCS\_GDA\_1994\_MGA\_56
     |  
     |  PCS\_GDA\_1994\_MGA\_57 = PrjCoordSysType.PCS\_GDA\_1994\_MGA\_57
     |  
     |  PCS\_GDA\_1994\_MGA\_58 = PrjCoordSysType.PCS\_GDA\_1994\_MGA\_58
     |  
     |  PCS\_GGRS\_1987\_GREEK\_GRID = PrjCoordSysType.PCS\_GGRS\_1987\_GREEK\_GRID
     |  
     |  PCS\_ID\_1974\_UTM\_46N = PrjCoordSysType.PCS\_ID\_1974\_UTM\_46N
     |  
     |  PCS\_ID\_1974\_UTM\_46S = PrjCoordSysType.PCS\_ID\_1974\_UTM\_46S
     |  
     |  PCS\_ID\_1974\_UTM\_47N = PrjCoordSysType.PCS\_ID\_1974\_UTM\_47N
     |  
     |  PCS\_ID\_1974\_UTM\_47S = PrjCoordSysType.PCS\_ID\_1974\_UTM\_47S
     |  
     |  PCS\_ID\_1974\_UTM\_48N = PrjCoordSysType.PCS\_ID\_1974\_UTM\_48N
     |  
     |  PCS\_ID\_1974\_UTM\_48S = PrjCoordSysType.PCS\_ID\_1974\_UTM\_48S
     |  
     |  PCS\_ID\_1974\_UTM\_49N = PrjCoordSysType.PCS\_ID\_1974\_UTM\_49N
     |  
     |  PCS\_ID\_1974\_UTM\_49S = PrjCoordSysType.PCS\_ID\_1974\_UTM\_49S
     |  
     |  PCS\_ID\_1974\_UTM\_50N = PrjCoordSysType.PCS\_ID\_1974\_UTM\_50N
     |  
     |  PCS\_ID\_1974\_UTM\_50S = PrjCoordSysType.PCS\_ID\_1974\_UTM\_50S
     |  
     |  PCS\_ID\_1974\_UTM\_51N = PrjCoordSysType.PCS\_ID\_1974\_UTM\_51N
     |  
     |  PCS\_ID\_1974\_UTM\_51S = PrjCoordSysType.PCS\_ID\_1974\_UTM\_51S
     |  
     |  PCS\_ID\_1974\_UTM\_52N = PrjCoordSysType.PCS\_ID\_1974\_UTM\_52N
     |  
     |  PCS\_ID\_1974\_UTM\_52S = PrjCoordSysType.PCS\_ID\_1974\_UTM\_52S
     |  
     |  PCS\_ID\_1974\_UTM\_53N = PrjCoordSysType.PCS\_ID\_1974\_UTM\_53N
     |  
     |  PCS\_ID\_1974\_UTM\_53S = PrjCoordSysType.PCS\_ID\_1974\_UTM\_53S
     |  
     |  PCS\_ID\_1974\_UTM\_54S = PrjCoordSysType.PCS\_ID\_1974\_UTM\_54S
     |  
     |  PCS\_INDIAN\_1954\_UTM\_47N = PrjCoordSysType.PCS\_INDIAN\_1954\_UTM\_47N
     |  
     |  PCS\_INDIAN\_1954\_UTM\_48N = PrjCoordSysType.PCS\_INDIAN\_1954\_UTM\_48N
     |  
     |  PCS\_INDIAN\_1975\_UTM\_47N = PrjCoordSysType.PCS\_INDIAN\_1975\_UTM\_47N
     |  
     |  PCS\_INDIAN\_1975\_UTM\_48N = PrjCoordSysType.PCS\_INDIAN\_1975\_UTM\_48N
     |  
     |  PCS\_JAD\_1969\_JAMAICA\_GRID = PrjCoordSysType.PCS\_JAD\_1969\_JAMAICA\_GRID
     |  
     |  PCS\_JAMAICA\_1875\_OLD\_GRID = PrjCoordSysType.PCS\_JAMAICA\_1875\_OLD\_GRID
     |  
     |  PCS\_JAPAN\_PLATE\_ZONE\_I = PrjCoordSysType.PCS\_JAPAN\_PLATE\_ZONE\_I
     |  
     |  PCS\_JAPAN\_PLATE\_ZONE\_II = PrjCoordSysType.PCS\_JAPAN\_PLATE\_ZONE\_II
     |  
     |  PCS\_JAPAN\_PLATE\_ZONE\_III = PrjCoordSysType.PCS\_JAPAN\_PLATE\_ZONE\_III
     |  
     |  PCS\_JAPAN\_PLATE\_ZONE\_IV = PrjCoordSysType.PCS\_JAPAN\_PLATE\_ZONE\_IV
     |  
     |  PCS\_JAPAN\_PLATE\_ZONE\_IX = PrjCoordSysType.PCS\_JAPAN\_PLATE\_ZONE\_IX
     |  
     |  PCS\_JAPAN\_PLATE\_ZONE\_V = PrjCoordSysType.PCS\_JAPAN\_PLATE\_ZONE\_V
     |  
     |  PCS\_JAPAN\_PLATE\_ZONE\_VI = PrjCoordSysType.PCS\_JAPAN\_PLATE\_ZONE\_VI
     |  
     |  PCS\_JAPAN\_PLATE\_ZONE\_VII = PrjCoordSysType.PCS\_JAPAN\_PLATE\_ZONE\_VII
     |  
     |  PCS\_JAPAN\_PLATE\_ZONE\_VIII = PrjCoordSysType.PCS\_JAPAN\_PLATE\_ZONE\_VIII
     |  
     |  PCS\_JAPAN\_PLATE\_ZONE\_X = PrjCoordSysType.PCS\_JAPAN\_PLATE\_ZONE\_X
     |  
     |  PCS\_JAPAN\_PLATE\_ZONE\_XI = PrjCoordSysType.PCS\_JAPAN\_PLATE\_ZONE\_XI
     |  
     |  PCS\_JAPAN\_PLATE\_ZONE\_XII = PrjCoordSysType.PCS\_JAPAN\_PLATE\_ZONE\_XII
     |  
     |  PCS\_JAPAN\_PLATE\_ZONE\_XIII = PrjCoordSysType.PCS\_JAPAN\_PLATE\_ZONE\_XIII
     |  
     |  PCS\_JAPAN\_PLATE\_ZONE\_XIV = PrjCoordSysType.PCS\_JAPAN\_PLATE\_ZONE\_XIV
     |  
     |  PCS\_JAPAN\_PLATE\_ZONE\_XIX = PrjCoordSysType.PCS\_JAPAN\_PLATE\_ZONE\_XIX
     |  
     |  PCS\_JAPAN\_PLATE\_ZONE\_XV = PrjCoordSysType.PCS\_JAPAN\_PLATE\_ZONE\_XV
     |  
     |  PCS\_JAPAN\_PLATE\_ZONE\_XVI = PrjCoordSysType.PCS\_JAPAN\_PLATE\_ZONE\_XVI
     |  
     |  PCS\_JAPAN\_PLATE\_ZONE\_XVII = PrjCoordSysType.PCS\_JAPAN\_PLATE\_ZONE\_XVII
     |  
     |  PCS\_JAPAN\_PLATE\_ZONE\_XVIII = PrjCoordSysType.PCS\_JAPAN\_PLATE\_ZONE\_XVII{\ldots}
     |  
     |  PCS\_JAPAN\_UTM\_51 = PrjCoordSysType.PCS\_JAPAN\_UTM\_51
     |  
     |  PCS\_JAPAN\_UTM\_52 = PrjCoordSysType.PCS\_JAPAN\_UTM\_52
     |  
     |  PCS\_JAPAN\_UTM\_53 = PrjCoordSysType.PCS\_JAPAN\_UTM\_53
     |  
     |  PCS\_JAPAN\_UTM\_54 = PrjCoordSysType.PCS\_JAPAN\_UTM\_54
     |  
     |  PCS\_JAPAN\_UTM\_55 = PrjCoordSysType.PCS\_JAPAN\_UTM\_55
     |  
     |  PCS\_JAPAN\_UTM\_56 = PrjCoordSysType.PCS\_JAPAN\_UTM\_56
     |  
     |  PCS\_KALIANPUR\_INDIA\_0 = PrjCoordSysType.PCS\_KALIANPUR\_INDIA\_0
     |  
     |  PCS\_KALIANPUR\_INDIA\_I = PrjCoordSysType.PCS\_KALIANPUR\_INDIA\_I
     |  
     |  PCS\_KALIANPUR\_INDIA\_IIA = PrjCoordSysType.PCS\_KALIANPUR\_INDIA\_IIA
     |  
     |  PCS\_KALIANPUR\_INDIA\_IIB = PrjCoordSysType.PCS\_KALIANPUR\_INDIA\_IIB
     |  
     |  PCS\_KALIANPUR\_INDIA\_IIIA = PrjCoordSysType.PCS\_KALIANPUR\_INDIA\_IIIA
     |  
     |  PCS\_KALIANPUR\_INDIA\_IIIB = PrjCoordSysType.PCS\_KALIANPUR\_INDIA\_IIIB
     |  
     |  PCS\_KALIANPUR\_INDIA\_IVA = PrjCoordSysType.PCS\_KALIANPUR\_INDIA\_IVA
     |  
     |  PCS\_KALIANPUR\_INDIA\_IVB = PrjCoordSysType.PCS\_KALIANPUR\_INDIA\_IVB
     |  
     |  PCS\_KERTAU\_MALAYA\_METERS = PrjCoordSysType.PCS\_KERTAU\_MALAYA\_METERS
     |  
     |  PCS\_KERTAU\_UTM\_47N = PrjCoordSysType.PCS\_KERTAU\_UTM\_47N
     |  
     |  PCS\_KERTAU\_UTM\_48N = PrjCoordSysType.PCS\_KERTAU\_UTM\_48N
     |  
     |  PCS\_KKJ\_FINLAND\_1 = PrjCoordSysType.PCS\_KKJ\_FINLAND\_1
     |  
     |  PCS\_KKJ\_FINLAND\_2 = PrjCoordSysType.PCS\_KKJ\_FINLAND\_2
     |  
     |  PCS\_KKJ\_FINLAND\_3 = PrjCoordSysType.PCS\_KKJ\_FINLAND\_3
     |  
     |  PCS\_KKJ\_FINLAND\_4 = PrjCoordSysType.PCS\_KKJ\_FINLAND\_4
     |  
     |  PCS\_KOC\_LAMBERT = PrjCoordSysType.PCS\_KOC\_LAMBERT
     |  
     |  PCS\_KUDAMS\_KTM = PrjCoordSysType.PCS\_KUDAMS\_KTM
     |  
     |  PCS\_LA\_CANOA\_UTM\_20N = PrjCoordSysType.PCS\_LA\_CANOA\_UTM\_20N
     |  
     |  PCS\_LA\_CANOA\_UTM\_21N = PrjCoordSysType.PCS\_LA\_CANOA\_UTM\_21N
     |  
     |  PCS\_LEIGON\_GHANA\_GRID = PrjCoordSysType.PCS\_LEIGON\_GHANA\_GRID
     |  
     |  PCS\_LISBON\_1890\_PORTUGAL\_BONNE = PrjCoordSysType.PCS\_LISBON\_1890\_PORTU{\ldots}
     |  
     |  PCS\_LISBON\_PORTUGUESE\_GRID = PrjCoordSysType.PCS\_LISBON\_PORTUGUESE\_GRI{\ldots}
     |  
     |  PCS\_LISBON\_PORTUGUESE\_OFFICIAL\_GRID = PrjCoordSysType.PCS\_LISBON\_PORTU{\ldots}
     |  
     |  PCS\_LOME\_UTM\_31N = PrjCoordSysType.PCS\_LOME\_UTM\_31N
     |  
     |  PCS\_LUZON\_PHILIPPINES\_I = PrjCoordSysType.PCS\_LUZON\_PHILIPPINES\_I
     |  
     |  PCS\_LUZON\_PHILIPPINES\_II = PrjCoordSysType.PCS\_LUZON\_PHILIPPINES\_II
     |  
     |  PCS\_LUZON\_PHILIPPINES\_III = PrjCoordSysType.PCS\_LUZON\_PHILIPPINES\_III
     |  
     |  PCS\_LUZON\_PHILIPPINES\_IV = PrjCoordSysType.PCS\_LUZON\_PHILIPPINES\_IV
     |  
     |  PCS\_LUZON\_PHILIPPINES\_V = PrjCoordSysType.PCS\_LUZON\_PHILIPPINES\_V
     |  
     |  PCS\_Lisboa\_Hayford\_Gauss\_IGeoE = PrjCoordSysType.PCS\_Lisboa\_Hayford\_Ga{\ldots}
     |  
     |  PCS\_Lisboa\_Hayford\_Gauss\_IPCC = PrjCoordSysType.PCS\_Lisboa\_Hayford\_Gau{\ldots}
     |  
     |  PCS\_MADEIRA\_1936\_UTM\_ZONE\_28N = PrjCoordSysType.PCS\_MADEIRA\_1936\_UTM\_Z{\ldots}
     |  
     |  PCS\_MALONGO\_1987\_UTM\_32S = PrjCoordSysType.PCS\_MALONGO\_1987\_UTM\_32S
     |  
     |  PCS\_MASSAWA\_UTM\_37N = PrjCoordSysType.PCS\_MASSAWA\_UTM\_37N
     |  
     |  PCS\_MERCHICH\_NORD\_MAROC = PrjCoordSysType.PCS\_MERCHICH\_NORD\_MAROC
     |  
     |  PCS\_MERCHICH\_SAHARA = PrjCoordSysType.PCS\_MERCHICH\_SAHARA
     |  
     |  PCS\_MERCHICH\_SUD\_MAROC = PrjCoordSysType.PCS\_MERCHICH\_SUD\_MAROC
     |  
     |  PCS\_MGI\_FERRO\_AUSTRIA\_CENTRAL = PrjCoordSysType.PCS\_MGI\_FERRO\_AUSTRIA\_{\ldots}
     |  
     |  PCS\_MGI\_FERRO\_AUSTRIA\_EAST = PrjCoordSysType.PCS\_MGI\_FERRO\_AUSTRIA\_EAS{\ldots}
     |  
     |  PCS\_MGI\_FERRO\_AUSTRIA\_WEST = PrjCoordSysType.PCS\_MGI\_FERRO\_AUSTRIA\_WES{\ldots}
     |  
     |  PCS\_MHAST\_UTM\_32S = PrjCoordSysType.PCS\_MHAST\_UTM\_32S
     |  
     |  PCS\_MINNA\_NIGERIA\_EAST\_BELT = PrjCoordSysType.PCS\_MINNA\_NIGERIA\_EAST\_B{\ldots}
     |  
     |  PCS\_MINNA\_NIGERIA\_MID\_BELT = PrjCoordSysType.PCS\_MINNA\_NIGERIA\_MID\_BEL{\ldots}
     |  
     |  PCS\_MINNA\_NIGERIA\_WEST\_BELT = PrjCoordSysType.PCS\_MINNA\_NIGERIA\_WEST\_B{\ldots}
     |  
     |  PCS\_MINNA\_UTM\_31N = PrjCoordSysType.PCS\_MINNA\_UTM\_31N
     |  
     |  PCS\_MINNA\_UTM\_32N = PrjCoordSysType.PCS\_MINNA\_UTM\_32N
     |  
     |  PCS\_MONTE\_MARIO\_ROME\_ITALY\_1 = PrjCoordSysType.PCS\_MONTE\_MARIO\_ROME\_IT{\ldots}
     |  
     |  PCS\_MONTE\_MARIO\_ROME\_ITALY\_2 = PrjCoordSysType.PCS\_MONTE\_MARIO\_ROME\_IT{\ldots}
     |  
     |  PCS\_MPORALOKO\_UTM\_32N = PrjCoordSysType.PCS\_MPORALOKO\_UTM\_32N
     |  
     |  PCS\_MPORALOKO\_UTM\_32S = PrjCoordSysType.PCS\_MPORALOKO\_UTM\_32S
     |  
     |  PCS\_NAD\_1927\_AK\_1 = PrjCoordSysType.PCS\_NAD\_1927\_AK\_1
     |  
     |  PCS\_NAD\_1927\_AK\_10 = PrjCoordSysType.PCS\_NAD\_1927\_AK\_10
     |  
     |  PCS\_NAD\_1927\_AK\_2 = PrjCoordSysType.PCS\_NAD\_1927\_AK\_2
     |  
     |  PCS\_NAD\_1927\_AK\_3 = PrjCoordSysType.PCS\_NAD\_1927\_AK\_3
     |  
     |  PCS\_NAD\_1927\_AK\_4 = PrjCoordSysType.PCS\_NAD\_1927\_AK\_4
     |  
     |  PCS\_NAD\_1927\_AK\_5 = PrjCoordSysType.PCS\_NAD\_1927\_AK\_5
     |  
     |  PCS\_NAD\_1927\_AK\_6 = PrjCoordSysType.PCS\_NAD\_1927\_AK\_6
     |  
     |  PCS\_NAD\_1927\_AK\_7 = PrjCoordSysType.PCS\_NAD\_1927\_AK\_7
     |  
     |  PCS\_NAD\_1927\_AK\_8 = PrjCoordSysType.PCS\_NAD\_1927\_AK\_8
     |  
     |  PCS\_NAD\_1927\_AK\_9 = PrjCoordSysType.PCS\_NAD\_1927\_AK\_9
     |  
     |  PCS\_NAD\_1927\_AL\_E = PrjCoordSysType.PCS\_NAD\_1927\_AL\_E
     |  
     |  PCS\_NAD\_1927\_AL\_W = PrjCoordSysType.PCS\_NAD\_1927\_AL\_W
     |  
     |  PCS\_NAD\_1927\_AR\_N = PrjCoordSysType.PCS\_NAD\_1927\_AR\_N
     |  
     |  PCS\_NAD\_1927\_AR\_S = PrjCoordSysType.PCS\_NAD\_1927\_AR\_S
     |  
     |  PCS\_NAD\_1927\_AZ\_C = PrjCoordSysType.PCS\_NAD\_1927\_AZ\_C
     |  
     |  PCS\_NAD\_1927\_AZ\_E = PrjCoordSysType.PCS\_NAD\_1927\_AZ\_E
     |  
     |  PCS\_NAD\_1927\_AZ\_W = PrjCoordSysType.PCS\_NAD\_1927\_AZ\_W
     |  
     |  PCS\_NAD\_1927\_BLM\_14N = PrjCoordSysType.PCS\_NAD\_1927\_BLM\_14N
     |  
     |  PCS\_NAD\_1927\_BLM\_15N = PrjCoordSysType.PCS\_NAD\_1927\_BLM\_15N
     |  
     |  PCS\_NAD\_1927\_BLM\_16N = PrjCoordSysType.PCS\_NAD\_1927\_BLM\_16N
     |  
     |  PCS\_NAD\_1927\_BLM\_17N = PrjCoordSysType.PCS\_NAD\_1927\_BLM\_17N
     |  
     |  PCS\_NAD\_1927\_CA\_I = PrjCoordSysType.PCS\_NAD\_1927\_CA\_I
     |  
     |  PCS\_NAD\_1927\_CA\_II = PrjCoordSysType.PCS\_NAD\_1927\_CA\_II
     |  
     |  PCS\_NAD\_1927\_CA\_III = PrjCoordSysType.PCS\_NAD\_1927\_CA\_III
     |  
     |  PCS\_NAD\_1927\_CA\_IV = PrjCoordSysType.PCS\_NAD\_1927\_CA\_IV
     |  
     |  PCS\_NAD\_1927\_CA\_V = PrjCoordSysType.PCS\_NAD\_1927\_CA\_V
     |  
     |  PCS\_NAD\_1927\_CA\_VI = PrjCoordSysType.PCS\_NAD\_1927\_CA\_VI
     |  
     |  PCS\_NAD\_1927\_CA\_VII = PrjCoordSysType.PCS\_NAD\_1927\_CA\_VII
     |  
     |  PCS\_NAD\_1927\_CO\_C = PrjCoordSysType.PCS\_NAD\_1927\_CO\_C
     |  
     |  PCS\_NAD\_1927\_CO\_N = PrjCoordSysType.PCS\_NAD\_1927\_CO\_N
     |  
     |  PCS\_NAD\_1927\_CO\_S = PrjCoordSysType.PCS\_NAD\_1927\_CO\_S
     |  
     |  PCS\_NAD\_1927\_CT = PrjCoordSysType.PCS\_NAD\_1927\_CT
     |  
     |  PCS\_NAD\_1927\_DE = PrjCoordSysType.PCS\_NAD\_1927\_DE
     |  
     |  PCS\_NAD\_1927\_FL\_E = PrjCoordSysType.PCS\_NAD\_1927\_FL\_E
     |  
     |  PCS\_NAD\_1927\_FL\_N = PrjCoordSysType.PCS\_NAD\_1927\_FL\_N
     |  
     |  PCS\_NAD\_1927\_FL\_W = PrjCoordSysType.PCS\_NAD\_1927\_FL\_W
     |  
     |  PCS\_NAD\_1927\_GA\_E = PrjCoordSysType.PCS\_NAD\_1927\_GA\_E
     |  
     |  PCS\_NAD\_1927\_GA\_W = PrjCoordSysType.PCS\_NAD\_1927\_GA\_W
     |  
     |  PCS\_NAD\_1927\_GU = PrjCoordSysType.PCS\_NAD\_1927\_GU
     |  
     |  PCS\_NAD\_1927\_HI\_1 = PrjCoordSysType.PCS\_NAD\_1927\_HI\_1
     |  
     |  PCS\_NAD\_1927\_HI\_2 = PrjCoordSysType.PCS\_NAD\_1927\_HI\_2
     |  
     |  PCS\_NAD\_1927\_HI\_3 = PrjCoordSysType.PCS\_NAD\_1927\_HI\_3
     |  
     |  PCS\_NAD\_1927\_HI\_4 = PrjCoordSysType.PCS\_NAD\_1927\_HI\_4
     |  
     |  PCS\_NAD\_1927\_HI\_5 = PrjCoordSysType.PCS\_NAD\_1927\_HI\_5
     |  
     |  PCS\_NAD\_1927\_IA\_N = PrjCoordSysType.PCS\_NAD\_1927\_IA\_N
     |  
     |  PCS\_NAD\_1927\_IA\_S = PrjCoordSysType.PCS\_NAD\_1927\_IA\_S
     |  
     |  PCS\_NAD\_1927\_ID\_C = PrjCoordSysType.PCS\_NAD\_1927\_ID\_C
     |  
     |  PCS\_NAD\_1927\_ID\_E = PrjCoordSysType.PCS\_NAD\_1927\_ID\_E
     |  
     |  PCS\_NAD\_1927\_ID\_W = PrjCoordSysType.PCS\_NAD\_1927\_ID\_W
     |  
     |  PCS\_NAD\_1927\_IL\_E = PrjCoordSysType.PCS\_NAD\_1927\_IL\_E
     |  
     |  PCS\_NAD\_1927\_IL\_W = PrjCoordSysType.PCS\_NAD\_1927\_IL\_W
     |  
     |  PCS\_NAD\_1927\_IN\_E = PrjCoordSysType.PCS\_NAD\_1927\_IN\_E
     |  
     |  PCS\_NAD\_1927\_IN\_W = PrjCoordSysType.PCS\_NAD\_1927\_IN\_W
     |  
     |  PCS\_NAD\_1927\_KS\_N = PrjCoordSysType.PCS\_NAD\_1927\_KS\_N
     |  
     |  PCS\_NAD\_1927\_KS\_S = PrjCoordSysType.PCS\_NAD\_1927\_KS\_S
     |  
     |  PCS\_NAD\_1927\_KY\_N = PrjCoordSysType.PCS\_NAD\_1927\_KY\_N
     |  
     |  PCS\_NAD\_1927\_KY\_S = PrjCoordSysType.PCS\_NAD\_1927\_KY\_S
     |  
     |  PCS\_NAD\_1927\_LA\_N = PrjCoordSysType.PCS\_NAD\_1927\_LA\_N
     |  
     |  PCS\_NAD\_1927\_LA\_S = PrjCoordSysType.PCS\_NAD\_1927\_LA\_S
     |  
     |  PCS\_NAD\_1927\_MA\_I = PrjCoordSysType.PCS\_NAD\_1927\_MA\_I
     |  
     |  PCS\_NAD\_1927\_MA\_M = PrjCoordSysType.PCS\_NAD\_1927\_MA\_M
     |  
     |  PCS\_NAD\_1927\_MD = PrjCoordSysType.PCS\_NAD\_1927\_MD
     |  
     |  PCS\_NAD\_1927\_ME\_E = PrjCoordSysType.PCS\_NAD\_1927\_ME\_E
     |  
     |  PCS\_NAD\_1927\_ME\_W = PrjCoordSysType.PCS\_NAD\_1927\_ME\_W
     |  
     |  PCS\_NAD\_1927\_MI\_C = PrjCoordSysType.PCS\_NAD\_1927\_MI\_C
     |  
     |  PCS\_NAD\_1927\_MI\_N = PrjCoordSysType.PCS\_NAD\_1927\_MI\_N
     |  
     |  PCS\_NAD\_1927\_MI\_S = PrjCoordSysType.PCS\_NAD\_1927\_MI\_S
     |  
     |  PCS\_NAD\_1927\_MN\_C = PrjCoordSysType.PCS\_NAD\_1927\_MN\_C
     |  
     |  PCS\_NAD\_1927\_MN\_N = PrjCoordSysType.PCS\_NAD\_1927\_MN\_N
     |  
     |  PCS\_NAD\_1927\_MN\_S = PrjCoordSysType.PCS\_NAD\_1927\_MN\_S
     |  
     |  PCS\_NAD\_1927\_MO\_C = PrjCoordSysType.PCS\_NAD\_1927\_MO\_C
     |  
     |  PCS\_NAD\_1927\_MO\_E = PrjCoordSysType.PCS\_NAD\_1927\_MO\_E
     |  
     |  PCS\_NAD\_1927\_MO\_W = PrjCoordSysType.PCS\_NAD\_1927\_MO\_W
     |  
     |  PCS\_NAD\_1927\_MS\_E = PrjCoordSysType.PCS\_NAD\_1927\_MS\_E
     |  
     |  PCS\_NAD\_1927\_MS\_W = PrjCoordSysType.PCS\_NAD\_1927\_MS\_W
     |  
     |  PCS\_NAD\_1927\_MT\_C = PrjCoordSysType.PCS\_NAD\_1927\_MT\_C
     |  
     |  PCS\_NAD\_1927\_MT\_N = PrjCoordSysType.PCS\_NAD\_1927\_MT\_N
     |  
     |  PCS\_NAD\_1927\_MT\_S = PrjCoordSysType.PCS\_NAD\_1927\_MT\_S
     |  
     |  PCS\_NAD\_1927\_NC = PrjCoordSysType.PCS\_NAD\_1927\_NC
     |  
     |  PCS\_NAD\_1927\_ND\_N = PrjCoordSysType.PCS\_NAD\_1927\_ND\_N
     |  
     |  PCS\_NAD\_1927\_ND\_S = PrjCoordSysType.PCS\_NAD\_1927\_ND\_S
     |  
     |  PCS\_NAD\_1927\_NE\_N = PrjCoordSysType.PCS\_NAD\_1927\_NE\_N
     |  
     |  PCS\_NAD\_1927\_NE\_S = PrjCoordSysType.PCS\_NAD\_1927\_NE\_S
     |  
     |  PCS\_NAD\_1927\_NH = PrjCoordSysType.PCS\_NAD\_1927\_NH
     |  
     |  PCS\_NAD\_1927\_NJ = PrjCoordSysType.PCS\_NAD\_1927\_NJ
     |  
     |  PCS\_NAD\_1927\_NM\_C = PrjCoordSysType.PCS\_NAD\_1927\_NM\_C
     |  
     |  PCS\_NAD\_1927\_NM\_E = PrjCoordSysType.PCS\_NAD\_1927\_NM\_E
     |  
     |  PCS\_NAD\_1927\_NM\_W = PrjCoordSysType.PCS\_NAD\_1927\_NM\_W
     |  
     |  PCS\_NAD\_1927\_NV\_C = PrjCoordSysType.PCS\_NAD\_1927\_NV\_C
     |  
     |  PCS\_NAD\_1927\_NV\_E = PrjCoordSysType.PCS\_NAD\_1927\_NV\_E
     |  
     |  PCS\_NAD\_1927\_NV\_W = PrjCoordSysType.PCS\_NAD\_1927\_NV\_W
     |  
     |  PCS\_NAD\_1927\_NY\_C = PrjCoordSysType.PCS\_NAD\_1927\_NY\_C
     |  
     |  PCS\_NAD\_1927\_NY\_E = PrjCoordSysType.PCS\_NAD\_1927\_NY\_E
     |  
     |  PCS\_NAD\_1927\_NY\_LI = PrjCoordSysType.PCS\_NAD\_1927\_NY\_LI
     |  
     |  PCS\_NAD\_1927\_NY\_W = PrjCoordSysType.PCS\_NAD\_1927\_NY\_W
     |  
     |  PCS\_NAD\_1927\_OH\_N = PrjCoordSysType.PCS\_NAD\_1927\_OH\_N
     |  
     |  PCS\_NAD\_1927\_OH\_S = PrjCoordSysType.PCS\_NAD\_1927\_OH\_S
     |  
     |  PCS\_NAD\_1927\_OK\_N = PrjCoordSysType.PCS\_NAD\_1927\_OK\_N
     |  
     |  PCS\_NAD\_1927\_OK\_S = PrjCoordSysType.PCS\_NAD\_1927\_OK\_S
     |  
     |  PCS\_NAD\_1927\_OR\_N = PrjCoordSysType.PCS\_NAD\_1927\_OR\_N
     |  
     |  PCS\_NAD\_1927\_OR\_S = PrjCoordSysType.PCS\_NAD\_1927\_OR\_S
     |  
     |  PCS\_NAD\_1927\_PA\_N = PrjCoordSysType.PCS\_NAD\_1927\_PA\_N
     |  
     |  PCS\_NAD\_1927\_PA\_S = PrjCoordSysType.PCS\_NAD\_1927\_PA\_S
     |  
     |  PCS\_NAD\_1927\_PR = PrjCoordSysType.PCS\_NAD\_1927\_PR
     |  
     |  PCS\_NAD\_1927\_RI = PrjCoordSysType.PCS\_NAD\_1927\_RI
     |  
     |  PCS\_NAD\_1927\_SC\_N = PrjCoordSysType.PCS\_NAD\_1927\_SC\_N
     |  
     |  PCS\_NAD\_1927\_SC\_S = PrjCoordSysType.PCS\_NAD\_1927\_SC\_S
     |  
     |  PCS\_NAD\_1927\_SD\_N = PrjCoordSysType.PCS\_NAD\_1927\_SD\_N
     |  
     |  PCS\_NAD\_1927\_SD\_S = PrjCoordSysType.PCS\_NAD\_1927\_SD\_S
     |  
     |  PCS\_NAD\_1927\_TN = PrjCoordSysType.PCS\_NAD\_1927\_TN
     |  
     |  PCS\_NAD\_1927\_TX\_C = PrjCoordSysType.PCS\_NAD\_1927\_TX\_C
     |  
     |  PCS\_NAD\_1927\_TX\_N = PrjCoordSysType.PCS\_NAD\_1927\_TX\_N
     |  
     |  PCS\_NAD\_1927\_TX\_NC = PrjCoordSysType.PCS\_NAD\_1927\_TX\_NC
     |  
     |  PCS\_NAD\_1927\_TX\_S = PrjCoordSysType.PCS\_NAD\_1927\_TX\_S
     |  
     |  PCS\_NAD\_1927\_TX\_SC = PrjCoordSysType.PCS\_NAD\_1927\_TX\_SC
     |  
     |  PCS\_NAD\_1927\_UTM\_10N = PrjCoordSysType.PCS\_NAD\_1927\_UTM\_10N
     |  
     |  PCS\_NAD\_1927\_UTM\_11N = PrjCoordSysType.PCS\_NAD\_1927\_UTM\_11N
     |  
     |  PCS\_NAD\_1927\_UTM\_12N = PrjCoordSysType.PCS\_NAD\_1927\_UTM\_12N
     |  
     |  PCS\_NAD\_1927\_UTM\_13N = PrjCoordSysType.PCS\_NAD\_1927\_UTM\_13N
     |  
     |  PCS\_NAD\_1927\_UTM\_14N = PrjCoordSysType.PCS\_NAD\_1927\_UTM\_14N
     |  
     |  PCS\_NAD\_1927\_UTM\_15N = PrjCoordSysType.PCS\_NAD\_1927\_UTM\_15N
     |  
     |  PCS\_NAD\_1927\_UTM\_16N = PrjCoordSysType.PCS\_NAD\_1927\_UTM\_16N
     |  
     |  PCS\_NAD\_1927\_UTM\_17N = PrjCoordSysType.PCS\_NAD\_1927\_UTM\_17N
     |  
     |  PCS\_NAD\_1927\_UTM\_18N = PrjCoordSysType.PCS\_NAD\_1927\_UTM\_18N
     |  
     |  PCS\_NAD\_1927\_UTM\_19N = PrjCoordSysType.PCS\_NAD\_1927\_UTM\_19N
     |  
     |  PCS\_NAD\_1927\_UTM\_20N = PrjCoordSysType.PCS\_NAD\_1927\_UTM\_20N
     |  
     |  PCS\_NAD\_1927\_UTM\_21N = PrjCoordSysType.PCS\_NAD\_1927\_UTM\_21N
     |  
     |  PCS\_NAD\_1927\_UTM\_22N = PrjCoordSysType.PCS\_NAD\_1927\_UTM\_22N
     |  
     |  PCS\_NAD\_1927\_UTM\_3N = PrjCoordSysType.PCS\_NAD\_1927\_UTM\_3N
     |  
     |  PCS\_NAD\_1927\_UTM\_4N = PrjCoordSysType.PCS\_NAD\_1927\_UTM\_4N
     |  
     |  PCS\_NAD\_1927\_UTM\_5N = PrjCoordSysType.PCS\_NAD\_1927\_UTM\_5N
     |  
     |  PCS\_NAD\_1927\_UTM\_6N = PrjCoordSysType.PCS\_NAD\_1927\_UTM\_6N
     |  
     |  PCS\_NAD\_1927\_UTM\_7N = PrjCoordSysType.PCS\_NAD\_1927\_UTM\_7N
     |  
     |  PCS\_NAD\_1927\_UTM\_8N = PrjCoordSysType.PCS\_NAD\_1927\_UTM\_8N
     |  
     |  PCS\_NAD\_1927\_UTM\_9N = PrjCoordSysType.PCS\_NAD\_1927\_UTM\_9N
     |  
     |  PCS\_NAD\_1927\_UT\_C = PrjCoordSysType.PCS\_NAD\_1927\_UT\_C
     |  
     |  PCS\_NAD\_1927\_UT\_N = PrjCoordSysType.PCS\_NAD\_1927\_UT\_N
     |  
     |  PCS\_NAD\_1927\_UT\_S = PrjCoordSysType.PCS\_NAD\_1927\_UT\_S
     |  
     |  PCS\_NAD\_1927\_VA\_N = PrjCoordSysType.PCS\_NAD\_1927\_VA\_N
     |  
     |  PCS\_NAD\_1927\_VA\_S = PrjCoordSysType.PCS\_NAD\_1927\_VA\_S
     |  
     |  PCS\_NAD\_1927\_VI = PrjCoordSysType.PCS\_NAD\_1927\_VI
     |  
     |  PCS\_NAD\_1927\_VT = PrjCoordSysType.PCS\_NAD\_1927\_VT
     |  
     |  PCS\_NAD\_1927\_WA\_N = PrjCoordSysType.PCS\_NAD\_1927\_WA\_N
     |  
     |  PCS\_NAD\_1927\_WA\_S = PrjCoordSysType.PCS\_NAD\_1927\_WA\_S
     |  
     |  PCS\_NAD\_1927\_WI\_C = PrjCoordSysType.PCS\_NAD\_1927\_WI\_C
     |  
     |  PCS\_NAD\_1927\_WI\_N = PrjCoordSysType.PCS\_NAD\_1927\_WI\_N
     |  
     |  PCS\_NAD\_1927\_WI\_S = PrjCoordSysType.PCS\_NAD\_1927\_WI\_S
     |  
     |  PCS\_NAD\_1927\_WV\_N = PrjCoordSysType.PCS\_NAD\_1927\_WV\_N
     |  
     |  PCS\_NAD\_1927\_WV\_S = PrjCoordSysType.PCS\_NAD\_1927\_WV\_S
     |  
     |  PCS\_NAD\_1927\_WY\_E = PrjCoordSysType.PCS\_NAD\_1927\_WY\_E
     |  
     |  PCS\_NAD\_1927\_WY\_EC = PrjCoordSysType.PCS\_NAD\_1927\_WY\_EC
     |  
     |  PCS\_NAD\_1927\_WY\_W = PrjCoordSysType.PCS\_NAD\_1927\_WY\_W
     |  
     |  PCS\_NAD\_1927\_WY\_WC = PrjCoordSysType.PCS\_NAD\_1927\_WY\_WC
     |  
     |  PCS\_NAD\_1983\_AK\_1 = PrjCoordSysType.PCS\_NAD\_1983\_AK\_1
     |  
     |  PCS\_NAD\_1983\_AK\_10 = PrjCoordSysType.PCS\_NAD\_1983\_AK\_10
     |  
     |  PCS\_NAD\_1983\_AK\_2 = PrjCoordSysType.PCS\_NAD\_1983\_AK\_2
     |  
     |  PCS\_NAD\_1983\_AK\_3 = PrjCoordSysType.PCS\_NAD\_1983\_AK\_3
     |  
     |  PCS\_NAD\_1983\_AK\_4 = PrjCoordSysType.PCS\_NAD\_1983\_AK\_4
     |  
     |  PCS\_NAD\_1983\_AK\_5 = PrjCoordSysType.PCS\_NAD\_1983\_AK\_5
     |  
     |  PCS\_NAD\_1983\_AK\_6 = PrjCoordSysType.PCS\_NAD\_1983\_AK\_6
     |  
     |  PCS\_NAD\_1983\_AK\_7 = PrjCoordSysType.PCS\_NAD\_1983\_AK\_7
     |  
     |  PCS\_NAD\_1983\_AK\_8 = PrjCoordSysType.PCS\_NAD\_1983\_AK\_8
     |  
     |  PCS\_NAD\_1983\_AK\_9 = PrjCoordSysType.PCS\_NAD\_1983\_AK\_9
     |  
     |  PCS\_NAD\_1983\_AL\_E = PrjCoordSysType.PCS\_NAD\_1983\_AL\_E
     |  
     |  PCS\_NAD\_1983\_AL\_W = PrjCoordSysType.PCS\_NAD\_1983\_AL\_W
     |  
     |  PCS\_NAD\_1983\_AR\_N = PrjCoordSysType.PCS\_NAD\_1983\_AR\_N
     |  
     |  PCS\_NAD\_1983\_AR\_S = PrjCoordSysType.PCS\_NAD\_1983\_AR\_S
     |  
     |  PCS\_NAD\_1983\_AZ\_C = PrjCoordSysType.PCS\_NAD\_1983\_AZ\_C
     |  
     |  PCS\_NAD\_1983\_AZ\_E = PrjCoordSysType.PCS\_NAD\_1983\_AZ\_E
     |  
     |  PCS\_NAD\_1983\_AZ\_W = PrjCoordSysType.PCS\_NAD\_1983\_AZ\_W
     |  
     |  PCS\_NAD\_1983\_CA\_I = PrjCoordSysType.PCS\_NAD\_1983\_CA\_I
     |  
     |  PCS\_NAD\_1983\_CA\_II = PrjCoordSysType.PCS\_NAD\_1983\_CA\_II
     |  
     |  PCS\_NAD\_1983\_CA\_III = PrjCoordSysType.PCS\_NAD\_1983\_CA\_III
     |  
     |  PCS\_NAD\_1983\_CA\_IV = PrjCoordSysType.PCS\_NAD\_1983\_CA\_IV
     |  
     |  PCS\_NAD\_1983\_CA\_V = PrjCoordSysType.PCS\_NAD\_1983\_CA\_V
     |  
     |  PCS\_NAD\_1983\_CA\_VI = PrjCoordSysType.PCS\_NAD\_1983\_CA\_VI
     |  
     |  PCS\_NAD\_1983\_CO\_C = PrjCoordSysType.PCS\_NAD\_1983\_CO\_C
     |  
     |  PCS\_NAD\_1983\_CO\_N = PrjCoordSysType.PCS\_NAD\_1983\_CO\_N
     |  
     |  PCS\_NAD\_1983\_CO\_S = PrjCoordSysType.PCS\_NAD\_1983\_CO\_S
     |  
     |  PCS\_NAD\_1983\_CT = PrjCoordSysType.PCS\_NAD\_1983\_CT
     |  
     |  PCS\_NAD\_1983\_DE = PrjCoordSysType.PCS\_NAD\_1983\_DE
     |  
     |  PCS\_NAD\_1983\_FL\_E = PrjCoordSysType.PCS\_NAD\_1983\_FL\_E
     |  
     |  PCS\_NAD\_1983\_FL\_N = PrjCoordSysType.PCS\_NAD\_1983\_FL\_N
     |  
     |  PCS\_NAD\_1983\_FL\_W = PrjCoordSysType.PCS\_NAD\_1983\_FL\_W
     |  
     |  PCS\_NAD\_1983\_GA\_E = PrjCoordSysType.PCS\_NAD\_1983\_GA\_E
     |  
     |  PCS\_NAD\_1983\_GA\_W = PrjCoordSysType.PCS\_NAD\_1983\_GA\_W
     |  
     |  PCS\_NAD\_1983\_GU = PrjCoordSysType.PCS\_NAD\_1983\_GU
     |  
     |  PCS\_NAD\_1983\_HI\_1 = PrjCoordSysType.PCS\_NAD\_1983\_HI\_1
     |  
     |  PCS\_NAD\_1983\_HI\_2 = PrjCoordSysType.PCS\_NAD\_1983\_HI\_2
     |  
     |  PCS\_NAD\_1983\_HI\_3 = PrjCoordSysType.PCS\_NAD\_1983\_HI\_3
     |  
     |  PCS\_NAD\_1983\_HI\_4 = PrjCoordSysType.PCS\_NAD\_1983\_HI\_4
     |  
     |  PCS\_NAD\_1983\_HI\_5 = PrjCoordSysType.PCS\_NAD\_1983\_HI\_5
     |  
     |  PCS\_NAD\_1983\_IA\_N = PrjCoordSysType.PCS\_NAD\_1983\_IA\_N
     |  
     |  PCS\_NAD\_1983\_IA\_S = PrjCoordSysType.PCS\_NAD\_1983\_IA\_S
     |  
     |  PCS\_NAD\_1983\_ID\_C = PrjCoordSysType.PCS\_NAD\_1983\_ID\_C
     |  
     |  PCS\_NAD\_1983\_ID\_E = PrjCoordSysType.PCS\_NAD\_1983\_ID\_E
     |  
     |  PCS\_NAD\_1983\_ID\_W = PrjCoordSysType.PCS\_NAD\_1983\_ID\_W
     |  
     |  PCS\_NAD\_1983\_IL\_E = PrjCoordSysType.PCS\_NAD\_1983\_IL\_E
     |  
     |  PCS\_NAD\_1983\_IL\_W = PrjCoordSysType.PCS\_NAD\_1983\_IL\_W
     |  
     |  PCS\_NAD\_1983\_IN\_E = PrjCoordSysType.PCS\_NAD\_1983\_IN\_E
     |  
     |  PCS\_NAD\_1983\_IN\_W = PrjCoordSysType.PCS\_NAD\_1983\_IN\_W
     |  
     |  PCS\_NAD\_1983\_KS\_N = PrjCoordSysType.PCS\_NAD\_1983\_KS\_N
     |  
     |  PCS\_NAD\_1983\_KS\_S = PrjCoordSysType.PCS\_NAD\_1983\_KS\_S
     |  
     |  PCS\_NAD\_1983\_KY\_N = PrjCoordSysType.PCS\_NAD\_1983\_KY\_N
     |  
     |  PCS\_NAD\_1983\_KY\_S = PrjCoordSysType.PCS\_NAD\_1983\_KY\_S
     |  
     |  PCS\_NAD\_1983\_LA\_N = PrjCoordSysType.PCS\_NAD\_1983\_LA\_N
     |  
     |  PCS\_NAD\_1983\_LA\_S = PrjCoordSysType.PCS\_NAD\_1983\_LA\_S
     |  
     |  PCS\_NAD\_1983\_MA\_I = PrjCoordSysType.PCS\_NAD\_1983\_MA\_I
     |  
     |  PCS\_NAD\_1983\_MA\_M = PrjCoordSysType.PCS\_NAD\_1983\_MA\_M
     |  
     |  PCS\_NAD\_1983\_MD = PrjCoordSysType.PCS\_NAD\_1983\_MD
     |  
     |  PCS\_NAD\_1983\_ME\_E = PrjCoordSysType.PCS\_NAD\_1983\_ME\_E
     |  
     |  PCS\_NAD\_1983\_ME\_W = PrjCoordSysType.PCS\_NAD\_1983\_ME\_W
     |  
     |  PCS\_NAD\_1983\_MI\_C = PrjCoordSysType.PCS\_NAD\_1983\_MI\_C
     |  
     |  PCS\_NAD\_1983\_MI\_N = PrjCoordSysType.PCS\_NAD\_1983\_MI\_N
     |  
     |  PCS\_NAD\_1983\_MI\_S = PrjCoordSysType.PCS\_NAD\_1983\_MI\_S
     |  
     |  PCS\_NAD\_1983\_MN\_C = PrjCoordSysType.PCS\_NAD\_1983\_MN\_C
     |  
     |  PCS\_NAD\_1983\_MN\_N = PrjCoordSysType.PCS\_NAD\_1983\_MN\_N
     |  
     |  PCS\_NAD\_1983\_MN\_S = PrjCoordSysType.PCS\_NAD\_1983\_MN\_S
     |  
     |  PCS\_NAD\_1983\_MO\_C = PrjCoordSysType.PCS\_NAD\_1983\_MO\_C
     |  
     |  PCS\_NAD\_1983\_MO\_E = PrjCoordSysType.PCS\_NAD\_1983\_MO\_E
     |  
     |  PCS\_NAD\_1983\_MO\_W = PrjCoordSysType.PCS\_NAD\_1983\_MO\_W
     |  
     |  PCS\_NAD\_1983\_MS\_E = PrjCoordSysType.PCS\_NAD\_1983\_MS\_E
     |  
     |  PCS\_NAD\_1983\_MS\_W = PrjCoordSysType.PCS\_NAD\_1983\_MS\_W
     |  
     |  PCS\_NAD\_1983\_MT = PrjCoordSysType.PCS\_NAD\_1983\_MT
     |  
     |  PCS\_NAD\_1983\_NC = PrjCoordSysType.PCS\_NAD\_1983\_NC
     |  
     |  PCS\_NAD\_1983\_ND\_N = PrjCoordSysType.PCS\_NAD\_1983\_ND\_N
     |  
     |  PCS\_NAD\_1983\_ND\_S = PrjCoordSysType.PCS\_NAD\_1983\_ND\_S
     |  
     |  PCS\_NAD\_1983\_NE = PrjCoordSysType.PCS\_NAD\_1983\_NE
     |  
     |  PCS\_NAD\_1983\_NH = PrjCoordSysType.PCS\_NAD\_1983\_NH
     |  
     |  PCS\_NAD\_1983\_NJ = PrjCoordSysType.PCS\_NAD\_1983\_NJ
     |  
     |  PCS\_NAD\_1983\_NM\_C = PrjCoordSysType.PCS\_NAD\_1983\_NM\_C
     |  
     |  PCS\_NAD\_1983\_NM\_E = PrjCoordSysType.PCS\_NAD\_1983\_NM\_E
     |  
     |  PCS\_NAD\_1983\_NM\_W = PrjCoordSysType.PCS\_NAD\_1983\_NM\_W
     |  
     |  PCS\_NAD\_1983\_NV\_C = PrjCoordSysType.PCS\_NAD\_1983\_NV\_C
     |  
     |  PCS\_NAD\_1983\_NV\_E = PrjCoordSysType.PCS\_NAD\_1983\_NV\_E
     |  
     |  PCS\_NAD\_1983\_NV\_W = PrjCoordSysType.PCS\_NAD\_1983\_NV\_W
     |  
     |  PCS\_NAD\_1983\_NY\_C = PrjCoordSysType.PCS\_NAD\_1983\_NY\_C
     |  
     |  PCS\_NAD\_1983\_NY\_E = PrjCoordSysType.PCS\_NAD\_1983\_NY\_E
     |  
     |  PCS\_NAD\_1983\_NY\_LI = PrjCoordSysType.PCS\_NAD\_1983\_NY\_LI
     |  
     |  PCS\_NAD\_1983\_NY\_W = PrjCoordSysType.PCS\_NAD\_1983\_NY\_W
     |  
     |  PCS\_NAD\_1983\_OH\_N = PrjCoordSysType.PCS\_NAD\_1983\_OH\_N
     |  
     |  PCS\_NAD\_1983\_OH\_S = PrjCoordSysType.PCS\_NAD\_1983\_OH\_S
     |  
     |  PCS\_NAD\_1983\_OK\_N = PrjCoordSysType.PCS\_NAD\_1983\_OK\_N
     |  
     |  PCS\_NAD\_1983\_OK\_S = PrjCoordSysType.PCS\_NAD\_1983\_OK\_S
     |  
     |  PCS\_NAD\_1983\_OR\_N = PrjCoordSysType.PCS\_NAD\_1983\_OR\_N
     |  
     |  PCS\_NAD\_1983\_OR\_S = PrjCoordSysType.PCS\_NAD\_1983\_OR\_S
     |  
     |  PCS\_NAD\_1983\_PA\_N = PrjCoordSysType.PCS\_NAD\_1983\_PA\_N
     |  
     |  PCS\_NAD\_1983\_PA\_S = PrjCoordSysType.PCS\_NAD\_1983\_PA\_S
     |  
     |  PCS\_NAD\_1983\_PR\_VI = PrjCoordSysType.PCS\_NAD\_1983\_PR\_VI
     |  
     |  PCS\_NAD\_1983\_RI = PrjCoordSysType.PCS\_NAD\_1983\_RI
     |  
     |  PCS\_NAD\_1983\_SC = PrjCoordSysType.PCS\_NAD\_1983\_SC
     |  
     |  PCS\_NAD\_1983\_SD\_N = PrjCoordSysType.PCS\_NAD\_1983\_SD\_N
     |  
     |  PCS\_NAD\_1983\_SD\_S = PrjCoordSysType.PCS\_NAD\_1983\_SD\_S
     |  
     |  PCS\_NAD\_1983\_TN = PrjCoordSysType.PCS\_NAD\_1983\_TN
     |  
     |  PCS\_NAD\_1983\_TX\_C = PrjCoordSysType.PCS\_NAD\_1983\_TX\_C
     |  
     |  PCS\_NAD\_1983\_TX\_N = PrjCoordSysType.PCS\_NAD\_1983\_TX\_N
     |  
     |  PCS\_NAD\_1983\_TX\_NC = PrjCoordSysType.PCS\_NAD\_1983\_TX\_NC
     |  
     |  PCS\_NAD\_1983\_TX\_S = PrjCoordSysType.PCS\_NAD\_1983\_TX\_S
     |  
     |  PCS\_NAD\_1983\_TX\_SC = PrjCoordSysType.PCS\_NAD\_1983\_TX\_SC
     |  
     |  PCS\_NAD\_1983\_UTM\_10N = PrjCoordSysType.PCS\_NAD\_1983\_UTM\_10N
     |  
     |  PCS\_NAD\_1983\_UTM\_11N = PrjCoordSysType.PCS\_NAD\_1983\_UTM\_11N
     |  
     |  PCS\_NAD\_1983\_UTM\_12N = PrjCoordSysType.PCS\_NAD\_1983\_UTM\_12N
     |  
     |  PCS\_NAD\_1983\_UTM\_13N = PrjCoordSysType.PCS\_NAD\_1983\_UTM\_13N
     |  
     |  PCS\_NAD\_1983\_UTM\_14N = PrjCoordSysType.PCS\_NAD\_1983\_UTM\_14N
     |  
     |  PCS\_NAD\_1983\_UTM\_15N = PrjCoordSysType.PCS\_NAD\_1983\_UTM\_15N
     |  
     |  PCS\_NAD\_1983\_UTM\_16N = PrjCoordSysType.PCS\_NAD\_1983\_UTM\_16N
     |  
     |  PCS\_NAD\_1983\_UTM\_17N = PrjCoordSysType.PCS\_NAD\_1983\_UTM\_17N
     |  
     |  PCS\_NAD\_1983\_UTM\_18N = PrjCoordSysType.PCS\_NAD\_1983\_UTM\_18N
     |  
     |  PCS\_NAD\_1983\_UTM\_19N = PrjCoordSysType.PCS\_NAD\_1983\_UTM\_19N
     |  
     |  PCS\_NAD\_1983\_UTM\_20N = PrjCoordSysType.PCS\_NAD\_1983\_UTM\_20N
     |  
     |  PCS\_NAD\_1983\_UTM\_21N = PrjCoordSysType.PCS\_NAD\_1983\_UTM\_21N
     |  
     |  PCS\_NAD\_1983\_UTM\_22N = PrjCoordSysType.PCS\_NAD\_1983\_UTM\_22N
     |  
     |  PCS\_NAD\_1983\_UTM\_23N = PrjCoordSysType.PCS\_NAD\_1983\_UTM\_23N
     |  
     |  PCS\_NAD\_1983\_UTM\_3N = PrjCoordSysType.PCS\_NAD\_1983\_UTM\_3N
     |  
     |  PCS\_NAD\_1983\_UTM\_4N = PrjCoordSysType.PCS\_NAD\_1983\_UTM\_4N
     |  
     |  PCS\_NAD\_1983\_UTM\_5N = PrjCoordSysType.PCS\_NAD\_1983\_UTM\_5N
     |  
     |  PCS\_NAD\_1983\_UTM\_6N = PrjCoordSysType.PCS\_NAD\_1983\_UTM\_6N
     |  
     |  PCS\_NAD\_1983\_UTM\_7N = PrjCoordSysType.PCS\_NAD\_1983\_UTM\_7N
     |  
     |  PCS\_NAD\_1983\_UTM\_8N = PrjCoordSysType.PCS\_NAD\_1983\_UTM\_8N
     |  
     |  PCS\_NAD\_1983\_UTM\_9N = PrjCoordSysType.PCS\_NAD\_1983\_UTM\_9N
     |  
     |  PCS\_NAD\_1983\_UT\_C = PrjCoordSysType.PCS\_NAD\_1983\_UT\_C
     |  
     |  PCS\_NAD\_1983\_UT\_N = PrjCoordSysType.PCS\_NAD\_1983\_UT\_N
     |  
     |  PCS\_NAD\_1983\_UT\_S = PrjCoordSysType.PCS\_NAD\_1983\_UT\_S
     |  
     |  PCS\_NAD\_1983\_VA\_N = PrjCoordSysType.PCS\_NAD\_1983\_VA\_N
     |  
     |  PCS\_NAD\_1983\_VA\_S = PrjCoordSysType.PCS\_NAD\_1983\_VA\_S
     |  
     |  PCS\_NAD\_1983\_VT = PrjCoordSysType.PCS\_NAD\_1983\_VT
     |  
     |  PCS\_NAD\_1983\_WA\_N = PrjCoordSysType.PCS\_NAD\_1983\_WA\_N
     |  
     |  PCS\_NAD\_1983\_WA\_S = PrjCoordSysType.PCS\_NAD\_1983\_WA\_S
     |  
     |  PCS\_NAD\_1983\_WI\_C = PrjCoordSysType.PCS\_NAD\_1983\_WI\_C
     |  
     |  PCS\_NAD\_1983\_WI\_N = PrjCoordSysType.PCS\_NAD\_1983\_WI\_N
     |  
     |  PCS\_NAD\_1983\_WI\_S = PrjCoordSysType.PCS\_NAD\_1983\_WI\_S
     |  
     |  PCS\_NAD\_1983\_WV\_N = PrjCoordSysType.PCS\_NAD\_1983\_WV\_N
     |  
     |  PCS\_NAD\_1983\_WV\_S = PrjCoordSysType.PCS\_NAD\_1983\_WV\_S
     |  
     |  PCS\_NAD\_1983\_WY\_E = PrjCoordSysType.PCS\_NAD\_1983\_WY\_E
     |  
     |  PCS\_NAD\_1983\_WY\_EC = PrjCoordSysType.PCS\_NAD\_1983\_WY\_EC
     |  
     |  PCS\_NAD\_1983\_WY\_W = PrjCoordSysType.PCS\_NAD\_1983\_WY\_W
     |  
     |  PCS\_NAD\_1983\_WY\_WC = PrjCoordSysType.PCS\_NAD\_1983\_WY\_WC
     |  
     |  PCS\_NAHRWAN\_1967\_UTM\_38N = PrjCoordSysType.PCS\_NAHRWAN\_1967\_UTM\_38N
     |  
     |  PCS\_NAHRWAN\_1967\_UTM\_39N = PrjCoordSysType.PCS\_NAHRWAN\_1967\_UTM\_39N
     |  
     |  PCS\_NAHRWAN\_1967\_UTM\_40N = PrjCoordSysType.PCS\_NAHRWAN\_1967\_UTM\_40N
     |  
     |  PCS\_NAPARIMA\_1972\_UTM\_20N = PrjCoordSysType.PCS\_NAPARIMA\_1972\_UTM\_20N
     |  
     |  PCS\_NGN\_UTM\_38N = PrjCoordSysType.PCS\_NGN\_UTM\_38N
     |  
     |  PCS\_NGN\_UTM\_39N = PrjCoordSysType.PCS\_NGN\_UTM\_39N
     |  
     |  PCS\_NON\_EARTH = PrjCoordSysType.PCS\_NON\_EARTH
     |  
     |  PCS\_NORD\_SAHARA\_UTM\_29N = PrjCoordSysType.PCS\_NORD\_SAHARA\_UTM\_29N
     |  
     |  PCS\_NORD\_SAHARA\_UTM\_30N = PrjCoordSysType.PCS\_NORD\_SAHARA\_UTM\_30N
     |  
     |  PCS\_NORD\_SAHARA\_UTM\_31N = PrjCoordSysType.PCS\_NORD\_SAHARA\_UTM\_31N
     |  
     |  PCS\_NORD\_SAHARA\_UTM\_32N = PrjCoordSysType.PCS\_NORD\_SAHARA\_UTM\_32N
     |  
     |  PCS\_NTF\_CENTRE\_FRANCE = PrjCoordSysType.PCS\_NTF\_CENTRE\_FRANCE
     |  
     |  PCS\_NTF\_CORSE = PrjCoordSysType.PCS\_NTF\_CORSE
     |  
     |  PCS\_NTF\_FRANCE\_I = PrjCoordSysType.PCS\_NTF\_FRANCE\_I
     |  
     |  PCS\_NTF\_FRANCE\_II = PrjCoordSysType.PCS\_NTF\_FRANCE\_II
     |  
     |  PCS\_NTF\_FRANCE\_III = PrjCoordSysType.PCS\_NTF\_FRANCE\_III
     |  
     |  PCS\_NTF\_FRANCE\_IV = PrjCoordSysType.PCS\_NTF\_FRANCE\_IV
     |  
     |  PCS\_NTF\_NORD\_FRANCE = PrjCoordSysType.PCS\_NTF\_NORD\_FRANCE
     |  
     |  PCS\_NTF\_SUD\_FRANCE = PrjCoordSysType.PCS\_NTF\_SUD\_FRANCE
     |  
     |  PCS\_NZGD\_1949\_NORTH\_ISLAND = PrjCoordSysType.PCS\_NZGD\_1949\_NORTH\_ISLAN{\ldots}
     |  
     |  PCS\_NZGD\_1949\_SOUTH\_ISLAND = PrjCoordSysType.PCS\_NZGD\_1949\_SOUTH\_ISLAN{\ldots}
     |  
     |  PCS\_OSGB\_1936\_BRITISH\_GRID = PrjCoordSysType.PCS\_OSGB\_1936\_BRITISH\_GRI{\ldots}
     |  
     |  PCS\_POINTE\_NOIRE\_UTM\_32S = PrjCoordSysType.PCS\_POINTE\_NOIRE\_UTM\_32S
     |  
     |  PCS\_PSAD\_1956\_PERU\_CENTRAL = PrjCoordSysType.PCS\_PSAD\_1956\_PERU\_CENTRA{\ldots}
     |  
     |  PCS\_PSAD\_1956\_PERU\_EAST = PrjCoordSysType.PCS\_PSAD\_1956\_PERU\_EAST
     |  
     |  PCS\_PSAD\_1956\_PERU\_WEST = PrjCoordSysType.PCS\_PSAD\_1956\_PERU\_WEST
     |  
     |  PCS\_PSAD\_1956\_UTM\_17S = PrjCoordSysType.PCS\_PSAD\_1956\_UTM\_17S
     |  
     |  PCS\_PSAD\_1956\_UTM\_18N = PrjCoordSysType.PCS\_PSAD\_1956\_UTM\_18N
     |  
     |  PCS\_PSAD\_1956\_UTM\_18S = PrjCoordSysType.PCS\_PSAD\_1956\_UTM\_18S
     |  
     |  PCS\_PSAD\_1956\_UTM\_19N = PrjCoordSysType.PCS\_PSAD\_1956\_UTM\_19N
     |  
     |  PCS\_PSAD\_1956\_UTM\_19S = PrjCoordSysType.PCS\_PSAD\_1956\_UTM\_19S
     |  
     |  PCS\_PSAD\_1956\_UTM\_20N = PrjCoordSysType.PCS\_PSAD\_1956\_UTM\_20N
     |  
     |  PCS\_PSAD\_1956\_UTM\_20S = PrjCoordSysType.PCS\_PSAD\_1956\_UTM\_20S
     |  
     |  PCS\_PSAD\_1956\_UTM\_21N = PrjCoordSysType.PCS\_PSAD\_1956\_UTM\_21N
     |  
     |  PCS\_PTRA08\_UTM25\_ITRF93 = PrjCoordSysType.PCS\_PTRA08\_UTM25\_ITRF93
     |  
     |  PCS\_PTRA08\_UTM26\_ITRF93 = PrjCoordSysType.PCS\_PTRA08\_UTM26\_ITRF93
     |  
     |  PCS\_PTRA08\_UTM28\_ITRF93 = PrjCoordSysType.PCS\_PTRA08\_UTM28\_ITRF93
     |  
     |  PCS\_PULKOVO\_1942\_GK\_10 = PrjCoordSysType.PCS\_PULKOVO\_1942\_GK\_10
     |  
     |  PCS\_PULKOVO\_1942\_GK\_10N = PrjCoordSysType.PCS\_PULKOVO\_1942\_GK\_10N
     |  
     |  PCS\_PULKOVO\_1942\_GK\_11 = PrjCoordSysType.PCS\_PULKOVO\_1942\_GK\_11
     |  
     |  PCS\_PULKOVO\_1942\_GK\_11N = PrjCoordSysType.PCS\_PULKOVO\_1942\_GK\_11N
     |  
     |  PCS\_PULKOVO\_1942\_GK\_12 = PrjCoordSysType.PCS\_PULKOVO\_1942\_GK\_12
     |  
     |  PCS\_PULKOVO\_1942\_GK\_12N = PrjCoordSysType.PCS\_PULKOVO\_1942\_GK\_12N
     |  
     |  PCS\_PULKOVO\_1942\_GK\_13 = PrjCoordSysType.PCS\_PULKOVO\_1942\_GK\_13
     |  
     |  PCS\_PULKOVO\_1942\_GK\_13N = PrjCoordSysType.PCS\_PULKOVO\_1942\_GK\_13N
     |  
     |  PCS\_PULKOVO\_1942\_GK\_14 = PrjCoordSysType.PCS\_PULKOVO\_1942\_GK\_14
     |  
     |  PCS\_PULKOVO\_1942\_GK\_14N = PrjCoordSysType.PCS\_PULKOVO\_1942\_GK\_14N
     |  
     |  PCS\_PULKOVO\_1942\_GK\_15 = PrjCoordSysType.PCS\_PULKOVO\_1942\_GK\_15
     |  
     |  PCS\_PULKOVO\_1942\_GK\_15N = PrjCoordSysType.PCS\_PULKOVO\_1942\_GK\_15N
     |  
     |  PCS\_PULKOVO\_1942\_GK\_16 = PrjCoordSysType.PCS\_PULKOVO\_1942\_GK\_16
     |  
     |  PCS\_PULKOVO\_1942\_GK\_16N = PrjCoordSysType.PCS\_PULKOVO\_1942\_GK\_16N
     |  
     |  PCS\_PULKOVO\_1942\_GK\_17 = PrjCoordSysType.PCS\_PULKOVO\_1942\_GK\_17
     |  
     |  PCS\_PULKOVO\_1942\_GK\_17N = PrjCoordSysType.PCS\_PULKOVO\_1942\_GK\_17N
     |  
     |  PCS\_PULKOVO\_1942\_GK\_18 = PrjCoordSysType.PCS\_PULKOVO\_1942\_GK\_18
     |  
     |  PCS\_PULKOVO\_1942\_GK\_18N = PrjCoordSysType.PCS\_PULKOVO\_1942\_GK\_18N
     |  
     |  PCS\_PULKOVO\_1942\_GK\_19 = PrjCoordSysType.PCS\_PULKOVO\_1942\_GK\_19
     |  
     |  PCS\_PULKOVO\_1942\_GK\_19N = PrjCoordSysType.PCS\_PULKOVO\_1942\_GK\_19N
     |  
     |  PCS\_PULKOVO\_1942\_GK\_20 = PrjCoordSysType.PCS\_PULKOVO\_1942\_GK\_20
     |  
     |  PCS\_PULKOVO\_1942\_GK\_20N = PrjCoordSysType.PCS\_PULKOVO\_1942\_GK\_20N
     |  
     |  PCS\_PULKOVO\_1942\_GK\_21 = PrjCoordSysType.PCS\_PULKOVO\_1942\_GK\_21
     |  
     |  PCS\_PULKOVO\_1942\_GK\_21N = PrjCoordSysType.PCS\_PULKOVO\_1942\_GK\_21N
     |  
     |  PCS\_PULKOVO\_1942\_GK\_22 = PrjCoordSysType.PCS\_PULKOVO\_1942\_GK\_22
     |  
     |  PCS\_PULKOVO\_1942\_GK\_22N = PrjCoordSysType.PCS\_PULKOVO\_1942\_GK\_22N
     |  
     |  PCS\_PULKOVO\_1942\_GK\_23 = PrjCoordSysType.PCS\_PULKOVO\_1942\_GK\_23
     |  
     |  PCS\_PULKOVO\_1942\_GK\_23N = PrjCoordSysType.PCS\_PULKOVO\_1942\_GK\_23N
     |  
     |  PCS\_PULKOVO\_1942\_GK\_24 = PrjCoordSysType.PCS\_PULKOVO\_1942\_GK\_24
     |  
     |  PCS\_PULKOVO\_1942\_GK\_24N = PrjCoordSysType.PCS\_PULKOVO\_1942\_GK\_24N
     |  
     |  PCS\_PULKOVO\_1942\_GK\_25 = PrjCoordSysType.PCS\_PULKOVO\_1942\_GK\_25
     |  
     |  PCS\_PULKOVO\_1942\_GK\_25N = PrjCoordSysType.PCS\_PULKOVO\_1942\_GK\_25N
     |  
     |  PCS\_PULKOVO\_1942\_GK\_26 = PrjCoordSysType.PCS\_PULKOVO\_1942\_GK\_26
     |  
     |  PCS\_PULKOVO\_1942\_GK\_26N = PrjCoordSysType.PCS\_PULKOVO\_1942\_GK\_26N
     |  
     |  PCS\_PULKOVO\_1942\_GK\_27 = PrjCoordSysType.PCS\_PULKOVO\_1942\_GK\_27
     |  
     |  PCS\_PULKOVO\_1942\_GK\_27N = PrjCoordSysType.PCS\_PULKOVO\_1942\_GK\_27N
     |  
     |  PCS\_PULKOVO\_1942\_GK\_28 = PrjCoordSysType.PCS\_PULKOVO\_1942\_GK\_28
     |  
     |  PCS\_PULKOVO\_1942\_GK\_28N = PrjCoordSysType.PCS\_PULKOVO\_1942\_GK\_28N
     |  
     |  PCS\_PULKOVO\_1942\_GK\_29 = PrjCoordSysType.PCS\_PULKOVO\_1942\_GK\_29
     |  
     |  PCS\_PULKOVO\_1942\_GK\_29N = PrjCoordSysType.PCS\_PULKOVO\_1942\_GK\_29N
     |  
     |  PCS\_PULKOVO\_1942\_GK\_30 = PrjCoordSysType.PCS\_PULKOVO\_1942\_GK\_30
     |  
     |  PCS\_PULKOVO\_1942\_GK\_30N = PrjCoordSysType.PCS\_PULKOVO\_1942\_GK\_30N
     |  
     |  PCS\_PULKOVO\_1942\_GK\_31 = PrjCoordSysType.PCS\_PULKOVO\_1942\_GK\_31
     |  
     |  PCS\_PULKOVO\_1942\_GK\_31N = PrjCoordSysType.PCS\_PULKOVO\_1942\_GK\_31N
     |  
     |  PCS\_PULKOVO\_1942\_GK\_32 = PrjCoordSysType.PCS\_PULKOVO\_1942\_GK\_32
     |  
     |  PCS\_PULKOVO\_1942\_GK\_32N = PrjCoordSysType.PCS\_PULKOVO\_1942\_GK\_32N
     |  
     |  PCS\_PULKOVO\_1942\_GK\_4 = PrjCoordSysType.PCS\_PULKOVO\_1942\_GK\_4
     |  
     |  PCS\_PULKOVO\_1942\_GK\_4N = PrjCoordSysType.PCS\_PULKOVO\_1942\_GK\_4N
     |  
     |  PCS\_PULKOVO\_1942\_GK\_5 = PrjCoordSysType.PCS\_PULKOVO\_1942\_GK\_5
     |  
     |  PCS\_PULKOVO\_1942\_GK\_5N = PrjCoordSysType.PCS\_PULKOVO\_1942\_GK\_5N
     |  
     |  PCS\_PULKOVO\_1942\_GK\_6 = PrjCoordSysType.PCS\_PULKOVO\_1942\_GK\_6
     |  
     |  PCS\_PULKOVO\_1942\_GK\_6N = PrjCoordSysType.PCS\_PULKOVO\_1942\_GK\_6N
     |  
     |  PCS\_PULKOVO\_1942\_GK\_7 = PrjCoordSysType.PCS\_PULKOVO\_1942\_GK\_7
     |  
     |  PCS\_PULKOVO\_1942\_GK\_7N = PrjCoordSysType.PCS\_PULKOVO\_1942\_GK\_7N
     |  
     |  PCS\_PULKOVO\_1942\_GK\_8 = PrjCoordSysType.PCS\_PULKOVO\_1942\_GK\_8
     |  
     |  PCS\_PULKOVO\_1942\_GK\_8N = PrjCoordSysType.PCS\_PULKOVO\_1942\_GK\_8N
     |  
     |  PCS\_PULKOVO\_1942\_GK\_9 = PrjCoordSysType.PCS\_PULKOVO\_1942\_GK\_9
     |  
     |  PCS\_PULKOVO\_1942\_GK\_9N = PrjCoordSysType.PCS\_PULKOVO\_1942\_GK\_9N
     |  
     |  PCS\_PULKOVO\_1995\_GK\_10 = PrjCoordSysType.PCS\_PULKOVO\_1995\_GK\_10
     |  
     |  PCS\_PULKOVO\_1995\_GK\_10N = PrjCoordSysType.PCS\_PULKOVO\_1995\_GK\_10N
     |  
     |  PCS\_PULKOVO\_1995\_GK\_11 = PrjCoordSysType.PCS\_PULKOVO\_1995\_GK\_11
     |  
     |  PCS\_PULKOVO\_1995\_GK\_11N = PrjCoordSysType.PCS\_PULKOVO\_1995\_GK\_11N
     |  
     |  PCS\_PULKOVO\_1995\_GK\_12 = PrjCoordSysType.PCS\_PULKOVO\_1995\_GK\_12
     |  
     |  PCS\_PULKOVO\_1995\_GK\_12N = PrjCoordSysType.PCS\_PULKOVO\_1995\_GK\_12N
     |  
     |  PCS\_PULKOVO\_1995\_GK\_13 = PrjCoordSysType.PCS\_PULKOVO\_1995\_GK\_13
     |  
     |  PCS\_PULKOVO\_1995\_GK\_13N = PrjCoordSysType.PCS\_PULKOVO\_1995\_GK\_13N
     |  
     |  PCS\_PULKOVO\_1995\_GK\_14 = PrjCoordSysType.PCS\_PULKOVO\_1995\_GK\_14
     |  
     |  PCS\_PULKOVO\_1995\_GK\_14N = PrjCoordSysType.PCS\_PULKOVO\_1995\_GK\_14N
     |  
     |  PCS\_PULKOVO\_1995\_GK\_15 = PrjCoordSysType.PCS\_PULKOVO\_1995\_GK\_15
     |  
     |  PCS\_PULKOVO\_1995\_GK\_15N = PrjCoordSysType.PCS\_PULKOVO\_1995\_GK\_15N
     |  
     |  PCS\_PULKOVO\_1995\_GK\_16 = PrjCoordSysType.PCS\_PULKOVO\_1995\_GK\_16
     |  
     |  PCS\_PULKOVO\_1995\_GK\_16N = PrjCoordSysType.PCS\_PULKOVO\_1995\_GK\_16N
     |  
     |  PCS\_PULKOVO\_1995\_GK\_17 = PrjCoordSysType.PCS\_PULKOVO\_1995\_GK\_17
     |  
     |  PCS\_PULKOVO\_1995\_GK\_17N = PrjCoordSysType.PCS\_PULKOVO\_1995\_GK\_17N
     |  
     |  PCS\_PULKOVO\_1995\_GK\_18 = PrjCoordSysType.PCS\_PULKOVO\_1995\_GK\_18
     |  
     |  PCS\_PULKOVO\_1995\_GK\_18N = PrjCoordSysType.PCS\_PULKOVO\_1995\_GK\_18N
     |  
     |  PCS\_PULKOVO\_1995\_GK\_19 = PrjCoordSysType.PCS\_PULKOVO\_1995\_GK\_19
     |  
     |  PCS\_PULKOVO\_1995\_GK\_19N = PrjCoordSysType.PCS\_PULKOVO\_1995\_GK\_19N
     |  
     |  PCS\_PULKOVO\_1995\_GK\_20 = PrjCoordSysType.PCS\_PULKOVO\_1995\_GK\_20
     |  
     |  PCS\_PULKOVO\_1995\_GK\_20N = PrjCoordSysType.PCS\_PULKOVO\_1995\_GK\_20N
     |  
     |  PCS\_PULKOVO\_1995\_GK\_21 = PrjCoordSysType.PCS\_PULKOVO\_1995\_GK\_21
     |  
     |  PCS\_PULKOVO\_1995\_GK\_21N = PrjCoordSysType.PCS\_PULKOVO\_1995\_GK\_21N
     |  
     |  PCS\_PULKOVO\_1995\_GK\_22 = PrjCoordSysType.PCS\_PULKOVO\_1995\_GK\_22
     |  
     |  PCS\_PULKOVO\_1995\_GK\_22N = PrjCoordSysType.PCS\_PULKOVO\_1995\_GK\_22N
     |  
     |  PCS\_PULKOVO\_1995\_GK\_23 = PrjCoordSysType.PCS\_PULKOVO\_1995\_GK\_23
     |  
     |  PCS\_PULKOVO\_1995\_GK\_23N = PrjCoordSysType.PCS\_PULKOVO\_1995\_GK\_23N
     |  
     |  PCS\_PULKOVO\_1995\_GK\_24 = PrjCoordSysType.PCS\_PULKOVO\_1995\_GK\_24
     |  
     |  PCS\_PULKOVO\_1995\_GK\_24N = PrjCoordSysType.PCS\_PULKOVO\_1995\_GK\_24N
     |  
     |  PCS\_PULKOVO\_1995\_GK\_25 = PrjCoordSysType.PCS\_PULKOVO\_1995\_GK\_25
     |  
     |  PCS\_PULKOVO\_1995\_GK\_25N = PrjCoordSysType.PCS\_PULKOVO\_1995\_GK\_25N
     |  
     |  PCS\_PULKOVO\_1995\_GK\_26 = PrjCoordSysType.PCS\_PULKOVO\_1995\_GK\_26
     |  
     |  PCS\_PULKOVO\_1995\_GK\_26N = PrjCoordSysType.PCS\_PULKOVO\_1995\_GK\_26N
     |  
     |  PCS\_PULKOVO\_1995\_GK\_27 = PrjCoordSysType.PCS\_PULKOVO\_1995\_GK\_27
     |  
     |  PCS\_PULKOVO\_1995\_GK\_27N = PrjCoordSysType.PCS\_PULKOVO\_1995\_GK\_27N
     |  
     |  PCS\_PULKOVO\_1995\_GK\_28 = PrjCoordSysType.PCS\_PULKOVO\_1995\_GK\_28
     |  
     |  PCS\_PULKOVO\_1995\_GK\_28N = PrjCoordSysType.PCS\_PULKOVO\_1995\_GK\_28N
     |  
     |  PCS\_PULKOVO\_1995\_GK\_29 = PrjCoordSysType.PCS\_PULKOVO\_1995\_GK\_29
     |  
     |  PCS\_PULKOVO\_1995\_GK\_29N = PrjCoordSysType.PCS\_PULKOVO\_1995\_GK\_29N
     |  
     |  PCS\_PULKOVO\_1995\_GK\_30 = PrjCoordSysType.PCS\_PULKOVO\_1995\_GK\_30
     |  
     |  PCS\_PULKOVO\_1995\_GK\_30N = PrjCoordSysType.PCS\_PULKOVO\_1995\_GK\_30N
     |  
     |  PCS\_PULKOVO\_1995\_GK\_31 = PrjCoordSysType.PCS\_PULKOVO\_1995\_GK\_31
     |  
     |  PCS\_PULKOVO\_1995\_GK\_31N = PrjCoordSysType.PCS\_PULKOVO\_1995\_GK\_31N
     |  
     |  PCS\_PULKOVO\_1995\_GK\_32 = PrjCoordSysType.PCS\_PULKOVO\_1995\_GK\_32
     |  
     |  PCS\_PULKOVO\_1995\_GK\_32N = PrjCoordSysType.PCS\_PULKOVO\_1995\_GK\_32N
     |  
     |  PCS\_PULKOVO\_1995\_GK\_4 = PrjCoordSysType.PCS\_PULKOVO\_1995\_GK\_4
     |  
     |  PCS\_PULKOVO\_1995\_GK\_4N = PrjCoordSysType.PCS\_PULKOVO\_1995\_GK\_4N
     |  
     |  PCS\_PULKOVO\_1995\_GK\_5 = PrjCoordSysType.PCS\_PULKOVO\_1995\_GK\_5
     |  
     |  PCS\_PULKOVO\_1995\_GK\_5N = PrjCoordSysType.PCS\_PULKOVO\_1995\_GK\_5N
     |  
     |  PCS\_PULKOVO\_1995\_GK\_6 = PrjCoordSysType.PCS\_PULKOVO\_1995\_GK\_6
     |  
     |  PCS\_PULKOVO\_1995\_GK\_6N = PrjCoordSysType.PCS\_PULKOVO\_1995\_GK\_6N
     |  
     |  PCS\_PULKOVO\_1995\_GK\_7 = PrjCoordSysType.PCS\_PULKOVO\_1995\_GK\_7
     |  
     |  PCS\_PULKOVO\_1995\_GK\_7N = PrjCoordSysType.PCS\_PULKOVO\_1995\_GK\_7N
     |  
     |  PCS\_PULKOVO\_1995\_GK\_8 = PrjCoordSysType.PCS\_PULKOVO\_1995\_GK\_8
     |  
     |  PCS\_PULKOVO\_1995\_GK\_8N = PrjCoordSysType.PCS\_PULKOVO\_1995\_GK\_8N
     |  
     |  PCS\_PULKOVO\_1995\_GK\_9 = PrjCoordSysType.PCS\_PULKOVO\_1995\_GK\_9
     |  
     |  PCS\_PULKOVO\_1995\_GK\_9N = PrjCoordSysType.PCS\_PULKOVO\_1995\_GK\_9N
     |  
     |  PCS\_QATAR\_GRID = PrjCoordSysType.PCS\_QATAR\_GRID
     |  
     |  PCS\_RT38\_STOCKHOLM\_SWEDISH\_GRID = PrjCoordSysType.PCS\_RT38\_STOCKHOLM\_S{\ldots}
     |  
     |  PCS\_SAD\_1969\_UTM\_17S = PrjCoordSysType.PCS\_SAD\_1969\_UTM\_17S
     |  
     |  PCS\_SAD\_1969\_UTM\_18N = PrjCoordSysType.PCS\_SAD\_1969\_UTM\_18N
     |  
     |  PCS\_SAD\_1969\_UTM\_18S = PrjCoordSysType.PCS\_SAD\_1969\_UTM\_18S
     |  
     |  PCS\_SAD\_1969\_UTM\_19N = PrjCoordSysType.PCS\_SAD\_1969\_UTM\_19N
     |  
     |  PCS\_SAD\_1969\_UTM\_19S = PrjCoordSysType.PCS\_SAD\_1969\_UTM\_19S
     |  
     |  PCS\_SAD\_1969\_UTM\_20N = PrjCoordSysType.PCS\_SAD\_1969\_UTM\_20N
     |  
     |  PCS\_SAD\_1969\_UTM\_20S = PrjCoordSysType.PCS\_SAD\_1969\_UTM\_20S
     |  
     |  PCS\_SAD\_1969\_UTM\_21N = PrjCoordSysType.PCS\_SAD\_1969\_UTM\_21N
     |  
     |  PCS\_SAD\_1969\_UTM\_21S = PrjCoordSysType.PCS\_SAD\_1969\_UTM\_21S
     |  
     |  PCS\_SAD\_1969\_UTM\_22N = PrjCoordSysType.PCS\_SAD\_1969\_UTM\_22N
     |  
     |  PCS\_SAD\_1969\_UTM\_22S = PrjCoordSysType.PCS\_SAD\_1969\_UTM\_22S
     |  
     |  PCS\_SAD\_1969\_UTM\_23S = PrjCoordSysType.PCS\_SAD\_1969\_UTM\_23S
     |  
     |  PCS\_SAD\_1969\_UTM\_24S = PrjCoordSysType.PCS\_SAD\_1969\_UTM\_24S
     |  
     |  PCS\_SAD\_1969\_UTM\_25S = PrjCoordSysType.PCS\_SAD\_1969\_UTM\_25S
     |  
     |  PCS\_SAPPER\_HILL\_UTM\_20S = PrjCoordSysType.PCS\_SAPPER\_HILL\_UTM\_20S
     |  
     |  PCS\_SAPPER\_HILL\_UTM\_21S = PrjCoordSysType.PCS\_SAPPER\_HILL\_UTM\_21S
     |  
     |  PCS\_SCHWARZECK\_UTM\_33S = PrjCoordSysType.PCS\_SCHWARZECK\_UTM\_33S
     |  
     |  PCS\_SPHERE\_BEHRMANN = PrjCoordSysType.PCS\_SPHERE\_BEHRMANN
     |  
     |  PCS\_SPHERE\_BONNE = PrjCoordSysType.PCS\_SPHERE\_BONNE
     |  
     |  PCS\_SPHERE\_CASSINI = PrjCoordSysType.PCS\_SPHERE\_CASSINI
     |  
     |  PCS\_SPHERE\_ECKERT\_I = PrjCoordSysType.PCS\_SPHERE\_ECKERT\_I
     |  
     |  PCS\_SPHERE\_ECKERT\_II = PrjCoordSysType.PCS\_SPHERE\_ECKERT\_II
     |  
     |  PCS\_SPHERE\_ECKERT\_III = PrjCoordSysType.PCS\_SPHERE\_ECKERT\_III
     |  
     |  PCS\_SPHERE\_ECKERT\_IV = PrjCoordSysType.PCS\_SPHERE\_ECKERT\_IV
     |  
     |  PCS\_SPHERE\_ECKERT\_V = PrjCoordSysType.PCS\_SPHERE\_ECKERT\_V
     |  
     |  PCS\_SPHERE\_ECKERT\_VI = PrjCoordSysType.PCS\_SPHERE\_ECKERT\_VI
     |  
     |  PCS\_SPHERE\_EQUIDISTANT\_CONIC = PrjCoordSysType.PCS\_SPHERE\_EQUIDISTANT\_{\ldots}
     |  
     |  PCS\_SPHERE\_EQUIDISTANT\_CYLINDRICAL = PrjCoordSysType.PCS\_SPHERE\_EQUIDI{\ldots}
     |  
     |  PCS\_SPHERE\_GALL\_STEREOGRAPHIC = PrjCoordSysType.PCS\_SPHERE\_GALL\_STEREO{\ldots}
     |  
     |  PCS\_SPHERE\_LOXIMUTHAL = PrjCoordSysType.PCS\_SPHERE\_LOXIMUTHAL
     |  
     |  PCS\_SPHERE\_MERCATOR = PrjCoordSysType.PCS\_SPHERE\_MERCATOR
     |  
     |  PCS\_SPHERE\_MILLER\_CYLINDRICAL = PrjCoordSysType.PCS\_SPHERE\_MILLER\_CYLI{\ldots}
     |  
     |  PCS\_SPHERE\_MOLLWEIDE = PrjCoordSysType.PCS\_SPHERE\_MOLLWEIDE
     |  
     |  PCS\_SPHERE\_PLATE\_CARREE = PrjCoordSysType.PCS\_SPHERE\_PLATE\_CARREE
     |  
     |  PCS\_SPHERE\_POLYCONIC = PrjCoordSysType.PCS\_SPHERE\_POLYCONIC
     |  
     |  PCS\_SPHERE\_QUARTIC\_AUTHALIC = PrjCoordSysType.PCS\_SPHERE\_QUARTIC\_AUTHA{\ldots}
     |  
     |  PCS\_SPHERE\_ROBINSON = PrjCoordSysType.PCS\_SPHERE\_ROBINSON
     |  
     |  PCS\_SPHERE\_SINUSOIDAL = PrjCoordSysType.PCS\_SPHERE\_SINUSOIDAL
     |  
     |  PCS\_SPHERE\_STEREOGRAPHIC = PrjCoordSysType.PCS\_SPHERE\_STEREOGRAPHIC
     |  
     |  PCS\_SPHERE\_TWO\_POINT\_EQUIDISTANT = PrjCoordSysType.PCS\_SPHERE\_TWO\_POIN{\ldots}
     |  
     |  PCS\_SPHERE\_VAN\_DER\_GRINTEN\_I = PrjCoordSysType.PCS\_SPHERE\_VAN\_DER\_GRIN{\ldots}
     |  
     |  PCS\_SPHERE\_WINKEL\_I = PrjCoordSysType.PCS\_SPHERE\_WINKEL\_I
     |  
     |  PCS\_SPHERE\_WINKEL\_II = PrjCoordSysType.PCS\_SPHERE\_WINKEL\_II
     |  
     |  PCS\_SUDAN\_UTM\_35N = PrjCoordSysType.PCS\_SUDAN\_UTM\_35N
     |  
     |  PCS\_SUDAN\_UTM\_36N = PrjCoordSysType.PCS\_SUDAN\_UTM\_36N
     |  
     |  PCS\_TANANARIVE\_UTM\_38S = PrjCoordSysType.PCS\_TANANARIVE\_UTM\_38S
     |  
     |  PCS\_TANANARIVE\_UTM\_39S = PrjCoordSysType.PCS\_TANANARIVE\_UTM\_39S
     |  
     |  PCS\_TC\_1948\_UTM\_39N = PrjCoordSysType.PCS\_TC\_1948\_UTM\_39N
     |  
     |  PCS\_TC\_1948\_UTM\_40N = PrjCoordSysType.PCS\_TC\_1948\_UTM\_40N
     |  
     |  PCS\_TIMBALAI\_1948\_RSO\_BORNEO = PrjCoordSysType.PCS\_TIMBALAI\_1948\_RSO\_B{\ldots}
     |  
     |  PCS\_TIMBALAI\_1948\_UTM\_49N = PrjCoordSysType.PCS\_TIMBALAI\_1948\_UTM\_49N
     |  
     |  PCS\_TIMBALAI\_1948\_UTM\_50N = PrjCoordSysType.PCS\_TIMBALAI\_1948\_UTM\_50N
     |  
     |  PCS\_TM65\_IRISH\_GRID = PrjCoordSysType.PCS\_TM65\_IRISH\_GRID
     |  
     |  PCS\_TOKYO\_PLATE\_ZONE\_I = PrjCoordSysType.PCS\_TOKYO\_PLATE\_ZONE\_I
     |  
     |  PCS\_TOKYO\_PLATE\_ZONE\_II = PrjCoordSysType.PCS\_TOKYO\_PLATE\_ZONE\_II
     |  
     |  PCS\_TOKYO\_PLATE\_ZONE\_III = PrjCoordSysType.PCS\_TOKYO\_PLATE\_ZONE\_III
     |  
     |  PCS\_TOKYO\_PLATE\_ZONE\_IV = PrjCoordSysType.PCS\_TOKYO\_PLATE\_ZONE\_IV
     |  
     |  PCS\_TOKYO\_PLATE\_ZONE\_IX = PrjCoordSysType.PCS\_TOKYO\_PLATE\_ZONE\_IX
     |  
     |  PCS\_TOKYO\_PLATE\_ZONE\_V = PrjCoordSysType.PCS\_TOKYO\_PLATE\_ZONE\_V
     |  
     |  PCS\_TOKYO\_PLATE\_ZONE\_VI = PrjCoordSysType.PCS\_TOKYO\_PLATE\_ZONE\_VI
     |  
     |  PCS\_TOKYO\_PLATE\_ZONE\_VII = PrjCoordSysType.PCS\_TOKYO\_PLATE\_ZONE\_VII
     |  
     |  PCS\_TOKYO\_PLATE\_ZONE\_VIII = PrjCoordSysType.PCS\_TOKYO\_PLATE\_ZONE\_VIII
     |  
     |  PCS\_TOKYO\_PLATE\_ZONE\_X = PrjCoordSysType.PCS\_TOKYO\_PLATE\_ZONE\_X
     |  
     |  PCS\_TOKYO\_PLATE\_ZONE\_XI = PrjCoordSysType.PCS\_TOKYO\_PLATE\_ZONE\_XI
     |  
     |  PCS\_TOKYO\_PLATE\_ZONE\_XII = PrjCoordSysType.PCS\_TOKYO\_PLATE\_ZONE\_XII
     |  
     |  PCS\_TOKYO\_PLATE\_ZONE\_XIII = PrjCoordSysType.PCS\_TOKYO\_PLATE\_ZONE\_XIII
     |  
     |  PCS\_TOKYO\_PLATE\_ZONE\_XIV = PrjCoordSysType.PCS\_TOKYO\_PLATE\_ZONE\_XIV
     |  
     |  PCS\_TOKYO\_PLATE\_ZONE\_XIX = PrjCoordSysType.PCS\_TOKYO\_PLATE\_ZONE\_XIX
     |  
     |  PCS\_TOKYO\_PLATE\_ZONE\_XV = PrjCoordSysType.PCS\_TOKYO\_PLATE\_ZONE\_XV
     |  
     |  PCS\_TOKYO\_PLATE\_ZONE\_XVI = PrjCoordSysType.PCS\_TOKYO\_PLATE\_ZONE\_XVI
     |  
     |  PCS\_TOKYO\_PLATE\_ZONE\_XVII = PrjCoordSysType.PCS\_TOKYO\_PLATE\_ZONE\_XVII
     |  
     |  PCS\_TOKYO\_PLATE\_ZONE\_XVIII = PrjCoordSysType.PCS\_TOKYO\_PLATE\_ZONE\_XVII{\ldots}
     |  
     |  PCS\_TOKYO\_UTM\_51 = PrjCoordSysType.PCS\_TOKYO\_UTM\_51
     |  
     |  PCS\_TOKYO\_UTM\_52 = PrjCoordSysType.PCS\_TOKYO\_UTM\_52
     |  
     |  PCS\_TOKYO\_UTM\_53 = PrjCoordSysType.PCS\_TOKYO\_UTM\_53
     |  
     |  PCS\_TOKYO\_UTM\_54 = PrjCoordSysType.PCS\_TOKYO\_UTM\_54
     |  
     |  PCS\_TOKYO\_UTM\_55 = PrjCoordSysType.PCS\_TOKYO\_UTM\_55
     |  
     |  PCS\_TOKYO\_UTM\_56 = PrjCoordSysType.PCS\_TOKYO\_UTM\_56
     |  
     |  PCS\_USER\_DEFINED = PrjCoordSysType.PCS\_USER\_DEFINED
     |  
     |  PCS\_VOIROL\_N\_ALGERIE\_ANCIENNE = PrjCoordSysType.PCS\_VOIROL\_N\_ALGERIE\_A{\ldots}
     |  
     |  PCS\_VOIROL\_S\_ALGERIE\_ANCIENNE = PrjCoordSysType.PCS\_VOIROL\_S\_ALGERIE\_A{\ldots}
     |  
     |  PCS\_VOIROL\_UNIFIE\_N\_ALGERIE = PrjCoordSysType.PCS\_VOIROL\_UNIFIE\_N\_ALGE{\ldots}
     |  
     |  PCS\_VOIROL\_UNIFIE\_S\_ALGERIE = PrjCoordSysType.PCS\_VOIROL\_UNIFIE\_S\_ALGE{\ldots}
     |  
     |  PCS\_WGS\_1972\_UTM\_10N = PrjCoordSysType.PCS\_WGS\_1972\_UTM\_10N
     |  
     |  PCS\_WGS\_1972\_UTM\_10S = PrjCoordSysType.PCS\_WGS\_1972\_UTM\_10S
     |  
     |  PCS\_WGS\_1972\_UTM\_11N = PrjCoordSysType.PCS\_WGS\_1972\_UTM\_11N
     |  
     |  PCS\_WGS\_1972\_UTM\_11S = PrjCoordSysType.PCS\_WGS\_1972\_UTM\_11S
     |  
     |  PCS\_WGS\_1972\_UTM\_12N = PrjCoordSysType.PCS\_WGS\_1972\_UTM\_12N
     |  
     |  PCS\_WGS\_1972\_UTM\_12S = PrjCoordSysType.PCS\_WGS\_1972\_UTM\_12S
     |  
     |  PCS\_WGS\_1972\_UTM\_13N = PrjCoordSysType.PCS\_WGS\_1972\_UTM\_13N
     |  
     |  PCS\_WGS\_1972\_UTM\_13S = PrjCoordSysType.PCS\_WGS\_1972\_UTM\_13S
     |  
     |  PCS\_WGS\_1972\_UTM\_14N = PrjCoordSysType.PCS\_WGS\_1972\_UTM\_14N
     |  
     |  PCS\_WGS\_1972\_UTM\_14S = PrjCoordSysType.PCS\_WGS\_1972\_UTM\_14S
     |  
     |  PCS\_WGS\_1972\_UTM\_15N = PrjCoordSysType.PCS\_WGS\_1972\_UTM\_15N
     |  
     |  PCS\_WGS\_1972\_UTM\_15S = PrjCoordSysType.PCS\_WGS\_1972\_UTM\_15S
     |  
     |  PCS\_WGS\_1972\_UTM\_16N = PrjCoordSysType.PCS\_WGS\_1972\_UTM\_16N
     |  
     |  PCS\_WGS\_1972\_UTM\_16S = PrjCoordSysType.PCS\_WGS\_1972\_UTM\_16S
     |  
     |  PCS\_WGS\_1972\_UTM\_17N = PrjCoordSysType.PCS\_WGS\_1972\_UTM\_17N
     |  
     |  PCS\_WGS\_1972\_UTM\_17S = PrjCoordSysType.PCS\_WGS\_1972\_UTM\_17S
     |  
     |  PCS\_WGS\_1972\_UTM\_18N = PrjCoordSysType.PCS\_WGS\_1972\_UTM\_18N
     |  
     |  PCS\_WGS\_1972\_UTM\_18S = PrjCoordSysType.PCS\_WGS\_1972\_UTM\_18S
     |  
     |  PCS\_WGS\_1972\_UTM\_19N = PrjCoordSysType.PCS\_WGS\_1972\_UTM\_19N
     |  
     |  PCS\_WGS\_1972\_UTM\_19S = PrjCoordSysType.PCS\_WGS\_1972\_UTM\_19S
     |  
     |  PCS\_WGS\_1972\_UTM\_1N = PrjCoordSysType.PCS\_WGS\_1972\_UTM\_1N
     |  
     |  PCS\_WGS\_1972\_UTM\_1S = PrjCoordSysType.PCS\_WGS\_1972\_UTM\_1S
     |  
     |  PCS\_WGS\_1972\_UTM\_20N = PrjCoordSysType.PCS\_WGS\_1972\_UTM\_20N
     |  
     |  PCS\_WGS\_1972\_UTM\_20S = PrjCoordSysType.PCS\_WGS\_1972\_UTM\_20S
     |  
     |  PCS\_WGS\_1972\_UTM\_21N = PrjCoordSysType.PCS\_WGS\_1972\_UTM\_21N
     |  
     |  PCS\_WGS\_1972\_UTM\_21S = PrjCoordSysType.PCS\_WGS\_1972\_UTM\_21S
     |  
     |  PCS\_WGS\_1972\_UTM\_22N = PrjCoordSysType.PCS\_WGS\_1972\_UTM\_22N
     |  
     |  PCS\_WGS\_1972\_UTM\_22S = PrjCoordSysType.PCS\_WGS\_1972\_UTM\_22S
     |  
     |  PCS\_WGS\_1972\_UTM\_23N = PrjCoordSysType.PCS\_WGS\_1972\_UTM\_23N
     |  
     |  PCS\_WGS\_1972\_UTM\_23S = PrjCoordSysType.PCS\_WGS\_1972\_UTM\_23S
     |  
     |  PCS\_WGS\_1972\_UTM\_24N = PrjCoordSysType.PCS\_WGS\_1972\_UTM\_24N
     |  
     |  PCS\_WGS\_1972\_UTM\_24S = PrjCoordSysType.PCS\_WGS\_1972\_UTM\_24S
     |  
     |  PCS\_WGS\_1972\_UTM\_25N = PrjCoordSysType.PCS\_WGS\_1972\_UTM\_25N
     |  
     |  PCS\_WGS\_1972\_UTM\_25S = PrjCoordSysType.PCS\_WGS\_1972\_UTM\_25S
     |  
     |  PCS\_WGS\_1972\_UTM\_26N = PrjCoordSysType.PCS\_WGS\_1972\_UTM\_26N
     |  
     |  PCS\_WGS\_1972\_UTM\_26S = PrjCoordSysType.PCS\_WGS\_1972\_UTM\_26S
     |  
     |  PCS\_WGS\_1972\_UTM\_27N = PrjCoordSysType.PCS\_WGS\_1972\_UTM\_27N
     |  
     |  PCS\_WGS\_1972\_UTM\_27S = PrjCoordSysType.PCS\_WGS\_1972\_UTM\_27S
     |  
     |  PCS\_WGS\_1972\_UTM\_28N = PrjCoordSysType.PCS\_WGS\_1972\_UTM\_28N
     |  
     |  PCS\_WGS\_1972\_UTM\_28S = PrjCoordSysType.PCS\_WGS\_1972\_UTM\_28S
     |  
     |  PCS\_WGS\_1972\_UTM\_29N = PrjCoordSysType.PCS\_WGS\_1972\_UTM\_29N
     |  
     |  PCS\_WGS\_1972\_UTM\_29S = PrjCoordSysType.PCS\_WGS\_1972\_UTM\_29S
     |  
     |  PCS\_WGS\_1972\_UTM\_2N = PrjCoordSysType.PCS\_WGS\_1972\_UTM\_2N
     |  
     |  PCS\_WGS\_1972\_UTM\_2S = PrjCoordSysType.PCS\_WGS\_1972\_UTM\_2S
     |  
     |  PCS\_WGS\_1972\_UTM\_30N = PrjCoordSysType.PCS\_WGS\_1972\_UTM\_30N
     |  
     |  PCS\_WGS\_1972\_UTM\_30S = PrjCoordSysType.PCS\_WGS\_1972\_UTM\_30S
     |  
     |  PCS\_WGS\_1972\_UTM\_31N = PrjCoordSysType.PCS\_WGS\_1972\_UTM\_31N
     |  
     |  PCS\_WGS\_1972\_UTM\_31S = PrjCoordSysType.PCS\_WGS\_1972\_UTM\_31S
     |  
     |  PCS\_WGS\_1972\_UTM\_32N = PrjCoordSysType.PCS\_WGS\_1972\_UTM\_32N
     |  
     |  PCS\_WGS\_1972\_UTM\_32S = PrjCoordSysType.PCS\_WGS\_1972\_UTM\_32S
     |  
     |  PCS\_WGS\_1972\_UTM\_33N = PrjCoordSysType.PCS\_WGS\_1972\_UTM\_33N
     |  
     |  PCS\_WGS\_1972\_UTM\_33S = PrjCoordSysType.PCS\_WGS\_1972\_UTM\_33S
     |  
     |  PCS\_WGS\_1972\_UTM\_34N = PrjCoordSysType.PCS\_WGS\_1972\_UTM\_34N
     |  
     |  PCS\_WGS\_1972\_UTM\_34S = PrjCoordSysType.PCS\_WGS\_1972\_UTM\_34S
     |  
     |  PCS\_WGS\_1972\_UTM\_35N = PrjCoordSysType.PCS\_WGS\_1972\_UTM\_35N
     |  
     |  PCS\_WGS\_1972\_UTM\_35S = PrjCoordSysType.PCS\_WGS\_1972\_UTM\_35S
     |  
     |  PCS\_WGS\_1972\_UTM\_36N = PrjCoordSysType.PCS\_WGS\_1972\_UTM\_36N
     |  
     |  PCS\_WGS\_1972\_UTM\_36S = PrjCoordSysType.PCS\_WGS\_1972\_UTM\_36S
     |  
     |  PCS\_WGS\_1972\_UTM\_37N = PrjCoordSysType.PCS\_WGS\_1972\_UTM\_37N
     |  
     |  PCS\_WGS\_1972\_UTM\_37S = PrjCoordSysType.PCS\_WGS\_1972\_UTM\_37S
     |  
     |  PCS\_WGS\_1972\_UTM\_38N = PrjCoordSysType.PCS\_WGS\_1972\_UTM\_38N
     |  
     |  PCS\_WGS\_1972\_UTM\_38S = PrjCoordSysType.PCS\_WGS\_1972\_UTM\_38S
     |  
     |  PCS\_WGS\_1972\_UTM\_39N = PrjCoordSysType.PCS\_WGS\_1972\_UTM\_39N
     |  
     |  PCS\_WGS\_1972\_UTM\_39S = PrjCoordSysType.PCS\_WGS\_1972\_UTM\_39S
     |  
     |  PCS\_WGS\_1972\_UTM\_3N = PrjCoordSysType.PCS\_WGS\_1972\_UTM\_3N
     |  
     |  PCS\_WGS\_1972\_UTM\_3S = PrjCoordSysType.PCS\_WGS\_1972\_UTM\_3S
     |  
     |  PCS\_WGS\_1972\_UTM\_40N = PrjCoordSysType.PCS\_WGS\_1972\_UTM\_40N
     |  
     |  PCS\_WGS\_1972\_UTM\_40S = PrjCoordSysType.PCS\_WGS\_1972\_UTM\_40S
     |  
     |  PCS\_WGS\_1972\_UTM\_41N = PrjCoordSysType.PCS\_WGS\_1972\_UTM\_41N
     |  
     |  PCS\_WGS\_1972\_UTM\_41S = PrjCoordSysType.PCS\_WGS\_1972\_UTM\_41S
     |  
     |  PCS\_WGS\_1972\_UTM\_42N = PrjCoordSysType.PCS\_WGS\_1972\_UTM\_42N
     |  
     |  PCS\_WGS\_1972\_UTM\_42S = PrjCoordSysType.PCS\_WGS\_1972\_UTM\_42S
     |  
     |  PCS\_WGS\_1972\_UTM\_43N = PrjCoordSysType.PCS\_WGS\_1972\_UTM\_43N
     |  
     |  PCS\_WGS\_1972\_UTM\_43S = PrjCoordSysType.PCS\_WGS\_1972\_UTM\_43S
     |  
     |  PCS\_WGS\_1972\_UTM\_44N = PrjCoordSysType.PCS\_WGS\_1972\_UTM\_44N
     |  
     |  PCS\_WGS\_1972\_UTM\_44S = PrjCoordSysType.PCS\_WGS\_1972\_UTM\_44S
     |  
     |  PCS\_WGS\_1972\_UTM\_45N = PrjCoordSysType.PCS\_WGS\_1972\_UTM\_45N
     |  
     |  PCS\_WGS\_1972\_UTM\_45S = PrjCoordSysType.PCS\_WGS\_1972\_UTM\_45S
     |  
     |  PCS\_WGS\_1972\_UTM\_46N = PrjCoordSysType.PCS\_WGS\_1972\_UTM\_46N
     |  
     |  PCS\_WGS\_1972\_UTM\_46S = PrjCoordSysType.PCS\_WGS\_1972\_UTM\_46S
     |  
     |  PCS\_WGS\_1972\_UTM\_47N = PrjCoordSysType.PCS\_WGS\_1972\_UTM\_47N
     |  
     |  PCS\_WGS\_1972\_UTM\_47S = PrjCoordSysType.PCS\_WGS\_1972\_UTM\_47S
     |  
     |  PCS\_WGS\_1972\_UTM\_48N = PrjCoordSysType.PCS\_WGS\_1972\_UTM\_48N
     |  
     |  PCS\_WGS\_1972\_UTM\_48S = PrjCoordSysType.PCS\_WGS\_1972\_UTM\_48S
     |  
     |  PCS\_WGS\_1972\_UTM\_49N = PrjCoordSysType.PCS\_WGS\_1972\_UTM\_49N
     |  
     |  PCS\_WGS\_1972\_UTM\_49S = PrjCoordSysType.PCS\_WGS\_1972\_UTM\_49S
     |  
     |  PCS\_WGS\_1972\_UTM\_4N = PrjCoordSysType.PCS\_WGS\_1972\_UTM\_4N
     |  
     |  PCS\_WGS\_1972\_UTM\_4S = PrjCoordSysType.PCS\_WGS\_1972\_UTM\_4S
     |  
     |  PCS\_WGS\_1972\_UTM\_50N = PrjCoordSysType.PCS\_WGS\_1972\_UTM\_50N
     |  
     |  PCS\_WGS\_1972\_UTM\_50S = PrjCoordSysType.PCS\_WGS\_1972\_UTM\_50S
     |  
     |  PCS\_WGS\_1972\_UTM\_51N = PrjCoordSysType.PCS\_WGS\_1972\_UTM\_51N
     |  
     |  PCS\_WGS\_1972\_UTM\_51S = PrjCoordSysType.PCS\_WGS\_1972\_UTM\_51S
     |  
     |  PCS\_WGS\_1972\_UTM\_52N = PrjCoordSysType.PCS\_WGS\_1972\_UTM\_52N
     |  
     |  PCS\_WGS\_1972\_UTM\_52S = PrjCoordSysType.PCS\_WGS\_1972\_UTM\_52S
     |  
     |  PCS\_WGS\_1972\_UTM\_53N = PrjCoordSysType.PCS\_WGS\_1972\_UTM\_53N
     |  
     |  PCS\_WGS\_1972\_UTM\_53S = PrjCoordSysType.PCS\_WGS\_1972\_UTM\_53S
     |  
     |  PCS\_WGS\_1972\_UTM\_54N = PrjCoordSysType.PCS\_WGS\_1972\_UTM\_54N
     |  
     |  PCS\_WGS\_1972\_UTM\_54S = PrjCoordSysType.PCS\_WGS\_1972\_UTM\_54S
     |  
     |  PCS\_WGS\_1972\_UTM\_55N = PrjCoordSysType.PCS\_WGS\_1972\_UTM\_55N
     |  
     |  PCS\_WGS\_1972\_UTM\_55S = PrjCoordSysType.PCS\_WGS\_1972\_UTM\_55S
     |  
     |  PCS\_WGS\_1972\_UTM\_56N = PrjCoordSysType.PCS\_WGS\_1972\_UTM\_56N
     |  
     |  PCS\_WGS\_1972\_UTM\_56S = PrjCoordSysType.PCS\_WGS\_1972\_UTM\_56S
     |  
     |  PCS\_WGS\_1972\_UTM\_57N = PrjCoordSysType.PCS\_WGS\_1972\_UTM\_57N
     |  
     |  PCS\_WGS\_1972\_UTM\_57S = PrjCoordSysType.PCS\_WGS\_1972\_UTM\_57S
     |  
     |  PCS\_WGS\_1972\_UTM\_58N = PrjCoordSysType.PCS\_WGS\_1972\_UTM\_58N
     |  
     |  PCS\_WGS\_1972\_UTM\_58S = PrjCoordSysType.PCS\_WGS\_1972\_UTM\_58S
     |  
     |  PCS\_WGS\_1972\_UTM\_59N = PrjCoordSysType.PCS\_WGS\_1972\_UTM\_59N
     |  
     |  PCS\_WGS\_1972\_UTM\_59S = PrjCoordSysType.PCS\_WGS\_1972\_UTM\_59S
     |  
     |  PCS\_WGS\_1972\_UTM\_5N = PrjCoordSysType.PCS\_WGS\_1972\_UTM\_5N
     |  
     |  PCS\_WGS\_1972\_UTM\_5S = PrjCoordSysType.PCS\_WGS\_1972\_UTM\_5S
     |  
     |  PCS\_WGS\_1972\_UTM\_60N = PrjCoordSysType.PCS\_WGS\_1972\_UTM\_60N
     |  
     |  PCS\_WGS\_1972\_UTM\_60S = PrjCoordSysType.PCS\_WGS\_1972\_UTM\_60S
     |  
     |  PCS\_WGS\_1972\_UTM\_6N = PrjCoordSysType.PCS\_WGS\_1972\_UTM\_6N
     |  
     |  PCS\_WGS\_1972\_UTM\_6S = PrjCoordSysType.PCS\_WGS\_1972\_UTM\_6S
     |  
     |  PCS\_WGS\_1972\_UTM\_7N = PrjCoordSysType.PCS\_WGS\_1972\_UTM\_7N
     |  
     |  PCS\_WGS\_1972\_UTM\_7S = PrjCoordSysType.PCS\_WGS\_1972\_UTM\_7S
     |  
     |  PCS\_WGS\_1972\_UTM\_8N = PrjCoordSysType.PCS\_WGS\_1972\_UTM\_8N
     |  
     |  PCS\_WGS\_1972\_UTM\_8S = PrjCoordSysType.PCS\_WGS\_1972\_UTM\_8S
     |  
     |  PCS\_WGS\_1972\_UTM\_9N = PrjCoordSysType.PCS\_WGS\_1972\_UTM\_9N
     |  
     |  PCS\_WGS\_1972\_UTM\_9S = PrjCoordSysType.PCS\_WGS\_1972\_UTM\_9S
     |  
     |  PCS\_WGS\_1984\_UTM\_10N = PrjCoordSysType.PCS\_WGS\_1984\_UTM\_10N
     |  
     |  PCS\_WGS\_1984\_UTM\_10S = PrjCoordSysType.PCS\_WGS\_1984\_UTM\_10S
     |  
     |  PCS\_WGS\_1984\_UTM\_11N = PrjCoordSysType.PCS\_WGS\_1984\_UTM\_11N
     |  
     |  PCS\_WGS\_1984\_UTM\_11S = PrjCoordSysType.PCS\_WGS\_1984\_UTM\_11S
     |  
     |  PCS\_WGS\_1984\_UTM\_12N = PrjCoordSysType.PCS\_WGS\_1984\_UTM\_12N
     |  
     |  PCS\_WGS\_1984\_UTM\_12S = PrjCoordSysType.PCS\_WGS\_1984\_UTM\_12S
     |  
     |  PCS\_WGS\_1984\_UTM\_13N = PrjCoordSysType.PCS\_WGS\_1984\_UTM\_13N
     |  
     |  PCS\_WGS\_1984\_UTM\_13S = PrjCoordSysType.PCS\_WGS\_1984\_UTM\_13S
     |  
     |  PCS\_WGS\_1984\_UTM\_14N = PrjCoordSysType.PCS\_WGS\_1984\_UTM\_14N
     |  
     |  PCS\_WGS\_1984\_UTM\_14S = PrjCoordSysType.PCS\_WGS\_1984\_UTM\_14S
     |  
     |  PCS\_WGS\_1984\_UTM\_15N = PrjCoordSysType.PCS\_WGS\_1984\_UTM\_15N
     |  
     |  PCS\_WGS\_1984\_UTM\_15S = PrjCoordSysType.PCS\_WGS\_1984\_UTM\_15S
     |  
     |  PCS\_WGS\_1984\_UTM\_16N = PrjCoordSysType.PCS\_WGS\_1984\_UTM\_16N
     |  
     |  PCS\_WGS\_1984\_UTM\_16S = PrjCoordSysType.PCS\_WGS\_1984\_UTM\_16S
     |  
     |  PCS\_WGS\_1984\_UTM\_17N = PrjCoordSysType.PCS\_WGS\_1984\_UTM\_17N
     |  
     |  PCS\_WGS\_1984\_UTM\_17S = PrjCoordSysType.PCS\_WGS\_1984\_UTM\_17S
     |  
     |  PCS\_WGS\_1984\_UTM\_18N = PrjCoordSysType.PCS\_WGS\_1984\_UTM\_18N
     |  
     |  PCS\_WGS\_1984\_UTM\_18S = PrjCoordSysType.PCS\_WGS\_1984\_UTM\_18S
     |  
     |  PCS\_WGS\_1984\_UTM\_19N = PrjCoordSysType.PCS\_WGS\_1984\_UTM\_19N
     |  
     |  PCS\_WGS\_1984\_UTM\_19S = PrjCoordSysType.PCS\_WGS\_1984\_UTM\_19S
     |  
     |  PCS\_WGS\_1984\_UTM\_1N = PrjCoordSysType.PCS\_WGS\_1984\_UTM\_1N
     |  
     |  PCS\_WGS\_1984\_UTM\_1S = PrjCoordSysType.PCS\_WGS\_1984\_UTM\_1S
     |  
     |  PCS\_WGS\_1984\_UTM\_20N = PrjCoordSysType.PCS\_WGS\_1984\_UTM\_20N
     |  
     |  PCS\_WGS\_1984\_UTM\_20S = PrjCoordSysType.PCS\_WGS\_1984\_UTM\_20S
     |  
     |  PCS\_WGS\_1984\_UTM\_21N = PrjCoordSysType.PCS\_WGS\_1984\_UTM\_21N
     |  
     |  PCS\_WGS\_1984\_UTM\_21S = PrjCoordSysType.PCS\_WGS\_1984\_UTM\_21S
     |  
     |  PCS\_WGS\_1984\_UTM\_22N = PrjCoordSysType.PCS\_WGS\_1984\_UTM\_22N
     |  
     |  PCS\_WGS\_1984\_UTM\_22S = PrjCoordSysType.PCS\_WGS\_1984\_UTM\_22S
     |  
     |  PCS\_WGS\_1984\_UTM\_23N = PrjCoordSysType.PCS\_WGS\_1984\_UTM\_23N
     |  
     |  PCS\_WGS\_1984\_UTM\_23S = PrjCoordSysType.PCS\_WGS\_1984\_UTM\_23S
     |  
     |  PCS\_WGS\_1984\_UTM\_24N = PrjCoordSysType.PCS\_WGS\_1984\_UTM\_24N
     |  
     |  PCS\_WGS\_1984\_UTM\_24S = PrjCoordSysType.PCS\_WGS\_1984\_UTM\_24S
     |  
     |  PCS\_WGS\_1984\_UTM\_25N = PrjCoordSysType.PCS\_WGS\_1984\_UTM\_25N
     |  
     |  PCS\_WGS\_1984\_UTM\_25S = PrjCoordSysType.PCS\_WGS\_1984\_UTM\_25S
     |  
     |  PCS\_WGS\_1984\_UTM\_26N = PrjCoordSysType.PCS\_WGS\_1984\_UTM\_26N
     |  
     |  PCS\_WGS\_1984\_UTM\_26S = PrjCoordSysType.PCS\_WGS\_1984\_UTM\_26S
     |  
     |  PCS\_WGS\_1984\_UTM\_27N = PrjCoordSysType.PCS\_WGS\_1984\_UTM\_27N
     |  
     |  PCS\_WGS\_1984\_UTM\_27S = PrjCoordSysType.PCS\_WGS\_1984\_UTM\_27S
     |  
     |  PCS\_WGS\_1984\_UTM\_28N = PrjCoordSysType.PCS\_WGS\_1984\_UTM\_28N
     |  
     |  PCS\_WGS\_1984\_UTM\_28S = PrjCoordSysType.PCS\_WGS\_1984\_UTM\_28S
     |  
     |  PCS\_WGS\_1984\_UTM\_29N = PrjCoordSysType.PCS\_WGS\_1984\_UTM\_29N
     |  
     |  PCS\_WGS\_1984\_UTM\_29S = PrjCoordSysType.PCS\_WGS\_1984\_UTM\_29S
     |  
     |  PCS\_WGS\_1984\_UTM\_2N = PrjCoordSysType.PCS\_WGS\_1984\_UTM\_2N
     |  
     |  PCS\_WGS\_1984\_UTM\_2S = PrjCoordSysType.PCS\_WGS\_1984\_UTM\_2S
     |  
     |  PCS\_WGS\_1984\_UTM\_30N = PrjCoordSysType.PCS\_WGS\_1984\_UTM\_30N
     |  
     |  PCS\_WGS\_1984\_UTM\_30S = PrjCoordSysType.PCS\_WGS\_1984\_UTM\_30S
     |  
     |  PCS\_WGS\_1984\_UTM\_31N = PrjCoordSysType.PCS\_WGS\_1984\_UTM\_31N
     |  
     |  PCS\_WGS\_1984\_UTM\_31S = PrjCoordSysType.PCS\_WGS\_1984\_UTM\_31S
     |  
     |  PCS\_WGS\_1984\_UTM\_32N = PrjCoordSysType.PCS\_WGS\_1984\_UTM\_32N
     |  
     |  PCS\_WGS\_1984\_UTM\_32S = PrjCoordSysType.PCS\_WGS\_1984\_UTM\_32S
     |  
     |  PCS\_WGS\_1984\_UTM\_33N = PrjCoordSysType.PCS\_WGS\_1984\_UTM\_33N
     |  
     |  PCS\_WGS\_1984\_UTM\_33S = PrjCoordSysType.PCS\_WGS\_1984\_UTM\_33S
     |  
     |  PCS\_WGS\_1984\_UTM\_34N = PrjCoordSysType.PCS\_WGS\_1984\_UTM\_34N
     |  
     |  PCS\_WGS\_1984\_UTM\_34S = PrjCoordSysType.PCS\_WGS\_1984\_UTM\_34S
     |  
     |  PCS\_WGS\_1984\_UTM\_35N = PrjCoordSysType.PCS\_WGS\_1984\_UTM\_35N
     |  
     |  PCS\_WGS\_1984\_UTM\_35S = PrjCoordSysType.PCS\_WGS\_1984\_UTM\_35S
     |  
     |  PCS\_WGS\_1984\_UTM\_36N = PrjCoordSysType.PCS\_WGS\_1984\_UTM\_36N
     |  
     |  PCS\_WGS\_1984\_UTM\_36S = PrjCoordSysType.PCS\_WGS\_1984\_UTM\_36S
     |  
     |  PCS\_WGS\_1984\_UTM\_37N = PrjCoordSysType.PCS\_WGS\_1984\_UTM\_37N
     |  
     |  PCS\_WGS\_1984\_UTM\_37S = PrjCoordSysType.PCS\_WGS\_1984\_UTM\_37S
     |  
     |  PCS\_WGS\_1984\_UTM\_38N = PrjCoordSysType.PCS\_WGS\_1984\_UTM\_38N
     |  
     |  PCS\_WGS\_1984\_UTM\_38S = PrjCoordSysType.PCS\_WGS\_1984\_UTM\_38S
     |  
     |  PCS\_WGS\_1984\_UTM\_39N = PrjCoordSysType.PCS\_WGS\_1984\_UTM\_39N
     |  
     |  PCS\_WGS\_1984\_UTM\_39S = PrjCoordSysType.PCS\_WGS\_1984\_UTM\_39S
     |  
     |  PCS\_WGS\_1984\_UTM\_3N = PrjCoordSysType.PCS\_WGS\_1984\_UTM\_3N
     |  
     |  PCS\_WGS\_1984\_UTM\_3S = PrjCoordSysType.PCS\_WGS\_1984\_UTM\_3S
     |  
     |  PCS\_WGS\_1984\_UTM\_40N = PrjCoordSysType.PCS\_WGS\_1984\_UTM\_40N
     |  
     |  PCS\_WGS\_1984\_UTM\_40S = PrjCoordSysType.PCS\_WGS\_1984\_UTM\_40S
     |  
     |  PCS\_WGS\_1984\_UTM\_41N = PrjCoordSysType.PCS\_WGS\_1984\_UTM\_41N
     |  
     |  PCS\_WGS\_1984\_UTM\_41S = PrjCoordSysType.PCS\_WGS\_1984\_UTM\_41S
     |  
     |  PCS\_WGS\_1984\_UTM\_42N = PrjCoordSysType.PCS\_WGS\_1984\_UTM\_42N
     |  
     |  PCS\_WGS\_1984\_UTM\_42S = PrjCoordSysType.PCS\_WGS\_1984\_UTM\_42S
     |  
     |  PCS\_WGS\_1984\_UTM\_43N = PrjCoordSysType.PCS\_WGS\_1984\_UTM\_43N
     |  
     |  PCS\_WGS\_1984\_UTM\_43S = PrjCoordSysType.PCS\_WGS\_1984\_UTM\_43S
     |  
     |  PCS\_WGS\_1984\_UTM\_44N = PrjCoordSysType.PCS\_WGS\_1984\_UTM\_44N
     |  
     |  PCS\_WGS\_1984\_UTM\_44S = PrjCoordSysType.PCS\_WGS\_1984\_UTM\_44S
     |  
     |  PCS\_WGS\_1984\_UTM\_45N = PrjCoordSysType.PCS\_WGS\_1984\_UTM\_45N
     |  
     |  PCS\_WGS\_1984\_UTM\_45S = PrjCoordSysType.PCS\_WGS\_1984\_UTM\_45S
     |  
     |  PCS\_WGS\_1984\_UTM\_46N = PrjCoordSysType.PCS\_WGS\_1984\_UTM\_46N
     |  
     |  PCS\_WGS\_1984\_UTM\_46S = PrjCoordSysType.PCS\_WGS\_1984\_UTM\_46S
     |  
     |  PCS\_WGS\_1984\_UTM\_47N = PrjCoordSysType.PCS\_WGS\_1984\_UTM\_47N
     |  
     |  PCS\_WGS\_1984\_UTM\_47S = PrjCoordSysType.PCS\_WGS\_1984\_UTM\_47S
     |  
     |  PCS\_WGS\_1984\_UTM\_48N = PrjCoordSysType.PCS\_WGS\_1984\_UTM\_48N
     |  
     |  PCS\_WGS\_1984\_UTM\_48S = PrjCoordSysType.PCS\_WGS\_1984\_UTM\_48S
     |  
     |  PCS\_WGS\_1984\_UTM\_49N = PrjCoordSysType.PCS\_WGS\_1984\_UTM\_49N
     |  
     |  PCS\_WGS\_1984\_UTM\_49S = PrjCoordSysType.PCS\_WGS\_1984\_UTM\_49S
     |  
     |  PCS\_WGS\_1984\_UTM\_4N = PrjCoordSysType.PCS\_WGS\_1984\_UTM\_4N
     |  
     |  PCS\_WGS\_1984\_UTM\_4S = PrjCoordSysType.PCS\_WGS\_1984\_UTM\_4S
     |  
     |  PCS\_WGS\_1984\_UTM\_50N = PrjCoordSysType.PCS\_WGS\_1984\_UTM\_50N
     |  
     |  PCS\_WGS\_1984\_UTM\_50S = PrjCoordSysType.PCS\_WGS\_1984\_UTM\_50S
     |  
     |  PCS\_WGS\_1984\_UTM\_51N = PrjCoordSysType.PCS\_WGS\_1984\_UTM\_51N
     |  
     |  PCS\_WGS\_1984\_UTM\_51S = PrjCoordSysType.PCS\_WGS\_1984\_UTM\_51S
     |  
     |  PCS\_WGS\_1984\_UTM\_52N = PrjCoordSysType.PCS\_WGS\_1984\_UTM\_52N
     |  
     |  PCS\_WGS\_1984\_UTM\_52S = PrjCoordSysType.PCS\_WGS\_1984\_UTM\_52S
     |  
     |  PCS\_WGS\_1984\_UTM\_53N = PrjCoordSysType.PCS\_WGS\_1984\_UTM\_53N
     |  
     |  PCS\_WGS\_1984\_UTM\_53S = PrjCoordSysType.PCS\_WGS\_1984\_UTM\_53S
     |  
     |  PCS\_WGS\_1984\_UTM\_54N = PrjCoordSysType.PCS\_WGS\_1984\_UTM\_54N
     |  
     |  PCS\_WGS\_1984\_UTM\_54S = PrjCoordSysType.PCS\_WGS\_1984\_UTM\_54S
     |  
     |  PCS\_WGS\_1984\_UTM\_55N = PrjCoordSysType.PCS\_WGS\_1984\_UTM\_55N
     |  
     |  PCS\_WGS\_1984\_UTM\_55S = PrjCoordSysType.PCS\_WGS\_1984\_UTM\_55S
     |  
     |  PCS\_WGS\_1984\_UTM\_56N = PrjCoordSysType.PCS\_WGS\_1984\_UTM\_56N
     |  
     |  PCS\_WGS\_1984\_UTM\_56S = PrjCoordSysType.PCS\_WGS\_1984\_UTM\_56S
     |  
     |  PCS\_WGS\_1984\_UTM\_57N = PrjCoordSysType.PCS\_WGS\_1984\_UTM\_57N
     |  
     |  PCS\_WGS\_1984\_UTM\_57S = PrjCoordSysType.PCS\_WGS\_1984\_UTM\_57S
     |  
     |  PCS\_WGS\_1984\_UTM\_58N = PrjCoordSysType.PCS\_WGS\_1984\_UTM\_58N
     |  
     |  PCS\_WGS\_1984\_UTM\_58S = PrjCoordSysType.PCS\_WGS\_1984\_UTM\_58S
     |  
     |  PCS\_WGS\_1984\_UTM\_59N = PrjCoordSysType.PCS\_WGS\_1984\_UTM\_59N
     |  
     |  PCS\_WGS\_1984\_UTM\_59S = PrjCoordSysType.PCS\_WGS\_1984\_UTM\_59S
     |  
     |  PCS\_WGS\_1984\_UTM\_5N = PrjCoordSysType.PCS\_WGS\_1984\_UTM\_5N
     |  
     |  PCS\_WGS\_1984\_UTM\_5S = PrjCoordSysType.PCS\_WGS\_1984\_UTM\_5S
     |  
     |  PCS\_WGS\_1984\_UTM\_60N = PrjCoordSysType.PCS\_WGS\_1984\_UTM\_60N
     |  
     |  PCS\_WGS\_1984\_UTM\_60S = PrjCoordSysType.PCS\_WGS\_1984\_UTM\_60S
     |  
     |  PCS\_WGS\_1984\_UTM\_6N = PrjCoordSysType.PCS\_WGS\_1984\_UTM\_6N
     |  
     |  PCS\_WGS\_1984\_UTM\_6S = PrjCoordSysType.PCS\_WGS\_1984\_UTM\_6S
     |  
     |  PCS\_WGS\_1984\_UTM\_7N = PrjCoordSysType.PCS\_WGS\_1984\_UTM\_7N
     |  
     |  PCS\_WGS\_1984\_UTM\_7S = PrjCoordSysType.PCS\_WGS\_1984\_UTM\_7S
     |  
     |  PCS\_WGS\_1984\_UTM\_8N = PrjCoordSysType.PCS\_WGS\_1984\_UTM\_8N
     |  
     |  PCS\_WGS\_1984\_UTM\_8S = PrjCoordSysType.PCS\_WGS\_1984\_UTM\_8S
     |  
     |  PCS\_WGS\_1984\_UTM\_9N = PrjCoordSysType.PCS\_WGS\_1984\_UTM\_9N
     |  
     |  PCS\_WGS\_1984\_UTM\_9S = PrjCoordSysType.PCS\_WGS\_1984\_UTM\_9S
     |  
     |  PCS\_WGS\_1984\_WORLD\_MERCATOR = PrjCoordSysType.PCS\_WGS\_1984\_WORLD\_MERCA{\ldots}
     |  
     |  PCS\_WORLD\_BEHRMANN = PrjCoordSysType.PCS\_WORLD\_BEHRMANN
     |  
     |  PCS\_WORLD\_BONNE = PrjCoordSysType.PCS\_WORLD\_BONNE
     |  
     |  PCS\_WORLD\_CASSINI = PrjCoordSysType.PCS\_WORLD\_CASSINI
     |  
     |  PCS\_WORLD\_ECKERT\_I = PrjCoordSysType.PCS\_WORLD\_ECKERT\_I
     |  
     |  PCS\_WORLD\_ECKERT\_II = PrjCoordSysType.PCS\_WORLD\_ECKERT\_II
     |  
     |  PCS\_WORLD\_ECKERT\_III = PrjCoordSysType.PCS\_WORLD\_ECKERT\_III
     |  
     |  PCS\_WORLD\_ECKERT\_IV = PrjCoordSysType.PCS\_WORLD\_ECKERT\_IV
     |  
     |  PCS\_WORLD\_ECKERT\_V = PrjCoordSysType.PCS\_WORLD\_ECKERT\_V
     |  
     |  PCS\_WORLD\_ECKERT\_VI = PrjCoordSysType.PCS\_WORLD\_ECKERT\_VI
     |  
     |  PCS\_WORLD\_EQUIDISTANT\_CONIC = PrjCoordSysType.PCS\_WORLD\_EQUIDISTANT\_CO{\ldots}
     |  
     |  PCS\_WORLD\_EQUIDISTANT\_CYLINDRICAL = PrjCoordSysType.PCS\_WORLD\_EQUIDIST{\ldots}
     |  
     |  PCS\_WORLD\_GALL\_STEREOGRAPHIC = PrjCoordSysType.PCS\_WORLD\_GALL\_STEREOGR{\ldots}
     |  
     |  PCS\_WORLD\_HOTINE = PrjCoordSysType.PCS\_WORLD\_HOTINE
     |  
     |  PCS\_WORLD\_LOXIMUTHAL = PrjCoordSysType.PCS\_WORLD\_LOXIMUTHAL
     |  
     |  PCS\_WORLD\_MERCATOR = PrjCoordSysType.PCS\_WORLD\_MERCATOR
     |  
     |  PCS\_WORLD\_MILLER\_CYLINDRICAL = PrjCoordSysType.PCS\_WORLD\_MILLER\_CYLIND{\ldots}
     |  
     |  PCS\_WORLD\_MOLLWEIDE = PrjCoordSysType.PCS\_WORLD\_MOLLWEIDE
     |  
     |  PCS\_WORLD\_PLATE\_CARREE = PrjCoordSysType.PCS\_WORLD\_PLATE\_CARREE
     |  
     |  PCS\_WORLD\_POLYCONIC = PrjCoordSysType.PCS\_WORLD\_POLYCONIC
     |  
     |  PCS\_WORLD\_QUARTIC\_AUTHALIC = PrjCoordSysType.PCS\_WORLD\_QUARTIC\_AUTHALI{\ldots}
     |  
     |  PCS\_WORLD\_ROBINSON = PrjCoordSysType.PCS\_WORLD\_ROBINSON
     |  
     |  PCS\_WORLD\_SINUSOIDAL = PrjCoordSysType.PCS\_WORLD\_SINUSOIDAL
     |  
     |  PCS\_WORLD\_STEREOGRAPHIC = PrjCoordSysType.PCS\_WORLD\_STEREOGRAPHIC
     |  
     |  PCS\_WORLD\_TWO\_POINT\_EQUIDISTANT = PrjCoordSysType.PCS\_WORLD\_TWO\_POINT\_{\ldots}
     |  
     |  PCS\_WORLD\_VAN\_DER\_GRINTEN\_I = PrjCoordSysType.PCS\_WORLD\_VAN\_DER\_GRINTE{\ldots}
     |  
     |  PCS\_WORLD\_WINKEL\_I = PrjCoordSysType.PCS\_WORLD\_WINKEL\_I
     |  
     |  PCS\_WORLD\_WINKEL\_II = PrjCoordSysType.PCS\_WORLD\_WINKEL\_II
     |  
     |  PCS\_XIAN\_1980\_3\_DEGREE\_GK\_25 = PrjCoordSysType.PCS\_XIAN\_1980\_3\_DEGREE\_{\ldots}
     |  
     |  PCS\_XIAN\_1980\_3\_DEGREE\_GK\_25N = PrjCoordSysType.PCS\_XIAN\_1980\_3\_DEGREE{\ldots}
     |  
     |  PCS\_XIAN\_1980\_3\_DEGREE\_GK\_26 = PrjCoordSysType.PCS\_XIAN\_1980\_3\_DEGREE\_{\ldots}
     |  
     |  PCS\_XIAN\_1980\_3\_DEGREE\_GK\_26N = PrjCoordSysType.PCS\_XIAN\_1980\_3\_DEGREE{\ldots}
     |  
     |  PCS\_XIAN\_1980\_3\_DEGREE\_GK\_27 = PrjCoordSysType.PCS\_XIAN\_1980\_3\_DEGREE\_{\ldots}
     |  
     |  PCS\_XIAN\_1980\_3\_DEGREE\_GK\_27N = PrjCoordSysType.PCS\_XIAN\_1980\_3\_DEGREE{\ldots}
     |  
     |  PCS\_XIAN\_1980\_3\_DEGREE\_GK\_28 = PrjCoordSysType.PCS\_XIAN\_1980\_3\_DEGREE\_{\ldots}
     |  
     |  PCS\_XIAN\_1980\_3\_DEGREE\_GK\_28N = PrjCoordSysType.PCS\_XIAN\_1980\_3\_DEGREE{\ldots}
     |  
     |  PCS\_XIAN\_1980\_3\_DEGREE\_GK\_29 = PrjCoordSysType.PCS\_XIAN\_1980\_3\_DEGREE\_{\ldots}
     |  
     |  PCS\_XIAN\_1980\_3\_DEGREE\_GK\_29N = PrjCoordSysType.PCS\_XIAN\_1980\_3\_DEGREE{\ldots}
     |  
     |  PCS\_XIAN\_1980\_3\_DEGREE\_GK\_30 = PrjCoordSysType.PCS\_XIAN\_1980\_3\_DEGREE\_{\ldots}
     |  
     |  PCS\_XIAN\_1980\_3\_DEGREE\_GK\_30N = PrjCoordSysType.PCS\_XIAN\_1980\_3\_DEGREE{\ldots}
     |  
     |  PCS\_XIAN\_1980\_3\_DEGREE\_GK\_31 = PrjCoordSysType.PCS\_XIAN\_1980\_3\_DEGREE\_{\ldots}
     |  
     |  PCS\_XIAN\_1980\_3\_DEGREE\_GK\_31N = PrjCoordSysType.PCS\_XIAN\_1980\_3\_DEGREE{\ldots}
     |  
     |  PCS\_XIAN\_1980\_3\_DEGREE\_GK\_32 = PrjCoordSysType.PCS\_XIAN\_1980\_3\_DEGREE\_{\ldots}
     |  
     |  PCS\_XIAN\_1980\_3\_DEGREE\_GK\_32N = PrjCoordSysType.PCS\_XIAN\_1980\_3\_DEGREE{\ldots}
     |  
     |  PCS\_XIAN\_1980\_3\_DEGREE\_GK\_33 = PrjCoordSysType.PCS\_XIAN\_1980\_3\_DEGREE\_{\ldots}
     |  
     |  PCS\_XIAN\_1980\_3\_DEGREE\_GK\_33N = PrjCoordSysType.PCS\_XIAN\_1980\_3\_DEGREE{\ldots}
     |  
     |  PCS\_XIAN\_1980\_3\_DEGREE\_GK\_34 = PrjCoordSysType.PCS\_XIAN\_1980\_3\_DEGREE\_{\ldots}
     |  
     |  PCS\_XIAN\_1980\_3\_DEGREE\_GK\_34N = PrjCoordSysType.PCS\_XIAN\_1980\_3\_DEGREE{\ldots}
     |  
     |  PCS\_XIAN\_1980\_3\_DEGREE\_GK\_35 = PrjCoordSysType.PCS\_XIAN\_1980\_3\_DEGREE\_{\ldots}
     |  
     |  PCS\_XIAN\_1980\_3\_DEGREE\_GK\_35N = PrjCoordSysType.PCS\_XIAN\_1980\_3\_DEGREE{\ldots}
     |  
     |  PCS\_XIAN\_1980\_3\_DEGREE\_GK\_36 = PrjCoordSysType.PCS\_XIAN\_1980\_3\_DEGREE\_{\ldots}
     |  
     |  PCS\_XIAN\_1980\_3\_DEGREE\_GK\_36N = PrjCoordSysType.PCS\_XIAN\_1980\_3\_DEGREE{\ldots}
     |  
     |  PCS\_XIAN\_1980\_3\_DEGREE\_GK\_37 = PrjCoordSysType.PCS\_XIAN\_1980\_3\_DEGREE\_{\ldots}
     |  
     |  PCS\_XIAN\_1980\_3\_DEGREE\_GK\_37N = PrjCoordSysType.PCS\_XIAN\_1980\_3\_DEGREE{\ldots}
     |  
     |  PCS\_XIAN\_1980\_3\_DEGREE\_GK\_38 = PrjCoordSysType.PCS\_XIAN\_1980\_3\_DEGREE\_{\ldots}
     |  
     |  PCS\_XIAN\_1980\_3\_DEGREE\_GK\_38N = PrjCoordSysType.PCS\_XIAN\_1980\_3\_DEGREE{\ldots}
     |  
     |  PCS\_XIAN\_1980\_3\_DEGREE\_GK\_39 = PrjCoordSysType.PCS\_XIAN\_1980\_3\_DEGREE\_{\ldots}
     |  
     |  PCS\_XIAN\_1980\_3\_DEGREE\_GK\_39N = PrjCoordSysType.PCS\_XIAN\_1980\_3\_DEGREE{\ldots}
     |  
     |  PCS\_XIAN\_1980\_3\_DEGREE\_GK\_40 = PrjCoordSysType.PCS\_XIAN\_1980\_3\_DEGREE\_{\ldots}
     |  
     |  PCS\_XIAN\_1980\_3\_DEGREE\_GK\_40N = PrjCoordSysType.PCS\_XIAN\_1980\_3\_DEGREE{\ldots}
     |  
     |  PCS\_XIAN\_1980\_3\_DEGREE\_GK\_41 = PrjCoordSysType.PCS\_XIAN\_1980\_3\_DEGREE\_{\ldots}
     |  
     |  PCS\_XIAN\_1980\_3\_DEGREE\_GK\_41N = PrjCoordSysType.PCS\_XIAN\_1980\_3\_DEGREE{\ldots}
     |  
     |  PCS\_XIAN\_1980\_3\_DEGREE\_GK\_42 = PrjCoordSysType.PCS\_XIAN\_1980\_3\_DEGREE\_{\ldots}
     |  
     |  PCS\_XIAN\_1980\_3\_DEGREE\_GK\_42N = PrjCoordSysType.PCS\_XIAN\_1980\_3\_DEGREE{\ldots}
     |  
     |  PCS\_XIAN\_1980\_3\_DEGREE\_GK\_43 = PrjCoordSysType.PCS\_XIAN\_1980\_3\_DEGREE\_{\ldots}
     |  
     |  PCS\_XIAN\_1980\_3\_DEGREE\_GK\_43N = PrjCoordSysType.PCS\_XIAN\_1980\_3\_DEGREE{\ldots}
     |  
     |  PCS\_XIAN\_1980\_3\_DEGREE\_GK\_44 = PrjCoordSysType.PCS\_XIAN\_1980\_3\_DEGREE\_{\ldots}
     |  
     |  PCS\_XIAN\_1980\_3\_DEGREE\_GK\_44N = PrjCoordSysType.PCS\_XIAN\_1980\_3\_DEGREE{\ldots}
     |  
     |  PCS\_XIAN\_1980\_3\_DEGREE\_GK\_45 = PrjCoordSysType.PCS\_XIAN\_1980\_3\_DEGREE\_{\ldots}
     |  
     |  PCS\_XIAN\_1980\_3\_DEGREE\_GK\_45N = PrjCoordSysType.PCS\_XIAN\_1980\_3\_DEGREE{\ldots}
     |  
     |  PCS\_XIAN\_1980\_GK\_13 = PrjCoordSysType.PCS\_XIAN\_1980\_GK\_13
     |  
     |  PCS\_XIAN\_1980\_GK\_13N = PrjCoordSysType.PCS\_XIAN\_1980\_GK\_13N
     |  
     |  PCS\_XIAN\_1980\_GK\_14 = PrjCoordSysType.PCS\_XIAN\_1980\_GK\_14
     |  
     |  PCS\_XIAN\_1980\_GK\_14N = PrjCoordSysType.PCS\_XIAN\_1980\_GK\_14N
     |  
     |  PCS\_XIAN\_1980\_GK\_15 = PrjCoordSysType.PCS\_XIAN\_1980\_GK\_15
     |  
     |  PCS\_XIAN\_1980\_GK\_15N = PrjCoordSysType.PCS\_XIAN\_1980\_GK\_15N
     |  
     |  PCS\_XIAN\_1980\_GK\_16 = PrjCoordSysType.PCS\_XIAN\_1980\_GK\_16
     |  
     |  PCS\_XIAN\_1980\_GK\_16N = PrjCoordSysType.PCS\_XIAN\_1980\_GK\_16N
     |  
     |  PCS\_XIAN\_1980\_GK\_17 = PrjCoordSysType.PCS\_XIAN\_1980\_GK\_17
     |  
     |  PCS\_XIAN\_1980\_GK\_17N = PrjCoordSysType.PCS\_XIAN\_1980\_GK\_17N
     |  
     |  PCS\_XIAN\_1980\_GK\_18 = PrjCoordSysType.PCS\_XIAN\_1980\_GK\_18
     |  
     |  PCS\_XIAN\_1980\_GK\_18N = PrjCoordSysType.PCS\_XIAN\_1980\_GK\_18N
     |  
     |  PCS\_XIAN\_1980\_GK\_19 = PrjCoordSysType.PCS\_XIAN\_1980\_GK\_19
     |  
     |  PCS\_XIAN\_1980\_GK\_19N = PrjCoordSysType.PCS\_XIAN\_1980\_GK\_19N
     |  
     |  PCS\_XIAN\_1980\_GK\_20 = PrjCoordSysType.PCS\_XIAN\_1980\_GK\_20
     |  
     |  PCS\_XIAN\_1980\_GK\_20N = PrjCoordSysType.PCS\_XIAN\_1980\_GK\_20N
     |  
     |  PCS\_XIAN\_1980\_GK\_21 = PrjCoordSysType.PCS\_XIAN\_1980\_GK\_21
     |  
     |  PCS\_XIAN\_1980\_GK\_21N = PrjCoordSysType.PCS\_XIAN\_1980\_GK\_21N
     |  
     |  PCS\_XIAN\_1980\_GK\_22 = PrjCoordSysType.PCS\_XIAN\_1980\_GK\_22
     |  
     |  PCS\_XIAN\_1980\_GK\_22N = PrjCoordSysType.PCS\_XIAN\_1980\_GK\_22N
     |  
     |  PCS\_XIAN\_1980\_GK\_23 = PrjCoordSysType.PCS\_XIAN\_1980\_GK\_23
     |  
     |  PCS\_XIAN\_1980\_GK\_23N = PrjCoordSysType.PCS\_XIAN\_1980\_GK\_23N
     |  
     |  PCS\_YOFF\_1972\_UTM\_28N = PrjCoordSysType.PCS\_YOFF\_1972\_UTM\_28N
     |  
     |  PCS\_ZANDERIJ\_1972\_UTM\_21N = PrjCoordSysType.PCS\_ZANDERIJ\_1972\_UTM\_21N
     |  
     |  ----------------------------------------------------------------------
     |  Data descriptors inherited from enum.Enum:
     |  
     |  name
     |      The name of the Enum member.
     |  
     |  value
     |      The value of the Enum member.
     |  
     |  ----------------------------------------------------------------------
     |  Data descriptors inherited from enum.EnumMeta:
     |  
     |  \_\_members\_\_
     |      Returns a mapping of member name->value.
     |      
     |      This mapping lists all enum members, including aliases. Note that this
     |      is a read-only view of the internal mapping.
    
    class ProjectionType(JEnum)
     |  An enumeration.
     |  
     |  Method resolution order:
     |      ProjectionType
     |      JEnum
     |      enum.IntEnum
     |      builtins.int
     |      enum.Enum
     |      builtins.object
     |  
     |  Data and other attributes defined here:
     |  
     |  PRJ\_ALBERS = ProjectionType.PRJ\_ALBERS
     |  
     |  PRJ\_BAIDU\_MERCATOR = ProjectionType.PRJ\_BAIDU\_MERCATOR
     |  
     |  PRJ\_BEHRMANN = ProjectionType.PRJ\_BEHRMANN
     |  
     |  PRJ\_BONNE = ProjectionType.PRJ\_BONNE
     |  
     |  PRJ\_BONNE\_SOUTH\_ORIENTATED = ProjectionType.PRJ\_BONNE\_SOUTH\_ORIENTATED
     |  
     |  PRJ\_CASSINI = ProjectionType.PRJ\_CASSINI
     |  
     |  PRJ\_CHINA\_AZIMUTHAL = ProjectionType.PRJ\_CHINA\_AZIMUTHAL
     |  
     |  PRJ\_CONFORMAL\_AZIMUTHAL = ProjectionType.PRJ\_CONFORMAL\_AZIMUTHAL
     |  
     |  PRJ\_ECKERT\_I = ProjectionType.PRJ\_ECKERT\_I
     |  
     |  PRJ\_ECKERT\_II = ProjectionType.PRJ\_ECKERT\_II
     |  
     |  PRJ\_ECKERT\_III = ProjectionType.PRJ\_ECKERT\_III
     |  
     |  PRJ\_ECKERT\_IV = ProjectionType.PRJ\_ECKERT\_IV
     |  
     |  PRJ\_ECKERT\_V = ProjectionType.PRJ\_ECKERT\_V
     |  
     |  PRJ\_ECKERT\_VI = ProjectionType.PRJ\_ECKERT\_VI
     |  
     |  PRJ\_EQUALAREA\_CYLINDRICAL = ProjectionType.PRJ\_EQUALAREA\_CYLINDRICAL
     |  
     |  PRJ\_EQUIDISTANT\_AZIMUTHAL = ProjectionType.PRJ\_EQUIDISTANT\_AZIMUTHAL
     |  
     |  PRJ\_EQUIDISTANT\_CONIC = ProjectionType.PRJ\_EQUIDISTANT\_CONIC
     |  
     |  PRJ\_EQUIDISTANT\_CYLINDRICAL = ProjectionType.PRJ\_EQUIDISTANT\_CYLINDRIC{\ldots}
     |  
     |  PRJ\_GALL\_STEREOGRAPHIC = ProjectionType.PRJ\_GALL\_STEREOGRAPHIC
     |  
     |  PRJ\_GAUSS\_KRUGER = ProjectionType.PRJ\_GAUSS\_KRUGER
     |  
     |  PRJ\_GNOMONIC = ProjectionType.PRJ\_GNOMONIC
     |  
     |  PRJ\_HOTINE = ProjectionType.PRJ\_HOTINE
     |  
     |  PRJ\_HOTINE\_AZIMUTH\_NATORIGIN = ProjectionType.PRJ\_HOTINE\_AZIMUTH\_NATOR{\ldots}
     |  
     |  PRJ\_HOTINE\_OBLIQUE\_MERCATOR = ProjectionType.PRJ\_HOTINE\_OBLIQUE\_MERCAT{\ldots}
     |  
     |  PRJ\_LAMBERT\_AZIMUTHAL\_EQUAL\_AREA = ProjectionType.PRJ\_LAMBERT\_AZIMUTHA{\ldots}
     |  
     |  PRJ\_LAMBERT\_CONFORMAL\_CONIC = ProjectionType.PRJ\_LAMBERT\_CONFORMAL\_CON{\ldots}
     |  
     |  PRJ\_LOXIMUTHAL = ProjectionType.PRJ\_LOXIMUTHAL
     |  
     |  PRJ\_MERCATOR = ProjectionType.PRJ\_MERCATOR
     |  
     |  PRJ\_MILLER\_CYLINDRICAL = ProjectionType.PRJ\_MILLER\_CYLINDRICAL
     |  
     |  PRJ\_MOLLWEIDE = ProjectionType.PRJ\_MOLLWEIDE
     |  
     |  PRJ\_NONPROJECTION = ProjectionType.PRJ\_NONPROJECTION
     |  
     |  PRJ\_OBLIQUE\_MERCATOR = ProjectionType.PRJ\_OBLIQUE\_MERCATOR
     |  
     |  PRJ\_OBLIQUE\_STEREOGRAPHIC = ProjectionType.PRJ\_OBLIQUE\_STEREOGRAPHIC
     |  
     |  PRJ\_ORTHO\_GRAPHIC = ProjectionType.PRJ\_ORTHO\_GRAPHIC
     |  
     |  PRJ\_PLATE\_CARREE = ProjectionType.PRJ\_PLATE\_CARREE
     |  
     |  PRJ\_POLYCONIC = ProjectionType.PRJ\_POLYCONIC
     |  
     |  PRJ\_QUARTIC\_AUTHALIC = ProjectionType.PRJ\_QUARTIC\_AUTHALIC
     |  
     |  PRJ\_RECTIFIED\_SKEWED\_ORTHOMORPHIC = ProjectionType.PRJ\_RECTIFIED\_SKEWE{\ldots}
     |  
     |  PRJ\_ROBINSON = ProjectionType.PRJ\_ROBINSON
     |  
     |  PRJ\_SANSON = ProjectionType.PRJ\_SANSON
     |  
     |  PRJ\_SINUSOIDAL = ProjectionType.PRJ\_SINUSOIDAL
     |  
     |  PRJ\_SPHERE\_MERCATOR = ProjectionType.PRJ\_SPHERE\_MERCATOR
     |  
     |  PRJ\_STEREOGRAPHIC = ProjectionType.PRJ\_STEREOGRAPHIC
     |  
     |  PRJ\_TRANSVERSE\_MERCATOR = ProjectionType.PRJ\_TRANSVERSE\_MERCATOR
     |  
     |  PRJ\_TWO\_POINT\_EQUIDISTANT = ProjectionType.PRJ\_TWO\_POINT\_EQUIDISTANT
     |  
     |  PRJ\_VAN\_DER\_GRINTEN\_I = ProjectionType.PRJ\_VAN\_DER\_GRINTEN\_I
     |  
     |  PRJ\_WINKEL\_I = ProjectionType.PRJ\_WINKEL\_I
     |  
     |  PRJ\_WINKEL\_II = ProjectionType.PRJ\_WINKEL\_II
     |  
     |  ----------------------------------------------------------------------
     |  Data descriptors inherited from enum.Enum:
     |  
     |  name
     |      The name of the Enum member.
     |  
     |  value
     |      The value of the Enum member.
     |  
     |  ----------------------------------------------------------------------
     |  Data descriptors inherited from enum.EnumMeta:
     |  
     |  \_\_members\_\_
     |      Returns a mapping of member name->value.
     |      
     |      This mapping lists all enum members, including aliases. Note that this
     |      is a read-only view of the internal mapping.
    
    class RasterJoinPixelFormat(JEnum)
     |  定义了镶嵌结果像素格式类型常量。
     |  
     |  :var RasterJoinPixelFormat.RJPMONO: 即 PixelFormat.UBIT1。
     |  :var RasterJoinPixelFormat.RJPFBIT: 即 PixelFormat.UBIT4
     |  :var RasterJoinPixelFormat.RJPBYTE: 即 PixelFormat.UBIT8
     |  :var RasterJoinPixelFormat.RJPTBYTE: 即 PixelFormat.BIT16
     |  :var RasterJoinPixelFormat.RJPRGB: 即 PixelFormat.RGB
     |  :var RasterJoinPixelFormat.RJPRGBAFBIT: 即 PixelFormat.RGBA
     |  :var RasterJoinPixelFormat.RJPLONGLONG: 即 PixelFormat.BIT64
     |  :var RasterJoinPixelFormat.RJPLONG: 即 PixelFormat.BIT32
     |  :var RasterJoinPixelFormat.RJPFLOAT: 即 PixelFormat.SINGLE
     |  :var RasterJoinPixelFormat.RJPDOUBLE: 即 PixelFormat.DOUBLE
     |  :var RasterJoinPixelFormat.RJPFIRST: 参与镶嵌的第一个栅格数据集的像素格式。
     |  :var RasterJoinPixelFormat.RJPLAST: 参与镶嵌的最后一个栅格数据集的像素格式。
     |  :var RasterJoinPixelFormat.RJPMAX: 参与镶嵌的栅格数据集中最大的像素格式。
     |  :var RasterJoinPixelFormat.RJPMIN: 参与镶嵌的栅格数据集中最小的像素格式。
     |  :var RasterJoinPixelFormat.RJPMAJORITY: 参与镶嵌的栅格数据集中出现频率最高的像素格式,如果像素格式出现的频率相同,取索引值最小的。
     |  
     |  Method resolution order:
     |      RasterJoinPixelFormat
     |      JEnum
     |      enum.IntEnum
     |      builtins.int
     |      enum.Enum
     |      builtins.object
     |  
     |  Data and other attributes defined here:
     |  
     |  RJPBYTE = RasterJoinPixelFormat.RJPBYTE
     |  
     |  RJPDOUBLE = RasterJoinPixelFormat.RJPDOUBLE
     |  
     |  RJPFBIT = RasterJoinPixelFormat.RJPFBIT
     |  
     |  RJPFIRST = RasterJoinPixelFormat.RJPFIRST
     |  
     |  RJPFLOAT = RasterJoinPixelFormat.RJPFLOAT
     |  
     |  RJPLAST = RasterJoinPixelFormat.RJPLAST
     |  
     |  RJPLONG = RasterJoinPixelFormat.RJPLONG
     |  
     |  RJPLONGLONG = RasterJoinPixelFormat.RJPLONGLONG
     |  
     |  RJPMAJORITY = RasterJoinPixelFormat.RJPMAJORITY
     |  
     |  RJPMAX = RasterJoinPixelFormat.RJPMAX
     |  
     |  RJPMIN = RasterJoinPixelFormat.RJPMIN
     |  
     |  RJPMONO = RasterJoinPixelFormat.RJPMONO
     |  
     |  RJPRGB = RasterJoinPixelFormat.RJPRGB
     |  
     |  RJPRGBAFBIT = RasterJoinPixelFormat.RJPRGBAFBIT
     |  
     |  RJPTBYTE = RasterJoinPixelFormat.RJPTBYTE
     |  
     |  ----------------------------------------------------------------------
     |  Data descriptors inherited from enum.Enum:
     |  
     |  name
     |      The name of the Enum member.
     |  
     |  value
     |      The value of the Enum member.
     |  
     |  ----------------------------------------------------------------------
     |  Data descriptors inherited from enum.EnumMeta:
     |  
     |  \_\_members\_\_
     |      Returns a mapping of member name->value.
     |      
     |      This mapping lists all enum members, including aliases. Note that this
     |      is a read-only view of the internal mapping.
    
    class RasterJoinType(JEnum)
     |  定义了镶嵌结果栅格值的统计类型常量。
     |  
     |  :var RasterJoinType.RJMFIRST: 栅格重叠区域镶嵌后取第一个栅格数据集中的值。
     |  :var RasterJoinType.RJMLAST: 栅格重叠区域镶嵌后取最后一个栅格数据集中的值。
     |  :var RasterJoinType.RJMMAX: 栅格重叠区域镶嵌后取所有栅格数据集中相应位置的最大值。
     |  :var RasterJoinType.RJMMIN: 栅格重叠区域镶嵌后取所有栅格数据集中相应位置的最小值。
     |  :var RasterJoinType.RJMMean: 栅格重叠区域镶嵌后取所有栅格数据集中相应位置的平均值。
     |  
     |  Method resolution order:
     |      RasterJoinType
     |      JEnum
     |      enum.IntEnum
     |      builtins.int
     |      enum.Enum
     |      builtins.object
     |  
     |  Data and other attributes defined here:
     |  
     |  RJMFIRST = RasterJoinType.RJMFIRST
     |  
     |  RJMLAST = RasterJoinType.RJMLAST
     |  
     |  RJMMAX = RasterJoinType.RJMMAX
     |  
     |  RJMMIN = RasterJoinType.RJMMIN
     |  
     |  RJMMean = RasterJoinType.RJMMean
     |  
     |  ----------------------------------------------------------------------
     |  Data descriptors inherited from enum.Enum:
     |  
     |  name
     |      The name of the Enum member.
     |  
     |  value
     |      The value of the Enum member.
     |  
     |  ----------------------------------------------------------------------
     |  Data descriptors inherited from enum.EnumMeta:
     |  
     |  \_\_members\_\_
     |      Returns a mapping of member name->value.
     |      
     |      This mapping lists all enum members, including aliases. Note that this
     |      is a read-only view of the internal mapping.
    
    class RasterResampleMode(JEnum)
     |  栅格重采样计算方式的类型常量
     |  
     |  :var RasterResampleMode.NEAREST:  最邻近法。最邻近法是将最邻近的栅格值赋予新栅格。该方法的优点是不会改变原始栅格值,简单且处理速度快,但该种方法最大会有半个格子大小的位移。适用于表示分类或某种专题的离散数据,如土地利用,植被类型等。
     |  :var RasterResampleMode.BILINEAR:  双线性内插法。双线性内插使用内插点在输入栅格中的 4 邻域进行加权平均来计算新栅格值,权值根据 4 邻域中每个格子中心距内插点的距离来决定。该种方法的重采样结果会比最邻近法的结果更光滑,但会改变原来的栅格值。适用于表示某种现象分布、地形表面的连续数据,如 DEM、气温、降雨量分布、坡度等,这些数据本来就是通过采样点内插得到的连续表面。
     |  :var RasterResampleMode.CUBIC: 三次卷积内插法。三次卷积内插法较为复杂,与双线性内插相似,同样会改变栅格值,不同之处在于它使用 16 邻域来加权计算,会使计算结果得到一些锐化的效果。该种方法同样会改变原来的栅格值,且有可能会超出输入栅格的值域范围,且计算量大。适用于航片和遥感影像的重采样。
     |  
     |  Method resolution order:
     |      RasterResampleMode
     |      JEnum
     |      enum.IntEnum
     |      builtins.int
     |      enum.Enum
     |      builtins.object
     |  
     |  Data and other attributes defined here:
     |  
     |  BILINEAR = RasterResampleMode.BILINEAR
     |  
     |  CUBIC = RasterResampleMode.CUBIC
     |  
     |  NEAREST = RasterResampleMode.NEAREST
     |  
     |  ----------------------------------------------------------------------
     |  Data descriptors inherited from enum.Enum:
     |  
     |  name
     |      The name of the Enum member.
     |  
     |  value
     |      The value of the Enum member.
     |  
     |  ----------------------------------------------------------------------
     |  Data descriptors inherited from enum.EnumMeta:
     |  
     |  \_\_members\_\_
     |      Returns a mapping of member name->value.
     |      
     |      This mapping lists all enum members, including aliases. Note that this
     |      is a read-only view of the internal mapping.
    
    class ReclassPixelFormat(JEnum)
     |  该类定义了栅格数据集的像元值的存储类型常量
     |  
     |  :var ReclassPixelFormat.BIT32: 整型
     |  :var ReclassPixelFormat.BIT64: 长整型
     |  :var ReclassPixelFormat.SINGLE: 单精度
     |  :var ReclassPixelFormat.DOUBLE: 双精度
     |  
     |  Method resolution order:
     |      ReclassPixelFormat
     |      JEnum
     |      enum.IntEnum
     |      builtins.int
     |      enum.Enum
     |      builtins.object
     |  
     |  Data and other attributes defined here:
     |  
     |  BIT32 = ReclassPixelFormat.BIT32
     |  
     |  BIT64 = ReclassPixelFormat.BIT64
     |  
     |  DOUBLE = ReclassPixelFormat.DOUBLE
     |  
     |  SINGLE = ReclassPixelFormat.SINGLE
     |  
     |  ----------------------------------------------------------------------
     |  Data descriptors inherited from enum.Enum:
     |  
     |  name
     |      The name of the Enum member.
     |  
     |  value
     |      The value of the Enum member.
     |  
     |  ----------------------------------------------------------------------
     |  Data descriptors inherited from enum.EnumMeta:
     |  
     |  \_\_members\_\_
     |      Returns a mapping of member name->value.
     |      
     |      This mapping lists all enum members, including aliases. Note that this
     |      is a read-only view of the internal mapping.
    
    class ReclassSegmentType(JEnum)
     |  该类定义了重分级区间类型常量。
     |  
     |  :var ReclassSegmentType.OPENCLOSE: 左开右闭,如 (number1, number2]。
     |  :var ReclassSegmentType.CLOSEOPEN: 左闭右开,如 [number1, number2)。
     |  
     |  Method resolution order:
     |      ReclassSegmentType
     |      JEnum
     |      enum.IntEnum
     |      builtins.int
     |      enum.Enum
     |      builtins.object
     |  
     |  Data and other attributes defined here:
     |  
     |  CLOSEOPEN = ReclassSegmentType.CLOSEOPEN
     |  
     |  OPENCLOSE = ReclassSegmentType.OPENCLOSE
     |  
     |  ----------------------------------------------------------------------
     |  Data descriptors inherited from enum.Enum:
     |  
     |  name
     |      The name of the Enum member.
     |  
     |  value
     |      The value of the Enum member.
     |  
     |  ----------------------------------------------------------------------
     |  Data descriptors inherited from enum.EnumMeta:
     |  
     |  \_\_members\_\_
     |      Returns a mapping of member name->value.
     |      
     |      This mapping lists all enum members, including aliases. Note that this
     |      is a read-only view of the internal mapping.
    
    class ReclassType(JEnum)
     |  该类定义了栅格重分级类型常量
     |  
     |  :var ReclassType.UNIQUE: 单值重分级,即对指定的某些单值进行重新赋值。
     |  :var ReclassType.RANGE: 范围重分级,即将一个区间内的值重新赋值为一个值。
     |  
     |  Method resolution order:
     |      ReclassType
     |      JEnum
     |      enum.IntEnum
     |      builtins.int
     |      enum.Enum
     |      builtins.object
     |  
     |  Data and other attributes defined here:
     |  
     |  RANGE = ReclassType.RANGE
     |  
     |  UNIQUE = ReclassType.UNIQUE
     |  
     |  ----------------------------------------------------------------------
     |  Data descriptors inherited from enum.Enum:
     |  
     |  name
     |      The name of the Enum member.
     |  
     |  value
     |      The value of the Enum member.
     |  
     |  ----------------------------------------------------------------------
     |  Data descriptors inherited from enum.EnumMeta:
     |  
     |  \_\_members\_\_
     |      Returns a mapping of member name->value.
     |      
     |      This mapping lists all enum members, including aliases. Note that this
     |      is a read-only view of the internal mapping.
    
    class RegionToPointMode(JEnum)
     |  面转点的方式
     |  
     |  :var RegionToPointMode.VERTEX: 节点模式,将面对象的每个节点都转换为一个点对象
     |  :var RegionToPointMode.INNER\_POINT: 内点模式,将面对象的内点转换为一个点对象
     |  :var RegionToPointMode.SUB\_INNER\_POINT: 子对象内点模式,将面对象的每个子对象的内点分别转换为一个点对象
     |  :var RegionToPointMode.TOPO\_INNER\_POINT: 拓扑内点模式,对复杂面对象进行保护性分解后得到的多个面对象的内点,分别转换为一个子对象。
     |  
     |  Method resolution order:
     |      RegionToPointMode
     |      JEnum
     |      enum.IntEnum
     |      builtins.int
     |      enum.Enum
     |      builtins.object
     |  
     |  Data and other attributes defined here:
     |  
     |  INNER\_POINT = RegionToPointMode.INNER\_POINT
     |  
     |  SUB\_INNER\_POINT = RegionToPointMode.SUB\_INNER\_POINT
     |  
     |  TOPO\_INNER\_POINT = RegionToPointMode.TOPO\_INNER\_POINT
     |  
     |  VERTEX = RegionToPointMode.VERTEX
     |  
     |  ----------------------------------------------------------------------
     |  Data descriptors inherited from enum.Enum:
     |  
     |  name
     |      The name of the Enum member.
     |  
     |  value
     |      The value of the Enum member.
     |  
     |  ----------------------------------------------------------------------
     |  Data descriptors inherited from enum.EnumMeta:
     |  
     |  \_\_members\_\_
     |      Returns a mapping of member name->value.
     |      
     |      This mapping lists all enum members, including aliases. Note that this
     |      is a read-only view of the internal mapping.
    
    class ResamplingMethod(JEnum)
     |  该类定义了创建金字塔类型常量。
     |  
     |  :var ResamplingMethod.AVERAGE: 平均值
     |  :var ResamplingMethod.NEAR: 邻近值
     |  
     |  Method resolution order:
     |      ResamplingMethod
     |      JEnum
     |      enum.IntEnum
     |      builtins.int
     |      enum.Enum
     |      builtins.object
     |  
     |  Data and other attributes defined here:
     |  
     |  AVERAGE = ResamplingMethod.AVERAGE
     |  
     |  NEAR = ResamplingMethod.NEAR
     |  
     |  ----------------------------------------------------------------------
     |  Data descriptors inherited from enum.Enum:
     |  
     |  name
     |      The name of the Enum member.
     |  
     |  value
     |      The value of the Enum member.
     |  
     |  ----------------------------------------------------------------------
     |  Data descriptors inherited from enum.EnumMeta:
     |  
     |  \_\_members\_\_
     |      Returns a mapping of member name->value.
     |      
     |      This mapping lists all enum members, including aliases. Note that this
     |      is a read-only view of the internal mapping.
    
    class SearchMode(JEnum)
     |  该类定义了内插时使用的样本点的查找方式类型常量。
     |  
     |  对于同一种插值方法,样本点的选择方法不同,得到的插值结果也会不同。SuperMap 提供四种插值查找方式,分别为不进行查找,块(QUADTREE) 查找,定长查找(KDTREE\_FIXED\_RADIUS)和 变长查找(KDTREE\_FIXED\_COUNT)。
     |  
     |  :var SearchMode.NONE: 不进行查找,使用所有的输入点进行内插分析。
     |  :var SearchMode.QUADTREE: 块查找方式,即根据设置的每个块内的点的最多数量对数据集进行分块,使用块内的点进行插值运算。
     |                            注意: 目前只对 Kriging、RBF 插值方法起作用,而对 IDW 插值方法不起作用。
     |  :var SearchMode.KDTREE\_FIXED\_RADIUS: 定长查找方式,即指定半径范围内所有的采样点都参与栅格单元的插值运算。该方式由查找半径(search\_radius)和
     |                                       期望参与运算的最少样点数(expected\_count)两个参数来最终确定参与运算的采样点。
     |                                       当计算某个位置的未知数值时,会以该位置为圆心,以设定的定长值(即查找半径)为半径,落在这个范围内的
     |                                       采样点都将参与运算;但如果设置了期望参与运算的最少点数,若查找半径范围内的点数达不到该数值,将自动
     |                                       扩展查找半径直到找到指定的数目的采样点。
     |  :var SearchMode.KDTREE\_FIXED\_COUNT: 变长查找方式,即距离栅格单元最近的指定数目的采样点参与插值运算。该方式由期望参与运算的最多样
     |                                      点数(expected\_count)和查找半径(search\_radius)两个参数来最终确定参与运算的采样点。当计算某
     |                                      个位置的未知数值时,会查找该位置附近的 N 个采样点,N 值即为设定的固定点数(即期望参与运算的最多样点
     |                                      数),那么这 N 个采样点都将参与运算;但如果设置了查找半径,若半径范围内的点数少于设置的固定点数,则
     |                                      范围之外的采样点被舍弃,不参与运算。
     |  
     |  Method resolution order:
     |      SearchMode
     |      JEnum
     |      enum.IntEnum
     |      builtins.int
     |      enum.Enum
     |      builtins.object
     |  
     |  Data and other attributes defined here:
     |  
     |  KDTREE\_FIXED\_COUNT = SearchMode.KDTREE\_FIXED\_COUNT
     |  
     |  KDTREE\_FIXED\_RADIUS = SearchMode.KDTREE\_FIXED\_RADIUS
     |  
     |  NONE = SearchMode.NONE
     |  
     |  QUADTREE = SearchMode.QUADTREE
     |  
     |  ----------------------------------------------------------------------
     |  Data descriptors inherited from enum.Enum:
     |  
     |  name
     |      The name of the Enum member.
     |  
     |  value
     |      The value of the Enum member.
     |  
     |  ----------------------------------------------------------------------
     |  Data descriptors inherited from enum.EnumMeta:
     |  
     |  \_\_members\_\_
     |      Returns a mapping of member name->value.
     |      
     |      This mapping lists all enum members, including aliases. Note that this
     |      is a read-only view of the internal mapping.
    
    class ShadowMode(JEnum)
     |  该类定义了晕渲图渲染方式类型常量。
     |  
     |  :var ShadowMode.IllUMINATION\_AND\_SHADOW: 渲染和阴影。同时考虑当地的光照角以及阴影的作用。
     |  :var ShadowMode.SHADOW: 阴影。只考虑区域是否位于阴影中。
     |  :var ShadowMode.IllUMINATION: 渲染。只考虑当地的光照角。
     |  
     |  Method resolution order:
     |      ShadowMode
     |      JEnum
     |      enum.IntEnum
     |      builtins.int
     |      enum.Enum
     |      builtins.object
     |  
     |  Data and other attributes defined here:
     |  
     |  IllUMINATION = ShadowMode.IllUMINATION
     |  
     |  IllUMINATION\_AND\_SHADOW = ShadowMode.IllUMINATION\_AND\_SHADOW
     |  
     |  SHADOW = ShadowMode.SHADOW
     |  
     |  ----------------------------------------------------------------------
     |  Data descriptors inherited from enum.Enum:
     |  
     |  name
     |      The name of the Enum member.
     |  
     |  value
     |      The value of the Enum member.
     |  
     |  ----------------------------------------------------------------------
     |  Data descriptors inherited from enum.EnumMeta:
     |  
     |  \_\_members\_\_
     |      Returns a mapping of member name->value.
     |      
     |      This mapping lists all enum members, including aliases. Note that this
     |      is a read-only view of the internal mapping.
    
    class SlopeType(JEnum)
     |  该类定义了坡度的单位类型常量。
     |  
     |  :var SlopeType.DEGREE: 以角度为单位来表示坡度。
     |  :var SlopeType.RADIAN: 以弧度为单位来表示坡度。
     |  :var SlopeType.PERCENT:  以百分数来表示坡度。 该百分数为垂直高度和水平距离的比值乘以100,即单位水平距离上的高度值乘以100, 或者说是坡度的正切值乘以100。
     |  
     |  Method resolution order:
     |      SlopeType
     |      JEnum
     |      enum.IntEnum
     |      builtins.int
     |      enum.Enum
     |      builtins.object
     |  
     |  Data and other attributes defined here:
     |  
     |  DEGREE = SlopeType.DEGREE
     |  
     |  PERCENT = SlopeType.PERCENT
     |  
     |  RADIAN = SlopeType.RADIAN
     |  
     |  ----------------------------------------------------------------------
     |  Data descriptors inherited from enum.Enum:
     |  
     |  name
     |      The name of the Enum member.
     |  
     |  value
     |      The value of the Enum member.
     |  
     |  ----------------------------------------------------------------------
     |  Data descriptors inherited from enum.EnumMeta:
     |  
     |  \_\_members\_\_
     |      Returns a mapping of member name->value.
     |      
     |      This mapping lists all enum members, including aliases. Note that this
     |      is a read-only view of the internal mapping.
    
    class SmoothMethod(JEnum)
     |  该类定义了光滑方法类型常量。用于从 Grid 或 DEM 数据生成等值线或等值面时对等值线或者等值面的边界线进行平滑。
     |  
     |  等值线的生成是通过对原栅格数据进行插值,然后连接等值点得到,所以得到的结果是棱角分明的折线,等值面的生成是通过对原栅格数据进行插值,然后连接等值
     |  点得到等值线,再由相邻等值线封闭组成的,所以得到的结果是棱角分明的多边形面,这两者均需要进行一定的光滑处理,SuperMap 提供两种光滑处理的方法,
     |  B 样条法和磨角法。
     |  
     |  :var SmoothMethod.NONE: 不进行光滑。
     |  :var SmoothMethod.BSPLINE: B 样条法。B 样条法是以一条通过折线中一些节点的 B 样条曲线代替原始折线来达到光滑的目的。B 样条曲线是贝塞尔曲线
     |                             的一种扩展。如下图所示,B 样条曲线不必通过原线对象的所有节点。除经过的原折线上的一些点外,曲线上的其他点通过
     |                             B 样条函数拟合得出。
     |  
     |                             .. image:: ../image/BSpline.png
     |  
     |                             对非闭合的线对象使用 B 样条法后,其两端点的相对位置保持不变。
     |  :var SmoothMethod.POLISH: 磨角法。磨角法是一种运算相对简单,处理速度比较快的光滑方法,但是效果比较局限。它的主要过程是将折线上的两条相邻的
     |                            线段,分别在距离夹角顶点三分之一线段长度处添加节点,将夹角两侧新添加的两节点相连,从而将原线段的节点磨平,故称
     |                            磨角法。下图为进行一次磨角法的过程示意图。
     |  
     |                            .. image:: ../image/Polish.png
     |  
     |                            可以多次磨角以得到接近光滑的线。对非闭合的线对象使用磨角法后,其两端点的相对位置保持不变。
     |  
     |  Method resolution order:
     |      SmoothMethod
     |      JEnum
     |      enum.IntEnum
     |      builtins.int
     |      enum.Enum
     |      builtins.object
     |  
     |  Data and other attributes defined here:
     |  
     |  BSPLINE = SmoothMethod.BSPLINE
     |  
     |  NONE = SmoothMethod.NONE
     |  
     |  POLISH = SmoothMethod.POLISH
     |  
     |  ----------------------------------------------------------------------
     |  Data descriptors inherited from enum.Enum:
     |  
     |  name
     |      The name of the Enum member.
     |  
     |  value
     |      The value of the Enum member.
     |  
     |  ----------------------------------------------------------------------
     |  Data descriptors inherited from enum.EnumMeta:
     |  
     |  \_\_members\_\_
     |      Returns a mapping of member name->value.
     |      
     |      This mapping lists all enum members, including aliases. Note that this
     |      is a read-only view of the internal mapping.
    
    class SpatialIndexType(JEnum)
     |  该类定义了空间索引类型常量。
     |  
     |  空间索引用于提高数据空间查询效率的数据结构。在 SuperMap 中提供了 R 树索引,四叉树索引,图幅索引和多级网格索引。以上几种索引仅适用于矢量数据集。
     |  
     |  同时,一个数据集在一种时刻只能使用一种索引,但是索引可以切换,即当对数据集创建完一种索引之后,必须删除旧的索引才能创建新的。数据集处于编辑状态时,
     |  系统自动维护当前的索引。特别地,当数据被多次编辑后,索引的效率将会受到不同程度的影响,通过系统的判断得知是否要求重新建立空间索引:
     |  
     |  - 当前版本 UDB 和 PostgreSQL 数据源只支持 R 树索引(RTree),DB2 数据源只支持多级网格索引(Multi\_Level\_Grid);
     |  - 数据库中的点数据集均不支持四叉树(QTree)索引和 R 树索引(RTree);
     |  - 网络数据集不支持任何类型的空间索引;
     |  - 复合数据集不支持多级网格索引;
     |  - 路由数据集不支持图幅索引(TILE);
     |  - 属性数据集不支持任何类型的空间索引;
     |  - 对于数据库类型的数据源,数据库记录要大于1000条时才可以创建索引。
     |  
     |  :var SpatialIndexType.NONE: 无空间索引就是没有空间索引,适用于数据量非常小的情况
     |  :var SpatialIndexType.RTREE: R 树索引是基于磁盘的索引结构,是 B 树(一维)在高维空间的自然扩展,易于与现有数据库系统集成,能够支持各种类型
     |                               的空间查询处理操作,在实践中得到了广泛的应用,是目前最流行的空间索引方法之一。R 树空间索引方法是设计一些包含
     |                               空间对象的矩形,将一些空间位置相近的目标对象,包含在这个矩形内,把这些矩形作为空间索引,它含有所包含的空间对
     |                               象的指针。
     |                               注意:
     |  
     |                               - 此索引适合于静态数据(对数据进行浏览、查询操作时)。
     |                               - 此索引支持数据的并发操作。
     |  
     |  :var SpatialIndexType.QTREE: 四叉树是一种重要的层次化数据集结构,主要用来表达二维坐标下空间层次关系,实际上它是一维二叉树在二维空间的扩展。
     |                               那么,四叉树索引就是将一张地图四等分,然后再每一个格子中再四等分,逐层细分,直至不能再分。现在在 SuperMap
     |                               中四叉树最多允许分成13层。基于希尔伯特(Hilbert)编码的排序规则,从四叉树中可确定索引类中每个对象实例的被索
     |                               引属性值是属于哪个最小范围。从而提高了检索效率
     |  :var SpatialIndexType.TILE: 图幅索引。在 SuperMap 中根据数据集的某一属性字段或根据给定的一个范围,将空间对象进行分类,通过索引进行管理已
     |                              分类的空间对象,以此提高查询检索速度
     |  :var SpatialIndexType.MULTI\_LEVEL\_GRID: 多级网格索引,又叫动态索引。多级网格索引结合了 R 树索引与四叉树索引的优点,提供非常好的并发编辑
     |                                          支持,具有很好的普适性。若不能确定数据适用于哪种空间索引,可为其建立多级网格索引。采用划分多层网
     |                                          格的方式来组织管理数据。网格索引的基本方法是将数据集按照一定的规则划分成相等或不相等的网格,记录
     |                                          每一个地理对象所占的网格位置。在 GIS 中常用的是规则网格。当用户进行空间查询时,首先计算出用户查
     |                                          询对象所在的网格,通过该网格快速查询所选地理对象,可以优化查询操作。
     |  
     |                                          当前版本中,定义网格的索引为一级,二级和三级,每一级都有各自的划分规则,第一级的网格最小,第二级
     |                                          和第三级的网格要相应得比前面的大。在建立多级网格索引时,根据具体数据及其分布的情况,网格的大小和
     |                                          网格索引的级数由系统自动给出,不需要用户进行设置。
     |  :var SpatialIndexType.PRIMARY: 原生索引,创建的是空间索引。在PostgreSQL空间数据扩展PostGIS中是GIST索引,意思是通用的搜索树。在SQLServer空间数据扩展SQLSpatial中是多级格网索引:
     |  
     |                                 - PostGIS的GIST索引是一种平衡的,树状结构的访问方法,主要使用了B-tree、R-tree、RD-tree索引算法。优点:适用于多维数据类型和集合数据类型,同样适用于其他的数据类型,GIST多字段索引在查询条件中包含索引字段的任何子集都会使用索引扫描。缺点:GIST索引创建耗时较长,占用空间也比较大。
     |                                 - SQLSpatial的多级格网索引最多可以设置四级,每一级按照等分格网的方式依次进行。在创建索引时,可以选择高、中、低三种格网密度,分别对应(4*4)、(8*8)和(16*16),目前默认中格网密度。
     |  
     |  Method resolution order:
     |      SpatialIndexType
     |      JEnum
     |      enum.IntEnum
     |      builtins.int
     |      enum.Enum
     |      builtins.object
     |  
     |  Data and other attributes defined here:
     |  
     |  MULTI\_LEVEL\_GRID = SpatialIndexType.MULTI\_LEVEL\_GRID
     |  
     |  NONE = SpatialIndexType.NONE
     |  
     |  PRIMARY = SpatialIndexType.PRIMARY
     |  
     |  QTREE = SpatialIndexType.QTREE
     |  
     |  RTREE = SpatialIndexType.RTREE
     |  
     |  TILE = SpatialIndexType.TILE
     |  
     |  ----------------------------------------------------------------------
     |  Data descriptors inherited from enum.Enum:
     |  
     |  name
     |      The name of the Enum member.
     |  
     |  value
     |      The value of the Enum member.
     |  
     |  ----------------------------------------------------------------------
     |  Data descriptors inherited from enum.EnumMeta:
     |  
     |  \_\_members\_\_
     |      Returns a mapping of member name->value.
     |      
     |      This mapping lists all enum members, including aliases. Note that this
     |      is a read-only view of the internal mapping.
    
    class SpatialQueryMode(JEnum)
     |  该类定义了空间查询操作模式类型常量。
     |  空间查询是通过几何对象之间的空间位置关系来构建过滤条件的一种查询方式。例如:通过空间查询可以找到被包含在面中的空间对象,相离或者相邻的空间对象等。
     |  
     |  :var SpatialQueryMode.NONE: 无空间查询
     |  :var SpatialQueryMode.IDENTITY: 重合空间查询模式。返回被搜索图层中与搜索对象完全重合的对象。注意:搜索对象与被搜索对象的类型必须相同;且两个对象的交集不为空,搜索对象的边界及内部分别和被搜索对象的外部交集为空。
     |                                  该关系适合的对象类型:
     |  
     |                                  - 搜索对象:点、线、面;
     |                                  - 被搜索对象:点、线、面。
     |  
     |                                  如图:
     |  
     |                                   .. image:: ../image/SQIdentical.png
     |  
     |  :var SpatialQueryMode.DISJOINT: 分离空间查询模式。返回被搜索图层中与搜索对象相离的对象。注意:搜索对象和被搜索对象相离,即无任何交集。
     |                                  该关系适合的对象类型:
     |  
     |                                  - 搜索对象:点、线、面;
     |                                  - 被搜索对象:点、线、面。
     |  
     |                                   如图:
     |  
     |                                    .. image:: ../image/SQDsjoint.png
     |  
     |  :var SpatialQueryMode.INTERSECT: 相交空间查询模式。返回与搜索对象相交的所有对象。注意:如果搜索对象是面,返回全部或部分被搜索对象包含的对象以及全部或部分包含搜索对象的对象;如果搜索对象不是面,返回全部或部分包含搜索对象的对象。
     |                                   该关系适合的对象类型:
     |  
     |                                   - 搜索对象:点、线、面;
     |                                   - 被搜索对象:点、线、面。
     |  
     |                                   如图:
     |  
     |                                    .. image:: ../image/SQIntersect.png
     |  
     |  :var SpatialQueryMode.TOUCH: 邻接空间查询模式。返回被搜索图层中其边界与搜索对象边界相触的对象。注意:搜索对象和被搜索对象的内部交集为空。
     |                               该关系不适合的对象类型为:
     |  
     |                               - 点查询点的空间关系。
     |  
     |                               如图:
     |  
     |                                .. image:: ../image/SQTouch.png
     |  
     |  :var SpatialQueryMode.OVERLAP: 叠加空间查询模式。返回被搜索图层中与搜索对象部分重叠的对象。
     |                                 该关系适合的对象类型为:
     |  
     |                                 - 线/线,面/面。其中,两个几何对象的维数必须一致,而且他们交集的维数也应该和几何对象的维数一样
     |  
     |                                 注意:点与任何一种几何对象都不存在部分重叠的情况
     |  
     |                                 如图:
     |  
     |                                  .. image:: ../image/SQOverlap.png
     |  
     |  :var SpatialQueryMode.CROSS: 交叉空间查询模式。返回被搜索图层中与搜索对象(线)相交的所有对象(线或面)。注意:搜索对象和被搜索对象内部的交集不能为空;参与交叉(Cross)关系运算的两个对象必须有一个是线对象。
     |                               该关系适合的对象类型:
     |  
     |                               - 搜索对象:线;
     |                               - 被搜索对象:线、面。
     |  
     |                               如图:
     |  
     |                                .. image:: ../image/SQCross.png
     |  
     |  :var SpatialQueryMode.WITHIN: 被包含空间查询模式。返回被搜索图层中完全包含搜索对象的对象。如果返回的对象是面,其必须全部包含(包括边接触)搜索对象;如果返回的对象是线,其必须完全包含搜索对象;如果返回的对象是点,其必须与搜索对象重合。该类型与包含(Contain)的查询模式正好相反。
     |                                该关系适合的对象类型:
     |  
     |                                - 搜索对象: 点、线、面;
     |                                - 被搜索对象: 点、线、面。
     |  
     |                                如图:
     |  
     |                                 .. image:: ../image/SQWithin.png
     |  
     |  :var SpatialQueryMode.CONTAIN: 包含空间查询模式。返回被搜索图层中完全被搜索对象包含的对象。注:搜索对象和被搜索对象的边界交集可以不为空;点查线/点查面/线查面,不存在包含情况。
     |                                 该关系适合的对象类型:
     |  
     |                                 - 搜索对象:点、线、面;
     |                                 - 被搜索对象:点、线、面。
     |  
     |                                 如图:
     |  
     |                                  .. image:: ../image/SQContain.png
     |  
     |  :var SpatialQueryMode.INNERINTERSECT: 内部相交查询模式,返回与搜索对象相交但不是仅接触的所有对象。也就是在相交算子的结果之上排除所有接触算子的结果。
     |  
     |  Method resolution order:
     |      SpatialQueryMode
     |      JEnum
     |      enum.IntEnum
     |      builtins.int
     |      enum.Enum
     |      builtins.object
     |  
     |  Data and other attributes defined here:
     |  
     |  CONTAIN = SpatialQueryMode.CONTAIN
     |  
     |  CROSS = SpatialQueryMode.CROSS
     |  
     |  DISJOINT = SpatialQueryMode.DISJOINT
     |  
     |  IDENTITY = SpatialQueryMode.IDENTITY
     |  
     |  INNERINTERSECT = SpatialQueryMode.INNERINTERSECT
     |  
     |  INTERSECT = SpatialQueryMode.INTERSECT
     |  
     |  NONE = SpatialQueryMode.NONE
     |  
     |  OVERLAP = SpatialQueryMode.OVERLAP
     |  
     |  TOUCH = SpatialQueryMode.TOUCH
     |  
     |  WITHIN = SpatialQueryMode.WITHIN
     |  
     |  ----------------------------------------------------------------------
     |  Data descriptors inherited from enum.Enum:
     |  
     |  name
     |      The name of the Enum member.
     |  
     |  value
     |      The value of the Enum member.
     |  
     |  ----------------------------------------------------------------------
     |  Data descriptors inherited from enum.EnumMeta:
     |  
     |  \_\_members\_\_
     |      Returns a mapping of member name->value.
     |      
     |      This mapping lists all enum members, including aliases. Note that this
     |      is a read-only view of the internal mapping.
    
    class SpatialStatisticsType(JEnum)
     |  数据集进行空间度量后的字段统计类型常量
     |  
     |  :var SpatialStatisticsType.MAX: 统计字段的最大值。只对数值型字段有效。
     |  :var SpatialStatisticsType.MIN: 统计字段的最小值。只对数值型字段有效。
     |  :var SpatialStatisticsType.SUM: 统计字段的和。只对数值型字段有效。
     |  :var SpatialStatisticsType.MEAN: 统计字段的平均值。只对数值型字段有效。
     |  :var SpatialStatisticsType.FIRST: 保留第一个对象的字段值。对数值、布尔、时间和文本型字段都有效。
     |  :var SpatialStatisticsType.LAST: 保留最后一个对象的字段值。对数值、布尔、时间和文本型字段都有效。
     |  :var SpatialStatisticsType.MEDIAN: 统计字段的中位数。只对数值型字段有效。
     |  
     |  Method resolution order:
     |      SpatialStatisticsType
     |      JEnum
     |      enum.IntEnum
     |      builtins.int
     |      enum.Enum
     |      builtins.object
     |  
     |  Data and other attributes defined here:
     |  
     |  FIRST = SpatialStatisticsType.FIRST
     |  
     |  LAST = SpatialStatisticsType.LAST
     |  
     |  MAX = SpatialStatisticsType.MAX
     |  
     |  MEAN = SpatialStatisticsType.MEAN
     |  
     |  MEDIAN = SpatialStatisticsType.MEDIAN
     |  
     |  MIN = SpatialStatisticsType.MIN
     |  
     |  SUM = SpatialStatisticsType.SUM
     |  
     |  ----------------------------------------------------------------------
     |  Data descriptors inherited from enum.Enum:
     |  
     |  name
     |      The name of the Enum member.
     |  
     |  value
     |      The value of the Enum member.
     |  
     |  ----------------------------------------------------------------------
     |  Data descriptors inherited from enum.EnumMeta:
     |  
     |  \_\_members\_\_
     |      Returns a mapping of member name->value.
     |      
     |      This mapping lists all enum members, including aliases. Note that this
     |      is a read-only view of the internal mapping.
    
    class StatisticMode(JEnum)
     |  该类定义了字段统计方法类型常量。对单一字段提供常用统计功能。SuperMap 提供的统计功能有6种,统计字段的最大值,最小值,平均值,总和,标准差以及方差。
     |  
     |  :var StatisticMode.MAX: 统计所选字段的最大值。
     |  :var StatisticMode.MIN: 统计所选字段的最小值。
     |  :var StatisticMode.AVERAGE: 统计所选字段的平均值。
     |  :var StatisticMode.SUM: 统计所选字段的总和。
     |  :var StatisticMode.STDDEVIATION: 统计所选字段的标准差。
     |  :var StatisticMode.VARIANCE: 统计所选字段的方差。
     |  
     |  Method resolution order:
     |      StatisticMode
     |      JEnum
     |      enum.IntEnum
     |      builtins.int
     |      enum.Enum
     |      builtins.object
     |  
     |  Data and other attributes defined here:
     |  
     |  AVERAGE = StatisticMode.AVERAGE
     |  
     |  MAX = StatisticMode.MAX
     |  
     |  MIN = StatisticMode.MIN
     |  
     |  STDDEVIATION = StatisticMode.STDDEVIATION
     |  
     |  SUM = StatisticMode.SUM
     |  
     |  VARIANCE = StatisticMode.VARIANCE
     |  
     |  ----------------------------------------------------------------------
     |  Data descriptors inherited from enum.Enum:
     |  
     |  name
     |      The name of the Enum member.
     |  
     |  value
     |      The value of the Enum member.
     |  
     |  ----------------------------------------------------------------------
     |  Data descriptors inherited from enum.EnumMeta:
     |  
     |  \_\_members\_\_
     |      Returns a mapping of member name->value.
     |      
     |      This mapping lists all enum members, including aliases. Note that this
     |      is a read-only view of the internal mapping.
    
    class StatisticsCompareType(JEnum)
     |  比较类型常量
     |  
     |  :var StatisticsCompareType.LESS: 小于。
     |  :var StatisticsCompareType.LESS\_OR\_EQUAL: 小于或等于。
     |  :var StatisticsCompareType.EQUAL: 等于。
     |  :var StatisticsCompareType.GREATER: 大于。
     |  :var StatisticsCompareType.GREATER\_OR\_EQUAL: 大于或等于。
     |  
     |  Method resolution order:
     |      StatisticsCompareType
     |      JEnum
     |      enum.IntEnum
     |      builtins.int
     |      enum.Enum
     |      builtins.object
     |  
     |  Data and other attributes defined here:
     |  
     |  EQUAL = StatisticsCompareType.EQUAL
     |  
     |  GREATER = StatisticsCompareType.GREATER
     |  
     |  GREATER\_OR\_EQUAL = StatisticsCompareType.GREATER\_OR\_EQUAL
     |  
     |  LESS = StatisticsCompareType.LESS
     |  
     |  LESS\_OR\_EQUAL = StatisticsCompareType.LESS\_OR\_EQUAL
     |  
     |  ----------------------------------------------------------------------
     |  Data descriptors inherited from enum.Enum:
     |  
     |  name
     |      The name of the Enum member.
     |  
     |  value
     |      The value of the Enum member.
     |  
     |  ----------------------------------------------------------------------
     |  Data descriptors inherited from enum.EnumMeta:
     |  
     |  \_\_members\_\_
     |      Returns a mapping of member name->value.
     |      
     |      This mapping lists all enum members, including aliases. Note that this
     |      is a read-only view of the internal mapping.
    
    class StatisticsFieldType(JEnum)
     |  点抽稀的统计类型,统计的是抽稀点原始点集的值
     |  
     |  :var StatisticsFieldType.AVERAGE:  统计平均值
     |  :var StatisticsFieldType.SUM: 统计和
     |  :var StatisticsFieldType.MAXVALUE: 最大值
     |  :var StatisticsFieldType.MINVALUE:  最小值
     |  :var StatisticsFieldType.VARIANCE: 方差
     |  :var StatisticsFieldType.SAMPLEVARIANCE: 样本方差
     |  :var StatisticsFieldType.STDDEVIATION: 标准差
     |  :var StatisticsFieldType.SAMPLESTDDEV: 样本标准差
     |  
     |  Method resolution order:
     |      StatisticsFieldType
     |      JEnum
     |      enum.IntEnum
     |      builtins.int
     |      enum.Enum
     |      builtins.object
     |  
     |  Data and other attributes defined here:
     |  
     |  AVERAGE = StatisticsFieldType.AVERAGE
     |  
     |  MAXVALUE = StatisticsFieldType.MAXVALUE
     |  
     |  MINVALUE = StatisticsFieldType.MINVALUE
     |  
     |  SAMPLESTDDEV = StatisticsFieldType.SAMPLESTDDEV
     |  
     |  SAMPLEVARIANCE = StatisticsFieldType.SAMPLEVARIANCE
     |  
     |  STDDEVIATION = StatisticsFieldType.STDDEVIATION
     |  
     |  SUM = StatisticsFieldType.SUM
     |  
     |  VARIANCE = StatisticsFieldType.VARIANCE
     |  
     |  ----------------------------------------------------------------------
     |  Data descriptors inherited from enum.Enum:
     |  
     |  name
     |      The name of the Enum member.
     |  
     |  value
     |      The value of the Enum member.
     |  
     |  ----------------------------------------------------------------------
     |  Data descriptors inherited from enum.EnumMeta:
     |  
     |  \_\_members\_\_
     |      Returns a mapping of member name->value.
     |      
     |      This mapping lists all enum members, including aliases. Note that this
     |      is a read-only view of the internal mapping.
    
    class StatisticsType(JEnum)
     |  字段统计类型常量
     |  
     |  :var StatisticsType.MAX: 统计字段的最大值
     |  :var StatisticsType.MIN: 统计字段的最小值
     |  :var StatisticsType.SUM: 统计字段的和
     |  :var StatisticsType.MEAN: 统计字段的平均值
     |  :var StatisticsType.FIRST: 保留第一个对象的字段值
     |  :var StatisticsType.LAST: 保留最后一个对象的字段值。
     |  
     |  Method resolution order:
     |      StatisticsType
     |      JEnum
     |      enum.IntEnum
     |      builtins.int
     |      enum.Enum
     |      builtins.object
     |  
     |  Data and other attributes defined here:
     |  
     |  FIRST = StatisticsType.FIRST
     |  
     |  LAST = StatisticsType.LAST
     |  
     |  MAX = StatisticsType.MAX
     |  
     |  MEAN = StatisticsType.MEAN
     |  
     |  MIN = StatisticsType.MIN
     |  
     |  SUM = StatisticsType.SUM
     |  
     |  ----------------------------------------------------------------------
     |  Data descriptors inherited from enum.Enum:
     |  
     |  name
     |      The name of the Enum member.
     |  
     |  value
     |      The value of the Enum member.
     |  
     |  ----------------------------------------------------------------------
     |  Data descriptors inherited from enum.EnumMeta:
     |  
     |  \_\_members\_\_
     |      Returns a mapping of member name->value.
     |      
     |      This mapping lists all enum members, including aliases. Note that this
     |      is a read-only view of the internal mapping.
    
    class StreamOrderType(JEnum)
     |  流域水系编号(即河流分级)方法类型常量
     |  
     |  :var StreamOrderType.STRAHLER: Strahler 河流分级法。Strahler 河流分级法由 Strahler 于 1957 年提出。其规则定义为:直接发
     |                                 源于河源的河流为 1 级河流;同级的两条河流交汇形成的河流的等级比原来增加 1 级;不同等级的两
     |                                 条河流交汇形成的河流的级等于原来河流中级等较高者。
     |  
     |                                 .. image:: ../image/Strahler.png
     |  
     |  :var StreamOrderType.SHREVE: Shreve 河流分级法。Shreve 河流分级法由 Shreve 于 1966 年提出。其规则定义为:直接发源于河源
     |                               的河流等级为 1 级,两条河流交汇形成的河流的等级为两条河流等级的和。例如,两条 1 级河流交汇形
     |                               成 2 级河流,一条 2 级河流和一条 3 级河流交汇形成一条 5 级河流。
     |  
     |                               .. image:: ../image/Shreve.png
     |  
     |  Method resolution order:
     |      StreamOrderType
     |      JEnum
     |      enum.IntEnum
     |      builtins.int
     |      enum.Enum
     |      builtins.object
     |  
     |  Data and other attributes defined here:
     |  
     |  SHREVE = StreamOrderType.SHREVE
     |  
     |  STRAHLER = StreamOrderType.STRAHLER
     |  
     |  ----------------------------------------------------------------------
     |  Data descriptors inherited from enum.Enum:
     |  
     |  name
     |      The name of the Enum member.
     |  
     |  value
     |      The value of the Enum member.
     |  
     |  ----------------------------------------------------------------------
     |  Data descriptors inherited from enum.EnumMeta:
     |  
     |  \_\_members\_\_
     |      Returns a mapping of member name->value.
     |      
     |      This mapping lists all enum members, including aliases. Note that this
     |      is a read-only view of the internal mapping.
    
    class StringAlignment(JEnum)
     |  该类定义了多行文本排版方式类型常量
     |  
     |  :var StringAlignment.LEFT:  左对齐
     |  :var StringAlignment.CENTER: 居中对齐
     |  :var StringAlignment.RIGHT: 右对齐
     |  :var StringAlignment.DISTRIBUTED: 分散对齐(两端对齐)
     |  
     |  Method resolution order:
     |      StringAlignment
     |      JEnum
     |      enum.IntEnum
     |      builtins.int
     |      enum.Enum
     |      builtins.object
     |  
     |  Data and other attributes defined here:
     |  
     |  CENTER = StringAlignment.CENTER
     |  
     |  DISTRIBUTED = StringAlignment.DISTRIBUTED
     |  
     |  LEFT = StringAlignment.LEFT
     |  
     |  RIGHT = StringAlignment.RIGHT
     |  
     |  ----------------------------------------------------------------------
     |  Data descriptors inherited from enum.Enum:
     |  
     |  name
     |      The name of the Enum member.
     |  
     |  value
     |      The value of the Enum member.
     |  
     |  ----------------------------------------------------------------------
     |  Data descriptors inherited from enum.EnumMeta:
     |  
     |  \_\_members\_\_
     |      Returns a mapping of member name->value.
     |      
     |      This mapping lists all enum members, including aliases. Note that this
     |      is a read-only view of the internal mapping.
    
    class TerrainInterpolateType(JEnum)
     |  地形插值类型常量
     |  
     |  :var TerrainInterpolateType.IDW: 距离反比权值插值法。参考 :py:attr:`.InterpolationAlgorithmType.IDW`
     |  :var TerrainInterpolateType.KRIGING: 克吕金内插法。参考 :py:attr:`.InterpolationAlgorithmType.KRIGING`
     |  :var TerrainInterpolateType.TIN: 不规则三角网。先将给定的线数据集生成一个TIN模型,然后根据给定的极值点信息以及湖信息生成DEM模型。
     |  
     |  Method resolution order:
     |      TerrainInterpolateType
     |      JEnum
     |      enum.IntEnum
     |      builtins.int
     |      enum.Enum
     |      builtins.object
     |  
     |  Data and other attributes defined here:
     |  
     |  IDW = TerrainInterpolateType.IDW
     |  
     |  KRIGING = TerrainInterpolateType.KRIGING
     |  
     |  TIN = TerrainInterpolateType.TIN
     |  
     |  ----------------------------------------------------------------------
     |  Data descriptors inherited from enum.Enum:
     |  
     |  name
     |      The name of the Enum member.
     |  
     |  value
     |      The value of the Enum member.
     |  
     |  ----------------------------------------------------------------------
     |  Data descriptors inherited from enum.EnumMeta:
     |  
     |  \_\_members\_\_
     |      Returns a mapping of member name->value.
     |      
     |      This mapping lists all enum members, including aliases. Note that this
     |      is a read-only view of the internal mapping.
    
    class TerrainStatisticType(JEnum)
     |  地形统计类型常量
     |  
     |  :var TerrainStatisticType.UNIQUE: 去重复点统计。
     |  :var TerrainStatisticType.MEAN: 平均数统计。
     |  :var TerrainStatisticType.MIN: 最小值统计。
     |  :var TerrainStatisticType.MAX: 最大值统计。
     |  :var TerrainStatisticType.MAJORITY: 众数指的是出现频率最高的栅格值。目前只用于栅格分带统计。
     |  :var TerrainStatisticType.MEDIAN: 中位数指的是按栅格值从小到大排列,位于中间位置的栅格值。目前只用于栅格分带统计。
     |  
     |  Method resolution order:
     |      TerrainStatisticType
     |      JEnum
     |      enum.IntEnum
     |      builtins.int
     |      enum.Enum
     |      builtins.object
     |  
     |  Data and other attributes defined here:
     |  
     |  MAJORITY = TerrainStatisticType.MAJORITY
     |  
     |  MAX = TerrainStatisticType.MAX
     |  
     |  MEAN = TerrainStatisticType.MEAN
     |  
     |  MEDIAN = TerrainStatisticType.MEDIAN
     |  
     |  MIN = TerrainStatisticType.MIN
     |  
     |  UNIQUE = TerrainStatisticType.UNIQUE
     |  
     |  ----------------------------------------------------------------------
     |  Data descriptors inherited from enum.Enum:
     |  
     |  name
     |      The name of the Enum member.
     |  
     |  value
     |      The value of the Enum member.
     |  
     |  ----------------------------------------------------------------------
     |  Data descriptors inherited from enum.EnumMeta:
     |  
     |  \_\_members\_\_
     |      Returns a mapping of member name->value.
     |      
     |      This mapping lists all enum members, including aliases. Note that this
     |      is a read-only view of the internal mapping.
    
    class TextAlignment(JEnum)
     |  该类定义了文本对齐类型常量。
     |  
     |  指定文本中的各子对象的对齐方式。文本对象的每个子对象的位置是由文本的锚点和文本的对齐方式共同确定的。当文本子对象的锚点固定,对齐方式确定文本子对象与锚点的相对位置,从而确定文本子对象的位置。
     |  
     |  :var TextAlignment.TOPLEFT: 左上角对齐。当文本的对齐方式为左上角对齐时,文本子对象的最小外接矩形的左上角点在该文本子对象的锚点位置
     |  :var TextAlignment.TOPCENTER: 顶部居中对齐。当文本的对齐方式为上面居中对齐时,文本子对象的最小外接矩形的上边线的中点在该文本子对象的锚点位置
     |  :var TextAlignment.TOPRIGHT: 右上角对齐。当文本的对齐方式为右上角对齐时,文本子对象的最小外接矩形的右上角点在该文本子对象的锚点位置
     |  :var TextAlignment.BASELINELEFT: 基准线左对齐。当文本的对齐方式为基准线左对齐时,文本子对象的基线的左端点在该文本子对象的锚点位置
     |  :var TextAlignment.BASELINECENTER: 基准线居中对齐。当文本的对齐方式为基准线居中对齐时,文本子对象的基线的中点在该文本子对象的锚点位置
     |  :var TextAlignment.BASELINERIGHT: 基准线右对齐。当文本的对齐方式为基准线右对齐时,文本子对象的基线的右端点在该文本子对象的锚点位置
     |  :var TextAlignment.BOTTOMLEFT: 左下角对齐。当文本的对齐方式为左下角对齐时,文本子对象的最小外接矩形的左下角点在该文本子对象的锚点位置
     |  :var TextAlignment.BOTTOMCENTER: 底部居中对齐。当文本的对齐方式为底线居中对齐时,文本子对象的最小外接矩形的底线的中点在该文本子对象的锚点位置
     |  :var TextAlignment.BOTTOMRIGHT: 右下角对齐。当文本的对齐方式为右下角对齐时,文本子对象的最小外接矩形的右下角点在该文本子对象的锚点位置
     |  :var TextAlignment.MIDDLELEFT: 左中对齐。当文本的对齐方式为左中对齐时,文本子对象的最小外接矩形的左边线的中点在该文本子对象的锚点位置
     |  :var TextAlignment.MIDDLECENTER: 中心对齐。当文本的对齐方式为中心对齐时,文本子对象的最小外接矩形的中心点在该文本子对象的锚点位置
     |  :var TextAlignment.MIDDLERIGHT: 右中对齐。当文本的对齐方式为右中对齐时,文本子对象的最小外接矩形的右边线的中点在该文本子对象的锚点位置
     |  
     |  Method resolution order:
     |      TextAlignment
     |      JEnum
     |      enum.IntEnum
     |      builtins.int
     |      enum.Enum
     |      builtins.object
     |  
     |  Data and other attributes defined here:
     |  
     |  BASELINECENTER = TextAlignment.BASELINECENTER
     |  
     |  BASELINELEFT = TextAlignment.BASELINELEFT
     |  
     |  BASELINERIGHT = TextAlignment.BASELINERIGHT
     |  
     |  BOTTOMCENTER = TextAlignment.BOTTOMCENTER
     |  
     |  BOTTOMLEFT = TextAlignment.BOTTOMLEFT
     |  
     |  BOTTOMRIGHT = TextAlignment.BOTTOMRIGHT
     |  
     |  MIDDLECENTER = TextAlignment.MIDDLECENTER
     |  
     |  MIDDLELEFT = TextAlignment.MIDDLELEFT
     |  
     |  MIDDLERIGHT = TextAlignment.MIDDLERIGHT
     |  
     |  TOPCENTER = TextAlignment.TOPCENTER
     |  
     |  TOPLEFT = TextAlignment.TOPLEFT
     |  
     |  TOPRIGHT = TextAlignment.TOPRIGHT
     |  
     |  ----------------------------------------------------------------------
     |  Data descriptors inherited from enum.Enum:
     |  
     |  name
     |      The name of the Enum member.
     |  
     |  value
     |      The value of the Enum member.
     |  
     |  ----------------------------------------------------------------------
     |  Data descriptors inherited from enum.EnumMeta:
     |  
     |  \_\_members\_\_
     |      Returns a mapping of member name->value.
     |      
     |      This mapping lists all enum members, including aliases. Note that this
     |      is a read-only view of the internal mapping.
    
    class TopologyRule(JEnum)
     |  该类定义了拓扑规则类型常量。
     |  
     |  该类主要用于对点、线和面数据进行拓扑检查,是拓扑检查的一个参数。根据相应的拓扑规则,返回不符合规则的对象。
     |  
     |  :var TopologyRule.REGION\_NO\_OVERLAP: 面内无重叠,用于对面数据进行拓扑检查。检查一个面数据集(或者面记录集)中相互有重叠的面对象。此规则
     |                                       多用于一个区域不能同时属于两个对象的情况。如行政区划面,相邻的区划之间要求不能有任何重叠,行政区划
     |                                       数据上必须是每个区域都有明确的地域定义。此类数据还包括:土地利用图斑、邮政编码覆盖区划、公民投票选
     |                                       区区划等。重叠部分作为错误生成到结果数据集中,错误数据集类型:面。注意:只对一个数据集或记录集本身进行检查。
     |  :var TopologyRule.REGION\_NO\_GAPS: 面内无缝隙,用于对面数据进行拓扑检查。返回一个面数据集(或者面记录集)中相邻面之间有空隙的面对象。此规
     |                                    则多用于检查一个面数据中,单个区域或相邻区域之间有空隙的情况。一般对于如土地利用图斑这样的数据,要求不
     |                                    能有未定义土地利用类型的斑块,可使用此规则进行检查。
     |  
     |                                    注意:
     |  
     |                                    - 只对一个数据集或记录集本身进行检查。
     |                                    - 若被检查的面数据集(或记录集)中存在自相交面对象,则检查可能失败或结果错误。建议检查时,先进行
     |                                      “面内无自相交”(REGION\_NO\_SELF\_INTERSECTION)规则检查,或自行对自相交面进行修改,确认无自相交
     |                                      对象后再进行“面内无缝隙”规则检查。
     |  
     |  :var TopologyRule.REGION\_NO\_OVERLAP\_WITH: 面与面无重叠,用于对面数据进行拓扑检查。检查两个面数据集中重叠的所有对象。此规则检查第一个面数
     |                                            据中,与第二个面数据有重叠的所有对象。如将水域数据与旱地数据叠加,可用此规则检查。重叠部分作为
     |                                            错误生成到结果数据集中,错误数据集类型:面
     |  :var TopologyRule.REGION\_COVERED\_BY\_REGION\_CLASS: 面被多个面覆盖,用于对面数据进行拓扑检查。检查第一个面数据集(或者面记录集)中没有
     |                                                    被第二个面数据集(或者面记录集)覆盖的对象。如:省界 Area1 中每一个省域都必须完全
     |                                                    被县界 Area2 中属于该省的面域所覆盖。未覆盖的部分将作为错误生成到结果数据集中,错误数据集类型:面
     |  
     |  :var TopologyRule.REGION\_COVERED\_BY\_REGION: 面被面包含,用于对面数据进行拓扑检查。
     |                                              检查第一个面数据集(或者面记录集)中没有被第二个面数据集(或者面记录集)中任何对象包含的对象。即面数据 1 的区域都必须被面数据 2 的某一个区域完全包含。
     |                                              未被包含的面对象整个将作为错误生成到结果数据集中,错误数据集类型:面。
     |  :var TopologyRule.REGION\_BOUNDARY\_COVERED\_BY\_LINE: 面边界被多条线覆盖,用于对面数据进行拓扑检查。
     |                                                     检查面数据集(或者面记录集)中对象的边界没有被线数据集(或者线记录集)中的线覆盖的对象。
     |                                                     通常用于行政区界或地块和存储有边界线属性的线数据进行检查。面数据中不能存储一些边界线的属性,
     |                                                     此时需要专门的边界线数据,存储区域边界的不同属性,要求边界线与区域边界线完全重合。
     |                                                     未被覆盖的边界将作为错误生成到结果数据集中,错误数据集类型:线。
     |  :var TopologyRule.REGION\_BOUNDARY\_COVERED\_BY\_REGION\_BOUNDARY: 面边界被边界覆盖,用于对面数据进行拓扑检查。
     |                                                                检查面数据集(或者面记录集)中边界没有被另一面数据集(或者面记录集)中对象(可以为多个)的边界覆盖的对象。
     |                                                                未被覆盖的边界将作为错误生成到结果数据集中,错误数据集类型:线。
     |  :var TopologyRule.REGION\_CONTAIN\_POINT: 面包含点,用于对面数据进行拓扑检查。
     |                                          检查面数据集(或者面记录集)中没有包含任何点数据集(或者点记录集)中点的对象。例如省域数据与省会数据进行检查,每个省内都必须有一个省会城市,不包含任何点数据的面对象,都将被检查出来。
     |                                          未包含点的面对象将作为错误生成到结果数据集中,错误数据集类型:面。
     |  :var TopologyRule.LINE\_NO\_INTERSECTION: 线内无相交,用于对线数据进行拓扑检查。
     |                                          检查一个线数据集(或者线记录集)中相互有相交(不包括端点和内部接触及端点和端点接触)的线对象。交点将作为错误生成到结果数据集中,错误数据集类型:点。
     |                                          注意:只对一个数据集或记录集本身进行检查。
     |  :var TopologyRule.LINE\_NO\_OVERLAP: 线内无重叠,用于对线数据进行拓扑检查。检查一个线数据集(或者线记录集)中相互有重叠的线对象。对象之间重叠的部分将作为错误生成到结果数据集中,错误数据集类型:线。
     |                                     注意:只对一个数据集或记录集本身进行检查。
     |  :var TopologyRule.LINE\_NO\_DANGLES: 线内无悬线,用于对线数据进行拓扑检查。检查一个线数据集(或者线记录集)中被定义为悬线的对象,包括过头线和长悬线。悬点将作为错误生成到结果数据集中,错误数据集类型:点。
     |                                     注意:只对一个数据集或记录集本身进行检查。
     |  :var TopologyRule.LINE\_NO\_PSEUDO\_NODES: 线内无假结点,用于对线数据进行拓扑检查。返回一个线数据集(或者线记录集)中包含假结点的线对象。假结点将作为错误生成到结果数据集中,错误数据集类型:点。
     |                                          注意:只对一个数据集或记录集本身进行检查。
     |  :var TopologyRule.LINE\_NO\_OVERLAP\_WITH: 线与线无重叠,用于对线数据进行拓扑检查。检查第一个线数据集(或者线记录集)中和第二个线数据集(或者线记录集)中的对象有重叠的所有对象。如交通路线中的公路和铁路不能出现重叠。
     |                                          重叠部分作为错误生成到结果数据集中,错误数据集类型:线。
     |  :var TopologyRule.LINE\_NO\_INTERSECT\_OR\_INTERIOR\_TOUCH: 线内无相交或无内部接触,用于对线数据进行拓扑检查。返回一个线数据集(或者线记录集)中和其它线对象相交的线对象,即除端点之间接触外其它所有的相交或内部接触的线对象。
     |                                                         交点作为错误生成到结果数据集中,错误数据集类型:点。
     |                                                         注意:线数据集(或者线记录集)中所有交点必须是线的端点,即相交的弧段必须被打断,否则就违反此规则(自交不检查)。
     |  :var TopologyRule.LINE\_NO\_SELF\_OVERLAP: 线内无自交叠,用于对线数据进行拓扑检查。检查一个线数据集(或者线记录集)内相互有交叠(交集是线)的线对象。自交叠部分(线)将作为错误生成到结果数据集中,错误数据集类型:线。
     |                                          注意:只对一个数据集或记录集本身进行检查。
     |  :var TopologyRule.LINE\_NO\_SELF\_INTERSECT: 线内无自相交,用于对线数据进行拓扑检查。检查一个线数据集(或者线记录集)内自相交的线对象(包括自交叠的情况)。
     |                                            交点将作为错误生成到结果数据集中,错误数据集类型:点。
     |                                            注意:只对一个数据集或记录集本身进行检查。
     |  :var TopologyRule.LINE\_BE\_COVERED\_BY\_LINE\_CLASS: 线被多条线完全覆盖,用于对线数据进行拓扑检查。
     |                                                   检查第一个线数据集(或者线记录集)中没有与第二个线数据集(或者线记录集)中的对象有重合的对象。
     |                                                   未被覆盖的部分将作为错误生成到结果数据集中,错误数据集类型:线。
     |                                                   注意:线数据集(或线记录集)中每一个对象,都必须被另一个线数据集(或者线记录集)中的一个或多个线对象覆盖。如Line1中的某条公交路线必须被Line2中的一系列相连的街道覆盖。
     |  :var TopologyRule.LINE\_COVERED\_BY\_REGION\_BOUNDARY: 线被面边界覆盖,用于对线数据进行拓扑检查。检查线数据集(或者线记录集)中没有与面数据集(或者面记录集)中某个对象的边界重合的对象。(可被多个面的边界覆盖)。
     |                                                     未被边界覆盖的部分将作为错误生成到结果数据集中,错误数据集类型:线。
     |  :var TopologyRule.LINE\_END\_POINT\_COVERED\_BY\_POINT: 线端点必须被点覆盖,用于对线数据进行拓扑检查。
     |                                                     检查线数据集(或者线记录集)中的端点没有与点数据集(或者点记录集)中任何一个点重合的对象。
     |                                                     未被覆盖的端点将作为错误生成到结果数据集中,错误数据集类型:点。
     |  :var TopologyRule.POINT\_COVERED\_BY\_LINE: 点必须在线上,用于对点数据进行拓扑检查。
     |                                           返回点数据集(或者点记录集)中没有被线数据集(或者线记录集)中的某个对象覆盖的对象。如高速公路上的收费站。
     |                                           未被覆盖的点将作为错误生成到结果数据集中,错误数据集类型:点。
     |  :var TopologyRule.POINT\_COVERED\_BY\_REGION\_BOUNDARY: 点必须在面的边界上,用于对点数据进行拓扑检查。
     |                                                      检查点数据集(或者点记录集)中没有在面数据集(或者面记录集)中某个对象的边界上的对象。
     |                                                      不在面边界上的点将作为错误生成到结果数据集中,错误数据集类型:点。
     |  :var TopologyRule.POINT\_CONTAINED\_BY\_REGION: 点被面完全包含,用于对点数据进行拓扑检查。
     |                                               检查点数据集(或者点记录集)中没有被面数据集(或者面记录集)中任何一个对象内部包含的点对象。
     |                                               不在面内的点将作为错误生成到结果数据集中,错误数据集类型:点。
     |  :var TopologyRule.POINT\_BECOVERED\_BY\_LINE\_END\_POINT: 点必须被线端点覆盖,用于对点数据进行拓扑检查。
     |                                                       返回点数据集(或者点记录集)中没有被线数据集(或者线记录集)中任意对象的端点覆盖的对象。
     |  :var TopologyRule.NO\_MULTIPART: 无复杂对象。检查一个数据集或记录集内包含子对象的复杂对象,适用于面和线。
     |                                  复杂对象将作为错误生成到结果数据集中,错误数据集类型:线或面。
     |  :var TopologyRule.POINT\_NO\_IDENTICAL: 无重复点,用于对点数据进行拓扑检查。检查点数据集中的重复点对象。点数据集内发生重叠的对象都将作为拓扑错误生成。
     |                                        所有重复的点将作为错误生成到结果数据集中,错误数据集类型:点。
     |                                        注意:只对一个数据集或记录集本身进行检查。
     |  :var TopologyRule.POINT\_NO\_CONTAINED\_BY\_REGION: 点不被面包含。检查点数据集(或者点记录集)中被面数据集(或者面记录集)中某一个对象内部包含的点对象。
     |                                                  被面包含的点将作为错误生成到结果数据集中,错误数据集类型:点。
     |                                                  注意:点若位于面边界上,则不违背此规则。
     |  :var TopologyRule.LINE\_NO\_INTERSECTION\_WITH\_REGION: 线不能和面相交或被包含。检查线数据集(或者线记录集)中和面数据集(或者面记录集)中的面对象相交或者被面对象包含的线对象。
     |                                                      线面交集部分将作为错误生成到结果数据集中,错误数据集类型:线。
     |  :var TopologyRule.REGION\_NO\_OVERLAP\_ON\_BOUNDARY: 面边界无交叠,用于对面数据进行拓扑检查。
     |                                                   检查面数据集或记录集中的面对象的边界与另一面数据集或记录集中的对象边界有交叠的部分。
     |                                                   边界重叠的部分将作为错误生成到结果数据集中,错误数据集类型:线。
     |  :var TopologyRule.REGION\_NO\_SELF\_INTERSECTION: 面内无自相交,用于对面数据进行拓扑检查。
     |                                                 检查面数据中是否存在自相交的对象。
     |                                                 面对象自相交的交点将作为错误生成到结果数据集中,错误数据集类型:点。
     |  :var TopologyRule.LINE\_NO\_INTERSECTION\_WITH: 线与线无相交,即线对象和线对象不能相交。
     |                                               检查第一个线数据集(或者线记录集)中没有与第二个线数据集(或者线记录集)中的对象有相交的对象。
     |                                               交点将作为错误生成到结果数据集中,错误数据集类型:点。
     |  :var TopologyRule.VERTEX\_DISTANCE\_GREATER\_THAN\_TOLERANCE: 节点距离必须大于容限。检查点、线、面类型的两个数据集内部或者两个数据集之间对象的节点距离是否小于容限。
     |                                                            不大于容限的节点将作为错误生成到结果数据集中,错误数据集类型:点。
     |                                                            注意:如果两节点重合,即距离为0,则不视为拓扑错误。
     |  :var TopologyRule.LINE\_EXIST\_INTERSECT\_VERTEX: 线段相交处必须存在交点。线、面类型的数据集内部或两个数据集之间,线段与线段十字相交处必须存在节点,且此节点至少存在于两个相交线段中的一个。
     |                                                 如不满足则将此交点计算出来作为错误生成到结果数据集中,错误数据集类型:点。
     |                                                 注意:两条线段端点相接的情况不违反规则。
     |  :var TopologyRule.VERTEX\_MATCH\_WITH\_EACH\_OTHER: 节点之间必须互相匹配,即容限范围内线段上存在垂足点。
     |                                                  检查线、面类型数据集内部或两个数据集之间,点数据集和线数据集、点数据集和面数据之间,在当前节点 P 的容限范围内,线段 L 上应存在一个节点 Q 在与之匹配,即 Q 在 P 的容限范围内。如不满足,则计算 P 到 L 的“垂足” A 点(即 A 与 P 匹配)作为错误生成到结果数据集中,错误数据集类型:点。
     |  :var TopologyRule.NO\_REDUNDANT\_VERTEX: 线或面边界无冗余节点。检查线数据集或面数据集中是否存在有冗余节点的对象。线对象或面对象边界上的两节点之间如果存在其他共线节点,则这些共线节点为冗余节点。
     |                                         冗余节点将作为错误生成到结果数据集中,错误数据类型:点
     |  :var TopologyRule.LINE\_NO\_SHARP\_ANGLE: 线内无打折。检查线数据集(或记录集)中线对象是否存在打折。若一条线上连续四个节点形成的两个夹角均小于所给的尖角角度容限,则认为线段在此处打折。
     |                                         产生尖角的第一个折点作为错误生成到结果数据集中,错误数据类型:点。
     |                                         注意:在使用 :py:meth:`topology\_validate` 方法对该规则检查时,通过该方法的 tolerance 参数设置尖角容
     |  :var TopologyRule.LINE\_NO\_SMALL\_DANGLES: 线内无短悬线,用于对线数据进行拓扑检查。检查线数据集(或记录集)中线对象是否是短悬线。一条悬线的长度小于悬线容限的线对象即为短悬线
     |                                           短悬线的端点作为错误生成到结果数据集中,错误数据类型:点。
     |                                           注意:在使用 :py:meth:`topology\_validate` 方法对该规则检查时,通过该方法的 tolerance 参数设置短悬线容限。
     |  :var TopologyRule.LINE\_NO\_EXTENDED\_DANGLES: 线内无长悬线,用于对线数据进行拓扑检查。检查线数据集(或记录集)中线对象是否是长悬线。一条悬线按其行进方向延伸了指定的长度(悬线容限)之后与某弧段有交点,则该线对象为长悬线。
     |                                              长悬线需要延长一端的端点作为错误生成到结果数据集中,错误数据类型:点。
     |                                              注意:在使用 :py:meth:`topology\_validate` 方法对该规则检查时,通过该方法的 tolerance 参数设置长悬线容限。
     |  :var TopologyRule.REGION\_NO\_ACUTE\_ANGLE: 面内无锐角,用于对面数据进行拓扑检查。检查面数据集(或记录集)中面对象是否存在锐角。若面边界线上连续三个节点形成的夹角小于所给的锐角角度容限,则认为此夹角为锐角。
     |                                           产生锐角的第二个节点作为错误生成到结果数据集中,错误数据类型:点。
     |                                           注意:在使用 :py:meth:`topology\_validate` 方法对该规则检查时,通过该方法的 tolerance 参数设置锐角容限。
     |  
     |  Method resolution order:
     |      TopologyRule
     |      JEnum
     |      enum.IntEnum
     |      builtins.int
     |      enum.Enum
     |      builtins.object
     |  
     |  Data and other attributes defined here:
     |  
     |  LINE\_BE\_COVERED\_BY\_LINE\_CLASS = TopologyRule.LINE\_BE\_COVERED\_BY\_LINE\_C{\ldots}
     |  
     |  LINE\_COVERED\_BY\_REGION\_BOUNDARY = TopologyRule.LINE\_COVERED\_BY\_REGION\_{\ldots}
     |  
     |  LINE\_END\_POINT\_COVERED\_BY\_POINT = TopologyRule.LINE\_END\_POINT\_COVERED\_{\ldots}
     |  
     |  LINE\_EXIST\_INTERSECT\_VERTEX = TopologyRule.LINE\_EXIST\_INTERSECT\_VERTEX
     |  
     |  LINE\_NO\_DANGLES = TopologyRule.LINE\_NO\_DANGLES
     |  
     |  LINE\_NO\_EXTENDED\_DANGLES = TopologyRule.LINE\_NO\_EXTENDED\_DANGLES
     |  
     |  LINE\_NO\_INTERSECTION = TopologyRule.LINE\_NO\_INTERSECTION
     |  
     |  LINE\_NO\_INTERSECTION\_WITH = TopologyRule.LINE\_NO\_INTERSECTION\_WITH
     |  
     |  LINE\_NO\_INTERSECTION\_WITH\_REGION = TopologyRule.LINE\_NO\_INTERSECTION\_W{\ldots}
     |  
     |  LINE\_NO\_INTERSECT\_OR\_INTERIOR\_TOUCH = TopologyRule.LINE\_NO\_INTERSECT\_O{\ldots}
     |  
     |  LINE\_NO\_OVERLAP = TopologyRule.LINE\_NO\_OVERLAP
     |  
     |  LINE\_NO\_OVERLAP\_WITH = TopologyRule.LINE\_NO\_OVERLAP\_WITH
     |  
     |  LINE\_NO\_PSEUDO\_NODES = TopologyRule.LINE\_NO\_PSEUDO\_NODES
     |  
     |  LINE\_NO\_SELF\_INTERSECT = TopologyRule.LINE\_NO\_SELF\_INTERSECT
     |  
     |  LINE\_NO\_SELF\_OVERLAP = TopologyRule.LINE\_NO\_SELF\_OVERLAP
     |  
     |  LINE\_NO\_SHARP\_ANGLE = TopologyRule.LINE\_NO\_SHARP\_ANGLE
     |  
     |  LINE\_NO\_SMALL\_DANGLES = TopologyRule.LINE\_NO\_SMALL\_DANGLES
     |  
     |  NO\_MULTIPART = TopologyRule.NO\_MULTIPART
     |  
     |  NO\_REDUNDANT\_VERTEX = TopologyRule.NO\_REDUNDANT\_VERTEX
     |  
     |  POINT\_BECOVERED\_BY\_LINE\_END\_POINT = TopologyRule.POINT\_BECOVERED\_BY\_LI{\ldots}
     |  
     |  POINT\_CONTAINED\_BY\_REGION = TopologyRule.POINT\_CONTAINED\_BY\_REGION
     |  
     |  POINT\_COVERED\_BY\_LINE = TopologyRule.POINT\_COVERED\_BY\_LINE
     |  
     |  POINT\_COVERED\_BY\_REGION\_BOUNDARY = TopologyRule.POINT\_COVERED\_BY\_REGIO{\ldots}
     |  
     |  POINT\_NO\_CONTAINED\_BY\_REGION = TopologyRule.POINT\_NO\_CONTAINED\_BY\_REGI{\ldots}
     |  
     |  POINT\_NO\_IDENTICAL = TopologyRule.POINT\_NO\_IDENTICAL
     |  
     |  REGION\_BOUNDARY\_COVERED\_BY\_LINE = TopologyRule.REGION\_BOUNDARY\_COVERED{\ldots}
     |  
     |  REGION\_BOUNDARY\_COVERED\_BY\_REGION\_BOUNDARY = TopologyRule.REGION\_BOUND{\ldots}
     |  
     |  REGION\_CONTAIN\_POINT = TopologyRule.REGION\_CONTAIN\_POINT
     |  
     |  REGION\_COVERED\_BY\_REGION = TopologyRule.REGION\_COVERED\_BY\_REGION
     |  
     |  REGION\_COVERED\_BY\_REGION\_CLASS = TopologyRule.REGION\_COVERED\_BY\_REGION{\ldots}
     |  
     |  REGION\_NO\_ACUTE\_ANGLE = TopologyRule.REGION\_NO\_ACUTE\_ANGLE
     |  
     |  REGION\_NO\_GAPS = TopologyRule.REGION\_NO\_GAPS
     |  
     |  REGION\_NO\_OVERLAP = TopologyRule.REGION\_NO\_OVERLAP
     |  
     |  REGION\_NO\_OVERLAP\_ON\_BOUNDARY = TopologyRule.REGION\_NO\_OVERLAP\_ON\_BOUN{\ldots}
     |  
     |  REGION\_NO\_OVERLAP\_WITH = TopologyRule.REGION\_NO\_OVERLAP\_WITH
     |  
     |  REGION\_NO\_SELF\_INTERSECTION = TopologyRule.REGION\_NO\_SELF\_INTERSECTION
     |  
     |  VERTEX\_DISTANCE\_GREATER\_THAN\_TOLERANCE = TopologyRule.VERTEX\_DISTANCE\_{\ldots}
     |  
     |  VERTEX\_MATCH\_WITH\_EACH\_OTHER = TopologyRule.VERTEX\_MATCH\_WITH\_EACH\_OTH{\ldots}
     |  
     |  ----------------------------------------------------------------------
     |  Data descriptors inherited from enum.Enum:
     |  
     |  name
     |      The name of the Enum member.
     |  
     |  value
     |      The value of the Enum member.
     |  
     |  ----------------------------------------------------------------------
     |  Data descriptors inherited from enum.EnumMeta:
     |  
     |  \_\_members\_\_
     |      Returns a mapping of member name->value.
     |      
     |      This mapping lists all enum members, including aliases. Note that this
     |      is a read-only view of the internal mapping.
    
    class Unit(JEnum)
     |  该类定义了表示单位的类型常量。
     |  
     |  :var Unit.MILIMETER: 毫米
     |  :var Unit.CENTIMETER: 厘米
     |  :var Unit.DECIMETER: 分米
     |  :var Unit.METER: 米
     |  :var Unit.KILOMETER: 千米
     |  :var Unit.INCH: 英寸
     |  :var Unit.FOOT: 英尺
     |  :var Unit.YARD: 码
     |  :var Unit.MILE: 英里
     |  :var Unit.SECOND: 秒,角度单位
     |  :var Unit.MINUTE: 分,角度单位
     |  :var Unit.DEGREE: 度,角度单位
     |  :var Unit.RADIAN: 弧度,弧度单位
     |  
     |  Method resolution order:
     |      Unit
     |      JEnum
     |      enum.IntEnum
     |      builtins.int
     |      enum.Enum
     |      builtins.object
     |  
     |  Data and other attributes defined here:
     |  
     |  CENTIMETER = Unit.CENTIMETER
     |  
     |  DECIMETER = Unit.DECIMETER
     |  
     |  DEGREE = Unit.DEGREE
     |  
     |  FOOT = Unit.FOOT
     |  
     |  INCH = Unit.INCH
     |  
     |  KILOMETER = Unit.KILOMETER
     |  
     |  METER = Unit.METER
     |  
     |  MILE = Unit.MILE
     |  
     |  MILIMETER = Unit.MILIMETER
     |  
     |  MINUTE = Unit.MINUTE
     |  
     |  RADIAN = Unit.RADIAN
     |  
     |  SECOND = Unit.SECOND
     |  
     |  YARD = Unit.YARD
     |  
     |  ----------------------------------------------------------------------
     |  Data descriptors inherited from enum.Enum:
     |  
     |  name
     |      The name of the Enum member.
     |  
     |  value
     |      The value of the Enum member.
     |  
     |  ----------------------------------------------------------------------
     |  Data descriptors inherited from enum.EnumMeta:
     |  
     |  \_\_members\_\_
     |      Returns a mapping of member name->value.
     |      
     |      This mapping lists all enum members, including aliases. Note that this
     |      is a read-only view of the internal mapping.
    
    class VCTVersion(JEnum)
     |  VCT 版本
     |  
     |  :var VCTVersion.CNSDTF\_VCT: 国家自然标准 1.0
     |  :var VCTVersion.LANDUSE\_VCT: 国家土地利用 2.0
     |  :var VCTVersion.LANDUSE\_VCT30: 国家土地利用 3.0
     |  
     |  Method resolution order:
     |      VCTVersion
     |      JEnum
     |      enum.IntEnum
     |      builtins.int
     |      enum.Enum
     |      builtins.object
     |  
     |  Data and other attributes defined here:
     |  
     |  CNSDTF\_VCT = VCTVersion.CNSDTF\_VCT
     |  
     |  LANDUSE\_VCT = VCTVersion.LANDUSE\_VCT
     |  
     |  LANDUSE\_VCT30 = VCTVersion.LANDUSE\_VCT30
     |  
     |  ----------------------------------------------------------------------
     |  Data descriptors inherited from enum.Enum:
     |  
     |  name
     |      The name of the Enum member.
     |  
     |  value
     |      The value of the Enum member.
     |  
     |  ----------------------------------------------------------------------
     |  Data descriptors inherited from enum.EnumMeta:
     |  
     |  \_\_members\_\_
     |      Returns a mapping of member name->value.
     |      
     |      This mapping lists all enum members, including aliases. Note that this
     |      is a read-only view of the internal mapping.
    
    class VariogramMode(JEnum)
     |  该类定义了克吕金(Kriging)插值时的半变函数类型常量。 定义克吕金(Kriging)插值时的半变函数类型。包括指数型、球型和高斯型。用户所选择的半变函
     |  数类型会影响未知点的预测,特别是曲线在原点处的不同形状有重要意义。曲线在原点处越陡,则较近领域对该预测值的影响就越大。因此输出表面就会越不光滑。
     |  每种类型都有各自适用的情况。
     |  
     |  :var VariogramMode.EXPONENTIAL: 指数函数(Exponential Variogram Mode)。这种类型适用于在空间自相关关系随距离增加成指数递减的情况。
     |                                  下图所示为空间自相关关系在无穷处完全消失。指数函数较为常用。
     |  
     |                                  .. image:: ../image/VariogramMode\_Exponential.png
     |  
     |  :var VariogramMode.GAUSSIAN: 高斯函数(Gaussian Variogram Mode)。
     |  
     |                              .. image:: ../image/variogrammode\_Gaussian.png
     |  
     |  :var VariogramMode.SPHERICAL: 球型函数(Spherical Variogram Mode)。这种类型显示了空间自相关关系逐渐减少的情况下(即半变函数值逐渐
     |                                增加),直到超出一定的距离,空间自相关关系为0。球型函数较为常用。
     |  
     |                                .. image:: ../image/VariogramMode\_Spherical.png
     |  
     |  Method resolution order:
     |      VariogramMode
     |      JEnum
     |      enum.IntEnum
     |      builtins.int
     |      enum.Enum
     |      builtins.object
     |  
     |  Data and other attributes defined here:
     |  
     |  EXPONENTIAL = VariogramMode.EXPONENTIAL
     |  
     |  GAUSSIAN = VariogramMode.GAUSSIAN
     |  
     |  SPHERICAL = VariogramMode.SPHERICAL
     |  
     |  ----------------------------------------------------------------------
     |  Data descriptors inherited from enum.Enum:
     |  
     |  name
     |      The name of the Enum member.
     |  
     |  value
     |      The value of the Enum member.
     |  
     |  ----------------------------------------------------------------------
     |  Data descriptors inherited from enum.EnumMeta:
     |  
     |  \_\_members\_\_
     |      Returns a mapping of member name->value.
     |      
     |      This mapping lists all enum members, including aliases. Note that this
     |      is a read-only view of the internal mapping.
    
    class VectorResampleType(JEnum)
     |  矢量数据集重采样方法类型常量
     |  
     |  :var VectorResampleType.RTBEND: 使用光栏采样算法进行重采样
     |  :var VectorResampleType.RTGENERAL: 使用道格拉斯算法进行重采样
     |  
     |  Method resolution order:
     |      VectorResampleType
     |      JEnum
     |      enum.IntEnum
     |      builtins.int
     |      enum.Enum
     |      builtins.object
     |  
     |  Data and other attributes defined here:
     |  
     |  RTBEND = VectorResampleType.RTBEND
     |  
     |  RTGENERAL = VectorResampleType.RTGENERAL
     |  
     |  ----------------------------------------------------------------------
     |  Data descriptors inherited from enum.Enum:
     |  
     |  name
     |      The name of the Enum member.
     |  
     |  value
     |      The value of the Enum member.
     |  
     |  ----------------------------------------------------------------------
     |  Data descriptors inherited from enum.EnumMeta:
     |  
     |  \_\_members\_\_
     |      Returns a mapping of member name->value.
     |      
     |      This mapping lists all enum members, including aliases. Note that this
     |      is a read-only view of the internal mapping.
    
    class WorkspaceType(JEnum)
     |  该类定义了工作空间类型常量。
     |  
     |  SuperMap 支持的文件型工作空间的类型有四种,SSXWU 格式和 SMWU 格式;SuperMap 支持的数据库型工作空间的类型有两种:Oracle 工作空间 和 SQL Server 工作空间。
     |  
     |  :var WorkspaceType.DEFAULT:   默认值, 表示工作空间未被保存时的工作空间类型。
     |  :var WorkspaceType.ORACLE: Oracle 工作空间。工作空间保存在 Oracle 数据库中。
     |  :var WorkspaceType.SQL: SQL Server 工作空间。工作空间保存在 SQL Server 数据库中。该常量仅在 Windows 平台版本中支持,在 Linux版本中不提供。
     |  :var WorkspaceType.DM: DM 工作空间。工作空间保存在DM 数据库中。
     |  :var WorkspaceType.MYSQL: MYSQL 工作空间。工作空间保存在MySQL 数据库中。
     |  :var WorkspaceType.PGSQL: PostgreSQL 工作空间。工作空间保存在PostgreSQL 数据库中。
     |  :var WorkspaceType.MONGO: MongoDB 工作空间。工作空间保存在 MongoDB 数据库中。
     |  :var WorkspaceType.SXWU: SXWU工作空间,只有 6R 版本的工作空间能存成类型为 SXWU 的工作空间文件。另存为 6R 版本的工作空间时,文件型工作空间只能存为 SXWU 或是 SMWU。
     |  :var WorkspaceType.SMWU: SMWU工作空间,只有 6R 版本的工作空间能存成类型为 SMWU 的工作空间文件。另存为 6R 版本的工作空间时,文件型工作空间只能存为 SXWU 或是 SMWU。该常量仅在 Windows 平台版本中支持,在 Linux版本中不提供。
     |  
     |  Method resolution order:
     |      WorkspaceType
     |      JEnum
     |      enum.IntEnum
     |      builtins.int
     |      enum.Enum
     |      builtins.object
     |  
     |  Data and other attributes defined here:
     |  
     |  DEFAULT = WorkspaceType.DEFAULT
     |  
     |  DM = WorkspaceType.DM
     |  
     |  MONGO = WorkspaceType.MONGO
     |  
     |  MYSQL = WorkspaceType.MYSQL
     |  
     |  ORACLE = WorkspaceType.ORACLE
     |  
     |  PGSQL = WorkspaceType.PGSQL
     |  
     |  SMWU = WorkspaceType.SMWU
     |  
     |  SQL = WorkspaceType.SQL
     |  
     |  SXWU = WorkspaceType.SXWU
     |  
     |  ----------------------------------------------------------------------
     |  Data descriptors inherited from enum.Enum:
     |  
     |  name
     |      The name of the Enum member.
     |  
     |  value
     |      The value of the Enum member.
     |  
     |  ----------------------------------------------------------------------
     |  Data descriptors inherited from enum.EnumMeta:
     |  
     |  \_\_members\_\_
     |      Returns a mapping of member name->value.
     |      
     |      This mapping lists all enum members, including aliases. Note that this
     |      is a read-only view of the internal mapping.
    
    class WorkspaceVersion(JEnum)
     |  该类定义了工作空间版本类型常量。
     |  
     |  :var WorkspaceVersion.UGC60: SuperMap UGC 6.0 工作空间
     |  :var WorkspaceVersion.UGC70: SuperMap UGC 7.0 工作空间
     |  
     |  Method resolution order:
     |      WorkspaceVersion
     |      JEnum
     |      enum.IntEnum
     |      builtins.int
     |      enum.Enum
     |      builtins.object
     |  
     |  Data and other attributes defined here:
     |  
     |  UGC60 = WorkspaceVersion.UGC60
     |  
     |  UGC70 = WorkspaceVersion.UGC70
     |  
     |  ----------------------------------------------------------------------
     |  Data descriptors inherited from enum.Enum:
     |  
     |  name
     |      The name of the Enum member.
     |  
     |  value
     |      The value of the Enum member.
     |  
     |  ----------------------------------------------------------------------
     |  Data descriptors inherited from enum.EnumMeta:
     |  
     |  \_\_members\_\_
     |      Returns a mapping of member name->value.
     |      
     |      This mapping lists all enum members, including aliases. Note that this
     |      is a read-only view of the internal mapping.

DATA
    \_\_all\_\_ = ['PixelFormat', 'BlockSizeOption', 'AreaUnit', 'Unit', 'Engi{\ldots}

FILE
    /opt/conda/lib/python3.6/site-packages/iobjectspy/enums.py



    \end{Verbatim}

    \begin{Verbatim}[commandchars=\\\{\}]
{\color{incolor}In [{\color{incolor}6}]:} \PY{n}{help}\PY{p}{(}\PY{n}{smo}\PY{o}{.}\PY{n}{ai}\PY{p}{)}
\end{Verbatim}


    \begin{Verbatim}[commandchars=\\\{\}]
Help on package iobjectspy.ai in iobjectspy:

NAME
    iobjectspy.ai

PACKAGE CONTENTS
    \_detection
    \_segmentation
    \_toolkit
    recognition
    utils

FILE
    /opt/conda/lib/python3.6/site-packages/iobjectspy/ai/\_\_init\_\_.py



    \end{Verbatim}

    \begin{Verbatim}[commandchars=\\\{\}]
{\color{incolor}In [{\color{incolor}7}]:} \PY{n}{help}\PY{p}{(}\PY{n}{smo}\PY{o}{.}\PY{n}{conversion}\PY{p}{)}
\end{Verbatim}


    \begin{Verbatim}[commandchars=\\\{\}]
Help on module iobjectspy.conversion in iobjectspy:

NAME
    iobjectspy.conversion

DESCRIPTION
    conversion 模块提供基本的数据导入和导出功能,通过使用 conversion 模块可以快速的将第三方的文件导入到 SuperMap 的数据源中,也可以将 SuperMap
    数据源 中的数据导出为 第三方文件。
    
    在 conversion 模块中,所有导入数据的接口中,output 参数输入结果数据集的数据源信息,可以为 Datasource 对象,也可以为 DatasourceConnectionInfo 对象,
    同时,也支持当前工作空间下数据源的别名,也支持 UDB 文件路径,DCF 文件路径等。
    
    
    >>> ds = Datasource.create(':memory:')
    >>> alias = ds.alias
    >>> shape\_file = 'E:/Point.shp'
    >>> result1 = import\_shape(shape\_file, ds)
    >>> result2 = import\_shape(shape\_file, alias)
    >>> result3 = import\_shape(shape\_file, 'E:/Point\_Out.udb')
    >>> result4 = import\_shape(shape\_file, 'E:/Point\_Out\_conn.dcf')
    
    
    而导入数据的结果返回一个 Dataset 或 str 的列表对象。当导入数据只生成一个数据集时,列表的个数为1,当导入数据生成多个数据集时,列表的个数可能大于1。
    列表中返回 Dataset 还是 str 是由输入的 output 参数决定的,当输入的 output 参数可以直接在当前工作空间中获取到数据源对象时,将会返回 Dataset 的列表,
    如果输入的 output 参数无法直接在当前工作空间中获取到数据源对象时,程序将自动尝试打开数据源或新建数据源(只支持新建 UDB 数据源),此时,返回的结果将是
    结果数据集的数据集名称,而完成数据导入后,结果数据源也会被关闭。所以如果用户需要继续基于导入后的结果数据集进行操作,则需要根据结果数据集名称和数据源信息再次开发数据源以获取数据集。
    
    
    所有导出数据集的接口,data 参数是被导出的数据集信息,data 参数接受输入一个数据集对象(Dataset)或数据源别名与数据集名称的组合(例如,'alias/dataset\_name', 'alias\textbackslash{}\textbackslash{}\textbackslash{}dataset\_name'),
    ,也支持数据源连接信息与数据集名称的组合(例如, 'E:/data.udb/dataset\_name'),值得注意的是,当输入的是数据源信息时,程序会自动打开数据源,但是不会自动关闭数据源,也就是打开后的数据源
    会存在当前工作空间中
    
    >>> export\_to\_shape('E:/data.udb/Point', 'E:/Point.shp', is\_over\_write=True)
    >>> ds = Datasource.open('E:/data.udb')
    >>> export\_to\_shape(ds['Point'], 'E:/Point.shp', is\_over\_write=True)
    >>> export\_to\_shape(ds.alias + '|Point', 'E:/Point.shp', is\_over\_write=True)
    >>> ds.close()

FUNCTIONS
    export\_to\_bmp(data, output, is\_over\_write=False, world\_file\_path=None, progress=None)
        导出数据集到 BMP 文件中
        
        :param data: 被导出的数据集
        :type data: DatasetImage or str
        :param str output: 结果文件路径
        :param bool is\_over\_write: 导出目录中存在同名文件时,是否强制覆盖。默认为 False
        :param str world\_file\_path: 导出的影像数据的坐标文件路径
        :param function progress: 进度信息处理函数,请参考 :py:class:`.StepEvent`
        :return: 是否导出成功
        :rtype: bool
    
    export\_to\_csv(data, output, is\_over\_write=False, attr\_filter=None, ignore\_fields=None, target\_file\_charset=None, is\_export\_field\_names=True, is\_export\_point\_as\_wkt=False, progress=None)
        导出数据集到 csv 文件中
        
        :param data: 被导出的数据集
        :type data: DatasetVector or str
        :param str output: 结果文件路径
        :param bool is\_over\_write: 导出目录中存在同名文件时,是否强制覆盖。默认为 False
        :param str attr\_filter: 导出目标文件的过滤信息
        :param ignore\_fields:  需要忽略的字段
        :type ignore\_fields: list[str]
        :param target\_file\_charset: 需要导出的文件的字符集类型
        :type target\_file\_charset: Charset or str
        :param bool is\_export\_field\_names:  是否写出字段名称。
        :param bool is\_export\_point\_as\_wkt: 是否将点以 WKT 方式写出。
        :param function progress: 进度信息处理函数,请参考 :py:class:`.StepEvent`
        :return: 是否导出成功
        :rtype: bool
    
    export\_to\_dbf(data, output, is\_over\_write=False, attr\_filter=None, ignore\_fields=None, target\_file\_charset=None, progress=None)
        导出数据集到 dbf 文件中
        
        :param data: 被导出的数据集,只支持导出属性表数据集
        :type data: DatasetVector or str
        :param str output: 结果文件路径
        :param bool is\_over\_write: 导出目录中存在同名文件时,是否强制覆盖。默认为 False
        :param str attr\_filter: 导出目标文件的过滤信息
        :param ignore\_fields:  需要忽略的字段
        :type ignore\_fields: list[str]
        :param target\_file\_charset: 需要导出的文件的字符集类型
        :type target\_file\_charset: Charset or str
        :param function progress: 进度信息处理函数,请参考 :py:class:`.StepEvent`
        :return: 是否导出成功
        :rtype: bool
    
    export\_to\_dwg(data, output, is\_over\_write=False, attr\_filter=None, ignore\_fields=None, cad\_version=CADVersion.CAD2007, is\_export\_border=False, is\_export\_xrecord=False, is\_export\_external\_data=False, style\_map\_file=None, progress=None)
        导出数据集到 DWG 文件中, Linux 平台不支持导出数据集为 DWG 文件。
        
        :param data: 被导出的数据集
        :type data: DatasetVector or str
        :param str output: 结果文件路径
        :param bool is\_over\_write: 导出目录中存在同名文件时,是否强制覆盖。默认为 False
        :param str attr\_filter: 导出目标文件的过滤信息
        :param ignore\_fields:  需要忽略的字段
        :type ignore\_fields: list[str]
        :param cad\_version: 导出的 DWG 文件的版本。
        :type cad\_version: CADVersion or str
        :param bool is\_export\_border: 导出cad面对像或矩形对象时是否导出边界。
        :param bool is\_export\_xrecord: 是否将用户自定义的字段以及属性字段作为扩展记录导出
        :param bool is\_export\_external\_data: 是否导出扩展字段
        :param str style\_map\_file: 风格对照表的路径
        :param function progress: 进度信息处理函数,请参考 :py:class:`.StepEvent`
        :return: 是否导出成功
        :rtype: bool
    
    export\_to\_dxf(data, output, is\_over\_write=False, attr\_filter=None, ignore\_fields=None, cad\_version=CADVersion.CAD2007, is\_export\_border=False, is\_export\_xrecord=False, is\_export\_external\_data=False, progress=None)
        导出数据集到 DXF 文件中,Linux 平台不支持导出数据集为 DXF 文件
        
        :param data: 被导出的数据集
        :type data: DatasetVector or str
        :param str output: 结果文件路径
        :param bool is\_over\_write: 导出目录中存在同名文件时,是否强制覆盖。默认为 False
        :param str attr\_filter: 导出目标文件的过滤信息
        :param ignore\_fields:  需要忽略的字段
        :type ignore\_fields: list[str]
        :param cad\_version: 导出的 DWG 文件的版本。
        :type cad\_version: CADVersion or str
        :param bool is\_export\_border: 导出cad面对像或矩形对象时是否导出边界。
        :param bool is\_export\_xrecord: 是否将用户自定义的字段以及属性字段作为扩展记录导出
        :param bool is\_export\_external\_data: 是否导出扩展字段
        :param function progress: 进度信息处理函数,请参考 :py:class:`.StepEvent`
        :return: 是否导出成功
        :rtype: bool
    
    export\_to\_e00(data, output, is\_over\_write=False, attr\_filter=None, ignore\_fields=None, target\_file\_charset=None, double\_precision=False, progress=None)
        导出数据集到 E00 文件中
        
        :param data: 被导出的数据集
        :type data: DatasetVector or str
        :param str output: 结果文件路径
        :param bool is\_over\_write: 导出目录中存在同名文件时,是否强制覆盖。默认为 False
        :param str attr\_filter: 导出目标文件的过滤信息
        :param ignore\_fields:  需要忽略的字段
        :type ignore\_fields: list[str]
        :param target\_file\_charset: 需要导出的文件的字符集类型
        :type target\_file\_charset: Charset or str
        :param bool double\_precision: 是否以双精度方式导出 E00,默认为 False。
        :param function progress: 进度信息处理函数,请参考 :py:class:`.StepEvent`
        :return: 是否导出成功
        :rtype: bool
    
    export\_to\_geojson(data, output, is\_over\_write=False, attr\_filter=None, ignore\_fields=None, target\_file\_charset=None, progress=None)
        导出数据集到 GeoJson 文件中
        
        :param data: 被导出的数据集集合
        :type data: DatasetVector or str or list[DatasetVector] or list[str]
        :param str output: 结果文件路径
        :param bool is\_over\_write: 导出目录中存在同名文件时,是否强制覆盖。默认为 False
        :param str attr\_filter: 导出目标文件的过滤信息
        :param ignore\_fields:  需要忽略的字段
        :type ignore\_fields: list[str]
        :param target\_file\_charset: 需要导出的文件的字符集类型
        :type target\_file\_charset: Charset or str
        :param function progress: 进度信息处理函数,请参考 :py:class:`.StepEvent`
        :return: 是否导出成功
        :rtype: bool
    
    export\_to\_gif(data, output, is\_over\_write=False, world\_file\_path=None, progress=None)
        导出数据集到 GIF 文件中
        
        :param data: 被导出的数据集
        :type data: DatasetImage or str
        :param str output: 结果文件路径
        :param bool is\_over\_write: 导出目录中存在同名文件时,是否强制覆盖。默认为 False
        :param str world\_file\_path: 导出的影像数据的坐标文件路径
        :param function progress: 进度信息处理函数,请参考 :py:class:`.StepEvent`
        :return: 是否导出成功
        :rtype: bool
    
    export\_to\_grd(data, output, is\_over\_write=False, progress=None)
        导出数据集到 GRD 文件中
        
        :param data: 被导出的数据集
        :type data: DatasetGrid or str
        :param str output: 结果文件路径
        :param bool is\_over\_write: 导出目录中存在同名文件时,是否强制覆盖。默认为 False
        :param function progress: 进度信息处理函数,请参考 :py:class:`.StepEvent`
        :return: 是否导出成功
        :rtype: bool
    
    export\_to\_img(data, output, is\_over\_write=False, progress=None)
        导出数据集到 IMG 文件中
        
        :param data: 被导出的数据集
        :type data: DatasetImage or DatasetGrid or str
        :param str output: 结果文件路径
        :param bool is\_over\_write: 导出目录中存在同名文件时,是否强制覆盖。默认为 False
        :param function progress: 进度信息处理函数,请参考 :py:class:`.StepEvent`
        :return: 是否导出成功
        :rtype: bool
    
    export\_to\_jpg(data, output, is\_over\_write=False, world\_file\_path=None, compression=None, progress=None)
        导出数据集到 JPG 文件中
        
        :param data: 被导出的数据集
        :type data: DatasetImage or str
        :param str output: 结果文件路径
        :param bool is\_over\_write: 导出目录中存在同名文件时,是否强制覆盖。默认为 False
        :param str world\_file\_path: 导出的影像数据的坐标文件路径
        :param int compression: 影像文件的压缩率,单位:百分比
        :param function progress: 进度信息处理函数,请参考 :py:class:`.StepEvent`
        :return: 是否导出成功
        :rtype: bool
    
    export\_to\_kml(data, output, is\_over\_write=False, attr\_filter=None, ignore\_fields=None, target\_file\_charset=None, progress=None)
        导出数据集到 KML 文件中
        
        :param data: 被导出的数据集集合
        :type data: DatasetVector or str or list[DatasetVector] or list[str]
        :param str output: 结果文件路径
        :param bool is\_over\_write: 导出目录中存在同名文件时,是否强制覆盖。默认为 False
        :param str attr\_filter: 导出目标文件的过滤信息
        :param ignore\_fields:  需要忽略的字段
        :type ignore\_fields: list[str]
        :param target\_file\_charset: 需要导出的文件的字符集类型
        :type target\_file\_charset: Charset or str
        :param function progress: 进度信息处理函数,请参考 :py:class:`.StepEvent`
        :return: 是否导出成功
        :rtype: bool
    
    export\_to\_kmz(data, output, is\_over\_write=False, attr\_filter=None, ignore\_fields=None, target\_file\_charset=None, progress=None)
        导出数据集到 KMZ 文件中
        
        :param data: 被导出的数据集集合
        :type data: DatasetVector or str or list[DatasetVector] or list[str]
        :param str output: 结果文件路径
        :param bool is\_over\_write: 导出目录中存在同名文件时,是否强制覆盖。默认为 False
        :param str attr\_filter: 导出目标文件的过滤信息
        :param ignore\_fields:  需要忽略的字段
        :type ignore\_fields: list[str]
        :param target\_file\_charset: 需要导出的文件的字符集类型
        :type target\_file\_charset: Charset or str
        :param function progress: 进度信息处理函数,请参考 :py:class:`.StepEvent`
        :return: 是否导出成功
        :rtype: bool
    
    export\_to\_mif(data, output, is\_over\_write=False, attr\_filter=None, ignore\_fields=None, target\_file\_charset=None, progress=None)
        导出数据集到 MIF 文件中
        
        :param data: 被导出的数据集
        :type data: DatasetVector or str
        :param str output: 结果文件路径
        :param bool is\_over\_write: 导出目录中存在同名文件时,是否强制覆盖。默认为 False
        :param str attr\_filter: 导出目标文件的过滤信息
        :param ignore\_fields:  需要忽略的字段
        :type ignore\_fields: list[str]
        :param target\_file\_charset: 需要导出的文件的字符集类型
        :type target\_file\_charset: Charset or str
        :param function progress: 进度信息处理函数,请参考 :py:class:`.StepEvent`
        :return: 是否导出成功
        :rtype: bool
    
    export\_to\_png(data, output, is\_over\_write=False, world\_file\_path=None, progress=None)
        导出数据集到 PNG 文件中
        
        :param data: 被导出的数据集
        :type data: DatasetImage or str
        :param str output: 结果文件路径
        :param bool is\_over\_write: 导出目录中存在同名文件时,是否强制覆盖。默认为 False
        :param str world\_file\_path: 导出的影像数据的坐标文件路径
        :param function progress: 进度信息处理函数,请参考 :py:class:`.StepEvent`
        :return: 是否导出成功
        :rtype: bool
    
    export\_to\_shape(data, output, is\_over\_write=False, attr\_filter=None, ignore\_fields=None, target\_file\_charset=None, progress=None)
        导出数据集到 Shape 文件中
        
        :param data: 被导出的数据集
        :type data: DatasetVector or str
        :param str output: 结果文件路径
        :param bool is\_over\_write: 导出目录中存在同名文件时,是否强制覆盖。默认为 False
        :param str attr\_filter: 导出目标文件的过滤信息
        :param ignore\_fields:  需要忽略的字段
        :type ignore\_fields: list[str]
        :param target\_file\_charset: 需要导出的文件的字符集类型
        :type target\_file\_charset: Charset or str
        :param function progress: 进度信息处理函数,请参考 :py:class:`.StepEvent`
        :return: 是否导出成功
        :rtype: bool
    
    export\_to\_simplejson(data, output, is\_over\_write=False, attr\_filter=None, ignore\_fields=None, target\_file\_charset=None, progress=None)
        导出数据集到 SimpleJson 文件中
        
        :param data: 被导出的数据集
        :type data: DatasetVector or str
        :param str output: 结果文件路径
        :param bool is\_over\_write: 导出目录中存在同名文件时,是否强制覆盖。默认为 False
        :param str attr\_filter: 导出目标文件的过滤信息
        :param ignore\_fields:  需要忽略的字段
        :type ignore\_fields: list[str]
        :param target\_file\_charset: 需要导出的文件的字符集类型
        :type target\_file\_charset: Charset or str
        :param function progress: 进度信息处理函数,请参考 :py:class:`.StepEvent`
        :return: 是否导出成功
        :rtype: bool
    
    export\_to\_sit(data, output, is\_over\_write=False, password=None, progress=None)
        导出数据集到 SIT 文件中
        
        :param data: 被导出的数据集
        :type data: DatasetImage or str
        :param str output: 结果文件路径
        :param bool is\_over\_write: 导出目录中存在同名文件时,是否强制覆盖。默认为 False
        :param str password: 密码
        :param function progress: 进度信息处理函数,请参考 :py:class:`.StepEvent`
        :return: 是否导出成功
        :rtype: bool
    
    export\_to\_tab(data, output, is\_over\_write=False, attr\_filter=None, ignore\_fields=None, target\_file\_charset=None, style\_map\_file=None, progress=None)
        导出数据集到 TAB 文件中
        
        :param data: 被导出的数据集
        :type data: DatasetVector or str
        :param str output: 结果文件路径
        :param bool is\_over\_write: 导出目录中存在同名文件时,是否强制覆盖。默认为 False
        :param str attr\_filter: 导出目标文件的过滤信息
        :param ignore\_fields:  需要忽略的字段
        :type ignore\_fields: list[str]
        :param target\_file\_charset: 需要导出的文件的字符集类型
        :type target\_file\_charset: Charset or str
        :param str style\_map\_file: 导出的风格对照表路径
        :param function progress: 进度信息处理函数,请参考 :py:class:`.StepEvent`
        :return: 是否导出成功
        :rtype: bool
    
    export\_to\_tif(data, output, is\_over\_write=False, export\_as\_tile=False, export\_transform\_file=True, progress=None)
        导出数据集到 TIF 文件中
        
        :param data: 被导出的数据集
        :type data: DatasetImage or DatasetGrid or str
        :param str output: 结果文件路径
        :param bool is\_over\_write: 导出目录中存在同名文件时,是否强制覆盖。默认为 False
        :param bool export\_as\_tile: 是否以块的方式导出,默认为 False
        :param bool export\_transform\_file: 是否将仿射转换信息导出外部文件,默认为 True,即导出到外部的 tfw 文件中,否则投影信息会导出到 tiff 文件中
        :param function progress: 进度信息处理函数,请参考 :py:class:`.StepEvent`
        :return: 是否导出成功
        :rtype: bool
    
    export\_to\_vct(data, config\_path, version, output, is\_over\_write=False, attr\_filter=None, ignore\_fields=None, target\_file\_charset=None, progress=None)
        导出数据集到 VCT 文件中
        
        :param data: 被导出的数据集集合
        :type data: DatasetVector or str or list[DatasetVector] or str
        :param str config\_path: VCT 配置文件路径
        :param version: VCT 版本
        :type version: VCTVersion or str
        :param str output: 结果文件路径
        :param bool is\_over\_write: 导出目录中存在同名文件时,是否强制覆盖。默认为 False
        :param str attr\_filter: 导出目标文件的过滤信息
        :param ignore\_fields:  需要忽略的字段
        :type ignore\_fields: list[str]
        :param target\_file\_charset: 需要导出的文件的字符集类型
        :type target\_file\_charset: Charset or str
        :param function progress: 进度信息处理函数,请参考 :py:class:`.StepEvent`
        :return: 是否导出成功
        :rtype: bool
    
    import\_aibingrid(source\_file, output, out\_dataset\_name=None, ignore\_mode='IGNORENONE', ignore\_values=None, is\_import\_as\_grid=False, is\_build\_pyramid=True, progress=None)
        导入 AIBinGrid 文件, Linux 平台不支持导入 AIBinGrid 文件。
        
        :param str source\_file: 被导入的 AIBinGrid 文件
        :param output: 结果数据源
        :type output: Datasource or DatasourceConnectionInfo or str
        :param str out\_dataset\_name: 结果数据集名称
        :param ignore\_mode: JPG 文件的忽略颜色值的模式
        :type ignore\_mode: IgnoreMode or str
        :param ignore\_values: 要忽略的颜色值
        :type ignore\_values: list[float] 要忽略的颜色值
        :param bool is\_import\_as\_grid: 是否导入为 Grid 数据集
        :param bool is\_build\_pyramid: 是否自动建立影像金字塔
        :param function progress: 进度信息处理函数,请参考 :py:class:`.StepEvent`
        :return: 导入后的结果数据集或结果数据集名称
        :rtype: list[DatasetGrid] or list[DatasetImage] or list[str]
    
    import\_bmp(source\_file, output, out\_dataset\_name=None, ignore\_mode='IGNORENONE', ignore\_values=None, world\_file\_path=None, is\_import\_as\_grid=False, is\_build\_pyramid=True, progress=None)
        导入 BMP 文件
        
        :param str source\_file: 被导入的 BMP 文件
        :param output: 结果数据源
        :type output: Datasource or DatasourceConnectionInfo or str
        :param str out\_dataset\_name: 结果数据集名称
        :param ignore\_mode: BMP 文件的忽略颜色值的模式
        :type ignore\_mode: IgnoreMode or str
        :param ignore\_values: 要忽略的颜色值
        :type ignore\_values: list[float] 要忽略的颜色值
        :param str world\_file\_path: 导入的源影像文件的坐标参考文件路径
        :param bool is\_import\_as\_grid: 是否导入为 Grid 数据集
        :param bool is\_build\_pyramid: 是否自动建立影像金字塔
        :param function progress: 进度信息处理函数,请参考 :py:class:`.StepEvent`
        :return: 导入后的结果数据集或结果数据集名称
        :rtype: list[DatasetGrid] or list[DatasetImage] or list[str]
    
    import\_csv(source\_file, output, out\_dataset\_name=None, import\_mode=None, separator=',', head\_is\_field=True, fields\_as\_point=None, field\_as\_geometry=None, is\_import\_empty=False, source\_file\_charset=None, progress=None)
        导入 CSV 文件
        
        :param str source\_file: 被导入的 csv 文件
        :param output: 结果数据源
        :type output: Datasource or  DatasourceConnectionInfo or str
        :param str out\_dataset\_name: 结果数据集名称
        :param import\_mode: 导入模式类型,可以为 ImportMode 枚举值或名称
        :type import\_mode: ImportMode or str
        :param str separator: 源 CSV 文件中字段的分隔符。默认以 ',' 作为分隔符
        :param bool head\_is\_field: CSV 文件的首行是否为字段名称
        :param fields\_as\_point: 指定字段为X、Y或者X、Y、Z坐标,如果符合条件,则生成点或者三维点数据集
        :type fields\_as\_point: list[str] or list[int]
        :param int field\_as\_geometry: 指定WKT串的Geometry索引位置
        :param bool is\_import\_empty: 是否导入空的数据集,默认为 False,即不导入
        :param source\_file\_charset: CSV 文件的原始字符集类型
        :type source\_file\_charset: Charset or str
        :param function progress: 进度信息处理函数,请参考 :py:class:`.StepEvent`
        :return: 导入后的结果数据集或结果数据集名称
        :rtype: list[DatasetVector] or list[str]
    
    import\_dbf(source\_file, output, out\_dataset\_name=None, import\_mode=None, is\_import\_empty=False, source\_file\_charset=None, progress=None)
        导入 dbf 文件到数据源中。
        
        :param str source\_file: 被导入的 dbf 文件
        :param output: 结果数据源
        :type output: Datasource or DatasourceConnectionInfo or str
        :param str out\_dataset\_name: 结果数据集名称
        :param import\_mode: 导入模式类型,可以为 ImportMode 枚举值或名称
        :type import\_mode: ImportMode or str
        :param bool is\_import\_empty: 否导入空的数据集,默认是不导入的。默认为 False
        :param source\_file\_charset: dbf 文件的原始字符集类型
        :type source\_file\_charset: Charset or str
        :param function progress: 进度信息处理函数,请参考 :py:class:`.StepEvent`
        :return: 导入后的结果数据集或结果数据集名称
        :rtype: list[DatasetVector] or list[str]
        
        
        >>> result = import\_dbf( 'E:/point.dbf', 'E:/import\_dbf\_out.udb')
        >>> print(len(result) == 1)
        >>> print(result[0])
    
    import\_dgn(source\_file, output, out\_dataset\_name=None, import\_mode=None, is\_import\_empty=False, is\_import\_as\_cad=True, style\_map\_file=None, is\_import\_by\_layer=False, is\_cell\_as\_point=False, progress=None)
        导入 DGN 文件
        
        :param str source\_file: 被导入的 dgn 文件
        :param output: 结果数据源
        :type output: Datasource or DatasourceConnectionInfo or str
        :param str out\_dataset\_name: 结果数据集名称
        :param import\_mode: 导入模式
        :type import\_mode: ImportMode or str
        :param bool is\_import\_empty: 是否导入空数据集,默认为 False
        :param bool is\_import\_as\_cad: 是否导入为 CAD 数据集,默认为 True
        :param str style\_map\_file: 设置风格对照表的存储路径。 风格对照表是指 SuperMap 系统与其它系统风格(包括:符号、线型、填充等)的对照文件。风格对照表只对 CAD 类型的数据,如 DXF、DWG、DGN 起作用。在设置风格对照表之前,必须保证数据是以CAD方式导入,且不忽略风格。
        :param bool is\_import\_by\_layer: 是否在导入后的数据中合并源数据中的 CAD 图层信息,CAD 是以图层信息来存储的,默认为 False,即所有 的图层信息都合并到了一个 CAD 数据集, 否则对应源数据中的每一个图层生成一个 CAD 数据集。
        :param bool is\_cell\_as\_point: 是否将 cell(单元)对象导入为点对象(cell header)还是除 cell header 外的所有要素对象。 默认导入为除 cell header 外的所有要素对象。
        :param function progress: 进度信息处理函数,请参考 :py:class:`.StepEvent`
        :return: 导入后的结果数据集或结果数据集名称
        :rtype: list[DatasetVector] or list[str]
    
    import\_dwg(source\_file, output, out\_dataset\_name=None, import\_mode=None, is\_import\_empty=False, is\_import\_as\_cad=True, is\_import\_by\_layer=False, ignore\_block\_attrs=True, block\_as\_point=False, import\_external\_data=False, import\_xrecord=True, import\_invisible\_layer=False, keep\_parametric\_part=False, ignore\_lwpline\_width=False, shx\_paths=None, curve\_segment=73, style\_map\_file=None, progress=None)
        导入 DWG 文件,Linux 平台不支持导入 DWG 文件。
        
        :param str source\_file: 被导入的 dwg 文件
        :param output: 结果数据源
        :type output: Datasource or DatasourceConnectionInfo or str
        :param str out\_dataset\_name: 结果数据集名称
        :param import\_mode: 数据集导入模式
        :type import\_mode: ImportMode or str
        :param bool is\_import\_empty: 是否导入空的数据集,默认为 False,即不导入
        :param bool is\_import\_as\_cad: 是否以 CAD 数据集方式导入
        :param bool is\_import\_by\_layer: 是否在导入后的数据中合并源数据中的 CAD 图层信息,CAD 是以图层信息来存储的,默认为 False,即所有的图层信息都合并到了一个 CAD 数据集, 否则对应源数据中的每一个图层生成一个 CAD 数据集。
        :param bool ignore\_block\_attrs: 是否数据导入时是否忽略块儿属性。默认为 True
        :param bool block\_as\_point: 将符号块导入为点对象还是复合对象,默认为 False, 即将原有的符号块作为复合对象导入,否则在符号块的位置用点对象代替。
        :param bool import\_external\_data: 否导入外部数据,外部数据为 CAD 中类似属性表的数据导入后格式为一些额外的字段,默认为 False,否则将外部数据追加在默认字段后面。
        :param bool import\_xrecord: 是否将用户自定义的字段以及属性字段作为扩展记录导入。
        :param bool import\_invisible\_layer: 是否导入不可见图层
        :param bool keep\_parametric\_part: 是否保留Acad数据中的参数化部分
        :param bool ignore\_lwpline\_width: 是否忽略多义线宽度,默认为 False。
        :param shx\_paths: shx 字体库的路径
        :type shx\_paths: list[str]
        :param int curve\_segment: 曲线拟合精度,默认为 73
        :param str style\_map\_file: 风格对照表的存储路径
        :param function progress: 进度信息处理函数,请参考 :py:class:`.StepEvent`
        :return: 导入后的结果数据集或结果数据集名称
        :rtype: list[DatasetVector] or list[str]
    
    import\_dxf(source\_file, output, out\_dataset\_name=None, import\_mode=None, is\_import\_empty=False, is\_import\_as\_cad=True, is\_import\_by\_layer=False, ignore\_block\_attrs=True, block\_as\_point=False, import\_external\_data=False, import\_xrecord=True, import\_invisible\_layer=False, keep\_parametric\_part=False, ignore\_lwpline\_width=False, shx\_paths=None, curve\_segment=73, style\_map\_file=None, progress=None)
        导入 DXF 文件,Linux 平台不支持导入 DXF 文件
        
        :param str source\_file: 被导入的 dxf 文件
        :param output: 结果数据源
        :type output: Datasource or DatasourceConnectionInfo or str
        :param str out\_dataset\_name: 结果数据集名称
        :param import\_mode: 数据集导入模式
        :type import\_mode: ImportMode or str
        :param bool is\_import\_empty: 是否导入空的数据集,默认为 False,即不导入
        :param bool is\_import\_as\_cad: 是否以 CAD 数据集方式导入
        :param bool is\_import\_by\_layer: 是否在导入后的数据中合并源数据中的 CAD 图层信息,CAD 是以图层信息来存储的,默认为 False,即所有的图层信息都合并到了一个 CAD 数据集, 否则对应源数据中的每一个图层生成一个 CAD 数据集。
        :param bool ignore\_block\_attrs: 是否数据导入时是否忽略块儿属性。默认为 True
        :param bool block\_as\_point: 将符号块导入为点对象还是复合对象,默认为 False, 即将原有的符号块作为复合对象导入,否则在符号块的位置用点对象代替。
        :param bool import\_external\_data: 否导入外部数据,外部数据为 CAD 中类似属性表的数据导入后格式为一些额外的字段,默认为 False,否则将外部数据追加在默认字段后面。
        :param bool import\_xrecord: 是否将用户自定义的字段以及属性字段作为扩展记录导入。
        :param bool import\_invisible\_layer: 是否导入不可见图层
        :param bool keep\_parametric\_part: 是否保留Acad数据中的参数化部分
        :param bool ignore\_lwpline\_width: 是否忽略多义线宽度,默认为 False。
        :param shx\_paths: shx 字体库的路径
        :type shx\_paths: list[str]
        :param int curve\_segment: 曲线拟合精度,默认为 73
        :param str style\_map\_file: 风格对照表的存储路径
        :param function progress: 进度信息处理函数,请参考 :py:class:`.StepEvent`
        :return: 导入后的结果数据集或结果数据集名称
        :rtype: list[DatasetVector] or list[str]
    
    import\_e00(source\_file, output, out\_dataset\_name=None, import\_mode=None, is\_ignore\_attrs=True, source\_file\_charset=None, progress=None)
        导入 E00 文件
        
        :param str source\_file: 被导入的 E00 文件
        :param output: 结果数据源
        :type output: Datasource or DatasourceConnectionInfo or str
        :param str out\_dataset\_name: 结果数据集名称
        :param import\_mode: 数据集导入模式
        :type import\_mode: ImportMode or str
        :param bool is\_ignore\_attrs: 是否忽略属性信息
        :param source\_file\_charset:  E00 文件的原始字符集
        :type source\_file\_charset: Charset or str
        :param function progress: 进度信息处理函数,请参考 :py:class:`.StepEvent`
        :return: 导入后的结果数据集或结果数据集名称
        :rtype: list[DatasetVector] or list[str]
    
    import\_ecw(source\_file, output, out\_dataset\_name=None, ignore\_mode='IGNORENONE', ignore\_values=None, multi\_band\_mode=None, is\_import\_as\_grid=False, progress=None)
        导入 ECW 文件
        
        :param str source\_file: 被导入的 ECW 文件
        :param output: 结果数据源
        :type output: Datasource or DatasourceConnectionInfo or str
        :param str out\_dataset\_name: 结果数据集名称
        :param ignore\_mode: ECW 文件的忽略颜色值的模式
        :type ignore\_mode: IgnoreMode or str
        :param ignore\_values: 要忽略的颜色值
        :type ignore\_values: list[float] 要忽略的颜色值
        :param multi\_band\_mode: 多波段导入模式,可以导入为多个单波段数据集、单个多波段数据集或单个单波段数据集。
        :type multi\_band\_mode: MultiBandImportMode or str
        :param bool is\_import\_as\_grid: 是否导入为 Grid 数据集
        :param function progress: 进度信息处理函数,请参考 :py:class:`.StepEvent`
        :return: 导入后的结果数据集或结果数据集名称
        :rtype: list[DatasetGrid]  or list[DatasetImage] or list[str]
    
    import\_geojson(source\_file, output, out\_dataset\_name=None, import\_mode=None, is\_import\_empty=False, is\_import\_as\_cad=False, source\_file\_charset=None, progress=None)
        导入 GeoJson 文件
        
        :param str source\_file: 被导入的 GeoJson 文件
        :param output: 结果数据源
        :type output: Datasource or DatasourceConnectionInfo or str
        :param str out\_dataset\_name: 结果数据集名称
        :param import\_mode: 导入模式
        :type import\_mode: ImportMode or str
        :param bool is\_import\_empty: 是否导入空的数据集,默认为 False
        :param bool is\_import\_as\_cad: 是否导入为 CAD 数据集
        :param source\_file\_charset: GeoJson 文件的原始字符集类型
        :type source\_file\_charset: Charset or str
        :param function progress: 进度信息处理函数,请参考 :py:class:`.StepEvent`
        :return: 导入后的结果数据集或结果数据集名称
        :rtype: list[DatasetVector] or list[str]
    
    import\_gif(source\_file, output, out\_dataset\_name=None, ignore\_mode='IGNORENONE', ignore\_values=None, world\_file\_path=None, is\_import\_as\_grid=False, is\_build\_pyramid=True, progress=None)
        导入 GIF 文件
        
        :param str source\_file: 被导入的 GIF 文件
        :param output: 结果数据源
        :type output: Datasource or DatasourceConnectionInfo or str
        :param str out\_dataset\_name: 结果数据集名称
        :param ignore\_mode: GIF 文件的忽略颜色值的模式
        :type ignore\_mode: IgnoreMode or str
        :param ignore\_values: 要忽略的颜色值
        :type ignore\_values: list[float] 要忽略的颜色值
        :param str world\_file\_path: 导入的源影像文件的坐标参考文件路径
        :param bool is\_import\_as\_grid: 是否导入为 Grid 数据集
        :param bool is\_build\_pyramid: 是否自动建立影像金字塔
        :param function progress: 进度信息处理函数,请参考 :py:class:`.StepEvent`
        :return: 导入后的结果数据集或结果数据集名称
        :rtype: list[DatasetGrid] or list[DatasetImage] or list[str]
    
    import\_grd(source\_file, output, out\_dataset\_name=None, ignore\_mode='IGNORENONE', ignore\_values=None, is\_build\_pyramid=True, progress=None)
        导入 GRD 文件
        
        :param str source\_file: 被导入的 GRD 文件
        :param output: 结果数据源
        :type output: Datasource or DatasourceConnectionInfo or str
        :param str out\_dataset\_name: 结果数据集名称
        :param ignore\_mode: JPG 文件的忽略颜色值的模式
        :type ignore\_mode: IgnoreMode or str
        :param ignore\_values: 要忽略的颜色值
        :type ignore\_values: list[float] 要忽略的颜色值
        :param bool is\_build\_pyramid: 是否自动建立影像金字塔
        :param function progress: 进度信息处理函数,请参考 :py:class:`.StepEvent`
        :return: 导入后的结果数据集或结果数据集名称
        :rtype: list[DatasetGrid] or list[str]
    
    import\_img(source\_file, output, out\_dataset\_name=None, ignore\_mode='IGNORENONE', ignore\_values=None, multi\_band\_mode=None, is\_import\_as\_grid=False, is\_build\_pyramid=True, progress=None)
        导入 Erdas Image 文件
        
        :param str source\_file: 被导入的 IMG 文件
        :param output: 结果数据源
        :type output: Datasource or DatasourceConnectionInfo or str
        :param str out\_dataset\_name: 结果数据集名称
        :param ignore\_mode: Erdas Image 的忽略颜色值的模式
        :type ignore\_mode: IgnoreMode or str
        :param ignore\_values: 要忽略的颜色值
        :type ignore\_values: list[float] 要忽略的颜色值
        :param multi\_band\_mode: 多波段导入模式,可以导入为多个单波段数据集、单个多波段数据集或单个单波段数据集。
        :type multi\_band\_mode: MultiBandImportMode or str
        :param bool is\_import\_as\_grid: 是否导入为 Grid 数据集
        :param bool is\_build\_pyramid: 是否自动建立影像金字塔
        :param function progress: 进度信息处理函数,请参考 :py:class:`.StepEvent`
        :return: 导入后的结果数据集或结果数据集名称
        :rtype: list[DatasetGrid] or list[DatasetImage] or list[str]
    
    import\_jp2(source\_file, output, out\_dataset\_name=None, ignore\_mode='IGNORENONE', ignore\_values=None, is\_import\_as\_grid=False, progress=None)
        导入 JP2 文件
        
        :param str source\_file: 被导入的 JP2 文件
        :param output: 结果数据源
        :type output: Datasource or DatasourceConnectionInfo or str
        :param str out\_dataset\_name: 结果数据集名称
        :param ignore\_mode: JPG 文件的忽略颜色值的模式
        :type ignore\_mode: IgnoreMode or str
        :param ignore\_values: 要忽略的颜色值
        :type ignore\_values: list[float] 要忽略的颜色值
        :param bool is\_import\_as\_grid: 是否导入为 Grid 数据集
        :param function progress: 进度信息处理函数,请参考 :py:class:`.StepEvent`
        :return: 导入后的结果数据集或结果数据集名称
        :rtype: list[DatasetGrid] or list[DatasetImage] or list[str]
    
    import\_jpg(source\_file, output, out\_dataset\_name=None, ignore\_mode='IGNORENONE', ignore\_values=None, world\_file\_path=None, is\_import\_as\_grid=False, is\_build\_pyramid=True, progress=None)
        导入 JPG 文件
        
        :param str source\_file: 被导入的 JPG 文件
        :param output: 结果数据源
        :type output: Datasource or DatasourceConnectionInfo or str
        :param str out\_dataset\_name: 结果数据集名称
        :param ignore\_mode: JPG 文件的忽略颜色值的模式
        :type ignore\_mode: IgnoreMode or str
        :param ignore\_values: 要忽略的颜色值
        :type ignore\_values: list[float] 要忽略的颜色值
        :param str world\_file\_path: 导入的源影像文件的坐标参考文件路径
        :param bool is\_import\_as\_grid: 是否导入为 Grid 数据集
        :param bool is\_build\_pyramid: 是否自动建立影像金字塔
        :param function progress: 进度信息处理函数,请参考 :py:class:`.StepEvent`
        :return: 导入后的结果数据集或结果数据集名称
        :rtype: list[DatasetGrid] or list[DatasetImage] or list[str]
    
    import\_kml(source\_file, output, out\_dataset\_name=None, import\_mode=None, is\_import\_empty=False, is\_import\_as\_cad=False, is\_ignore\_invisible\_object=True, source\_file\_charset=None, progress=None)
        导入 KML 文件
        
        :param str source\_file: 被导入的 KML 文件
        :param output: 结果数据源
        :type output: Datasource or DatasourceConnectionInfo or str
        :param str out\_dataset\_name: 结果数据集名称
        :type import\_mode: ImportMode or str
        :param bool is\_import\_empty: 是否导入空的数据集
        :param bool is\_import\_as\_cad: 是否以 CAD 数据集方式导入
        :param bool is\_ignore\_invisible\_object: 是否忽略不可见对象
        :param source\_file\_charset: KML 文件的原始字符集
        :type source\_file\_charset: Charset or str
        :param function progress: 进度信息处理函数,请参考 :py:class:`.StepEvent`
        :return: 导入后的结果数据集或结果数据集名称
        :rtype: list[DatasetVector] or list[str]
    
    import\_kmz(source\_file, output, out\_dataset\_name=None, import\_mode=None, is\_import\_empty=False, is\_import\_as\_cad=False, is\_ignore\_invisible\_object=True, source\_file\_charset=None, progress=None)
        导入 KMZ 文件
        
        :param str source\_file: 被导入的 KMZ 文件
        :param output: 结果数据源
        :type output: Datasource or DatasourceConnectionInfo or str
        :param str out\_dataset\_name: 结果数据集名称
        :param import\_mode: 数据集导入模式
        :type import\_mode: ImportMode or str
        :param bool is\_import\_empty: 是否导入空的数据集
        :param bool is\_import\_as\_cad: 是否以 CAD 数据集方式导入
        :param bool is\_ignore\_invisible\_object: 是否忽略不可见对象
        :param source\_file\_charset: KMZ 文件的原始字符集
        :type source\_file\_charset: Charset or str
        :param function progress: 进度信息处理函数,请参考 :py:class:`.StepEvent`
        :return: 导入后的结果数据集或结果数据集名称
        :rtype: list[DatasetVector] or list[str]
    
    import\_mapgis(source\_file, output, out\_dataset\_name=None, import\_mode=None, is\_import\_as\_cad=True, color\_index\_file\_path=None, import\_network\_topology=False, source\_file\_charset=None, progress=None)
        导入 MapGIS 文件,Linux 平台不支持导入 MapGIS 文件。
        
        :param str source\_file: 被导入的 MAPGIS 文件
        :param output: 结果数据源
        :type output: Datasource or DatasourceConnectionInfo or str
        :param str out\_dataset\_name: 结果数据集名称
        :param import\_mode: 数据集导入模式
        :type import\_mode: ImportMode or str
        :param bool is\_import\_as\_cad: 是否以 CAD 数据集方式导入
        :param str color\_index\_file\_path: MAPGIS 导入数据时的颜色索引表文件路径,默认文件路径为系统路径下的 MapGISColor.wat
        :param bool import\_network\_topology: 导入时是否导入网络数据集
        :param source\_file\_charset: MAPGIS 文件的原始字符集
        :type source\_file\_charset: Charset or str
        :param function progress: 进度信息处理函数,请参考 :py:class:`.StepEvent`
        :return: 导入后的结果数据集或结果数据集名称
        :rtype: list[DatasetVector] or list[str]
    
    import\_mif(source\_file, output, out\_dataset\_name=None, import\_mode=None, is\_ignore\_attrs=True, is\_import\_as\_cad=False, style\_map\_file=None, source\_file\_charset=None, progress=None)
        导入 MIF 文件
        
        :param str source\_file: 被导入的 mif 文件
        :param output: 结果数据源
        :type output: Datasource or DatasourceConnectionInfo or str
        :param str out\_dataset\_name: 结果数据集名称
        :param import\_mode: 数据集导入模式
        :type import\_mode: ImportMode or str
        :param bool is\_ignore\_attrs: 导入 MIF 格式数据时是否忽略该数据的属性,包括矢量数据的属性信息。
        :param bool is\_import\_as\_cad: 是否以 CAD 数据集方式导入
        :param source\_file\_charset: mif 文件的原始字符集
        :type source\_file\_charset: Charset or str
        :param str style\_map\_file: 风格对照表的存储路径
        :param function progress: 进度信息处理函数,请参考 :py:class:`.StepEvent`
        :return: 导入后的结果数据集或结果数据集名称
        :rtype: list[DatasetVector] or list[str]
    
    import\_mrsid(source\_file, output, out\_dataset\_name=None, ignore\_mode='IGNORENONE', ignore\_values=None, multi\_band\_mode=None, is\_import\_as\_grid=False, progress=None)
        导入 MrSID 文件, Linux 平台不支持导入 MrSID 文件。
        
        :param str source\_file: 被导入的 MrSID 文件
        :param output: 结果数据源
        :type output: Datasource or DatasourceConnectionInfo or str
        :param str out\_dataset\_name: 结果数据集名称
        :param ignore\_mode: MrSID 文件的忽略颜色值的模式
        :type ignore\_mode: IgnoreMode or str
        :param ignore\_values: 要忽略的颜色值
        :type ignore\_values: list[float] 要忽略的颜色值
        :param multi\_band\_mode: 多波段导入模式,可以导入为多个单波段数据集、单个多波段数据集或单个单波段数据集。
        :type multi\_band\_mode: MultiBandImportMode or str
        :param bool is\_import\_as\_grid: 是否导入为 Grid 数据集
        :param function progress: 进度信息处理函数,请参考 :py:class:`.StepEvent`
        :return: 导入后的结果数据集或结果数据集名称
        :rtype: list[DatasetGrid] or list[DatasetImage] or list[str]
    
    import\_osm(source\_file, output, out\_dataset\_name=None, import\_mode=None, source\_file\_charset=None, progress=None)
        导入 OSM 矢量数据, Linux 平台不支持 OSM 文件
        
        :param str source\_file: 被导入的 OSM 文件
        :param output: 结果数据源
        :type output: Datasource or DatasourceConnectionInfo or str
        :param str out\_dataset\_name: 结果数据集名称
        :param import\_mode: 数据集导入模式
        :type import\_mode: ImportMode or str
        :param source\_file\_charset:  OSM 文件的原始字符集
        :type source\_file\_charset: Charset or str
        :param function progress: 进度信息处理函数,请参考 :py:class:`.StepEvent`
        :return: 导入后的结果数据集或结果数据集名称
        :rtype: list[DatasetVector] or list[str]
    
    import\_png(source\_file, output, out\_dataset\_name=None, ignore\_mode='IGNORENONE', ignore\_values=None, world\_file\_path=None, is\_import\_as\_grid=False, is\_build\_pyramid=True, progress=None)
        导入 Portal Network Graphic (PNG) 文件
        
        :param str source\_file: 被导入的 PNG 文件
        :param output: 结果数据源
        :type output: Datasource or DatasourceConnectionInfo or str
        :param str out\_dataset\_name: 结果数据集名称
        :param ignore\_mode: PNG 文件的忽略颜色值的模式
        :type ignore\_mode: IgnoreMode or str
        :param ignore\_values: 要忽略的颜色值
        :type ignore\_values: list[float] 要忽略的颜色值
        :param str world\_file\_path: 导入的源影像文件的坐标参考文件路径
        :param bool is\_import\_as\_grid: 是否导入为 Grid 数据集
        :param bool is\_build\_pyramid: 是否自动建立影像金字塔
        :param function progress: 进度信息处理函数,请参考 :py:class:`.StepEvent`
        :return: 导入后的结果数据集或结果数据集名称
        :rtype: list[DatasetGrid] or list[DatasetImage] or list[str]
    
    import\_shape(source\_file, output, out\_dataset\_name=None, import\_mode=None, is\_ignore\_attrs=False, is\_import\_empty=False, source\_file\_charset=None, is\_import\_as\_3d=False, progress=None)
        导入 shape 文件到数据源中。
        
        :param str source\_file: 被导入的 shape 文件
        :param output: 结果数据源
        :type output: Datasource or  DatasourceConnectionInfo or str
        :param str out\_dataset\_name: 结果数据集名称
        :param import\_mode: 导入模式类型,可以为 ImportMode 枚举值或名称
        :type import\_mode: ImportMode or str
        :param bool is\_ignore\_attrs: 是否忽略属性信息,默认值为 False
        :param bool is\_import\_empty: 否导入空的数据集,默认是不导入的。默认为 False
        :param source\_file\_charset: shape 文件的原始字符集类型
        :type source\_file\_charset: Charset or str
        :param bool is\_import\_as\_3d: 是否导入为 3D 数据集
        :param function progress: 进度信息处理函数,请参考 :py:class:`.StepEvent`
        :return: 导入后的结果数据集或结果数据集名称
        :rtype: list[DatasetVector] or list[str]
        
        >>> result = import\_shape( 'E:/point.shp', 'E:/import\_shp\_out.udb')
        >>> print(len(result) == 1)
        >>> print(result[0])
    
    import\_simplejson(source\_file, output, out\_dataset\_name=None, import\_mode=None, is\_import\_empty=False, source\_file\_charset=None, progress=None)
        导入 SimpleJson 文件
        
        :param str source\_file: 被导入的 SimpleJson 文件
        :param output: 结果数据源
        :type output: Datasource or DatasourceConnectionInfo or str
        :param str out\_dataset\_name: 结果数据集名称
        :param import\_mode: 数据集导入模式
        :type import\_mode: ImportMode or str
        :param bool is\_import\_empty: 是否导入空数据集
        :param source\_file\_charset:  SimpleJson 文件的原始字符集
        :type source\_file\_charset: Charset or str
        :param function progress: 进度信息处理函数,请参考 :py:class:`.StepEvent`
        :return: 导入后的结果数据集或结果数据集名称
        :rtype: list[DatasetVector] or list[str]
    
    import\_sit(source\_file, output, out\_dataset\_name=None, ignore\_mode='IGNORENONE', ignore\_values=None, multi\_band\_mode=None, is\_import\_as\_grid=False, password=None, progress=None)
        导入 SIT 文件
        
        :param str source\_file: 被导入的 SIT 文件
        :param output: 结果数据源
        :type output: Datasource or DatasourceConnectionInfo or str
        :param str out\_dataset\_name: 结果数据集名称
        :param ignore\_mode: SIT 文件的忽略颜色值的模式
        :type ignore\_mode: IgnoreMode or str
        :param ignore\_values: 要忽略的颜色值
        :type ignore\_values: list[float] 要忽略的颜色值
        :param multi\_band\_mode: 多波段导入模式,可以导入为多个单波段数据集、单个多波段数据集或单个单波段数据集。
        :type multi\_band\_mode: MultiBandImportMode or str
        :param bool is\_import\_as\_grid: 是否导入为 Grid 数据集
        :param str password: 密码
        :param function progress: 进度信息处理函数,请参考 :py:class:`.StepEvent`
        :return: 导入后的结果数据集或结果数据集名称
        :rtype: list[DatasetGrid] or list[DatasetImage] or list[str]
    
    import\_tab(source\_file, output, out\_dataset\_name=None, import\_mode=None, is\_ignore\_attrs=True, is\_import\_empty=False, is\_import\_as\_cad=False, style\_map\_file=None, source\_file\_charset=None, progress=None)
        导入 TAB 文件
        
        :param str source\_file: 被导入的 TAB 文件
        :param output: 结果数据源
        :type output: Datasource or DatasourceConnectionInfo or str
        :param str out\_dataset\_name: 结果数据集名称
        :param import\_mode: 数据集导入模式
        :type import\_mode: ImportMode or str
        :param bool is\_ignore\_attrs: 导入 TAB 格式数据时是否忽略该数据的属性,包括矢量数据的属性信息。
        :param bool is\_import\_empty: 是否导入空的数据集,默认为 False,即不导入
        :param bool is\_import\_as\_cad: 是否以 CAD 数据集方式导入
        :param source\_file\_charset: mif 文件的原始字符集
        :type source\_file\_charset: Charset or str
        :param str style\_map\_file: 风格对照表的存储路径
        :param function progress: 进度信息处理函数,请参考 :py:class:`.StepEvent`
        :return: 导入后的结果数据集或结果数据集名称
        :rtype: list[DatasetVector] or list[str]
    
    import\_tif(source\_file, output, out\_dataset\_name=None, ignore\_mode='IGNORENONE', ignore\_values=None, multi\_band\_mode=None, world\_file\_path=None, is\_import\_as\_grid=False, is\_build\_pyramid=True, progress=None)
        导入 TIF 文件
        
        :param str source\_file: 被导入的 TIF 文件
        :param output: 结果数据源
        :type output: Datasource or DatasourceConnectionInfo or str
        :param str out\_dataset\_name: 结果数据集名称
        :param ignore\_mode: Tiff/BigTIFF/GeoTIFF 文件的忽略颜色值的模式
        :type ignore\_mode: IgnoreMode or str
        :param ignore\_values: 要忽略的颜色值
        :type ignore\_values: list[float] 要忽略的颜色值
        :param multi\_band\_mode: 多波段导入模式,可以导入为多个单波段数据集、单个多波段数据集或单个单波段数据集。
        :type multi\_band\_mode: MultiBandImportMode or str
        :param str world\_file\_path: 导入的源影像文件的坐标参考文件路径
        :param bool is\_import\_as\_grid: 是否导入为 Grid 数据集
        :param bool is\_build\_pyramid: 是否自动建立影像金字塔
        :param function progress: 进度信息处理函数,请参考 :py:class:`.StepEvent`
        :return: 导入后的结果数据集或结果数据集名称
        :rtype: list[DatasetGrid] or list[DatasetImage] or list[str]
    
    import\_usgsdem(source\_file, output, out\_dataset\_name=None, ignore\_mode='IGNORENONE', ignore\_values=None, is\_build\_pyramid=True, progress=None)
        导入 USGSDEM 文件
        
        :param str source\_file: 被导入的 JPG 文件
        :param output: 结果数据源
        :type output: Datasource or DatasourceConnectionInfo or str
        :param str out\_dataset\_name: 结果数据集名称
        :param ignore\_mode: JPG 文件的忽略颜色值的模式
        :type ignore\_mode: IgnoreMode or str
        :param ignore\_values: 要忽略的颜色值
        :type ignore\_values: list[float] 要忽略的颜色值
        :param bool is\_build\_pyramid: 是否自动建立影像金字塔
        :param function progress: 进度信息处理函数,请参考 :py:class:`.StepEvent`
        :return: 导入后的结果数据集或结果数据集名称
        :rtype: list[DatasetGrid] or list[str]
    
    import\_vct(source\_file, output, out\_dataset\_name=None, import\_mode=None, is\_import\_empty=False, source\_file\_charset=None, layers=None, progress=None)
        导入 VCT 文件
        
        :param str source\_file: 被导入的 VCT 文件
        :param output: 结果数据源
        :type output: Datasource or DatasourceConnectionInfo or str
        :param str out\_dataset\_name: 结果数据集名称
        :param import\_mode: 数据集导入模式
        :type import\_mode: ImportMode or str
        :param bool is\_import\_empty: 是否导入空数据集
        :param source\_file\_charset:  VCT 文件的原始字符集
        :type source\_file\_charset: Charset or str
        :param layers: 需要导入的图层名称,设置为 None 时将全部导入。
        :type layers: str or list[str]
        :param function progress: 进度信息处理函数,请参考 :py:class:`.StepEvent`
        :return: 导入后的结果数据集或结果数据集名称
        :rtype: list[DatasetVector] or list[str]

DATA
    \_\_all\_\_ = ['import\_shape', 'import\_dbf', 'import\_csv', 'import\_mapgis'{\ldots}

FILE
    /opt/conda/lib/python3.6/site-packages/iobjectspy/conversion.py



    \end{Verbatim}

    \begin{Verbatim}[commandchars=\\\{\}]
{\color{incolor}In [{\color{incolor}8}]:} \PY{n}{help}\PY{p}{(}\PY{n}{smo}\PY{o}{.}\PY{n}{analyst}\PY{p}{)}
\end{Verbatim}


    \begin{Verbatim}[commandchars=\\\{\}]
Help on module iobjectspy.analyst in iobjectspy:

NAME
    iobjectspy.analyst

DESCRIPTION
    ananlyst 模块提供了常用的空间数据处理和分析的功能,用户使用analyst 模块可以进行缓冲区分析( :py:meth:`create\_buffer` )、叠加分析( :py:meth:`overlay` )、
    创建泰森多边形( :py:meth:`create\_thiessen\_polygons` )、拓扑构面( :py:meth:`topology\_build\_regions` )、密度聚类( :py:meth:`kernel\_density` )、
    插值分析( :py:meth:`interpolate` ),栅格代数运算( :py:meth:`expression\_math\_analyst` )等功能。
    
    
    在 analyst 模块的所有接口中,对输入数据参数要求为数据集( :py:class:`.Dataset` , :py:class:`.DatasetVector` , :py:class:`.DatasetImage` , :py:class:`.DatasetGrid` )的参数,
    都接受直接输入一个数据集对象(Dataset)或数据源别名与数据集名称的组合(例如,'alias/dataset\_name', 'alias\textbackslash{}\textbackslash{}\textbackslash{}dataset\_name'),也支持数据源连接信息与数据集名称的组合(例如,'E:/data.udb/dataset\_name')。
    
        - 支持设置数据集
    
            >>> ds = Datasource.open('E:/data.udb')
            >>> create\_buffer(ds['point'], 10, 10, unit='Meter', out\_data='E:/buffer\_out.udb')
    
        - 支持设置数据集别名和数据集名称组合
    
            >>> create\_buffer(ds.alias + '/point' + , 10, 10, unit='Meter', out\_data='E:/buffer\_out.udb')
            >>> create\_buffer(ds.alias + '\textbackslash{}\textbackslash{}point', 10, 10, unit='Meter', out\_data='E:/buffer\_out.udb')
            >>> create\_buffer(ds.alias + '|point', 10, 10, unit='Meter', out\_data='E:/buffer\_out.udb')
    
        - 支持设置 udb 文件路径和数据集名称组合
    
            >>> create\_buffer('E:/data.udb/point', 10, 10, unit='Meter', out\_data='E:/buffer\_out.udb')
    
        - 支持设置数据源连接信息和数据集名称组合,数据源连接信息包括 dcf 文件、xml 字符串等,具体参考 :py:meth:`.DatasourceConnectionInfo.make`
    
            >>> create\_buffer('E:/data\_ds.dcf/point', 10, 10, unit='Meter', out\_data='E:/buffer\_out.udb')
    
    .. Note:: 当输入的是数据源信息时,程序会自动打开数据源,但是接口运行结束时不会自动关闭数据源,也就是打开后的数据源会存在当前工作空间中
    
    
    在 analyst 模块中所有接口中,对输出数据参数要求为数据源( :py:class:`.Datasource` )的,均接受 Datasource 对象,也可以为 :py:class:`.DatasourceConnectionInfo` 对象,
    同时,也支持当前工作空间下数据源的别名,也支持 UDB 文件路径,DCF 文件路径等。
    
        - 支持设置 udb 文件路径
    
            >>> create\_buffer('E:/data.udb/point', 10, 10, unit='Meter', out\_data='E:/buffer\_out.udb')
    
        - 支持设置数据源对象
    
            >>> ds = Datasource.open('E:/buffer\_out.udb')
            >>> create\_buffer('E:/data.udb/point', 10, 10, unit='Meter', out\_data=ds)
            >>> ds.close()
    
        - 支持设置数据源别名
    
            >>> ds\_conn = DatasourceConnectionInfo('E:/buffer\_out.udb', alias='my\_datasource')
            >>> create\_buffer('E:/data.udb/point', 10, 10, unit='Meter', out\_data='my\_datasource')
    
    
    .. Note:: 如果输出数据的参数输入的是数据源连接信息或 UDB 文件路径等,程序会自动打开数据源,如果是 UDB 数据源而本地不存在,还会自动新建一个UDB数据源,但需要确保UDB数据源所在的文件目录存在而且可写。
              在功能完成后,如果数据源是由程序自动打开或创建的,会被自动关闭掉(这里与输入数据为Dataset不同,输入数据中被自动打开的数据源不会自动关闭)。所以,对于有些接口
              输出结果为数据集的,就会返回结果数据集的名称,如果传入的是数据源对象,返回的便是结果数据集。

CLASSES
    builtins.object
        iobjectspy.\_jsuperpy.analyst.AnalyzingPatternsResult
            iobjectspy.\_jsuperpy.analyst.IncrementalResult
        iobjectspy.\_jsuperpy.analyst.BasicStatisticsAnalystResult
        iobjectspy.\_jsuperpy.analyst.GWRSummary
        iobjectspy.\_jsuperpy.analyst.GridHistogram
        iobjectspy.\_jsuperpy.analyst.NeighbourShape
            iobjectspy.\_jsuperpy.analyst.NeighbourShapeAnnulus
            iobjectspy.\_jsuperpy.analyst.NeighbourShapeCircle
            iobjectspy.\_jsuperpy.analyst.NeighbourShapeRectangle
            iobjectspy.\_jsuperpy.analyst.NeighbourShapeWedge
        iobjectspy.\_jsuperpy.analyst.PreprocessOptions
        iobjectspy.\_jsuperpy.analyst.ProcessingOptions
        iobjectspy.\_jsuperpy.analyst.ReclassMappingTable
        iobjectspy.\_jsuperpy.analyst.ReclassSegment
        iobjectspy.\_jsuperpy.analyst.StatisticsField
    iobjectspy.\_jsuperpy.analyst.InterpolationParameter(builtins.object)
        iobjectspy.\_jsuperpy.analyst.InterpolationDensityParameter
        iobjectspy.\_jsuperpy.analyst.InterpolationIDWParameter
        iobjectspy.\_jsuperpy.analyst.InterpolationKrigingParameter
        iobjectspy.\_jsuperpy.analyst.InterpolationRBFParameter
    
    class AnalyzingPatternsResult(builtins.object)
     |  分析模式结果类。该类用于获取分析模式计算的结果,包括结果指数、期望、方差、Z得分和P值等。
     |  
     |  Methods defined here:
     |  
     |  \_\_init\_\_(self)
     |      Initialize self.  See help(type(self)) for accurate signature.
     |  
     |  \_\_str\_\_(self)
     |      Return str(self).
     |  
     |  ----------------------------------------------------------------------
     |  Data descriptors defined here:
     |  
     |  \_\_dict\_\_
     |      dictionary for instance variables (if defined)
     |  
     |  \_\_weakref\_\_
     |      list of weak references to the object (if defined)
     |  
     |  expectation
     |      float: 分析模式结果中的期望值
     |  
     |  index
     |      float: 分析模式结果中的莫兰指数或GeneralG指数
     |  
     |  p\_value
     |      float: 分析模式结果中的P值
     |  
     |  variance
     |      float: 分析模式结果中的方差值
     |  
     |  z\_score
     |      float: 分析模式结果中的Z得分
    
    class BasicStatisticsAnalystResult(builtins.object)
     |  栅格基本统计分析结果类
     |  
     |  Methods defined here:
     |  
     |  \_\_init\_\_(self)
     |      Initialize self.  See help(type(self)) for accurate signature.
     |  
     |  \_\_str\_\_(self)
     |      Return str(self).
     |  
     |  to\_dict(self)
     |      输出为 dict 对象
     |      
     |      :rtype: dict
     |  
     |  ----------------------------------------------------------------------
     |  Data descriptors defined here:
     |  
     |  \_\_dict\_\_
     |      dictionary for instance variables (if defined)
     |  
     |  \_\_weakref\_\_
     |      list of weak references to the object (if defined)
     |  
     |  first\_quartile
     |      float: 栅格基本统计分析计算所得的第一四分值
     |  
     |  kurtosis
     |      float: 栅格基本统计分析计算所得的峰度
     |  
     |  max
     |      float: 栅格基本统计分析计算所得的最大值
     |  
     |  mean
     |      float: 栅格基本统计分析计算所得的最小值
     |  
     |  median
     |      float: 栅格基本统计分析计算所得的中位数
     |  
     |  min
     |      float:
     |  
     |  skewness
     |      float: 栅格基本统计分析计算所得的偏度
     |  
     |  std
     |      float: 栅格基本统计分析计算所得的均方差(标准差)
     |  
     |  third\_quartile
     |      float: 栅格基本统计分析计算所得的第三四分值
    
    class GWRSummary(builtins.object)
     |  地理加权回归结果汇总类。该类给出了地理加权回归分析的结果汇总,例如带宽、相邻数、残差平方和、AICc和判定系数等。
     |  
     |  Methods defined here:
     |  
     |  \_\_init\_\_(self)
     |      Initialize self.  See help(type(self)) for accurate signature.
     |  
     |  \_\_str\_\_(self)
     |      Return str(self).
     |  
     |  ----------------------------------------------------------------------
     |  Data descriptors defined here:
     |  
     |  AIC
     |      float: 地理加权回归结果汇总中的AIC。与AICc类似,是衡量模型拟合优良性的一种标准,可以权衡所估计模型的复杂度和模型拟
     |      合数据的优良性,在评价模型时是兼顾了简洁性和精确性。表明增加自由参数的数目提高了拟合的优良性,AIC鼓励数据的
     |      拟合性,但是应尽量避免出现过度拟合的情况。所以优先考虑AIC值较小的,是寻找可以最好的解释数据但包含最少自由参
     |      数的模型。
     |  
     |  AICc
     |      float: 地理加权回归结果汇总中的AICc。当数据增加时,AICc收敛为AIC,也是模型性能的一种度量,有助与比较不同的回归模型。
     |      考虑到模型复杂性,具有较低AICc值的模型将更好的拟合观测数据。AICc不是拟合度的绝对度量,但对于比较用于同一因变
     |      量且具有不同解释变量的模型非常有用。如果两个模型的AICc值相差大于3,具有较低AICc值的模型将视为更佳的模型。
     |  
     |  Edf
     |      float: 地理加权回归结果汇总中的有效自由度。数据的数目与有效的参数数量(EffectiveNumber)的差值,不一定是整数,可用
     |      来计算多个诊断测量值。自由度较大的模型拟合度会较差,能够较好的反应数据的真实情况,统计量会变得比较可靠;反之,
     |      拟合效果会较好,但是不能较好的反应数据的真实情况,模型数据的独立性被削弱,关联度增加。
     |  
     |  R2
     |      float: 地理加权回归结果汇总中的判定系数(R2)。判定系数是拟合度的一种度量,其值在0.0和1.0范围内变化,值越大模型越好。
     |      此值可解释为回归模型所涵盖的因变量方差的比例。R2计算的分母为因变量值的平方和,添加一个解释变量不会更改分母但是
     |      会更改分子,这将出现改善模型拟合的情况,但是也可能假象。
     |  
     |  R2\_adjusted
     |      float: 地理加权回归结果汇总中的校正的判定系数。校正的判定系数的计算将按分子和分母的自由度对它们进行正规化。这具有对
     |      模型中变量数进行补偿的效果,由于校正的R2值通常小于R2值。但是,执行校正时,无法将该值的解释作为所解释方差的比例。
     |      自由度的有效值是带宽的函数,因此,AICc是对模型进行比较的首选方式。
     |  
     |  \_\_dict\_\_
     |      dictionary for instance variables (if defined)
     |  
     |  \_\_weakref\_\_
     |      list of weak references to the object (if defined)
     |  
     |  band\_width
     |      float: 地理加权回归结果汇总中的带宽范围。
     |      
     |      * 用于各个局部估计的带宽范围,它控制模型中的平滑程度。通常,你可以选择默认的带宽范围,方法是:设置带宽确定
     |        方式(kernel\_type)方法选择 :py:attr:`.BandWidthType.AICC` 或 :py:attr:`.BandWidthType.CV`,这两个选项都将尝试识别最佳带宽范围。
     |      
     |      * 由于"最佳"条件对于 AIC 和 CV 并不相同,都会得到相对的最优 AICc 值和 CV 值,因而通常会获得不同的最佳值。
     |      
     |      * 可以通过设置带宽类型(kernel\_type)方法提供精确的带宽范围。
     |  
     |  effective\_number
     |      float: 地理加权回归结果汇总中的有效的参数数量。反映了估计值的方差与系数估计值的偏差之间的折衷,该值与带宽的选择有关,
     |      可用来计算多个诊断测量值。对于较大的带宽,系数的有效数量将接近实际参数数量,局部系数估计值将具有较小的方差,
     |      但是偏差将会非常大;对于较小的带宽,系数的有效数量将接近观测值的数量,局部系数估计值将具有较大的方差,但是偏
     |      差将会变小。
     |  
     |  neighbours
     |      int: 地理加权回归结果汇总中的相邻数目。
     |      
     |      * 用于各个局部估计的相邻数目,它控制模型中的平滑程度。通常,你可以选择默认的相邻点值,方法是:设置带宽确定方式(kernel\_type)
     |        方法选择 :py:attr:`.BandWidthType.AICC` 或 :py:attr:`.BandWidthType.CV`,这两个选项都将尝试识别最佳自适应相邻点数目。
     |      
     |      * 由于"最佳"条件对于 AIC 和 CV 并不相同,都会得到相对的最优 AICc 值和 CV 值,因而通常会获得不同的最佳值。
     |      
     |      * 可以通过设置带宽类型(kernel\_type)方法提供精确的自适应相邻点数目。
     |  
     |  residual\_squares
     |      float: 地理加权回归结果汇总中的残差平方和。残差平方和为实际值与估计值(或拟合值)的平方之和。此测量值越小,模型越
     |      拟合观测数据,即拟合程度越好。
     |  
     |  sigma
     |      float: 地理加权回归结果汇总中的残差估计标准差。残差的估计标准差,为剩余平方和除以残差的有效自由度的平方根。此统计值
     |      越小,模型拟合效果越好。
    
    class GridHistogram(builtins.object)
     |  创建给定栅格数据集的直方图。
     |  
     |  直方图,又称柱状图,由一系列高度不等的矩形块来表示一份数据的分布情况。一般横轴表示类别,纵轴表示分布情况。
     |  
     |  栅格直方图的横轴表示栅格值的分组,栅格值将被划分到这 N(默认为 100)个组中,即每个组对应着一个栅格值范围;纵轴表示频数,即
     |  栅格值在每组的值范围内的单元格的个数。
     |  
     |  下图是栅格直方图的示意图。该栅格数据的最小值和最大值分别为 0 和 100,取组数为 10,得出每组的频数,绘制如下的直方图。矩形块
     |  上方标注了该组的频数,例如,第 6 组的栅格值范围为 [50,60),栅格数据中值在此范围内的单元格共有 3 个,因此该组的频数为 3。
     |  
     |  .. image:: ../image/BuildHistogram.png
     |  
     |  注:直方图分组的最后一组的值范围为前闭后闭,其余均为前闭后开。
     |  
     |  在通过此方法获得栅格数据集的直方图(GridHistogram)对象后,可以通过该对象的 get\_frequencies 方法返回每个组的频数,还可以通过
     |  get\_group\_count 方法重新指定栅格直方图的组数,然后再通过 get\_frequencies 方法返回每组的频数。
     |  
     |  下图为创建栅格直方图的一个实例。本例中,最小栅格值为 250,最大栅格值为 1243,组数为 500,获取各组的频数,绘制出如右侧所示的
     |  栅格直方图。从右侧的栅格直方图,可以非常直观的了解栅格数据集栅格值的分布情况。
     |  
     |  .. image:: ../image/BuildHistogram\_1.png
     |  
     |  Methods defined here:
     |  
     |  \_\_init\_\_(self, source\_data, group\_count, function\_type=None, progress=None)
     |      构造栅格直方图对象
     |      
     |      :param source\_data:  指定的栅格数据集
     |      :type source\_data: DatasetGrid or str
     |      :param group\_count:  指定的直方图的组数。必须大于 0。
     |      :type group\_count: int
     |      :param function\_type: FunctionType
     |      :type function\_type: 指定的变换函数类型。
     |      :param progress: 进度信息处理函数,具体参考 :py:class:`.StepEvent`
     |      :type progress: function
     |  
     |  get\_frequencies(self)
     |      返回栅格直方图每个组的频数。直方图的每个组都对应了一个栅格值范围,值在这个范围内的所有单元格的个数即为该组的频数。
     |      
     |      :return:  返回栅格直方图每个组的频数。
     |      :rtype: list[int]
     |  
     |  get\_group\_count(self)
     |      返回栅格直方图横轴上的组数。
     |      
     |      :return: 返回栅格直方图横轴上的组数。
     |      :rtype: int
     |  
     |  get\_segments(self)
     |      返回栅格直方图每个组的区间信息。
     |      
     |      :return:  栅格直方图每个组的区间信息。
     |      :rtype: list[GridHistogram.HistogramSegmentInfo]
     |  
     |  set\_group\_count(self, count)
     |      设置栅格直方图横轴上的组数。
     |      
     |      :param int count: 栅格直方图横轴上的组数。必须大于 0。
     |      :rtype: self
     |  
     |  ----------------------------------------------------------------------
     |  Data descriptors defined here:
     |  
     |  \_\_dict\_\_
     |      dictionary for instance variables (if defined)
     |  
     |  \_\_weakref\_\_
     |      list of weak references to the object (if defined)
     |  
     |  ----------------------------------------------------------------------
     |  Data and other attributes defined here:
     |  
     |  HistogramSegmentInfo = <class 'iobjectspy.\_jsuperpy.analyst.GridHistog{\ldots}
     |      栅格直方图每个分段区间的信息类。
    
    class IncrementalResult(AnalyzingPatternsResult)
     |  增量空间自相关结果类。该类用于获取增量空间自相关计算的结果,包括结果增量距离、莫兰指数、期望、方差、Z得分和P值等。
     |  
     |  Method resolution order:
     |      IncrementalResult
     |      AnalyzingPatternsResult
     |      builtins.object
     |  
     |  Methods defined here:
     |  
     |  \_\_init\_\_(self)
     |      Initialize self.  See help(type(self)) for accurate signature.
     |  
     |  \_\_str\_\_(self)
     |      Return str(self).
     |  
     |  ----------------------------------------------------------------------
     |  Data descriptors defined here:
     |  
     |  distance
     |      float: 增量空间自相关结果中的增量距离
     |  
     |  ----------------------------------------------------------------------
     |  Data descriptors inherited from AnalyzingPatternsResult:
     |  
     |  \_\_dict\_\_
     |      dictionary for instance variables (if defined)
     |  
     |  \_\_weakref\_\_
     |      list of weak references to the object (if defined)
     |  
     |  expectation
     |      float: 分析模式结果中的期望值
     |  
     |  index
     |      float: 分析模式结果中的莫兰指数或GeneralG指数
     |  
     |  p\_value
     |      float: 分析模式结果中的P值
     |  
     |  variance
     |      float: 分析模式结果中的方差值
     |  
     |  z\_score
     |      float: 分析模式结果中的Z得分
    
    class InterpolationDensityParameter(InterpolationParameter)
     |  点密度差值(Density)插值参数类。点密度插值方法,用于表达采样点的密度分布情况。
     |  点密度插值的结果栅格的分辨率设置需要结合点数据集范围大小来取值,一般结果栅格行列值(即结果栅格数据集范围除以分辨率)在 500
     |  以内即可以较好的体现出密度走势。由于点密度插值暂时只支持定长搜索模式,因此搜索半径(search\_radius)值设置较为重要,此值需要用户根据待插值点数据分布状况和点数据集范围进行设置。
     |  
     |  Method resolution order:
     |      InterpolationDensityParameter
     |      InterpolationParameter
     |      builtins.object
     |  
     |  Methods defined here:
     |  
     |  \_\_init\_\_(self, resolution, search\_radius=0.0, expected\_count=12, bounds=None)
     |      构造点密度差值插值参数类对象
     |      
     |      :param float resolution: 插值运算时使用的分辨率
     |      :param float search\_radius: 查找参与运算点的查找半径
     |      :param int expected\_count: 期望参与插值运算的点数
     |      :param Rectangle bounds: 插值分析的范围,用于确定运行结果的范围
     |  
     |  set\_expected\_count(self, value)
     |      设置期望参与插值运算的点数
     |      
     |      :param int value: 表示期望参与运算的最少样点数
     |      :return: self
     |      :rtype: InterpolationDensityParameter
     |  
     |  set\_search\_radius(self, value)
     |      设置查找参与运算点的查找半径。单位与用于插值的点数据集(或记录集所属的数据集)的单位相同。查找半径决定了参与运算点的查找范围,当计算某个位置
     |      的未知数值时,会以该位置为圆心,以search\_radius为半径,落在这个范围内的采样点都将参与运算,即该位置的预测值由该范围内采样点的数值决定。
     |      
     |      :param float value: 查找参与运算点的查找半径
     |      :return: self
     |      :rtype: InterpolationDensityParameter
     |  
     |  ----------------------------------------------------------------------
     |  Data descriptors defined here:
     |  
     |  expected\_count
     |      int: 返回期望参与插值运算的点数,表示期望参与运算的最少样点数
     |  
     |  search\_mode
     |      SearchMode: 在插值运算时,查找参与运算点的方式,只支持定长查找(KDTREE\_FIXED\_RADIUS)方式
     |  
     |  search\_radius
     |      float: 查找参与运算点的查找半径
     |  
     |  ----------------------------------------------------------------------
     |  Methods inherited from InterpolationParameter:
     |  
     |  set\_bounds(self, value)
     |      设置插值分析的范围,用于确定运行结果的范围
     |      
     |      :param Rectangle value: 插值分析的范围,用于确定运行结果的范围
     |      :return: self
     |      :rtype: InterpolationParameter
     |  
     |  set\_resolution(self, value)
     |      设置插值运算时使用的分辨率。
     |      
     |      :param float value: 插值运算时使用的分辨率
     |      :return: self
     |      :rtype: InterpolationParameter
     |  
     |  ----------------------------------------------------------------------
     |  Data descriptors inherited from InterpolationParameter:
     |  
     |  \_\_dict\_\_
     |      dictionary for instance variables (if defined)
     |  
     |  \_\_weakref\_\_
     |      list of weak references to the object (if defined)
     |  
     |  bounds
     |      Rectangle: 插值分析的范围,用于确定运行结果的范围
     |  
     |  resolution
     |      float: 插值运算时使用的分辨率
     |  
     |  type
     |      InterpolationAlgorithmType: 插值分支所支持的算法的类型
    
    class InterpolationIDWParameter(InterpolationParameter)
     |  距离反比权值插值(Inverse Distance Weighted)参数类,
     |  
     |  Method resolution order:
     |      InterpolationIDWParameter
     |      InterpolationParameter
     |      builtins.object
     |  
     |  Methods defined here:
     |  
     |  \_\_init\_\_(self, resolution, search\_mode=SearchMode.KDTREE\_FIXED\_COUNT, search\_radius=0.0, expected\_count=12, power=1, bounds=None)
     |      构造 IDW 插值参数类。
     |      
     |      :param float resolution: 插值运算时使用的分辨率
     |      :param search\_mode: 查找方式,不支持 QUADTREE
     |      :type search\_mode: SearchMode or str
     |      :param float search\_radius: 查找参与运算点的查找半径
     |      :param int expected\_count: 期望参与插值运算的点数
     |      :param int power: 距离权重计算的幂次,幂次值越低,内插结果越平滑,幂次值越高,内插结果细节越详细。此参数应为一个大于0的值。如果不指定此参数,方法缺省将其设置为1
     |      :param Rectangle bounds: 插值分析的范围,用于确定运行结果的范围
     |  
     |  set\_expected\_count(self, value)
     |      设置期望参与插值运算的点数。如果设置 search\_mode 为 KDTREE\_FIXED\_RADIUS ,同时指定参与插值运算点的个数,当查找范围内的点数小于指定的点数时赋为空值。
     |      
     |      :param int value: 表示期望参与运算的最少样点数
     |      :return: self
     |      :rtype: InterpolationIDWParameter
     |  
     |  set\_power(self, value)
     |      设置距离权重计算的幂次。幂次值越低,内插结果越平滑,幂次值越高,内插结果细节越详细。此参数应为一个大于0的值。如果不指定此参数,方法缺省
     |      将其设置为1。
     |      
     |      :param int value: 距离权重计算的幂次
     |      :return: self
     |      :rtype: InterpolationIDWParameter
     |  
     |  set\_search\_mode(self, value)
     |      设置在插值运算时,查找参与运算点的方式。不支持 QUADTREE
     |      
     |      :param value: 在插值运算时,查找参与运算点的方式
     |      :type value:  SearchMode or str
     |      :return: self
     |      :rtype: InterpolationIDWParameter
     |  
     |  set\_search\_radius(self, value)
     |      设置查找参与运算点的查找半径。单位与用于插值的点数据集(或记录集所属的数据集)的单位相同。查找半径决定了参与运算点的查找范围,当计算某个位置
     |      的未知数值时,会以该位置为圆心,以search\_radius为半径,落在这个范围内的采样点都将参与运算,即该位置的预测值由该范围内采样点的数值决定。
     |      
     |      如果设置 search\_mode 为KDTREE\_FIXED\_COUNT,同时指定查找参与运算点的范围,当查找范围内的点数小于指定的点数时赋为空值,当查找范围内的点数
     |      大于指定的点数时,则返回距离插值点最近的指定个数的点进行插值。
     |      
     |      :param float value: 查找参与运算点的查找半径
     |      :return: self
     |      :rtype: InterpolationIDWParameter
     |  
     |  ----------------------------------------------------------------------
     |  Data descriptors defined here:
     |  
     |  expected\_count
     |      int: 期望参与插值运算的点数,如果设置 search\_mode 为 KDTREE\_FIXED\_RADIUS ,同时指定参与插值运算点的个数,当查找范围内的点数小于指定的点数时赋为空值。
     |  
     |  power
     |      int: 距离权重计算的幂次
     |  
     |  search\_mode
     |      SearchMode: 在插值运算时,查找参与运算点的方式,不支持 QUADTREE
     |  
     |  search\_radius
     |      float: 查找参与运算点的查找半径
     |  
     |  ----------------------------------------------------------------------
     |  Methods inherited from InterpolationParameter:
     |  
     |  set\_bounds(self, value)
     |      设置插值分析的范围,用于确定运行结果的范围
     |      
     |      :param Rectangle value: 插值分析的范围,用于确定运行结果的范围
     |      :return: self
     |      :rtype: InterpolationParameter
     |  
     |  set\_resolution(self, value)
     |      设置插值运算时使用的分辨率。
     |      
     |      :param float value: 插值运算时使用的分辨率
     |      :return: self
     |      :rtype: InterpolationParameter
     |  
     |  ----------------------------------------------------------------------
     |  Data descriptors inherited from InterpolationParameter:
     |  
     |  \_\_dict\_\_
     |      dictionary for instance variables (if defined)
     |  
     |  \_\_weakref\_\_
     |      list of weak references to the object (if defined)
     |  
     |  bounds
     |      Rectangle: 插值分析的范围,用于确定运行结果的范围
     |  
     |  resolution
     |      float: 插值运算时使用的分辨率
     |  
     |  type
     |      InterpolationAlgorithmType: 插值分支所支持的算法的类型
    
    class InterpolationKrigingParameter(InterpolationParameter)
     |  克吕金(Kriging)内插法参数。
     |  
     |  Kriging 法为地质统计学上一种空间资料内插处理方法,主要的目的是利用各数据点间变异数(variance)的大小来推求某一未知点与各已知点的权重关系,再
     |  由各数据点的值和其与未知点的权重关系推求未知点的值。Kriging 法最大的特色不仅是提供一个最小估计误差的预测值,并且可明确的指出误差值的大小。一般
     |  而言,许多地质参数,如地形面本身即具有连续的性质,故在一短距离内的任两点必有空间上的关系。反之,在一不规则面上的两点若相距甚远,则在统计意义上可
     |  视为互为独立 (stastically indepedent),这种随距离而改变的空间上连续性,可用半变异图 (semivariogram) 来表现。因此,若想由已知的散乱点来
     |  推求某一未知点的值,则可利用半变异图推求各已知点及欲求值点的空间关系。再由此空间参数推求半变异数,由各数据点间的半变异数可推求未知点与已知点间的
     |  权重关系,进而推求出未知点的值。克吕金法的优点是以空间统计学作为其坚实的理论基础。物理含义明确;不但能估计测定参数的空间变异分布,而且还可以估算
     |  参数的方差分布。克吕金法的缺点是计算步骤较烦琐,计算量大,且变异函数有时需要根据经验人为选定。
     |  
     |  克吕金插值法可以采用两种方式来获取参与插值的采样点,进而获得相应位置点的预测值,一个是在待计算预测值位置点周围一定范围内,获取该范围内的所有采样
     |  点,通过特定的插值计算公式获得该位置点的预测值;另一个是在待计算预测值位置点周围获取一定数目的采样点,通过特定的插值计算公式获得该位置点的预测值。
     |  
     |  克吕金插值过程是一个多步骤的处理过程,包括:
     |      - 创建变异图和协方差函数来估计统计相关(也称为空间自相关)的值;
     |      - 预测待计算位置点的未知值。
     |  
     |  半变异函数与半变异图:
     |      - 计算所有采样点中相距 h 个单位的任意两点的半变异函数值,那么任意两点的距离 h 一般是唯一的,将所有的点对的距离与相应的半变函数值快速显示在以 h
     |        为 X 坐标轴和以半变函数值为 Y 坐标轴的坐标空间内,就得到了半变异图。相距距离愈小的点其半变异数愈小,而随着距离的增加,任两点间的空间相依关系愈
     |        小,使得半变异函数值趋向于一稳定值。此稳定值我们称之为基台值(Sill);而达到基台值时的最小 h 值称之为自相关阈值(Range)。
     |  
     |  块金效应:
     |      - 当点间距离为 0(比如,步长=0)时,半变函数值为 0。然而,在一个无限小的距离内,半变函数通常显示出块金效应,这是一个大于 0 的值。如果半变函数
     |        在Y周上的截距式 2 ,则块金效应值为 2。
     |      - 块金效应属于测量误差,或者是小于采样步长的小距离上的空间变化,或者两者兼而有之。测量误差主要是由于观测仪器的内在误差引起的。自然现象的空间变异
     |        范围很大(可以在很小的尺度上,也可以在很大的尺度上)。小于步长尺度上的变化就表现为块金的一部分。
     |  
     |  半变异图的获得是进行空间插值预测的关键步骤之一,克吕金法的主要应用之一就是预测非采样点的属性值,半变异图提供了采样点的空间自相关信息,根据半变
     |  异图,选择合适的半变异模型,即拟合半变异图的曲线模型。
     |  
     |  不同的模型将会影响所获得的预测结果,如果接近原点处的半变异函数曲线越陡,则较近领域对该预测值的影响就越大。因此输出表面就会越不光滑。
     |  
     |  SuperMap 支持的半变函数模型有指数型、球型和高斯型。详细信息参见 VariogramMode 类
     |  
     |  Method resolution order:
     |      InterpolationKrigingParameter
     |      InterpolationParameter
     |      builtins.object
     |  
     |  Methods defined here:
     |  
     |  \_\_init\_\_(self, resolution, krighing\_type=InterpolationAlgorithmType.KRIGING, search\_mode=SearchMode.KDTREE\_FIXED\_COUNT, search\_radius=0.0, expected\_count=12, max\_point\_count\_in\_node=50, max\_point\_count\_for\_interpolation=200, variogram=VariogramMode.SPHERICAL, angle=0.0, mean=0.0, exponent=Exponent.EXP1, nugget=0.0, range\_value=0.0, sill=0.0, bounds=None)
     |      构造 克吕金插值参数对象。
     |      
     |      :param float resolution: 插值运算时使用的分辨率
     |      :param krighing\_type: 插值分析的算法类型。支持设置 KRIGING, SimpleKRIGING, UniversalKRIGING 三种,默认使用 KRIGING。
     |      :type krighing\_type: InterpolationAlgorithmType or str
     |      :param search\_mode: 查找模式。
     |      :type search\_mode: SearchMode or str
     |      :param float search\_radius:  查找参与运算点的查找半径。单位与用于插值的点数据集(或记录集所属的数据集)的单位相同。查找半径决定了参与
     |                                   运算点的查找范围,当计算某个位置的未知数值时,会以该位置为圆心,search\_radius 为半径,落在这个范围内的
     |                                   采样点都将参与运算,即该位置的预测值由该范围内采样点的数值决定。
     |      :param int expected\_count:  期望参与插值运算的点数,当查找方式为变长查找时,表示期望参与运算的最多样点数。
     |      :param int max\_point\_count\_in\_node: 单个块内最多查找点数。当用QuadTree的查找插值点时,才可以设置块内最多点数。
     |      :param int max\_point\_count\_for\_interpolation: 设置块查找时,最多参与插值的点数。注意,该值必须大于零。当用QuadTree的查找插值点时,才可以设置最多参与插值的点数
     |      :param variogram:  克吕金(Kriging)插值时的半变函数类型。默认值为 VariogramMode.SPHERICAL
     |      :type variogram: VariogramMode or str
     |      :param float angle:  克吕金算法中旋转角度值
     |      :param float mean: 插值字段的平均值,即采样点插值字段值总和除以采样点数目。
     |      :param exponent: 用于插值的样点数据中趋势面方程的阶数
     |      :type exponent: Exponent or str
     |      :param float nugget: 块金效应值。
     |      :param float range\_value: 自相关阈值。
     |      :param float sill: 基台值
     |      :param Rectangle bounds: 插值分析的范围,用于确定运行结果的范围
     |  
     |  set\_angle(self, value)
     |      设置克吕金算法中旋转角度值
     |      
     |      :param float value: 克吕金算法中旋转角度值
     |      :return: self
     |      :rtype: InterpolationKrigingParameter
     |  
     |  set\_expected\_count(self, value)
     |      设置期望参与插值运算的点数
     |      
     |      :param int value: 表示期望参与运算的最少样点数
     |      :return: self
     |      :rtype: InterpolationIDWParameter
     |  
     |  set\_exponent(self, value)
     |      设置用于插值的样点数据中趋势面方程的阶数
     |      
     |      :param value: 用于插值的样点数据中趋势面方程的阶数
     |      :type value: Exponent or str
     |      :return: self
     |      :rtype: InterpolationKrigingParameter
     |  
     |  set\_max\_point\_count\_for\_interpolation(self, value)
     |      设置块查找时,最多参与插值的点数。注意,该值必须大于零。当用QuadTree的查找插值点时,才可以设置最多参与插值的点数
     |      
     |      :param int value: 块查找时,最多参与插值的点数
     |      :return: self
     |      :rtype: InterpolationKrigingParameter
     |  
     |  set\_max\_point\_count\_in\_node(self, value)
     |      设置单个块内最多查找点数。当用QuadTree的查找插值点时,才可以设置块内最多点数。
     |      
     |      :param int value: 单个块内最多查找点数。当用QuadTree的查找插值点时,才可以设置块内最多点数
     |      :return: self
     |      :rtype: InterpolationKrigingParameter
     |  
     |  set\_mean(self, value)
     |      设置插值字段的平均值,即采样点插值字段值总和除以采样点数目。
     |      
     |      :param float value: 插值字段的平均值,即采样点插值字段值总和除以采样点数目。
     |      :return: self
     |      :rtype: InterpolationKrigingParameter
     |  
     |  set\_nugget(self, value)
     |      设置块金效应值。
     |      
     |      :param float value: 块金效应值。
     |      :return: self
     |      :rtype: InterpolationKrigingParameter
     |  
     |  set\_range(self, value)
     |      设置自相关阈值
     |      
     |      :param float value: 自相关阈值
     |      :return: self
     |      :rtype: InterpolationKrigingParameter
     |  
     |  set\_search\_mode(self, value)
     |      设置在插值运算时,查找参与运算点的方式
     |      
     |      :param value: 在插值运算时,查找参与运算点的方式
     |      :type value:  SearchMode or str
     |      :return: self
     |      :rtype: InterpolationIDWParameter
     |  
     |  set\_search\_radius(self, value)
     |      设置查找参与运算点的查找半径。单位与用于插值的点数据集(或记录集所属的数据集)的单位相同。查找半径决定了参与运算点的查找范围,当计算某个位置
     |      的未知数值时,会以该位置为圆心,以 search\_radius为半径,落在这个范围内的采样点都将参与运算,即该位置的预测值由该范围内采样点的数值决定。
     |      
     |      查找模式设置为“变长查找”(KDTREE\_FIXED\_COUNT),将使用最大查找半径范围内的固定数目的样点值进行插值,最大查找半径为点数据集的区域范围对
     |      应的矩形的对角线长度的 0.2 倍。
     |      
     |      :param float value: 查找参与运算点的查找半径
     |      :return: self
     |      :rtype: InterpolationIDWParameter
     |  
     |  set\_sill(self, value)
     |      设置基台值
     |      
     |      :param float value: 基台值
     |      :return: self
     |      :rtype: InterpolationKrigingParameter
     |  
     |  set\_variogram\_mode(self, value)
     |      设置克吕金(Kriging)插值时的半变函数类型。默认值为 VariogramMode.SPHERICAL
     |      
     |      :param value: 克吕金(Kriging)插值时的半变函数类型
     |      :type value: VariogramMode or
     |      :return: self
     |      :rtype: InterpolationKrigingParameter
     |  
     |  ----------------------------------------------------------------------
     |  Data descriptors defined here:
     |  
     |  angle
     |      float: 克吕金算法中旋转角度值
     |  
     |  expected\_count
     |      int: 期望参与插值运算的点数
     |  
     |  exponent
     |      Exponent: 用于插值的样点数据中趋势面方程的阶数
     |  
     |  max\_point\_count\_for\_interpolation
     |      int:块查找时,最多参与插值的点数
     |  
     |  max\_point\_count\_in\_node
     |      int:  单个块内最多查找点数
     |  
     |  mean
     |      float: 插值字段的平均值,即采样点插值字段值总和除以采样点数目。
     |  
     |  nugget
     |      float:  块金效应值。
     |  
     |  range
     |      float: 自相关阈值
     |  
     |  search\_mode
     |      SearchMode: 在插值运算时,查找参与运算点的方式
     |  
     |  search\_radius
     |      float: 查找参与运算点的查找半径
     |  
     |  sill
     |      float: 基台值
     |  
     |  variogram\_mode
     |      VariogramMode: 克吕金(Kriging)插值时的半变函数类型。默认值为 VariogramMode.SPHERICAL
     |  
     |  ----------------------------------------------------------------------
     |  Methods inherited from InterpolationParameter:
     |  
     |  set\_bounds(self, value)
     |      设置插值分析的范围,用于确定运行结果的范围
     |      
     |      :param Rectangle value: 插值分析的范围,用于确定运行结果的范围
     |      :return: self
     |      :rtype: InterpolationParameter
     |  
     |  set\_resolution(self, value)
     |      设置插值运算时使用的分辨率。
     |      
     |      :param float value: 插值运算时使用的分辨率
     |      :return: self
     |      :rtype: InterpolationParameter
     |  
     |  ----------------------------------------------------------------------
     |  Data descriptors inherited from InterpolationParameter:
     |  
     |  \_\_dict\_\_
     |      dictionary for instance variables (if defined)
     |  
     |  \_\_weakref\_\_
     |      list of weak references to the object (if defined)
     |  
     |  bounds
     |      Rectangle: 插值分析的范围,用于确定运行结果的范围
     |  
     |  resolution
     |      float: 插值运算时使用的分辨率
     |  
     |  type
     |      InterpolationAlgorithmType: 插值分支所支持的算法的类型
    
    class InterpolationRBFParameter(InterpolationParameter)
     |  径向基函数 RBF(Radial Basis Function)插值法参数类
     |  
     |  Method resolution order:
     |      InterpolationRBFParameter
     |      InterpolationParameter
     |      builtins.object
     |  
     |  Methods defined here:
     |  
     |  \_\_init\_\_(self, resolution, search\_mode=SearchMode.KDTREE\_FIXED\_COUNT, search\_radius=0.0, expected\_count=12, max\_point\_count\_in\_node=50, max\_point\_count\_for\_interpolation=200, smooth=0.100000001490116, tension=40, bounds=None)
     |      构造径向基函数插值法参数类对象。
     |      
     |      :param float resolution: 插值运算时使用的分辨率
     |      :param search\_mode: 查找模式。
     |      :type search\_mode: SearchMode or str
     |      :param float search\_radius:  查找参与运算点的查找半径。单位与用于插值的点数据集(或记录集所属的数据集)的单位相同。查找半径决定了参与
     |                                   运算点的查找范围,当计算某个位置的未知数值时,会以该位置为圆心,search\_radius 为半径,落在这个范围内的
     |                                   采样点都将参与运算,即该位置的预测值由该范围内采样点的数值决定。
     |      :param int expected\_count:  期望参与插值运算的点数,当查找方式为变长查找时,表示期望参与运算的最多样点数。
     |      :param int max\_point\_count\_in\_node: 单个块内最多查找点数。当用QuadTree的查找插值点时,才可以设置块内最多点数。
     |      :param int max\_point\_count\_for\_interpolation: 设置块查找时,最多参与插值的点数。注意,该值必须大于零。当用QuadTree的查找插值点时,才可以设置最多参与插值的点数
     |      :param float smooth: 光滑系数,值域为 [0,1]
     |      :param float tension: 张力系数
     |      :param Rectangle bounds: 插值分析的范围,用于确定运行结果的范围
     |  
     |  set\_expected\_count(self, value)
     |      设置期望参与插值运算的点数
     |      
     |      :param int value: 表示期望参与运算的最少样点数
     |      :return: self
     |      :rtype: InterpolationRBFParameter
     |  
     |  set\_max\_point\_count\_for\_interpolation(self, value)
     |      设置块查找时,最多参与插值的点数。注意,该值必须大于零。当用QuadTree的查找插值点时,才可以设置最多参与插值的点数
     |      
     |      :param int value: 块查找时,最多参与插值的点数
     |      :return: self
     |      :rtype: InterpolationRBFParameter
     |  
     |  set\_max\_point\_count\_in\_node(self, value)
     |      设置单个块内最多查找点数。当用QuadTree的查找插值点时,才可以设置块内最多点数。
     |      
     |      :param int value: 单个块内最多查找点数。当用QuadTree的查找插值点时,才可以设置块内最多点数
     |      :return: self
     |      :rtype: InterpolationRBFParameter
     |  
     |  set\_search\_mode(self, value)
     |      设置在插值运算时,查找参与运算点的方式。
     |      
     |      :param value: 在插值运算时,查找参与运算点的方式
     |      :type value:  SearchMode or str
     |      :return: self
     |      :rtype: InterpolationRBFParameter
     |  
     |  set\_search\_radius(self, value)
     |      设置查找参与运算点的查找半径。单位与用于插值的点数据集(或记录集所属的数据集)的单位相同。查找半径决定了参与运算点的查找范围,当计算某个位置
     |      的未知数值时,会以该位置为圆心,以 search\_radiu s为半径,落在这个范围内的采样点都将参与运算,即该位置的预测值由该范围内采样点的数值决定。
     |      
     |      查找模式设置为“变长查找”(KDTREE\_FIXED\_COUNT),将使用最大查找半径范围内的固定数目的样点值进行插值,最大查找半径为点数据集的区域范围对
     |      应的矩形的对角线长度的 0.2 倍。
     |      
     |      :param float value: 查找参与运算点的查找半径
     |      :return: self
     |      :rtype: InterpolationRBFParameter
     |  
     |  set\_smooth(self, value)
     |      设置光滑系数
     |      
     |      :param float value: 光滑系数
     |      :return: self
     |      :rtype: InterpolationRBFParameter
     |  
     |  set\_tension(self, value)
     |      设置张力系数
     |      
     |      :param float value: 张力系数
     |      :return: self
     |      :rtype: InterpolationRBFParameter
     |  
     |  ----------------------------------------------------------------------
     |  Data descriptors defined here:
     |  
     |  expected\_count
     |      int: 期望参与插值运算的点数
     |  
     |  max\_point\_count\_for\_interpolation
     |      int:块查找时,最多参与插值的点数
     |  
     |  max\_point\_count\_in\_node
     |      int:  单个块内最多查找点数
     |  
     |  search\_mode
     |      SearchMode: 在插值运算时,查找参与运算点的方式,不支持 KDTREE\_FIXED\_RADIUS
     |  
     |  search\_radius
     |      float: 查找参与运算点的查找半径
     |  
     |  smooth
     |      float: 光滑系数
     |  
     |  tension
     |      float: 张力系数
     |  
     |  ----------------------------------------------------------------------
     |  Methods inherited from InterpolationParameter:
     |  
     |  set\_bounds(self, value)
     |      设置插值分析的范围,用于确定运行结果的范围
     |      
     |      :param Rectangle value: 插值分析的范围,用于确定运行结果的范围
     |      :return: self
     |      :rtype: InterpolationParameter
     |  
     |  set\_resolution(self, value)
     |      设置插值运算时使用的分辨率。
     |      
     |      :param float value: 插值运算时使用的分辨率
     |      :return: self
     |      :rtype: InterpolationParameter
     |  
     |  ----------------------------------------------------------------------
     |  Data descriptors inherited from InterpolationParameter:
     |  
     |  \_\_dict\_\_
     |      dictionary for instance variables (if defined)
     |  
     |  \_\_weakref\_\_
     |      list of weak references to the object (if defined)
     |  
     |  bounds
     |      Rectangle: 插值分析的范围,用于确定运行结果的范围
     |  
     |  resolution
     |      float: 插值运算时使用的分辨率
     |  
     |  type
     |      InterpolationAlgorithmType: 插值分支所支持的算法的类型
    
    class NeighbourShape(builtins.object)
     |  邻域形状基类。邻域按照形状可分为:矩形邻域、圆形邻域、环形邻域和扇形邻域。邻域形状的相关参数设置
     |  
     |  Methods defined here:
     |  
     |  \_\_init\_\_(self)
     |      Initialize self.  See help(type(self)) for accurate signature.
     |  
     |  ----------------------------------------------------------------------
     |  Data descriptors defined here:
     |  
     |  \_\_dict\_\_
     |      dictionary for instance variables (if defined)
     |  
     |  \_\_weakref\_\_
     |      list of weak references to the object (if defined)
     |  
     |  shape\_type
     |      NeighbourShapeType: 域分析的邻域形状类型
    
    class NeighbourShapeAnnulus(NeighbourShape)
     |  环形邻域形状类
     |  
     |  Method resolution order:
     |      NeighbourShapeAnnulus
     |      NeighbourShape
     |      builtins.object
     |  
     |  Methods defined here:
     |  
     |  \_\_init\_\_(self, inner\_radius, outer\_radius)
     |      构造环形邻域形状类对象
     |      
     |      :param float inner\_radius: 内环半径
     |      :param float outer\_radius: 外环半径
     |  
     |  set\_inner\_radius(self, value)
     |      设置内环半径
     |      
     |      :param float value: 内环半径
     |      :return: self
     |      :rtype: NeighbourShapeAnnulus
     |  
     |  set\_outer\_radius(self, value)
     |      设置外环半径
     |      
     |      :param float value: 外环半径
     |      :return: self
     |      :rtype: NeighbourShapeAnnulus
     |  
     |  ----------------------------------------------------------------------
     |  Data descriptors defined here:
     |  
     |  inner\_radius
     |      float: 内环半径
     |  
     |  outer\_radius
     |      float: 外环半径
     |  
     |  ----------------------------------------------------------------------
     |  Data descriptors inherited from NeighbourShape:
     |  
     |  \_\_dict\_\_
     |      dictionary for instance variables (if defined)
     |  
     |  \_\_weakref\_\_
     |      list of weak references to the object (if defined)
     |  
     |  shape\_type
     |      NeighbourShapeType: 域分析的邻域形状类型
    
    class NeighbourShapeCircle(NeighbourShape)
     |  圆形邻域形状类
     |  
     |  Method resolution order:
     |      NeighbourShapeCircle
     |      NeighbourShape
     |      builtins.object
     |  
     |  Methods defined here:
     |  
     |  \_\_init\_\_(self, radius)
     |      构造圆形邻域形状类对象
     |      
     |      :param float radius: 圆形邻域的半径
     |  
     |  set\_radius(self, value)
     |      设置圆形邻域的半径
     |      
     |      :param float value: 圆形邻域的半径
     |      :return: self
     |      :rtype: NeighbourShapeCircle
     |  
     |  ----------------------------------------------------------------------
     |  Data descriptors defined here:
     |  
     |  radius
     |      float: 圆形邻域的半径
     |  
     |  ----------------------------------------------------------------------
     |  Data descriptors inherited from NeighbourShape:
     |  
     |  \_\_dict\_\_
     |      dictionary for instance variables (if defined)
     |  
     |  \_\_weakref\_\_
     |      list of weak references to the object (if defined)
     |  
     |  shape\_type
     |      NeighbourShapeType: 域分析的邻域形状类型
    
    class NeighbourShapeRectangle(NeighbourShape)
     |  矩形邻域形状类
     |  
     |  Method resolution order:
     |      NeighbourShapeRectangle
     |      NeighbourShape
     |      builtins.object
     |  
     |  Methods defined here:
     |  
     |  \_\_init\_\_(self, width, height)
     |      构造矩形邻域形状类对象
     |      
     |      :param float width: 矩形邻域的宽
     |      :param float height: 矩形邻域的高
     |  
     |  set\_height(self, value)
     |      设置矩形邻域的高
     |      
     |      :param float value:  矩形邻域的高
     |      :return: self
     |      :rtype:  NeighbourShapeRectangle
     |  
     |  set\_width(self, value)
     |      设置矩形邻域的宽
     |      
     |      :param float value: 矩形邻域的宽
     |      :return: self
     |      :rtype:  NeighbourShapeRectangle
     |  
     |  ----------------------------------------------------------------------
     |  Data descriptors defined here:
     |  
     |  height
     |      float: 矩形邻域的高
     |  
     |  width
     |      float: 矩形邻域的宽
     |  
     |  ----------------------------------------------------------------------
     |  Data descriptors inherited from NeighbourShape:
     |  
     |  \_\_dict\_\_
     |      dictionary for instance variables (if defined)
     |  
     |  \_\_weakref\_\_
     |      list of weak references to the object (if defined)
     |  
     |  shape\_type
     |      NeighbourShapeType: 域分析的邻域形状类型
    
    class NeighbourShapeWedge(NeighbourShape)
     |  扇形邻域形状类
     |  
     |  Method resolution order:
     |      NeighbourShapeWedge
     |      NeighbourShape
     |      builtins.object
     |  
     |  Methods defined here:
     |  
     |  \_\_init\_\_(self, radius, start\_angle, end\_angle)
     |      构造扇形邻域形状类对象
     |      
     |      :param float radius: 形邻域的半径
     |      :param float start\_angle: 扇形邻域的起始角度。单位为度。规定水平向右为 0 度,顺时针旋转计算角度。
     |      :param float end\_angle: 扇形邻域的终止角度。单位为度。规定水平向右为 0 度,顺时针旋转计算角度。
     |  
     |  set\_end\_angle(self, value)
     |      设置扇形邻域的终止角度。单位为度。规定水平向右为 0 度,顺时针旋转计算角度。
     |      
     |      :param float value:
     |      :return: self
     |      :rtype: NeighbourShapeWedge
     |  
     |  set\_radius(self, value)
     |      设置扇形邻域的半径
     |      
     |      :param float value: 扇形邻域的半径
     |      :return: self
     |      :rtype: NeighbourShapeWedge
     |  
     |  set\_start\_angle(self, value)
     |      设置扇形邻域的起始角度。单位为度。规定水平向右为 0 度,顺时针旋转计算角度。
     |      
     |      :param float value: 扇形邻域的起始角度。单位为度。规定水平向右为 0 度,顺时针旋转计算角度。
     |      :return: self
     |      :rtype: NeighbourShapeWedge
     |  
     |  ----------------------------------------------------------------------
     |  Data descriptors defined here:
     |  
     |  end\_angle
     |      float: 扇形邻域的终止角度。单位为度。规定水平向右为 0 度,顺时针旋转计算角度。
     |  
     |  radius
     |      float: 扇形邻域的半径
     |  
     |  start\_angle
     |      float: 扇形邻域的起始角度。单位为度。规定水平向右为 0 度,顺时针旋转计算角度。
     |  
     |  ----------------------------------------------------------------------
     |  Data descriptors inherited from NeighbourShape:
     |  
     |  \_\_dict\_\_
     |      dictionary for instance variables (if defined)
     |  
     |  \_\_weakref\_\_
     |      list of weak references to the object (if defined)
     |  
     |  shape\_type
     |      NeighbourShapeType: 域分析的邻域形状类型
    
    class PreprocessOptions(builtins.object)
     |  拓扑预处理参数类
     |  
     |  Methods defined here:
     |  
     |  \_\_init\_\_(self, arcs\_inserted=False, vertex\_arc\_inserted=False, vertexes\_snapped=False, polygons\_checked=False, vertex\_adjusted=False)
     |      构造拓扑预处理参数类对象
     |      
     |      :param bool arcs\_inserted: 是否进行线段间求交插入节点
     |      :param bool vertex\_arc\_inserted: 否进行节点与线段间插入节点
     |      :param bool vertexes\_snapped: 是否进行节点捕捉
     |      :param bool polygons\_checked: 是否进行多边形走向调整
     |      :param bool vertex\_adjusted: 是否进行节点位置调整
     |  
     |  set\_arcs\_inserted(self, value)
     |      设置是否进行线段间求交插入节点
     |      
     |      :param bool value: 是否进行线段间求交插入节点
     |      :return: self
     |      :rtype: PreprocessOptions
     |  
     |  set\_polygons\_checked(self, value)
     |      设置是否进行多边形走向调整
     |      
     |      :param bool value: 是否进行多边形走向调整
     |      :return: self
     |      :rtype: PreprocessOptions
     |  
     |  set\_vertex\_adjusted(self, value)
     |      设置是否进行节点位置调整
     |      
     |      :param bool value: 是否进行节点位置调整
     |      :return: self
     |      :rtype: PreprocessOptions
     |  
     |  set\_vertex\_arc\_inserted(self, value)
     |      设置否进行节点与线段间插入节点
     |      
     |      :param bool value: 否进行节点与线段间插入节点
     |      :return: self
     |      :rtype: PreprocessOptions
     |  
     |  set\_vertexes\_snapped(self, value)
     |      设置是否进行节点捕捉
     |      
     |      :param bool value: 是否进行节点捕捉
     |      :return: self
     |      :rtype: PreprocessOptions
     |  
     |  ----------------------------------------------------------------------
     |  Data descriptors defined here:
     |  
     |  \_\_dict\_\_
     |      dictionary for instance variables (if defined)
     |  
     |  \_\_weakref\_\_
     |      list of weak references to the object (if defined)
     |  
     |  arcs\_inserted
     |      bool: 是否进行线段间求交插入节点
     |  
     |  polygons\_checked
     |      bool: 是否进行多边形走向调整
     |  
     |  vertex\_adjusted
     |      bool: 是否进行节点位置调整
     |  
     |  vertex\_arc\_inserted
     |      bool: 否进行节点与线段间插入节点
     |  
     |  vertexes\_snapped
     |      bool: 是否进行节点捕捉
    
    class ProcessingOptions(builtins.object)
     |  拓扑处理参数类。该类提供了关于拓扑处理的设置信息。
     |  
     |  如果未通过 set\_vertex\_tolerance,set\_overshoots\_tolerance 和 set\_undershoots\_tolerance 方法设置节点容限、短悬线容限和长悬线容限,
     |  或设置为0,系统将使用数据集的容限中相应的容限值进行处理
     |  
     |  Methods defined here:
     |  
     |  \_\_init\_\_(self, pseudo\_nodes\_cleaned=False, overshoots\_cleaned=False, redundant\_vertices\_cleaned=False, undershoots\_extended=False, duplicated\_lines\_cleaned=False, lines\_intersected=False, adjacent\_endpoints\_merged=False, overshoots\_tolerance=1e-10, undershoots\_tolerance=1e-10, vertex\_tolerance=1e-10, filter\_vertex\_recordset=None, arc\_filter\_string=None, filter\_mode=None)
     |      构造拓扑处理参数类
     |      
     |      :param bool pseudo\_nodes\_cleaned: 是否去除假结点
     |      :param bool overshoots\_cleaned: 是否去除短悬线。
     |      :param bool redundant\_vertices\_cleaned: 是否去除冗余点
     |      :param bool undershoots\_extended: 是否进行长悬线延伸。
     |      :param bool duplicated\_lines\_cleaned: 是否去除重复线
     |      :param bool lines\_intersected: 是否进行弧段求交。
     |      :param bool adjacent\_endpoints\_merged: 是否进行邻近端点合并。
     |      :param float overshoots\_tolerance:  短悬线容限,该容限用于在去除短悬线时判断悬线是否是短悬线。
     |      :param float undershoots\_tolerance: 长悬线容限,该容限用于在长悬线延伸时判断悬线是否延伸。单位与进行拓扑处理的数据集单位相同。
     |      :param float vertex\_tolerance: 节点容限。该容限用于邻近端点合并、弧段求交、去除假结点和去除冗余点。单位与进行拓扑处理的数据集单位相同。
     |      :param Recordset filter\_vertex\_recordset: 弧段求交的过滤点记录集,即此记录集中的点位置线段不进行求交打断。
     |      :param str arc\_filter\_string: 弧段求交的过滤线表达式。 在进行弧段求交时,通过该属性可以指定一个字段表达式,符合该表达式的线对象将不被打断。
     |                                    该表达式是否有效,与 filter\_mode 弧段求交过滤模式有关
     |      :param filter\_mode: 弧段求交的过滤模式。
     |      :type filter\_mode: ArcAndVertexFilterMode or str
     |  
     |  set\_adjacent\_endpoints\_merged(self, value)
     |      设置是否进行邻近端点合并。
     |      
     |      如果多条弧段端点的距离小于节点容限,那么这些点就会被合并成为一个结点,该结点位置是原有点的几何平均(即 X、Y 分别为所有原有点 X、Y 的平均值)。
     |      
     |      用于判断邻近端点的节点容限,可通过 :py:meth:`set\_vertex\_tolerance` 设置,如果不设置或设置为0,将使用数据集的容限中的节点容限。
     |      
     |      需要注意的是,如果有两个邻近端点,那么合并的结果就会是一个假结点,还需要进行去除假结点的操作。
     |      
     |      :param bool value: 是否进行邻近端点合并
     |      :return: ProcessingOptions
     |      :rtype: self
     |  
     |  set\_arc\_filter\_string(self, value)
     |      设置弧段求交的过滤线表达式。
     |      
     |      在进行弧段求交时,通过该属性可以指定一个字段表达式,符合该表达式的线对象将不被打断。详细介绍请参见 :py:meth:`set\_lines\_intersected`  方法。
     |      
     |      :param str value:
     |      :return: ProcessingOptions
     |      :rtype: self
     |  
     |  set\_duplicated\_lines\_cleaned(self, value)
     |      设置是否去除重复线
     |      
     |      重复线:两条弧段若其所有节点两两重合,则可认为是重复线。重复线的判断不考虑方向。
     |      
     |      去除重复线的目的是为避免建立拓扑多边形时产生面积为零的多边形对象,因此,重复的线对象只应保留其中一个,多余的应删除。
     |      
     |      通常,出现重复线多是由于弧段求交造成的。
     |      
     |      :param bool value: 是否去除重复线
     |      :return: ProcessingOptions
     |      :rtype: self
     |  
     |  set\_filter\_mode(self, value)
     |      设置弧段求交的过滤模式
     |      
     |      :param value: 弧段求交的过滤模式
     |      :type value: ArcAndVertexFilterMode
     |      :return: ProcessingOptions
     |      :rtype: self
     |  
     |  set\_filter\_vertex\_recordset(self, value)
     |      设置弧段求交的过滤点记录集,即此记录集中的点位置线段不进行求交打断。
     |      
     |      如果过滤点在线对象上或到线对象的距离在容限范围内,在过滤点到线对象的垂足位置上线对象不被打断。详细介绍请参见 :py:meth:`set\_lines\_intersected` 方法。
     |      
     |      注意:过滤点记录集是否有效,与 :py:meth:`set\_filter\_mode` 方法设置的弧段求交过滤模式有关。可参见 :py:class:`.ArcAndVertexFilterMode` 类。
     |      
     |      :param Recordset value:
     |      :return: ProcessingOptions
     |      :rtype: self
     |  
     |  set\_lines\_intersected(self, value)
     |      设置是否进行弧段求交。
     |      
     |      线数据建立拓扑关系之前,首先要进行弧段求交计算,根据交点分解成若干线对象,一般而言,在二维坐标系统中凡是与其他线有交点的线对象都需要从交点处打断,如十字路口。且此方法是后续错误处理方法的基础。
     |      在实际应用中,相交线段完全打断的处理方式在很多时候并不能很好地满足研究需求。例如,一条高架铁路横跨一条公路,在二维坐标上来看是两个相交的线对象,但事实上并没有相交
     |      ,如果打断将可能影响进一步的分析。在交通领域还有很多类似的实际场景,如河流水系与交通线路的相交,城市中错综复杂的立交桥等,对于某些相交点是否打断,
     |      需要根据实际应用来灵活处理,而不能因为在二维平面上相交就一律打断。
     |      
     |      这种情况可以通过设置过滤线表达式( :py:meth:`set\_arc\_filter\_string` )和过滤点记录集 ( :py:meth:`set\_vertex\_filter\_recordset` ) 来
     |      确定哪些线对象以及哪些相交位置处不打断:
     |      
     |        - 过滤线表达式用于查询出不需要打断的线对象
     |        - 过滤点记录集中的点对象所在位置处不打断
     |      
     |      这两个参数单独或组合使用构成了弧段求交的四种过滤模式,还有一种是不进行过滤。过滤模式通过 :py:meth:`set\_filter\_mode` 方法设置。对于上面的例子,使用不同的过滤模式,弧段求交的结果也不相同。关于过滤模式的详细介绍请参阅 :py:class:`.ArcAndVertexFilterMode` 类。
     |      
     |      注意:进行弧段求交处理时,可通过 :py:meth:`set\_vertex\_tolerance` 方法设置节点容限(如不设置,将使用数据集的节点容限),用于判断过滤点是否有效。若过滤点到线对象的距离在设置的容限范围内,则线对象在过滤点到其的垂足位置上不被打断。
     |      
     |      :param bool value: 是否进行弧段求交
     |      :return: ProcessingOptions
     |      :rtype: self
     |  
     |  set\_overshoots\_cleaned(self, value)
     |      设置是否去除短悬线。去除短悬线指如果一条悬线的长度小于悬线容限,则在进行去除短悬线操作时就会把这条悬线删除。通过 set\_overshoots\_tolerance 方法可以指定短悬线容限,如不指定则使用数据集的短悬线容限。
     |      
     |      悬线:如果一个线对象的端点没有与其它任意一个线对象的端点相连,则这个端点称之为悬点。具有悬点的线对象称之为悬线。
     |      
     |      :param bool value: 是否去除短悬线,True 表示去除,False 表示不去除。
     |      :return: ProcessingOptions
     |      :rtype: self
     |  
     |  set\_overshoots\_tolerance(self, value)
     |      设置短悬线容限,该容限用于在去除短悬线时判断悬线是否是短悬线。单位与进行拓扑处理的数据集单位相同。
     |      
     |      “悬线”的定义:如果一个线对象的端点没有与其它任意一个线对象的端点相连,则这个端点称之为悬点。具有悬点的线对象称之为悬线。
     |      
     |      :param float value:  短悬线容限
     |      :return: ProcessingOptions
     |      :rtype: self
     |  
     |  set\_pseudo\_nodes\_cleaned(self, value)
     |      设置是否去除假结点。结点又称为弧段连接点,至少连接三条弧段的才可称为一个结点。如果弧段连接点只连接了一条弧段(岛屿的情况)或连接了两条弧段(即它是两条弧段的公共端点),则该结点被称为假结点
     |      
     |      :param bool value: 是否去除假结点,True 表示去除,False 表示不去除。
     |      :return: ProcessingOptions
     |      :rtype: self
     |  
     |  set\_redundant\_vertices\_cleaned(self, value)
     |      设置是否去除冗余点。任意弧段上两节点之间的距离小于节点容限时,其中一个即被认为是一个冗余点,在进行拓扑处理时可以去除。
     |      
     |      :param bool value: : 是否去除冗余点,True 表示去除,False 表示不去除。
     |      :return: ProcessingOptions
     |      :rtype: self
     |  
     |  set\_undershoots\_extended(self, value)
     |      设置是否进行长悬线延伸。 如果一条悬线按其行进方向延伸了指定的长度(悬线容限)之后与某弧段有交点,则拓扑处理后会将该悬线自动延伸到某弧段上,
     |      称为长悬线延伸。通过 set\_undershoots\_tolerance 方法可以指定长悬线容限,如不指定则使用数据集的长悬线容限。
     |      
     |      :param bool value:  是否进行长悬线延伸
     |      :return: ProcessingOptions
     |      :rtype: self
     |  
     |  set\_undershoots\_tolerance(self, value)
     |      设置长悬线容限,该容限用于在长悬线延伸时判断悬线是否延伸。单位与进行拓扑处理的数据集单位相同。
     |      
     |      :param float value: 长悬线容限
     |      :return: ProcessingOptions
     |      :rtype: self
     |  
     |  set\_vertex\_tolerance(self, value)
     |      设置节点容限。该容限用于邻近端点合并、弧段求交、去除假结点和去除冗余点。单位与进行拓扑处理的数据集单位相同。
     |      
     |      :param float value: 节点容限
     |      :return: ProcessingOptions
     |      :rtype: self
     |  
     |  ----------------------------------------------------------------------
     |  Data descriptors defined here:
     |  
     |  \_\_dict\_\_
     |      dictionary for instance variables (if defined)
     |  
     |  \_\_weakref\_\_
     |      list of weak references to the object (if defined)
     |  
     |  adjacent\_endpoints\_merged
     |      bool: 是否进行邻近端点合并
     |  
     |  arc\_filter\_string
     |      str:  弧段求交的过滤线表达式。 在进行弧段求交时,通过该属性可以指定一个字段表达式,符合该表达式的线对象将不被打断。该表达式是否有效,与 filter\_mode 弧段求交过滤模式有关
     |  
     |  duplicated\_lines\_cleaned
     |      bool: 是否去除重复线
     |  
     |  filter\_mode
     |      ArcAndVertexFilterMode: 弧段求交的过滤模式
     |  
     |  filter\_vertex\_recordset
     |      Recordset: 弧段求交的过滤点记录集,即此记录集中的点位置线段不进行求交打断
     |  
     |  lines\_intersected
     |      bool: 是否进行弧段求交
     |  
     |  overshoots\_cleaned
     |      bool:  是否去除短悬线
     |  
     |  overshoots\_tolerance
     |      float: 短悬线容限,该容限用于在去除短悬线时判断悬线是否是短悬线
     |  
     |  pseudo\_nodes\_cleaned
     |      bool: 是否去除假结点
     |  
     |  redundant\_vertices\_cleaned
     |      bool: 是否去除冗余点
     |  
     |  undershoots\_extended
     |      bool: 是否进行长悬线延伸
     |  
     |  undershoots\_tolerance
     |      float: 长悬线容限,该容限用于在长悬线延伸时判断悬线是否延伸。单位与进行拓扑处理的数据集单位相同
     |  
     |  vertex\_tolerance
     |      float: 节点容限。该容限用于邻近端点合并、弧段求交、去除假结点和去除冗余点。单位与进行拓扑处理的数据集单位相同
    
    class ReclassMappingTable(builtins.object)
     |  栅格重分级映射表类。提供对源栅格数据集进行单值或范围的重分级,且包含对无值数据和未分级单元格的处理。
     |  
     |  重分级映射表,用于说明源数据和结果数据值之间的对应关系。这种对应关系由这几部分内容表达:重分级类型、重分级区间集合、无值和未分级数据的处理。
     |  
     |  - 重分级的类型
     |    重分级有两种类型,单值重分级和范围重分级。单值重分级是对指定的某些单值进行重新赋值,如将源栅格中值为100的单元格,赋值为1输出到结果
     |    栅格中;范围重分级将一个区间内的值重新赋值为一个值,如将源栅格中栅格值在[100,500)范围内的单元格,重新赋值为200输出到结果栅格中。通过该类的 :py:meth:`set\_reclass\_type` 方法来设置重分级类型。
     |  
     |  - 重分级区间集合
     |    重分级的区间集合规定了源栅格某个栅格值或者一定区间内的栅格值与重分级后的新值的对应关系,通过该类的  :py:meth:`set\_segments` 方法设置。
     |    该集合由若干重分级区间(ReclassSegment)对象构成。该对象用于设置每个重分级区间的信息,包括要重新赋值的源栅格单值或区间的起始值、终止值,重分级区间的类型,
     |    以及栅格重分级的区间值或源栅格单值对应的新值等,详见 :py:class:`.ReclassSegment` 类。
     |  
     |  - 无值和未分级数据的处理
     |    对源栅格数据中的无值,可以通过该类的 :py:meth:`set\_retain\_no\_value` 方法来设置是否保持无值,如果为 False,即不保持为无值,则可通过 :py:meth:`set\_change\_no\_value\_to` 方法为无值数据指定一个值。
     |  
     |    对在重分级映射表中未涉及的栅格值,可以通过该类的 :py:meth:`set\_retain\_missing\_value` 方法来设置是否保持其原值,如果为 False,即不保持原值,则可通过 :py:meth:`set\_change\_missing\_valueT\_to` 方法为其指定一个值。
     |  
     |  此外,该类还提供了将重分级映射表数据导出为 XML 字符串及 XML 文件的方法和导入 XML 字符串或文件的方法。当多个输入的栅格数据需要应用相同的分级范围时,可以将其导出为重分级映射表文件,
     |  当对后续数据进行分级时,直接导入该重分级映射表文件,进而可以批量处理导入的栅格数据。有关栅格重分级映射表文件的格式和标签含义请参见 to\_xml 方法。
     |  
     |  Methods defined here:
     |  
     |  \_\_init\_\_(self)
     |      Initialize self.  See help(type(self)) for accurate signature.
     |  
     |  from\_dict(self, values)
     |      从 dict 对象中读取重分级映射表信息
     |      
     |      :param dict values: 包含重分级映射表信息的 dict 对象
     |      :return: self
     |      :rtype: ReclassMappingTable
     |  
     |  set\_change\_missing\_value\_to(self, value)
     |      设置不在指定区间或单值内的栅格的指定值。如果 :py:meth:`is\_retain\_no\_value` 为 True 时,则该设置无效。
     |      
     |      :param float value: 不在指定区间或单值内的栅格的指定值
     |      :return: self
     |      :rtype: ReclassMappingTable
     |  
     |  set\_change\_no\_value\_to(self, value)
     |      设置无值数据的指定值。:py:meth:`is\_retain\_no\_value` 为 True 时,该设置无效。
     |      
     |      :param float value: 无值数据的指定值
     |      :return: self
     |      :rtype: ReclassMappingTable
     |  
     |  set\_reclass\_type(self, value)
     |      设置栅格重分级类型
     |      
     |      :param value: 栅格重分级类型,默认值为 UNIQUE
     |      :type value: ReclassType or str
     |      :return: self
     |      :rtype: ReclassMappingTable
     |  
     |  set\_retain\_missing\_value(self, value)
     |      设置源数据集中不在指定区间或单值之外的数据是否保留原值。
     |      
     |      :param bool value:  源数据集中不在指定区间或单值之外的数据是否保留原值。
     |      :return: self
     |      :rtype: ReclassMappingTable
     |  
     |  set\_retain\_no\_value(self, value)
     |      设置是否将源数据集中的无值数据保持为无值。设置是否将源数据集中的无值数据保持为无值。
     |      - 当 set\_retain\_no\_value 方法设置为 True 时,表示保持源数据集中的无值数据为无值;
     |      - 当 set\_retain\_no\_value 方法设置为 False 时,表示将源数据集中的无值数据设置为指定的值( :py:meth:`set\_change\_no\_value\_to` )
     |      
     |      :param bool value:
     |      :return: self
     |      :rtype: ReclassMappingTable
     |  
     |  set\_segments(self, value)
     |      设置重分级区间集合
     |      
     |      :param value: 重分级区间集合。当 value 为 str 时,支持使用 ';' 分隔多个ReclassSegment,每个 ReclassSegment使用 ','分隔 起始值、终止值、新值和分区类型。例如:
     |                      '0,100,50,CLOSEOPEN; 100,200,150,CLOSEOPEN'
     |      :type value: list[ReclassSegment] or str
     |      :return: self
     |      :rtype: ReclassMappingTable
     |  
     |  to\_dict(self)
     |      将当前信息输出到 dict 中
     |      
     |      :return: 包含当前信息的字典对象
     |      :rtype: dict
     |  
     |  to\_xml(self)
     |      将当前对象信息输出为 xml 字符串
     |      
     |      :return: xml 字符串
     |      :rtype: str
     |  
     |  to\_xml\_file(self, xml\_file)
     |      该方法用于将对重分级映射表对象的参数设置写入一个 XML 文件,称为栅格重分级映射表文件,其后缀名为 .xml,下面是一个栅格重分级映射表文件的例子:
     |      
     |      重分级映射表文件中各标签的含义如下:
     |      
     |      - <SmXml:ReclassType></SmXml:ReclassType> 标签:重分级类型。1表示单值重分级,2表示范围重分级。
     |      - <SmXml:SegmentCount></SmXml:SegmentCount> 标签:重分级区间集合,count 参数表示重分级的级数。
     |      - <SmXml:Range></SmXml:Range> 标签:重分级区间,重分级类型为单值重分级,格式为:区间起始值--区间终止值:新值-区间类型。对于区间类型,0表示左开右闭,1表示左闭右开。
     |      - <SmXml:Unique></SmXml:Unique> 标签:重分级区间,重分级类型为范围重分级,格式为:原值:新值。
     |      - <SmXml:RetainMissingValue></SmXml:RetainMissingValue> 标签:未分级单元格是否保留原值。0表示不保留,1表示保留。
     |      - <SmXml:RetainNoValue></SmXml:RetainNoValue> 标签:无值数据是否保持无值。0表示不保持,0表示不保持。
     |      - <SmXml:ChangeMissingValueTo></SmXml:ChangeMissingValueTo> 标签:为未分级单元格的指定的值。
     |      - <SmXml:ChangeNoValueTo></SmXml:ChangeNoValueTo> 标签:为无值数据的指定的值。
     |      
     |      
     |      :param str xml\_file: xml 文件路径
     |      :type xml\_file:
     |      :return: 导出成功返回 True,否则返回 False
     |      :rtype: bool
     |  
     |  ----------------------------------------------------------------------
     |  Static methods defined here:
     |  
     |  from\_xml(xml)
     |      从存储在XML格式字符串中的参数值导入到映射表数据中,并返回一个新的对象。
     |      
     |      :param str xml: XML格式字符串
     |      :return:  栅格重分级映射表对象
     |      :rtype: ReclassMappingTable
     |  
     |  from\_xml\_file(xml\_file)
     |      从已保存的XML格式的映射表文件中导入映射表数据,并返回一个新的对象。
     |      
     |      :param str xml\_file: XML文件
     |      :return:  栅格重分级映射表对象
     |      :rtype: ReclassMappingTable
     |  
     |  make\_from\_dict(values)
     |      从 dict 对象中读取重分级映射表信息构造新的对象
     |      
     |      :param dict values: 包含重分级映射表信息的 dict 对象
     |      :return: 重分级映射表对象
     |      :rtype: ReclassMappingTable
     |  
     |  ----------------------------------------------------------------------
     |  Data descriptors defined here:
     |  
     |  \_\_dict\_\_
     |      dictionary for instance variables (if defined)
     |  
     |  \_\_weakref\_\_
     |      list of weak references to the object (if defined)
     |  
     |  change\_missing\_value\_to
     |      float: 返回不在指定区间或单值内的栅格的指定值。
     |  
     |  change\_no\_value\_to
     |      float: 返回无值数据的指定值
     |  
     |  is\_retain\_missing\_value
     |      bool: 源数据集中不在指定区间或单值之外的数据是否保留原值
     |  
     |  is\_retain\_no\_value
     |      bool: 返回是否将源数据集中的无值数据保持为无值。
     |  
     |  reclass\_type
     |      ReclassType: 返回栅格重分级类型
     |  
     |  segments
     |      list[ReclassSegment]: 返回重分级区间集合。 每一个 ReclassSegment 就是一个区间范围或者是一个旧值和一个新值的对应关系。
    
    class ReclassSegment(builtins.object)
     |  栅格重分级区间类。该类主要用于重分级区间信息的相关设置,包括区间的起始值、终止值等。
     |  
     |  该类用于设置在进行重分级时,重分级映射表中每个重分级区间的参数,重分级类型不同,需要设置的属性也有所不同。
     |  
     |  - 当重分级类型为单值重分级时,需要通过 :py:meth:`set\_start\_value` 方法指定需要被重新赋值的源栅格的单值,并通过 :py:meth:`set\_new\_value` 方法设置该值对应的新值。
     |  - 当重分级类型为范围重分级时,需要通过 :py:meth:`set\_start\_value` 方法指定需要重新赋值的源栅格值区间的起始值,通过 :py:meth:`set\_end\_value` 方法设置区间的终止值,
     |    并通过 :py:meth:`set\_new\_value` 方法设置该区间对应的新值,还可以由 :py:meth:`set\_segment\_type` 方法设置区间类型是“左开右闭”还是“左闭右开”。
     |  
     |  Methods defined here:
     |  
     |  \_\_init\_\_(self, start\_value=None, end\_value=None, new\_value=None, segment\_type=None)
     |      构造栅格重分级区间对象
     |      
     |      :param float start\_value:  栅格重分级区间的起始值
     |      :param float end\_value: 栅格重分级区间的终止值
     |      :param float new\_value: 栅格重分级的区间值或旧值对应的新值
     |      :param segment\_type:  栅格重分级区间类型
     |      :type segment\_type: ReclassSegmentType or str
     |  
     |  from\_dict(self, values)
     |      从dict中读取信息
     |      
     |      :param values: 包含 ReclassSegment 信息的 dict
     |      :type values: dict
     |      :return: self
     |      :rtype: ReclassSegment
     |  
     |  set\_end\_value(self, value)
     |      栅格重分级区间的终止值
     |      
     |      :param float value: 栅格重分级区间的终止值
     |      :return: self
     |      :rtype: ReclassSegment
     |  
     |  set\_new\_value(self, value)
     |      栅格重分级的区间值或旧值对应的新值
     |      
     |      :param float value: 栅格重分级的区间值或旧值对应的新值
     |      :return: self
     |      :rtype: ReclassSegment
     |  
     |  set\_segment\_type(self, value)
     |      设置栅格重分级区间类型
     |      
     |      :param value: 栅格重分级区间类型
     |      :type value: ReclassSegmentType or str
     |      :return: self
     |      :rtype: ReclassSegment
     |  
     |  set\_start\_value(self, value)
     |      设置栅格重分级区间的起始值
     |      
     |      :param float value: 栅格重分级区间的起始值
     |      :return: self
     |      :rtype: ReclassSegment
     |  
     |  to\_dict(self)
     |      将当前对象信息输出到 dict
     |      
     |      :return: 包含当前对象信息的 dict 对象
     |      :rtype: dict
     |  
     |  ----------------------------------------------------------------------
     |  Static methods defined here:
     |  
     |  make\_from\_dict(values)
     |      从dict中读取信息构造 ReclassSegment 对象
     |      
     |      :param values: 包含 ReclassSegment 信息的 dict
     |      :type values: dict
     |      :return: 栅格重分级区间对象
     |      :rtype: ReclassSegment
     |  
     |  ----------------------------------------------------------------------
     |  Data descriptors defined here:
     |  
     |  \_\_dict\_\_
     |      dictionary for instance variables (if defined)
     |  
     |  \_\_weakref\_\_
     |      list of weak references to the object (if defined)
     |  
     |  end\_value
     |      float: 栅格重分级区间的终止值
     |  
     |  new\_value
     |      float: 栅格重分级的区间值或旧值对应的新值
     |  
     |  segment\_type
     |      ReclassSegmentType: 栅格重分级区间类型
     |  
     |  start\_value
     |      float: 栅格重分级区间的起始值
    
    class StatisticsField(builtins.object)
     |  对字段进行统计的信息。主要用于 :py:meth:`summary\_points`
     |  
     |  Methods defined here:
     |  
     |  \_\_init\_\_(self, source\_field=None, stat\_type=None, result\_field=None)
     |      初始化对象
     |      
     |      :param str source\_field: 被统计的字段名称
     |      :param stat\_type: 统计类型
     |      :type stat\_type: StatisticsFieldType or str
     |      :param str result\_field: 结果字段名称
     |  
     |  set\_result\_field(self, value)
     |      设置结果字段名称
     |      
     |      :param str value: 结果字段名称
     |      :return: self
     |      :rtype: StatisticsField
     |  
     |  set\_source\_field(self, value)
     |      设置被统计的字段名称
     |      
     |      :param str value: 字段名称
     |      :return: self
     |      :rtype: StatisticsField
     |  
     |  set\_stat\_type(self, value)
     |      设置字段统计类型
     |      
     |      :param value: 字段统计类型
     |      :type value: StatisticsFieldType or str
     |      :return: self
     |      :rtype: StatisticsField
     |  
     |  ----------------------------------------------------------------------
     |  Data descriptors defined here:
     |  
     |  \_\_dict\_\_
     |      dictionary for instance variables (if defined)
     |  
     |  \_\_weakref\_\_
     |      list of weak references to the object (if defined)
     |  
     |  result\_field
     |      str: 结果字段名称
     |  
     |  source\_field
     |      str: 被统计的字段名称
     |  
     |  stat\_type
     |      StatisticsFieldType: 字段统计类型

FUNCTIONS
    GWR(source, explanatory\_fields, model\_field, kernel\_function='GAUSSIAN', band\_width\_type='AICC', distance\_tolerance=0.0, kernel\_type='FIXED', neighbors=2, out\_data=None, out\_dataset\_name=None, progress=None)
        空间关系建模介绍:
        
         * 用户可以通过空间关系建模来解决以下问题:
        
           * 为什么某一现象会持续的发生,是什么因素导致了这种情况?
           * 导致某一事故发生率比预期的要高的因素有那些?有没有什么方法来减少整个城市或特定区域内的事故发生率?
           * 对某种现象建模以预测其他地点或者其他时间的数值?
        
         * 通过回归分析,你可以对空间关系进行建模、检查和研究,可以帮助你解释所观测到的空间模型后的诸多因素。比如线性关系是正或者
           是负;对于正向关系,即存在正相关性,某一变量随着另一个变量增加而增加;反之,某一变量随着另一个变量增加而减小;或者两个变量无关系。
        
        
        地理加权回归分析。
        
        * 地理加权回归分析结果信息包含一个结果数据集和地理加权回归结果汇总(请参阅 GWRSummary 类)。
        * 结果数据集包含交叉验证(CVScore)、预测值(Predicted)、回归系数(Intercept、C1\_解释字段名)、残差(Residual)、标准误
          (StdError)、系数标准误(SE\_Intercept、SE1\_解释字段名)、伪t值(TV\_Intercept、TV1\_解释字段名)和Studentised残差(StdResidual)等。
        
        说明:
        
          * 地理加权回归分析是一种用于空间变化关系的线性回归的局部形式,可用来在空间变化依赖和独立变量之间的关系研究。对地理要素所
            关联的数据变量之间的关系进行建模,从而可以对未知值进行预测或者更好地理解可对要建模的变量产生影响的关键因素。回归方法使
            你可以对空间关系进行验证并衡量空间关系的稳固性。
          * 交叉验证(CVScore):交叉验证在回归系数估计时不包括回归点本身即只根据回归点周围的数据点进行回归计算。该值就是每个回归
            点在交叉验证中得到的估计值与实际值之差,它们的平方和为CV值。作为一个模型性能指标。
          * 预测值(Predicted):这些值是地理加权回归得到的估计值(或拟合值)。
          * 回归系数(Intercept):它是地理加权回归模型的回归系数,为回归模型的回归截距,表示所有解释变量均为零时因变量的预测值。
          * 回归系数(C1\_解释字段名):它是解释字段的回归系数,表示解释变量与因变量之间的关系强度和类型。如果回归系数为正,则解释
            变量与因变量之间的关系为正向的;反之,则存在负向关系。如果关系很强,则回归系数也相对较大;关系较弱时,则回归系数接近于0。
          * 残差(Residual):这些是因变量无法解释的部分,是估计值和实际值之差,标准化残差的平均值为0,标准差为1。残差可用于确定模
            型的拟合程度,残差较小表明模型拟合效果较好,可以解释大部分预测值,说明这个回归方程是有效的。
          * 标准误(StdError):估计值的标准误差,用于衡量每个估计值的可靠性。较小的标准误表明拟合值与实际值的差异程度越小,模型拟合效果越好。
          * 系数标准误(SE\_Intercept、SE1\_解释字段名):这些值用于衡量每个回归系数估计值的可靠性。系数的标准误差与实际系数相比较小
            时,估计值的可信度会更高。较大的标准误差可能表示存在局部多重共线性问题。
          * 伪t值(TV\_Intercept、TV1\_解释字段名):是对各个回归系数的显著性检验。当T值大于临界值时,拒绝零假设,回归系数显著即回归系
            估计值是可靠的;当T值小于临界值时,则接受零假设,回归系数不显著。
          * Studentised残差(StdResidual):残差和标准误的比值,该值可用来判断数据是否异常,若数据都在(-2,2)区间内,表明数据具
            有正态性和方差齐性;若数据超出(-2,2)区间,表明该数据为异常数据,无方差齐性和正态性。
        
        
        :param source: 待计算的数据集。可以为点、线、面数据集。
        :type source: DatasetVector or str
        :param explanatory\_fields: 解释字段的名称的集合
        :type explanatory\_fields: list[str] or str
        :param str model\_field: 建模字段的名称
        :param kernel\_function: 核函数类型
        :type kernel\_function: KernelFunction or str
        :param band\_width\_type: 带宽确定方式
        :type band\_width\_type: BandWidthType or str
        :param float distance\_tolerance: 带宽范围
        :param kernel\_type: 带宽类型
        :type kernel\_type: KernelType or str
        :param int neighbors: 相邻数目。只有当带宽类型设置为 :py:attr:`.KernelType.ADAPTIVE` 且宽确定方式设置为 :py:attr:`.BandWidthType.BANDWIDTH` 时有效。
        :param out\_data: 用于存储结果数据集的数据源
        :type out\_data: Datasource or DatasourceConnectionInfo or str
        :param str out\_dataset\_name: 结果数据集名称
        :param progress: 进度信息,具体参考 :py:class:`.StepEvent`
        :type progress: function
        :return: 返回一个两个元素的 tuple,tuple 的第一个元素为 :py:class:`.GWRSummary` ,第二个元素为地理加权回归结果数据集。
        :rtype: tuple[GWRSummary, DatasetVector]
    
    aggregate\_grid(input\_data, scale, aggregation\_type, is\_expanded, is\_ignore\_no\_value, out\_data=None, out\_dataset\_name=None, progress=None)
        栅格数据聚合,返回结果栅格数据集。
        栅格聚合操作是以整数倍缩小栅格分辨率,生成一个新的分辨率较粗的栅格的过程。此时,每个像元由原栅格数据的一组像元聚合而成,其值由其包含的原栅格的值共
        同决定,可以取包含栅格的和、最大值、最小值、平均值、中位数。如缩小n(n为大于1的整数)倍,则聚合后栅格的行、列的数目均为原栅格的1/n,也就是单元格
        大小是原来的n倍。聚合可以通过对数据进行概化,达到清除不需要的信息或者删除微小错误的目的。
        
        注意:如果原栅格数据的行列数不是 scale 的整数倍,使用 is\_expanded 参数来处理零头。
        
        - is\_expanded 为 true,则在零头加上一个数,使之成为一个整数倍,扩大的范围其栅格值均为无值,因此,结果数据集的范围会比原始的大一些。
        
        - is\_expanded 为 false,去掉零头,结果数据集的范围会比原始的小一些。
        
        :param input\_data: 指定的进行聚合操作的栅格数据集。
        :type input\_data: DatasetGrid or str
        :param int scale: 指定的结果栅格与输入栅格之间栅格大小的比例。取值为大于 1 的整型数值。
        :param aggregation\_type: 聚合操作类型
        :type aggregation\_type: AggregationType
        :param bool is\_expanded: 指定是否处理零头。当原栅格数据的行列数不是 scale 的整数倍时,栅格边界处则会出现零头。
        :param bool is\_ignore\_no\_value:  在聚合范围内含有无值数据时聚合操作的计算方式。如果为 True,使用聚合范围内除无值外的其余栅格值来计算;如果为 False,则聚合结果为无值。
        :param out\_data: 结果数据集所在的数据源
        :type out\_data: Datasource or DatasourceConnectionInfo or str
        :param str out\_dataset\_name: 结果数据集名称
        :param function progress: 进度信息处理函数,具体参考 :py:class:`.StepEvent`
        :return: 结果数据集或数据集名称
        :rtype: DatasetGrid or str
    
    aggregate\_points(input\_data, min\_pile\_point, distance, unit=None, class\_field=None, out\_data=None, out\_dataset\_name='AggregateResult', progress=None)
        对点数据集进行聚类,使用密度聚类算法,返回聚类后的类别或同一簇构成的多边形。
        对点集合进行空间位置的聚类,使用密度聚类方法 DBSCAN,它能将具有足够高密度的区域划分为簇,并可以在带有噪声的空间数据中发现任意形状的聚类。它定义
        簇为密度相连的点的最大集合。DBSCAN 使用阈值 e 和 MinPts 来控制簇的生成。其中,给定对象半径 e 内的区域称为该对象的 e一邻域。如果一个对象的
        e一邻域至少包含最小数目 MinPtS 个对象,则称该对象为核心对象。给定一个对象集合 D,如果 P 是在 Q 的 e一邻域内,而 Q 是一个核心对象,我们说对象
        P 从对象 Q 出发是直接密度可达的。DBSCAN 通过检查数据里中每个点的 e-领域来寻找聚类,如果一个点 P 的 e-领域包含多于 MinPts 个点,则创建一个
        以 P 作为核心对象的新簇,然后,DBSCAN反复地寻找从这些核心对象直接密度可达的对象并加入该簇,直到没有新的点可以被添加。
        
        :param input\_data: 输入的点数据集
        :type input\_data: DatasetVector or str
        :param int min\_pile\_point:  密度聚类点数目阈值,必须大于等于2。阈值越大表示能聚类为一簇的条件越苛刻。
        :param float distance: 密度聚类半径。
        :param unit:  密度聚类半径的单位。
        :type unit: Unit or str
        :param str class\_field: 输入的点数据集中用于保存密度聚类的结果聚类类别的字段,如果不为空,则必须是点数据集中合法的字段名称。
                                要求字段类型为INT16, INT32 或 INT64,如果字段名有效但不存在,将会创建一个 INT32 的字段。
                                参数有效,则会将聚类类别保存在此字段中。
        :param out\_data: 结果数据源信息,结果数据源信息不能与 class\_field同时为空,如果结果数据源有效时,将会生成结果面对象。
        :type out\_data: Datasource or DatasourceConnectionInfo or st
        :param str out\_dataset\_name: 结果数据集名称
        :param function progress: 进度信息处理函数,具体参考 :py:class:`.StepEvent`
        :return: 结果数据集或数据集名称,如果输入的结果数据源为空,将会返回一个布尔值,True 表示聚类成功,False 表示聚类失败。
        :rtype: DatasetVector or str or bool
        
        
        >>> result = aggregate\_points('E:/data.udb/point', 4, 100, 'Meter', 'SmUserID', out\_data='E:/aggregate\_out.udb')
    
    altitude\_statistics(point\_data, grid\_data, out\_data=None, out\_dataset\_name=None)
        高程统计,统计二维点数据集中每个点对应的栅格值,并生成一个三维点数据集,三维点对象的 Z 值即为被统计的栅格像素的高程值。
        
        :param point\_data: 二维点数据集
        :type point\_data: DatasetVector or str
        :param grid\_data: 被统计的栅格数据集
        :type grid\_data: DatasetGrid or str
        :param out\_data: 用于存储结果数据的数据源。
        :type out\_data: Datasource or DatasourceConnectionInfo or str
        :param out\_dataset\_name: 结果数据集的名称
        :type out\_dataset\_name: str
        :return: 统计三维数据集或数据集名称
        :rtype: DatasetGrid or str
    
    area\_solar\_radiation\_days(grid\_data, latitude, start\_day, end\_day=160, hour\_start=0, hour\_end=24, day\_interval=5, hour\_interval=0.5, transmittance=0.5, z\_factor=1.0, out\_data=None, out\_total\_grid\_name='TotalGrid', out\_direct\_grid\_name=None, out\_diffuse\_grid\_name=None, out\_duration\_grid\_name=None, progress=None)
        计算多天的区域太阳辐射总量,即整个DEM范围内每个栅格的太阳辐射情况。需要指定每天的开始时点、结束时点和开始日期、结束日期。
        
        :param grid\_data: 待计算太阳辐射的DEM栅格数据
        :type grid\_data: DatasetGrid or str
        :param latitude: 待计算区域的平均纬度
        :type latitude: float
        :param start\_day: 起始日期,可以是 "\%Y-\%m-\%d" 格式的字符串,如果为 int,则表示一年中的第几天
        :type start\_day: datetime.date or str or int
        :param end\_day: 终止日期,可以是 "\%Y-\%m-\%d" 格式的字符串,如果为 int,则表示一年中的第几天
        :type end\_day: datetime.date or str or int
        :param hour\_start: 起始时点,如果输入float 时,可以输入一个 [0,24]范围内的数值,表示一天中的第几个小时。也可以输入一个 datetime.datatime 或 "\%H:\%M:\%S" 格式的字符串
        :type hour\_start: float or str or datetime.datetime
        :param hour\_end: 终止时点,如果输入float 时,可以输入一个 [0,24]范围内的数值,表示一天中的第几个小时。也可以输入一个 datetime.datatime 或 "\%H:\%M:\%S" 格式的字符串
        :type hour\_end:  float or str or datetime.datetime
        :param int day\_interval: 天数间隔,单位为天
        :param float hour\_interval: 小时间隔,单位为小时。
        :param float transmittance: 太阳辐射穿过大气的透射率,值域为[0,1]。
        :param float z\_factor: 高程缩放系数
        :param out\_data: 用于存储结果数据的数据源。
        :type out\_data: Datasource or DatasourceConnectionInfo or str
        :param str out\_total\_grid\_name: 总辐射量结果数据集名称,数据集名称必须合法
        :param str out\_direct\_grid\_name: 直射辐射量结果数据集名称,数据集名称必须合法,且接口内不会自动获取有效的数据集名称
        :param str out\_diffuse\_grid\_name: 散射辐射量结果数据集名称,数据集名称必须合法,且接口内不会自动获取有效的数据集名称
        :param str out\_duration\_grid\_name: 太阳直射持续时间结果数据集名称,数据集名称必须合法,且接口内不会自动获取有效的数据集名称
        :param progress: 进度信息处理函数,具体参考 :py:class:`.StepEvent`
        :type progress: function
        :return: 返回一个四个元素的 tuple:
        
                   * 第一个为总辐射量结果数据集,
                   * 如果设置了直射辐射量结果数据集名称,第二个为直射辐射量结果数据集,否则为 None,
                   * 如果设置散射辐射量结果数据集的名称,第三个为散射辐射量结果数据集,否则为 None
                   * 如果设置太阳直射持续时间结果数据集的名称,第四个为太阳直射持续时间结果数据集,否则为 None
        
        :rtype: tuple[DatasetGrid] or tuple[str]
    
    area\_solar\_radiation\_hours(grid\_data, latitude, day, hour\_start=0, hour\_end=24, hour\_interval=0.5, transmittance=0.5, z\_factor=1.0, out\_data=None, out\_total\_grid\_name='TotalGrid', out\_direct\_grid\_name=None, out\_diffuse\_grid\_name=None, out\_duration\_grid\_name=None, progress=None)
        计算一天内的太阳辐射,需要指定开始时点、结束时点及开始日期作为要计算的日期
        
        :param grid\_data: 待计算太阳辐射的DEM栅格数据
        :type grid\_data: DatasetGrid or str
        :param latitude: 待计算区域的平均纬度
        :type latitude: float
        :param day: 待计算的指定日期。可以是 "\%Y-\%m-\%d" 格式的字符串,如果为 int,则表示一年中的第几天。
        :type day:  datetime.date or str or int
        :param hour\_start: 起始时点,如果输入float 时,可以输入一个 [0,24]范围内的数值,表示一天中的第几个小时。也可以输入一个 datetime.datatime 或 "\%H:\%M:\%S" 格式的字符串
        :type hour\_start: float or str or datetime.datetime
        :param hour\_end: 终止时点,如果输入float 时,可以输入一个 [0,24]范围内的数值,表示一天中的第几个小时。也可以输入一个 datetime.datatime 或 "\%H:\%M:\%S" 格式的字符串
        :type hour\_end:  float or str or datetime.datetime
        :param float hour\_interval: 小时间隔,单位为小时。
        :param float transmittance: 太阳辐射穿过大气的透射率,值域为[0,1]。
        :param float z\_factor: 高程缩放系数
        :param out\_data: 用于存储结果数据的数据源。
        :type out\_data: Datasource or DatasourceConnectionInfo or str
        :param str out\_total\_grid\_name: 总辐射量结果数据集名称,数据集名称必须合法
        :param str out\_direct\_grid\_name: 直射辐射量结果数据集名称,数据集名称必须合法,且接口内不会自动获取有效的数据集名称
        :param str out\_diffuse\_grid\_name: 散射辐射量结果数据集名称,数据集名称必须合法,且接口内不会自动获取有效的数据集名称
        :param str out\_duration\_grid\_name: 太阳直射持续时间结果数据集名称,数据集名称必须合法,且接口内不会自动获取有效的数据集名称
        :param progress: 进度信息处理函数,具体参考 :py:class:`.StepEvent`
        :type progress: function
        :return: 返回一个四个元素的 tuple:
        
                   * 第一个为总辐射量结果数据集,
                   * 如果设置了直射辐射量结果数据集名称,第二个为直射辐射量结果数据集,否则为 None,
                   * 如果设置散射辐射量结果数据集的名称,第三个为散射辐射量结果数据集,否则为 None
                   * 如果设置太阳直射持续时间结果数据集的名称,第四个为太阳直射持续时间结果数据集,否则为 None
        
        :rtype: tuple[DatasetGrid] or tuple[str]
    
    auto\_correlation(source, assessment\_field, concept\_model='INVERSEDISTANCE', distance\_method='EUCLIDEAN', distance\_tolerance=-1.0, exponent=1.0, k\_neighbors=1, is\_standardization=False, weight\_file\_path=None, progress=None)
        分析模式介绍:
        
            分析模式可评估一组数据是形成离散空间模式、聚类空间模式或者随机空间模式。
        
            * 分析模式用来计算的数据可以是点、线、面。对于点、线和面对象,在距离计算中会使用对象的质心。对象的质心为所有子对象的加权平均中心。点对象的加权项为1(即质心为自身),线对象的加权项是长度,而面对象的加权项是面积。
            * 分析模式类采用推论式统计,会在进行统计检验时预先建立"零假设",假设要素或要素之间相关的值都表现为随机空间模式。
            * 分析结果计算中会给出一个P值用来表示"零假设"的正确概率,用以判定是接受"零假设"还是拒绝"零假设"。
            * 分析结果计算中会给出一个Z得分用来表示标准差的倍数,用以判定数据是呈聚类、离散或随机。
            * 要拒绝"零假设",就必须要承担可能做出错误选择(即错误的拒绝"零假设")的风险。
        
              下表显示了不同置信度下未经校正的临界P值和临界Z得分:
        
              .. image:: ../image/AnalyzingPatterns.png
        
            * 用户可以通过分析模式来解决以下问题:
        
                * 数据集中的要素或数据集中要素关联的值是否发生空间聚类?
                * 数据集的聚类程度是否会随时间变化?
        
            分析模式包括空间自相关分析( :py:func:`auto\_correlation` )、平均最近邻分析( :py:func:`average\_nearest\_neighbor` )、
            高低值聚类分析( :py:func:`high\_or\_low\_clustering` )、增量空间自相关分析( :py:func:`incremental\_auto\_correlation` )等。
        
        对矢量数据集进行空间自相关分析,并返回空间自相关分析结果。空间自相关返回的结果包括莫兰指数、期望、方差、z得分、P值,
        请参阅 :py:class:`.AnalyzingPatternsResult` 类。
        
        .. image:: ../image/AnalyzingPatterns\_autoCorrelation.png
        
        
        :param source: 待计算的数据集。可以为点、线、面数据集。
        :type source: DatasetVector or str
        :param str assessment\_field: 评估字段的名称。仅数值字段有效。
        :param concept\_model: 空间关系概念化模型。默认值 :py:attr:`.ConceptualizationModel.INVERSEDISTANCE`。
        :type concept\_model: ConceptualizationModel or str
        :param distance\_method: 距离计算方法类型
        :type distance\_method: DistanceMethod or str
        :param float distance\_tolerance: 中断距离容限。仅对概念化模型设置为 :py:attr:`.ConceptualizationModel.INVERSEDISTANCE` 、
                                         :py:attr:`.ConceptualizationModel.INVERSEDISTANCESQUARED` 、
                                         :py:attr:`.ConceptualizationModel.FIXEDDISTANCEBAND` 、
                                         :py:attr:`.ConceptualizationModel.ZONEOFINDIFFERENCE` 时有效。
        
                                         为"反距离"和"固定距离"模型指定中断距离。"-1"表示计算并应用默认距离,此默认值为保证每个要
                                         素至少有一个相邻的要素;"0"表示为未应用任何距离,则每个要素都是相邻要素。
        
        :param float exponent: 反距离幂指数。仅对概念化模型设置为 :py:attr:`.ConceptualizationModel.INVERSEDISTANCE` 、
                                         :py:attr:`.ConceptualizationModel.INVERSEDISTANCESQUARED` 、
                                         :py:attr:`.ConceptualizationModel.ZONEOFINDIFFERENCE` 时有效。
        :param int k\_neighbors:  相邻数目,目标要素周围最近的K个要素为相邻要素。仅对概念化模型设置为 :py:attr:`.ConceptualizationModel.KNEARESTNEIGHBORS` 时有效。
        :param bool is\_standardization: 是否对空间权重矩阵进行标准化。若进行标准化,则每个权重都会除以该行的和。
        :param str weight\_file\_path: 空间权重矩阵文件路径
        :param progress: 进度信息处理函数,具体参考 :py:class:`.StepEvent`
        :type progress: function
        :return: 空间自相关结果
        :rtype: AnalyzingPatternsResult
    
    average\_nearest\_neighbor(source, study\_area, distance\_method='EUCLIDEAN', progress=None)
        对矢量数据集进行平均最近邻分析,并返回平均最近邻分析结果数组。
        
        * 平均最近邻返回的结果包括最近邻指数、预期平均距离、平均观测距离、z得分、P值,请参阅 :py:class:`.AnalyzingPatternsResult` 类。
        
        * 给定的研究区域面积大小必须大于等于0;如果研究区域面积等于0,则会自动生成输入数据集的最小面积外接矩形,用该矩形的面积来进行计算。
          该默认值为: 0 。
        
        * 距离计算方法类型可以指定相邻要素之间的距离计算方式(参阅 :py:class:`.DistanceMethod` )。如果输入数据集为地理坐标系,则会采用弦测量方法来
          计算距离。地球表面上的任意两点,两点之间的弦距离为穿过地球体连接两点的直线长度。
        
        
        .. image:: ../image/AnalyzingPatterns\_AverageNearestNeighbor.png
        
        
        关于分析模式介绍,请参考 :py:func:`auto\_correlation`
        
        :param source: 待计算的数据集。可以为点、线、面数据集。
        :type source: DatasetVector or str
        :param float study\_area: 研究区域面积
        :param distance\_method: 距离计算方法
        :type distance\_method: DistanceMethod or str
        :param progress: 进度信息处理函数,具体参考 :py:class:`.StepEvent`
        :type progress: function
        :return: 平均最近邻分析结果
        :rtype: AnalyzingPatternsResult
    
    basin(direction\_grid, out\_data=None, out\_dataset\_name=None, progress=None)
        关于水文分析:
        
        * 水文分析基于数字高程模型(DEM)栅格数据建立水系模型,用于研究流域水文特征和模拟地表水文过程,并对未来的地表水文情况做出预估。水文分析模型能够帮助我们分析洪水的范围,定位径流污染源,预测地貌改变对径流的影响等,广泛应用于区域规划、农林、灾害预测、道路设计等诸多行业和领域。
        
        * 地表水的汇流情况很大程度上决定于地表形状,而 DEM 数据能够表达区域地貌形态的空间分布,在描述流域地形,如流域边界、坡度和坡向、河网提取等方面具有突出优势,因而非常适用于水文分析。
        
        * SuperMap 提供的水文分析主要内容有填充洼地、计算流向、计算流长、计算累积汇水量、流域划分、河流分级、连接水系及水系矢量化等。
        
            * 水文分析的一般流程为:
        
              .. image:: ../image/HydrologyAnalyst\_2.png
        
            * 如何获得栅格水系?
        
              水文分析中很多功能都需要基于栅格水系数据,如提取矢量水系(:py:func:`stream\_to\_line` 方法)、河流分级(:py:func:`stream\_order` 方法)、
              连接水系(::py:func:`stream\_link` 方法)等。
        
              通常,可以从累积汇水量栅格中提取栅格水系数据。在累积汇水量栅格中,单元格的值越大,代表该区域的累积汇水量越大。累积汇水量
              较高的单元格可视为河谷,因此,可以通过设定一个阈值,提取累积汇水量大于该值的单元格,这些单元格就构成栅格水系。值得说明的
              是,对于不同级别的河谷、不同区域的相同级别的河谷,该值可能不同,因此该阈值的确定需要依据研究区域的实际地形地貌并通过不断的试验来确定。
        
              在 SuperMap 中,要求用于进一步分析(提取矢量水系、河流分级、连接水系等)的栅格水系为一个二值栅格,这可以通过栅格代数运算
              来实现,使大于或等于累积汇水量阈值的单元格为 1,否则为 0,如下图所示。
        
              .. image:: ../image/HydrologyAnalyst\_3.png
        
              因此,提取栅格水系的过程如下:
        
               1. 获得累积汇水量栅格,可通过 :py:func:`flow\_accumulation` 方法实现。
               2. 通过栅格代数运算 :py:func:`expression\_math\_analyst` 方法对累积汇水量栅格进行关系运算,就可以得到满足要求的栅格水系数据。假设设定
                  阈值为 1000,则运算表达式为:"[Datasource.FlowAccumulationDataset]>1000"。除此,使用 Con(x,y,z) 函数也可以得到想
                  要的结果,即表达式为:"Con([Datasource.FlowAccumulationDataset]>1000,1,0)"。
        
        
        根据流向栅格计算流域盆地。流域盆地即为集水区域,是用于描述流域的方式之一。
        
        计算流域盆地是依据流向数据为每个单元格分配唯一盆地的过程,如下图所示,流域盆地是描述流域的方式之一,展现了那些所有相互连接且处于同一流域盆地的栅格。
        
        .. image:: ../image/Basin.png
        
        
        :param direction\_grid: 流向栅格数据集。
        :type direction\_grid: DatasetGrid or str
        :param out\_data: 存储结果数据集的数据源
        :type out\_data: Datasource or DatasourceConnectionInfo or str
        :param str out\_dataset\_name: 结果数据集名称
        :param progress: 进度信息处理函数,具体参考 :py:class:`.StepEvent`
        :type progress: function
        :return: 流域盆地栅格数据集或数据集名称
        :rtype: DatasetGrid or str
    
    build\_lake(dem\_grid, lake\_data, elevation, progress=None)
        挖湖,即修改面数据集区域范围内的 DEM 数据集的高程值为指定的数值。
        挖湖是指根据已有的湖泊面数据,在 DEM 数据集上显示湖泊信息。如下图所示,挖湖之后,DEM 在湖泊面数据对应位置的栅格值变成指定的高程值,且整个湖泊区域栅格值相同。
        
        .. image:: ../image/BuildLake.png
        
        :param dem\_grid:  指定的待挖湖的 DEM 栅格数据集。
        :type dem\_grid: DatasetGrid or str
        :param lake\_data:  指定的湖区域,为面数据集。
        :type lake\_data: DatasetVector or str
        :param elevation: 指定的湖区域的高程字段或指定的高程值。如果为 str,则要求字段类型为数值型。如果指定为 None 或空字符串,或湖区域数据集中不存在指定的
                          字段,则按照湖区域边界对应 DEM 栅格上的最小高程进行挖湖。高程值的单位与 DEM 栅格数据集的栅格值单位相同。
        :type elevation: str or float
        :param progress: 进度信息处理函数,具体参考 :py:class:`.StepEvent`
        :type progress: function
        :return: 成功返回 True,否则返回 False
        :rtype: bool
    
    build\_quad\_mesh(quad\_mesh\_region, left\_bottom, left\_top, right\_bottom, right\_top, cols=0, rows=0, out\_col\_field=None, out\_row\_field=None, out\_data=None, out\_dataset\_name=None, progress=None)
        对单个简单面对象进行网格剖分。
        流体问题是一个连续性的问题,为了简化对其的研究以及建模处理的方便,对研究区域进行离散化处理,其思路就是建立离散的网格,网格划分就是对连续的物理区域进行剖分,把它分成若干个网格,并确定各个网格中的节点,用网格内的一个值来代替整个网格区域的基本情况,网格作为计算与分析的载体,其质量的好坏对后期的数值模拟的精度和计算效率有重要的影响。
        
        网格剖分的步骤:
        
         1.数据预处理,包含去除重复点等。给定一个合理的容限,去除重复点,使得最后的网格划分结果更趋合理,不会出现看起来从1个点
           (实际是重复点)出发有多条线的现象。
        
         2.多边形分解:对于复杂的多边形区域,我们采用分块逐步划分的方法来进行网格的构建,将一个复杂的不规则多边形区域划分为多个简
            单的单连通区域,然后对每个单连通区域执行网格划分程序,最后再将各个子区域网格拼接起来构成对整个区域的划分。
        
         3.选择四个角点:这4个角点对应着网格划分的计算区域上的4个顶点,其选择会对划分的结果造成影响。其选择应尽量在原区域近似四边
            形的四个顶点上,同时要考虑整体的流势。
        
            .. image:: ../image/SelectPoint.png
        
         4.为了使划分的网格呈现四边形的特征,构成多边形的顶点数据(不在同一直线上)需参与构网。
        
         5.进行简单区域网格划分。
        
        注:简单多边形:多边形内任何直线或边都不会交叉。
        
            .. image:: ../image/QuadMeshPart.png
        
        说明:
        
         RightTopIndex 为右上角点索引号,LeftTopIndex 为左上角点索引号,RightBottomIndex 为右下角点索引号,LeftBottomIndex
         为左下角点索引号。则 nCount1=(RightTopIndex- LeftTopIndex+1)和 nCount2=(RightBottomIndex- LeftBottomIndex+1),
         如果:nCount1不等于nCount2,则程序不处理。
        
        水文分析的相关介绍,请参考 :py:func:`basin`
        
        
        :param quad\_mesh\_region: 网格剖分的面对象
        :type quad\_mesh\_region: GeoRegion
        :param Point2D left\_bottom: 网格剖分的区域多边形左下角点坐标。四个角点选择依据:4个角点对应着网格剖分的计算区域上的4个顶点,
                                    其选择会对剖分的结果造成影响。其选择应尽量在原区域近似四边形的四个顶点上,同时要考虑整体的流势。
        :param Point2D left\_top: 网格剖分的区域多边形左上角点坐标
        :param Point2D right\_bottom: 网格剖分的区域多边形右下角点坐标
        :param Point2D right\_top: 网格剖分的区域多边形右上角点坐标
        :param int cols: 网格剖分的列方向节点数。默认值为0,表示不参与处理;若不为0,但是此值若小于多边形列方向的最大点数减一,则
                         以多边形列方向的最大点数减一作为列数(cols);若大于多边形列方向的最大点数减一,则会自动加点,使列方
                         向的数目为 cols。
                         举例来讲:如果用户希望将一矩形面对象划分为2*3(高*宽)=6个小矩形,则列方向数目(cols)为3。
        :param int rows: 网格剖分的行方向节点数。默认值为0,表示不参与处理;若不为0,但是此值小于多边形行方向的最大点数减一,则以
                         多边形行方向的最大点数减一作为行数(rows);若大于多边形行方向的最大点数减一,则会自动加点,使行方向的
                         数目为 rows。举例来讲:如果用户希望将一矩形面对象划分为2*3(高*宽)=6个小矩形,则行方向数目(rows)为2。
        :param str out\_col\_field: 格网剖分结果对象的列属性字段名称。此字段用来保存剖分结果对象的列号。
        :param str out\_row\_field: 格网剖分结果对象的行属性字段名称。此字段用来保存剖分结果对象的行号。
        :param out\_data: 存放剖分结果数据集的数据源。
        :type out\_data: DatasourceConnectionInfo or Datasource or str
        :param str out\_dataset\_name: 剖分结果数据集的名称。
        :param progress: 进度信息处理函数,具体参考 :py:class:`.StepEvent`
        :type progress: function
        :return: 剖分后的结果数据集,剖分出的多个面以子对象形式返回。
        :rtype: DatasetVector or str
    
    build\_terrain(source\_datas, lake\_dataset=None, lake\_altitude\_field=None, clip\_data=None, erase\_data=None, interpolate\_type='IDW', resample\_len=0.0, z\_factor=1.0, is\_process\_flat\_area=False, encode\_type='NONE', pixel\_format='SINGLE', cell\_size=0.0, out\_data=None, out\_dataset\_name=None, progress=None)
        根据指定的地形构建参数信息创建地形。
        DEM(Digital Elevation Model,数字高程模型)主要用于描述区域地貌形态的空间分布,是地面特性为高程和海拔高程的数字地面模型(DTM),
        通常通过高程测量点(或从等高线中进行采样提取高程点)进行数据内插而成。此方法用于构建地形,即对具有高程信息的点或线数据集通过插值生成 DEM 栅格。
        
        .. image:: ../image/BuildTerrain\_1.png
        
        可以通过 source\_datas 参数指定用于构建地形的数据集,支持仅高程点、仅等高线以及支持高程点和等高线共同构建。
        
        
        :param source\_datas: 用于构建的点数据集和线数据集,以及数据集的高程字段。要求数据集的坐标系相同。
        :type source\_datas: dict[DatasetVector,str] or dict[str,str]
        :param lake\_dataset:  湖泊面数据集。在结果数据集中,湖泊面数据集区域范围内的高程值小于周边相邻的高程值。
        :type lake\_dataset: DatasetVector or str
        :param str lake\_altitude\_field: 湖泊面数据集的高程字段
        :param clip\_data: 设置用于裁剪的数据集。构建地形时,仅位于裁剪区域内的 DEM 结果被保留,区域外的部分被赋予无值。
        
                          .. image:: ../image/BuildTerrainParameter\_1.png
        
        :type clip\_data: DatasetVector or str
        :param erase\_data: 用于擦除的数据集。构建地形时,位于擦除区域内的结果 DEM 栅格值为无值。仅在 interpolate\_type 设置为 TIN 时有效。
        
                           .. image:: ../image/BuildTerrainParameter\_2.png
        
        :type erase\_data: DatasetVector or str
        :param interpolate\_type: 地形插值类型。默认值为 IDW。
        :type interpolate\_type: TerrainInterpolateType  or str
        :param float resample\_len: 采样距离。只对线数据集有效。单位与用于构建地形的线数据集单位一致。仅在 interpolate\_type 设置为TIN时有效。
                             首先对线数据集进行重采样过滤掉一些比较密集的节点,然后再生成 TIN 模型,提高生成速度。
        :param float z\_factor: 高程缩放系数
        :param bool is\_process\_flat\_area: 是否处理平坦区域。等值线生成DEM能较好地处理山顶山谷,点生成DEM也可以处理平坦区域,但效
                                          果没有等值线生成DEM处理的好,主要原因是根据点判断平坦区域结果较为粗糙。
        :param encode\_type: 编码方式。对于栅格数据集,目前支持的编码方式有未编码、SGL、LZW 三种方式
        :type encode\_type: EncodeType or str
        :param pixel\_format: 结果数据集的像素格式
        :type pixel\_format: PixelFormat or str
        :param float cell\_size: 结果数据集的栅格单元的大小,如果指定为 0 或负数,则系统会使用 L/500(L 是指源数据集的区域范围对应的矩形的对角线长度)作为单元格大小。
        :param out\_data: 用于存储结果数据的数据源。
        :type out\_data: Datasource or DatasourceConnectionInfo or str
        :param out\_dataset\_name: 结果数据集的名称
        :type out\_dataset\_name: str
        :param progress: 进度信息处理函数,具体参考 :py:class:`.StepEvent`
        :type progress: function
        :return: 结果数据集或数据集名称
        :rtype: Dataset or str
    
    build\_weight\_matrix(source, unique\_id\_field, file\_path, concept\_model='INVERSEDISTANCE', distance\_method='EUCLIDEAN', distance\_tolerance=-1.0, exponent=1.0, k\_neighbors=1, is\_standardization=False, progress=None)
        构建空间权重矩阵。
        
         * 空间权重矩阵文件旨在生成、存储、重用和共享一组要素之间关系的空间关系概念化模型。文件采用的是二进制文件格式创建,要素关系
           存储为稀疏矩阵。
        
         * 该方法会生成一个空间权重矩阵文件,文件格式为 ‘*.swmb’。生成的空间权重矩阵文件可用来进行分析,只要将空间关系概念化模型设
           置为 :py:attr:`.ConceptualizationModel.SPATIALWEIGHTMATRIXFILE` 并且通过 weight\_file\_path 参数指定创建的空间权重矩阵
           文件的完整路径。
        
        :param source: 待构建空间权重矩阵的数据集,支持点线面。
        :type source: DatasetVector or str
        :param str unique\_id\_field: 唯一ID字段名,必须是数值型字段。
        :param str file\_path: 空间权重矩阵文件保存路径。
        :param concept\_model: 概念化模型
        :type concept\_model: ConceptualizationModel or str
        :param distance\_method: 距离计算方法类型
        :type distance\_method: DistanceMethod or str
        :param float distance\_tolerance: 中断距离容限。仅对概念化模型设置为 :py:attr:`.ConceptualizationModel.INVERSEDISTANCE` 、
                                         :py:attr:`.ConceptualizationModel.INVERSEDISTANCESQUARED` 、
                                         :py:attr:`.ConceptualizationModel.FIXEDDISTANCEBAND` 、
                                         :py:attr:`.ConceptualizationModel.ZONEOFINDIFFERENCE` 时有效。
        
                                         为"反距离"和"固定距离"模型指定中断距离。"-1"表示计算并应用默认距离,此默认值为保证每个要
                                         素至少有一个相邻的要素;"0"表示为未应用任何距离,则每个要素都是相邻要素。
        
        :param float exponent: 反距离幂指数。仅对概念化模型设置为 :py:attr:`.ConceptualizationModel.INVERSEDISTANCE` 、
                               :py:attr:`.ConceptualizationModel.INVERSEDISTANCESQUARED` 、
                               :py:attr:`.ConceptualizationModel.ZONEOFINDIFFERENCE` 时有效。
        
        :param int k\_neighbors: 相邻数目。仅对概念化模型设置为 :py:attr:`.ConceptualizationModel.KNEARESTNEIGHBORS` 时有效。
        :param bool is\_standardization: 是否对空间权重矩阵进行标准化。若进行标准化,则每个权重都会除以该行的和。
        :param progress: 进度信息,具体参考 :py:class:`.StepEvent`
        :type progress: function
        :return: 如果构建空间权重矩阵,返回 True,否则返回 False
        :rtype: bool
    
    calculate\_aspect(input\_data, out\_data=None, out\_dataset\_name=None, progress=None)
        计算坡向,并返回坡向栅格数据集,即坡向图。
        坡向是指坡面的朝向,它表示地形表面某处最陡的下坡方向。坡向反映了斜坡所面对的方向,任意斜坡的倾斜方向可取 0~360 度中的任意方向,所以坡向计算的
        结果范围为 0~360 度。从正北方向(0 度)开始顺时针计算
        
        :param input\_data: 指定的待计算坡向的栅格数据集
        :type input\_data: DatasetGrid or str
        :param out\_data: 结果数据集所在的数据源
        :type out\_data: Datasource or DatasourceConnectionInfo or str
        :param str out\_dataset\_name: 结果数据集名称
        :param function progress: 进度信息处理函数,具体参考 :py:class:`.StepEvent`
        :return: 结果数据集或数据集名称
        :rtype: DatasetGrid or str
    
    calculate\_hill\_shade(input\_data, shadow\_mode, azimuth, altitude\_angle, z\_factor, out\_data=None, out\_dataset\_name=None, progress=None)
        三维晕渲图是指通过模拟实际地表的本影与落影的方式反映地形起伏状况的栅格图。通过采用假想的光源照射地表,结合栅格数据得到的坡度坡向信息, 得到各像元
        的灰度值,面向光源的斜坡的灰度值较高,背向光源的灰度值较低,即为阴影区,从而形象表现出实际地表的地貌和地势。 由栅格数据计算得出的这种山体阴影图
        往往具有非常逼真的立体效果,因而称其为三维晕渲图。
        
        .. image:: ../image/CalculateHillShade.png
        
        三维晕渲图在描述地表三维状况和地形分析中都具有比较重要的价值,当将其他专题信息叠加在三维晕渲图之上时,将会更加提高三维晕渲图的应用价值和直观效果。
        
        在生成三维晕渲图时,需要指定假想光源的位置,该位置由光源的方位角和高度角确定。方位角确定光源的方向,高度角是光源照射时倾斜角度。例如,当光源的方位角
        为 315 度,高度角为 45 度时,其与地表的相对位置如下图所示。
        
        .. image:: ../image/CalculateHillShade\_1.png
        
        三维晕渲图有三种类型:渲染阴影效果、渲染效果和阴影效果,通过 :py:class`ShadowMode` 类来指定。
        
        :param input\_data: 指定的待生成三维晕渲图的栅格数据集
        :type input\_data: DatasetGrid or str
        :param shadow\_mode: 三维晕渲图的渲染类型
        :type shadow\_mode: ShadowMode or str
        :param float azimuth: 指定的光源方位角。用于确定光源的方向,是从光源所在位置的正北方向线起,依顺时针方向到光源与目标方向线
                              的夹角,范围为 0-360 度,以正北方向为 0 度,依顺时针方向递增。
        
                              .. image:: ../image/Azimuth.png
        
        :param float altitude\_angle: 指定的光源高度角。用于确定光源照射的倾斜角度,是光源与目标的方向线与水平面间的夹角,范围为
                                     0-90 度。当光源高度角为 90 度时,光源正射地表。
        
                                     .. image:: ../image/AltitudeAngle.png
        
        :param float z\_factor:  指定的高程缩放系数。该值是指在栅格中,栅格值(Z 坐标,即高程值)相对于 X 和 Y 坐标的单位变换系数。通常有 X,Y,Z 都参加的计算中,需要将高程值乘以一个高程缩放系数,使得三者单位一致。例如,X、Y 方向上的单位是米,而 Z 方向的单位是英尺,由于 1 英尺等于 0.3048 米,则需要指定缩放系数为 0.3048。如果设置为 1.0,表示不缩放。
        :param out\_data: 结果数据集所在的数据源
        :type out\_data: Datasource or DatasourceConnectionInfo or str
        :param str out\_dataset\_name: 结果数据集名称
        :param function progress: 进度信息处理函数,具体参考 :py:class:`.StepEvent`
        :return: 结果数据集或数据集名称
        :rtype: DatasetGrid or str
    
    calculate\_ortho\_image(input\_data, colors, no\_value\_color, out\_data=None, out\_dataset\_name=None, progress=None)
        根据给定的颜色集合生成正射三维影像。
        
        正射影像是采用数字微分纠正技术,通过周边邻近栅格的高程得到当前点的合理日照强度,进行正射影像纠正。
        
        :param input\_data: 指定的待计算三维正射影像的 DEM 栅格。
        :type input\_data: DatasetGrid or str
        :param colors: 三维投影后的颜色集合。输入如果为 dict,则表示高程值与颜色值的对应关系。
                    可以不必在高程颜色对照表中列出待计算栅格的所有栅格值(高程值)及其对应颜色,未在高程颜色对照表中列出的高程值,其在结果影像中的颜色将通过插值得出。
        :type colors: Colors or dict[float,tuple]
        :param no\_value\_color: 无值栅格的颜色
        :type no\_value\_color: tuple or int
        :param out\_data: 结果数据集所在的数据源
        :type out\_data: Datasource or DatasourceConnectionInfo or str
        :param str out\_dataset\_name: 结果数据集名称
        :param function progress: 进度信息处理函数,具体参考 :py:class:`.StepEvent`
        :return: 结果数据集或数据集名称
        :rtype: DatasetGrid or str
    
    calculate\_slope(input\_data, slope\_type, z\_factor, out\_data=None, out\_dataset\_name=None, progress=None)
        计算坡度,并返回坡度栅格数据集,即坡度图。 坡度是地表面上某一点的切面和水平面所成的夹角。坡度值越大,表示地势越陡峭
        
        注意:
            计算坡度时,要求待计算的栅格值(即高程)的单位与 x,y 坐标的单位相同。如果不一致,可通过高程缩放系数(方法中对应 zFactor 参数)来调整。
            但注意,当高程值单位与坐标单位间的换算无法通过固定值来调节时,则需要通过其他途径对数据进行处理。最常见的情况之一是 DEM 栅格采用地理坐标系时,
            单位为度,而高程值单位为米,此时建议对 DEM 栅格进行投影转换,将 x,y 坐标转换为平面坐标。
        
        :param input\_data: 指定的的待计算坡度的栅格数据集
        :type input\_data: DatasetGrid or str
        :param slope\_type: 坡度的单位类型
        :type slope\_type: SlopeType or str
        :param float z\_factor: 指定的高程缩放系数。该值是指在栅格中,栅格值(Z 坐标,即高程值)相对于 X 和 Y 坐标的单位变换系数。通常有 X,Y,Z 都参加的计算中,需要将高程值乘以一个高程缩放系数,使得三者单位一致。例如,X、Y 方向上的单位是米,而 Z 方向的单位是 英尺,由于 1 英尺等于 0.3048 米,则需要指定缩放系数为 0.3048。如果设置为 1.0,表示不缩放。
        :param out\_data: 结果数据集所在的数据源
        :type out\_data: Datasource or DatasourceConnectionInfo or str
        :param str out\_dataset\_name: 结果数据集名称
        :param function progress: 进度信息处理函数,具体参考 :py:class:`.StepEvent`
        :return: 结果数据集或数据集名称
        :rtype: DatasetGrid or str
    
    clip\_raster(input\_data, clip\_region, is\_clip\_in\_region=True, is\_exact\_clip=False, out\_data=None, out\_dataset\_name=None, progress=None)
        对栅格或影像数据集进行裁剪,结果存储为一个新的栅格或影像数据集。有时,我们的研究范围或者感兴趣区域较小,仅涉及当前栅格数据
        的一部分,这时可以对栅格数据进行裁剪,即通过一个 GeoRegion 对象作为裁剪区域对栅格数据进行裁剪,提取该区域内(外)的栅格数
        据生成一个新的数据集,此外,还可以选择进行精确裁剪或显示裁剪。
        
        :param input\_data:  指定的要进行裁剪的数据集,支持栅格数据集和影像数据集。
        :type input\_data: DatasetGrid or DatasetImage or str
        :param clip\_region: 裁剪区域
        :type clip\_region: GeoRegion or Rectangle
        :param bool is\_clip\_in\_region: 是否对裁剪区内的数据集进行裁剪。若为 True,则对裁剪区域内的数据集进行裁剪,若为 False,则对裁剪区域外的数据集进行裁剪。
        :param bool is\_exact\_clip: 是否使用精确裁剪。若为 True,表示使用精确裁剪对栅格或影像数据集进行裁剪,False 表示使用显示裁剪:
        
                                    - 采用显示裁剪时,系统会按照像素分块(详见 DatasetGrid.block\_size\_option、DatasetImage.block\_size\_option 方法)的大小,
                                      对栅格或影像数据集进行裁剪。此时只有裁剪区域内的数据被保留,即如果裁剪区的边界没有恰好与单元格的边界重合,那么单元格将被分割,
                                      位于裁剪区的部分会保留下来;位于裁剪区域外,且在被裁剪的那部分栅格所在块的总范围内的栅格仍有栅格值,但不显示。此种方式适用于大数据的裁剪。
        
                                    - 采用精确裁剪时,系统在裁剪区域边界,会根据裁剪区域压盖的单元格的中心点的位置确定是否保留该单元格。如果使用区域内裁剪方式,单元格的中心点位于裁剪区内则保留,反之不保留。
        
        :param out\_data: 结果数据集所在的数据源或直接生成 tif 文件
        :type out\_data: Datasource or DatasourceConnectionInfo or str
        :param str out\_dataset\_name: 结果数据集名称。如果设置直接生成 tif 文件,则此参数无效。
        :param function progress: 进度信息处理函数,具体参考 :py:class:`.StepEvent`
        :return: 结果数据集或数据集名称或第三方影像文件路径。
        :rtype: DatasetGrid or DatasetImage or str
        
        >>> clip\_region = Rectangle(875.5, 861.2, 1172.6, 520.9)
        >>> result = clip\_raster(data\_dir + 'example\_data.udb/seaport', clip\_region, True, False, out\_data=out\_dir + 'clip\_seaport.tif')
        >>> result = clip\_raster(data\_dir + 'example\_data.udb/seaport', clip\_region, True, False, out\_data=out\_dir + 'clip\_out.udb')
    
    clip\_vector(input\_data, clip\_region, is\_clip\_in\_region=True, is\_erase\_source=False, out\_data=None, out\_dataset\_name=None, progress=None)
        对矢量数据集进行裁剪,结果存储为一个新的矢量数据集。
        
        :param input\_data: 指定的要进行裁剪的矢量数据集,支持点、线、面、文本、CAD 数据集。
        :type input\_data: DatasetVector or str
        :param GeoRegion  clip\_region: 指定的裁剪区域
        :param bool is\_clip\_in\_region: 指定是否对裁剪区内的数据集进行裁剪。若为 True,则对裁剪区域内的数据集进行裁剪,若为 False ,则对裁剪区域外的数据集进行裁剪。
        :param bool is\_erase\_source: 指定是否擦除裁剪区域,若为 True,表示对裁剪区域进行擦除,若为 False,则不对裁剪区域进行擦除。
        :param out\_data: 结果数据集所在的数据源
        :type out\_data: Datasource or DatasourceConnectionInfo or str
        :param str out\_dataset\_name: 结果数据集名称
        :param function progress: 进度信息处理函数,具体参考 :py:class:`.StepEvent`
        :return: 结果数据集或数据集名称
        :rtype: DatasetVector or str
    
    cluster\_outlier\_analyst(source, assessment\_field, concept\_model='INVERSEDISTANCE', distance\_method='EUCLIDEAN', distance\_tolerance=-1.0, exponent=1.0, is\_FDR\_adjusted=False, k\_neighbors=1, is\_standardization=False, weight\_file\_path=None, out\_data=None, out\_dataset\_name=None, progress=None)
        聚类分布介绍:
        
            聚类分布可识别一组数据具有统计显著性的热点、冷点或者空间异常值。
        
            聚类分布用来计算的数据可以是点、线、面。对于点、线和面对象,在距离计算中会使用对象的质心。对象的质心为所有子对象的加权
            平均中心。点对象的加权项为1(即质心为自身),线对象的加权项是长度,而面对象的加权项是面积。
        
            用户可以通过聚类分布计算来解决以下问题:
        
                1. 聚类或冷点和热点出现在哪里?
                2. 空间异常值的出现位置在哪里?
                3. 哪些要素十分相似?
        
            聚类分布包括聚类和异常值分析(:py:func:`cluster\_outlier\_analyst`)、热点分析(:py:func:`hot\_spot\_analyst`)、
            优化热点分析(:py:func:`optimized\_hot\_spot\_analyst`)等
        
        
        聚类和异常值分析,返回结果矢量数据集。
        
         * 结果数据集中包括局部莫兰指数(ALMI\_MoranI)、z得分(ALMI\_Zscore)、P值(ALMI\_Pvalue)和聚类和异常值类型(ALMI\_Type)。
         * z得分和P值都是统计显著性的度量,用于逐要素的判断是否拒绝"零假设"。置信区间字段会识别具有统计显著性的聚类和异常值。如果,
           要素的Z得分是一个较高的正值,则表示周围的要素拥有相似值(高值或低值),聚类和异常值类型字段将具有统计显著性的高值聚类表示
           为"HH",将具有统计显著性的低值聚类表示为"LL";如果,要素的Z得分是一个较低的负值值,则表示有一个具有统计显著性的空间数据异常
           值,聚类和异常值类型字段将指出低值要素围绕高值要素表示为"HL",将高值要素围绕低值要素表示为"LH"。
         * 在没有设置 is\_FDR\_adjusted,统计显著性以P值和Z字段为基础,否则,确定置信度的关键P值会降低以兼顾多重测试和空间依赖性。
        
         .. image:: ../image/ClusteringDistributions\_clusterOutlierAnalyst.png
        
        
        :param source: 待计算的数据集。可以为点、线、面数据集。
        :type source: DatasetVector or str
        :param str assessment\_field: 评估字段的名称。仅数值字段有效。
        :param concept\_model: 空间关系概念化模型。默认值 :py:attr:`.ConceptualizationModel.INVERSEDISTANCE`。
        :type concept\_model: ConceptualizationModel or str
        :param distance\_method: 距离计算方法类型
        :type distance\_method: DistanceMethod or str
        :param float distance\_tolerance: 中断距离容限。仅对概念化模型设置为 :py:attr:`.ConceptualizationModel.INVERSEDISTANCE` 、
                                         :py:attr:`.ConceptualizationModel.INVERSEDISTANCESQUARED` 、
                                         :py:attr:`.ConceptualizationModel.FIXEDDISTANCEBAND` 、
                                         :py:attr:`.ConceptualizationModel.ZONEOFINDIFFERENCE` 时有效。
        
                                         为"反距离"和"固定距离"模型指定中断距离。"-1"表示计算并应用默认距离,此默认值为保证每个要
                                         素至少有一个相邻的要素;"0"表示为未应用任何距离,则每个要素都是相邻要素。
        
        :param float exponent: 反距离幂指数。仅对概念化模型设置为 :py:attr:`.ConceptualizationModel.INVERSEDISTANCE` 、
                                         :py:attr:`.ConceptualizationModel.INVERSEDISTANCESQUARED` 、
                                         :py:attr:`.ConceptualizationModel.ZONEOFINDIFFERENCE` 时有效。
        :param bool is\_FDR\_adjusted: 是否进行FDR(错误发现率)校正。若进行FDR(错误发现率)校正,则统计显著性将以错误发现率校正为基础,否则,统计显著性将以P值和z得分字段为基础。
        :param int k\_neighbors:  相邻数目,目标要素周围最近的K个要素为相邻要素。仅对概念化模型设置为 :py:attr:`.ConceptualizationModel.KNEARESTNEIGHBORS` 时有效。
        :param bool is\_standardization: 是否对空间权重矩阵进行标准化。若进行标准化,则每个权重都会除以该行的和。
        :param str weight\_file\_path: 空间权重矩阵文件路径
        :param out\_data: 结果数据源
        :type out\_data: Datasource or DatasourceConnectionInfo or str
        :param str out\_dataset\_name: 结果数据集名称
        :param progress: 进度信息,具体参考 :py:class:`.StepEvent`
        :type progress: function
        :return: 结果数据集或数据集名称
        :rtype: DatasetVector or str
    
    collect\_events(source, out\_data=None, out\_dataset\_name=None, progress=None)
        收集事件,将事件数据转换成加权数据。
        
         * 结果点数据集中包含一个 Counts 字段,该字段会保存每个唯一位置所有质心的总和。
        
         * 收集事件只会处理质心坐标完全相同的对象,并且只会保留一个质心,去除其余的重复点。
        
         * 对于点、线和面对象,在距离计算中会使用对象的质心。对象的质心为所有子对象的加权平均中心。点对象的加权项为1(即质心为自身),
           线对象的加权项是长度,而面对象的加权项是面积。
        
        
        :param source: 待收集的数据集。可以为点、线、面数据集。
        :type source: DatasetVector or str
        :param out\_data: 用于存储结果点数据集的数据源。
        :type out\_data: Datasource or DatasourceConnectionInfo or str
        :param str out\_dataset\_name: 结果点数据集名称。
        :param progress: 进度信息,具体参考 :py:class:`.StepEvent`
        :type progress: function
        :return: 结果数据集或数据集名称
        :rtype: DatasetVector or str
    
    compute\_min\_distance(source, reference, min\_distance, max\_distance, out\_data=None, out\_dataset\_name=None, progress=None)
        最近距离计算。求算“被计算记录集”中每一个对象到“参考记录集”中在查询范围内的所有对象的距离中的最小值(即最近距离),并将最近距离信息保存到一个新的属性表数据集中。
        最近距离计算功能用于计算“被计算记录集”中每一个对象(称为“被计算对象”)到“参考记录集”中在查询范围内的所有对象(称为“参考对象”)的距离中的最小值,也就是最近距离,计算的结果为一个纯属性表数据集,记录了“被计算对象”到最近的“参考对象”的距离信息,使用三个属性字段存储,分别为:Source\_ID(“被计算对象”的 SMID)、根据参考对象的类型可能为 Point\_ID、Line\_ID、Region\_ID(“参考对象”的 SMID)以及 Distance(前面二者的距离值)。如果被计算对象与多个参考对象具有最近距离,则属性表中相应的添加多条记录。
        
        * 支持的数据类型
        
          “被计算记录集”仅支持二维点记录集,“参考记录集”可以是为从二维点、线、面数据集以及二维网络数据集获得的记录集。从二维网络数据集可以获得存有弧段的记录集,或存有结点的记录集(从网络数据集的子集获取),将这两种记录集作为“参考记录集”,可用于查找最近的弧段或最近的结点。
        
          “被计算记录集”和“参考记录集”可以是同一个记录集,也可以是从同一个数据集查询出的不同记录集,这两种情况下,不会计算对象到自身的距离。
        
        * 查询范围
        
          查询范围由用户指定的一个最小距离和一个最大距离构成,用于过滤不参与计算的“参考对象”,即从“被计算对象”出发,只有与其距离介于最小距离和最大距离之间(包括等于)的“参考对象”参与计算。如果将查询范围设置为从“0”到“-1”,则表示计算到“参考记录集”中所有对象的最近距离。
        
          如下图所示,红色圆点来自“被计算记录集”,方块来自“参考记录集”,粉色区域表示查询范围,则只有位于查询范围内的蓝色方块参与最近距离计算,也就是说本例的计算的结果只包含红色圆点与距其最近的蓝色方块的 SMID 和距离值
        
          .. image:: ../image/ComputeDistance.png
        
        * 注意事项:
        
          * “被计算记录集”和“参考记录集”所属的数据集的必须具有相同的坐标系。
        
          * 如下图所示,点到线对象的距离,是计算点到整个线对象的最小距离,即在线上找到一点与被计算点的距离最短;同样的,点到面对象的距离,是计算点到面对象的整个边界的最小距离。
        
            .. image:: ../image/ComputeDistance\_1.png
        
          * 计算两个对象间距离时,出现包含或(部分)重叠的情况时,距离均为 0。例如点对象在线对象上,二者间距离为 0。
        
        :param source:  指定的被计算记录集。只支持二维点记录集和数据集
        :type source: DatasetVector or Recordset or str
        :param reference: 指定的参考记录集。支持二维点、线、面记录集和数据集
        :type reference: DatasetVector or Recordset or str
        :param min\_distance: 指定的查询范围的最小距离。取值范围为大于或等于 0。单位与被计算记录集所属数据集的单位相同。
        :type min\_distance: float
        :param max\_distance:  指定的查询范围的最大距离。取值范围为大于 0 的值及 -1。当设置为 -1 时,表示不限制最大距离。单位与被计算记录集所属数据集的单位相同。
        :type max\_distance: float
        :param out\_data: 指定的用于存储结果属性表数据集的数据源。
        :type out\_data: Datasource or DatasourceConnectionInfo or str
        :param out\_dataset\_name:  指定的结果属性表数据集的名称。
        :type out\_dataset\_name: str
        :param progress: 进度信息处理函数,具体参考 :py:class:`.StepEvent`
        :type progress: function
        :return:  结果数据集或数据集名称
        :rtype: DatasetVector
    
    compute\_point\_aspect(input\_data, specified\_point)
        计算 DEM 栅格上指定点处的坡向。 DEM 栅格上指定点处的坡向,与坡向图(calculate\_aspect 方法)的计算方法相同,是将该点所在单元格与其周围的相
        邻的八个单元格所形成的 3 × 3 平面作为计算单元,通过三阶反距离平方权差分法计算水平高程变化率和垂直高程变化率从而得出坡向。更多介绍,请参阅 :py:meth:`calculate\_aspect` 方法。
        
        注意:
            当指定点所在的单元格为无值时,计算结果为 -1,这与生成坡向图不同;当指定的点位于 DEM 栅格的数据集范围之外时,计算结果为 -1。
        
        :param input\_data: 指定的待计算坡向的栅格数据集
        :type input\_data: DatasetGrid or str
        :param Point2D specified\_point:  指定的地理坐标点。
        :return: 指定点处的坡向。单位为度。
        :rtype: float
    
    compute\_point\_slope(input\_data, specified\_point, slope\_type, z\_factor)
        计算 DEM 栅格上指定点处的坡度。
        DEM 栅格上指定点处的坡度,与坡度图(calculate\_slope 方法)的计算方法相同,是将该点所在单元格与其周围的相邻的八个单元格所形成的 3 × 3 平面作
        为计算单元,通过三阶反距离平方权差分法计算水平高程变化率和垂直高程变化率从而得出坡度。更多介绍,请参阅 calculate\_slope 方法。
        
        注意:
            当指定点所在的单元格为无值时,计算结果为 -1,这与生成坡度图不同;当指定的点位于 DEM 栅格的数据集范围之外时,计算结果为 -1。
        
        :param input\_data: 指定的待计算坡向的栅格数据集
        :type input\_data: DatasetGrid or str
        :param Point2D specified\_point: 指定的地理坐标点。
        :param slope\_type: 指定的坡度单位类型。可以用角度、弧度或百分数来表示。以使用角度为例,坡度计算的结果范围为 0~90 度。
        :type slope\_type: SlopeType or str
        :param float z\_factor: 指定的高程缩放系数。该值是指在 DEM 栅格中,栅格值(Z 坐标,即高程值)相对于 X 和 Y 坐标的单位变换系数。通常有 X,Y,Z 都参加的计算中,需要将高程值乘以一个高程缩放系数,使得三者单位一致。例如,X、Y 方向上的单位是米,而 Z 方向的单位是英尺,由于 1 英尺等于 0.3048 米,则需要指定缩放系数为 0.3048。如果设置为 1.0,表示不缩放。
        :return: 指定点处的坡度。单位为 type 参数指定的类型。
        :rtype: float
    
    compute\_range\_distance(source, reference, min\_distance, max\_distance, out\_data=None, out\_dataset\_name=None, progress=None)
        范围距离计算。求算“被计算记录集”中每一个对象到“参考记录集”中在查询范围内的每一个对象的距离,并将距离信息保存到一个新的属性表数据集中。
        
        该功能用于计算记录集 A 中每一个对象到记录集 B 中在查询范围内的每一个对象的距离,记录集 A 称为“被计算记录集”,当中的对象称作“被计算对象”,记录集 B 称为“参考记录集”,当中的对象称作“参考对象”。“被计算记录集”和“参考记录集”可以是同一个记录集,也可以是从同一个数据集查询出的不同记录集,这两种情况下,不会计算对象到自身的距离。
        
        查询范围由一个最小距离和一个最大距离构成,用于过滤不参与计算的“参考对象”,即从“被计算对象”出发,只有与其距离介于最小距离和最大距离之间(包括等于)的“参考对象”参与计算。
        
        如下图所示,红色圆点为“被计算对象”,方块为“参考对象”,粉色区域表示查询范围,则只有位于查询范围内的蓝色方块参与距离计算,也就是说本例的计算的结果只包含红色圆点与粉色区域内的蓝色方块的 SMID 和距离值。
        
        .. image:: ../image/ComputeDistance.png
        
        范围距离计算的结果为一个纯属性表数据集,记录了“被计算对象”到“参考对象”的距离信息,使用三个属性字段存储,分别为:Source\_ID(“被计算对象”的 SMID)、根据参考对象的类型可能为 Point\_ID、Line\_ID、Region\_ID(“参考对象”的 SMID)以及 Distance(前面二者的距离值)。
        
        注意事项:
        
         * “被计算记录集”和“参考记录集”所属的数据集的必须具有相同的坐标系。
        
         * 如下图所示,点到线对象的距离,是计算点到整个线对象的最小距离,即在线上找到一点与被计算点的距离最短;同样的,点到面对象的距离,是计算点到面对象的整个边界的最小距离。
        
           .. image:: ../image/ComputeDistance\_1.png
        
         * 计算两个对象间距离时,出现包含或(部分)重叠的情况时,距离均为 0。例如点对象在线对象上,二者间距离为 0。
        
        
        :param source: 指定的被计算记录集。只支持二维点记录集或数据集
        :type source: DatasetVector or Recordset or str
        :param reference: 指定的参考记录集。只支持二维点、线、面记录集或数据集
        :type reference: DatasetVector or Recordset or str
        :param min\_distance: 指定的查询范围的最小距离。取值范围为大于或等于 0。 单位与被计算记录集所属数据集的单位相同。
        :type min\_distance: float
        :param max\_distance: 指定的查询范围的最大距离。取值范围为大于或等于 0,且必须大于或等于最小距离。单位与被计算记录集所属数据集的单位相同。
        :type max\_distance: float
        :param out\_data: 指定的用于存储结果属性表数据集的数据源。
        :type out\_data: Datasource or DatasourceConnectionInfo or str
        :param out\_dataset\_name: 指定的结果属性表数据集的名称。
        :type out\_dataset\_name: str
        :param progress: 进度信息处理函数,具体参考 :py:class:`.StepEvent`
        :type progress: function
        :return: 结果数据集或数据集名称
        :rtype: DatasetVector
    
    compute\_range\_raster(input\_data, count, progress=None)
        计算栅格像元值的自然断点中断值
        
        :param input\_data: 栅格数据集
        :type input\_data: DatasetGrid or str
        :param count: 自然分段的个数
        :type count: int
        :param progress: 进度信息处理函数,具体参考 :py:class:`.StepEvent`
        :return: 自然分段的中断值(包括像元的最大和最小值)
        :rtype: Array
    
    compute\_range\_vector(input\_data, value\_field, count, progress=None)
        计算矢量自然断点中断值
        
        :param input\_data: 矢量数据集
        :type input\_data: DatasetVector or str
        :param value\_field: 分段的标准字段
        :type value\_field: str
        :param count: 自然分段的个数
        :type count: int
        :param progress: 进度信息处理函数,具体参考 :py:class:`.StepEvent`
        :return: 自然分段的中断值(包括属性的最大和最小值)
        :rtype: Array
    
    compute\_surface\_area(input\_data, region)
        计算表面面积,即计算所选多边形区域内的 DEM 栅格拟合的三维曲面的总的表面面积。
        
        :param input\_data: 指定的待计算表面面积的 DEM 栅格。
        :type input\_data: DatasetGrid or str
        :param GeoRegion region: 指定的用于计算表面面积的多边形
        :return: 表面面积的值。单位为平方米。返回 -1 表示计算失败。
        :rtype: float
    
    compute\_surface\_distance(input\_data, line)
        计算栅格表面距离,即计算在 DEM 栅格拟合的三维曲面上沿指定的线段或折线段的曲面距离。
        
        注意:
            - 表面量算所量算的距离是曲面上的,因而比平面上的值要大。
            - 当用于量算的线超出了 DEM 栅格的范围时,会先按数据集范围对线对象进行裁剪,按照位于数据集范围内的那部分线来计算表面距离。
        
        :param input\_data:  指定的待计算表面距离的 DEM 栅格。
        :type input\_data: DatasetGrid or str
        :param GeoLine line: 用于计算表面距离的二维线。
        :return: 表面距离的值。单位为米。
        :rtype: float
    
    compute\_surface\_volume(input\_data, region, base\_value)
        计算表面体积,即计算所选多边形区域内的 DEM 栅格拟合的三维曲面与一个基准平面之间的空间上的体积。
        
        :param input\_data: 待计算体积的 DEM 栅格。
        :type input\_data: DatasetGrid or str
        :param GeoRegion region: 用于计算体积的多边形。
        :param float base\_value: 基准平面的值。单位与待计算的 DEM 栅格的栅格值单位相同。
        :return: 指定的基准平面的值。单位与待计算的 DEM 栅格的栅格值单位相同。
        :rtype: float
    
    cost\_distance(input\_data, cost\_grid, max\_distance=-1.0, cell\_size=None, out\_data=None, out\_distance\_grid\_name=None, out\_direction\_grid\_name=None, out\_allocation\_grid\_name=None, progress=None)
        根据给定的参数,生成耗费距离栅格,以及耗费方向栅格和耗费分配栅格。
        
        实际应用中,直线距离往往不能满足要求。例如,从 B 到最近源 A 的直线距离与从 C 到最近源 A 的直线距离相同,若 BA 路段交通拥堵,而 CA 路段交通畅
        通,则其时间耗费必然不同;此外,通过直线距离对应的路径到达最近源时常常是不可行的,例如,遇到河流、高山等障碍物就需要绕行,这时就需要考虑其耗费距离。
        
        该方法根据源数据集和耗费栅格生成相应的耗费距离栅格、耗费方向栅格(可选)和耗费分配栅格(可选)。源数据可以是矢量数据(点、线、面),也可以是栅格数据。
        对于栅格数据,要求除标识源以外的单元格为无值。
        
         * 耗费距离栅格的值表示该单元格到最近源的最小耗费值(可以是各种类型的耗费因子,也可以是各感兴趣的耗费因子的加权)。最近源
           是当前单元格到达所有的源中耗费最小的一个源。耗费栅格中为无值的单元格在输出的耗费距离栅格中仍为无值。
        
           单元格到达源的耗费的计算方法是,从待计算单元格的中心出发,到达最近源的最小耗费路径在每个单元格上经过的距离乘以耗费栅格
           上对应单元格的值,将这些值累加即为单元格到源的耗费值。因此,耗费距离的计算与单元格大小和耗费栅格有关。在下面的示意图中,
           源栅格和耗费栅格的单元格大小(cell\_size)均为2,单元格(2,1)到达源(0,0)的最小耗费路线如右图中红线所示:
        
           .. image:: ../image/CostDistance\_1.png
        
           那么单元格(2,1)到达源的最小耗费(即耗费距离)为:
        
           .. image:: ../image/CostDistance\_2.png
        
         * 耗费方向栅格的值表达的是从该单元格到达最近的源的最小耗费路径的行进方向。在耗费方向栅格中,可能的行进方向共有八个(正北、
           正南、正西、正东、西北、西南、东南、东北),使用1到8八个整数对这八个方向进行编码,如下图所示。注意,源所在的单元格在耗费
           方向栅格中的值为0,耗费栅格中为无值的单元格在输出的耗费方向栅格中将被赋值为15。
        
           .. image:: ../image/CostDistance\_3.png
        
         * 耗费分配栅格的值为单元格的最近源的值(源为栅格时,为最近源的栅格值;源为矢量对象时,为最近源的 SMID),单元格到达最近的
           源具有最小耗费距离。耗费栅格中为无值的单元格在输出的耗费分配栅格中仍为无值。
        
           下图为生成耗费距离的示意图。其中,在耗费栅格上,使用蓝色箭头标识了单元格到达最近源的行进路线,耗费方向栅格的值即标示了
           当前单元格到达最近源的最小耗费路线的行进方向。
        
           .. image:: ../image/CostDistance\_4.png
        
        下图为生成耗费距离栅格的一个实例,其中源数据集为点数据集,耗费栅格为对应区域的坡度栅格的重分级结果,生成了耗费距离栅格、耗费方向栅格和耗费分配栅格。
        
        .. image:: ../image/CostDistance.png
        
        :param input\_data: 生成距离栅格的源数据集。源是指感兴趣的研究对象或地物,如学校、道路或消防栓等。包含源的数据集,即为源数据集。源数据集可以为
                            点、线、面数据集,也可以为栅格数据集,栅格数据集中具有有效值的栅格为源,对于无值则视为该位置没有源。
        :type input\_data: DatasetVector or DatasetGrid or str
        :param DatasetGrid cost\_grid:  耗费栅格。其栅格值不能为负值。该数据集为一个栅格数据集,每个单元格的值表示经过此单元格时的单位耗费。
        :param float max\_distance: 生成距离栅格的最大距离,大于该距离的栅格其计算结果取无值。若某个栅格单元格 A 到最近源之间的最短距离大于该值,则结果数据集中该栅格的值取无值。
        :param float cell\_size: 结果数据集的分辨率,是生成距离栅格的可选参数
        :param out\_data: 结果数据集所在的数据源
        :type out\_data:  Datasource or DatasourceConnectionInfo or str
        :param str out\_distance\_grid\_name: 结果距离栅格数据集的名称。如果名称为空,将自动获取有效的数据集名称。
        :param str out\_direction\_grid\_name: 方向栅格数据集的名称,如果为空,将不生成方向栅格数据集
        :param str out\_allocation\_grid\_name:  分配栅格数据集的名称,如果为空,将不生成 分配栅格数据集
        :param function progress: 进度信息处理函数,具体参考 :py:class:`.StepEvent`
        :return: 如果生成成功,返回结果数据集或数据集名称的元组,其中第一个为距离栅格数据集,第二个为方向栅格数据集,第三个为分配栅格数据集,如果没有设置方向栅格数据集名称和
                 分配栅格数据集名称,对应的值为 None
        :rtype: tuple[DataetGrid] or tuple[str]
    
    cost\_path(input\_data, distance\_dataset, direction\_dataset, compute\_type, out\_data=None, out\_dataset\_name=None, progress=None)
        根据耗费距离栅格和耗费方向栅格,分析从目标出发到达最近源的最短路径栅格。
        该方法根据给定的目标数据集,以及通过“生成耗费距离栅格”功能得到的耗费距离栅格和耗费方向栅格,来计算每个目标对象到达最近的源的最短路径,也就是最小
        耗费路径。该方法不需要指定源所在的数据集,因为源的位置在距离栅格和方向栅格中能够体现出来,即栅格值为 0 的单元格。生成的最短路径栅格是一个二值栅
        格,值为 1 的单元格表示路径,其他单元格的值为 0。
        
        例如,将购物商场(一个点数据集)作为源,各居民小区(一个面数据集)作为目标,分析从各居民小区出发,如何到达距其最近的购物商场。实现的过程是,首先
        针对源(购物商场)生成距离栅格和方向栅格,然后将居民小区作为目标区域,通过最短路径分析,得到各居民小区(目标)到最近购物商场(源)的最短路径。该
        最短路径包含两种含义:通过直线距离栅格与直线方向栅格,将得到最小直线距离路径;通过耗费距离栅格与耗费方向栅格,则得到最小耗费路径。
        
        注意,该方法中要求输入的耗费距离栅格和耗费方向栅格必须是匹配的,也就是说二者应是同一次使用“生成耗费距离栅格”功能生成的。此外,有三种计算最短路径
        的方式:像元路径、区域路径和单一路径,具体含义请参见 :py:class:`.ComputeType` 类。
        
        :param input\_data: 目标所在的数据集。可以为点、线、面或栅格数据集。如果是栅格数据,要求除标识目标以外的单元格为无值。
        :type input\_data: DatasetVector or DatasetGrid or DatasetImage or str
        :param distance\_dataset: 耗费距离栅格数据集。
        :type distance\_dataset: DatasetGrid or str
        :param direction\_dataset:  耗费方向栅格数据集
        :type direction\_dataset: DatasetGrid or str
        :param compute\_type: 栅格距离最短路径分析的计算方式
        :type compute\_type: ComputeType or str
        :param out\_data: 结果数据集所在的数据源
        :type out\_data: Datasource or DatasourceConnectionInfo or str
        :param str out\_dataset\_name: 结果数据集名称
        :param function progress: 进度信息处理函数,具体参考 :py:class:`.StepEvent`
        :return: 结果数据集或数据集名称
        :rtype: DatasetVector or str
    
    cost\_path\_line(source\_point, target\_point, cost\_grid, smooth\_method=None, smooth\_degree=0, progress=None)
        根据给定的参数,计算源点和目标点之间的最小耗费路径(一个二维矢量线对象)。该方法用于根据给定的源点、目标点和耗费栅格,计算源点与目标点之间的最小耗费路径
        
        下图为计算两点间最小耗费路径的实例。该例以 DEM 栅格的坡度的重分级结果作为耗费栅格,分析给定的源点和目标点之间的最小耗费路径。
        
        .. image:: ../image/CostPathLine.png
        
        :param Point2D source\_point: 指定的源点
        :param Point2D target\_point: 指定的目标点
        :param DatasetGrid cost\_grid:  耗费栅格。其栅格值不能为负值。该数据集为一个栅格数据集,每个单元格的值表示经过此单元格时的单位耗费。
        :param smooth\_method: 计算两点(源和目标)间最短路径时对结果路线进行光滑的方法
        :type smooth\_method: SmoothMethod or str
        :param int smooth\_degree: 计算两点(源和目标)间最短路径时对结果路线进行光滑的光滑度。
                                    光滑度的值越大,光滑度的值越大,则结果矢量线的光滑度越高。当 smooth\_method 不为 NONE 时有效。光滑度的有效取值与光滑方法有关,光滑方法有 B 样条法和磨角法:
                                    - 光滑方法为 B 样条法时,光滑度的有效取值为大于等于2的整数,建议取值范围为[2,10]。
                                    - 光滑方法为磨角法时,光滑度代表一次光滑过程中磨角的次数,设置为大于等于1的整数时有效
        :param function progress: 进度信息处理函数,具体参考 :py:class:`.StepEvent`
        :return: 返回表示最短路径的线对象和最短路径的花费
        :rtype: tuple[GeoLine,float]
    
    create\_buffer(input\_data, distance\_left, distance\_right=None, unit=None, end\_type=None, segment=24, is\_save\_attributes=True, is\_union\_result=False, out\_data=None, out\_dataset\_name='BufferResult', progress=None)
        创建矢量数据集或记录集的缓冲。
        
        缓冲区分析是围绕空间对象,使用一个或多个与这些对象的距离值(称为缓冲半径)作为半径,生成一个或多个区域的过程。缓冲区也可以理解为空间对象的一种影响或服务范围。
        
        缓冲区分析的基本作用对象是点、线、面。SuperMap 支持对二维点、线、面数据集(或记录集)和网络数据集进行缓冲区分析。其中,对网络数据集进行缓冲区分析时,是对其中的弧段作缓冲区。缓冲区的类型可以分析单重缓冲区(或称简单缓冲区)和多重缓冲区。下面以简单缓冲区为例分别介绍点、线、面的缓冲区。
        
        * 点缓冲区
          点的缓冲区是以点对象为圆心,以给定的缓冲距离为半径生成的圆形区域。当缓冲距离足够大时,两个或多个点对象的缓冲区可能有重叠。选择合并缓冲区时,重叠部分将被合并,最终得到的缓冲区是一个复杂面对象。
        
          .. image:: ../image/PointBuffer.png
        
        * 线缓冲区
          线的缓冲区是沿线对象的法线方向,分别向线对象的两侧平移一定的距离而得到两条线,并与在线端点处形成的光滑曲线(也可以形成平头)接合形成的封闭区域。同样,当缓冲距离足够大时,两个或多个线对象的缓冲区可能有重叠。合并缓冲区的效果与点的合并缓冲区相同。
        
          .. image:: ../image/LineBuffer.png
        
          线对象两侧的缓冲宽度可以不一致,从而生成左右不等缓冲区;也可以只在线对象的一侧创建单边缓冲区。此时只能生成平头缓冲区。
        
          .. image:: ../image/LineBuffer\_1.png
        
        * 面缓冲区
        
          面的缓冲区生成方式与线的缓冲区类似,区别是面的缓冲区仅在面边界的一侧延展或收缩。当缓冲半径为正值时,缓冲区向面对象边界的外侧扩展;为负值时,向边界内收缩。同样,当缓冲距离足够大时,两个或多个线对象的缓冲区可能有重叠。也可以选择合并缓冲区,其效果与点的合并缓冲区相同。
        
          .. image:: ../image/RegionBuffer.png
        
        * 多重缓冲区是指在几何对象的周围,根据给定的若干缓冲区半径,建立相应数据量的缓冲区。对于线对象,还可以建立单边多重缓冲区,但注意不支持对网络数据集创建。
        
          .. image:: ../image/MultiBuffer.png
        
        
        缓冲区分析在 GIS 空间分析中经常用到,且往往结合叠加分析来共同解决实际问题。缓冲区分析在农业、城市规划、生态保护、防洪抗灾、军事、地质、环境等诸多领域都有应用。
        
        例如扩建道路时,可根据道路扩宽宽度对道路创建缓冲区,然后将缓冲区图层与建筑图层叠加,通过叠加分析查找落入缓冲区而需要被拆除的建筑;又如,为了保护环境和耕地,可对湿地、森林、草地和耕地进行缓冲区分析,在缓冲区内不允许进行工业建设。
        
        说明:
        
        *  对于面对象,在做缓冲区分析前最好先经过拓扑检查,排除面内相交的情况,所谓面内相交,指的是面对象自身相交,如图所示,图中数字代表面对象的节点顺序。
        
        .. image:: ../image/buffer\_regioninter.png
        
        * 对“负半径”的说明
        
            * 如果缓冲区半径为数值型,则仅面数据支持负半径;
            * 如果缓冲区半径为字段或字段表达式,如果字段或字段表达式的值为负值,对于点、线数据取其绝对值;对于面数据,若合并缓冲区,则取其绝对值,若不合并,则按照负半径处理。
        
        
        :param input\_data: 指定的创建缓冲区的源矢量记录集是数据集。支持点、线、面数据集和记录集。
        :type input\_data: Recordset or DatasetVector or str
        :param distance\_left: (左)缓冲区的距离。如果为字符串,则表示(左)缓冲距离所在的字段,即每个几何对象创建缓冲区时使用字段中存储的值作为缓冲半径。对于线对象,表示左缓冲区半径,对于点和面对象,表示缓冲区半径。
        :type distance\_left: float or str
        :param distance\_right: 右缓冲区的距离,如果为字符串,则表示右缓冲距离所在的字段,即每个线几何对象创建缓冲区时使用字段中存储的值作为右缓冲半径。该参数只对线对象有效。
        :type distance\_right: float or str
        :param unit: 缓冲区距离半径单位,只支持距离单位,不支持角度和弧度单位。
        :type unit: Unit or str
        :param end\_type: 缓冲区端点类型。用以区分线对象缓冲区分析时的端点是圆头缓冲还是平头缓冲。对于点或面对象,只支持圆头缓冲
        :type end\_type: BufferEndType or str
        :param int segment: 半圆弧线段个数,即用多少个线段来模拟一个半圆,必须大于等于4。
        :param  bool is\_save\_attributes: 是否保留进行缓冲区分析的对象的字段属性。当合并结果面数据集时,该参数无效。即当 isUnion 参数为 false 时有效。
        :param bool is\_union\_result: 是否合并缓冲区,即是否将源数据各对象生成的所有缓冲区域进行合并运算后返回。对于面对象而言,要求源数据集中的面对象不相交。
        :param out\_data: 存储结果数据的数据源
        :type out\_data: Datasource
        :param str out\_dataset\_name: 结果数据集名称
        :param function progress: 进度信息处理函数,具体参考 :py:class:`.StepEvent`
        :return: 结果数据集或数据集名称
        :rtype: DatasetVector or str
    
    create\_line\_one\_side\_multi\_buffer(input\_data, radius, is\_left, unit=None, segment=24, is\_save\_attributes=True, is\_union\_result=False, is\_ring=True, out\_data=None, out\_dataset\_name='BufferResult', progress=None)
        创建矢量线数据集单边多重缓冲区。缓冲区介绍请参考 :py:meth:`create\_buffer`。
        线的单边多重缓冲区,是指在线对象的一侧生成多重缓冲区。左侧是指沿线对象的节点序列方向的左侧,右侧为节点序列方向的右侧。
        
        .. image:: ../image/LineOneSideMultiBuffer.png
        
        :param input\_data: 指定的创建多重缓冲区的源矢量数据集。只支持线数据集或线记录集
        :type input\_data: DatasetVector or Recordset
        :param radius: 指定的多重缓冲区半径列表。单位由 unit 参数指定。
        :type radius: list[float] or tuple[float] or str
        :param bool is\_left: 是否生成左缓冲区。设置为 True,在线的左侧生成缓冲区,否则在右侧生成缓冲区。
        :param unit:  指定的缓冲区半径单位。
        :type unit: BufferRadiusUnit
        :param int segment:  指定的弧段拟合数
        :param bool is\_save\_attributes: 是否保留进行缓冲区分析的对象的字段属性。当合并结果面数据集时,该参数无效,即当 is\_union\_result 为 False 时有效。
        :param bool is\_union\_result:  是否合并缓冲区,即是否将源数据各对象生成的所有缓冲区域进行合并运算后返回。
        :param bool is\_ring:  是否生成环状缓冲区。设置为 True,则生成多重缓冲区时外圈缓冲区是以环状区域与内圈数据相邻的;设置为 False,则外围缓冲区是一个包含了内圈数据的区域。
        :param out\_data: 存储结果数据的数据源
        :type out\_data: Datasource
        :param str out\_dataset\_name: 结果数据集名称
        :param function progress: 进度信息处理函数,具体参考 :py:class:`.StepEvent`
        :return: 结果数据集或数据集名称
        :rtype: DatasetVector or str
    
    create\_multi\_buffer(input\_data, radius, unit=None, segment=24, is\_save\_attributes=True, is\_union\_result=False, is\_ring=True, out\_data=None, out\_dataset\_name='BufferResult', progress=None)
        创建矢量数据集多重缓冲区。缓冲区介绍请参考 :py:meth:`create\_buffer`
        
        :param input\_data: 指定的创建多重缓冲区的源矢量数据集或记录集。支持点、线、面数据集和网络数据集。对网络数据集进行分析,是对其中的弧段作缓冲区。
        :type input\_data: DatasetVector or Recordset
        :param radius: 指定的多重缓冲区半径列表。单位由 unit 参数指定。
        :type radius: list[float] or tuple[float]
        :param unit: 指定的缓冲区半径单位。
        :type unit:  BufferRadiusUnit or str
        :param int segment: 指定的弧段拟合数。
        :param bool is\_save\_attributes: 是否保留进行缓冲区分析的对象的字段属性。当合并结果面数据集时,该参数无效,即当 is\_union\_result 为 False 时有效。
        :param bool is\_union\_result:  是否合并缓冲区,即是否将源数据各对象生成的所有缓冲区域进行合并运算后返回。
        :param bool is\_ring:  是否生成环状缓冲区。设置为 True,则生成多重缓冲区时外圈缓冲区是以环状区域与内圈数据相邻的;设置为 False,则外围缓冲区是一个包含了内圈数据的区域。
        :param out\_data: 存储结果数据的数据源
        :type out\_data: Datasource
        :param str out\_dataset\_name: 结果数据集名称
        :param function progress: 进度信息处理函数,具体参考 :py:class:`.StepEvent`
        :return: 结果数据集或数据集名称
        :rtype: DatasetVector or str
    
    create\_thiessen\_polygons(input\_data, clip\_region, field\_stats=None, out\_data=None, out\_dataset\_name=None, progress=None)
        创建泰森多边形。
        荷兰气候学家 A.H.Thiessen 提出了一种根据离散分布的气象站的降雨量来计算平均降雨量的方法,即将所有相邻气象站连成三角形,作这些三角形各边的垂直平分线,
        于是每个气象站周围的若干垂直平分线便围成一个多边形。用这个多边形内所包含的一个唯一气象站的降雨强度来表示这个多边形区域内的降雨强度,并称这个多边形为泰森多边形。
        
        泰森多边形的特性:
        
            - 每个泰森多边形内仅含有一个离散点数据;
            - 泰森多边形内的点到相应离散点的距离最近;
            - 位于泰森多边形边上的点到其两边的离散点的距离相等。
            - 泰森多边形可用于定性分析、统计分析、邻近分析等。例如,可以用离散点的性质来描述泰森多边形区域的性质;可用离散点的数据来计算泰森多边形区域的数据
            - 判断一个离散点与其它哪些离散点相邻时,可根据泰森多边形直接得出,且若泰森多边形是n边形,则就与n个离散点相邻;当某一数据点落入某一泰森多边形中时,它与相应的离散点最邻近,无需计算距离。
        
        
        邻近分析是 GIS 领域里又一个最为基础的分析功能之一,邻近分析是用来发现事物之间的某种邻近关系。邻近分析类所提供的进行邻近分析的方法都是实现泰森多边形的建立,
        就是根据所提供的点数据建立泰森多边形,从而获得点之间的邻近关系。泰森多边形用于将点集合中的点的周围区域分配给相应的点,使位于这个点所拥有的区域(即该点所关联的泰森多边形)
        内的任何地点离这个点的距离都要比离其他点的距离要小,同时,所建立的泰森多边形还满足上述所有的泰森多边形法的理论。
        
        泰森多边形是如何创建的?利用下面的图示来理解泰森多边形建立的过程:
        
            - 对待建立泰森多边形的点数据进行由左向右,由上到下的扫描,如果某个点距离之前刚刚扫描过的点的距离小于给定的邻近容限值,那么分析时将忽略该点;
            - 基于扫描检查后符合要求的所有点建立不规则三角网,即构建 Delaunay 三角网;
            - 画出每个三角形边的中垂线,由这些中垂线构成泰森多边形的边,而中垂线的交点是相应的泰森多边形的顶点;
            - 用于建立泰森多边形的点的点位将成为相应的泰森多边形的锚点。
        
        
        :param input\_data: 输入的点数据,可以为点数据集、点记录集或 :py:class:`.Point2D` 的列表
        :type input\_data: DatasetVector or Recordset or list[Point2D]
        :param GeoRegion clip\_region:  指定的裁剪结果数据的裁剪区域。该参数可以为空,如果为空,结果数据集将不进行裁剪
        :param field\_stats:  统计字段名称和对应的统计类型,输入为一个list,list中存储的每个元素为tuple,tuple的大小为2,第一个元素为被统计的字段名称,第二个元素为统计类型。
                             当 stats\_fields 为 str 时,支持设置 ',' 分隔多个字段,例如 "field1:SUM, field2:MAX, field3:MIN"
        :type field\_stats: list[str,StatisticsType] or list[str,str] or str
        :param out\_data: 结果面对象所在的数据源。如果 out\_data 为空,则会将生成的泰森多边形面几何对象直接返回
        :type out\_data:  Datasource or DatasourceConnectionInfo or str
        :param str out\_dataset\_name: 结果数据集名称,当 out\_data 不为空时才有效。
        :param function progress: 进度信息处理函数,具体参考 :py:class:`.StepEvent`
        :return: 如果 out\_data 为空,将返回 list[GeoRegion],否则返回结果数据集或数据集名称。
        :rtype: DatasetVector or str or list[GeoRegion]
    
    density\_based\_clustering(input\_data, min\_pile\_point\_count, search\_distance, unit, out\_data=None, out\_dataset\_name=None, progress=None)
        密度聚类的DBSCAN实现
        
        该方法根据给定的搜索半径(search\_distance)和该范围内需包含的最少点数(min\_pile\_point\_count)将空间点数据中密度足够大且空间相近的区域相连,并消除噪声的干扰,以达到较好的聚类效果。
        
        :param input\_data: 指定的要聚类的矢量数据集,支持点数据集。
        :type input\_data: DatasetVector or str
        :param min\_pile\_point\_count: 每类包含的最少点数
        :type min\_pile\_point\_count: int
        :param search\_distance: 搜索邻域的距离
        :type search\_distance: int
        :param unit: 搜索距离的单位
        :type unit: Unit
        :param out\_data: 结果数据集所在的数据源
        :type out\_data: Datasource or DatasourceConnectionInfo or str
        :param str out\_dataset\_name: 结果数据集名称
        :param function progress: 进度信息处理函数,具体参考 :py:class:`.StepEvent`
        :return: 结果数据集或数据集名称
        :rtype: DatasetVector or str
    
    density\_interpolate(input\_data, z\_value\_field, pixel\_format, resolution, search\_radius=0.0, expected\_count=12, bounds=None, z\_value\_scale=1.0, out\_data=None, out\_dataset\_name=None, progress=None)
        使用点密度插值方法对点数据集或记录集进行插值。具体参考 :py:meth:`interpolate` 和 :py:class:`.InterpolationDensityParameter`
        
        :param input\_data:  需要进行插值分析的点数据集或点记录集
        :type input\_data: DatasetVector or str or Recordset
        :param str z\_value\_field: 存储用于进行插值分析的值的字段名称。插值分析不支持文本类型的字段。
        :param pixel\_format: 指定结果栅格数据集存储的像素,不支持 BIT64
        :type pixel\_format: PixelFormat or str
        :param float resolution: 插值运算时使用的分辨率
        :param float search\_radius: 查找参与运算点的查找半径。单位与用于插值的点数据集(或记录集所属的数据集)的单位相同。查找半径决定了参与运算点的查找范围,当计算某个位置的未知数值时,会以该位置为圆心,以search\_radius为半径,落在这个范围内的采样点都将参与运算,即该位置的预测值由该范围内采样点的数值决定。
        :param int expected\_count: 期望参与插值运算的点数
        :param Rectangle bounds: 插值分析的范围,用于确定运行结果的范围
        :param float z\_value\_scale:  插值分析值的缩放比率
        :param out\_data: 结果数据集所在的数据源
        :type out\_data: Datasource or DatasourceConnectionInfo or str
        :param str out\_dataset\_name: 结果数据集名称
        :param function progress: 进度信息处理函数,具体参考 :py:class:`.StepEvent`
        :return: 结果数据集或数据集名称
        :rtype: DatasetGrid or str
    
    dissolve(input\_data, dissolve\_type, dissolve\_fields, field\_stats=None, attr\_filter=None, is\_null\_value\_able=True, is\_preprocess=True, tolerance=1e-10, out\_data=None, out\_dataset\_name='DissolveResult', progress=None)
        融合是指将融合字段值相同的对象合并为一个简单对象或复杂对象。适用于线对象和面对象。子对象是构成简单对象和复杂对象的基本对象。简单对象由一个子对象组成,
        即简单对象本身;复杂对象由两个或两个以上相同类型的子对象组成。
        
        :param input\_data: 待融合的矢量数据集。必须为线数据集或面数据集。
        :type input\_data: DatasetVector or str
        :param dissolve\_type: 融合类型
        :type dissolve\_type: DissolveType or str
        :param dissolve\_fields: 融合字段,融合字段的字段值相同的记录才会融合。当 dissolve\_fields 为 str 时,支持设置 ',' 分隔多个字段,例如 "field1,field2,field3"
        :type dissolve\_fields: list[str] or str
        :param field\_stats:  统计字段名称和对应的统计类型。stats\_fields 为 list,list中每个元素为一个tuple,tuple的第一个元素为被统计的字段,第二个元素为统计类型。
                             当 stats\_fields 为 str 时,支持设置 ',' 分隔多个字段,例如 "field1:SUM, field2:MAX, field3:MIN"
        :type field\_stats: list[tuple[str,StatisticsType]] or list[tuple[str,str]] or str
        :param str attr\_filter: 数据集融合时对象的过滤表达式
        :param float tolerance: 融合容限
        :param bool is\_null\_value\_able: 是否处理融合字段值为空的对象
        :param bool is\_preprocess: 是否进行拓扑预处理
        :param out\_data: 结果数据保存的数据源。如果为空,则结果数据集保存到输入数据集所在的数据源。
        :type out\_data: Datasource or DatasourceConnectionInfo or str
        :param str out\_dataset\_name: 结果数据集名称
        :param function progress: 进度信息处理函数,具体参考 :py:class:`.StepEvent`
        :return: 结果数据集或数据集名称
        :rtype: DatasetVector or str
        
        
        >>> result = dissolve('E:/data.udb/zones', 'SINGLE', 'SmUserID', 'Area:SUM', tolerance=0.000001, out\_data='E:/dissolve\_out.udb')
    
    divide\_math\_analyst(first\_operand, second\_operand, user\_region=None, out\_data=None, out\_dataset\_name=None, progress=None)
        栅格除法运算。将输入的两个栅格数据集的栅格值逐个像元地相除。栅格代数运算的具体使用,参考 :py:meth:`expression\_math\_analyst`
        
        如果输入两个像素类型(PixelFormat)均为整数类型的栅格数据集,则输出整数类型的结果数据集;否则,输出浮点型的结果数据集。如果输入的两个栅格数据集
        的像素类型精度不同,则运算的结果数据集的像素类型与二者中精度较高者保持一致。
        
        :param first\_operand: 指定的第一栅格数据集。
        :type first\_operand: DatasetGrid or str
        :param second\_operand:  指定的第二栅格数据集。
        :type second\_operand: DatasetGrid or str
        :param GeoRegion user\_region: 用户指定的有效计算区域。如果为 None,则表示计算全部区域,如果参与运算的数据集范围不一致,将使用所有数据集的范围的交集作为计算区域。
        :param out\_data: 结果数据集所在的数据源
        :type out\_data: Datasource or DatasourceConnectionInfo or str
        :param str out\_dataset\_name: 结果数据集名称
        :param function progress: 进度信息处理函数,具体参考 :py:class:`.StepEvent`
        :return: 结果数据集或数据集名称
        :rtype: DatasetGrid or str
    
    dual\_line\_to\_center\_line(source\_line, max\_width, min\_width, out\_data=None, out\_dataset\_name=None, progress=None)
        根据给定的宽度从双线记录集或数据集中提取中心线。
        该功能一般用于提取双线道路或河流的中心线。双线要求连续且平行或基本平行,提取效果如下图。
        
        .. image:: ../image/DualLineToCenterLine.png
        
        注意:
        
         * 双线一般为双线道路或双线河流,可以是线数据,也可以是面数据。
         * max\_width 和 min\_width 参数用于指定记录集中双线的最大宽度和最小宽度,用于提取最小和最大宽度之间的双线的中心线。小于最小宽度、大于最大宽度部分的双线不提取中心线,且大于最大宽度的双线保留,小于最小宽度的双线丢弃。
         * 对于双线道路或双线河流中比较复杂的交叉口,如五叉六叉,或者双线的最大宽度和最小宽度相差较大的情形,提取的结果可能不理想。
        
        :param source\_line: 指定的双线记录集或数据集。要求为面类型的数据集或记录集。
        :type source\_line: DatasetVector or Recordset or str
        :param max\_width: 指定的双线的最大宽度。要求为大于 0 的值。单位与双线记录集所属的数据集相同。
        :type max\_width: float
        :param min\_width: 指定的双线的最小宽度。要求为大于或等于 0 的值。单位与双线记录集所属的数据集相同。
        :type min\_width: float
        :param out\_data: 指定的用于存储结果中心线数据集的数据源。
        :type out\_data: Datasource or DatasourceConnectionInfo or str
        :param out\_dataset\_name: 指定的结果中心线数据集的名称。
        :type out\_dataset\_name: str
        :param progress: 进度信息处理函数,具体参考 :py:class:`.StepEvent`
        :type progress: function
        :return: 结果数据集对象或结果数据集名称
        :rtype: DatasetVector or str
    
    edge\_match(source, target, edge\_match\_mode, tolerance=None, is\_union=False, edge\_match\_line=None, out\_data=None, out\_dataset\_name=None, progress=None)
        图幅接边,对两个二维线数据集进行自动接边。
        
        :param source: 接边源数据集。只能是二维线数据集。
        :type source: DatasetVector
        :param target: 接边目标数据。只能是二维线数据集,与接边源数据有相同的坐标系。
        :type target: DatasetVector
        :param edge\_match\_mode: 接边模式。
        :type edge\_match\_mode: EdgeMatchMode or str
        :param tolerance: 接边容限。单位与进行接边的数据集的单位相同。
        :type tolerance: float
        :param is\_union: 是否进行接边融合。
        :type is\_union: bool
        :param edge\_match\_line: 数据接边的接边线。在接边方式为交点位置接边 EdgeMatchMode.THE\_INTERSECTION 的时候用来计算交点,
                                不设置将按照数据集范围自动计算接边线来计算交点。
                                设置接边线后,发生接边关联的对象的端点将尽可能的靠到接边线上。
        :type edge\_match\_line: GeoLine
        :param out\_data: 接边关联数据所在的数据源。
        :type out\_data: Datasource or DatasourceConnectionInfo or str
        :param out\_dataset\_name: 接边关联数据的数据集名称。
        :type out\_dataset\_name: str
        :param progress: 进度信息处理函数,具体参考 :py:class:`.StepEvent`
        :type progress: function
        :return: 如果设置了接边关联数据集且接边成功,则返回接边关联数据集对象或数据集名称。如果没有设置接边关联数据集,将不会生成
                 接边关联数据集,则返回是否进行接边成功。
        :rtype: DatasetVector or str or bool
    
    eliminate(source, region\_tolerance, vertex\_tolerance, is\_delete\_single\_region=False, progress=None)
        碎多边形合并,即将数据集中小于指定面积的多边形合并到相邻的多边形中。目前仅支持将碎多边形合并到与其相邻的具有最大面积的多边形中。
        
        在数据制作和处理过程中,或对不精确的数据进行叠加后,都可能产生一些细碎而无用的多边形,称为碎多边形。可以通过“碎多边形合并”
        功能将这些细碎多边形合并到相邻的多边形中,或删除孤立的碎多边形(没有与其他多边形相交或者相切的多边形),以达到简化数据的目的。
        
        一般面积远远小于数据集中其他对象的多边形才被认为是“碎多边形”,通常是同一数据集中最大面积的百万分之一到万分之一间,但可以依
        据实际研究的需求来设置最小多边形容限。如下图所示的数据中,在较大的多边形的边界上,有很多无用的碎多边形。
        
        .. image:: ../image/Eliminate\_1.png
        
        下图是对该数据进行“碎多边形合并”处理后的结果,与上图对比可以看出,碎多边形都被合并到了相邻的较大的多边形中。
        
        .. image:: ../image/Eliminate\_2.png
        
        注意:
        
            * 该方法适用于两个面具有公共边界的情况,处理后会把公共边界去除。
            * 进行碎多边形合并处理后,数据集内的对象数量可能减少。
        
        
        :param source: 指定的待进行碎多边形合并的数据集。只支持矢量二维面数据集,指定其他类型的数据集会抛出异常。
        :type source: DatasetVector or str
        :param region\_tolerance: 指定的最小多边形容限。单位与系统计算的面积(SMAREA 字段)的单位一致。将 SMAREA 字段的值与该容限值对比,小于该值的多边形将被消除。取值范围为大于等于0,指定为小于0的值会抛出异常。
        :type region\_tolerance: float
        :param vertex\_tolerance: 指定的节点容限。单位与进行碎多边形合并的数据集单位相同。若两个节点之间的距离小于此容限值,则合并过程中会自动将这两个节点合并为一个节点。取值范围大于等于0,指定为小于0的值会抛出异常。
        :type vertex\_tolerance: float
        :param is\_delete\_single\_region: 指定是否删除孤立的小多边形。如果为 true 会删除孤立的小多边形,否则不删除。
        :type is\_delete\_single\_region: bool
        :param progress: 进度信息处理函数,具体参考 :py:class:`.StepEvent`
        :type progress: function
        :return: 整合成功返回 True,失败返回 False
        :rtype: bool
    
    expression\_math\_analyst(expression, pixel\_format, out\_data, is\_ingore\_no\_value=True, user\_region=None, out\_dataset\_name=None, progress=None)
        栅格代数运算类。用于提供对一个或多个栅格数据集的数学运算及函数运算。
        
        栅格代数运算的思想是运用代数学的观点对地理特征和现象进行空间分析。实质上,是对多个栅格数据集(DatasetGrid)进行数学运算以及函数运算。运算结果
        栅格的像元值是由输入的一个或多个栅格同一位置的像元的值通过代数规则运算得到的。
        
        栅格分析中很多功能都是基于栅格代数运算的,作为栅格分析的核心内容,栅格代数运算用途十分广泛,能够帮助我们解决各种类型的实际问题。如建筑工程中的计
        算填挖方量,将工程实施前的DEM栅格与实施后的DEM栅格相减,就能够从结果栅格中得到施工前后的高程差,将结果栅格的像元值与像元所代表的实际面积相乘,
        就可以得知工程的填方量与挖方量;又如,想要提取2000年全国范围内平均降雨量介于20毫米和50毫米的地区,可以通过“20<年平均降雨量<50”关系运算表达式,
        对年平均降雨量栅格数据进行运算而获得。
        
        通过该类的方法进行栅格代数运算主要有以下两种途径:
        
            - 使用该类提供的基础运算方法。该类提供了六个用于进行基础运算的方法,包括 plus(加法运算)、minus(减法运算)、multiply(乘法运算)、
              divide(除法运算)、to\_int(取整运算)和 to\_float(浮点运算)。使用这几个方法可以完成一个或多个栅格数据对应栅格值的算术运算。对于相
              对简单的运算,可以通过多次调用这几个方法来实现,如 (A/B)-(A/C)。
            - 执行运算表达式。使用表达式不仅可以对一个或多个栅格数据集实现运算符运算,还能够进行函数运算。运算符包括算术运算符、关系运算符和布尔运算符,
              算术运算主要包括加法(+)、减法(-)、乘法(*)、除法(/);布尔运算主要包括和(And)、或(Or)、异或(Xor)、非(Not);关系运算主要包括
              =、<、>、<>、>=、<=。注意,对于布尔运算和关系运算均有三种可能的输出结果:真=1、假=0及无值(只要有一个输入值为无值,结果即为无值)。
        
        此外,还支持 21 种常用的函数运算,如下图所示:
        
        .. image:: ../image/MathAnalyst\_Function.png
        
        
        执行栅格代数运算表达式,支持自定义表达式栅格运算,通过自定义表达式可以进行算术运算、条件运算、逻辑运算、函数运算(常用函数、三角函数)以及复合运算。
        栅格代数运算表达式的组成需要遵循以下规则:
        
            - 运算表达式应为一个形如下式的字符串:
        
                [DatasourceAlias1.Raster1] + [DatasourceAlias2.Raster2]
                使用“ [数据源别名.数据集名] ”来指定参加运算的栅格数据集;注意要使用方括号把名字括起来。
        
            - 栅格代数运算支持四则运算符("+" 、"-" 、"*" 、"/" )、条件运算符(">" 、">=" 、"<" 、"<=" 、"<>" 、"==" )、逻辑运算符("|" 、"\&" 、"Not()" 、"\^{}" )和一些常用数学函数("abs()" 、"acos()" 、"asin()" 、"atan()" 、"acot()" 、"cos()" 、"cosh()" 、"cot()" 、"exp()" 、"floor()" 、"mod(,)" 、"ln()" 、"log()" 、"pow(,)" 、"sin()" 、"sinh()" 、"sqrt()" 、"tan()" 、"tanh()" 、"Isnull()" 、"Con(,,)" 、"Pick(,,,..)" )。
            - 代数运算的表达式中各个函数之间可以嵌套使用,直接用条件运算符计算的栅格结果都为二值(如大于、小于等),即满足条件的用1代替,不满足的用0代替,若想使用其他值来表示满足条件和不满足条件的取值,可以使用条件提取函数Con(,,)。例如:"Con(IsNull([SURFACE\_ANALYST.Dem3] ) ,100,Con([SURFACE\_ANALYST.Dem3] > 100,[SURFACE\_ANALYST.Dem3] ,-9999) ) " ,该表达式的含义是:栅格数据集 Dem3 在别名为 SURFACE\_ANALYST 的数据源中,将其中无值栅格变为 100,剩余栅格中,大于100 的,值保持不变,小于等于 100 的,值改成 -9999。
            - 如果栅格计算中有小于零的负值,注意要加小括号,如:[DatasourceAlias1.Raster1] - ([DatasourceAlias2.Raster2])。
            - 表达式中,运算符连接的操作数可以是一个栅格数据集,也可以是数字或者数学函数。
            - 数学函数的自变量可以为一个数值,也可以为某个数据集,或者是一个数据集或多个数据集的运算表达式。
            - 表达式必须是没有回车的单行表达式。
            - 表达式中必须至少含有一个输入栅格数据集。
        
        注意:
        
            - 参与运算的两个数据集,如果其像素类型(PixelFormat)不同,则运算的结果数据集的像素类型与二者中精度较高者保持一致。例如,一个为32位整型,一个为单精度浮点型,那么进行加法运算后,结果数据集的像素类型将为单精度浮点型。
            - 对于栅格数据集中的无值数据,如果忽略无值,则无论何种运算,结果仍为无值;如果不忽略无值,意味着无值将参与运算。例如,两栅格数据集 A 和 B 相加,A 某单元格为无值,值为-9999,B 对应单元格值为3000,如果不忽略无值,则运算结果该单元格值为-6999。
        
        :param str expression:  自定义的栅格运算表达式。
        :param pixel\_format: 指定的结果数据集的像素格式。注意,如果指定的像素类型的精度低于参与运算的栅格数据集像素类型的精度,运算结果可能不正确。
        :type pixel\_format: PixelFormat or str
        :param out\_data: 结果数据集所在的数据源
        :type out\_data: Datasource or DatasourceConnectionInfo or str
        :param bool is\_ingore\_no\_value:  是否忽略无值栅格数据。true 表示忽略无值数据,即无值栅格不参与运算。
        :param GeoRegion user\_region: 用户指定的有效计算区域。如果为 None,则表示计算全部区域,如果参与运算的数据集范围不一致,将使用所有数据集
                                     的范围的交集作为计算区域。
        :param str out\_dataset\_name: 结果数据集名称
        :param function progress: 进度信息处理函数,具体参考 :py:class:`.StepEvent`
        :return: 结果数据集或数据集名称
        :rtype: DatasetGrid or str
    
    fill\_sink(surface\_grid, exclude\_area=None, out\_data=None, out\_dataset\_name=None, progress=None)
        对 DEM 栅格数据填充伪洼地。
        洼地是指周围栅格都比其高的区域,分为自然洼地和伪洼地。
        
         * 自然洼地,是实际存在的洼地,是地表真实形态的反映,如冰川或喀斯特地貌、采矿区、坑洞等,一般远少于伪洼地;
         * 伪洼地,主要是由数据处理造成的误差、不合适的插值方法导致,在 DEM 栅格数据中很常见。
        
        在确定流向时,由于洼地高程低于周围栅格的高程,一定区域内的流向都将指向洼地,导致水流在洼地聚集不能流出,引起汇水网络的中断,
        因此,填充洼地通常是进行合理流向计算的前提。
        
        在填充某处洼地后,有可能产生新的洼地,因此,填充洼地是一个不断重复识别洼地、填充洼地的过程,直至所有洼地被填充且不再产生新
        的洼地。下图为填充洼地的剖面示意图。
        
        .. image:: ../image/FillSink.png
        
        该方法可以指定一个点或面数据集,用于指示的真实洼地或需排除的洼地,这些洼地不会被填充。使用准确的该类数据,将获得更为真实的
        无伪洼地地形,使后续分析更为可靠。
        
        用于指示洼地的数据,如果是点数据集,其中的一个或多个点位于洼地内即可,最理想的情形是点指示该洼地区域的汇水点;如果是面数据
        集,每个面对象应覆盖一个洼地区域。
        
        可以通过 exclude\_area 参数,指定一个点或面数据集,用于指示的真实洼地或需排除的洼地,这些洼地不会被填充。使用准确的该类数据,
        将获得更为真实的无伪洼地地形,使后续分析更为可靠。用于指示洼地的数据,如果是点数据集,其中的一个或多个点位于洼地内即可,最
        理想的情形是点指示该洼地区域的汇水点;如果是面数据集,每个面对象应覆盖一个洼地区域。
        
        如果 exclude\_area 为 None,则会将 DEM 栅格中所有洼地填充,包括伪洼地和真实洼地
        
        
        水文分析的相关介绍,请参考 :py:func:`basin`
        
        :param surface\_grid: 指定的要进行填充洼地的 DEM 数据
        :type surface\_grid: DatasetGrid or str
        :param exclude\_area: 指定的用于指示已知自然洼地或要排除的洼地的点或面数据。如果是点数据集,一个或多个点所在的区域指示为洼地;
                             如果是面数据集,每个面对象对应一个洼地区域。如果为 None,则会将 DEM 栅格中所有洼地填充,包括伪洼地和真实洼地
        :type exclude\_area: DatasetVector or str
        :param out\_data: 用于存储结果数据集的数据源
        :type out\_data: DatasourceConnectionInfo or Datasource or str
        :param str out\_dataset\_name: 结果数据集的名称。
        :param progress: 进度信息处理函数,具体参考 :py:class:`.StepEvent`
        :type progress: function
        :return: 无伪洼地的 DEM 栅格数据集或数据集名称。如果填充伪洼地失败,则返回 None。
        :rtype: DatasetVector or str
    
    flow\_accumulation(direction\_grid, weight\_grid=None, out\_data=None, out\_dataset\_name=None, progress=None)
        根据流向栅格计算累积汇水量。可应用权重数据集计算加权累积汇水量。
        累积汇水量是指流向某个单元格的所有上游单元格的水流累积量,是基于流向数据计算得出的。
        
        累积汇水量的值可以帮助我们识别河谷和分水岭。单元格的累积汇水量较高,说明该地地势较低,可视为河谷;为0说明该地地势较高,可能为分水岭。因此,累积汇水量是提取流域的各种特征参数(如流域面积、周长、排水密度等)的基础。
        
        计算累积汇水量的基本思路是:假定栅格数据中的每个单元格处有一个单位的水量,依据水流方向图顺次计算每个单元格所能累积到的水量(不包括当前单元格的水量)。
        
        下图显示了由水流方向计算累积汇水量的过程。
        
        .. image:: ../image/FlowAccumulation\_1.png
        
        下图为流向栅格和基于其生成的累积汇水量栅格。
        
        .. image:: ../image/FlowAccumulation\_2.png
        
        在实际应用中,每个单元格的水量不一定相同,往往需要指定权重数据来获取符合需求的累积汇水量。使用了权重数据后,累积汇水量的计算过程中,每个单元格的水量不再是一个单位,而是乘以权重(权重数据集的栅格值)后的值。例如,将某时期的平均降雨量作为权重数据,计算所得的累积汇水量就是该时期的流经每个单元格的雨量。
        
        注意,权重栅格必须与流向栅格具有相同的范围和分辨率。
        
        
        水文分析的相关介绍,请参考 :py:func:`basin`
        
        
        :param direction\_grid: 流向栅格数据。
        :type direction\_grid: DatasetGrid or str
        :param weight\_grid: 权重栅格数据。设置为 None 表示不使用权重数据集。
        :type weight\_grid: DatasetGrid or str
        :param out\_data: 用于存储结果数据集的数据源
        :type out\_data: DatasourceConnectionInfo or Datasource or str
        :param str out\_dataset\_name: 结果数据集的名称。
        :param progress: 进度信息处理函数,具体参考 :py:class:`.StepEvent`
        :type progress: function
        :return: 累积汇水量栅格数据集或数据集名称。如果计算失败,则返回 None。
        :rtype: DatasetVector or str
    
    flow\_direction(surface\_grid, force\_flow\_at\_edge, out\_data=None, out\_dataset\_name=None, out\_drop\_grid\_name=None, progress=None)
        对 DEM 栅格数据计算流向。为保证流向计算的正确性,建议使用填充伪洼地之后的 DEM 栅格数据。
        
        流向,即水文表面水流的方向。计算流向是水文分析的关键步骤之一。水文分析的很多功能需要基于流向栅格,如计算累积汇水量、计算流
        长和流域等。
        
        SuperMap 使用最大坡降法(D8,Deterministic Eight-node)计算流向。这种方法通过计算单元格的最陡下降方向作为水流的方向。中心
        单元格与相邻单元格的高程差与距离的比值称为高程梯度。最陡下降方向即为中心单元格与高程梯度最大的单元格所构成的方向,也就是中
        心栅格的流向。单元格的流向的值,是通过对其周围的8个邻域栅格进行编码来确定的。如下图所示,若中心单元格的水流方向是左边,则其
        水流方向被赋值16;若流向右边,则赋值1。
        
        在 SuperMap 中,通过对中心栅格的 8 个邻域栅格编码(如下图所示),中心栅格的水流方向便可由其中的某一值来确定。例如,若中心
        栅格的水流方向是左边,则其水流方向被赋值 16;若流向右边,则赋值 1。
        
        .. image:: ../image/FlowDirection\_1.png
        
        计算流向时,需要注意栅格边界单元格的处理。位于栅格边界的单元格比较特殊,通过 forceFlowAtEdge 参数可以指定其流向是否向外,
        如果向外,则边界栅格的流向值如下图(左)所示,否则,位于边界上的单元格将赋为无值,如下图(右)所示。
        
        .. image:: ../image/FlowDirection\_2.png
        
        计算 DEM 数据每个栅格的流向得到流向栅格。下图显示了基于无洼地的 DEM 数据生成的流向栅格。
        
        .. image:: ../image/FlowDirection\_3.png
        
        
        水文分析的相关介绍,请参考 :py:func:`basin`
        
        
        :param surface\_grid: 用于计算流向的 DEM 数据
        :type surface\_grid: DatasetGrid or str
        :param bool force\_flow\_at\_edge: 指定是否强制边界的栅格流向为向外。如果为 True,则 DEM 栅格边缘处的所有单元的流向都是从栅格向外流动。
        :param out\_data: 用于存储结果数据集的数据源
        :type out\_data: DatasourceConnectionInfo or Datasource or str
        :param str out\_dataset\_name: 结果流向数据集的名称
        :param str out\_drop\_grid\_name: 结果高程梯度栅格数据集名称。可选参数。用于计算流向的中间结果。中心单元格与相邻单元格的高程差与距离的比值称
                                       为高程梯度。如下图所示,为流向计算的一个实例,该实例中生成了高程梯度栅格
        
                                       .. image:: ../image/FlowDirection.png
        
        :param progress: 进度信息处理函数,具体参考 :py:class:`.StepEvent`
        :type progress: function
        :return: 返回一个2个元素的tuple,第一个元素为 结果流向栅格数据集或数据集名称,如果设置了结果高程梯度栅格数据集名称,
                 则第二个元素为结果高程梯度栅格数据集或数据集名称,否则为 None
        :rtype: tuple[DatasetGrid,DatasetGrid] or tuple[str,str]
    
    flow\_length(direction\_grid, up\_stream, weight\_grid=None, out\_data=None, out\_dataset\_name=None, progress=None)
        根据流向栅格计算流长,即计算每个单元格沿着流向到其流向起始点或终止点之间的距离。可应用权重数据集计算加权流长。
        
        流长,是指每个单元格沿着流向到其流向起始点或终止点之间的距离,包括上游方向和下游方向的长度。水流长度直接影响地面径流的速度,
        进而影响地面土壤的侵蚀力,因此在水土保持方面具有重要意义,常作为土壤侵蚀、水土流失情况的评价因素。
        
        流长的计算基于流向数据,流向数据表明水流的方向,该数据集可由流向分析创建;权重数据定义了每个单元格的水流阻力。流长一般用于
        洪水的计算,水流往往会受到诸如坡度、土壤饱和度、植被覆盖等许多因素的阻碍,此时对这些因素建模,需要提供权重数据集。
        
        流长有两种计算方式:
        
         * 顺流而下:计算每个单元格沿流向到下游流域汇水点之间的最长距离。
         * 溯流而上:计算每个单元格沿流向到上游分水线顶点的最长距离。
        
        下图分别为以顺流而下和溯流而上计算得出的流长栅格:
        
        .. image:: ../image/FlowLength.png
        
        权重数据定义了每个栅格单元间的水流阻力,应用权重所获得的流长为加权距离(即距离乘以对应权重栅格的值)。例如,将流长分析应用
        于洪水的计算,洪水流往往会受到诸如坡度、土壤饱和度、植被覆盖等许多因素的阻碍,此时对这些因素建模,需要提供权重数据集。
        
        注意,权重栅格必须与流向栅格具有相同的范围和分辨率。
        
        水文分析的相关介绍,请参考 :py:func:`basin`
        
        
        :param direction\_grid: 指定的流向栅格数据。
        :type direction\_grid: DatasetGrid or str
        :param bool up\_stream: 指定顺流而下计算还是溯流而上计算。True 表示溯流而上,False 表示顺流而下。
        :param weight\_grid:  指定的权重栅格数据。设置为 None 表示不使用权重数据集。
        :type weight\_grid: DatasetGrid or str
        :param out\_data: 用于存储结果数据集的数据源
        :type out\_data: DatasourceConnectionInfo or Datasource or str
        :param str out\_dataset\_name: 结果流长栅格数据集的名称
        :param progress: 进度信息处理函数,具体参考 :py:class:`.StepEvent`
        :type progress: function
        :return: 结果流长栅格数据集或数据集名称
        :rtype: DatasetGrid or  str
    
    grid\_basic\_statistics(grid\_data, function\_type=None, progress=None)
        栅格基本统计分析,可指定变换函数类型。用于对栅格数据集进行基本的统计分析,包括最大值、最小值、平均值和标准差等。
        
        指定变换函数时,用来统计的数据是原始栅格值经过函数变换后得到的值。
        
        :param grid\_data: 待统计的栅格数据
        :type grid\_data: DatasetGrid or str
        :param function\_type: 变换函数类型
        :type function\_type: FunctionType or str
        :param progress: function
        :type progress: 进度信息处理函数,具体参考 :py:class:`.StepEvent`
        :return: 基本统计分析结果
        :rtype: BasicStatisticsAnalystResult
    
    grid\_common\_statistics(grid\_data, compare\_datasets\_or\_value, compare\_type, is\_ignore\_no\_value, out\_data=None, out\_dataset\_name=None, progress=None)
        栅格常用统计分析,将一个栅格数据集逐行逐列按照某种比较方式与一个(或多个)栅格数据集,或一个固定值进行比较,比较结果为“真”的像元值为 1,为“假”的像元值为 0。
        
        关于无值的说明:
        
         * 当待统计源数据集的栅格有无值时,如果忽略无值,则统计结果栅格也为无值,否则使用该无值参与统计;当各比较数据集的栅格有无值时,
           如果忽略无值,则此次统计(待统计栅格与该比较数据集的计算)不计入结果,否则使用该无值进行比较。
         * 当无值不参与运算(即忽略无值)时,统计结果数据集中无值的值,由结果栅格的像素格式决定,为最大像元值,例如,结果栅格数据集像素
           格式为 PixelFormat.UBIT8,即每个像元使用 8 个比特表示,则无值的值为 255。在此方法中,结果栅格的像素格式是由比较栅格数据集
           的数量来决定的。比较数据集得个数、结果栅格的像素格式和结果栅格中无值的值三者的对应关系如下所示:
        
        .. image:: ../image/CommonStatistics.png
        
        :param grid\_data:  指定的待统计的栅格数据。
        :type grid\_data: DatasetGrid or str
        :param compare\_datasets\_or\_value: 指定的比较的数据集集合或固定值。指定固定值时,固定值的单位与待统计的栅格数据集的栅格值单位相同。
        :type compare\_datasets\_or\_value: list[DatasetGrid] or list[str] or float
        :param compare\_type: 指定的比较类型
        :type compare\_type: StatisticsCompareType or str
        :param is\_ignore\_no\_value: 指定是否忽略无值。如果为 true,即忽略无值,则计算区域内的无值不参与计算,结果栅格值仍为无值;若为 false,则计算区域内的无值参与计算。
        :type is\_ignore\_no\_value: bool
        :param out\_data: 用于存储结果数据的数据源。
        :type out\_data: Datasource or DatasourceConnectionInfo or str
        :param out\_dataset\_name: 结果数据集的名称
        :type out\_dataset\_name: str
        :param progress: 进度信息处理函数,具体参考 :py:class:`.StepEvent`
        :type progress: function
        :return: 统计结果栅格数据集或数据集名称
        :rtype: DatasetGrid or str
    
    grid\_extract\_isoline(extracted\_grid, interval, datum\_value=0.0, expected\_z\_values=None, resample\_tolerance=0.0, smooth\_method='BSPLINE', smoothness=0, clip\_region=None, out\_data=None, out\_dataset\_name=None, progress=None)
        用于从栅格数据集中提取等值线,并将结果保存为数据集。
        
        等值线是由一系列具有相同值的点连接而成的光滑曲线或折线,如等高线、等温线。等值线的分布反映了栅格表面上值的变化,等值线分布越密集的地方, 表示栅格表面值的变化比较剧烈,例如,如果为等高线,则越密集,坡度越陡峭,反之坡度越平缓。通过提取等值线,可以找到高程、温度、降水等的值相同的位置, 同时等值线的分布状况也可以显示出变化的陡峭和平缓区。
        
        如下所示,上图为某个区域的 DEM 栅格数据,下图是从上图中提取的等高线。DEM 栅格数据的高程信息是存储在每一个栅格单元中的,栅格是有大小的,栅格的大小取决于栅格数据的分辨率 ,即每一个栅格单元代表实际地面上的相应地块的大小,因此,栅格数据不能很精确的反应每一位置上的高程信息 ,而矢量数据在这方面相对具有很大的优势,因此,从栅格数据中提取等高线 ,把栅格数据变成矢量数据,就可以突出显示数据的细节部分,便于分析,例如,从等高线数据中可以明显的区分地势的陡峭与舒缓的部位,可以区分出山脊山谷
        
        .. image:: ../image/SurfaceAnalyst\_1.png
        
        .. image:: ../image/SurfaceAnalyst\_2.png
        
        SuperMap 提供两种方法来提取等值线:
        
            * 通过设置基准值(datum\_value)和等值距(interval)来提取等间距的等值线。该方法是以等值距为间隔向基准值的前后两个方向
              计算提取哪些高程的等值线。例如,高程范围为15-165的 DEM 栅格数据,设置基准值为50,等值距为20,则提取等值线的高程分别
              为:30、50、70、90、110、130和150。
            * 通过 expected\_z\_values 方法指定一个 Z 值的集合,则只提取高程为集合中值的等值线/面。例如,高程范围为0-1000的 DEM 栅
              格数据,指定 Z 值集合为[20,300,800],那么提取的结果就只有 20、300、800 三条等值线或三者构成的等值面。
        
        注意:
            * 如果同时调用了上面两种方法所需设置的属性,那么只有 expected\_z\_values 方法有效,即只提取指定的值的等值线。因此,想要
              提取等间距的等值线,就不能调用 expected\_z\_values 方法。
        
        
        :param extracted\_grid: 指定的提取操作需要的参数。
        :type extracted\_grid: DatasetGrid or str
        :param  float interval:  等值距,等值距是两条等值线之间的间隔值,必须大于0.
        :param datum\_value: 设置等值线的基准值。基准值与等值距(interval)共同决定提取哪些高程上的等值线。基准值作为一个生成等值
                            线的初始起算值,以等值距为间隔向其前后两个方向计算,因此并不一定是最小等值线的值。例如,高程范围为
                            220-1550 的 DEM 栅格数据,如果设基准值为 500,等值距为 50,则提取等值线的结果是:最小等值线值为 250,
                            最大等值线值为 1550。
        
                            当同时设置 expected\_z\_values 时,只会考虑 expected\_z\_values 设置的值,即只提取高程为这些值的等值线。
        :type datum\_value: float
        
        :param expected\_z\_values: 期望分析结果的 Z 值集合。Z 值集合存储一系列数值,该数值为待提取等值线的值。即,仅高程值在Z值集
                                  合中的等值线会被提取。
                                  当同时设置 datum\_value 时,只会考虑 expected\_z\_values 设置的值,即只提取高程为这些值的等值线。
        :type expected\_z\_values: list[float] or str
        :param resample\_tolerance: 重采样的距离容限系数。通过对提取出的等值线行重采样,可以简化最终提取的等值线数据。SuperMap 在
                                   提取等值线/面时使用的重采样方法为光栏法(VectorResampleType.RTBEND),该方法需要一个重采样
                                   距离容限进行采样控制。它的值由重采样的距离容限系数乘以源栅格分辨率得出,一般取值为源栅格分辨率
                                   的 0~1 倍。
        
                                   重采样的距离容限系数默认为 0,即不进行任何采样,保证结果正确,但通过设置合理的参数,可以加快执
                                   行速度。容限值越大,等值线边界的控制点越少,此时可能出现等值线相交的情况。因此,推荐用户先使
                                   用默认值来提取等值线。
        :type resample\_tolerance: float
        :param smooth\_method: 滑处理所使用的方法
        :type smooth\_method: SmoothMethod or str
        :param smoothness: 设置等值线或等值面的光滑度。 光滑度为 0 或 1表示不进行光滑处理,值越大则光滑度越高。等值线提取时,光滑度可自由设置
        :type smoothness: int
        :param clip\_region: 指定的裁剪面对象。如果不需要对操作结果进行裁剪,可以使用 None 值取代该参数。
        :type clip\_region: GeoRegion
        :param out\_data: 用于存放结果数据集的数据源。如果为空,则会直接返回等值线对象的列表。
        :type out\_data: Datasource or DatasourceConnectionInfo or str
        :param out\_dataset\_name:  指定的提取结果数据集的名称。
        :type out\_dataset\_name: str
        :param progress: function
        :type progress: 进度信息处理函数,具体参考 :py:class:`.StepEvent`
        :return: 提取等值线得到的数据集或数据集名称,或等值线对象列表。
        :rtype: DatasetVector or str or list[GeoLine]
    
    grid\_extract\_isoregion(extracted\_grid, interval, datum\_value=0.0, expected\_z\_values=None, resample\_tolerance=0.0, smooth\_method='BSPLINE', smoothness=0, clip\_region=None, out\_data=None, out\_dataset\_name=None, progress=None)
        用于从栅格数据集中提取等值面。
        
        SuperMap 提供两种方法来提取等值面:
        
        * 通过设置基准值(datum\_value)和等值距(interval)来提取等间距的等值面。该方法是以等值距为间隔向基准值的前后两个方向计算
          提取哪些高程的等值线。例如,高程范围为15-165的 DEM 栅格数据,设置基准值为50,等值距为20,则提取等值线的高程分别为:
          30、50、70、90、110、130和150。
        * 通过 expected\_z\_values 方法指定一个 Z 值的集合,则只提取高程为集合中值的等值面。例如,高程范围为0-1000的 DEM 栅格数据,
          指定 Z 值集合为[20,300,800],那么提取的结果就只有20、300、800三者构成的等值面。
        
        注意:
        
         * 如果同时调用了上面两种方法所需设置的属性,那么只有 setExpectedZValues 方法有效,即只提取指定的值的等值面。
           因此,想要提取等间距的等值面,就不能调用 expected\_z\_values 方法。
        
        :param extracted\_grid: DatasetGrid or str
        :type extracted\_grid:  指定的待提取的栅格数据集。
        :param float interval:  等值距,等值距是两条等值线之间的间隔值,必须大于0
        :param datum\_value: 设置等值线的基准值。基准值与等值距(interval)共同决定提取哪些高程上的等值面。基准值作为一个生成等值
                            线的初始起算值,以等值距为间隔向其前后两个方向计算,因此并不一定是最小等值面的值。例如,高程范围为
                            220-1550 的 DEM 栅格数据,如果设基准值为 500,等值距为 50,则提取等值线的结果是:最小等值线值为 250,
                            最大等值线值为 1550。
        
                            当同时设置 expected\_z\_values 时,只会考虑 expected\_z\_values 设置的值,即只提取高程为这些值的等值线。
        :type datum\_value: float
        :param expected\_z\_values: 期望分析结果的 Z 值集合。Z 值集合存储一系列数值,该数值为待提取等值线的值。即,仅高程值在Z值集
                                  合中的等值线会被提取。
                                  当同时设置 datum\_value 时,只会考虑 expected\_z\_values 设置的值,即只提取高程为这些值的等值线。
        :type expected\_z\_values: list[float] or str
        :param resample\_tolerance: 重采样的距离容限系数。通过对提取出的等值线行重采样,可以简化最终提取的等值线数据。SuperMap 在
                                   提取等值线/面时使用的重采样方法为光栏法(VectorResampleType.RTBEND),该方法需要一个重采样
                                   距离容限进行采样控制。它的值由重采样的距离容限系数乘以源栅格分辨率得出,一般取值为源栅格分辨率
                                   的 0~1 倍。
                                   重采样的距离容限系数默认为 0,即不进行任何采样,保证结果正确,但通过设置合理的参数,可以加快执
                                   行速度。容限值越大,等值线边界的控制点越少,此时可能出现等值线相交的情况。因此,推荐用户先使
                                   用默认值来提取等值线。
        :type resample\_tolerance: float
        :param smooth\_method: 滑处理所使用的方法
        :type smooth\_method: SmoothMethod or str
        :param smoothness: 设置等值面的光滑度。 光滑度为 0 或 1表示不进行光滑处理,值越大则光滑度越高。
                           对于等值面的提取,采用先提取等值线然后生成等值面的方式,若将光滑度设置为2,
                           则中间结果数据集,即等值线对象的点数将为原始数据集点数的2倍,当光滑度设定值不断增大时,点数将成2的指数倍
                           增长,这将大大降低等值面提取的效率甚至可能导致提取失败。
        :type smoothness: int
        :param clip\_region: 指定的裁剪面对象。
        :type clip\_region: GeoRegion
        :param out\_data: 用于存放结果数据集的数据源。如果为空,则直接返回等值面对象列表
        :type out\_data: Datasource or DatasourceConnectionInfo or str
        :param out\_dataset\_name:  指定的提取结果数据集的名称。
        :type out\_dataset\_name: str
        :param progress: function
        :type progress: 进度信息处理函数,具体参考 :py:class:`.StepEvent`
        :return: 提取等值面得到的数据集或数据集名称,或等值面对象列表
        :rtype: DatasetVector or str or list[GeoRegion]
    
    grid\_neighbour\_statistics(grid\_data, neighbour\_shape, is\_ignore\_no\_value=True, grid\_stat\_mode='SUM', unit\_type='CELL', out\_data=None, out\_dataset\_name=None, progress=None)
        栅格邻域统计分析。
        
        邻域统计分析,是对输入数据集中的每个像元的指定扩展区域中的像元进行统计,将运算结果作为像元的值。统计的方法包括:总和、
        最大值、最小值、众数、少数、中位数等,请参见 GridStatisticsMode 枚举类型。目前提供的邻域范围类型(请参见 NeighbourShapeType
        枚举类型)有:矩形、圆形、圆环和扇形。
        
        下图为邻域统计的原理示意,假设使用“总和”作为统计方法做矩形邻域统计,邻域大小为 3×3,那么对于图中位于第二行第三列的单元格,
        它的值则由以其为中心向周围扩散得到的一个 3×3 的矩形内所有像元值的和来决定。
        
        
        .. image:: ../image/NeighbourStatistics.png
        
        邻域统计的应用十分广泛。例如:
        
        * 对表示物种种类分布的栅格计算每个邻域内的生物种类(统计方法:种类),从而观察该地区的物种丰度;
        * 对坡度栅格统计邻域内的坡度差(统计方法:值域),从而评估该区域的地形起伏状况;
        
          .. image:: ../image/NeighbourStatistics\_1.png
        
        * 邻域统计还用于图像处理,如统计邻域内的平均值(称为均值滤波)或中位数(称为中值滤波)可以达到平滑的效果,从而去除噪声或过多的细节,等等。
        
          .. image:: ../image/NeighbourStatistics\_2.png
        
        
        :param grid\_data: 指定的待统计的栅格数据。
        :type grid\_data: DatasetGrid or str
        :param neighbour\_shape: 邻域形状
        :type neighbour\_shape: NeighbourShape
        :param is\_ignore\_no\_value: 指定是否忽略无值。如果为 true,即忽略无值,则计算区域内的无值不参与计算,结果栅格值仍为无值;若为 false,则计算区域内的无值参与计算。
        :type is\_ignore\_no\_value: bool
        :param grid\_stat\_mode: 邻域分析的统计方法
        :type grid\_stat\_mode: GridStatisticsMode or str
        :param unit\_type: 邻域统计的单位类型
        :type unit\_type: NeighbourUnitType or str
        :param out\_data: 用于存储结果数据的数据源。
        :type out\_data: Datasource or DatasourceConnectionInfo or str
        :param out\_dataset\_name: 结果数据集的名称
        :type out\_dataset\_name: str
        :param progress: 进度信息处理函数,具体参考 :py:class:`.StepEvent`
        :type progress: function
        :return: 统计结果栅格数据集或数据集名称
        :rtype: DatasetGrid or str
    
    hierarchical\_density\_based\_clustering(input\_data, min\_pile\_point\_count, out\_data=None, out\_dataset\_name=None, progress=None)
        密度聚类的HDBSCAN实现
        
        该方法是对DBSCAN方法的改进,只需给定空间邻域范围内的最少点数(min\_pile\_point\_count)。在DBSCAN的基础上,计算不同的搜索半径选择最稳定的空间聚类分布作为密度聚类结果。
        
        :param input\_data: 指定的要聚类的矢量数据集,支持点数据集。
        :type input\_data: DatasetVector or str
        :param min\_pile\_point\_count: 每类包含的最少点数
        :type min\_pile\_point\_count: int
        :param out\_data: 结果数据集所在的数据源
        :type out\_data: Datasource or DatasourceConnectionInfo or str
        :param str out\_dataset\_name: 结果数据集名称
        :param function progress: 进度信息处理函数,具体参考 :py:class:`.StepEvent`
        :return: 结果数据集或数据集名称
        :rtype: DatasetVector or str
    
    high\_or\_low\_clustering(source, assessment\_field, concept\_model='INVERSEDISTANCE', distance\_method='EUCLIDEAN', distance\_tolerance=-1.0, exponent=1.0, k\_neighbors=1, is\_standardization=False, weight\_file\_path=None, progress=None)
        对矢量数据集进行高低值聚类分析,并返回高低值聚类分析结果。 高低值聚类返回的结果包括GeneralG指数、期望、方差、z得分、P值,
        请参阅 :py:class:`.AnalyzingPatternsResult` 类。
        
        .. image:: ../image/AnalyzingPatterns\_highOrLowClustering.png
        
        关于分析模式介绍,请参考 :py:func:`auto\_correlation`
        
        
        :param source: 待计算的数据集。可以为点、线、面数据集。
        :type source: DatasetVector or str
        :param str assessment\_field: 评估字段的名称。仅数值字段有效。
        :param concept\_model: 空间关系概念化模型。默认值 :py:attr:`.ConceptualizationModel.INVERSEDISTANCE`。
        :type concept\_model: ConceptualizationModel or str
        :param distance\_method: 距离计算方法类型
        :type distance\_method: DistanceMethod or str
        :param float distance\_tolerance: 中断距离容限。仅对概念化模型设置为 :py:attr:`.ConceptualizationModel.INVERSEDISTANCE` 、
                                         :py:attr:`.ConceptualizationModel.INVERSEDISTANCESQUARED` 、
                                         :py:attr:`.ConceptualizationModel.FIXEDDISTANCEBAND` 、
                                         :py:attr:`.ConceptualizationModel.ZONEOFINDIFFERENCE` 时有效。
        
                                         为"反距离"和"固定距离"模型指定中断距离。"-1"表示计算并应用默认距离,此默认值为保证每个要
                                         素至少有一个相邻的要素;"0"表示为未应用任何距离,则每个要素都是相邻要素。
        
        :param float exponent: 反距离幂指数。仅对概念化模型设置为 :py:attr:`.ConceptualizationModel.INVERSEDISTANCE` 、
                                         :py:attr:`.ConceptualizationModel.INVERSEDISTANCESQUARED` 、
                                         :py:attr:`.ConceptualizationModel.ZONEOFINDIFFERENCE` 时有效。
        :param int k\_neighbors:  相邻数目,目标要素周围最近的K个要素为相邻要素。仅对概念化模型设置为 :py:attr:`.ConceptualizationModel.KNEARESTNEIGHBORS` 时有效。
        :param bool is\_standardization: 是否对空间权重矩阵进行标准化。若进行标准化,则每个权重都会除以该行的和。
        :param str weight\_file\_path: 空间权重矩阵文件路径
        :param progress: 进度信息处理函数,具体参考 :py:class:`.StepEvent`
        :type progress: function
        :return: 高低值聚类结果
        :rtype: AnalyzingPatternsResult
    
    hot\_spot\_analyst(source, assessment\_field, concept\_model='INVERSEDISTANCE', distance\_method='EUCLIDEAN', distance\_tolerance=-1.0, exponent=1.0, is\_FDR\_adjusted=False, k\_neighbors=1, is\_standardization=False, self\_weight\_field=None, weight\_file\_path=None, out\_data=None, out\_dataset\_name=None, progress=None)
        热点分析,返回结果矢量数据集。
        
         * 结果数据集中包括z得分(Gi\_Zscore)、P值(Gi\_Pvalue)和置信区间(Gi\_ConfInvl)。
        
         * z得分和P值都是统计显著性的度量,用于逐要素的判断是否拒绝"零假设"。置信区间字段会识别具有统计显著性的热点和冷点。置信区间
           为+3和-3的要素反映置信度为99\%的统计显著性,置信区间为+2和-2的要素反映置信度为95\%的统计显著性,置信区间为+1和-1的要素反映置
           信度为90\%的统计显著性,而置信区间为0的要素的聚类则没有统计意义。
        
         * 在没有设置 is\_FDR\_adjusted 方法的情况下,统计显著性以P值和Z字段为基础,否则,确定置信度的关键P值会降低以兼顾多重测试和空间依赖性。
        
        .. image:: ../image/ClusteringDistributions\_hotSpotAnalyst.png
        
        
        关于聚类分布介绍,参考 :py:func:`cluster\_outlier\_analyst`
        
        :param source: 待计算的数据集。可以为点、线、面数据集。
        :type source: DatasetVector or str
        :param str assessment\_field: 评估字段的名称。仅数值字段有效。
        :param concept\_model: 空间关系概念化模型。默认值 :py:attr:`.ConceptualizationModel.INVERSEDISTANCE`。
        :type concept\_model: ConceptualizationModel or str
        :param distance\_method: 距离计算方法类型
        :type distance\_method: DistanceMethod or str
        :param float distance\_tolerance: 中断距离容限。仅对概念化模型设置为 :py:attr:`.ConceptualizationModel.INVERSEDISTANCE` 、
                                         :py:attr:`.ConceptualizationModel.INVERSEDISTANCESQUARED` 、
                                         :py:attr:`.ConceptualizationModel.FIXEDDISTANCEBAND` 、
                                         :py:attr:`.ConceptualizationModel.ZONEOFINDIFFERENCE` 时有效。
        
                                         为"反距离"和"固定距离"模型指定中断距离。"-1"表示计算并应用默认距离,此默认值为保证每个要
                                         素至少有一个相邻的要素;"0"表示为未应用任何距离,则每个要素都是相邻要素。
        
        :param float exponent: 反距离幂指数。仅对概念化模型设置为 :py:attr:`.ConceptualizationModel.INVERSEDISTANCE` 、
                               :py:attr:`.ConceptualizationModel.INVERSEDISTANCESQUARED` 、
                               :py:attr:`.ConceptualizationModel.ZONEOFINDIFFERENCE` 时有效。
        :param bool is\_FDR\_adjusted: 是否进行 FDR(错误发现率)校正。若进行FDR(错误发现率)校正,则统计显著性将以错误发现率校正为基础,否则,统计显著性将以P值和z得分字段为基础。
        :param int k\_neighbors:  相邻数目,目标要素周围最近的K个要素为相邻要素。仅对概念化模型设置为 :py:attr:`.ConceptualizationModel.KNEARESTNEIGHBORS` 时有效。
        :param bool is\_standardization: 是否对空间权重矩阵进行标准化。若进行标准化,则每个权重都会除以该行的和。
        :param str self\_weight\_field: 自身权重字段的名称,仅数值字段有效。
        :param str weight\_file\_path: 空间权重矩阵文件路径
        :param out\_data: 结果数据源
        :type out\_data: Datasource or DatasourceConnectionInfo or str
        :param str out\_dataset\_name: 结果数据集名称
        :param progress: 进度信息,具体参考 :py:class:`.StepEvent`
        :type progress: function
        :return: 结果数据集或数据集名称
        :rtype: DatasetVector or str
    
    idw\_interpolate(input\_data, z\_value\_field, pixel\_format, resolution, search\_mode=SearchMode.KDTREE\_FIXED\_COUNT, search\_radius=0.0, expected\_count=12, power=1, bounds=None, z\_value\_scale=1.0, out\_data=None, out\_dataset\_name=None, progress=None)
        使用 IDW 插值方法对点数据集或记录集进行插值。具体参考 :py:meth:`interpolate` 和 :py:class:`.InterpolationIDWParameter`
        
        :param input\_data:  需要进行插值分析的点数据集或点记录集
        :type input\_data: DatasetVector or str or Recordset
        :param str z\_value\_field: 存储用于进行插值分析的值的字段名称。插值分析不支持文本类型的字段。
        :param pixel\_format: 指定结果栅格数据集存储的像素,不支持 BIT64
        :type pixel\_format: PixelFormat or str
        :param float resolution: 插值运算时使用的分辨率
        :param search\_mode: 插值运算时,查找参与运算点的方式。不支持 QUADTREE
        :type search\_mode: SearchMode or str
        :param float search\_radius: 查找参与运算点的查找半径。单位与用于插值的点数据集(或记录集所属的数据集)的单位相同。查找半径决定了参与运算点的查找范围,当计算某个位置的未知数值时,会以该位置为圆心,以search\_radius为半径,落在这个范围内的采样点都将参与运算,即该位置的预测值由该范围内采样点的数值决定。
                                    如果设置 search\_mode 为KDTREE\_FIXED\_COUNT,同时指定查找参与运算点的范围,当查找范围内的点数小于指定的点数时赋为空值,当查找范围内的点数大于指定的点数时,则返回距离插值点最近的指定个数的点进行插值。
        :param int expected\_count: 期望参与插值运算的点数。如果设置 search\_mode 为 KDTREE\_FIXED\_RADIUS ,同时指定参与插值运算点的个数,当查找范围内的点数小于指定的点数时赋为空值。
        :param int power: 距离权重计算的幂次。幂次值越低,内插结果越平滑,幂次值越高,内插结果细节越详细。此参数应为一个大于0的值。如果不指定此参数,方法缺省将其设置为1。
        :param Rectangle bounds: 插值分析的范围,用于确定运行结果的范围
        :param float z\_value\_scale: 插值分析值的缩放比率
        :param out\_data: 结果数据集所在的数据源
        :type out\_data: Datasource or DatasourceConnectionInfo or str
        :param str out\_dataset\_name: 结果数据集名称
        :param function progress: 进度信息处理函数,具体参考 :py:class:`.StepEvent`
        :return: 结果数据集或数据集名称
        :rtype: DatasetGrid or str
    
    incremental\_auto\_correlation(source, assessment\_field, begin\_distance=0.0, distance\_method='EUCLIDEAN', incremental\_distance=0.0, incremental\_number=10, is\_standardization=False, progress=None)
        对矢量数据集进行增量空间自相关分析,并返回增量空间自相关分析结果数组。增量空间自相关返回的结果包括增量距离、莫兰指数、期望、方差、z得分、P值,
        请参阅 :py:class:`.IncrementalResult` 类。
        
        增量空间自相关会为一系列的增量距离运行空间自相关方法(参考 :py:func:`auto\_correlation` ),空间关系概念化模型默认为固定距离
        模型(参阅 :py:attr:`.ConceptualizationModel.FIXEDDISTANCEBAND` )
        
        关于分析模式介绍,请参考 :py:func:`auto\_correlation`
        
        :param source: 待计算的数据集。可以为点、线、面数据集。
        :type source: DatasetVector or str
        :param str assessment\_field: 评估字段的名称。仅数值字段有效。
        :param float begin\_distance: 增量空间自相关开始分析的起始距离。
        :param distance\_method: 距离计算方法类型
        :type distance\_method: DistanceMethod or str
        :param float incremental\_distance: 距离增量,增量空间自相关每次分析的间隔距离。
        :param int incremental\_number: 递增的距离段数目。为增量空间自相关指定分析数据集的次数,该值的范围为:2 \textasciitilde{} 30。
        :param bool is\_standardization: 是否对空间权重矩阵进行标准化。若进行标准化,则每个权重都会除以该行的和。
        :param progress: 进度信息,具体参考 :py:class:`.StepEvent`
        :type progress: function
        :return: 增量空间自相关分析结果列表。
        :rtype: list[IncrementalResult]
    
    integrate(source, tolerance, unit=None, progress=None)
        整合, 将容限范围内的节点捕捉在一起。节点容限较大会导致要素重叠或导致面和线对象被删除,还可能导致不被期望移动的节点发生移动。
        所以,选取容限值时应当根据实际情形设置合理的容限值。
        
        注意:整合功能将直接修改源数据集。
        
        :param source: 指定的待整合的数据集。可以为点、线、面数据集。
        :type source: DatasetVector or str
        :param tolerance: 指定的节点容限。
        :type tolerance: float
        :param unit:  指定的节点容限单位。
        :type unit: Unit or str
        :param progress:
        :type progress: function
        :return: 进度信息处理函数,具体参考 :py:class:`.StepEvent`
        :rtype: bool
    
    interpolate(input\_data, parameter, z\_value\_field, pixel\_format, z\_value\_scale=1.0, out\_data=None, out\_dataset\_name=None, progress=None)
        插值分析类。该类提供插值分析功能,用于对离散的点数据进行插值得到栅格数据集。插值分析可以将有限的采样点数据,通过插值对采样点周围的数值情况进行预测,
        从而掌握研究区域内数据的总体分布状况,而使采样的离散点不仅仅反映其所在位置的数值情况,而且可以反映区域的数值分布。
        
        为什么要进行插值?
        
        由于地理空间要素之间存在着空间关联性,即相互邻近的事物总是趋于同质,也就是具有相同或者相似的特征,举个例子,街道的一边下雨了,那么街道的另一边在大
        多数情况下也一定在下雨,如果在更大的区域范围,一个乡镇的气候应当与其接壤的另一的乡镇的气候相同,等等,基于这样的推理,我们就可以利用已知地点的信息
        来间接获取与其相邻的其他地点的信息,而插值分析就是基于这样的思想产生的,也是插值重要的应用价值之一。
        
        将某个区域的采样点数据插值生成栅格数据,实际上是将研究区域按照给定的格网尺寸(分辨率)进行栅格化,栅格数据中每一个栅格单元对应一块区域,栅格单元的
        值由其邻近的采样点的数值通过某种插值方法计算得到,因此,就可以预测采样点周围的数值情况,进而了解整个区域的数值分布情况。其中,插值方法主要有距离反
        比权值插值法、克吕金(Kriging)内插法、径向基函数RBF(Radial Basis Function)插值。
        利用插值分析功能能够预测任何地理点数据的未知值,如高程、降雨量、化学物浓度、噪声级等等。
        
        
        :param input\_data:  需要进行插值分析的点数据集或点记录集
        :type input\_data: DatasetVector or str or Recordset
        :param InterpolationParameter parameter: 插值方法需要的参数信息
        :param str z\_value\_field: 存储用于进行插值分析的值的字段名称。插值分析不支持文本类型的字段。
        :param pixel\_format: 指定结果栅格数据集存储的像素,不支持 BIT64
        :type pixel\_format: PixelFormat or str
        :param float z\_value\_scale: 插值分析值的缩放比率
        :param out\_data: 结果数据集所在的数据源
        :type out\_data: Datasource or DatasourceConnectionInfo or str
        :param str out\_dataset\_name: 结果数据集名称
        :param function progress: 进度信息处理函数,具体参考 :py:class:`.StepEvent`
        :return: 结果数据集或数据集名称
        :rtype: DatasetGrid or str
    
    interpolate\_points(points, values, parameter, pixel\_format, prj, out\_data, z\_value\_scale=1.0, out\_dataset\_name=None, progress=None)
        对点数组进行插值分析,并返回分析结果
        
        :param points: 需要进行插值分析的点数据
        :type points: list[Point2D]
        :param values:  点数组对应的用于插值分析的值。
        :type values: list[float]
        :param InterpolationParameter parameter:  插值方法需要的参数信息
        :param  pixel\_format: 指定结果栅格数据集存储的像素,不支持 BIT64
        :type pixel\_format: PixelFormat or str
        :param PrjCoordSys prj: 点数组的坐标系统。生成的结果数据集也参照该坐标系统。
        :param out\_data:  结果数据集所在的数据源
        :type out\_data: Datasource or DatasourceConnectionInfo or str
        :param float z\_value\_scale: 插值分析值的缩放比率
        :param str out\_dataset\_name: 结果数据集名称
        :param function progress: 进度信息处理函数,具体参考 :py:class:`.StepEvent`
        :return: 结果数据集或数据集名称
    
    kernel\_density(input\_data, value\_field, search\_radius, resolution, bounds=None, out\_data=None, out\_dataset\_name=None, progress=None)
        对点数据集或线数据集进行核密度分析,并返回分析结果。
        核密度分析,即使用核函数,来计算点或线邻域范围内的每单位面积量值。其结果是中间值大周边值小的光滑曲面,在邻域边界处降为0。
        
        :param input\_data: 需要进行核密度分析的点数据集或线数据集。
        :type input\_data: DatasetVector or str
        :param str value\_field: 存储用于进行密度分析的测量值的字段名称。若传 None 则所有几何对象都按值为1处理。不支持文本类型的字段。
        :param float search\_radius: 栅格邻域内用于计算密度的查找半径。单位与用于分析的数据集的单位相同。当计算某个栅格位置的未知数值时,会以该位置
                                    为圆心,以该属性设置的值为半径,落在这个范围内的采样对象都将参与运算,即该位置的预测值由该范围内采样对象的测量
                                    值决定。查找半径越大,生成的密度栅格越平滑且概化程度越高。值越小,生成的栅格所显示的信息越详细。
        
        :param float resolution: 密度分析结果栅格数据的分辨率
        :param Rectangle bounds: 密度分析的范围,用于确定运行结果所得到的栅格数据集的范围
        :param out\_data: 结果数据集所在的数据源
        :type out\_data: Datasource or DatasourceConnectionInfo or str
        :param str out\_dataset\_name: 结果数据集名称
        :param function progress: 进度信息处理函数,具体参考 :py:class:`.StepEvent`
        :return: 结果数据集或数据集名称
        :rtype: DatasetGrid or str
        
        >>> kernel\_density(data\_dir + 'example\_data.udb/taxi', 'passenger\_count', 0.01, 0.001, out\_data=out\_dir + 'density\_result.udb'
    
    kriging\_interpolate(input\_data, z\_value\_field, pixel\_format, resolution, krighing\_type='KRIGING', search\_mode=SearchMode.KDTREE\_FIXED\_COUNT, search\_radius=0.0, expected\_count=12, max\_point\_count\_in\_node=50, max\_point\_count\_for\_interpolation=200, variogram\_mode=VariogramMode.SPHERICAL, angle=0.0, mean=0.0, exponent=Exponent.EXP1, nugget=0.0, range\_value=0.0, sill=0.0, bounds=None, z\_value\_scale=1.0, out\_data=None, out\_dataset\_name=None, progress=None)
        使用克吕金插值方法对点数据集或记录集进行插值。具体参考 :py:meth:`interpolate` 和 :py:class:`.InterpolationKrigingParameter`
        
        :param input\_data:  需要进行插值分析的点数据集或点记录集
        :type input\_data: DatasetVector or str or Recordset
        :param str z\_value\_field: 存储用于进行插值分析的值的字段名称。插值分析不支持文本类型的字段。
        :param pixel\_format: 指定结果栅格数据集存储的像素,不支持 BIT64
        :type pixel\_format: PixelFormat or str
        :param float resolution: 插值运算时使用的分辨率
        :param krighing\_type: 插值分析的算法类型。支持设置 KRIGING, SimpleKRIGING, UniversalKRIGING 三种,默认使用 KRIGING。
        :type krighing\_type: InterpolationAlgorithmType or str
        :param search\_mode: 查找模式。
        :type search\_mode: SearchMode or str
        :param float search\_radius:  查找参与运算点的查找半径。单位与用于插值的点数据集(或记录集所属的数据集)的单位相同。查找半径决定了参与
                                         运算点的查找范围,当计算某个位置的未知数值时,会以该位置为圆心,search\_radius 为半径,落在这个范围内的
                                         采样点都将参与运算,即该位置的预测值由该范围内采样点的数值决定。
        :param int expected\_count:  期望参与插值运算的点数,当查找方式为变长查找时,表示期望参与运算的最多样点数。
        :param int max\_point\_count\_in\_node: 单个块内最多查找点数。当用QuadTree的查找插值点时,才可以设置块内最多点数。
        :param int max\_point\_count\_for\_interpolation: 设置块查找时,最多参与插值的点数。注意,该值必须大于零。当用QuadTree的查找插值点时,才可以设置最多参与插值的点数
        :param variogram:  克吕金(Kriging)插值时的半变函数类型。默认值为 VariogramMode.SPHERICAL
        :type variogram: VariogramMode or str
        :param float angle:  克吕金算法中旋转角度值
        :param float mean: 插值字段的平均值,即采样点插值字段值总和除以采样点数目。
        :param exponent: 用于插值的样点数据中趋势面方程的阶数
        :type exponent: Exponent or str
        :param float nugget: 块金效应值。
        :param float range\_value: 自相关阈值。
        :param float sill: 基台值
        :param Rectangle bounds: 插值分析的范围,用于确定运行结果的范围
        :param float z\_value\_scale: 插值分析值的缩放比率
        :param out\_data: 结果数据集所在的数据源
        :type out\_data: Datasource or DatasourceConnectionInfo or str
        :param str out\_dataset\_name: 结果数据集名称
        :param function progress: 进度信息处理函数,具体参考 :py:class:`.StepEvent`
        :return: 结果数据集或数据集名称
        :rtype: DatasetGrid or str
    
    measure\_central\_element(source, group\_field=None, weight\_field=None, self\_weight\_field=None, distance\_method='EUCLIDEAN', stats\_fields=None, out\_data=None, out\_dataset\_name=None, progress=None)
        关于空间度量:
        
            空间度量用来计算的数据可以是点、线、面。对于点、线和面对象,在距离计算中会使用对象的质心。对象的质心为所有子对象的加权
            平均中心。点对象的加权项为1(即质心为自身),线对象的加权项是长度,而面对象的加权项是面积。
        
            用户可以通过空间度量计算来解决以下问题:
        
                1. 数据的中心在哪里?
        
                2. 数据的分布呈什么形状和方向?
        
                3. 数据是如何分散布局?
        
            空间度量包括中心要素( :py:func:`measure\_central\_element` )、方向分布( :py:func:`measure\_directional` )、
            标准距离( :py:func:`measure\_standard\_distance` )、方向平均值( :py:func:`measure\_linear\_directional\_mean` )、
            平均中心( :py:func:`measure\_mean\_center` )、中位数中心( :py:func:`measure\_median\_center` )等。
        
        计算矢量数据的中心要素,返回结果矢量数据集。
        
         * 中心要素是与其他所有对象质心的累积距离最小,位于最中心的对象。
        
         * 如果设置了分组字段,则结果矢量数据集将包含 “分组字段名\_Group” 字段。
        
         * 实际上,距其他所有对象质心的累积距离最小的中心要素可能会有多个,但中心要素方法只会输出SmID 字段值最小的对象。
        
        :param source: 待计算的数据集。可以为点、线、面数据集。
        :type source: DatasetVector or str
        :param str group\_field: 分组字段的名称
        :param str weight\_field: 权重字段的名称
        :param str self\_weight\_field: 自身权重字段的名称
        :param distance\_method: 距离计算方法类型
        :type distance\_method: DistanceMethod or str
        :param stats\_fields: 统计字段的类型,为一个字典类型,字典类型的 key 为字段名,value 为统计类型。
        :type stats\_fields: list[tuple[str,SpatialStatisticsType]] or list[tuple[str,str]] or str
        :param out\_data: 用于存储结果数据集的数据源
        :type out\_data: DatasourceConnectionInfo or Datasource or str
        :param str out\_dataset\_name: 结果数据集名称
        :param progress: 进度信息处理函数,具体参考 :py:class:`.StepEvent`
        :type progress: function
        :return: 结果矢量数据集或数据集名称
        :rtype: DatasetVector or str
    
    measure\_directional(source, group\_field=None, ellipse\_size='SINGLE', stats\_fields=None, out\_data=None, out\_dataset\_name=None, progress=None)
        计算矢量数据的方向分布,返回结果矢量数据集。
        
         * 方向分布是根据所有对象质心的平均中心(有权重,为加权)为圆点,计算x和y坐标的标准差为轴得到的标准差椭圆。
        
         * 标准差椭圆的圆心x和y坐标、两个标准距离(长半轴和短半轴)、椭圆的方向,分别储存在结果矢量数据集中的CircleCenterX、
           CircleCenterY、SemiMajorAxis、SemiMinorAxis、RotationAngle字段中。如果设置了分组字段,则结果矢量数据集将包含
           “分组字段名\_Group” 字段。
        
         * 椭圆的方向RotationAngle字段中的正值表示正椭圆(长半轴的方向为X轴方向, 短半轴的方向为Y轴方向))按逆时针旋转,负值表示
           正椭圆按顺时针旋转。
        
         * 输出的椭圆大小有三个级别:Single(一个标准差)、Twice(二个标准差)和Triple(三个标准差),详细介绍请参见 :py:class:`.EllipseSize` 类。
        
         * 用于计算方向分布的标准差椭圆算法是由D. Welty Lefever在1926年提出,用来度量数据的方向和分布。首先确定椭圆的圆心,即平均
           中心(有权重,为加权);然后确定椭圆的方向;最后确定长轴和短轴的长度。
        
         .. image:: ../image/MeasureDirection.png
        
        
        关于空间度量介绍,请参考 :py:func:`measure\_central\_element`
        
        
        :param source: 待计算的数据集。可以为点、线、面数据集。
        :type source: DatasetVector or str
        :param str group\_field: 分组字段名称
        :param ellipse\_size: 椭圆大小类型
        :type ellipse\_size: EllipseSize or str
        :param stats\_fields: 统计字段的类型,为一个list类型,list 中存储2个元素的tuple,tuple的第一个元素为被统计的字段,第二个元素为统计类型
        :type stats\_fields: list[tuple[str,SpatialStatisticsType]] or list[tuple[str,str]] or str
        :param out\_data: 用于存储结果数据集的数据源
        :type out\_data: DatasourceConnectionInfo or Datasource or str
        :param str out\_dataset\_name: 结果数据集名称
        :param progress: 进度信息处理函数,具体参考 :py:class:`.StepEvent`
        :type progress: function
        :return: 结果矢量数据集
        :rtype: DatasetVector or str
    
    measure\_linear\_directional\_mean(source, group\_field=None, weight\_field=None, is\_orientation=False, stats\_fields=None, out\_data=None, out\_dataset\_name=None, progress=None)
        计算线数据集的方向平均值,并返回结果矢量数据集。
        
         * 线性方向平均值是根据所有线对象的质心的平均中心点为其中心、长度等于所有输入线对象的平均长度、方位或方向为由所有输入线对象
           的起点和终点(每个线对象都只会使用起点和终点来确定方向)计算得到的平均方位或平均方向创建的线对象。
        
         * 线对象的平均中心x和y坐标、平均长度、罗盘角、方向平均值、圆方差,分别储存在结果矢量数据集中的AverageX、AverageY、
           AverageLength、CompassAngle、DirectionalMean、CircleVariance字段中。如果设置了分组字段,则结果矢量数据集将包含
           “分组字段名\_Group” 字段。
        
         * 线对象的罗盘角(CompassAngle)字段表示以正北方为基准方向按顺时针旋转;方向平均值(DirectionalMean)字段表示以正东方为
           基准方向按逆时针旋转;圆方差(CircleVariance)表示方向或方位偏离方向平均值的程度,如果输入线对象具有非常相似(或完全相同)
           的方向则该值会非常小,反之则相反。
        
        
         .. image:: ../image/MeasureLinearDirectionalMean.png
        
        
        关于空间度量介绍,请参考 :py:func:`measure\_central\_element`
        
        :param source: 待计算的数据集。为线数据集。
        :type source: DatasetVector or str
        :param str group\_field: 分组字段名称
        :param str weight\_field: 权重字段名称
        :param bool is\_orientation: 是否忽略起点和终点的方向。为 False 时,将在计算方向平均值时使用起始点和终止点的顺序;为 True 时,将忽略起始点和终止点的顺序。
        :param stats\_fields:  统计字段的类型,为一个list类型,list 中存储2个元素的tuple,tuple的第一个元素为被统计的字段,第二个元素为统计类型
        :type stats\_fields: list[tuple[str,SpatialStatisticsType]] or list[tuple[str,str]] or str
        :param out\_data: 用于存储结果数据集的数据源
        :type out\_data: DatasourceConnectionInfo or Datasource or str
        :param str out\_dataset\_name: 结果数据集名称
        :param progress: 进度信息处理函数,具体参考 :py:class:`.StepEvent`
        :type progress: function
        :return: 结果数据集或数据集名称
        :rtype: DatasetVector or str
    
    measure\_mean\_center(source, group\_field=None, weight\_field=None, stats\_fields=None, out\_data=None, out\_dataset\_name=None, progress=None)
        计算矢量数据的平均中心,返回结果矢量数据集。
        
         * 平均中心是根据输入的所有对象质心的平均x和y坐标构造的点。
        
         * 平均中心的x和y坐标分别储存在结果矢量数据集中的SmX和SmY字段中。如果设置了分组字段,则结果矢量数据集将包含 “分组字段名\_Group” 字段。
        
         .. image:: ../image/MeasureMeanCenter.png
        
        
        关于空间度量介绍,请参考 :py:func:`measure\_central\_element`
        
        :param source: 待计算的数据集。可以为点、线、面数据集。
        :type source: DatasetVector or str
        :param str group\_field: 分组字段
        :param str weight\_field: 权重字段
        :param stats\_fields: 统计字段的类型,为一个list类型,list 中存储2个元素的tuple,tuple的第一个元素为被统计的字段,第二个元素为统计类型
        :type stats\_fields: list[tuple[str,SpatialStatisticsType]] or list[tuple[str,str]] or str
        :param out\_data: 用于存储结果数据集的数据源
        :type out\_data: DatasourceConnectionInfo or Datasource or str
        :param str out\_dataset\_name: 结果数据集名称
        :param progress: 进度信息处理函数,具体参考 :py:class:`.StepEvent`
        :type progress: function
        :return: 结果数据集或数据集名称
        :rtype: DatasetVector or str
    
    measure\_median\_center(source, group\_field, weight\_field, stats\_fields=None, out\_data=None, out\_dataset\_name=None, progress=None)
        计算矢量数据的中位数中心,返回结果矢量数据集。
        
         * 中位数中心是根据输入的所有对象质心,使用迭代算法找出到所有对象质心的欧式距离最小的点。
        
         * 中位数中心的x和y坐标分别储存在结果矢量数据集中的SmX和SmY字段中。如果设置了分组字段,则结果矢量数据集将包含
           “分组字段名\_Group” 字段。
        
         * 实际上,距所有对象质心的距离最小的点可能有多个,但中位数中心方法只会返回一个点。
        
         * 用于计算中位数中心的算法是由Kuhn,Harold W.和Robert E. Kuenne在1962年提出的迭代加权最小二乘法(Weiszfeld算法),之后由
           James E. Burt和Gerald M. Barber进一步概括。首先以平均中心(有权重,为加权)作为起算点,利用加权最小二乘法得到候选点,将
           候选点重新作为起算点代入计算得到新的候选点,迭代计算直到候选点到所有对象质心的欧式距离最小为止。
        
         .. image:: ../image/MeasureMedianCenter.png
        
        关于空间度量介绍,请参考 :py:func:`measure\_central\_element`
        
        :param source: 待计算的数据集。可以为点、线、面数据集。
        :type source: DatasetVector or str
        :param str group\_field: 分组字段
        :param str weight\_field: 权重字段
        :param stats\_fields: 统计字段的类型,为一个list类型,list 中存储2个元素的tuple,tuple的第一个元素为被统计的字段,第二个元素为统计类型
        :type stats\_fields: list[tuple[str,SpatialStatisticsType]] or list[tuple[str,str]] or str
        :param out\_data: 用于存储结果数据集的数据源
        :type out\_data: DatasourceConnectionInfo or Datasource or str
        :param str out\_dataset\_name: 结果数据集名称
        :param progress: 进度信息处理函数,具体参考 :py:class:`.StepEvent`
        :type progress: function
        :return: 结果数据集或数据集名称
        :rtype: DatasetVector or str
    
    measure\_standard\_distance(source, group\_field, weight\_field, ellipse\_size='SINGLE', stats\_fields=None, out\_data=None, out\_dataset\_name=None, progress=None)
        计算矢量数据的标准距离,返回结果矢量数据集。
        
         * 标准距离是根据所有对象质心的平均中心(有权重,为加权)为圆心,计算x和y坐标的标准距离为半径得到的圆。
        
         * 圆的圆心x和y坐标、标准距离(圆的半径),分别储存在结果矢量数据集中的CircleCenterX、CircleCenterY、StandardDistance字
           段中。如果设置了分组字段,则结果矢量数据集将包含 “分组字段名\_Group” 字段。
        
         * 输出的圆大小有三个级别:Single(一个标准差)、Twice(二个标准差)和Triple(三个标准差),详细介绍请参见 :py:class:`.EllipseSize` 枚举类型。
        
         .. image:: ../image/MeasureStandardDistance.png
        
        关于空间度量介绍,请参考 :py:func:`measure\_central\_element`
        
        :param source: 待计算的数据集。为线数据集
        :type source: DatasetVector or str
        :param str group\_field: 分组字段
        :param str weight\_field: 权重字段
        :param ellipse\_size: 椭圆大小类型
        :type ellipse\_size: EllipseSize or str
        :param stats\_fields: 统计字段的类型,为一个list类型,list 中存储2个元素的tuple,tuple的第一个元素为被统计的字段,第二个元素为统计类型
        :type stats\_fields: list[tuple[str,SpatialStatisticsType]] or list[tuple[str,str]] or str
        :param out\_data: 用于存储结果数据集的数据源
        :type out\_data: DatasourceConnectionInfo or Datasource or str
        :param str out\_dataset\_name: 结果数据集名称
        :param progress: 进度信息处理函数,具体参考 :py:class:`.StepEvent`
        :type progress: function
        :return: 结果数据集或数据集名称
        :rtype: DatasetVector or str
    
    minus\_math\_analyst(first\_operand, second\_operand, user\_region=None, out\_data=None, out\_dataset\_name=None, progress=None)
        栅格减法运算。逐个像元地从第一个栅格数据集的栅格值中减去第二个数据集的栅格值。进行此运算时,输入栅格数据集的顺序很重要,顺序不同,结果通常也是不相同的。栅格代数运算的具体使用,参考 :py:meth:`expression\_math\_analyst`
        
        如果输入两个像素类型(PixelFormat)均为整数类型的栅格数据集,则输出整数类型的结果数据集;否则,输出浮点型的结果数据集。如果输入的两个栅格数据集
        的像素类型精度不同,则运算的结果数据集的像素类型与二者中精度较高者保持一致。
        
        :param first\_operand: 指定的第一栅格数据集。
        :type first\_operand: DatasetGrid or str
        :param second\_operand:  指定的第二栅格数据集。
        :type second\_operand: DatasetGrid or str
        :param GeoRegion user\_region: 用户指定的有效计算区域。如果为 None,则表示计算全部区域,如果参与运算的数据集范围不一致,将使用所有数据集的范围的交集作为计算区域。
        :param out\_data: 结果数据集所在的数据源
        :type out\_data: Datasource or DatasourceConnectionInfo or str
        :param str out\_dataset\_name: 结果数据集名称
        :param function progress: 进度信息处理函数,具体参考 :py:class:`.StepEvent`
        :return: 结果数据集或数据集名称
        :rtype: DatasetGrid or str
    
    multiply\_math\_analyst(first\_operand, second\_operand, user\_region=None, out\_data=None, out\_dataset\_name=None, progress=None)
        栅格乘法运算。将输入的两个栅格数据集的栅格值逐个像元地相乘。栅格代数运算的具体使用,参考 :py:meth:`expression\_math\_analyst`
        
        如果输入两个像素类型(PixelFormat)均为整数类型的栅格数据集,则输出整数类型的结果数据集;否则,输出浮点型的结果数据集。如果输入的两个栅格数据集
        的像素类型精度不同,则运算的结果数据集的像素类型与二者中精度较高者保持一致。
        
        :param first\_operand: 指定的第一栅格数据集。
        :type first\_operand: DatasetGrid or str
        :param second\_operand:  指定的第二栅格数据集。
        :type second\_operand: DatasetGrid or str
        :param GeoRegion user\_region: 用户指定的有效计算区域。如果为 None,则表示计算全部区域,如果参与运算的数据集范围不一致,将使用所有数据集的范围的交集作为计算区域。
        :param out\_data: 结果数据集所在的数据源
        :type out\_data: Datasource or DatasourceConnectionInfo or str
        :param str out\_dataset\_name: 结果数据集名称
        :param function progress: 进度信息处理函数,具体参考 :py:class:`.StepEvent`
        :return: 结果数据集或数据集名称
        :rtype: DatasetGrid or str
    
    optimized\_hot\_spot\_analyst(source, assessment\_field=None, aggregation\_method='NETWORKPOLYGONS', aggregating\_polygons=None, bounding\_polygons=None, out\_data=None, out\_dataset\_name=None, progress=None)
        优化的热点分析,返回结果矢量数据集。
        
         * 结果数据集中包括z得分(Gi\_Zscore)、P值(Gi\_Pvalue)和置信区间(Gi\_ConfInvl),详细介绍请参阅 :py:func:`hot\_spot\_analyst` 方法结果。
        
         * z得分和P值都是统计显著性的度量,用于逐要素的判断是否拒绝"零假设"。置信区间字段会识别具有统计显著性的热点和冷点。置信区间
           为+3和-3的要素反映置信度为99\%的统计显著性,置信区间为+2和-2的要素反映置信度为95\%的统计显著性,置信区间为+1和-1的要素反映
           置信度为90\%的统计显著性,而置信区间为0的要素的聚类则没有统计意义。
        
         * 如果提供分析字段,则会直接执行热点分析; 如果未提供分析字段,则会利用提供的聚合方法(参阅 :py:class:`AggregationMethod`)聚
           合所有输入事件点以获得计数,从而作为分析字段执行热点分析。
        
         * 执行热点分析时,默认概念化模型为 :py:attr:`.ConceptualizationModel.FIXEDDISTANCEBAND` 、错误发现率(FDR)为 True ,
           统计显著性将使用错误发现率(FDR)校正法自动兼顾多重测试和空间依赖性。
        
         .. image:: ../image/ClusteringDistributions\_OptimizedHotSpotAnalyst.png
        
        关于聚类分布介绍,参考 :py:func:`cluster\_outlier\_analyst`
        
        :param source: 待计算的数据集。如果设置了评估字段,可以为点、线、面数据集,否则,则必须为点数据集。
        :type source: DatasetVector or str
        :param str assessment\_field: 评估字段的名称。
        :param aggregation\_method: 聚合方法。如果未设置提供分析字段,则需要为优化的热点分析提供的聚合方法。
        
                                   * 如果设置为 :py:attr:`.AggregationMethod.AGGREGATIONPOLYGONS` ,则必须设置 aggregating\_polygons
                                   * 如果设置为 :py:attr:`.AggregationMethod.NETWORKPOLYGONS` ,如果设置了 bounding\_polygons,则使用
                                     bounding\_polygons 进行聚合,如果没有设置 bounding\_polygons, 则使用点数据集的地理范围进行聚合。
                                   * 如果设置为 :py:attr:`.AggregationMethod.SNAPNEARBYPOINTS` , aggregating\_polygons 和 bounding\_polygons 都无效。
        
        :type aggregation\_method: AggregationMethod or str
        :param aggregating\_polygons: 聚合事件点以获得事件计数的面数据集。如果未提供分析字段(assessment\_field) 且 aggregation\_method
                                     设置为 :py:attr:`.AggregationMethod.AGGREGATIONPOLYGONS` 时,提供聚合事件点以获得事件计数的面数据集。
                                     如果设置了评估字段,此参数无效。
        :type aggregating\_polygons: DatasetVector or str
        :param bounding\_polygons: 事件点发生区域的边界面数据集。必须为面数据集。如果未提供分析字段(assessment\_field)且 aggregation\_method
                                  设置为 :py:attr:`.AggregationMethod.NETWORKPOLYGONS` 时,提供事件点发生区域的边界面数据集。
        :type bounding\_polygons: DatasetVector or str
        :param out\_data: 结果数据源信息
        :type out\_data: Datasource or DatasourceConnectionInfo or str
        :param str out\_dataset\_name: 结果数据集名称
        :param progress: 进度信息,具体参考 :py:class:`.StepEvent`
        :type progress: function
        :return: 结果数据集或数据集名称
        :rtype: DatasetVector or str
    
    ordering\_density\_based\_clustering(input\_data, min\_pile\_point\_count, search\_distance, unit, cluster\_sensitivity, out\_data=None, out\_dataset\_name=None, progress=None)
        密度聚类的OPTICS实现
        
        该方法在DBSCAN的基础上,额外计算了每个点的可达距离,并基于排序信息和聚类系数(cluster\_sensitivity)得到聚类结果。该方法对于搜索半径(search\_distance)和该范围内需包含的最少点数(min\_pile\_point\_count)不是很敏感,主要决定结果的是聚类系数(cluster\_sensitivity)
        
        概念定义:
        - 可达距离:取核心点的核心距离和其到周围临近点距离的最大值。
        - 核心点:某个点在搜索半径内,存在点的个数不小于每类包含的最少点数(min\_pile\_point\_count)。
        - 核心距离:某个点成为核心点的最小距离。
        - 聚类系数:为1\textasciitilde{}100的整数,是对聚类类别多少的标准量化,系数为1时聚类类别最少、100最多。
        
        :param input\_data: 指定的要聚类的矢量数据集,支持点数据集。
        :type input\_data: DatasetVector or str
        :param min\_pile\_point\_count: 每类包含的最少点数
        :type min\_pile\_point\_count: int
        :param search\_distance: 搜索邻域的距离
        :type search\_distance: int
        :param unit: 搜索距离的单位
        :type unit: Unit
        :param cluster\_sensitivity: 聚类系数
        :type cluster\_sensitivity: int
        :param out\_data: 结果数据集所在的数据源
        :type out\_data: Datasource or DatasourceConnectionInfo or str
        :param str out\_dataset\_name: 结果数据集名称
        :param function progress: 进度信息处理函数,具体参考 :py:class:`.StepEvent`
        :return: 结果数据集或数据集名称
        :rtype: DatasetVector or str
    
    overlay(source\_input, overlay\_input, overlay\_mode, source\_retained=None, overlay\_retained=None, tolerance=1e-10, out\_data=None, out\_dataset\_name='OverlayOutput', progress=None)
        叠加分析用于对输入的两个数据集或记录集之间进行各种叠加分析运算,如裁剪(clip)、擦除(erase)、合并(union)、相交(intersect)、同一(identity)、
        对称差(xOR)和更新(update)。叠加分析是 GIS 中的一项非常重要的空间分析功能。是指在统一空间参考系统下,通过对两个数据集进行的一系列集合运算,
        产生新数据集的过程。叠加分析广泛应用r于资源管理、城市建设评估、国土管理、农林牧业、统计等领域。因此,通过此叠加分析类可实现对空间数据的加工和分析,
        提取用户需要的新的空间几何信息,并且对数据的属性信息进行处理。
        
            - 进行叠加分析的两个数据集中,被称作输入数据集(在 SuperMap GIS 中称作第一数据集)的那个数据集,其类型可以是点、线、面等;另一个被称作叠加数据集(在 SuperMap GIS 中称作第二数据集)的数据集,其类型一般是面类型。
            - 应注意面数据集或记录集中本身应避免包含重叠区域,否则叠加分析结果可能出错。
            - 叠加分析的数据必须为具有相同地理参考的数据,包括输入数据和结果数据。
            - 在叠加分析的数据量较大的情况下,需对结果数据集创建空间索引,以提高数据的显示速度
            - 所有叠加分析的结果都不考虑数据集的系统字段
        
        需要注意:
            - 当 source\_input 为数据集时,overlay\_input 可以为数据集、记录集和面几何对象列表
            - 当 source\_input 为记录集时,overlay\_input 可以为数据集、记录集和面几何对象列表
            - 当 source\_input 为几何对象列表时,overlay\_input 可以为数据集、记录集和面几何对象列表
            - 当 source\_input 为几何对象列表时,必须设置有效的结果数据源信息
        
        
        :param source\_input: 叠加分析的源数据,可以是数据集、记录集和几何对象列表。当叠加分析模式为 update、xor 和 union 时,源数据只支持面数据。
                             当叠加分析模式为 clip、intersect、erase 和 identity 时,源数据支持点线面。
        :type source\_input: DatasetVector or Recordset or list[Geometry]
        :param overlay\_input: 参与计算的叠加数据,必须为面类型数据,可以是数据集、记录集和几何对象列表
        :type overlay\_input: DatasetVector or Recordset or list[Geometry]
        :param overlay\_mode: 叠加分析模式
        :type overlay\_mode: OverlayMode or str
        :param source\_retained: 源数据集或记录集中需要保留的字段。当 source\_retained 为 str 时,支持设置 ',' 分隔多个字段,例如 "field1,field2,field3"
        :type source\_retained: list[str] or str
        :param overlay\_retained: 参与计算的叠加数据需要保留的字段。当 overlay\_retained 为 str 时,支持设置 ',' 分隔多个字段,例如 "field1,field2,field3"。
                                 对于裁剪 (CLIP) 和擦除 (ERASE) 无效
        :type overlay\_retained: list[str] or str
        :param float tolerance: 叠加分析的容限值
        :param out\_data: 结果数据保存的数据源。如果为空,则结果数据集保存到叠加分析源数据集所在的数据源。
        :type out\_data: Datasource or DatasourceConnectionInfo or str
        :param str out\_dataset\_name: 结果数据集名称
        :param function progress: 进度信息处理函数,具体参考 :py:class:`.StepEvent`
        :return: 结果数据集或数据集名称
        :rtype: DatasetVector or str
    
    path\_line(target\_point, distance\_dataset, direction\_dataset, smooth\_method=None, smooth\_degree=0)
        根据距离栅格和方向栅格,分析从目标点出发到达最近源的最短路径(一个二维矢量线对象)。 该方法根据距离栅格和方向栅格,分析给定的目标点到达最近源的最短路径。其中距离栅格和方向栅格可以是耗费距离栅格和耗费方向栅格,也可以是表面距离栅格和表面方向栅格。
        
            - 当距离栅格为耗费距离栅格,方向栅格为耗费方向栅格时,该方法计算得出的是最小耗费路径。耗费距离栅格和耗费方向栅格可以通过 costDistance 方法生成。注意,此方法要求二者是同一次生成的结果。
            - 当距离栅格为表面距离栅格,方向栅格为表面方向栅格时,该方法计算得出的是最短表面距离路径。表面距离栅格和表面方向栅格可以通过 surfaceDistance 方法生成。同样,此方法要求二者是同一次生成的结果。
        
        源的位置在距离栅格和方向栅格中能够体现出来,即栅格值为 0 的单元格。源可以是一个,也可以有多个。当有多个源时,最短路径是目标点到达其最近的源的路径。
        
        下图为源、表面栅格、耗费栅格和目标点,其中耗费栅格是对表面栅格计算坡度后重分级的结果。
        
        .. image:: ../image/PathLine\_2.png
        
        使用如上图所示的源和表面栅格生成表面距离栅格和表面方向栅格,然后计算目标点到最近源的最短表面距离路径;使用源和耗费栅格生成耗费距离栅格和耗费方向栅格,然后计算目标点到最近源的最小耗费路径。得到的结果路径如下图所示:
        
        .. image:: ../image/PathLine\_3.png
        
        
        :param Point2D target\_point: 指定的目标点。
        :param DatasetGrid distance\_dataset: 指定的距离栅格。可以是耗费距离栅格或表面距离栅格。
        :param DatasetGrid direction\_dataset: 指定的方向栅格。与距离栅格对应,可以是耗费方向栅格或表面方向栅格。
        :param smooth\_method: 计算两点(源和目标)间最短路径时对结果路线进行光滑的方法
        :type smooth\_method: SmoothMethod or str
        :param int smooth\_degree: 计算两点(源和目标)间最短路径时对结果路线进行光滑的光滑度。
                                    光滑度的值越大,光滑度的值越大,则结果矢量线的光滑度越高。当 smooth\_method 不为 NONE 时有效。光滑度的有效取值与光滑方法有关,光滑方法有 B 样条法和磨角法:
                                    - 光滑方法为 B 样条法时,光滑度的有效取值为大于等于2的整数,建议取值范围为[2,10]。
                                    - 光滑方法为磨角法时,光滑度代表一次光滑过程中磨角的次数,设置为大于等于1的整数时有效
        :return: 返回表示最短路径的线对象和最短路径的花费
        :rtype: tuple[GeoLine,float]
    
    pickup\_border(input\_data, is\_preprocess=True, extract\_ids=None, out\_data=None, out\_dataset\_name=None, progress=None)
        提取面(或线)的边界,并保存为线数据集。若多个面(或线)共边界(线段),该边界(线段)只会被提取一次。
        
        不支持重叠面提取边界。
        
        :param input\_data: 指定的面或线数据集。
        :type input\_data: DatasetVector or str
        :param bool is\_preprocess: 是否进行拓扑预处理
        :param extract\_ids:  指定的面ID数组,可选参数,仅会提取给定ID数组对应的面对象边界。
        :type extract\_ids: list[int] or str
        :param out\_data: 用于存储结果数据集的数据源。
        :type out\_data: Datasource or DatasourceConnectionInfo or str
        :param str out\_dataset\_name: 结果数据集名称
        :param function progress: 进度信息处理函数,具体参考 :py:class:`.StepEvent`
        :return: 结果数据集或数据集名称
        :rtype: DatasetVector or str
    
    plus\_math\_analyst(first\_operand, second\_operand, user\_region=None, out\_data=None, out\_dataset\_name=None, progress=None)
        栅格加法运算。将输入的两个栅格数据集的栅格值逐个像元地相加。 栅格代数运算的具体使用,参考 :py:meth:`expression\_math\_analyst`
        
        如果输入两个像素类型(PixelFormat)均为整数类型的栅格数据集,则输出整数类型的结果数据集;否则,输出浮点型的结果数据集。如果输入的两个栅格数据集
        的像素类型精度不同,则运算的结果数据集的像素类型与二者中精度较高者保持一致。
        
        :param first\_operand: 指定的第一栅格数据集。
        :type first\_operand: DatasetGrid or str
        :param second\_operand:  指定的第二栅格数据集。
        :type second\_operand: DatasetGrid or str
        :param GeoRegion user\_region: 用户指定的有效计算区域。如果为 None,则表示计算全部区域,如果参与运算的数据集范围不一致,将使用所有数据集的范围的交集作为计算区域。
        :param out\_data: 结果数据集所在的数据源
        :type out\_data: Datasource or DatasourceConnectionInfo or str
        :param str out\_dataset\_name: 结果数据集名称
        :param function progress: 进度信息处理函数,具体参考 :py:class:`.StepEvent`
        :return: 结果数据集或数据集名称
        :rtype: DatasetGrid or str
    
    point3ds\_extract\_isoline(extracted\_points, resolution, interval, terrain\_interpolate\_type=None, datum\_value=0.0, expected\_z\_values=None, resample\_tolerance=0.0, smooth\_method='BSPLINE', smoothness=0, clip\_region=None, out\_data=None, out\_dataset\_name=None, progress=None)
        用于从三维点集合中提取等值线,并将结果保存为数据集。方法的实现原理是先利用点集合中存储的三维信息(高程或者温度等),也就是
        除了点的坐标信息的数据, 对点数据进行插值分析,得到栅格数据集(方法实现的中间结果,栅格值为单精度浮点型),然后从栅格数据集
        中提取等值线。
        
        点数据提取等值线介绍参考 :py:meth:`point\_extract\_isoline`
        
        注意:
        
         * 从点数据(点数据集/记录集/三维点集合)中提取等值线(面)时,插值得出的中间结果栅格的分辨率如果太小,会导致提取等值线(面)
           失败。这里提供一个判断方法:使用点数据的 Bounds 的长和宽分别除以设置的分辨率,也就是中间结果栅格的行列数,如果行列数任何一
           个大于10000,即认为分辨率设置的过小了,此时系统会抛出异常
        
        :param extracted\_points: 指定的待提取等值线的点串,该点串中的点是三维点,每一个点存储了 X,Y 坐标信息和只有一个三维度的信息(例如:高程信息等)。
        :type extracted\_points: list[Point3D]
        :param resolution: 指定的中间结果(栅格数据集)的分辨率。
        :type resolution: float
        :param float interval:  等值距,等值距是两条等值线之间的间隔值,必须大于0
        :param terrain\_interpolate\_type: 地形插值类型。
        :type terrain\_interpolate\_type: TerrainInterpolateType or str
        :param datum\_value: 设置等值线的基准值。基准值与等值距(interval)共同决定提取哪些高程上的等值线。基准值作为一个生成等值
                            线的初始起算值,以等值距为间隔向其前后两个方向计算,因此并不一定是最小等值线的值。例如,高程范围为
                            220-1550 的 DEM 栅格数据,如果设基准值为 500,等值距为 50,则提取等值线的结果是:最小等值线值为 250,
                            最大等值线值为 1550。
        
                            当同时设置 expected\_z\_values 时,只会考虑 expected\_z\_values 设置的值,即只提取高程为这些值的等值线。
        :type datum\_value: float
        :param expected\_z\_values: 期望分析结果的 Z 值集合。Z 值集合存储一系列数值,该数值为待提取等值线的值。即,仅高程值在Z值集
                                  合中的等值线会被提取。
                                  当同时设置 datum\_value 时,只会考虑 expected\_z\_values 设置的值,即只提取高程为这些值的等值线。
        :type expected\_z\_values: list[float] or str
        :param resample\_tolerance: 重采样的距离容限系数。通过对提取出的等值线行重采样,可以简化最终提取的等值线数据。SuperMap 在
                                   提取等值线/面时使用的重采样方法为光栏法(VectorResampleType.RTBEND),该方法需要一个重采样
                                   距离容限进行采样控制。它的值由重采样的距离容限系数乘以源栅格分辨率得出,一般取值为源栅格分辨率
                                   的 0~1 倍。
                                   重采样的距离容限系数默认为 0,即不进行任何采样,保证结果正确,但通过设置合理的参数,可以加快执
                                   行速度。容限值越大,等值线边界的控制点越少,此时可能出现等值线相交的情况。因此,推荐用户先使
                                   用默认值来提取等值线。
        :type resample\_tolerance: float
        :param smooth\_method: 滑处理所使用的方法
        :type smooth\_method: SmoothMethod or str
        :param smoothness: 设置等值线或等值面的光滑度。 光滑度为 0 或 1表示不进行光滑处理,值越大则光滑度越高。等值线提取时,光滑度可自由设置;
        :type smoothness: int
        :param clip\_region: 指定的裁剪面对象。
        :type clip\_region: GeoRegion
        :param out\_data: 用于存放结果数据集的数据源。如果为空,则直接返回等值线对象列表
        :type out\_data: Datasource or DatasourceConnectionInfo or str
        :param out\_dataset\_name:  指定的提取结果数据集的名称。
        :type out\_dataset\_name: str
        :param progress: function
        :type progress: 进度信息处理函数,具体参考 :py:class:`.StepEvent`
        :return: 提取等值线得到的数据集或数据集名称,或等值线对象列表
        :rtype: DatasetVector or str or list[GeoLine]
    
    point3ds\_extract\_isoregion(extracted\_points, resolution, interval, terrain\_interpolate\_type=None, datum\_value=0.0, expected\_z\_values=None, resample\_tolerance=0.0, smooth\_method='BSPLINE', smoothness=0, clip\_region=None, out\_data=None, out\_dataset\_name=None, progress=None)
        用于从三维点集合中提取等值面,并将结果保存为数据集。方法的实现原理是先利用点集合中存储的第三维信息(高程或者温度等),也就
        是除了点的坐标信息的数据, 对点数据使用 IDW 插值法(InterpolationAlgorithmType.IDW)进行插值分析,得到栅格数据集(方法实现
        的中间结果,栅格值为单精度浮点型),接着从栅格数据集中提取等值面。
        
        点数据提取等值面介绍,参考 :py:meth:`points\_extract\_isoregion`
        
        :param extracted\_points: 指定的待提取等值面的点串,该点串中的点是三维点,每一个点存储了 X,Y 坐标信息和只有一个第三维度的信息(例如:高程信息等)。
        :type extracted\_points: list[Point3D]
        :param resolution: 指定的中间结果(栅格数据集)的分辨率
        :type resolution: float
        :param float interval:  等值距,等值距是两条等值线之间的间隔值,必须大于0
        :param terrain\_interpolate\_type: 指定的地形插值类型。
        :type terrain\_interpolate\_type: TerrainInterpolateType or str
        :param datum\_value: 设置等值线的基准值。基准值与等值距(interval)共同决定提取哪些高程上的等值面。基准值作为一个生成等值
                            线的初始起算值,以等值距为间隔向其前后两个方向计算,因此并不一定是最小等值面的值。例如,高程范围为
                            220-1550 的 DEM 栅格数据,如果设基准值为 500,等值距为 50,则提取等值线的结果是:最小等值线值为 250,
                            最大等值线值为 1550。
        
                            当同时设置 expected\_z\_values 时,只会考虑 expected\_z\_values 设置的值,即只提取高程为这些值的等值线。
        :type datum\_value: float
        :param expected\_z\_values: 期望分析结果的 Z 值集合。Z 值集合存储一系列数值,该数值为待提取等值线的值。即,仅高程值在Z值集
                                  合中的等值线会被提取。
                                  当同时设置 datum\_value 时,只会考虑 expected\_z\_values 设置的值,即只提取高程为这些值的等值线。
        :type expected\_z\_values: list[float] or str
        :param resample\_tolerance: 重采样的距离容限系数。通过对提取出的等值线行重采样,可以简化最终提取的等值线数据。SuperMap 在
                                   提取等值线/面时使用的重采样方法为光栏法(VectorResampleType.RTBEND),该方法需要一个重采样
                                   距离容限进行采样控制。它的值由重采样的距离容限系数乘以源栅格分辨率得出,一般取值为源栅格分辨率
                                   的 0~1 倍。
                                   重采样的距离容限系数默认为 0,即不进行任何采样,保证结果正确,但通过设置合理的参数,可以加快执
                                   行速度。容限值越大,等值线边界的控制点越少,此时可能出现等值线相交的情况。因此,推荐用户先使
                                   用默认值来提取等值线。
        :type resample\_tolerance: float
        :param smooth\_method: 滑处理所使用的方法
        :type smooth\_method: SmoothMethod or str
        :param smoothness: 设置等值面的光滑度。 光滑度为 0 或 1表示不进行光滑处理,值越大则光滑度越高。
                           对于等值面的提取,采用先提取等值线然后生成等值面的方式,若将光滑度设置为2,
                           则中间结果数据集,即等值线对象的点数将为原始数据集点数的2倍,当光滑度设定值不断增大时,点数将成2的指数倍
                           增长,这将大大降低等值面提取的效率甚至可能导致提取失败。
        :type smoothness: int
        :param clip\_region: 指定的裁剪面对象。
        :type clip\_region: GeoRegion
        :param out\_data: 用于存放结果数据集的数据源。如果为空,则直接返回等值面对象列表
        :type out\_data: Datasource or DatasourceConnectionInfo or str
        :param out\_dataset\_name:  指定的提取结果数据集的名称。
        :type out\_dataset\_name: str
        :param progress: function
        :type progress: 进度信息处理函数,具体参考 :py:class:`.StepEvent`
        :return: 提取等值面得到的数据集或数据集名称,或等值面对象列表
        :rtype: DatasetVector or str or list[GeoRegion]
    
    point\_density(input\_data, value\_field, resolution, neighbour\_shape, neighbour\_unit='CELL', bounds=None, out\_data=None, out\_dataset\_name=None, progress=None)
        对点数据集进行点密度分析,并返回分析结果。
        简单点密度分析,即计算每个点的指定邻域形状内的每单位面积量值。计算方法为指定测量值除以邻域面积。点的邻域叠加处,其密度值也相加。
        每个输出栅格的密度均为叠加在栅格上的所有邻域密度值之和。结果栅格值的单位为原数据集单位的平方的倒数,即若原数据集单位为米,则结果栅格值的单位
        为每平方米。注意对于地理坐标数据集,结果栅格值的单位为“每平方度”,是没有实际意义的。
        
        :param input\_data: 需要进行核密度分析的点数据集或线数据集。
        :type input\_data: DatasetVector or str
        :param str value\_field: 存储用于进行密度分析的测量值的字段名称。若传 None 则所有几何对象都按值为1处理。不支持文本类型的字段。
        :param float resolution: 密度分析结果栅格数据的分辨率
        :param neighbour\_shape: 计算密度的查找邻域形状。如果输入值为 str,则要求格式为:
                                - 'CIRCLE,radius', 例如 'CIRCLE, 10'
                                - 'RECTANGLE,width,height',例如 'RECTANGLE,5.0,10.0'
                                - 'ANNULUS,inner\_radius,outer\_radius',例如 'ANNULUS,5.0,10.0'
                                - 'WEDGE,radius,start\_angle,end\_angle',例如 'WEDGE,10.0,0,45'
        :type neighbour\_shape: NeighbourShape or str
        :param neighbour\_unit: 邻域统计的单位类型。可以使用栅格坐标或地理坐标。
        :type neighbour\_unit: NeighbourUnitType or str
        :param Rectangle bounds: 密度分析的范围,用于确定运行结果所得到的栅格数据集的范围
        :param out\_data: 结果数据集所在的数据源
        :type out\_data: Datasource or DatasourceConnectionInfo or str
        :param str out\_dataset\_name: 结果数据集名称
        :param function progress: 进度信息处理函数,具体参考 :py:class:`.StepEvent`
        :return: 结果数据集或数据集名称
        :rtype: DatasetGrid or str
        
        
        >>> point\_density(data\_dir + 'example\_data.udb/taxi', 'passenger\_count', 0.0001, 'CIRCLE,0.001', 'MAP', out\_data=out\_dir + 'density\_result.udb')
    
    point\_extract\_isoline(extracted\_point, z\_value\_field, resolution, interval, terrain\_interpolate\_type=None, datum\_value=0.0, expected\_z\_values=None, resample\_tolerance=0.0, smooth\_method='BSPLINE', smoothness=0, clip\_region=None, out\_data=None, out\_dataset\_name=None, progress=None)
        用于从点数据集中提取等值线,并将结果保存为数据集。方法的实现原理类似“从点数据集中提取等值线”的方法,不同之处在于,
        这里的操作对象是点数据集,因此, 实现的过程是先对点数据集中的点数据使用 IDW 插值法('InterpolationAlgorithmType.IDW` )
        进行插值分析,得到栅格数据集(方法实现的中间结果,栅格值为单精度浮点型),然后从栅格数据集中提取等值线。
        
        点数据中的点是分散分布,点数据能够很好的表现位置信息,但对于点本身的其他属性信息却表现不出来,例如,已经获取了某个研究区域的
        大量采样点的高程信息,如下所示 (上图),从图上并不能看出地势高低起伏的趋势,看不出哪里地势陡峭、哪里地形平坦,如果我们运用
        等值线的原理,将这些点数据所蕴含的信息以等值线的形式表现出来, 即将相邻的具有相同高程值的点连接起来 ,形成下面下图所示的等
        高线图,那么关于这个区域的地形信息就明显的表现出来了。不同的点数据提取的等值线具有不同的含义,主要依据点数据多代表的信息而定,
        如果点的值代表温度,那么提取的等值线就是等温线;如果点的值代表雨量,那么提取的等值线就是等降水量线,等等。
        
        .. image:: ../image/SurfaceAnalyst\_3.png
        
        .. image:: ../image/SurfaceAnalyst\_4.png
        
        注意:
        
         * 从点数据(点数据集/记录集/三维点集合)中提取等值线(面)时,插值得出的中间结果栅格的分辨率如果太小,会导致提取等值线(面)
           失败。这里提供一个判断方法:使用点数据的 Bounds 的长和宽分别除以设置的分辨率,也就是中间结果栅格的行列数,如果行列数任何一
           个大于10000,即认为分辨率设置的过小了,此时系统会抛出异常
        
        :param extracted\_point: 指定的待提取的点数据集或记录集
        :type extracted\_point: DatasetVector or str or Recordset
        :param z\_value\_field: 指定的用于提取操作的字段名称。提取等值线时,将使用该字段中的值,对点数据集进行插值分析。
        :type z\_value\_field: str
        :param resolution: 指定的中间结果(栅格数据集)的分辨率。
        :type resolution: float
        :param float interval:  等值距,等值距是两条等值线之间的间隔值,必须大于0
        :param terrain\_interpolate\_type: 地形插值类型。
        :type terrain\_interpolate\_type: TerrainInterpolateType or str
        :param datum\_value: 设置等值线的基准值。基准值与等值距(interval)共同决定提取哪些高程上的等值线。基准值作为一个生成等值
                            线的初始起算值,以等值距为间隔向其前后两个方向计算,因此并不一定是最小等值线的值。例如,高程范围为
                            220-1550 的 DEM 栅格数据,如果设基准值为 500,等值距为 50,则提取等值线的结果是:最小等值线值为 250,
                            最大等值线值为 1550。
        
                            当同时设置 expected\_z\_values 时,只会考虑 expected\_z\_values 设置的值,即只提取高程为这些值的等值线。
        :type datum\_value: float
        :param expected\_z\_values: 期望分析结果的 Z 值集合。Z 值集合存储一系列数值,该数值为待提取等值线的值。即,仅高程值在Z值集
                                  合中的等值线会被提取。
                                  当同时设置 datum\_value 时,只会考虑 expected\_z\_values 设置的值,即只提取高程为这些值的等值线。
        :type expected\_z\_values: list[float] or str
        :param resample\_tolerance: 重采样的距离容限系数。通过对提取出的等值线行重采样,可以简化最终提取的等值线数据。SuperMap 在
                                   提取等值线/面时使用的重采样方法为光栏法(VectorResampleType.RTBEND),该方法需要一个重采样
                                   距离容限进行采样控制。它的值由重采样的距离容限系数乘以源栅格分辨率得出,一般取值为源栅格分辨率
                                   的 0~1 倍。
        
                                   重采样的距离容限系数默认为 0,即不进行任何采样,保证结果正确,但通过设置合理的参数,可以加快执
                                   行速度。容限值越大,等值线边界的控制点越少,此时可能出现等值线相交的情况。因此,推荐用户先使
                                   用默认值来提取等值线。
        :type resample\_tolerance: float
        :param smooth\_method: 滑处理所使用的方法
        :type smooth\_method: SmoothMethod or str
        :param smoothness: 设置等值线或等值面的光滑度。 光滑度为 0 或 1表示不进行光滑处理,值越大则光滑度越高。等值线提取时,光滑度可自由设置
        :type smoothness: int
        :param clip\_region: 指定的裁剪面对象。
        :type clip\_region: GeoRegion
        :param out\_data: 用于存放结果数据集的数据源。 如果为空,则会直接返回等值线对象的列表。
        :type out\_data: Datasource or DatasourceConnectionInfo or str
        :param out\_dataset\_name:  指定的提取结果数据集的名称。
        :type out\_dataset\_name: str
        :param progress: function
        :type progress: 进度信息处理函数,具体参考 :py:class:`.StepEvent`
        :return: 提取等值线得到的数据集或数据集名称,或等值线对象列表
        :rtype: DatasetVector or str or list[GeoLine]
    
    points\_extract\_isoregion(extracted\_point, z\_value\_field, interval, resolution=None, terrain\_interpolate\_type=None, datum\_value=0.0, expected\_z\_values=None, resample\_tolerance=0.0, smooth\_method='BSPLINE', smoothness=0, clip\_region=None, out\_data=None, out\_dataset\_name=None, progress=None)
        用于从点数据集中提取等值面。方法的实现原理是先对点数据集使用 IDW 插值法(InterpolationAlgorithmType.IDW)进行插值分析,
        得到栅格数据集(方法实现的中间结果,栅格值为单精度浮点型),接着从栅格数据集中提取等值线, 最终由等值线构成等值面。
        
        等值面是由相邻的等值线封闭组成的面。等值面的变化可以很直观的表示出相邻等值线之间的变化,诸如高程、温度、降水、污染或大气压
        力等用等值面来表示是非常直观、 有效的。等值面分布的效果与等值线的分布相同,也是反映了栅格表面上的变化,等值面分布越密集的地
        方,表示栅格表面值有较大的变化,反之则表示栅格表面值变化较少; 等值面越窄的地方,表示栅格表面值有较大的变化,反之则表示栅格
        表面值变化较少。
        
        如下所示,上图为存储了高程信息的点数据集,下图为从上图点数据集中提取的等值面,从等值面数据中可以明显的分析出地形的起伏变化,
        等值面越密集, 越狭窄的地方表示地势越陡峭,反之,等值面越稀疏,较宽的地方表示地势较舒缓,变化较小。
        
        .. image:: ../image/SurfaceAnalyst\_5.png
        
        .. image:: ../image/SurfaceAnalyst\_6.png
        
        注意:
        
         * 从点数据(点数据集/记录集/三维点集合)中提取等值面时,插值得出的中间结果栅格的分辨率如果太小,会导致提取等值面
           失败。这里提供一个判断方法:使用点数据的 Bounds 的长和宽分别除以设置的分辨率,也就是中间结果栅格的行列数,如果行列数任何一个
           大于10000,即认为分辨率设置的过小了,此时系统会抛出异常。
        
        :param extracted\_point: 指定的待提取的点数据集或记录集
        :type extracted\_point: DatasetVector or str or Recordset
        :param z\_value\_field: 指定的用于提取操作的字段名称。提取等值面时,将使用该字段中的值,对点数据集进行插值分析。
        :type z\_value\_field: str
        :param float interval:  等值距,等值距是两条等值线之间的间隔值,必须大于0
        :param resolution: 指定的中间结果(栅格数据集)的分辨率。
        :type resolution: float
        :param terrain\_interpolate\_type: 指定的地形插值类型。
        :type terrain\_interpolate\_type: TerrainStatisticType
        :param datum\_value: 设置等值线的基准值。基准值与等值距(interval)共同决定提取哪些高程上的等值面。基准值作为一个生成等值
                            线的初始起算值,以等值距为间隔向其前后两个方向计算,因此并不一定是最小等值面的值。例如,高程范围为
                            220-1550 的 DEM 栅格数据,如果设基准值为 500,等值距为 50,则提取等值线的结果是:最小等值线值为 250,
                            最大等值线值为 1550。
        
                            当同时设置 expected\_z\_values 时,只会考虑 expected\_z\_values 设置的值,即只提取高程为这些值的等值线。
        :type datum\_value: float
        :param expected\_z\_values: 期望分析结果的 Z 值集合。Z 值集合存储一系列数值,该数值为待提取等值线的值。即,仅高程值在Z值集
                                  合中的等值线会被提取。
                                  当同时设置 datum\_value 时,只会考虑 expected\_z\_values 设置的值,即只提取高程为这些值的等值线。
        :type expected\_z\_values: list[float] or str
        :param resample\_tolerance: 重采样的距离容限系数。通过对提取出的等值线行重采样,可以简化最终提取的等值线数据。SuperMap 在
                                   提取等值线/面时使用的重采样方法为光栏法(VectorResampleType.RTBEND),该方法需要一个重采样
                                   距离容限进行采样控制。它的值由重采样的距离容限系数乘以源栅格分辨率得出,一般取值为源栅格分辨率
                                   的 0~1 倍。
                                   重采样的距离容限系数默认为 0,即不进行任何采样,保证结果正确,但通过设置合理的参数,可以加快执
                                   行速度。容限值越大,等值线边界的控制点越少,此时可能出现等值线相交的情况。因此,推荐用户先使
                                   用默认值来提取等值线。
        :type resample\_tolerance: float
        :param smooth\_method: 滑处理所使用的方法
        :type smooth\_method: SmoothMethod or str
        :param smoothness: 设置等值面的光滑度。 光滑度为 0 或 1表示不进行光滑处理,值越大则光滑度越高。
                           对于等值面的提取,采用先提取等值线然后生成等值面的方式,若将光滑度设置为2,
                           则中间结果数据集,即等值线对象的点数将为原始数据集点数的2倍,当光滑度设定值不断增大时,点数将成2的指数倍
                           增长,这将大大降低等值面提取的效率甚至可能导致提取失败。
        :type smoothness: int
        :param clip\_region: 指定的裁剪面对象。
        :type clip\_region: GeoRegion
        :param out\_data: 用于存放结果数据集的数据源。如果为空,则直接返回等值面对象列表
        :type out\_data: Datasource or DatasourceConnectionInfo or str
        :param out\_dataset\_name:  指定的提取结果数据集的名称。
        :type out\_dataset\_name: str
        :param progress: function
        :type progress: 进度信息处理函数,具体参考 :py:class:`.StepEvent`
        :return: 提取等值面得到的数据集或数据集名称,或等值面对象列表
        :rtype: DatasetVector or str or list[GeoRegion]
    
    pour\_points(direction\_grid, accumulation\_grid, area\_limit, out\_data=None, out\_dataset\_name=None, progress=None)
        根据流向栅格和累积汇水量栅格生成汇水点栅格。
        
        汇水点位于流域的边界上,通常为边界上的最低点,流域内的水从汇水点流出,所以汇水点必定具有较高的累积汇水量。根据这一特点,就可以基于累积汇水量和流向栅格来提取汇水点。
        
        汇水点的确定需要一个累积汇水量阈值,累积汇水量栅格中大于或等于该阈值的位置将作为潜在的汇水点,再依据流向最终确定汇水点的位置。该阈值的确定十分关键,影响着汇水点的数量、位置以及子流域的大小和范围等。合理的阈值,需要考虑流域范围内的土壤特征、坡度特征、气候条件等多方面因素,根据实际研究的需求来确定,因此具有较大难度。
        
        获得了汇水点栅格后,可以结合流向栅格来进行流域的分割( :py:func:`watershed` 方法)。
        
        水文分析的相关介绍,请参考 :py:func:`basin`
        
        :param direction\_grid: 流向栅格数据
        :type direction\_grid: DatasetGrid or str
        :param accumulation\_grid: 累积汇水量栅格数据
        :type accumulation\_grid: DatasetGrid or str
        :param int area\_limit: 汇水量限制值
        :param out\_data: 用于存储结果数据集的数据源
        :type out\_data: DatasourceConnectionInfo or Datasource or str
        :param str out\_dataset\_name: 结果栅格数据集的名称
        :param progress: 进度信息处理函数,具体参考 :py:class:`.StepEvent`
        :type progress: function
        :return: 结果栅格数据集或数据集名称
        :rtype: DatasetGrid or  str
    
    preprocess(inputs, arcs\_inserted=True, vertex\_arc\_inserted=True, vertexes\_snapped=True, polygons\_checked=True, vertex\_adjusted=True, precisions=None, tolerance=1e-10, options=None, progress=None)
        对给定的拓扑数据集进行拓扑预处理。
        
        :param inputs: 输入数据集或记录集,如果是数据集,不能是只读。
        :type inputs: DatasetVector or list[DatasetVector] or str or list[str] or Recordset or list[Recordset]
        :param bool arcs\_inserted: 是否进行线段间求交插入节点
        :param bool vertex\_arc\_inserted: 否进行节点与线段间插入节点
        :param bool vertexes\_snapped: 是否进行节点捕捉
        :param bool polygons\_checked: 是否进行多边形走向调整
        :param bool vertex\_adjusted: 是否进行节点位置调整
        :param precisions: 指定的精度等级数组。精度等级的值越小,代表对应记录集的精度越高,数据质量越好。在进行顶点捕捉时,低精度的记录集中的点将被捕捉到高精度记录集中的点的位置上。精度等级数组必须与要进行拓扑预处理的记录集集合元素数量相同并一一对应。
        :type precisions: list[int]
        :param float tolerance: 指定的处理时需要的容限控制。单位与进行拓扑预处理的记录集单位相同。
        :param PreprocessOption options: 拓扑预处理参数类对象,如果此参数不为空,将优先使用此参数为拓扑预处理参数
        :param function progress: 进度信息处理函数,具体参考 :py:class:`.StepEvent`
        :return: 拓扑预处理是否成功
        :rtype: bool
    
    raster\_mosaic(inputs, back\_or\_no\_value, back\_tolerance, join\_method, join\_pixel\_format, cell\_size, encode\_type='NONE', valid\_rect=None, out\_data=None, out\_dataset\_name=None, progress=None)
        栅格数据集镶嵌。支持栅格数据集和影像数据集。
        
        栅格数据的镶嵌是指将两个或两个以上栅格数据按照地理坐标组成一个栅格数据。有时由于待研究分析的区域很大,或者感兴趣的目标对象
        分布很广,涉及到多个栅格数据集或者多幅影像,就需要进行镶嵌。下图展示了六幅相邻的栅格数据镶嵌为一幅数据。
        
        .. image:: ../image/Mosaic\_1.png
        
        进行栅格数据镶嵌时,需要注意以下要点:
        
         * 待镶嵌栅格必须具有相同的坐标系
           镶嵌要求所有栅格数据集或影像数据集具有相同的坐标系,否则镶嵌结果可能出错。可以在镶嵌前通过投影转换统一所有带镶嵌栅格的
           坐标系。
        
         * 重叠区域的处理
           镶嵌时,经常会出现两幅或多幅栅格数据之间有重叠区域的情况(如下图,两幅影像在红色框内的区域是重叠的),此时需要指定对重
           叠区域栅格的取值方式。SuperMap 提供了五种重叠区域取值方式,使用者可根据实际需求选择适当的方式,详见 :py:class:`.RasterJoinType` 类。
        
           .. image:: ../image/Mosaic\_2.png
        
         * 关于无值和背景色及其容限的说明
           待镶嵌的栅格数据有两种:栅格数据集和影像数据集。对于栅格数据集,该方法可以指定无值及无值的容限,对于影像数据集,该方法
           可以指定背景色及其容限。
        
           * 待镶嵌数据为栅格数据集:
        
              * 当待镶嵌的数据为栅格数据集时,栅格值为 back\_or\_no\_value 参数所指定的值的单元格,以及在 back\_tolerance 参数指定的容限范
                围内的单元格被视为无值,这些单元格不会参与镶嵌时的计算(叠加区域的计算),而栅格的原无值单元格则不再是无值数据从而参与运算。
        
              * 需要注意,无值的容限是用户指定的无值的值的容限,与栅格中原无值无关。
        
           * 待镶嵌数据为影像数据集
        
              * 当待镶嵌的数据为影像数据集时,栅格值为 back\_or\_no\_value 参数所指定的值的单元格,以及在 back\_tolerance 参数指定的容限
                范围内单元格被视为背景色,这些单元格不参与镶嵌时的计算。例如,指定无值的值为 a,指定的无值的容限为 b,则栅格值在
                [a-b,a+b] 范围内的单元格均不参与计算。
        
              * 注意,影像数据集中栅格值代表的是一个颜色。影像数据集的栅格值对应为 RGB 颜色,因此,如果想要将某种颜色设为背景色,
                为 back\_or\_no\_value 参数指定的值应为将该颜色(RGB 值)转为 32 位整型之后的值,系统内部会根据像素格式再进行相应的转换。
        
              * 对于背景色的容限值的设置,与背景色的值的指定方式相同:该容限值为一个 32 位整型值,在系统内部被转换为对应背景色
                R、G、B 的三个容限值,例如,指定为背景色的颜色为 (100,200,60),指定的容限值为 329738,该值对应的 RGB 值为
                (10,8,5),则值在 (90,192,55) 和 (110,208,65) 之间的颜色均被视为背景色,不参与计算。
        
        注意:
        
        将两个或以上高像素格式的栅格镶嵌成低像素格式的栅格时,结果栅格值可能超出值域,导致错误,因此不建议进行此种操作。
        
        
        :param inputs: 指定的待镶嵌的数据集的集合。
        :type inputs: list[Dataset] or list[str] or str
        :param back\_or\_no\_value: 指定的栅格背景颜色或无值的值。可以使用一个 float 或 tuple 表示一个 RGB 或 RGBA 值
        :type back\_or\_no\_value: float or tuple
        :param back\_tolerance: 指定的栅格背景颜色或无值的容限。可以使用一个 float 或 tuple 表示一个 RGB 或 RGBA 值
        :type back\_tolerance: float or tuple
        :param join\_method: 指定的镶嵌方法,即镶嵌时重叠区域的取值方式。
        :type join\_method: RasterJoinType or str
        :param join\_pixel\_format: 指定的镶嵌结果栅格数据的像素格式。
        :type join\_pixel\_format: RasterJoinPixelFormat or str
        :param float cell\_size: 指定的镶嵌结果数据集的单元格大小。
        :param encode\_type: 指定的镶嵌结果数据集的编码方式。
        :type encode\_type: EncodeType or str
        :param valid\_rect: 指定的镶嵌结果数据集的有效范围。
        :type valid\_rect: Rectangle
        :param out\_data: 指定的用于存储镶嵌结果数据集的数据源信息
        :type out\_data:  Datasource or DatasourceConnectionInfo or str
        :param str out\_dataset\_name: 指定的镶嵌结果数据集的名称。
        :param progress: 进度信息,具体参考 :py:class:`.StepEvent`
        :type progress: function
        :return: 镶嵌结果数据集
        :rtype: Dataset
    
    raster\_to\_vector(input\_data, value\_field, out\_dataset\_type=DatasetType.POINT, back\_or\_no\_value=-9999, back\_or\_no\_value\_tolerance=0.0, specifiedvalue=None, specifiedvalue\_tolerance=0.0, valid\_region=None, is\_thin\_raster=True, smooth\_method=None, smooth\_degree=0.0, out\_data=None, out\_dataset\_name=None, progress=None)
        通过指定转换参数设置将栅格数据集转换为矢量数据集。
        
        :param input\_data: 待转换的栅格数据集或影像数据集
        :type input\_data: DatasetGrid or DatasetImage or str
        :param str value\_field:  结果矢量数据集中存储值的字段
        :param out\_dataset\_type: 结果数据集类型,支持点、线和面数据集。当结果数据集类型为线数据聚集时,is\_thin\_raster, smooth\_method, smooth\_degree 才有效。
        :type out\_dataset\_type: DatasetType or str
        :param back\_or\_no\_value: 设置栅格的背景色或表示无值的值,只在栅格转矢量时有效。 允许用户指定一个值来标识那些不需要转换的单元格:
        
                                  - 当被转换的栅格数据为栅格数据集时,栅格值为指定的值的单元格被视为无值,这些单元格不会被转换,而栅格的原无值将作为有效值来参与转换。
                                  - 当被转化的栅格数据为影像数据集时,栅格值为指定的值的单元格被视为背景色,从而不参与转换。
        
                                 需要注意,影像数据集中栅格值代表的是一个颜色或颜色的索引值,与其像素格式(PixelFormat)有关。对于 BIT32、UBIT32、RGBA、RGB 和 BIT16
        
                                 格式的影像数据集,其栅格值对应为 RGB 颜色,可以使用一个 tuple 或 int 来表示 RGB 值 或 RGBA 值
        
                                 对于 UBIT8 和 UBIT4 格式的影像数据集,其栅格值对应的是颜色的索引值,因此,应为该属性设置的值为被视为背景色的颜色对应的索引值。
        :type back\_or\_no\_value: int or tuple
        :param back\_or\_no\_value\_tolerance: 栅格背景色的容限或无值的容限,只在栅格转矢量时有效。用于配合 back\_or\_no\_value 方法(指定栅格无值或者背景色)来共同确定栅格数据中哪些值不被转换:
        
                                            - 当被转换的栅格数据为栅格数据集时,如果指定为无值的值为 a,指定的无值的容限为 b,则栅格值在[a-b,a+b]范围内的单元格均被视为无值。需要注意,无值的容限是用户指定的无值的值的容限,与栅格中原无值无关。
                                            - 当被转化的栅格数据为影像数据集时,该容限值为一个32位整型值或tuple,tuple用于表示 RGB值或RGBA值。
                                            - 该值代表的意义与影像数据集的像素格式有关:对于栅格值对应 RGB 颜色的影像数据集,该值在系统内部被转为分别对应 R、G、B 的三个容限值,
                                              例如,指定为背景色的颜色为(100,200,60),指定的容限值为329738,该值对应的 RGB 值为(10,8,5),则值在 (90,192,55) 和 (110,208,65)
                                              之间的颜色均为背景色;对于栅格值为颜色索引值的影像数据集,该容限值为颜色索引值的容限,在该容限范围内的栅格值均视为背景色。
        
        :type back\_or\_no\_value\_tolerance: int or float or tuple
        :param specifiedvalue: 栅格按值转矢量时指定的栅格值。只将具有该值的栅格转为矢量。
        :type specifiedvalue: int or float or tuple
        :param specifiedvalue\_tolerance: 栅格按值转矢量时指定的栅格值的容限
        :type specifiedvalue\_tolerance: int or float or tuple
        :param valid\_region: 转换的有效区域
        :type valid\_region: GeoRegion or Rectangle
        :param bool is\_thin\_raster: 转换之前是否进行栅格细化。
        :param smooth\_method: 光滑方法,只在栅格转为矢量线数据时有效
        :type smooth\_method: SmoothMethod or str
        :param int smooth\_degree: 光滑度。光滑度的值越大,光滑度的值越大,则结果矢量线的光滑度越高。当 smooth\_method 不为 NONE 时有效。光滑度的有效取值与光滑方法有关,光滑方法有 B 样条法和磨角法:
        
                                    - 光滑方法为 B 样条法时,光滑度的有效取值为大于等于2的整数,建议取值范围为[2,10]。
                                    - 光滑方法为磨角法时,光滑度代表一次光滑过程中磨角的次数,设置为大于等于1的整数时有效
        
        :param out\_data: 结果数据集所在的数据源
        :type out\_data: Datasource or DatasourceConnectionInfo or str
        :param str out\_dataset\_name: 结果数据集名称
        :param function progress: 进度信息处理函数,具体参考 :py:class:`.StepEvent`
        :return: 结果数据集或数据集名称
        :rtype: DatasetVector or str
    
    rbf\_interpolate(input\_data, z\_value\_field, pixel\_format, resolution, search\_mode=SearchMode.KDTREE\_FIXED\_COUNT, search\_radius=0.0, expected\_count=12, max\_point\_count\_in\_node=50, max\_point\_count\_for\_interpolation=200, smooth=0.100000001490116, tension=40, bounds=None, z\_value\_scale=1.0, out\_data=None, out\_dataset\_name=None, progress=None)
        使用径向基函数(RBF) 插值方法对点数据集或记录集进行插值。具体参考 :py:meth:`interpolate` 和 :py:class:`.InterpolationRBFParameter`
        
        :param input\_data:  需要进行插值分析的点数据集或点记录集
        :type input\_data: DatasetVector or str or Recordset
        :param str z\_value\_field: 存储用于进行插值分析的值的字段名称。插值分析不支持文本类型的字段。
        :param pixel\_format: 指定结果栅格数据集存储的像素,不支持 BIT64
        :type pixel\_format: PixelFormat or str
        :param float resolution: 插值运算时使用的分辨率
        :param search\_mode: 查找模式。
        :type search\_mode: SearchMode or str
        :param float search\_radius: 查找参与运算点的查找半径。单位与用于插值的点数据集(或记录集所属的数据集)的单位相同。查找半径决定了参与运算点的查找范围,当计算某个位置的未知数值时,会以该位置为圆心,search\_radius 为半径,落在这个范围内的采样点都将参与运算,即该位置的预测值由该范围内采样点的数值决定。
        :param int expected\_count: 期望参与插值运算的点数,当查找方式为变长查找时,表示期望参与运算的最多样点数。
        :param int max\_point\_count\_in\_node: 单个块内最多查找点数。当用QuadTree的查找插值点时,才可以设置块内最多点数。
        :param int max\_point\_count\_for\_interpolation: 设置块查找时,最多参与插值的点数。注意,该值必须大于零。当用QuadTree的查找插值点时,才可以设置最多参与插值的点数
        :param float smooth: 光滑系数,值域为 [0,1]
        :param float tension: 张力系数
        :param Rectangle bounds: 插值分析的范围,用于确定运行结果的范围
        :param float z\_value\_scale: 插值分析值的缩放比率
        :param out\_data: 结果数据集所在的数据源
        :type out\_data: Datasource or DatasourceConnectionInfo or str
        :param str out\_dataset\_name: 结果数据集名称
        :param function progress: 进度信息处理函数,具体参考 :py:class:`.StepEvent`
        :return: 结果数据集或数据集名称
        :rtype: DatasetGrid or str
    
    reclass\_grid(input\_data, re\_pixel\_format, segments=None, reclass\_type='UNIQUE', is\_retain\_no\_value=True, change\_no\_value\_to=None, is\_retain\_missing\_value=False, change\_missing\_value\_to=None, reclass\_map=None, out\_data=None, out\_dataset\_name=None, progress=None)
        栅格数据重分级,返回结果栅格数据集。
        栅格重分级就是对源栅格数据的栅格值进行重新分类和按照新的分类标准赋值,其结果是用新的值取代了栅格数据的原栅格值。对于已知的栅格数据,有时为了便于看清趋势,找出栅格值的规律,或者为了方便进一步的分析,重分级是很必要的:
        
            - 通过重分级,可以使用新值来替代单元格的旧值,以达到更新数据的目的。例如,在处理土地类型变更时,将已经开垦为耕地的荒地赋予新的栅格值;
            - 通过重分级,可以对大量的栅格值进行分组归类,同组的单元格赋予相同的值来简化数据。例如,将旱地、水浇地、水田等都归为农业用地;
            - 通过重分级,可以对多种栅格数据按照统一的标准进行分类。例如,某个建筑物的选址的影响因素包括土壤和坡度,则对输入的土壤类型和坡度的栅格数据,可以按照 1-10 的等级标准来进行重分级,便于进一步的选址分析;
            - 通过重分级,可以将某些不希望参与分析的单元格设为无值,也可以为原先为无值的单元格补充新测定的值,便于进一步的分析处理。
        
        例如,常常需要对栅格表面进行坡度分析得到坡度数据,来辅助与地形有关的分析。但我们可能需要知道坡度属于哪个等级而不是具体的坡度数值,来帮助我们了解地形的陡峭程度,从而辅助进一步的分析,如选址、分析道路铺设线路等。此时可以使用重分级,将不同的坡度划分到对应的等级中。
        
        :param input\_data:  指定的用于栅格重采样的数据集。支持影像数据集,包括多波段影像
        :type input\_data: DatasetImage or DatasetGrid or str
        :param re\_pixel\_format: 结果数据集的栅格值的存储类型
        :type re\_pixel\_format: ReclassPixelFormat
        :param segments: 重分级区间集合。重分级区间集合。当 segments 为 str 时,支持使用 ';' 分隔多个ReclassSegment,每个 ReclassSegment使用 ','分隔 起始值、终止值、新值和分区类型。例如: '0,100,50,CLOSEOPEN; 100,200,150,CLOSEOPEN'
        :type segments: list[ReclassSegment] or str
        :param reclass\_type: 栅格重分级类型
        :type reclass\_type: ReclassType or str
        :param bool is\_retain\_no\_value: 是否将源数据集中的无值数据保持为无值
        :param float change\_no\_value\_to: 无值数据的指定值。 is\_retain\_no\_value 设置为 False 时,该设置有效,否则无效。
        :param bool is\_retain\_missing\_value: 源数据集中不在指定区间或单值之外的数据是否保留原值
        :param float change\_missing\_value\_to: 不在指定区间或单值内的栅格的指定值,is\_retain\_no\_value 设置为 False 时,该设置有效,否则无效。
        :param ReclassMappingTable reclass\_map: 栅格重分级映射表类。如果该对象不为空,使用该对象设置的值进行栅格重分级。
        :param out\_data: 结果数据集所在的数据源
        :type out\_data: Datasource or DatasourceConnectionInfo or str
        :param str out\_dataset\_name: 结果数据集名称
        :param function progress: 进度信息处理函数,具体参考 :py:class:`.StepEvent`
        :return: 结果数据集或数据集名称
        :rtype: DatasetGrid or DatasetImage or str
    
    region\_to\_center\_line(region\_data, out\_data=None, out\_dataset\_name=None, progress=None)
        提取面数据集或记录集的中心线,一般用于提取河流的中心线。
        
        该方法用于提取面对象的中心线。如果面包含岛洞,提取时会绕过岛洞,采用最短路径绕过。如下图。
        
        .. image:: ../image/RegionToCenterLine\_1.png
        
        如果面对象不是简单的长条形,而是具有分叉结构,则提取的中心线是最长的一段。如下图所示。
        
        .. image:: ../image/RegionToCenterLine\_2.png
        
        
        :param region\_data: 指定的待提取中心线的面记录集或面数据集
        :type region\_data: Recordset or DatasetVector
        :param out\_data: 结果数据源信息或数据源对象
        :type out\_data: Datasource or DatasourceConnectionInfo or str
        :param out\_dataset\_name: 结果中心线数据集名称
        :type out\_dataset\_name: str
        :param progress:  进度信息处理函数,具体参考 :py:class:`.StepEvent`
        :type progress: function
        :return: 结果数据集对象或结果数据集名称
        :rtype: DatasetVector or str
    
    resample\_raster(input\_data, new\_cell\_size, resample\_mode, out\_data=None, out\_dataset\_name=None, progress=None)
        栅格数据重采样,返回结果数据集。
        
        栅格数据经过了配准或纠正、投影等几何操作后,栅格的像元中心位置通常会发生变化,其在输入栅格中的位置不一定是整数的行列号,因此需要根据输出栅格上每个格子在输入栅格中的位置,对输入栅格按一定规则进行重采样,进行栅格值的插值计算,建立新的栅格矩阵。不同分辨率的栅格数据之间进行代数运算时,需要将栅格大小统一到一个指定的分辨率上,此时也需要对栅格进行重采样。
        
        栅格重采样有三种常用方法:最邻近法、双线性内插法和三次卷积法。有关这三种重采样方法较为详细的介绍,请参见 ResampleMode 类。
        
        :param input\_data:  指定的用于栅格重采样的数据集。支持影像数据集,包括多波段影像
        :type input\_data: DatasetImage or DatasetGrid or str
        :param float new\_cell\_size: 指定的结果栅格的单元格大小
        :param resample\_mode: 重采样计算方法
        :type resample\_mode: ResampleMode or str
        :param out\_data: 结果数据集所在的数据源
        :type out\_data: Datasource or DatasourceConnectionInfo or str
        :param str out\_dataset\_name: 结果数据集名称
        :param function progress: 进度信息处理函数,具体参考 :py:class:`.StepEvent`
        :return: 结果数据集或数据集名称
        :rtype: DatasetImage or DatasetGrid or str
    
    resample\_vector(input\_data, distance, resample\_type=VectorResampleType.RTBEND, is\_preprocess=True, tolerance=1e-10, is\_save\_small\_geometry=False, out\_data=None, out\_dataset\_name=None, progress=None)
        对矢量数据集进行重采样,支持线数据集、面数据集和网络数据集。 矢量数据重采样是按照一定规则剔除一些节点,以达到对数据进行简化的目的(如下图所示),
        其结果可能由于使用不同的重采样方法而不同。SuperMap 提供了两种重采样方法,具体参考 :py:class:`.VectorResampleType`
        
        .. image:: ../image/VectorResample.png
        
        该方法可以对线数据集、面数据集和网络数据集进行重采样。对面数据集重采样时,实质是对面对象的边界进行重采样。对于多个面对象的公共边界,如果进行了
        拓扑预处理只对其中一个多边形的该公共边界重采样一次,其他多边形的该公共边界会依据该多边形重采样的结果进行调整使之贴合,因此不会出现缝隙。
        
        注意: 重采样容限过大时,可能影响数据正确性,如出现两多边形的公共边界处出现相交的情况。
        
        :param input\_data: 需要进行重采样的矢量数据集,支持线数据集、面数据集和网络数据集
        :type input\_data: DatasetVector or str
        :param float distance: 设置重采样距离。单位与数据集坐标系单位相同。重采样距离可设置为大于 0 的浮点型数值。但如果设置的值小于默认值,将使用默认值。设置的重采样容限值越大,采样结果数据越简化
        :param resample\_type: 重采样方法。重采样支持光栏采样算法和道格拉斯算法。具体参考 :py:class:`.VectorResampleType` 。默认使用光栏采样。
        :type resample\_type: VectorResampleType or str
        :param bool is\_preprocess: 是否进行拓扑预处理。只对面数据集有效,如果数据集不进行拓扑预处理,可能会导致缝隙,除非能确保数据中两个相邻面公共线部分的节点坐标完全一致。
        :param float tolerance: 进行拓扑预处理时的节点捕捉容限,单位与数据集单位相同。
        :param bool is\_save\_small\_geometry: 是否保留小对象。小对象是指面积为0的对象,重采样过程有可能产生小对象。true 表示保留小对象,false 表示不保留。
        :param out\_data: 结果数据源所在半径,如果此参数为空,将直接对原始数据做采样,也就是会改变原始数据。如果此参数不为空,将会先复制原始数据到此数据源中,
                         再对复制得到的数据集进行采样处理。out\_data 所指向数据源可以与源数据集所在的数据源相同。
        :type out\_data: Datasource or DatasourceConnectionInfo or str
        :param str out\_dataset\_name: 结果数据集名称,当 out\_data 不为空时才有效。
        :param function progress: 进度信息处理函数,具体参考 :py:class:`.StepEvent`
        :return: 结果数据集或数据集名称
        :rtype: DatasetVector or str
    
    simplify\_building(source\_data, width\_threshold, height\_threshold, save\_failed=False, out\_data=None, out\_dataset\_name=None)
        面对象的直角多边形拟合
        如果一串连续的节点到最小面积外接矩形的下界的距离大于 height\_threshold,且节点的总宽度大于 width\_threshold,则对连续节点进行拟合。
        
        :param source\_data: 需要处理的面数据集
        :type source\_data: DatasetVector or str
        :param float width\_threshold: 点到最小面积外接矩形的左右边界的阈值
        :param float height\_threshold: 点到最小面积外接矩形的上下边界的阈值
        :param bool save\_failed: 面对象进行直角化失败时,是否保存源面对象,如果为 False,则结果数据集中不含失败的面对象。
        :param out\_data: 用于存储结果数据集的数据源。
        :type out\_data: Datasource or DatasourceConnectionInfo or str
        :param str out\_dataset\_name: 结果数据集名称
        :return: 结果数据集或数据集名称
        :rtype:  DatasetVector or str
    
    slice\_grid(input\_data, number\_zones, base\_output\_zones, out\_data=None, out\_dataset\_name=None, progress=None)
        自然分割重分级,适用于分布不均匀的数据。
        
        Jenks自然间断法:
        
        该重分级方法利用的是Jenks自然间断法。Jenks自然间断法基于数据中固有的自然分组,这是方差最小化分级的形式,间断通常不均匀,且间断 选择在值出现剧
        烈变动的地方,所以该方法能对相似值进行恰当分组并可使各分级间差异最大化。Jenks间断点分级法会将相似值(聚类值)放置在同一类中,所以该方法适用于
        分布不均匀的数据值。
        
        :param input\_data: 指定的进行重分级操作的栅格数据集。
        :type input\_data: DatasetGrid or str
        :param int number\_zones: 将栅格数据集重分级的区域数量。
        :param int base\_output\_zones:  结果栅格数据集中最低区域的值
        :param out\_data: 结果数据集所在的数据源
        :type out\_data: Datasource or DatasourceConnectionInfo or str
        :param str out\_dataset\_name: 结果数据集名称
        :param function progress: 进度信息处理函数,具体参考 :py:class:`.StepEvent`
        :return: 结果数据集或数据集名称
        :rtype: DatasetGrid or str
        
        
        设置分级区域数为9,将待分级栅格数据的最小值到最大值自然分割为9份。最低区域值设为1,重分级后的值以1为起始值每级递增。
        
        >>> slice\_grid('E:/data.udb/DEM', 9, 1, 'E:/Slice\_out.udb')
    
    smooth\_vector(input\_data, smoothness, out\_data=None, out\_dataset\_name=None, progress=None)
        对矢量数据集进行光滑,支持线数据集、面数据集和网络数据集
        
        * 光滑的目的
        
            当折线或多边形的边界的线段过多时,就可能影响对原始特征的描述,不利用进一步的处理或分析,或显示和打印效果不够理想,因此需要对数据简化。简化的方法
            一般有重采样(:py:meth:`resample\_vector`)和光滑。光滑是通过增加节点的方式使用曲线或直线段来代替原始折线的方法。需要注意,对折线进行光滑后,
            其长度通常会变短,折线上线段的方向也会发生明显改变,但两个端点的相对位置不会变化;面对象经过光滑后,其面积通常会变小。
        
        * 光滑方法与光滑系数的设置
        
            该方法采用 B 样条法对矢量数据集进行光滑。有关 B 样条法的介绍可参见 SmoothMethod 类。光滑系数(方法中对应 smoothness 参数)影响着光滑的程度,
            光滑系数越大,结果数据越光滑。光滑系数的建议取值范围为[2,10]。该方法支持对线数据集、面数据集和网络数据集进行光滑。
        
            * 对线数据集设置不同光滑系数的光滑效果:
        
            .. image:: ../image/Smooth\_1.png
        
            * 对面数据集设置不同光滑系数的光滑效果:
        
            .. image:: ../image/Smooth\_2.png
        
        
        :param input\_data: 需要进行光滑处理的数据集,支持线数据集、面数据集和网络数据集
        :type input\_data: DatasetVector or str
        :param int smoothness: 指定的光滑系数。取大于等于 2 的值有效,该值越大,线对象或面对象边界的节点数越多,也就越光滑。建议取值范围为[2,10]。
        :param out\_data: 结果数据源所在半径,如果此参数为空,将直接对原始数据做光滑,也就是会改变原始数据。如果此参数不为空,将会先复制原始数据到此数据源中,
                         再对复制得到的数据集进行光滑处理。out\_data 所指向数据源可以与源数据集所在的数据源相同。
        :type out\_data: Datasource or DatasourceConnectionInfo or str
        :param str out\_dataset\_name: 结果数据集名称,当 out\_data 不为空时才有效。
        :param function progress: 进度信息处理函数,具体参考 :py:class:`.StepEvent`
        :return: 结果数据集或数据集名称
        :rtype: DatasetVector or str
    
    split\_lines\_by\_regions(line\_input, region\_input, progress=None)
        用面对象分割线对象。在提取线对象的左右多边形(即 pickupLeftRightRegions() 方法)操作前,需要调用该方法分割线对象,否则会出现一个线对象对应多个左(右)多边形的情形。
        如下图:线对象 AB,如果不用面对象进行分割,则 AB 的左多边形有两个,分别为1,3;右多边形也有两个,分别为1和3,进行分割操作后,线对象 AB 分割为 AC 与 CB,此时 AC 与 CB 各自对应的左、右多边形分别只有一个。
        
        .. image:: ../image/SplitLinesByRegions.png
        
        
        :param line\_input:  指定的被分割的线记录集或数据集。
        :type line\_input: DatasetVector or Recordset
        :param region\_input: 指定的用于分割线记录集的面记录集或数据集。
        :type region\_input: DatasetVector or Recordset
        :param function progress:
        :return: 成功返回 True,失败返回 False。
        :rtype: bool
    
    straight\_distance(input\_data, max\_distance=-1.0, cell\_size=None, out\_data=None, out\_distance\_grid\_name=None, out\_direction\_grid\_name=None, out\_allocation\_grid\_name=None, progress=None)
        根据给定的参数,生成直线距离栅格,以及直线方向栅格和直线分配栅格。
        
        该方法用于对源数据集生成相应的直线距离栅格、直线方向栅格(可选)和直线分配栅格(可选),三个结果数据集的区域范围与源数据集的范围一致。生成直线距
        离栅格的源数据可以是矢量数据(点、线、面),也可以是栅格数据。对于栅格数据,要求除标识源以外的单元格为无值。
        
        * 直线距离栅格的值代表该单元格到最近的源的欧氏距离(即直线距离)。最近源是当前单元格到达所有源中直线距离最短的一个源。对于每个
          单元格,它的中心与源的中心相连的直线即为单元格到源的距离,计算的方法是通过二者形成的直角三角形的两条直角边来计算,因此直线
          距离的计算只与单元格大小(即分辨率)有关。下图为直线距离计算的示意图,其中源栅格的单元格大小(cell\_size)为10。
        
          .. image:: ../image/StraightDistance\_1.png
        
          那么第三行第三列的单元格到源的距离L为:
        
          .. image:: ../image/StraightDistance\_2.png
        
        * 直线方向栅格的值表示该单元格到最近的源的方位角,单位为度。以正东方向为90度,正南为180度,正西为270度,正北为360度,顺时针方向旋转,范围为0-360度,并规定对应源的栅格值为0度。
        
        * 直线分配栅格的值为单元格的最近源的值(源为栅格时,为最近源的栅格值;源为矢量对象时,为最近源的 SMID),因此从直线分配栅格中可以得知每个单元格的最近的源是哪个。
        
        下图为生成直线距离的示意图。单元格大小均为2。
        
        .. image:: ../image/StraightDistance\_3.png
        
        直线距离栅格通常用于分析经过的路线没有障碍或等同耗费的情况,例如,救援飞机飞往最近的医院时,空中没有障碍物,因此采用哪条路线的耗费均相同,此时通过直线距离栅格就可以确定从救援飞机所在地点到周围各医院的距离;根据直线分配栅格可以获知离救援飞机所在地点最近的医院;由直线方向栅格可以确定最近的医院在救援飞机所在地点的方位。
        
        然而,在救援汽车开往最近医院的实例中,因为地表有各种类型的障碍物,采用不同的路线的耗费不尽相同,这时,就需要使用耗费距离栅格来进行分析。有关耗费距离栅格请参见 CostDistance 方法。
        
        下图为生成直线距离栅格的一个实例,其中源数据集为点数据集,生成了直线距离栅格、直线方向栅格和直线分配栅格。
        
        .. image:: ../image/StraightDistance.png
        
        
        注意:当数据集的最小外接矩形(bounds)为某些特殊情形时,结果数据集的 Bounds 按以下规则取值:
        
        * 当源数据集的 Bounds 的高和宽均为 0 (如只有一个矢量点)时,结果数据集的 Bounds 的高和宽,均取源数据集 Bounds 的左边界值(Left)和下边界值(Right)二者绝对值较小的一个。
        * 当源数据集的 Bounds 的高为 0 而宽不为 0 (如只有一条水平线)时,结果数据集的 Bounds 的高和宽,均等于源数据集 Bounds 的宽。
        * 当源数据集的 Bounds 的宽为 0 而高不为 0 (如只有一条竖直线)时,结果数据集的 Bounds 的高和宽,均等于源数据集 Bounds 的高。
        
        
        :param input\_data: 生成距离栅格的源数据集。源是指感兴趣的研究对象或地物,如学校、道路或消防栓等。包含源的数据集,即为源数据集。源数据集可以为
                            点、线、面数据集,也可以为栅格数据集,栅格数据集中具有有效值的栅格为源,对于无值则视为该位置没有源。
        :type input\_data: DatasetVector or DatasetGrid or DatasetImage or str
        :param float max\_distance: 生成距离栅格的最大距离,大于该距离的栅格其计算结果取无值。若某个栅格单元格 A 到最近源之间的最短距离大于该值,则结果数据集中该栅格的值取无值。
        :param float cell\_size: 结果数据集的分辨率,是生成距离栅格的可选参数
        :param out\_data: 结果数据集所在的数据源
        :type out\_data:  Datasource or DatasourceConnectionInfo or str
        :param str out\_distance\_grid\_name: 结果距离栅格数据集的名称。如果名称为空,将自动获取有效的数据集名称。
        :param str out\_direction\_grid\_name: 方向栅格数据集的名称,如果为空,将不生成方向栅格数据集
        :param str out\_allocation\_grid\_name:  分配栅格数据集的名称,如果为空,将不生成 分配栅格数据集
        :param function progress: 进度信息处理函数,具体参考 :py:class:`.StepEvent`
        :return: 如果生成成功,返回结果数据集或数据集名称的元组,其中第一个为距离栅格数据集,第二个为方向栅格数据集,第三个为分配栅格数据集,如果没有设置方向栅格数据集名称和
                 分配栅格数据集名称,对应的值为 None
        :rtype: tuple[DataetGrid] or tuple[str]
    
    stream\_link(stream\_grid, direction\_grid, out\_data=None, out\_dataset\_name=None, progress=None)
        连接水系,即根据栅格水系和流向栅格为每条河流赋予唯一值。
        连接水系基于栅格水系和流向栅格,为水系中的每条河流分别赋予唯一值,值为整型。连接后的水系网络记录了水系节点的连接信息,体现了
        水系的网络结构。
        
        如下图所示,连接水系后,每条河段都有唯一的栅格值。图中红色的点为交汇点,即河段与河段相交的位置。河段是河流的一部分,它连接
        两个相邻交汇点,或连接一个交汇点和汇水点,或连接一个交汇点和分水线。因此,连接水系可用于确定流域盆地的汇水点。
        
        .. image:: ../image/StreamLink\_1.png
        
        下图连接水系的一个实例。
        
        .. image:: ../image/StreamLink\_2.png
        
        
        水文分析的相关介绍,请参考 :py:func:`basin`
        
        :param stream\_grid: 栅格水系数据
        :type stream\_grid: DatasetGrid or str
        :param direction\_grid: 流向栅格数据
        :type direction\_grid: DatasetGrid or str
        :param out\_data: 用于存储结果数据集的数据源
        :type out\_data: DatasourceConnectionInfo or Datasource or str
        :param str out\_dataset\_name: 结果栅格数据集的名称
        :param progress: 进度信息处理函数,具体参考 :py:class:`.StepEvent`
        :type progress: function
        :return: 连接后的栅格水系,为一个栅格数据集。返回结果栅格数据集或数据集名称
        :rtype: DatasetGrid or  str
    
    stream\_order(stream\_grid, direction\_grid, order\_type, out\_data=None, out\_dataset\_name=None, progress=None)
        对河流进行分级,根据河流等级为栅格水系编号。
        
        流域中的河流分为干流和支流,在水文学中,根据河流的流量、形态等因素对河流进行分级。在水文分析中,可以从河流的级别推断出河流的某些特征。
        
        该方法以栅格水系为基础,依据流向栅格对河流分级,结果栅格的值即代表该条河流的等级,值越大,等级越高。SuperMap 提供两种河流
        分级方法:Strahler 法和 Shreve 法。有关这两种方法的介绍请参见 :py:class`StreamOrderType` 枚举类型。
        
        如下图所示,是河流分级的一个实例。根据 Shreve 河流分级法,该区域的河流被分为14个等级。
        
        .. image:: ../image/StreamOrder.png
        
        水文分析的相关介绍,请参考 :py:func:`basin`
        
        :param stream\_grid: 栅格水系数据
        :type stream\_grid: DatasetGrid or str
        :param direction\_grid: 流向栅格数据
        :type direction\_grid: DatasetGrid or str
        :param order\_type: 流域水系编号方法
        :type order\_type: StreamOrderType or str
        :param out\_data: 用于存储结果数据集的数据源
        :type out\_data: DatasourceConnectionInfo or Datasource or str
        :param str out\_dataset\_name: 结果栅格数据集的名称
        :param progress: 进度信息处理函数,具体参考 :py:class:`.StepEvent`
        :type progress: function
        :return: 编号后的栅格流域水系网络,为一个栅格数据集。返回结果数据集或数据集名称。
        :rtype: DatasetGrid or  str
    
    stream\_to\_line(stream\_grid, direction\_grid, order\_type, out\_data=None, out\_dataset\_name=None, progress=None)
        提取矢量水系,即将栅格水系转化为矢量水系。
        
        提取矢量水系是基于流向栅格,将栅格水系转化为矢量水系(一个矢量线数据集)的过程。得到矢量水系后,就可以进行各种基于矢量的计
        算、处理和空间分析,如构建水系网络。下图为 DEM 数据以及对应的矢量水系。
        
        .. image:: ../image/StreamToLine.png
        
        通过该方法获得的矢量水系数据集,保留了河流的等级和流向信息。
        
         * 在提取矢量水系的同时,系统计算每条河流的等级,并在结果数据集中自动添加一个名为“StreamOrder”的属性字段来存储该值。分级的方式可
           通过 order\_type 参数设置。
        
         * 流向信息存储在结果数据集中名为“Direction”的字段中,以0或1来表示,0表示流向与该线对象的几何方向一致,1表示与线对象的几何
           方向相反。通过该方法获得的矢量水系的流向均与其几何方向相同,即“Direction”字段值都为0。在对矢量水系构建水系网络后,可直接
           使用(或根据实际需要进行修改)该字段作为流向字段。
        
        水文分析的相关介绍,请参考 :py:func:`basin`
        
        
        :param stream\_grid: 栅格水系数据
        :type stream\_grid: DatasetGrid or str
        :param direction\_grid: 流向栅格数据
        :type direction\_grid: DatasetGrid or str
        :param order\_type: 河流分级方法
        :type order\_type: StreamOrderType or str
        :param out\_data: 用于存储结果数据集的数据源
        :type out\_data: DatasourceConnectionInfo or Datasource or str
        :param str out\_dataset\_name: 结果矢量水系数据集名称
        :param progress: 进度信息处理函数,具体参考 :py:class:`.StepEvent`
        :type progress: function
        :return: 矢量水系数据集或数据集名称
        :rtype: DatasetVector or  str
    
    summary\_points(input\_data, radius, unit=None, stats=None, is\_random\_save\_point=False, is\_save\_attrs=False, out\_data=None, out\_dataset\_name=None, progress=None)
        根据指定的距离抽稀点数据集,即用一个点表示指定距离范围内的所有点。 该方法支持不同的单位,并且可以选择点抽稀的方式,还可以对抽稀点原始点集做统计。
        在结果数据集 resultDatasetName 中,会新建SourceObjID 和 StatisticsObjNum 两个字段。SourceObjID 字段存储抽稀后得到的点对象在原始数据集
        中的 SmID, StatisticsObjNum 表示当前点所代表的所有点数目,包括被抽稀的点和其自身。
        
        :param input\_data: 待抽稀的点数据集
        :type input\_data: DatasetVector or str or Recordset
        :param float radius:  抽稀点的半径。任取一个坐标点,在此坐标点半径内的所有点坐标通过此点表示。需注意选择抽稀点的半径的单位。
        :param unit: 抽稀点半径的单位。
        :type unit: Unit or str
        :param stats: 对抽稀点原始点集做统计。需要设置统计的字段名,统计结果的字段名和统计模式。当该数组为空表示不做统计。当 stats 为 str 时,支持设置以 ';'
                      分隔多个 StatisticsField,每个 StatisticsField 使用 ',' 分隔 'source\_field,stat\_type,result\_name',例如:
                      'field1,AVERAGE,field1\_avg; field2,MINVALUE,field2\_min'
        :type stats: list[StatisticsField] or str
        :param bool is\_random\_save\_point:  是否随机保存抽稀点。True表示从抽稀半径范围内的点集中随机取一个点保存,False表示取抽稀半径范围内点集中距点集内所有点的距离之和最小的点。
        :param bool is\_save\_attrs: 是否保留属性字段
        :param out\_data: 结果数据集所在的数据源
        :type out\_data: Datasource or DatasourceConnectionInfo or str
        :param str out\_dataset\_name: 结果数据集名称
        :param function progress: 进度信息处理函数,具体参考 :py:class:`.StepEvent`
        :return: 结果数据集或数据集名称
        :rtype: DatasetVector or str
    
    surface\_distance(input\_data, surface\_grid\_dataset, max\_distance=-1.0, cell\_size=None, max\_upslope\_degrees=90.0, max\_downslope\_degree=90.0, out\_data=None, out\_distance\_grid\_name=None, out\_direction\_grid\_name=None, out\_allocation\_grid\_name=None, progress=None)
        根据给定的参数,生成表面距离栅格,以及表面方向栅格和表面分配栅格。 该方法根据源数据集和表面栅格生成相应的表面距离栅格、表面方向栅格(可选)和表面
        分配栅格(可选)。源数据可以是矢量数据(点、线、面),也可以是栅格数据。对于栅格数据,要求除标识源以外的单元格为无值。
        
        * 表面距离栅格的值表示表面栅格上该单元格到最近源的表面最短距离。最近源是指当前单元格到达所有的源中表面距离最短的一个源。表面栅格中为无值的单元格在输出的表面距离栅格中仍为无值。
          从当前单元格(设为 g1)到达下一个单元格(设为 g2)的表面距离 d 的计算方法为:
        
          .. image:: ../image/SurfaceDistance\_1.png
        
          其中,b 为 g1 的栅格值(即高程)与 g2 的栅格值的差;a 为 g1 与 g2 的中心点之间的直线距离,其值考虑两种情况,当 g2 是与
          g1 相邻的上、下、左、右四个单元格之一时,a 的值等于单元格大小;当 g2 是与 g1 对角相邻的四个单元格之一时,a 的值为单元格大小乘以根号 2。
        
          当前单元格到达最近源的距离值就是沿着最短路径的表面距离值。在下面的示意图中,源栅格和表面栅格的单元格大小(CellSize)均为
          1,单元格(2,1)到达源(0,0)的表面最短路径如右图中红线所示:
        
          .. image:: ../image/SurfaceDistance\_2.png
        
          那么单元格(2,1)到达源的最短表面距离为:
        
          .. image:: ../image/SurfaceDistance\_3.png
        
        * 表面方向栅格的值表达的是从该单元格到达最近源的最短表面距离路径的行进方向。在表面方向栅格中,可能的行进方向共有八个(正北、
          正南、正西、正东、西北、西南、东南、东北),使用 1 到 8 八个整数对这八个方向进行编码,如下图所示。注意,源所在的单元格在表面方向栅格中的值为 0,表面栅格中为无值的单元格在输出的表面方向栅格中将被赋值为 15。
        
          .. image:: ../image/CostDistance\_3.png
        
        * 表面分配栅格的值为单元格的最近源的值(源为栅格时,为最近源的栅格值;源为矢量对象时,为最近源的 SMID),单元格到达最近的源具有最短表面距离。表面栅格中为无值的单元格在输出的表面分配栅格中仍为无值。
          下图为生成表面距离的示意图。其中,在表面栅格上,根据结果表面方向栅格,使用蓝色箭头标识了单元格到达最近源的行进方向。
        
          SurfaceDistance\_4.png
        
        通过上面的介绍,可以了解到,结合表面距离栅格及对应的方向、分配栅格,可以知道表面栅格上每个单元格最近的源是哪个,表面距离是多少以及如何到达该最近源。
        
        注意,生成表面距离时可以指定最大上坡角度(max\_upslope\_degrees)和最大下坡角度(max\_downslope\_degree),从而在寻找最近源时
        避免经过上下坡角度超过指定值的单元格。从当前单元格行进到下一个高程更高的单元格为上坡,上坡角度即上坡方向与水平面的夹角,如果
        上坡角度大于给定值,则不会考虑此行进方向;从当前单元格行进到下一个高程小于当前高程的单元格为下坡,下坡角度即下坡方向与水平面
        的夹角,同样的,如果下坡角度大于给定值,则不会考虑此行进方向。如果由于上下坡角度限制,使得当前单元格没能找到最近源,那么在
        表面距离栅格中该单元格的值为无值,在方向栅格和分配栅格中也为无值。
        
        下图为生成表面距离栅格的一个实例,其中源数据集为点数据集,表面栅格为对应区域的 DEM 栅格,生成了表面距离栅格、表面方向栅格和表面分配栅格。
        
        .. image:: ../image/SurfaceDistance.png
        
        
        :param input\_data: 生成距离栅格的源数据集。源是指感兴趣的研究对象或地物,如学校、道路或消防栓等。包含源的数据集,即为源数据集。源数据集可以为
                            点、线、面数据集,也可以为栅格数据集,栅格数据集中具有有效值的栅格为源,对于无值则视为该位置没有源。
        :type input\_data: DatasetVector or DatasetGrid or DatasetImage or str
        :param surface\_grid\_dataset: 表面栅格
        :type surface\_grid\_dataset: DatasetGrid or str
        :param float max\_distance: 生成距离栅格的最大距离,大于该距离的栅格其计算结果取无值。若某个栅格单元格 A 到最近源之间的最短距离大于该值,则结果数据集中该栅格的值取无值。
        :param float cell\_size: 结果数据集的分辨率,是生成距离栅格的可选参数
        :param float max\_upslope\_degrees: 最大上坡角度。单位为度,取值范围为大于或等于0。默认值为 90 度,即不考虑上坡角度。
                                          如果指定了最大上坡角度,则选择路线的时候会考虑地形的上坡的角度。从当前单元格行进到下一个高程更高的单元格
                                          为上坡,上坡角度即上坡方向与水平面的夹角。如果上坡角度大于给定值,则不会考虑此行进方向,即给出的路线不会
                                          经过上坡角度大于该值的区域。可想而知,可能会因为该值的设置而导致没有符合条件的路线。此外,由于坡度的表示
                                          范围为0到90度,因此,虽然可以指定为一个大于90度的值,但产生的效果与指定为90度相同,即不考虑上坡角度。
        :param float max\_downslope\_degree: 设置最大下坡角度。单位为度,取值范围为大于或等于0。
                                          如果指定了最大下坡角度,则选择路线的时候会考虑地形的下坡的角度。从当前单元格行进到下一个高程小于当前高
                                          程的单元格为下坡,下坡角度即下坡方向与水平面的夹角。如果下坡角度大于给定值,则不会考虑此行进方向,即给
                                          出的路线不会经过下坡角度大于该值的区域。可想而知,可能会因为该值的设置而导致没有符合条件的路线。此外,
                                          由于坡度的表示范围为0到90度,因此,虽然可以指定为一个大于90度的值,但产生的效果与指定为90度相同,即不
                                          考虑下坡角度。
        :param out\_data: 结果数据集所在的数据源
        :type out\_data:  Datasource or DatasourceConnectionInfo or str
        :param str out\_distance\_grid\_name: 结果距离栅格数据集的名称。如果名称为空,将自动获取有效的数据集名称。
        :param str out\_direction\_grid\_name: 方向栅格数据集的名称,如果为空,将不生成方向栅格数据集
        :param str out\_allocation\_grid\_name:  分配栅格数据集的名称,如果为空,将不生成 分配栅格数据集
        :param function progress: 进度信息处理函数,具体参考 :py:class:`.StepEvent`
        :return: 如果生成成功,返回结果数据集或数据集名称的元组,其中第一个为距离栅格数据集,第二个为方向栅格数据集,第三个为分配栅格数据集,如果没有设置方向栅格数据集名称和
                 分配栅格数据集名称,对应的值为 None
        :rtype: tuple[DataetGrid] or tuple[str]
    
    surface\_path\_line(source\_point, target\_point, surface\_grid\_dataset, max\_upslope\_degrees=90.0, max\_downslope\_degree=90.0, smooth\_method=None, smooth\_degree=0, progress=None)
        根据给定的参数,计算源点和目标点之间的最短表面距离路径(一个二维矢量线对象)。该方法用于根据给定的源点、目标点和表面栅格,计算源点与目标点之间的最短表面距离路径。
        
        设置最大上坡角度(max\_upslope\_degrees)和最大下坡角度(max\_downslope\_degree)可以使分析得出的路线不经过过于陡峭的地形。
        但注意,如果指定了上下坡角度限制,也可能得不到分析结果,这与最大上下坡角度的值和表面栅格所表达的地形有关。下图展示了将最
        大上坡角度和最大下坡角度分别均设置为 5 度、10 度和 90 度(即不限制上下坡角度)时的表面距离最短路径,由于对上下坡角度做出
        了限制,因此表面距离最短路径是以不超过最大上下坡角度为前提而得出的。
        
        .. image:: ../image/SurfacePathLine.png
        
        :param Point2D source\_point: 指定的源点。
        :param Point2D target\_point:  指定的目标点。
        :param surface\_grid\_dataset: 表面栅格
        :type surface\_grid\_dataset: DatasetGrid or str
        :param float max\_upslope\_degrees: 最大上坡角度。单位为度,取值范围为大于或等于0。默认值为 90 度,即不考虑上坡角度。
                                          如果指定了最大上坡角度,则选择路线的时候会考虑地形的上坡的角度。从当前单元格行进到下一个高程更高的单元格
                                          为上坡,上坡角度即上坡方向与水平面的夹角。如果上坡角度大于给定值,则不会考虑此行进方向,即给出的路线不会
                                          经过上坡角度大于该值的区域。可想而知,可能会因为该值的设置而导致没有符合条件的路线。此外,由于坡度的表示
                                          范围为0到90度,因此,虽然可以指定为一个大于90度的值,但产生的效果与指定为90度相同,即不考虑上坡角度。
        :param float max\_downslope\_degree: 设置最大下坡角度。单位为度,取值范围为大于或等于0。
                                          如果指定了最大下坡角度,则选择路线的时候会考虑地形的下坡的角度。从当前单元格行进到下一个高程小于当前高
                                          程的单元格为下坡,下坡角度即下坡方向与水平面的夹角。如果下坡角度大于给定值,则不会考虑此行进方向,即给
                                          出的路线不会经过下坡角度大于该值的区域。可想而知,可能会因为该值的设置而导致没有符合条件的路线。此外,
                                          由于坡度的表示范围为0到90度,因此,虽然可以指定为一个大于90度的值,但产生的效果与指定为90度相同,即不
                                          考虑下坡角度。
        :param smooth\_method: 计算两点(源和目标)间最短路径时对结果路线进行光滑的方法
        :type smooth\_method: SmoothMethod or str
        :param int smooth\_degree: 计算两点(源和目标)间最短路径时对结果路线进行光滑的光滑度。
                                    光滑度的值越大,光滑度的值越大,则结果矢量线的光滑度越高。当 smooth\_method 不为 NONE 时有效。光滑度的有效取值与光滑方法有关,光滑方法有 B 样条法和磨角法:
                                    - 光滑方法为 B 样条法时,光滑度的有效取值为大于等于2的整数,建议取值范围为[2,10]。
                                    - 光滑方法为磨角法时,光滑度代表一次光滑过程中磨角的次数,设置为大于等于1的整数时有效
        :param function progress: 进度信息处理函数,具体参考 :py:class:`.StepEvent`
        :return: 返回表示最短路径的线对象和最短路径的花费
        :rtype: tuple[GeoLine,float]
    
    thin\_raster(source, back\_or\_no\_value, back\_or\_no\_value\_tolerance, out\_data=None, out\_dataset\_name=None, progress=None)
        栅格细化,通常在将栅格转换为矢量线数据前使用。
        
        栅格数据细化处理可以减少栅格数据中用于标识线状地物的单元格的数量,从而提高矢量化的速度和精度。一般作为栅格转线矢量数据之
        前的预处理,使转换的效果更好。例如一幅扫描的等高线图上可能使用 5、6 个单元格来显示一条等高线的宽度,细化处理后,等高线的
        宽度就只用一个单元格来显示了,有利于更好地进行矢量化。
        
        .. image:: ../image/ThinRaster.png
        
        关于无值/背景色及其容限的说明:
        
        进行栅格细化时,允许用户标识那些不需要细化的单元格。对于栅格数据集,通过无值及其容限来确定这些值,对于影像数据集,则通过背景色及其容限来确定。
        
        * 当对栅格数据集进行细化时,栅格值为 back\_or\_no\_value 参数指定的值的单元格被视为无值,不参与细化,而栅格的原无值将作为有效值来参与细化;
          同时,在 back\_or\_no\_value\_tolerance 参数指定的无值的容限范围内的单元格也不参与细化。例如,指定无值的值为 a,指定的无值的容限为 b,
          则栅格值在 [a-b,a+b] 范围内的单元格均不参与细化。
        
        * 当对影像数据集进行细化时,栅格值为指定的值的单元格被视为背景色,不参与细化;同时,在 back\_or\_no\_value\_tolerance 参数指
          定的背景色的容限范围内的单元格也不参与细化。
        
        需要注意,影像数据集中栅格值代表的是一个颜色值,因此,如果想要将某种颜色设为背景色,为 back\_or\_no\_value 参数指定的值应为
        将该颜色(RGB 值)转为 32 位整型之后的值,系统内部会根据像素格式再进行相应的转换。背景色的容限同样为一个 32 位整型值。该
        值在系统内部被转为分别对应 R、G、B 的三个容限值,例如,指定为背景色的颜色为 (100,200,60),指定的容限值为 329738,该值对应
        的 RGB 值为 (10,8,5),则值在 (90,192,55) 和 (110,208,65) 之间的颜色均不参与细化。
        
        注意:对于栅格数据集,如果指定的无值的值,在待细化的栅格数据集的值域范围外,会分析失败,返回 None。
        
        :param source: 指定的待细化的栅格数据集。支持影像数据集。
        :type source: DatasetImage or DatasetGrid or str
        :param back\_or\_no\_value: 指定栅格的背景色或表示无值的值。可以使用一个 int 或 tuple 来表示一个 RGB 或 RGBA 值。
        :type back\_or\_no\_value: int or tuple
        :param back\_or\_no\_value\_tolerance: 栅格背景色的容限或无值的容限。可以使用一个 float 或 tuple 来表示一个 RGB 或 RGBA 值。
        :type back\_or\_no\_value\_tolerance: float or tuple
        :param out\_data: 用于存储结果数据的数据源。
        :type out\_data: Datasource or DatasourceConnectionInfo or str
        :param out\_dataset\_name: 结果数据集的名称
        :type out\_dataset\_name: str
        :param progress: 进度信息处理函数,具体参考 :py:class:`.StepEvent`
        :type progress: function
        :return: 结果数据集或数据集名称
        :rtype: Dataset or str
    
    to\_float\_math\_analyst(input\_data, user\_region=None, out\_data=None, out\_dataset\_name=None, progress=None)
        栅格浮点运算。将输入的栅格数据集的栅格值转换成浮点型。 如果输入的栅格值为双精度浮点型,进行浮点运算后的结果栅格值也转换为单精度浮点型。
        
        :param input\_data: 指定的第一栅格数据集。
        :type input\_data: DatasetGrid or str
        :param GeoRegion user\_region: 用户指定的有效计算区域。如果为 None,则表示计算全部区域,如果参与运算的数据集范围不一致,将使用所有数据集的范围的交集作为计算区域。
        :param out\_data: 结果数据集所在的数据源
        :type out\_data: Datasource or DatasourceConnectionInfo or str
        :param str out\_dataset\_name: 结果数据集名称
        :param function progress: 进度信息处理函数,具体参考 :py:class:`.StepEvent`
        :return: 结果数据集或数据集名称
        :rtype: DatasetGrid or str
    
    to\_int\_math\_analyst(input\_data, user\_region=None, out\_data=None, out\_dataset\_name=None, progress=None)
        栅格取整运算。提供对输入的栅格数据集的栅格值进行取整运算。取整运算的结果是去除栅格值的小数部分,只保留栅格值的整数。如果输入栅格值为整数类型,进行取整运算后的结果与输入栅格值相同。
        
        :param input\_data: 指定的第一栅格数据集。
        :type input\_data: DatasetGrid or str
        :param GeoRegion user\_region: 用户指定的有效计算区域。如果为 None,则表示计算全部区域,如果参与运算的数据集范围不一致,将使用所有数据集的范围的交集作为计算区域。
        :param out\_data: 结果数据集所在的数据源
        :type out\_data: Datasource or DatasourceConnectionInfo or str
        :param str out\_dataset\_name: 结果数据集名称
        :param function progress: 进度信息处理函数,具体参考 :py:class:`.StepEvent`
        :return: 结果数据集或数据集名称
        :rtype: DatasetGrid or str
    
    topology\_build\_regions(input\_data, out\_data=None, out\_dataset\_name=None, progress=None)
        用于将线数据集或者网络数据集,通过拓扑处理来构建面数据集。在进行拓扑构面前,最好能使用拓扑处理 :py:meth:`topology\_processing` 对数据集进行拓扑处理。
        
        :param input\_data: 指定的用于进行多边形拓扑处理的源数据集,只能是线数据集或网络数据集。
        :type input\_data: DatasetVector or str
        :param out\_data: 用于存储结果数据集的数据源。
        :type out\_data: Datasource or DatasourceConnectionInfo or str
        :param str out\_dataset\_name: 结果数据集名称
        :param function progress: 进度信息处理函数,具体参考 :py:class:`.StepEvent`
        :return: 结果数据集或数据集名称
        :rtype: DatasetVector or str
    
    topology\_processing(input\_data, pseudo\_nodes\_cleaned=True, overshoots\_cleaned=True, redundant\_vertices\_cleaned=True, undershoots\_extended=True, duplicated\_lines\_cleaned=True, lines\_intersected=True, adjacent\_endpoints\_merged=True, overshoots\_tolerance=1e-10, undershoots\_tolerance=1e-10, vertex\_tolerance=1e-10, filter\_vertex\_recordset=None, arc\_filter\_string=None, filter\_mode=None, options=None, progress=None)
        根据拓扑处理选项对给定的数据集进行拓扑处理。将直接修改原始数据。
        
        :param input\_data: 指定的拓扑处理的数据集。
        :type input\_data: DatasetVector or str
        :param bool pseudo\_nodes\_cleaned: 是否去除假结点
        :param bool overshoots\_cleaned: 是否去除短悬线。
        :param bool redundant\_vertices\_cleaned: 是否去除冗余点
        :param bool undershoots\_extended: 是否进行长悬线延伸。
        :param bool duplicated\_lines\_cleaned: 是否去除重复线
        :param bool lines\_intersected: 是否进行弧段求交。
        :param bool adjacent\_endpoints\_merged: 是否进行邻近端点合并。
        :param float overshoots\_tolerance:  短悬线容限,该容限用于在去除短悬线时判断悬线是否是短悬线。
        :param float undershoots\_tolerance: 长悬线容限,该容限用于在长悬线延伸时判断悬线是否延伸。单位与进行拓扑处理的数据集单位相同。
        :param float vertex\_tolerance: 节点容限。该容限用于邻近端点合并、弧段求交、去除假结点和去除冗余点。单位与进行拓扑处理的数据集单位相同。
        :param Recordset filter\_vertex\_recordset: 弧段求交的过滤点记录集,即此记录集中的点位置线段不进行求交打断。
        :param str arc\_filter\_string: 弧段求交的过滤线表达式。 在进行弧段求交时,通过该属性可以指定一个字段表达式,符合该表达式的线对象将不被打断。
                                      该表达式是否有效,与 filter\_mode 弧段求交过滤模式有关
        :param filter\_mode: 弧段求交的过滤模式。
        :type filter\_mode: ArcAndVertexFilterMode or str
        :param options: 拓扑处理参数类,如果 options 不为空,拓扑处理将会使用此参数设置的值。
        :type options: ProcessingOptions or None
        :param function progress: 进度信息处理函数,具体参考 :py:class:`.StepEvent`
        :return: 是否拓扑处理成功
        :rtype: bool
    
    topology\_validate(source\_data, validating\_data, rule, tolerance, validate\_region=None, out\_data=None, out\_dataset\_name=None, progress=None)
        对数据集或记录集进行拓扑错误检查,返回含有拓扑错误的结果数据集。
        
        该方法的 tolerance 参数用于指定使用 rule 参数指定的拓扑规则对数据集检查时涉及的容限。例如,使用“线内无打折”(TopologyRule.LINE\_NO\_SHARP\_ANGLE)规则检查时,tolerance 参数设置的为尖角容限(一个角度值)。
        
        对于以下拓扑检查算子在调用该方法对数据进行拓扑检查之前,建议先对相应的数据进行拓扑预处理(即调用 :py:meth:`preprocess` 方法),否则检查的结果可能不正确:
        
            - REGION\_NO\_OVERLAP\_WITH
            - REGION\_COVERED\_BY\_REGION\_CLASS
            - REGION\_COVERED\_BY\_REGION
            - REGION\_BOUNDARY\_COVERED\_BY\_LINE
            - REGION\_BOUNDARY\_COVERED\_BY\_REGION\_BOUNDARY
            - REGION\_NO\_OVERLAP\_ON\_BOUNDARY
            - REGION\_CONTAIN\_POINT
            - LINE\_NO\_OVERLAP\_WITH
            - LINE\_BE\_COVERED\_BY\_LINE\_CLASS
            - LINE\_END\_POINT\_COVERED\_BY\_POINT
            - POINT\_NO\_CONTAINED\_BY\_REGION
            - POINT\_COVERED\_BY\_LINE
            - POINT\_COVERED\_BY\_REGION\_BOUNDARY
            - POINT\_CONTAINED\_BY\_REGION
            - POINT\_BECOVERED\_BY\_LINE\_END\_POINT
        
        对于以下拓扑检查算法需要设置参考数据集或记录集:
        
            - REGION\_NO\_OVERLAP\_WITH
            - REGION\_COVERED\_BY\_REGION\_CLASS
            - REGION\_COVERED\_BY\_REGION
            - REGION\_BOUNDARY\_COVERED\_BY\_LINE
            - REGION\_BOUNDARY\_COVERED\_BY\_REGION\_BOUNDARY
            - REGION\_CONTAIN\_POINT
            - REGION\_NO\_OVERLAP\_ON\_BOUNDARY
            - POINT\_BECOVERED\_BY\_LINE\_END\_POINT
            - POINT\_NO\_CONTAINED\_BY\_REGION
            - POINT\_CONTAINED\_BY\_REGION
            - POINT\_COVERED\_BY\_LINE
            - POINT\_COVERED\_BY\_REGION\_BOUNDARY
            - LINE\_NO\_OVERLAP\_WITH
            - LINE\_NO\_INTERSECT\_OR\_INTERIOR\_TOUCH
            - LINE\_BE\_COVERED\_BY\_LINE\_CLASS
            - LINE\_NO\_INTERSECTION\_WITH
            - LINE\_NO\_INTERSECTION\_WITH\_REGION
            - LINE\_EXIST\_INTERSECT\_VERTEX
            - VERTEX\_DISTANCE\_GREATER\_THAN\_TOLERANCE
            - VERTEX\_MATCH\_WITH\_EACH\_OTHER
        
        
        :param source\_data: 被检查的数据集或记录集
        :type source\_data: DatasetVector or str or Recordset
        :param validating\_data:  用于检查的参考记录集。如果使用的拓扑规则不需要参考记录集,则设置为 None
        :type validating\_data: DatasetVector or str or Recordset
        :param rule: 拓扑检查类型
        :type rule: TopologyRule or str
        :param float tolerance:   指定的拓扑错误检查时使用的容限。单位与进行拓扑错误检查的数据集单位相同。
        :param GeoRegion validate\_region: 被检查区域,None,则默认对整个拓扑数据集(validating\_data)进行检查,否则对 validate\_region 区域进行拓扑检查。
        :param out\_data: 结果数据集所在的数据源
        :type out\_data: Datasource or DatasourceConnectionInfo or str
        :param str out\_dataset\_name: 结果数据集名称
        :param function progress: 进度信息处理函数,具体参考 :py:class:`.StepEvent`
        :return: 结果数据集或数据集名称
        :rtype: DatasetVector or str
    
    update\_attributes(source\_data, target\_data, spatial\_relation, update\_fields, interval=1e-06)
        矢量数据集属性更新,将 source\_data 中的属性,根据 spatial\_relation 指定的空间关系,更新到 target\_data 数据集中。
        例如,有一份点数据和面数据,需要将点数据集中的属性值取平均值,然后将值写到包含点的面对象中,可以通过以下代码实现::
        
        >>> result = update\_attributes('ds/points', 'ds/zones', 'WITHIN', [('trip\_distance', 'mean'), ('', 'count')])
        
        spatial\_relation 参数是指源数据集( source\_data)对目标被更新数据集(target\_data)的空间关系。
        
        :param source\_data: 源数据集。源数据集提供属性数据,将源数据集中的属性值根据空间关系更新到目标数据集中。
        :type source\_data:  DatasetVector or str
        :param target\_data: 目标数据集。被写入属性数据的数据集。
        :type target\_data: DatasetVector or str
        :param spatial\_relation: 空间关系类型,源数据(查询对象)对目标数据(被查询对象)的空间关系,具体参考 :py:class:`SpatialQueryMode`
        :type spatial\_relation: SpatialQueryMode or str
        :param update\_fields: 字段统计信息,可能有多个源数据中对象与目标数据对象满足空间关系,需要对源数据的属性字段值进行汇总统计,将统计的结果写入到目标数据集中
                              为一个list,list中每个元素为一个 tuple,tuple的大小为2,tuple的第一个元素为被统计的字段名称,tuple的第二个元素为统计类型。
        :type update\_fields: list[tuple(str,AttributeStatisticsMode)] or list[tuple(str,str)] or str
        :param interval: 节点容限
        :type interval: float
        :return: 是否属性更新成功。更新成功返回 True,否则返回 False。
        :rtype: bool
    
    vector\_to\_raster(input\_data, value\_field, clip\_region=None, cell\_size=None, pixel\_format=PixelFormat.SINGLE, out\_data=None, out\_dataset\_name=None, progress=None)
        通过指定转换参数设置将矢量数据集转换为栅格数据集。
        
        :param input\_data: 待转换的矢量数据集。支持点、线和面数据集
        :type input\_data: DatasetVector or str
        :param str value\_field: 矢量数据集中存储栅格值的字段
        :param clip\_region: 转换的有效区域
        :type clip\_region: GeoRegion or Rectangle
        :param float cell\_size: 结果栅格数据集的单元格大小
        :param pixel\_format: 如果将矢量数据转为像素格式 为 UBIT1、UBIT4 和 UBIT8 的栅格数据集,矢量数据中值为 0 的对象在结果栅格中会丢失。
        :type pixel\_format: PixelFormat or str
        :param out\_data: 结果数据集所在的数据源
        :type out\_data: Datasource or DatasourceConnectionInfo or str
        :param str out\_dataset\_name: 结果数据集名称
        :param function progress: 进度信息处理函数,具体参考 :py:class:`.StepEvent`
        :return: 结果数据集或数据集名称
        :rtype: DatasetGrid or str
    
    watershed(direction\_grid, pour\_points\_or\_grid, out\_data=None, out\_dataset\_name=None, progress=None)
        流域分割,即生成指定汇水点(汇水点栅格数据集)的流域盆地。
        
        将一个流域划分为若干个子流域的过程称为流域分割。通过 :py:meth:`basin` 方法,可以获取较大的流域,但实际分析中,可能需要将较大的流域划
        分出更小的流域(称为子流域)。
        
        确定流域的第一步是确定该流域的汇水点,那么,流域分割同样首先要确定子流域的汇水点。与使用 basin 方法计算流域盆地不同,子流
        域的汇水点可以在栅格的边界上,也可能位于栅格的内部。该方法要求输入一个汇水点栅格数据,该数据可通过提取汇水点功能( :py:meth:`pour\_points` 方法)
        获得。此外,还可以使用另一个重载方法,输入表示汇水点的二维点集合来分割流域。
        
        水文分析的相关介绍,请参考 :py:func:`basin`
        
        :param direction\_grid: 流向栅格数据
        :type direction\_grid: DatasetGrid or str
        :param pour\_points\_or\_grid: 汇水点栅格数据或指定的汇水点(二维点列表),汇水点使用地理坐标单位。
        :type pour\_points\_or\_grid: DatasetGrid or str or list[Point2D]
        :param out\_data: 用于存储结果数据集的数据源
        :type out\_data: DatasourceConnectionInfo or Datasource or str
        :param str out\_dataset\_name: 结果汇水点的流域盆地栅格数据集名称
        :param progress: 进度信息处理函数,具体参考 :py:class:`.StepEvent`
        :type progress: function
        :return: 汇水点的流域盆地栅格数据集或数据集名称
        :rtype: DatasetGrid or  str
    
    weight\_matrix\_file\_to\_table(file\_path, out\_data, out\_dataset\_name=None, progress=None)
        空间权重矩阵文件转换成属性表。
        
        结果属性表包含源唯一ID字段(UniqueID)、相邻要素唯一ID字段(NeighborsID)、权重字段(Weight)。
        
        :param str file\_path: 空间权重矩阵文件路径。
        :param out\_data: 用于存储结果属性表的数据源
        :type out\_data: Datasource or DatasourceConnectionInfo or str
        :param str out\_dataset\_name: 结果属性表名称
        :param progress: 进度信息,具体参考 :py:class:`.StepEvent`
        :type progress: function
        :return: 结果属性表数据集或数据集名称。
        :rtype: DatasetVector or str
    
    zonal\_statistics\_on\_raster\_value(value\_data, zonal\_data, zonal\_field, is\_ignore\_no\_value=True, grid\_stat\_mode='SUM', out\_data=None, out\_dataset\_name=None, out\_table\_name=None, progress=None)
        栅格分带统计,方法中值数据为栅格的数据集,带数据可以是矢量或栅格数据。
        
        栅格分带统计,是以某种统计方法对区域内的单元格的值进行统计,将每个区域内的统计值赋给该区域所覆盖的所有单元格,从而得到结果栅格。栅格分带统计涉及两种数据,值数据和带数据。值数据即被统计的栅格数据,带数据为标识统计区域的数据,可以为栅格或矢量面数据。下图为使用栅格带数据进行分带统计的算法示意,其中灰色单元格代表无值数据。
        
        .. image:: ../image/ZonalStatisticsOnRasterValue\_1.png
        
        当带数据为栅格数据集时,连续的栅格值相同的单元格作为一个带(区域);当带数据为矢量面数据集时,要求其属性表中有一个标识带的字段,以数值来区分不同的带,如果两个及以上的面对象(可以相邻,也可以不相邻)的标识值相同,则进行分带统计时,它们将作为一个带进行统计,即在结果栅格中,这些面对象对应位置的栅格值都是这些面对象范围内的所有单元格的栅格值的统计值。
        
        分带统计的结果包含两部分:一是分带统计结果栅格,每个带内的栅格值相同,即按照统计方法计算所得的值;二是一个记录了每个分带内统计信息的属性表,包含 ZONALID(带的标识)、PIXELCOUNT(带内单元格数)、MININUM(最小值)、MAXIMUM(最大值)、RANGE\_VALUE(值域)、SUM\_VALUE(和)、MEAN(平均值)、STD(标准差)、VARIETY(种类)、MAJORITY(众数)、MINORITY(少数)、MEDIAN(中位数)等字段。
        
        下面通过一个实例来了解分带统计的应用。
        
          1. 如下图所示,左图是 DEM 栅格值,将其作为值数据,右图为对应区域的行政区划,将其作为带数据,进行分带统计;
        
          .. image:: ../image/ZonalStatisticsOnRasterValue\_2.png
        
          2. 使用上面的数据,将最大值作为统计方法,进行分带统计。结果包括如下图所示的结果栅格,以及对应的统计信息属性表(略)。结果栅格中,每个带内的栅格值均相等,即在该带范围内的值栅格中最大的栅格值,也就是高程值。该例统计了该地区每个行政区内最高的高程。
        
          .. image:: ../image/ZonalStatisticsOnRasterValue\_3.png
        
        
        注意,分带统计的结果栅格的像素类型(PixelFormat)与指定的分带统计类型(通过 ZonalStatisticsAnalystParameter 类的 setStatisticsMode 方法设置)有关:
        
         * 当统计类型为种类(VARIETY)时,结果栅格像素类型为 BIT32;
         * 当统计类型为最大值(MAX)、最小值(MIN)、值域(RANGE)时,结果栅格的像素类型与源栅格保持一致;
         * 当统计类型为平均值(MEAN)、标准差(STDEV)、总和(SUM)、众数(MAJORITY)、最少数(MINORITY)、中位数(MEDIAN)时,结果栅格的像素类型为 DOUBLE。
        
        :param value\_data: 需要被统计的值数据
        :type value\_data: DatasetGrid or str
        :param zonal\_data: 待统计的分带数据集。仅支持像素格式(PixelFormat)为 UBIT1、UBIT4、UBIT8 和 UBIT16 的栅格数据集或矢量面数据集。
        :type zonal\_data: DatasetGrid or DatasetVector or str
        :param str zonal\_field: 矢量分带数据中用于标识带的字段。字段类型只支持32位整型。
        :param bool is\_ignore\_no\_value: 统计时是否忽略无值数据。 如果为 True,表示无值栅格不参与运算;若为 False,表示有无值参与的运算,结果仍为无值
        :param grid\_stat\_mode: 分带统计类型
        :type grid\_stat\_mode:  GridStatisticsMode or str
        :param out\_data: 用于存储结果数据的数据源。
        :type out\_data: Datasource or DatasourceConnectionInfo or str
        :param out\_dataset\_name: 结果数据集的名称
        :type out\_dataset\_name: str
        :param out\_table\_name: 分析结果属性表的名称
        :type out\_table\_name: str
        :param progress: 进度信息处理函数,具体参考 :py:class:`.StepEvent`
        :type progress: function
        :return: 返回一个 tuple,tuple 有两个元素,第一个为结果数据集或名称,第二个为结果属性表数据集或名称
        :rtype: tuple[DatasetGrid, DatasetGrid] or tuple[str,str]

DATA
    \_\_all\_\_ = ['create\_buffer', 'overlay', 'dissolve', 'aggregate\_points',{\ldots}

FILE
    /opt/conda/lib/python3.6/site-packages/iobjectspy/analyst.py



    \end{Verbatim}

    \begin{Verbatim}[commandchars=\\\{\}]
{\color{incolor}In [{\color{incolor}9}]:} \PY{n}{help}\PY{p}{(}\PY{n}{smo}\PY{o}{.}\PY{n}{data}\PY{p}{)}
\end{Verbatim}


    \begin{Verbatim}[commandchars=\\\{\}]
Help on module iobjectspy.data in iobjectspy:

NAME
    iobjectspy.data

CLASSES
    builtins.Exception(builtins.BaseException)
        iobjectspy.\_jsuperpy.data.ex.DatasourceCreatedFailedError
        iobjectspy.\_jsuperpy.data.ex.DatasourceOpenedFailedError
        iobjectspy.\_jsuperpy.data.ex.DatasourceReadOnlyError
    builtins.RuntimeError(builtins.Exception)
        iobjectspy.\_jsuperpy.data.ex.ObjectDisposedError
    builtins.object
        iobjectspy.\_jsuperpy.data.dt.Colors
        iobjectspy.\_jsuperpy.data.dt.DatasetGridInfo
        iobjectspy.\_jsuperpy.data.dt.DatasetImageInfo
        iobjectspy.\_jsuperpy.data.dt.DatasetVectorInfo
        iobjectspy.\_jsuperpy.data.dt.JoinItem
        iobjectspy.\_jsuperpy.data.dt.LinkItem
        iobjectspy.\_jsuperpy.data.dt.QueryParameter
        iobjectspy.\_jsuperpy.data.dt.SpatialIndexInfo
        iobjectspy.\_jsuperpy.data.dt.TimeCondition
        iobjectspy.\_jsuperpy.data.geo.Feature
        iobjectspy.\_jsuperpy.data.geo.FieldInfo
        iobjectspy.\_jsuperpy.data.geo.Point2D
        iobjectspy.\_jsuperpy.data.geo.Point3D
        iobjectspy.\_jsuperpy.data.geo.Rectangle
        iobjectspy.\_jsuperpy.data.geo.TextPart
        iobjectspy.\_jsuperpy.data.geo.TextStyle
        iobjectspy.\_jsuperpy.data.prj.CoordSysTranslator
        iobjectspy.\_jsuperpy.data.step.StepEvent
    iobjectspy.\_jsuperpy.data.\_jvm.JVMBase(builtins.object)
        iobjectspy.\_jsuperpy.data.ds.Datasource
        iobjectspy.\_jsuperpy.data.ds.DatasourceConnectionInfo
        iobjectspy.\_jsuperpy.data.dt.Dataset
            iobjectspy.\_jsuperpy.data.dt.DatasetGrid
            iobjectspy.\_jsuperpy.data.dt.DatasetImage
            iobjectspy.\_jsuperpy.data.dt.DatasetTopology
            iobjectspy.\_jsuperpy.data.dt.DatasetVector
            iobjectspy.\_jsuperpy.data.dt.DatasetVolume
        iobjectspy.\_jsuperpy.data.dt.Recordset
        iobjectspy.\_jsuperpy.data.geo.Geometry
            iobjectspy.\_jsuperpy.data.geo.GeoLine
            iobjectspy.\_jsuperpy.data.geo.GeoPoint
            iobjectspy.\_jsuperpy.data.geo.GeoPoint3D
            iobjectspy.\_jsuperpy.data.geo.GeoRegion
            iobjectspy.\_jsuperpy.data.geo.GeoText
        iobjectspy.\_jsuperpy.data.op.GeometriesRelation
        iobjectspy.\_jsuperpy.data.prj.CoordSysTransParameter
        iobjectspy.\_jsuperpy.data.prj.GeoCoordSys
        iobjectspy.\_jsuperpy.data.prj.GeoDatum
        iobjectspy.\_jsuperpy.data.prj.GeoPrimeMeridian
        iobjectspy.\_jsuperpy.data.prj.GeoSpheroid
        iobjectspy.\_jsuperpy.data.prj.PrjCoordSys
        iobjectspy.\_jsuperpy.data.prj.PrjParameter
        iobjectspy.\_jsuperpy.data.prj.Projection
        iobjectspy.\_jsuperpy.data.ws.Workspace
        iobjectspy.\_jsuperpy.data.ws.WorkspaceConnectionInfo
    
    class Colors(builtins.object)
     |  颜色集合类。该类主要作用是提供颜色序列。提供各种渐变色和随机色的生成,以及 SuperMap 预定义渐变色的生成。
     |  
     |  Methods defined here:
     |  
     |  \_\_delitem\_\_(self, key)
     |  
     |  \_\_getitem\_\_(self, key)
     |  
     |  \_\_init\_\_(self, seq=None)
     |      Initialize self.  See help(type(self)) for accurate signature.
     |  
     |  \_\_iter\_\_(self)
     |  
     |  \_\_len\_\_(self)
     |  
     |  \_\_setitem\_\_(self, key, value)
     |  
     |  \_\_str\_\_(self)
     |      Return str(self).
     |  
     |  append(self, value)
     |      添加一个颜色值到颜色集合中
     |      
     |      :param value: RGB 颜色值或 RGBA 颜色值
     |      :type value: tuple[int] or int
     |  
     |  clear(self)
     |      清空所有颜色值
     |  
     |  extend(self, iterable)
     |      添加一个颜色值的集合
     |      
     |      :param iterable: 颜色值集合
     |      :type iterable: range[int] or range[tuple]
     |  
     |  index(self, value, start=None, end=None)
     |      返回颜色值的序号
     |      
     |      :param value: RGB 颜色值或 RGBA 颜色值
     |      :type value: tuple[int] or int
     |      :param int start: 开始查找位置
     |      :param int end: 终止查找位置
     |      :return: 满足条件的颜色值所在的位置
     |      :rtype: int
     |  
     |  insert(self, index, value)
     |      添加颜色到指定的位置
     |      
     |      :param int  index: 指定的位置
     |      :param value:  RGB 颜色值或 RGBA 颜色值
     |      :type value: tuple[int] or int
     |  
     |  pop(self, index=None)
     |      删除指定位置的颜色值,并返回颜色值。当 index 为 None 时删除最后一个颜色值
     |      
     |      :param int index: 指定的位置
     |      :return: 被删除的颜色值
     |      :rtype: tuple
     |  
     |  remove(self, value)
     |      删除指定的颜色值
     |      
     |      :param value: 被删除的颜色值
     |      :type value: tuple[int] or int
     |  
     |  values(self)
     |      返回所有的颜色值
     |      
     |      
     |      :rtype: list[tuple]
     |  
     |  ----------------------------------------------------------------------
     |  Static methods defined here:
     |  
     |  make\_gradient(count, gradient\_type, reverse=False, gradient\_colors=None)
     |      给定颜色的数量和控制颜色生成一组渐变色,或生成生成系统预定义渐变色。gradient\_colors 和 gradient\_type 不能同时有效,但 gradient\_type 有效时
     |      会优先使用 gradient\_type 生成系统预定义的渐变色。
     |      
     |      :param int count: 要生成的渐变色的颜色总数。
     |      :param gradient\_type: 渐变颜色的类型。
     |      :type gradient\_type:  ColorGradientType or str
     |      :param bool reverse: 是否反向生成渐变色,即是否从终止色到起始色生成渐变色。仅对 gradient\_type 有效时起作用。
     |      :param Colors gradient\_colors: 渐变颜色集。即生成渐变色的控制颜色。
     |      :return:
     |      :rtype:
     |  
     |  make\_random(count, colors=None)
     |      用于生成一定数量的随机颜色。
     |      
     |      :param int count: 间隔色个数
     |      :param Colors colors: 控制色集合。
     |      :return: 由间隔色个数和控制色集合生成的随机颜色表。
     |      :rtype: Colors
     |  
     |  ----------------------------------------------------------------------
     |  Data descriptors defined here:
     |  
     |  \_\_dict\_\_
     |      dictionary for instance variables (if defined)
     |  
     |  \_\_weakref\_\_
     |      list of weak references to the object (if defined)
    
    class CoordSysTransParameter(iobjectspy.\_jsuperpy.data.\_jvm.JVMBase)
     |  投影转换参照系转换参数类,通常包括平移、旋转和比例因子。
     |  
     |  在进行投影转换时,如果源投影和目标投影的地理坐标系不同,则需要进行参照系转换。SuperMap 提供常用的六种参照系转换方法,详见 CoordSysTransMethod
     |  方法。不同的参照系转换方法需要指定不同的转换参数:
     |  
     |  - 三参数转换法(GeocentricTranslation)、莫洛金斯基转换法(Molodensky)、简化的莫洛金斯基转换法(MolodenskyAbridged)属于精度较低的
     |    几种转换方法,在数据精度要求不高的情况下一般可以采用这几种方法。这三种转换法需要给定三个平移转换参数:X 轴坐标偏移量(set\_translate\_x)、Y轴
     |    坐标偏移量(set\_translate\_y)和 Z 轴坐标偏移量(set\_translate\_z)。
     |  - 位置矢量法(PositionVector)、基于地心的七参数转换法(CoordinateFrame)、布尔莎方法(BursaWolf)属于精度较高的几种转换方法。需要七个
     |    参数来进行调整和转换,包括除上述的三个平移转换参数外,还需要设置三个旋转转换参数(X 轴旋转角度(set\_rotate\_x)、Y 轴旋转角度(set\_rotate\_y)
     |    和 Z 轴旋转角度(set\_rotate\_z))和投影比例尺差参数(set\_scale\_difference)。
     |  
     |  Method resolution order:
     |      CoordSysTransParameter
     |      iobjectspy.\_jsuperpy.data.\_jvm.JVMBase
     |      builtins.object
     |  
     |  Methods defined here:
     |  
     |  \_\_getstate\_\_(self)
     |  
     |  \_\_init\_\_(self)
     |      Initialize self.  See help(type(self)) for accurate signature.
     |  
     |  \_\_setstate\_\_(self, state)
     |  
     |  clone(self)
     |      复制对象
     |      
     |      :rtype: CoordSysTransParameter
     |  
     |  from\_xml(self, xml)
     |      根据 XML 字符串构建 CoordSysTransParameter 对象,成功返回 True
     |      
     |      :param str xml:
     |      :rtype: bool
     |  
     |  set\_rotate\_x(self, value)
     |      设置 X 轴的旋转角度。用于不同大地参照系之间的转换。单位为弧度。
     |      
     |      :param float value: X 轴的旋转角度
     |      :return: self
     |      :rtype: CoordSysTransParameter
     |  
     |  set\_rotate\_y(self, value)
     |      设置 Y 轴的旋转角度。用于不同大地参照系之间的转换。单位为弧度。
     |      
     |      :param float value: Y 轴的旋转角度
     |      :return: self
     |      :rtype: CoordSysTransParameter
     |  
     |  set\_rotate\_z(self, value)
     |      设置 Z 轴的旋转角度。用于不同大地参照系之间的转换。单位为弧度。
     |      
     |      :param float value: Z 轴的旋转角度
     |      :return: self
     |      :rtype: CoordSysTransParameter
     |  
     |  set\_rotation\_origin\_x(self, value)
     |      设置旋转原点的 X 坐标的量
     |      
     |      :param float value: 旋转原点的 X 坐标的量
     |      :return: self
     |      :rtype: CoordSysTransParameter
     |  
     |  set\_rotation\_origin\_y(self, value)
     |      设置旋转原点的 Y 坐标的量
     |      
     |      :param float value: 旋转原点的 Y 坐标的量
     |      :return: self
     |      :rtype: CoordSysTransParameter
     |  
     |  set\_rotation\_origin\_z(self, value)
     |      设置旋转原点的 Z 坐标的量
     |      
     |      :param float value: 旋转原点的 Z 坐标的量
     |      :return: self
     |      :rtype: CoordSysTransParameter
     |  
     |  set\_scale\_difference(self, value)
     |      设置投影比例尺差。单位为百万分之一。用于不同大地参照系之间的转换
     |      
     |      :param float value: 投影比例尺差
     |      :return: self
     |      :rtype: CoordSysTransParameter
     |  
     |  set\_translate\_x(self, value)
     |      设置 X 轴的坐标偏移量。单位为米
     |      
     |      :param float value: X 轴的坐标偏移量
     |      :return: self
     |      :rtype: CoordSysTransParameter
     |  
     |  set\_translate\_y(self, value)
     |      设置 Y 轴的坐标偏移量。单位为米
     |      
     |      :param float value: Y 轴的坐标偏移量
     |      :return: self
     |      :rtype: CoordSysTransParameter
     |  
     |  set\_translate\_z(self, value)
     |      设置 Z 轴的坐标偏移量。单位为米
     |      
     |      :param float value:  Z 轴的坐标偏移量
     |      :return: self
     |      :rtype: CoordSysTransParameter
     |  
     |  to\_xml(self)
     |      将该 CoordSysTransParameter 对象输出为 XML 字符串。
     |      
     |      :rtype: str
     |  
     |  ----------------------------------------------------------------------
     |  Data descriptors defined here:
     |  
     |  rotate\_x
     |      float: X 轴的旋转角度
     |  
     |  rotate\_y
     |      float: Y 轴的旋转角度
     |  
     |  rotate\_z
     |      float: Z 轴的旋转角度
     |  
     |  rotation\_origin\_x
     |      float: 旋转原点的X坐标
     |  
     |  rotation\_origin\_y
     |      float:  旋转原点的 Y 坐标的量
     |  
     |  rotation\_origin\_z
     |      float: 旋转原点的Z坐标的量
     |  
     |  scale\_difference
     |      float: 投影比例尺差。单位为百万分之一。用于不同大地参照系之间的转换
     |  
     |  translate\_x
     |      float: 返回 X 轴的坐标偏移量。单位为米
     |  
     |  translate\_y
     |      float: 返回 Y 轴的坐标偏移量。单位为米
     |  
     |  translate\_z
     |      float: 返回 Z 轴的坐标偏移量。单位为米
     |  
     |  ----------------------------------------------------------------------
     |  Data descriptors inherited from iobjectspy.\_jsuperpy.data.\_jvm.JVMBase:
     |  
     |  \_\_dict\_\_
     |      dictionary for instance variables (if defined)
     |  
     |  \_\_weakref\_\_
     |      list of weak references to the object (if defined)
    
    class CoordSysTranslator(builtins.object)
     |  投影转换类。主要用于投影坐标之间及投影坐标系之间的转换。
     |  
     |  投影转换一般有三种工作方式:地理(经纬度)坐标和投影坐标之间的转换使用 forward() 方法、投影坐标和地理(经纬度)坐标之间的转换使用inverse() 方法 、
     |  两种投影坐标系之间的转换使用convert() 方法。
     |  
     |  注意:当前版本不支持光栅数据的投影转换。即在同一数据源中,投影转换只转换矢量数据部分。地理坐标系(Geographic coordinate system)也称为地理
     |  坐标系统,是以经纬度为地图的存储单位的。很明显,地理坐标系是球面坐标系统。如果将地球上的数字化信息存放到球面坐标系统上,就需要有这样的椭球体具有
     |  如下特点:可以量化计算的,具有长半轴(Semimajor Axis),短半轴(Semiminor Axis),偏心率(Flattening),中央子午线(prime meridian)及大地基准面(datum)。
     |  
     |  投影坐标系统(Projection coordinate system)实质上便是平面坐标系统,其地图单位通常为米。将球面坐标转化为平面坐标的过程便称为投影。所以每一个
     |  投影坐标系统都必定会有地理坐标系统(Geographic Coordinate System)参数。 因此就存在着投影坐标之间的转换以及投影坐标系之间的转换。
     |  
     |  在进行投影转换时,对文本对象(GeoText)投影转换后,文本对象的字高和角度会相应地转换,如果用户不需要这样的改变,需要对转换后的文本对象修正其字高和角度。
     |  
     |  Static methods defined here:
     |  
     |  convert(source\_data, target\_prj\_coordsys, coordsys\_trans\_parameter, coord\_sys\_trans\_method, source\_prj\_coordsys=None, out\_data=None, out\_dataset\_name=None)
     |      根据源投影坐标系与目标投影坐标系对输入数据进行投影转换。根据是否设置了有效的结果数据源信息,可以直接修改源数据或者将转换后的结果数据存储到结果数据源中。
     |      
     |      :param source\_data: 被转换的数据。直接将数据集、几何对象,二维点序列和几何对象序列进行转换。
     |      :type source\_data: DatasetVector or Geometry or list[Point2D] or list[Geometry]
     |      :param PrjCoordSys target\_prj\_coordsys: 目标投影坐标系对象。
     |      :param CoordSysTransParameter coordsys\_trans\_parameter:
     |      :type coordsys\_trans\_parameter: 投影坐标系转换参数。包括坐标的平移量、旋转角度、投影比例尺差,详情请参见 :py:class:`CoordSysTransParameter` 类。
     |      :param coord\_sys\_trans\_method:  投影转换的方法。详情参见 :py:class:`CoordSysTransMethod`。在进行投影转换时,如果源投影和目标投影的地理坐标系相同,该参数的设置不起作用。
     |      :type coord\_sys\_trans\_method: CoordSysTransMethod
     |      :param PrjCoordSys source\_prj\_coordsys: 源投影坐标系对象。当被转换的数据为数据集对象时,此参数无效,会使用数据集的投影坐标系信息。
     |      :param out\_data: 结果数据源。当结果数据源有效时,会将转换后的结果存储到新的结果数据集中,否则将直接修改原数据的点坐标。
     |      :type out\_data: Datasource or DatasourceConnectionInfo or str
     |      :param out\_dataset\_name: 结果数据集名称。当 out\_datasource 有效时才起作用
     |      :type out\_dataset\_name: str
     |      :return: 根据是否设置了结果数据源对象:
     |      
     |              - 如果设置了结果数据源对象,转换成功,会把转换后的结果写到结果数据集中并返回结果数据集名称或结果数据集对象,如果转换失败,返回 None。
     |              - 如果没设置结果数据源对象,转换成功,会直接修改输入的源数据的点坐标并返回 True,否则返回 False。
     |      
     |      :rtype: DatasetVector or str bool
     |  
     |  forward(data, prj\_coordsys, out\_data=None, out\_dataset\_name=None)
     |      在同一地理坐标系下,该方法用于将指定的 Point2D 列表中的点二维点对象从地理坐标转换到投影坐标
     |      
     |      :param data: 被转换的二维点列表
     |      :type data: list[Point2D] or tuple[Point2D]
     |      :param prj\_coordsys: 二维点对象所在的投影坐标系
     |      :type prj\_coordsys: PrjCoordSys
     |      :param out\_data: 结果数据源对象,可以选择将转换后得到的点保存到数据源中。如果为空,将返回转换后得到点的列表
     |      :type out\_data: Datasource or DatasourceConnectionInfo or str
     |      :param out\_dataset\_name: 结果数据集名称,out\_datasource 有效时才起作用
     |      :type out\_dataset\_name: str
     |      :return: 转换失败返回None,转换成功,如果设置了有效的 out\_datasource,返回结果数据集或数据集名称,否则,返回转换后得到的点的列表。
     |      :rtype: DatasetVector or str or list[Point2D]
     |  
     |  inverse(data, prj\_coordsys, out\_data=None, out\_dataset\_name=None)
     |      在同一投影坐标系下,该方法用于将指定的 Point2D 列表中的二维点对象从投影坐标转换到地理坐标。
     |      
     |      :param data: 被转换的二维点列表
     |      :type data: list[Point2D] or tuple[Point2D]
     |      :param prj\_coordsys: 二维点对象所在的投影坐标系
     |      :type prj\_coordsys: PrjCoordSys
     |      :param out\_data: 结果数据源对象,可以选择将转换后得到的点保存到数据源中。如果为空,将返回转换后得到点的列表
     |      :type out\_data: Datasource or DatasourceConnectionInfo or str
     |      :param out\_dataset\_name: 结果数据集名称,out\_datasource 有效时才起作用
     |      :type out\_dataset\_name: str
     |      :return: 转换失败返回None,转换成功,如果设置了有效的 out\_datasource,返回结果数据集或数据集名称,否则,返回转换后得到的点的列表。
     |      :rtype: DatasetVector or str or list[Point2D]
     |  
     |  ----------------------------------------------------------------------
     |  Data descriptors defined here:
     |  
     |  \_\_dict\_\_
     |      dictionary for instance variables (if defined)
     |  
     |  \_\_weakref\_\_
     |      list of weak references to the object (if defined)
    
    class Dataset(iobjectspy.\_jsuperpy.data.\_jvm.JVMBase)
     |  数据集(矢量数据集,栅格数据集,影像数据集等)的基类,提供各种数据集数据集公共的属性,方法。
     |  数据集一般为存储在一起的相关数据的集合;根据数据类型的不同,分为矢量数据集和栅格数据集,以及为了处理特定问题而设计的如拓扑数据集,网络数据集
     |  等。数据集是 GIS 数据组织的最小单位。其中矢量数据集是由同种类型空间要素组成的集合,所以也可以称为要素集。根据要素的空间特征的不同,矢量数据集
     |  又分为点数据集,线数据集,面数据集等,各矢量数据集是空间特征和性质相同而组织在一起的数据的集合。而栅格数据集由像元阵列组成,在表现要素上比矢量
     |  数据集欠缺,但是可以很好的表现空间现象的位置关系。
     |  
     |  Method resolution order:
     |      Dataset
     |      iobjectspy.\_jsuperpy.data.\_jvm.JVMBase
     |      builtins.object
     |  
     |  Methods defined here:
     |  
     |  \_\_init\_\_(self)
     |      Initialize self.  See help(type(self)) for accurate signature.
     |  
     |  \_\_repr\_\_(self)
     |      Return repr(self).
     |  
     |  \_\_str\_\_(self)
     |      Return str(self).
     |  
     |  close(self)
     |      用于关闭当前数据集
     |  
     |  is\_open(self)
     |      判断数据集是否已经打开,数据集打开返回 True,否则返回 False
     |      
     |      :rtype: bool
     |  
     |  is\_readonly(self)
     |      判断数据集是否是只读。如果数据集是只读,无法进行任何改写数据集的操作。 数据集是只读返回 True,否则返回 False。
     |      
     |      :rtype: bool
     |  
     |  open(self)
     |      打开数据集, 打开数据集成功返返回 True,否则返回 False
     |      
     |      :rtype: bool
     |  
     |  rename(self, new\_name)
     |      修改数据集名称
     |      
     |      :param str new\_name: 新的数据集名称
     |      :return: 修改成功返回True,否则返回False
     |      :rtype: bool
     |  
     |  set\_bounds(self, rc)
     |      设置数据集中包含所有对象的最小外接矩形。对于矢量数据集来说,为数据集中所有对象的最小外接矩形;对于栅格数据集来说,为
     |      当前栅格或影像的地理范围。
     |      
     |      :param Rectangle rc: 数据集中包含所有对象的最小外接矩形。
     |      :return: self
     |      :rtype: Dataset
     |  
     |  set\_description(self, value)
     |      设置用户加入的对数据集的描述信息。
     |      
     |      :param str value: 用户加入的对数据集的描述信息。
     |  
     |  set\_prj\_coordsys(self, value)
     |      设置数据集的投影信息
     |      
     |      :param PrjCoordSys value: 投影信息
     |  
     |  to\_json(self)
     |      输出数据集的信息到 json 字符串中。数据集的 json 串内容包括数据源的连接信息和数据集名称两项。
     |      
     |      :rtype: str
     |      
     |      示例::
     |       >>> ds = Workspace().get\_datasource('data')
     |       >>> print(ds[0].to\_json())
     |       \{"name": "location", "datasource": \{"type": "UDB", "alias": "data", "server": "E:/data.udb", "is\_readonly": false\}\}
     |  
     |  ----------------------------------------------------------------------
     |  Static methods defined here:
     |  
     |  from\_json(value)
     |      从数据集的 json 字符串中获取数据集,如果数据源没有打开,将自动打开数据源。
     |      
     |      :param str value:  json 字符串
     |      :return: 数据集对象
     |      :rtype: Dataet
     |  
     |  ----------------------------------------------------------------------
     |  Data descriptors defined here:
     |  
     |  bounds
     |      Rectangle: 返回数据集中包含所有对象的最小外接矩形。对于矢量数据集来说,为数据集中所有对象的最小外接矩形;对于栅格数据集来说,为当前栅格或影像的
     |      地理范围。
     |  
     |  datasource
     |      Datasource : 返回当前数据集所属的数据源对象
     |  
     |  description
     |      str: 返回用户加入的对数据集的描述信息。
     |  
     |  encode\_type
     |      EncodeType: 返回此数据集数据存储时的编码方式。对数据集采用压缩编码方式,可以减少数据存储所占用的空间,降低数据传输时的网络负载和服务器的负载。矢量数
     |      据集支持的编码方式有Byte,Int16,Int24,Int32,SGL,LZW,DCT,也可以指定为不使用编码方式。光栅数据支持的编码方式有DCT,SGL,LZW
     |      或不使用编码方式。具体请参见 :py:class:`EncodeType` 类型
     |  
     |  name
     |      str: 返回数据集名称
     |  
     |  prj\_coordsys
     |      PrjCoordSys: 返回数据集的投影信息.
     |  
     |  table\_name
     |      str: 返回数据集的表名。对数据库型数据源,返回此数据集在数据库中所对应的数据表名称;对文件型数据源,返回此数据集的存储属性的表名称.
     |  
     |  type
     |      DatasetType: 返回数据集类型
     |  
     |  ----------------------------------------------------------------------
     |  Data descriptors inherited from iobjectspy.\_jsuperpy.data.\_jvm.JVMBase:
     |  
     |  \_\_dict\_\_
     |      dictionary for instance variables (if defined)
     |  
     |  \_\_weakref\_\_
     |      list of weak references to the object (if defined)
    
    class DatasetGrid(Dataset)
     |  栅格数据集类。栅格数据集类,该类用于描述栅格数据,例如高程数据集和土地利用图。栅格数据采用网格形式组织并使用二维的栅格的像素值来记录数据,每个栅
     |  格(cell)代表一个像素要素,栅格值可以描述各种数据信息。栅格数据集中每一个栅格(cell)存储的是表示地物的属性值,属性值可以是土壤类型、密度值、
     |  高程、温度、湿度等。
     |  
     |  Method resolution order:
     |      DatasetGrid
     |      Dataset
     |      iobjectspy.\_jsuperpy.data.\_jvm.JVMBase
     |      builtins.object
     |  
     |  Methods defined here:
     |  
     |  \_\_init\_\_(self)
     |      Initialize self.  See help(type(self)) for accurate signature.
     |  
     |  build\_pyramid(self, resample\_method=None, progress=None)
     |      给栅格数据创建指定类型的金字塔,目的是提高栅格数据的显示速度。。金字塔只能针对原始的数据进行创建;用户仅能给一个数据集创建一次金字塔,如果想
     |      再次创建需要将原来创建的金字塔进行删除,当该栅格数据集显示的时候,已创建的金字塔都将被访问。下图所示为不同比例尺下金字塔的建立过程。
     |      
     |      :param resample\_method: 建金字塔方法的类型
     |      :type resample\_method: ResamplingMethod or str
     |      :param function progress: 进度信息处理函数,参考 :py:class:`.StepEvent`
     |      :return:  创建是否成功,成功返回 True,失败返回 False
     |      :rtype: bool
     |  
     |  build\_statistics(self)
     |      对栅格数据集执行统计操作,返回该栅格数据集的统计结果对象。统计的结果包括栅格数据集的最大值、最小值、均值、中值、众数、稀数、方差、标准差等。
     |      
     |      :return: 包含 最大值、最小值、均值、中值、众数、稀数、方差、标准差 的 dict 对象。其中 dict 中的 key 值:
     |      
     |              - average : 平均值
     |              - majority : 众数
     |              - minority : 稀数
     |              - max : 最大值
     |              - median : 中值
     |              - min : 最小值
     |              - stdDev : 标准差
     |              - var : 方差
     |              - is\_dirty :  是否为“脏”数据
     |      
     |      :rtype: dict
     |  
     |  build\_value\_table(self, out\_data=None, out\_dataset\_name=None)
     |      创建栅格值属性表,其类型为属性表数据集类型TABULAR。
     |      栅格数据集的像素格式为SINGLE 和 DOUBLE ,无法创建属性表,即调用该方法返回为 None。
     |      返回属性表数据集含有系统字段和两个记录栅格信息字段,GRIDVALUE 记录栅格值,GRIDCOUNT 记录栅格值对应的像元个数。
     |      
     |      :param out\_data: 结果数据集所在的数据源
     |      :type out\_data: Datasource or DatasourceConnectionInfo or str
     |      :param str out\_dataset\_name: 结果数据集名称
     |      :return: 结果数据集或数据集名称
     |      :rtype: DatasetVector or str
     |  
     |  calculate\_extremum(self)
     |      计算栅格数据集的极值,即最大值和最小值。建议:栅格数据集在一些分析或者操作之后,建议调用此接口,计算一下最大最小值。
     |      
     |      :return: 如果计算成功返回 true,否则返回 false。
     |      :rtype: bool
     |  
     |  get\_value(self, col, row)
     |      根据给定的行数和列数返回栅格数据集的栅格所对应的栅格值。注意:该方法的参数值的行、列数从零开始计数。
     |      
     |      :param int col: 指定的栅格数据集的列。
     |      :param int row: 指定栅格数据集的行。
     |      :return: 栅格数据集的栅格所对应的栅格值。
     |      :rtype: float
     |  
     |  grid\_to\_xy(self, col, row)
     |      根据指定的行数和列数所对应的栅格点转换为地理坐标系下的点,即 X, Y 坐标。
     |      
     |      :param int col: 指定的列
     |      :param int row: 指定的行
     |      :return: 地理坐标系下的对应的点坐标。
     |      :rtype: Point2D
     |  
     |  has\_pyramid(self)
     |      栅格数据集是否已创建金字塔。
     |      
     |      :rtype: bool
     |  
     |  remove\_pyramid(self)
     |      删除已创建的金字塔
     |      
     |      :rtype: bool
     |  
     |  set\_clip\_region(self, region)
     |      设置栅格数据集的显示区域。
     |      当用户设置此方法后,栅格数据集就按照给定的区域进行显示,区域之外的都不显示。
     |      
     |      注意:
     |      
     |       - 当用户所设定的栅格数据集的地理范围(即调用 :py:meth:`set\_geo\_reference` 方法)与所设定的裁剪区域无重叠区域,栅格数据集不显示。
     |       - 当重新设置栅格数据集的地理范围,不自动修改栅格数据集的裁剪区域。
     |      
     |      :param region: 栅格数据集的显示区域。
     |      :type region: GeoRegion or Rectangle
     |  
     |  set\_color\_table(self, colors)
     |      设置数据集的颜色表
     |      
     |      :param Colors colors: 颜色集合
     |  
     |  set\_geo\_reference(self, rect)
     |      将栅格数据集对应到地理坐标系中指定的地理范围。
     |      
     |      :param  Rectangle rect: 指定的地理范围
     |  
     |  set\_no\_value(self, value)
     |      设置栅格数据集的空值,当此数据集为空值时,用户可采用-9999来表示
     |      
     |      :param float value: 空值
     |  
     |  set\_value(self, col, row, value)
     |      根据给定的行数和列数设置栅格数据集的栅格所对应的栅格值。注意:该方法的参数值的行、列数从零开始计数。
     |      
     |      :param int col: 指定的栅格数据集的列。
     |      :param int row: 指定栅格数据集的行。
     |      :param float value: 指定的栅格数据集的栅格所对应的栅格值。
     |      :return: 栅格数据集的栅格所对应的修改之前的栅格值。
     |      :rtype: float
     |  
     |  update(self, dataset)
     |      根据指定的栅格数据集更新。
     |      注意:指定的栅格数据集和被更新的栅格数据集的编码方式(EncodeType)和像素类型(PixelFormat)必须保持一致
     |      
     |      :param dataset:  指定的栅格数据集。
     |      :type dataset: DatasetGrid or str
     |      :return: 如果更新成功,返回 True,否则返回 False。
     |      :rtype: bool
     |  
     |  update\_pyramid(self, rect)
     |      指定范围更新栅格数据集影像金字塔。
     |      
     |      :param Rectangle rect:  更新金字塔的指定影像范围
     |      :return: 更新成功,返回 True,否则返回 False。
     |      :rtype: bool
     |  
     |  xy\_to\_grid(self, point)
     |      将地理坐标系下的点(X Y)转换为栅格数据集中对应的栅格。
     |      
     |      :param point: 地理坐标系下的点
     |      :type point: Point2D
     |      :return: 栅格数据集对应的栅格,分别返回列和行
     |      :rtype: tuple[int]
     |  
     |  ----------------------------------------------------------------------
     |  Data descriptors defined here:
     |  
     |  block\_size\_option
     |      BlockSizeOption: 数据集的像素分块类型
     |  
     |  clip\_region
     |      GeoRegion: 栅格数据集的显示区域
     |  
     |  color\_table
     |      Colors: 数据集的颜色表
     |  
     |  column\_block\_count
     |      int: 栅格数据集分块后的所得到的总列数。
     |  
     |  height
     |      int:  栅格数据集的栅格数据的高度。单位为像素
     |  
     |  max\_value
     |      float:栅格数据集中栅格值的最大值。
     |  
     |  min\_value
     |      float:栅格数据集中栅格值的最小值
     |  
     |  no\_value
     |      float: 栅格数据集的空值,当此数据集为空值时,用户可采用-9999来表示
     |  
     |  pixel\_format
     |      PixelFormat: 栅格数据存储的像素格式。每个象素采用不同的字节进行表示,单位是比特(bit)。
     |  
     |  row\_block\_count
     |      int: 栅格数据经过分块后所得到的总行数。
     |  
     |  width
     |      int: 栅格数据集的栅格数据的宽度。单位为像素
     |  
     |  ----------------------------------------------------------------------
     |  Methods inherited from Dataset:
     |  
     |  \_\_repr\_\_(self)
     |      Return repr(self).
     |  
     |  \_\_str\_\_(self)
     |      Return str(self).
     |  
     |  close(self)
     |      用于关闭当前数据集
     |  
     |  is\_open(self)
     |      判断数据集是否已经打开,数据集打开返回 True,否则返回 False
     |      
     |      :rtype: bool
     |  
     |  is\_readonly(self)
     |      判断数据集是否是只读。如果数据集是只读,无法进行任何改写数据集的操作。 数据集是只读返回 True,否则返回 False。
     |      
     |      :rtype: bool
     |  
     |  open(self)
     |      打开数据集, 打开数据集成功返返回 True,否则返回 False
     |      
     |      :rtype: bool
     |  
     |  rename(self, new\_name)
     |      修改数据集名称
     |      
     |      :param str new\_name: 新的数据集名称
     |      :return: 修改成功返回True,否则返回False
     |      :rtype: bool
     |  
     |  set\_bounds(self, rc)
     |      设置数据集中包含所有对象的最小外接矩形。对于矢量数据集来说,为数据集中所有对象的最小外接矩形;对于栅格数据集来说,为
     |      当前栅格或影像的地理范围。
     |      
     |      :param Rectangle rc: 数据集中包含所有对象的最小外接矩形。
     |      :return: self
     |      :rtype: Dataset
     |  
     |  set\_description(self, value)
     |      设置用户加入的对数据集的描述信息。
     |      
     |      :param str value: 用户加入的对数据集的描述信息。
     |  
     |  set\_prj\_coordsys(self, value)
     |      设置数据集的投影信息
     |      
     |      :param PrjCoordSys value: 投影信息
     |  
     |  to\_json(self)
     |      输出数据集的信息到 json 字符串中。数据集的 json 串内容包括数据源的连接信息和数据集名称两项。
     |      
     |      :rtype: str
     |      
     |      示例::
     |       >>> ds = Workspace().get\_datasource('data')
     |       >>> print(ds[0].to\_json())
     |       \{"name": "location", "datasource": \{"type": "UDB", "alias": "data", "server": "E:/data.udb", "is\_readonly": false\}\}
     |  
     |  ----------------------------------------------------------------------
     |  Static methods inherited from Dataset:
     |  
     |  from\_json(value)
     |      从数据集的 json 字符串中获取数据集,如果数据源没有打开,将自动打开数据源。
     |      
     |      :param str value:  json 字符串
     |      :return: 数据集对象
     |      :rtype: Dataet
     |  
     |  ----------------------------------------------------------------------
     |  Data descriptors inherited from Dataset:
     |  
     |  bounds
     |      Rectangle: 返回数据集中包含所有对象的最小外接矩形。对于矢量数据集来说,为数据集中所有对象的最小外接矩形;对于栅格数据集来说,为当前栅格或影像的
     |      地理范围。
     |  
     |  datasource
     |      Datasource : 返回当前数据集所属的数据源对象
     |  
     |  description
     |      str: 返回用户加入的对数据集的描述信息。
     |  
     |  encode\_type
     |      EncodeType: 返回此数据集数据存储时的编码方式。对数据集采用压缩编码方式,可以减少数据存储所占用的空间,降低数据传输时的网络负载和服务器的负载。矢量数
     |      据集支持的编码方式有Byte,Int16,Int24,Int32,SGL,LZW,DCT,也可以指定为不使用编码方式。光栅数据支持的编码方式有DCT,SGL,LZW
     |      或不使用编码方式。具体请参见 :py:class:`EncodeType` 类型
     |  
     |  name
     |      str: 返回数据集名称
     |  
     |  prj\_coordsys
     |      PrjCoordSys: 返回数据集的投影信息.
     |  
     |  table\_name
     |      str: 返回数据集的表名。对数据库型数据源,返回此数据集在数据库中所对应的数据表名称;对文件型数据源,返回此数据集的存储属性的表名称.
     |  
     |  type
     |      DatasetType: 返回数据集类型
     |  
     |  ----------------------------------------------------------------------
     |  Data descriptors inherited from iobjectspy.\_jsuperpy.data.\_jvm.JVMBase:
     |  
     |  \_\_dict\_\_
     |      dictionary for instance variables (if defined)
     |  
     |  \_\_weakref\_\_
     |      list of weak references to the object (if defined)
    
    class DatasetGridInfo(builtins.object)
     |  栅格数据集信息类。该类包括了返回和设置栅格数据集的相应的设置信息等,例如栅格数据集的名称、宽度、高度、像素格式、编码方式、存储分块大小和空值等。
     |  
     |  Methods defined here:
     |  
     |  \_\_init\_\_(self, name=None, width=None, height=None, pixel\_format=None, encode\_type=None, block\_size\_option=BlockSizeOption.BS\_256)
     |      构造栅格数据集信息对象
     |      
     |      :param str name: 数据集名称
     |      :param int width: 数据集的宽度,单位为像素
     |      :param int height: 数据集的高度,单位为像素
     |      :param pixel\_format: 数据集存储的像素格式
     |      :type pixel\_format: PixelFormat or str
     |      :param encode\_type: 数据集存储的编码方式
     |      :type encode\_type: EncodeType or str
     |      :param block\_size\_option: 数据集的像素分块类型
     |      :type block\_size\_option:  BlockSizeOption
     |  
     |  from\_dict(self, values)
     |      从 dict 中读取 DatasetGridInfo 信息
     |      
     |      :param dict values:
     |      :return: self
     |      :rtype: DatasetGridInfo
     |  
     |  set\_block\_size\_option(self, value)
     |      设置数据集的像素分块类型。以正方形方式进行分块存储。其中在进行分块过程中,如果
     |      栅格数据不足以进行完整地分块,那么采用空格补充完整进行存储。默认值为 BlockSizeOption.BS\_256。
     |      
     |      :param value: 栅格数据集的像素分块
     |      :type value: BlockSizeOption or str
     |      :return: self
     |      :rtype: DatasetGridInfo
     |  
     |  set\_bounds(self, value)
     |      设置栅格数据集的地理范围。
     |      
     |      :param Rectangle value: 栅格数据集的地理范围。
     |      :return: self
     |      :rtype: DatasetGridInfo
     |  
     |  set\_encode\_type(self, value)
     |      设置格栅格数据集数据存储时的编码方式。对数据集采用压缩编码方式,可以减少数据存储所占用的空间,降低数据传输时的网络负载和服务器的负载。
     |      光栅数据支持的编码方式有 DCT,SGL,LZW 或不使用编码方式。
     |      
     |      :param value: 栅格数据集数据存储时的编码方式
     |      :type value: EncodeType or str
     |      :return: self
     |      :rtype: DatasetGridInfo
     |  
     |  set\_height(self, value)
     |      设置栅格数据集的栅格数据的高度。单位为像素。
     |      
     |      :param float value: 栅格数据集的栅格数据的高度
     |      :return: self
     |      :rtype: DatasetGridInfo
     |  
     |  set\_max\_value(self, value)
     |      设置栅格数据集栅格行列中的最大值。
     |      
     |      :param float value: 栅格数据集栅格行列中的最大值
     |      :return: self
     |      :rtype: DatasetGridInfo
     |  
     |  set\_min\_value(self, value)
     |      设置栅格数据集栅格行列中的最小值
     |      
     |      :param float value: 栅格数据集栅格行列中的最小值
     |      :return: self
     |      :rtype: DatasetGridInfo
     |  
     |  set\_name(self, value)
     |      设置数据集的名称
     |      
     |      :param str value: 数据集名称
     |      :return: self
     |      :rtype: DatasetGridInfo
     |  
     |  set\_no\_value(self, value)
     |      设置栅格数据集的空值,当此数据集为空值时,用户可采用-9999来表示。
     |      
     |      :param float value: 栅格数据集的空值
     |      :return: self
     |      :rtype: DatasetGridInfo
     |  
     |  set\_pixel\_format(self, value)
     |      设置栅格数据集的存储的像素格式
     |      
     |      :param value: 栅格数据集的存储的像素格式
     |      :type value: PixelFormat or str
     |      :return: self
     |      :rtype: DatasetGridInfo
     |  
     |  set\_width(self, value)
     |      设置栅格数据集的栅格数据的宽度。单位为像素。
     |      
     |      :param int value: 栅格数据集的栅格数据的宽度。单位为像素。
     |      :return: self
     |      :rtype: DatasetGridInfo
     |  
     |  to\_dict(self)
     |      将当前对象信息输出为 dict
     |      
     |      :rtype: dict
     |  
     |  ----------------------------------------------------------------------
     |  Static methods defined here:
     |  
     |  make\_from\_dict(values)
     |      从 dict 中读取信息构建 DatasetGridInfo 对象
     |      
     |      :param dict values:
     |      :rtype: DatasetGridInfo
     |  
     |  ----------------------------------------------------------------------
     |  Data descriptors defined here:
     |  
     |  \_\_dict\_\_
     |      dictionary for instance variables (if defined)
     |  
     |  \_\_weakref\_\_
     |      list of weak references to the object (if defined)
     |  
     |  block\_size\_option
     |      BlockSizeOption: 数据集的像素分块类型
     |  
     |  bounds
     |      Rectangle: 栅格数据集的地理范围.
     |  
     |  encode\_type
     |      EncodeType: 返回栅格数据集数据存储时的编码方式。对数据集采用压缩编码方式,可以减少数据存储所占用的空间,降低数据传输时的网络负载和服务器的负载。
     |      光栅数据支持的编码方式有 DCT,SGL,LZW 或不使用编码方式
     |  
     |  height
     |      int:  栅格数据集的栅格数据的高度。单位为像素
     |  
     |  max\_value
     |      float: 栅格数据集栅格行列中的最大值
     |  
     |  min\_value
     |      float: 格数据集栅格行列中的最小值
     |  
     |  name
     |      str: 数据集名称
     |  
     |  no\_value
     |      float: 栅格数据集的空值,当此数据集为空值时,用户可采用-9999来表示
     |  
     |  pixel\_format
     |      PixelFormat: 栅格数据存储的像素格式。每个象素采用不同的字节进行表示,单位是比特(bit)。
     |  
     |  width
     |      int: 栅格数据集的栅格数据的宽度。单位为像素
    
    class DatasetImage(Dataset)
     |  影像数据集类。 影像数据集类,该类用于描述影像数据,不具备属性信息,例如影像地图、多波段影像和实物地图等。 光栅数据采用网格形式组织并使用二维栅格
     |  的像素值来记录数据,每个栅格(cell)代表一个像素要素,栅格值可以描述各种数据信息。影像数据集中每一个栅格存储的是一个颜色值或颜色的索引值(RGB 值)。
     |  
     |  Method resolution order:
     |      DatasetImage
     |      Dataset
     |      iobjectspy.\_jsuperpy.data.\_jvm.JVMBase
     |      builtins.object
     |  
     |  Methods defined here:
     |  
     |  \_\_init\_\_(self)
     |      Initialize self.  See help(type(self)) for accurate signature.
     |  
     |  add\_band(self, datasets, indexes=None)
     |      向指定的多波段影像数据集中按照指定的索引追加多个波段
     |      
     |      :param datasets: 影像数据集合
     |      :type datasets: list[DatasetImage] or DatasetImage
     |      :param indexes:  要追加的波段索引,当输入的是单个 DatasetImage 数据才有效。
     |      :type indexes: list[int]
     |      :return: 添加的波段个数
     |      :rtype: int
     |  
     |  build\_pyramid(self, progress=None)
     |      给影像数据集创建金字塔。目的是提高影像数据集的显示速度。金字塔只能针对原始的数据进行创建;一次仅能给一个数据集创建金字塔,当显示该影像数据集的
     |      时候,已创建的金字塔都将被访问。
     |      
     |      :param function progress: 进度信息处理函数,参考 :py:class:`.StepEvent`
     |      :return:  创建是否成功,成功返回 True,失败返回 False
     |      :rtype: bool
     |  
     |  build\_statistics(self)
     |      对影像数据集执行统计操作,返回该影像数据集的统计结果对象。统计的结果包括影像数据集的最大值、最小值、均值、中值、众数、稀数、方差、标准差等。
     |      
     |      :return: 返回一个dict,dict中包含每个波段的统计结果,统计结果为包含最大值、最小值、均值、中值、众数、稀数、方差、标准差 的 dict 对象。其中 dict 中的 key 值:
     |      
     |              - average : 平均值
     |              - majority : 众数
     |              - minority : 稀数
     |              - max : 最大值
     |              - median : 中值
     |              - min : 最小值
     |              - stdDev : 标准差
     |              - var : 方差
     |              - is\_dirty :  是否为“脏”数据
     |      
     |      :rtype: dict[dict]
     |  
     |  calculate\_extremum(self, band=0)
     |      计算影像数据指定波段的极值,即最大值和最小值。
     |      
     |      :param int band:  要计算极值的影像数据的波段序号。
     |      :return: 如果计算成功返回 true,否则返回 false。
     |      :rtype: bool
     |  
     |  delete\_band(self, start\_index, count=1)
     |      根据指定索引号删除某个波段
     |      
     |      :param int start\_index: 指定删除波段的开始索引号。
     |      :param int count:  要删除的波段的个数。
     |      :return: 删除成功返回 true;否则返回 false。
     |      :rtype: bool
     |  
     |  get\_band\_index(self, name)
     |      获取指定波段名称所在的序号
     |      
     |      :param str name: 波段名称
     |      :return: 波段所在的序号
     |      :rtype: int
     |  
     |  get\_band\_name(self, band)
     |      返回指定序号的波段的名称。
     |      
     |      :param int band: 波段序号
     |      :return: 波段名称
     |      :rtype: str
     |  
     |  get\_max\_value(self, band=0)
     |      获取影像数据集指定波段的最大像素值
     |      
     |      :param int band: 指定的波段索引号,从 0 开始。
     |      :return: 影像数据集指定波段的最大像素值
     |      :rtype: float
     |  
     |  get\_min\_value(self, band=0)
     |      获取影像数据集指定波段的最小像素值
     |      
     |      :param int band: 指定的波段索引号,从 0 开始。
     |      :return: 影像数据集指定波段的最小像素值
     |      :rtype: float
     |  
     |  get\_no\_value(self, band=0)
     |      返回影像数据集指定波段的无值。
     |      
     |      :param int band: 指定的波段索引号,从 0 开始
     |      :return: 影像数据集中指定波段的无值
     |      :rtype: float
     |  
     |  get\_palette(self, band=0)
     |      获取影像数据集指定波段的颜色调色板
     |      
     |      :param int band:  指定的波段索引号,从 0 开始。
     |      :return: 影像数据集指定波段的颜色调色板
     |      :rtype: Colors
     |  
     |  get\_pixel\_format(self, band)
     |      返回影像数据集指定波段的像素格式。
     |      
     |      :param int band: 指定的波段索引号,从 0 开始。
     |      :return: 影像数据集指定波段的像素格式。
     |      :rtype: PixelFormat
     |  
     |  get\_value(self, col, row, band)
     |      根据给定的行数和列数返回影像数据集的栅格所对应的像素值。注意:该方法的参数值的行、列数从零开始计数。
     |      
     |      :param int col: 指定的影像数据集的列。
     |      :param int row: 指定的影像数据集的行。
     |      :param int band: 指定的波段数
     |      :return: 影像数据集中所对应的像素值。
     |      :rtype: float or tuple
     |  
     |  has\_pyramid(self)
     |      影像数据集是否已创建金字塔。
     |      
     |      :rtype: bool
     |  
     |  image\_to\_xy(self, col, row)
     |      根据指定的行数和列数所对应的影像点转换为地理坐标系下的点,即 X, Y 坐标。
     |      
     |      :param int col: 指定的列
     |      :param int row: 指定的行
     |      :return: 地理坐标系下的对应的点坐标。
     |      :rtype: Point2D
     |  
     |  remove\_pyramid(self)
     |      影像已创建的金字塔
     |      
     |      :rtype: bool
     |  
     |  set\_band\_name(self, band, name)
     |      设置指定序号的波段的名称。
     |      
     |      :param int band:  波段序号
     |      :param str name: 波段名称
     |  
     |  set\_clip\_region(self, region)
     |      设置影像数据集的显示区域。
     |      当用户设置此方法后,影像格数据集就按照给定的区域进行显示,区域之外的都不显示。
     |      
     |      注意:
     |      
     |      - 当用户所设定的影像数据集的地理范围(即调用 :py:meth:`set\_geo\_reference` 方法)与所设定的裁剪区域无重叠区域,影像数据集不显示。
     |      - 当重新设置影像数据集的地理范围,不自动修改影像数据集的裁剪区域。
     |      
     |      :param region: 影像数据集的显示区域。
     |      :type region: GeoRegion or Rectangle
     |  
     |  set\_geo\_reference(self, rect)
     |      将影像数据集对应到地理坐标系中指定的地理范围。
     |      
     |      :param  Rectangle rect: 指定的地理范围
     |  
     |  set\_no\_value(self, value, band)
     |      设置影像数据集指定波段的无值。
     |      
     |      :param float value: 指定的无值。
     |      :param int band: 指定的波段索引号,从 0 开始。
     |  
     |  set\_palette(self, colors, band)
     |      设置影像数据集指定波段的颜色调色板
     |      
     |      :param Colors colors: 颜色调色板。
     |      :param int band:  指定的波段索引号,从 0 开始。
     |  
     |  set\_value(self, col, row, value, band)
     |      根据给定的行数和列数设置影像数据集的所对应的像素值。注意:该方法的参数值的行、列数从零开始计数。
     |      
     |      :param int col: 指定的影像数据集的列。
     |      :param int row: 指定的影像数据集的行。
     |      :param value: 指定的影像数据集的所对应的像素值。
     |      :type value: tuple or float
     |      :param int band: 指定的波段序号
     |      :return: 影像数据集中所对应的修改之前的像素值。
     |      :rtype: float
     |  
     |  update(self, dataset)
     |      根据指定的影像数据集更新。
     |      注意:指定的影像数据集和被更新的影像数据集的编码方式(EncodeType)和像素类型(PixelFormat)必须保持一致。
     |      
     |      :param dataset: 指定的影像数据集。
     |      :type dataset: DatasetImage or str
     |      :return: 如果更新成功,返回 True,否则返回 False。
     |      :rtype: bool
     |  
     |  update\_pyramid(self, rect)
     |      指定范围更新影像数据集影像金字塔。
     |      
     |      :param Rectangle rect:  更新金字塔的指定影像范围
     |      :return: 更新成功,返回 True,否则返回 False。
     |      :rtype: bool
     |  
     |  xy\_to\_image(self, point)
     |      将地理坐标系下的点(X Y)转换为影像数据集中对应的像素值。
     |      
     |      :param point: 地理坐标系下的点
     |      :type point: Point2D
     |      :return: 影像数据集对应的影像点
     |      :rtype: tuple[int]
     |  
     |  ----------------------------------------------------------------------
     |  Data descriptors defined here:
     |  
     |  band\_count
     |      int: 返回波段数目
     |  
     |  block\_size\_option
     |      BlockSizeOption: 数据集的像素分块类型
     |  
     |  clip\_region
     |      GeoRegion: 影像数据集的显示区域
     |  
     |  height
     |      int:  影像数据集的影像数据的高度。单位为像素
     |  
     |  width
     |      int: 影像数据集的影像数据的宽度。单位为像素
     |  
     |  ----------------------------------------------------------------------
     |  Methods inherited from Dataset:
     |  
     |  \_\_repr\_\_(self)
     |      Return repr(self).
     |  
     |  \_\_str\_\_(self)
     |      Return str(self).
     |  
     |  close(self)
     |      用于关闭当前数据集
     |  
     |  is\_open(self)
     |      判断数据集是否已经打开,数据集打开返回 True,否则返回 False
     |      
     |      :rtype: bool
     |  
     |  is\_readonly(self)
     |      判断数据集是否是只读。如果数据集是只读,无法进行任何改写数据集的操作。 数据集是只读返回 True,否则返回 False。
     |      
     |      :rtype: bool
     |  
     |  open(self)
     |      打开数据集, 打开数据集成功返返回 True,否则返回 False
     |      
     |      :rtype: bool
     |  
     |  rename(self, new\_name)
     |      修改数据集名称
     |      
     |      :param str new\_name: 新的数据集名称
     |      :return: 修改成功返回True,否则返回False
     |      :rtype: bool
     |  
     |  set\_bounds(self, rc)
     |      设置数据集中包含所有对象的最小外接矩形。对于矢量数据集来说,为数据集中所有对象的最小外接矩形;对于栅格数据集来说,为
     |      当前栅格或影像的地理范围。
     |      
     |      :param Rectangle rc: 数据集中包含所有对象的最小外接矩形。
     |      :return: self
     |      :rtype: Dataset
     |  
     |  set\_description(self, value)
     |      设置用户加入的对数据集的描述信息。
     |      
     |      :param str value: 用户加入的对数据集的描述信息。
     |  
     |  set\_prj\_coordsys(self, value)
     |      设置数据集的投影信息
     |      
     |      :param PrjCoordSys value: 投影信息
     |  
     |  to\_json(self)
     |      输出数据集的信息到 json 字符串中。数据集的 json 串内容包括数据源的连接信息和数据集名称两项。
     |      
     |      :rtype: str
     |      
     |      示例::
     |       >>> ds = Workspace().get\_datasource('data')
     |       >>> print(ds[0].to\_json())
     |       \{"name": "location", "datasource": \{"type": "UDB", "alias": "data", "server": "E:/data.udb", "is\_readonly": false\}\}
     |  
     |  ----------------------------------------------------------------------
     |  Static methods inherited from Dataset:
     |  
     |  from\_json(value)
     |      从数据集的 json 字符串中获取数据集,如果数据源没有打开,将自动打开数据源。
     |      
     |      :param str value:  json 字符串
     |      :return: 数据集对象
     |      :rtype: Dataet
     |  
     |  ----------------------------------------------------------------------
     |  Data descriptors inherited from Dataset:
     |  
     |  bounds
     |      Rectangle: 返回数据集中包含所有对象的最小外接矩形。对于矢量数据集来说,为数据集中所有对象的最小外接矩形;对于栅格数据集来说,为当前栅格或影像的
     |      地理范围。
     |  
     |  datasource
     |      Datasource : 返回当前数据集所属的数据源对象
     |  
     |  description
     |      str: 返回用户加入的对数据集的描述信息。
     |  
     |  encode\_type
     |      EncodeType: 返回此数据集数据存储时的编码方式。对数据集采用压缩编码方式,可以减少数据存储所占用的空间,降低数据传输时的网络负载和服务器的负载。矢量数
     |      据集支持的编码方式有Byte,Int16,Int24,Int32,SGL,LZW,DCT,也可以指定为不使用编码方式。光栅数据支持的编码方式有DCT,SGL,LZW
     |      或不使用编码方式。具体请参见 :py:class:`EncodeType` 类型
     |  
     |  name
     |      str: 返回数据集名称
     |  
     |  prj\_coordsys
     |      PrjCoordSys: 返回数据集的投影信息.
     |  
     |  table\_name
     |      str: 返回数据集的表名。对数据库型数据源,返回此数据集在数据库中所对应的数据表名称;对文件型数据源,返回此数据集的存储属性的表名称.
     |  
     |  type
     |      DatasetType: 返回数据集类型
     |  
     |  ----------------------------------------------------------------------
     |  Data descriptors inherited from iobjectspy.\_jsuperpy.data.\_jvm.JVMBase:
     |  
     |  \_\_dict\_\_
     |      dictionary for instance variables (if defined)
     |  
     |  \_\_weakref\_\_
     |      list of weak references to the object (if defined)
    
    class DatasetImageInfo(builtins.object)
     |  影像数据集信息类,该类用于设置影像数据集的创建信息,包括名称、宽度、高度、波段数和存储分块大小等。
     |  
     |  通过该类设置影像数据集的创建信息时,需要注意:
     |  
     |  - 需要指定影像的波段数,波段数可以设置为 0,创建之后可以再向影像中添加波段;
     |  - 所有波段被设置为相同的像素格式和编码方式,创建影像成功后,可以根据需求,再为每个波段设置不同的像素格式和其编码类型。
     |  
     |  Methods defined here:
     |  
     |  \_\_init\_\_(self, name=None, width=None, height=None, pixel\_format=None, encode\_type=None, block\_size\_option=BlockSizeOption.BS\_256, band\_count=None)
     |      构造影像数据集信息对象
     |      
     |      :param str name: 数据集名称
     |      :param int width: 数据集的宽度,单位为像素
     |      :param int height: 数据集的高度,单位为像素
     |      :param pixel\_format: 数据集存储的像素格式
     |      :type pixel\_format: PixelFormat or str
     |      :param encode\_type: 数据集存储的编码方式
     |      :type encode\_type: EncodeType or str
     |      :param block\_size\_option: 数据集的像素分块类型
     |      :type block\_size\_option:  BlockSizeOption
     |      :param int band\_count: 波段数目
     |  
     |  from\_dict(self, values)
     |      从 dict 中读取 DatasetImageInfo 信息
     |      
     |      :param dict values:
     |      :return: self
     |      :rtype: DatasetImageInfo
     |  
     |  set\_band\_count(self, value)
     |      设置影像数据集的波段数目。创建影像数据集时,波段数可以设置为 0,此时,像素格式(pixel\_format)和编码格式(encode\_type)的设置是无效的
     |      ,因为这些信息是针对波段而言,因此波段为 0 时无法保存。此影像数据集的像素格式和编码格式,将以向其中添加的第一个波段的相关信息为准。
     |      
     |      :param int value: 波段数目。
     |      :return: self
     |      :rtype: DatasetImageInfo
     |  
     |  set\_block\_size\_option(self, value)
     |      设置数据集的像素分块类型。以正方形方式进行分块存储。其中在进行分块过程中,如果果影像数据不足以进行完整地分块,那么采用空格补充完整进行存储。默认值为 BlockSizeOption.BS\_256。
     |      
     |      :param value: 影像数据集的像素分块
     |      :type value: BlockSizeOption or str
     |      :return: self
     |      :rtype: DatasetImageInfo
     |  
     |  set\_bounds(self, value)
     |      设置影像数据集的地理范围。
     |      
     |      :param Rectangle value: 影像数据集的地理范围。
     |      :return: self
     |      :rtype: DatasetImageInfo
     |  
     |  set\_encode\_type(self, value)
     |      设置影像数据集数据存储时的编码方式。对数据集采用压缩编码方式,可以减少数据存储所占用的空间,降低数据传输时的网络负载和服务器的负载。
     |      光栅数据支持的编码方式有 DCT,SGL,LZW 或不使用编码方式。
     |      
     |      :param value: 像数据集数据存储时的编码方式
     |      :type value: EncodeType or str
     |      :return: self
     |      :rtype: DatasetImageInfo
     |  
     |  set\_height(self, value)
     |      设置影像数据集的影像数据的高度。单位为像素。
     |      
     |      :param int value: 影像数据集的影像数据的高度。单位为像素。
     |      :return: self
     |      :rtype: DatasetImageInfo
     |  
     |  set\_name(self, value)
     |      设置数据集的名称
     |      
     |      :param str value: 数据集名称
     |      :return: self
     |      :rtype: DatasetImageInfo
     |  
     |  set\_pixel\_format(self, value)
     |      设置影像数据集的存储的像素格式。影像数据集不支持 DOUBLE、SINGLE、BIT64 类型的像素格式。
     |      
     |      :param value: 影像数据集的存储的像素格式
     |      :type value: PixelFormat or str
     |      :return: self
     |      :rtype: DatasetImageInfo
     |  
     |  set\_width(self, value)
     |      设置影像数据集的影像数据的宽度。单位为像素。
     |      
     |      :param int value: 影像数据集的影像数据的宽度。单位为像素。
     |      :return: self
     |      :rtype: DatasetImageInfo
     |  
     |  to\_dict(self)
     |      将当前对象信息输出为 dict
     |      
     |      :rtype: dict
     |  
     |  ----------------------------------------------------------------------
     |  Static methods defined here:
     |  
     |  make\_from\_dict(values)
     |      从 dict 中读取信息构建 DatasetImageInfo 对象
     |      
     |      :param dict values:
     |      :rtype: DatasetImageInfo
     |  
     |  ----------------------------------------------------------------------
     |  Data descriptors defined here:
     |  
     |  \_\_dict\_\_
     |      dictionary for instance variables (if defined)
     |  
     |  \_\_weakref\_\_
     |      list of weak references to the object (if defined)
     |  
     |  band\_count
     |      int: 波段数目
     |  
     |  block\_size\_option
     |      BlockSizeOption: 数据集的像素分块类型
     |  
     |  bounds
     |      Rectangle: 影像数据集的地理范围.
     |  
     |  encode\_type
     |      EncodeType: 返回影像数据集数据存储时的编码方式。对数据集采用压缩编码方式,可以减少数据存储所占用的空间,降低数据传输时的网络负载和服务器的负载。
     |      光栅数据支持的编码方式有 DCT,SGL,LZW 或不使用编码方式
     |  
     |  height
     |      int:  影像数据集的影像数据的高度。单位为像素
     |  
     |  name
     |      str: 数据集名称
     |  
     |  pixel\_format
     |      PixelFormat: 影像数据存储的像素格式。每个象素采用不同的字节进行表示,单位是比特(bit)。
     |  
     |  width
     |      int: 影像数据集的影像数据的宽度。单位为像素
    
    class DatasetTopology(Dataset)
     |  数据集(矢量数据集,栅格数据集,影像数据集等)的基类,提供各种数据集数据集公共的属性,方法。
     |  数据集一般为存储在一起的相关数据的集合;根据数据类型的不同,分为矢量数据集和栅格数据集,以及为了处理特定问题而设计的如拓扑数据集,网络数据集
     |  等。数据集是 GIS 数据组织的最小单位。其中矢量数据集是由同种类型空间要素组成的集合,所以也可以称为要素集。根据要素的空间特征的不同,矢量数据集
     |  又分为点数据集,线数据集,面数据集等,各矢量数据集是空间特征和性质相同而组织在一起的数据的集合。而栅格数据集由像元阵列组成,在表现要素上比矢量
     |  数据集欠缺,但是可以很好的表现空间现象的位置关系。
     |  
     |  Method resolution order:
     |      DatasetTopology
     |      Dataset
     |      iobjectspy.\_jsuperpy.data.\_jvm.JVMBase
     |      builtins.object
     |  
     |  Methods defined here:
     |  
     |  \_\_init\_\_(self)
     |      Initialize self.  See help(type(self)) for accurate signature.
     |  
     |  ----------------------------------------------------------------------
     |  Methods inherited from Dataset:
     |  
     |  \_\_repr\_\_(self)
     |      Return repr(self).
     |  
     |  \_\_str\_\_(self)
     |      Return str(self).
     |  
     |  close(self)
     |      用于关闭当前数据集
     |  
     |  is\_open(self)
     |      判断数据集是否已经打开,数据集打开返回 True,否则返回 False
     |      
     |      :rtype: bool
     |  
     |  is\_readonly(self)
     |      判断数据集是否是只读。如果数据集是只读,无法进行任何改写数据集的操作。 数据集是只读返回 True,否则返回 False。
     |      
     |      :rtype: bool
     |  
     |  open(self)
     |      打开数据集, 打开数据集成功返返回 True,否则返回 False
     |      
     |      :rtype: bool
     |  
     |  rename(self, new\_name)
     |      修改数据集名称
     |      
     |      :param str new\_name: 新的数据集名称
     |      :return: 修改成功返回True,否则返回False
     |      :rtype: bool
     |  
     |  set\_bounds(self, rc)
     |      设置数据集中包含所有对象的最小外接矩形。对于矢量数据集来说,为数据集中所有对象的最小外接矩形;对于栅格数据集来说,为
     |      当前栅格或影像的地理范围。
     |      
     |      :param Rectangle rc: 数据集中包含所有对象的最小外接矩形。
     |      :return: self
     |      :rtype: Dataset
     |  
     |  set\_description(self, value)
     |      设置用户加入的对数据集的描述信息。
     |      
     |      :param str value: 用户加入的对数据集的描述信息。
     |  
     |  set\_prj\_coordsys(self, value)
     |      设置数据集的投影信息
     |      
     |      :param PrjCoordSys value: 投影信息
     |  
     |  to\_json(self)
     |      输出数据集的信息到 json 字符串中。数据集的 json 串内容包括数据源的连接信息和数据集名称两项。
     |      
     |      :rtype: str
     |      
     |      示例::
     |       >>> ds = Workspace().get\_datasource('data')
     |       >>> print(ds[0].to\_json())
     |       \{"name": "location", "datasource": \{"type": "UDB", "alias": "data", "server": "E:/data.udb", "is\_readonly": false\}\}
     |  
     |  ----------------------------------------------------------------------
     |  Static methods inherited from Dataset:
     |  
     |  from\_json(value)
     |      从数据集的 json 字符串中获取数据集,如果数据源没有打开,将自动打开数据源。
     |      
     |      :param str value:  json 字符串
     |      :return: 数据集对象
     |      :rtype: Dataet
     |  
     |  ----------------------------------------------------------------------
     |  Data descriptors inherited from Dataset:
     |  
     |  bounds
     |      Rectangle: 返回数据集中包含所有对象的最小外接矩形。对于矢量数据集来说,为数据集中所有对象的最小外接矩形;对于栅格数据集来说,为当前栅格或影像的
     |      地理范围。
     |  
     |  datasource
     |      Datasource : 返回当前数据集所属的数据源对象
     |  
     |  description
     |      str: 返回用户加入的对数据集的描述信息。
     |  
     |  encode\_type
     |      EncodeType: 返回此数据集数据存储时的编码方式。对数据集采用压缩编码方式,可以减少数据存储所占用的空间,降低数据传输时的网络负载和服务器的负载。矢量数
     |      据集支持的编码方式有Byte,Int16,Int24,Int32,SGL,LZW,DCT,也可以指定为不使用编码方式。光栅数据支持的编码方式有DCT,SGL,LZW
     |      或不使用编码方式。具体请参见 :py:class:`EncodeType` 类型
     |  
     |  name
     |      str: 返回数据集名称
     |  
     |  prj\_coordsys
     |      PrjCoordSys: 返回数据集的投影信息.
     |  
     |  table\_name
     |      str: 返回数据集的表名。对数据库型数据源,返回此数据集在数据库中所对应的数据表名称;对文件型数据源,返回此数据集的存储属性的表名称.
     |  
     |  type
     |      DatasetType: 返回数据集类型
     |  
     |  ----------------------------------------------------------------------
     |  Data descriptors inherited from iobjectspy.\_jsuperpy.data.\_jvm.JVMBase:
     |  
     |  \_\_dict\_\_
     |      dictionary for instance variables (if defined)
     |  
     |  \_\_weakref\_\_
     |      list of weak references to the object (if defined)
    
    class DatasetVector(Dataset)
     |  矢量数据集类。用于对矢量数据集进行描述,并对之进行相应的管理和操作。对矢量数据集的操作主要包括数据查询、修改、删除、建立索引等。
     |  
     |  Method resolution order:
     |      DatasetVector
     |      Dataset
     |      iobjectspy.\_jsuperpy.data.\_jvm.JVMBase
     |      builtins.object
     |  
     |  Methods defined here:
     |  
     |  \_\_init\_\_(self)
     |      Initialize self.  See help(type(self)) for accurate signature.
     |  
     |  \_\_str\_\_(self)
     |      Return str(self).
     |  
     |  append(self, data, fields=None)
     |      往当前数据集中追加记录。写入的数据可以:
     |      
     |      - Recordset 或 Recordset 的列表。需要确保被写入的 Recordset 的数据集类型与当前数据集类型相同,属性表结构与当前数据集的属性表结构相同,否则,可能会导致属性数据写入失败
     |      - DatasetVector 或 DatasetVector 的列表。需要确保被写入的 DatasetVector 的数据集类型与当前数据集类型相同,属性表结构与当前数据集的属性表结构相同,否则,可能会导致属性数据写入失败
     |      - Point2D or Point2D 的列表。写入数据为 Point2D 时不支持设置 fields。当前数据集必须为点数据集或 CAD 数据集。
     |      - Rectangle or Rectangle 的列表。写入数据为 Rectangle 时不支持设置 fields。当前数据集必须为面数据集或 CAD 数据集。
     |      - Geometry or Geometry 的列表。写入数据为 Geometry 时不支持设置 fields。当 Geometry 类型为:
     |      
     |          - 点,当前数据集必须为点数据集或 CAD 数据集。
     |          - 线,当前数据集必须为线数据集或 CAD 数据集
     |          - 面,当前数据集必须为面数据集或 CAD 数据集
     |          - 文本,当前数据集必须为文本数据集或 CAD 数据集
     |      
     |      - Feature 或 Feature 的列表。当设置了 fields 时必须确保 fields 中 Feature 的字段类型与数据集的字段是匹配的,否则可能导致写入属性丢
     |        失。当 fields 为空时,必须确保 Feature 中字段与当前数据集的属性字段完全匹配。当 Feature 含有空间对象时,空间对象的类型为:
     |      
     |          - 点,当前数据集必须为点数据集或 CAD 数据集。
     |          - 线,当前数据集必须为线数据集或 CAD 数据集
     |          - 面,当前数据集必须为面数据集或 CAD 数据集
     |          - 文本,当前数据集必须为文本数据集或 CAD 数据集
     |      
     |      
     |      :param data: 被写入的数据
     |      :type data: Recordset or DatasetVector or Geometry or Rectangle or Point2D or Feature or list[Recordset] or list[DatasetVector] or list[Geometry] or list[Rectangle] or list[Point2D] or list[Feature]
     |      :param dict fields: 字段映射,key 为 被写入数据中的字段名称,value 为当前数据集的字段名称。
     |      :return:  写入数据成功返回 True,否则返回 False
     |      :rtype: bool
     |  
     |  append\_fields(self, source, source\_link\_field, target\_link\_field, source\_fields, target\_fields=None)
     |      从源数据集向目标数据集追加字段,并根据关联字段查询结果对字段进行赋值。
     |      
     |      注意:
     |      
     |       * 如果指定的源数据集中被追加到目标数据集的字段名集合的某字段在源数据集中不存在,则忽略此字段,只追加源数据集中存在的字段;
     |       * 如果指定了追加字段在目标数据集中相对应的字段名集合,则按所指定的字段名在目标数据集中创建所追加的字段;当指定的字段名在目标数据集中已存在时,则自动加\_x(1、2、3{\ldots})进行字段的创建;
     |       * 如果在目标数据集中创建字段失败,则忽略此字段,继续追加其它字段;
     |       * 必须指定源字段名集合,否则追加不成功;
     |       * 可以不必指定目标字段名集合,一旦指定目标字段名集合,则此集合中字段名必须与源字段名集合中的字段名一一对应。
     |      
     |      
     |      :param source: 源数据集
     |      :type source: DatasetVector or str
     |      :param str  source\_link\_field: 源数据集中的与目标数据集的关联字段。
     |      :param str target\_link\_field: 目标数据集中的与源数据集的关联字段。
     |      :param source\_fields: 源数据集中被追加到目标数据集的字段名集合。
     |      :type source\_fields: list[str] or str
     |      :param target\_fields: 追加字段在目标数据集中相对应的字段名集合。
     |      :type target\_fields: list[str] or str
     |      :return: 一个布尔值,表示追加字段是否成功,成功返回 true,否则返回 false。
     |      :rtype: bool
     |  
     |  build\_field\_index(self, field\_names, index\_name)
     |      为数据集的非空间字段创建索引
     |      
     |      :param field\_names:  非空间字段名称
     |      :type field\_names: list[str] or str
     |      :param str index\_name: 索引名称
     |      :return: 创建成功返回 true,否则返回 false
     |      :rtype: bool
     |  
     |  build\_spatial\_index(self, spatial\_index\_info)
     |      根据指定的空间索引信息或索引类型为矢量数据集创建空间索引。
     |      注意:
     |      
     |          - 数据库中的点数据集不支持四叉树(QTree)索引和 R 树索引(RTree);
     |          - 网络数据集不支持任何类型的空间索引;
     |          - 属性数据集不支持任何类型的空间索引;
     |          - 路由数据集不支持图幅索引(TILE);
     |          - 复合数据集不支持多级网格索引;
     |          - 数据库记录要大于1000条时才可以创建索引。
     |      
     |      :param spatial\_index\_info: 空间索引信息,或者空间索引类型,当为空间索引类型时,可以为枚举值或名称
     |      :type spatial\_index\_info: SpatialIndexInfo or SpatialIndexType
     |      :return: 创建索引成功返回 True,否则返回 False。
     |      :rtype: bool
     |  
     |  compute\_bounds(self)
     |      重新计算数据集的空间范围。
     |      
     |      :rtype: Rectangle
     |  
     |  create\_field(self, field\_info)
     |      创建字段
     |      
     |      :param FieldInfo field\_info:  字段信息,如果字段的类型是必填字段,必须设置默认值,没有设置默认值时,添加失败。
     |      :rtype: bool
     |      :return: 创建字段成功返回 True,否则返回 False
     |  
     |  create\_fields(self, field\_infos)
     |      创建多个字段
     |      
     |      :param list[FieldInfo] field\_infos: 字段信息集合
     |      :return: 创建字段成功返回 True,否则返回 False
     |      :rtype: bool
     |  
     |  delete\_records(self, ids)
     |      通过 ID 数组删除数据集中的记录。
     |      
     |      :param list[int] ids: 待删除记录的 ID 数组
     |      :return: 删除成功返回 True,否则返回 False
     |      :rtype: bool
     |  
     |  drop\_field\_index(self, index\_name)
     |      根据索引名指定字段,删除该字段的索引
     |      
     |      :param str index\_name: 字段索引名称
     |      :return:  删除成功返回 True,否则返回 False
     |      :rtype: bool
     |  
     |  drop\_spatial\_index(self)
     |      删除空间索引,删除成功返回 True,否则返回 False
     |      
     |      :rtype: bool
     |  
     |  get\_available\_field\_name(self, name)
     |      根据传入参数生成一个合法的字段名。
     |      
     |      :param str name: 字段名称
     |      :rtype: str
     |  
     |  get\_features(self, attr\_filter=None, has\_geometry=True, fields=None)
     |      根据指定的属性过滤条件获取要素对象
     |      
     |      :param str attr\_filter: 属性过滤条件,默认为 None,即返回当前数据集的所有要素对象
     |      :param bool has\_geometry:  是否获取几何对象,为 False 时将只返回字段值
     |      :param fields:  结果字段名称
     |      :type fields: list[str] or str
     |      :return: 满足指定条件的所有要素对象
     |      :rtype: list[Feature]
     |  
     |  get\_field\_count(self)
     |      返回数据集所有的字段的数目
     |      
     |      :rtype: int
     |  
     |  get\_field\_indexes(self)
     |      返回当前数据集属性表建的索引与建索引的字段的关系映射对象。其中键值为索引值,映射值为索引所在字段。
     |      
     |      :rtype: dict
     |  
     |  get\_field\_info(self, item)
     |      获取指定名称或序号的字段
     |      
     |      :param item:  字段名称或序号
     |      :type item: int or str
     |      :return: 字段信息
     |      :rtype: FieldInfo
     |  
     |  get\_geometries(self, attr\_filter=None)
     |      根据指定的属性过滤条件获取几何对象
     |      
     |      :param str attr\_filter: 属性过滤条件,默认为 None,即返回当前数据集的所有几何对象
     |      :return: 满足指定条件的所有几何对象
     |      :rtype: list[Geometry]
     |  
     |  get\_record\_count(self)
     |      返回矢量数据集中全部记录的数目。
     |      
     |      :rtype: int
     |  
     |  get\_recordset(self, is\_empty=False, cursor\_type=CursorType.DYNAMIC, fields=None)
     |      根据给定的参数来返回空的记录集或者返回包括所有记录的记录集对象。
     |      
     |      :param bool is\_empty: 是否返回空的记录集参数。为 true 时返回空记录集。为 false 时返回包含所有记录的记录集合对象。
     |      :param cursor\_type: 游标类型,以便用户控制查询出来的记录集的属性。当游标类型为动态时,记录集可以被修改,当游标类型为静态时,记录集为只读。可以为枚举值或名称
     |      :type cursor\_type:  CursorType or str
     |      :param fields:  需要输出的结果字段名称,如果为 None 则保留所有的字段
     |      :type fields: list[str] or str
     |      :return: 满足条件的记录集对象
     |      :rtype: Recordset
     |  
     |  get\_spatial\_index\_type(self)
     |      获取空间索引类型
     |      
     |      :rtype: SpatialIndexType
     |  
     |  get\_tolerance\_dangle(self)
     |      获取短悬线容限
     |      
     |      :rtype: float
     |  
     |  get\_tolerance\_extend(self)
     |      获取长悬线容限
     |      
     |      :rtype: float
     |  
     |  get\_tolerance\_grain(self)
     |      获取颗粒容限
     |      
     |      :rtype: float
     |  
     |  get\_tolerance\_node\_snap(self)
     |      获取节点容限
     |      
     |      :rtype: float
     |  
     |  get\_tolerance\_small\_polygon(self)
     |      获取最小多边形容限
     |      
     |      :rtype: float
     |  
     |  index\_of\_field(self, name)
     |      获取指定字段名称序号
     |      
     |      :param str name: 字段名称
     |      :return: 如果字段存在返回字段的序号,否则返回 -1
     |      :rtype: int
     |  
     |  is\_available\_field\_name(self, name)
     |      判断指定的字段名称是否是合法而且没有被占用的字段名称
     |      
     |      :param str name:  字段名称
     |      :return: 字段名称合法且没有被占用返回 True,否则返回 False
     |      :rtype: bool
     |  
     |  is\_file\_cache(self)
     |      返回是否使用文件形式的缓存。文件形式的缓存可以提高浏览速度。
     |      注意:文件形式的缓存只对 Oracle 数据源下已创建图幅索引的矢量数据集有效。
     |      
     |      :rtype: bool
     |  
     |  is\_spatial\_index\_dirty(self)
     |      判断当前数据集的空间索引是否需要重建。因为在修改数据过程后,可能需要重建空间索引。
     |      注意:
     |      
     |       - 当矢量数据集无空间索引时,如果其记录条数已达到建立空间索引的要求,则返回 True,建议用户创建空间索引;否则返回 False。
     |       - 如果矢量数据集已有空间索引(图库索引除外),但其记录条数已经不能达到建立空间索引的要求时,返回 True。
     |      
     |      :rtype: bool
     |  
     |  is\_spatial\_index\_type\_supported(self, spatial\_index\_type)
     |      判断当前数据集是否支持指定的类型的空间索引。
     |      
     |      :param spatial\_index\_type: 空间索引类型,可以为枚举值或名称
     |      :type spatial\_index\_type: SpatialIndexType or str
     |      :return: 如果支持指定的空间索引类型,返回值为 true,否则为 false。
     |      :rtype: bool
     |  
     |  query(self, query\_param=None)
     |      通过设置查询条件对矢量数据集进行查询,该方法默认查询空间信息与属性信息。
     |      
     |      :param QueryParameter query\_param: 查询条件
     |      :return: 满足查询条件的结果记录集
     |      :rtype: Recordset
     |  
     |  query\_with\_bounds(self, bounds, attr\_filter=None, cursor\_type=CursorType.DYNAMIC)
     |      根据地理范围查询记录集
     |      
     |      :param Rectangle bounds: 已知的空间范围
     |      :param str attr\_filter: 查询过滤条件,相当于 SQL 语句中的 Where 子句部分
     |      :param cursor\_type: 游标类型,可以为枚举值或名称
     |      :type cursor\_type: CursorType or str
     |      :return: 满足查询条件的结果记录集
     |      :rtype: Recordset
     |  
     |  query\_with\_distance(self, geometry, distance, unit=None, attr\_filter=None, cursor\_type=CursorType.DYNAMIC)
     |      用于查询数据集中落在指定空间对象的缓冲区内,并且满足一定条件的记录。
     |      
     |      :param geometry: 用于查询的空间对象。
     |      :type geometry: Geometry or Point2D or Rectangle
     |      :param float distance: 查询半径
     |      :param unit: 查询半径的单位,如果为 None 则查询半径的单位与数据集的单位相同。
     |      :type unit: Unit or str
     |      :param str attr\_filter:  查询过滤条件,相当于 SQL 语句中的 Where 子句部分
     |      :param cursor\_type: 游标类型,可以为枚举值或名称
     |      :type cursor\_type: CursorType or str
     |      :return: 满足查询条件的结果记录集
     |      :rtype: Recordset
     |  
     |  query\_with\_filter(self, attr\_filter=None, cursor\_type=CursorType.DYNAMIC, result\_fields=None, has\_geometry=True)
     |      根据指定的属性过滤条件查询记录集
     |      
     |      :param str attr\_filter:  查询过滤条件,相当于 SQL 语句中的 Where 子句部分
     |      :param cursor\_type: 游标类型,可以为枚举值或名称
     |      :type cursor\_type: CursorType or str
     |      :param result\_fields: 结果字段名称
     |      :type result\_fields: list[str] or str
     |      :param bool has\_geometry: 是否包含几何对象
     |      :return: 满足查询条件的结果记录集
     |      :rtype: Recordset
     |  
     |  query\_with\_ids(self, ids, id\_field\_name='SmID', cursor\_type=CursorType.DYNAMIC)
     |      根据跟定的 ID 数组,查询满足记录的记录集
     |      
     |      :param list[int] ids: ID 数组
     |      :param str id\_field\_name: 数据集中用于表示 ID 的字段名称。默认为 “SmID”
     |      :param cursor\_type:   游标类型
     |      :type cursor\_type: CursorType or str
     |      :return: 满足查询条件的结果记录集
     |      :rtype: Recordset
     |  
     |  re\_build\_spatial\_index(self)
     |      在原有的空间索引的基础上进行重建,如果原来的空间索引被破坏,那么重建成功之后还可以继续使用。
     |      
     |      :return: 重建索引成功返回 Ture,否则返回 False。
     |      :rtype: bool
     |  
     |  remove\_field(self, item)
     |      删除指定的字段
     |      
     |      :param item:  字段名称或序号
     |      :type item: int or str
     |      :return: 删除成功返回 True,否则返回 False
     |      :rtype: bool
     |  
     |  reset\_tolerance\_as\_default(self)
     |      将所有的容限设为缺省值,单位与矢量数据集坐标系单位相同:
     |      
     |       - 节点容限的默认值为数据集宽度的1/1000000;
     |       - 颗粒容限的默认值为数据集宽度的1/1000;
     |       - 短悬线容限的默认值为数据集宽度的1/10000;
     |       - 长悬线容限的默认值为数据集宽度的1/10000;
     |       - 最小多边形容限的默认值为0。
     |  
     |  set\_charset(self, value)
     |      设置数据集的字符集
     |      
     |      :param value: 数据集的字符集
     |      :type value: Charset or str
     |  
     |  set\_file\_cache(self, value)
     |      设置是否使用文件形式的缓存。
     |      
     |      :param bool value: 是否使用文件形式的缓存
     |  
     |  set\_tolerance\_dangle(self, value)
     |      设置短悬线容限
     |      
     |      :param float value: 短悬线容限
     |  
     |  set\_tolerance\_extend(self, value)
     |      设置长悬线容限
     |      
     |      :param float value: 长悬线容限
     |  
     |  set\_tolerance\_grain(self, value)
     |      设置颗粒容限
     |      
     |      :param float value: 颗粒容限
     |  
     |  set\_tolerance\_node\_snap(self, value)
     |      设置节点容限
     |      
     |      :param float value: 节点容限
     |  
     |  set\_tolerance\_small\_polygon(self, value)
     |      设置最小多边形容限
     |      
     |      :param float value:  最小多边形容限
     |  
     |  stat(self, item, stat\_mode)
     |      对指定的字段按照给定的方式进行统计。
     |      当前版本提供了6种统计方式。统计字段的最大值,最小值,平均值,总和,标准差,以及方差。
     |      当前版本支持的统计字段类型为布尔,字节,双精度,单精度,16位整型,32位整型。
     |      
     |      :param item:  字段名称或序号
     |      :type item: str or int
     |      :param stat\_mode: 字段统计模式
     |      :type stat\_mode: StatisticMode or str
     |      :return: 统计结果
     |      :rtype: float
     |  
     |  truncate(self)
     |      清除矢量数据集中的所有记录。
     |      
     |      :return: 清除记录是否成功,成功返回 True,失败返回 False。
     |      :rtype: bool
     |  
     |  update\_field(self, item, value, attr\_filter=None)
     |      根据指定的需要更新的字段名称,用指定的用于更新的字段值更新符合 attributeFilter 条件的所有记录的字段值。需要更新的字段不能够为系统字段,
     |      也就是说待更新字段不可以为 sm 开头的字段(smUserID 除外)。
     |      
     |      :param item:  字段名称或序号
     |      :type item: str or int
     |      :param value: 指定用于更新的字段值。
     |      :type value: int or float or str or datetime.datetime or bytes or bytearray
     |      :param str attr\_filter: 要更新记录的查询条件,如果 attributeFilter 为空字符串,则更新表中所有的记录
     |      :return: 更新字段成功返回 True,否则返回 False
     |      :rtype: bool
     |  
     |  update\_field\_express(self, item, express, attr\_filter=None)
     |      根据指定的需要更新的字段名,用指定的表达式计算结果更新符合查询条件的所有记录的字段值。需要更新的字段不能够为系统字段,也就是说不可以为 Sm 开头的字段(smUserID 除外)。
     |      
     |      :param item:  字段名称或序号
     |      :type item: str or int
     |      :param str express: 指定的表达式,表达式可以是字段的运算或函数的运算。例如:"SMID" 、"abs(SMID)"、"SMID+1"、 " '字符串'"。
     |      :param str attr\_filter: 要更新记录的查询条件,如果 attributeFilter 为空字符串,则更新表中所有的记录
     |      :return: 更新字段成功返回 True,否则返回 False
     |      :rtype: bool
     |  
     |  ----------------------------------------------------------------------
     |  Data descriptors defined here:
     |  
     |  charset
     |      Charset: 矢量数据集的字符集
     |  
     |  child\_dataset
     |      DatasetVector: 矢量数据集的子数据集。主要用于网络数据集
     |  
     |  field\_infos
     |      list[FieldInfo]: 数据集的所有字段信息
     |  
     |  parent\_dataset
     |      DatasetVector: 矢量数据集的父数据集。主要用于网络数据集
     |  
     |  ----------------------------------------------------------------------
     |  Methods inherited from Dataset:
     |  
     |  \_\_repr\_\_(self)
     |      Return repr(self).
     |  
     |  close(self)
     |      用于关闭当前数据集
     |  
     |  is\_open(self)
     |      判断数据集是否已经打开,数据集打开返回 True,否则返回 False
     |      
     |      :rtype: bool
     |  
     |  is\_readonly(self)
     |      判断数据集是否是只读。如果数据集是只读,无法进行任何改写数据集的操作。 数据集是只读返回 True,否则返回 False。
     |      
     |      :rtype: bool
     |  
     |  open(self)
     |      打开数据集, 打开数据集成功返返回 True,否则返回 False
     |      
     |      :rtype: bool
     |  
     |  rename(self, new\_name)
     |      修改数据集名称
     |      
     |      :param str new\_name: 新的数据集名称
     |      :return: 修改成功返回True,否则返回False
     |      :rtype: bool
     |  
     |  set\_bounds(self, rc)
     |      设置数据集中包含所有对象的最小外接矩形。对于矢量数据集来说,为数据集中所有对象的最小外接矩形;对于栅格数据集来说,为
     |      当前栅格或影像的地理范围。
     |      
     |      :param Rectangle rc: 数据集中包含所有对象的最小外接矩形。
     |      :return: self
     |      :rtype: Dataset
     |  
     |  set\_description(self, value)
     |      设置用户加入的对数据集的描述信息。
     |      
     |      :param str value: 用户加入的对数据集的描述信息。
     |  
     |  set\_prj\_coordsys(self, value)
     |      设置数据集的投影信息
     |      
     |      :param PrjCoordSys value: 投影信息
     |  
     |  to\_json(self)
     |      输出数据集的信息到 json 字符串中。数据集的 json 串内容包括数据源的连接信息和数据集名称两项。
     |      
     |      :rtype: str
     |      
     |      示例::
     |       >>> ds = Workspace().get\_datasource('data')
     |       >>> print(ds[0].to\_json())
     |       \{"name": "location", "datasource": \{"type": "UDB", "alias": "data", "server": "E:/data.udb", "is\_readonly": false\}\}
     |  
     |  ----------------------------------------------------------------------
     |  Static methods inherited from Dataset:
     |  
     |  from\_json(value)
     |      从数据集的 json 字符串中获取数据集,如果数据源没有打开,将自动打开数据源。
     |      
     |      :param str value:  json 字符串
     |      :return: 数据集对象
     |      :rtype: Dataet
     |  
     |  ----------------------------------------------------------------------
     |  Data descriptors inherited from Dataset:
     |  
     |  bounds
     |      Rectangle: 返回数据集中包含所有对象的最小外接矩形。对于矢量数据集来说,为数据集中所有对象的最小外接矩形;对于栅格数据集来说,为当前栅格或影像的
     |      地理范围。
     |  
     |  datasource
     |      Datasource : 返回当前数据集所属的数据源对象
     |  
     |  description
     |      str: 返回用户加入的对数据集的描述信息。
     |  
     |  encode\_type
     |      EncodeType: 返回此数据集数据存储时的编码方式。对数据集采用压缩编码方式,可以减少数据存储所占用的空间,降低数据传输时的网络负载和服务器的负载。矢量数
     |      据集支持的编码方式有Byte,Int16,Int24,Int32,SGL,LZW,DCT,也可以指定为不使用编码方式。光栅数据支持的编码方式有DCT,SGL,LZW
     |      或不使用编码方式。具体请参见 :py:class:`EncodeType` 类型
     |  
     |  name
     |      str: 返回数据集名称
     |  
     |  prj\_coordsys
     |      PrjCoordSys: 返回数据集的投影信息.
     |  
     |  table\_name
     |      str: 返回数据集的表名。对数据库型数据源,返回此数据集在数据库中所对应的数据表名称;对文件型数据源,返回此数据集的存储属性的表名称.
     |  
     |  type
     |      DatasetType: 返回数据集类型
     |  
     |  ----------------------------------------------------------------------
     |  Data descriptors inherited from iobjectspy.\_jsuperpy.data.\_jvm.JVMBase:
     |  
     |  \_\_dict\_\_
     |      dictionary for instance variables (if defined)
     |  
     |  \_\_weakref\_\_
     |      list of weak references to the object (if defined)
    
    class DatasetVectorInfo(builtins.object)
     |  矢量数据集信息类。包括了矢量数据集的信息,如矢量数据集的名称,数据集的类型,编码方式,是否选用文件缓存等。文件缓存只针对图幅索引而言
     |  
     |  Methods defined here:
     |  
     |  \_\_init\_\_(self, name=None, dataset\_type=None, encode\_type=None, is\_file\_cache=True)
     |      构造矢量数据集信息类
     |      
     |      :param str name: 数据集名称
     |      :param dataset\_type:  数据集类型
     |      :type dataset\_type: DatasetType or str
     |      :param encode\_type:  数据集的压缩编码方式。支持四种压缩编码方式,即单字节,双字节,三字节和四字节编码方式
     |      :type encode\_type: EncodeType or str
     |      :param bool is\_file\_cache: 是否使用文件形式的缓存。文件形式的缓存只针对图幅索引有用
     |  
     |  \_\_repr\_\_(self)
     |      Return repr(self).
     |  
     |  from\_dict(self, values)
     |      从 dict 中读取 DatasetVectorInfo 对象
     |      
     |      :param dict values: 包含矢量数据集信息的 dict 对象,具体查看 :py:meth:`to\_dict`
     |      :rtype: self
     |      :rtype: DatasetVectorInfo
     |  
     |  set\_encode\_type(self, value)
     |      设置数据集的压缩编码方式
     |      
     |      :param value: 数据集的压缩编码方式
     |      :type value: EncodeType or str
     |      :return: self
     |      :rtype: DatasetVectorInfo
     |  
     |  set\_file\_cache(self, value)
     |      设置是否使用文件形式的缓存。文件形式的缓存只针对图幅索引有用。
     |      
     |      :param bool value: 是否使用文件形式的缓存
     |      :return: self
     |      :rtype: DatasetVectorInfo
     |  
     |  set\_name(self, value)
     |      设置数据集的名称
     |      
     |      :param str value: 数据集名称
     |      :return: self
     |      :rtype: DatasetVectorInfo
     |  
     |  set\_type(self, value)
     |      设置数据集的类型
     |      
     |      :param value: 数据集类型
     |      :type value: DatasetType or str
     |      :return: self
     |      :rtype: DatasetVectorInfo
     |  
     |  to\_dict(self)
     |      将当前对象的信息输出为 dict 对象
     |      
     |      :rtype: dict
     |  
     |  ----------------------------------------------------------------------
     |  Static methods defined here:
     |  
     |  make\_from\_dict(values)
     |      从 dict 中构造 DatasetVectorInfo 对象
     |      
     |      :param dict values: 包含矢量数据集信息的 dict 对象,具体查看 :py:meth:`to\_dict`
     |      :rtype: DatasetVectorInfo
     |  
     |  ----------------------------------------------------------------------
     |  Data descriptors defined here:
     |  
     |  \_\_dict\_\_
     |      dictionary for instance variables (if defined)
     |  
     |  \_\_weakref\_\_
     |      list of weak references to the object (if defined)
     |  
     |  encode\_type
     |      EncodeType: 数据集的压缩编码方式。支持四种压缩编码方式,即单字节,双字节,三字节和四字节编码方式
     |  
     |  is\_file\_cache
     |      bool: 是否使用文件形式的缓存。文件形式的缓存只针对图幅索引有用
     |  
     |  name
     |      str:   数据集名称,数据集的名称限制:数据集名称的长度限制为30个字符(也就是可以为30个英文字母或者15个汉字),组成数据集名称的字符可以
     |      为字母、汉字、数字和下划线,数据集名称不可以用数字和下划线开头,如果用字母开头,数据集名称不可以和数据库的保留关键字冲突。
     |  
     |  type
     |      DatasetType: 数据集类型
    
    class DatasetVolume(Dataset)
     |  数据集(矢量数据集,栅格数据集,影像数据集等)的基类,提供各种数据集数据集公共的属性,方法。
     |  数据集一般为存储在一起的相关数据的集合;根据数据类型的不同,分为矢量数据集和栅格数据集,以及为了处理特定问题而设计的如拓扑数据集,网络数据集
     |  等。数据集是 GIS 数据组织的最小单位。其中矢量数据集是由同种类型空间要素组成的集合,所以也可以称为要素集。根据要素的空间特征的不同,矢量数据集
     |  又分为点数据集,线数据集,面数据集等,各矢量数据集是空间特征和性质相同而组织在一起的数据的集合。而栅格数据集由像元阵列组成,在表现要素上比矢量
     |  数据集欠缺,但是可以很好的表现空间现象的位置关系。
     |  
     |  Method resolution order:
     |      DatasetVolume
     |      Dataset
     |      iobjectspy.\_jsuperpy.data.\_jvm.JVMBase
     |      builtins.object
     |  
     |  Methods defined here:
     |  
     |  \_\_init\_\_(self)
     |      Initialize self.  See help(type(self)) for accurate signature.
     |  
     |  ----------------------------------------------------------------------
     |  Methods inherited from Dataset:
     |  
     |  \_\_repr\_\_(self)
     |      Return repr(self).
     |  
     |  \_\_str\_\_(self)
     |      Return str(self).
     |  
     |  close(self)
     |      用于关闭当前数据集
     |  
     |  is\_open(self)
     |      判断数据集是否已经打开,数据集打开返回 True,否则返回 False
     |      
     |      :rtype: bool
     |  
     |  is\_readonly(self)
     |      判断数据集是否是只读。如果数据集是只读,无法进行任何改写数据集的操作。 数据集是只读返回 True,否则返回 False。
     |      
     |      :rtype: bool
     |  
     |  open(self)
     |      打开数据集, 打开数据集成功返返回 True,否则返回 False
     |      
     |      :rtype: bool
     |  
     |  rename(self, new\_name)
     |      修改数据集名称
     |      
     |      :param str new\_name: 新的数据集名称
     |      :return: 修改成功返回True,否则返回False
     |      :rtype: bool
     |  
     |  set\_bounds(self, rc)
     |      设置数据集中包含所有对象的最小外接矩形。对于矢量数据集来说,为数据集中所有对象的最小外接矩形;对于栅格数据集来说,为
     |      当前栅格或影像的地理范围。
     |      
     |      :param Rectangle rc: 数据集中包含所有对象的最小外接矩形。
     |      :return: self
     |      :rtype: Dataset
     |  
     |  set\_description(self, value)
     |      设置用户加入的对数据集的描述信息。
     |      
     |      :param str value: 用户加入的对数据集的描述信息。
     |  
     |  set\_prj\_coordsys(self, value)
     |      设置数据集的投影信息
     |      
     |      :param PrjCoordSys value: 投影信息
     |  
     |  to\_json(self)
     |      输出数据集的信息到 json 字符串中。数据集的 json 串内容包括数据源的连接信息和数据集名称两项。
     |      
     |      :rtype: str
     |      
     |      示例::
     |       >>> ds = Workspace().get\_datasource('data')
     |       >>> print(ds[0].to\_json())
     |       \{"name": "location", "datasource": \{"type": "UDB", "alias": "data", "server": "E:/data.udb", "is\_readonly": false\}\}
     |  
     |  ----------------------------------------------------------------------
     |  Static methods inherited from Dataset:
     |  
     |  from\_json(value)
     |      从数据集的 json 字符串中获取数据集,如果数据源没有打开,将自动打开数据源。
     |      
     |      :param str value:  json 字符串
     |      :return: 数据集对象
     |      :rtype: Dataet
     |  
     |  ----------------------------------------------------------------------
     |  Data descriptors inherited from Dataset:
     |  
     |  bounds
     |      Rectangle: 返回数据集中包含所有对象的最小外接矩形。对于矢量数据集来说,为数据集中所有对象的最小外接矩形;对于栅格数据集来说,为当前栅格或影像的
     |      地理范围。
     |  
     |  datasource
     |      Datasource : 返回当前数据集所属的数据源对象
     |  
     |  description
     |      str: 返回用户加入的对数据集的描述信息。
     |  
     |  encode\_type
     |      EncodeType: 返回此数据集数据存储时的编码方式。对数据集采用压缩编码方式,可以减少数据存储所占用的空间,降低数据传输时的网络负载和服务器的负载。矢量数
     |      据集支持的编码方式有Byte,Int16,Int24,Int32,SGL,LZW,DCT,也可以指定为不使用编码方式。光栅数据支持的编码方式有DCT,SGL,LZW
     |      或不使用编码方式。具体请参见 :py:class:`EncodeType` 类型
     |  
     |  name
     |      str: 返回数据集名称
     |  
     |  prj\_coordsys
     |      PrjCoordSys: 返回数据集的投影信息.
     |  
     |  table\_name
     |      str: 返回数据集的表名。对数据库型数据源,返回此数据集在数据库中所对应的数据表名称;对文件型数据源,返回此数据集的存储属性的表名称.
     |  
     |  type
     |      DatasetType: 返回数据集类型
     |  
     |  ----------------------------------------------------------------------
     |  Data descriptors inherited from iobjectspy.\_jsuperpy.data.\_jvm.JVMBase:
     |  
     |  \_\_dict\_\_
     |      dictionary for instance variables (if defined)
     |  
     |  \_\_weakref\_\_
     |      list of weak references to the object (if defined)
    
    class Datasource(iobjectspy.\_jsuperpy.data.\_jvm.JVMBase)
     |  数据源定义了一致的数据访问接口和规范。数据源的物理存储既可以是文件方式,也可以是数据库方式。区别不同存储方式的主要依据是其所采用的数据引擎类型:
     |  采用 UDB 引擎时,数据源以文件方式存储(*.udb,*.udd)——文件型数据源文件用.udb 文件存储空间数据,采用空间数据库引擎时,数据源存储在指定的
     |  DBMS 中。每个数据源都存在于一个工作空间中,不同的数据源通过数据源别名进行区分。通过数据源对象,可以对数据集进行创建、删除、复制等操作。
     |  
     |  
     |  使用 create\_vector\_dataset 快速创建矢量数据集::
     |  
     |      >>> ds = Datasource.create('E:/data.udb')
     |      >>> location\_dt = ds.create\_vector\_dataset('location', 'Point')
     |      >>> print(location\_dt.name)
     |      location
     |  
     |  
     |  追加数据到点数据集中::
     |  
     |      >>> location\_dt.append([Point2D(1,2), Point2D(2,3), Point2D(3,4)])
     |      >>> print(location\_dt.get\_record\_count())
     |      3
     |  
     |  数据源可以直接写入几何对象,要素对象,点数据等::
     |  
     |      >>> rect = location\_dt.bounds
     |      >>> location\_coverage = ds.write\_spatial\_data([rect], 'location\_coverage')
     |      >>> print(location\_coverage.get\_record\_count())
     |      1
     |      >>> ds.close()
     |  
     |  Method resolution order:
     |      Datasource
     |      iobjectspy.\_jsuperpy.data.\_jvm.JVMBase
     |      builtins.object
     |  
     |  Methods defined here:
     |  
     |  \_\_getitem\_\_(self, item)
     |  
     |  \_\_init\_\_(self)
     |      Initialize self.  See help(type(self)) for accurate signature.
     |  
     |  \_\_repr\_\_(self)
     |      Return repr(self).
     |  
     |  \_\_str\_\_(self)
     |      Return str(self).
     |  
     |  change\_password(self, old\_password, new\_password)
     |      修改已经打开的数据源的密码
     |      
     |      :param str old\_password: 旧密码
     |      :param str new\_password: 新的密码
     |      :return: 成功返回True,否则返回False
     |      :rtype: bool
     |  
     |  close(self)
     |      关闭当前数据源。
     |      
     |      :return: 成功关闭返回 True,否则返回 False
     |      :rtype: bool
     |  
     |  contains(self, name)
     |      检查当前数据源中是否有指定名称的数据集
     |      
     |      :param str name: 数据集名称
     |      :return: 当前数据源含有指定名称的数据集返回 True,否则返回 False
     |      :rtype: bool
     |  
     |  copy\_dataset(self, source, out\_dataset\_name=None, encode\_type=None, progress=None)
     |      复制数据集。复制数据集之前必须保证当前数据源已经打开而且可写。复制数据集时,可通过 EncodeType 参数来对数据集的编码方式进行修改。
     |      有关数据集存储的编码方式请参见 EncodeType 枚举类型。由于CAD数据集不支持任何编码,对 CAD 数据集进行复制操作时设置的 EncodeType 无效
     |      
     |      :param source:  要复制的源数据集。可以为数据集对象,也可以是数据源别名和数据集名称的组合,数据源名称和数据集名称组合可以使用 "|","\textbackslash{}\textbackslash{}\textbackslash{}","/"任意一种。
     |                      例如::
     |      
     |                      >>> source = 'ds\_alias/point\_dataset'
     |      
     |                      或者::
     |      
     |                      >>> source = 'ds\_alias|point\_dataset'
     |      
     |      :type source: Dataset or str
     |      :param str out\_dataset\_name:  目标数据集的名称。当名称为空或者不合法时,会自动获取到一个合法的数据集名称
     |      :param encode\_type: 数据集的编码方式。可以为 :py:class:`EncodeType` 枚举值或名称。
     |      :type encode\_type:  EncodeType or str
     |      :param function progress: 处理进度信息的函数,具体参考 :py:class:`.StepEvent`。
     |      :return: 复制成功返回结果数据集对象,否则返回 None
     |      :rtype: Dataset
     |  
     |  create\_dataset(self, dataset\_info, adjust\_name=False)
     |      创建数据集。根据指定的数据集信息创建数据集,如果数据集名称不合法或者已经存在,创建数据集会失败,用户可以设定 adjust\_name 为 True 自动
     |      获取一个合法的数据集名称。
     |      
     |      :param dataset\_info: 数据集信息
     |      :type dataset\_info: DatasetVectorInfo or DatasetImageInfo or DatasetGridInfo
     |      :param bool adjust\_name: 当数据集名称不合法时,是否自动调整数据集名称,使用一个合法的数据集名称。默认为 False。
     |      :return: 创建成功则返回结果数据集对象,否则返回None
     |      :rtype: bool
     |  
     |  create\_dataset\_from\_template(self, template, name, adjust\_name=False)
     |      根据指定的模板数据集,创建新的数据集对象。
     |      
     |      :param Dataset template: 模板数据集
     |      :param str name: 数据集名称
     |      :param bool adjust\_name: 当数据集名称不合法时,是否自动调整数据集名称,使用一个合法的数据集名称。默认为 False。
     |      :return: 创建成功则返回结果数据集对象,否则返回None
     |      :rtype: Dataset
     |  
     |  create\_vector\_dataset(self, name, dataset\_type, adjust\_name=False)
     |      根据数据集名称和类型,创建矢量数据集对象。
     |      
     |      :param str name: 数据集名称
     |      :param dataset\_type: 数据集类型,可以为数据集类型枚举值或名称。支持 TABULAR, POINT, LINE, REGION, TEXT, CAD, POINT3D, LINE3D, REGION3D
     |      :type dataset\_type: DatasetType or str
     |      :param bool adjust\_name: 当数据集名称不合法时,是否自动调整数据集名称,使用一个合法的数据集名称。默认为 False。
     |      :return: 创建成功则返回结果数据集对象,否则返回None
     |      :rtype: DatasetVector
     |  
     |  delete(self, item)
     |      删除指定的数据集,可以为数据集名称或序号
     |      
     |      :param item: 要删除的数据集的名称或序号
     |      :type item: str or int
     |      :return: 删除数据集成功返回True,否则返回False
     |      :rtype: bool
     |  
     |  delete\_all(self)
     |      删除当前数据源下所有的数据集
     |  
     |  field\_to\_point\_dataset(self, source\_dataset, x\_field, y\_field, out\_dataset\_name=None)
     |      从一个矢量数据集的属性表中的 X、Y 坐标字段创建点数据集。即以该矢量数据集的属性表中的 X 、Y 坐标字段作为数据集的 X、Y 坐标来创建点数据集。
     |      
     |      :param source\_dataset: 关联属性表中带有坐标字段的矢量数据集
     |      :type source\_dataset: DatasetVector or str
     |      :param str x\_field: 表示点横坐标的字段。
     |      :param str y\_field: 表示点纵坐标的字段。
     |      :param str out\_dataset\_name: 目标数据集的名称。当名称为空或者不合法时,会自动获取到一个合法的数据集名称
     |      :return: 成功返回一个点数据集,否则返回None
     |      :rtype: DatasetVector
     |  
     |  flush(self, dataset\_name=None)
     |      将内存中暂未写入数据库中的数据保存到数据库
     |      
     |      :param str dataset\_name: 需要刷新的数据集名称。当传入长度为空的字符串或None,表示对所有数据集进行刷新;否则对指定名字的数据集进行刷新。
     |      :return:  成功返回True,否则返回False
     |      :rtype: bool
     |  
     |  get\_available\_dataset\_name(self, name)
     |      返回一个数据源中未被使用的数据集的名称。数据集的名称限制:数据集名称的长度限制为30个字符(也就是可以为30个英文字母或者15个汉字),
     |      组成数据集名称的字符可以为字母、汉字、数字和下划线,数据集名称不可以用数字和下划线开头,数据集名称不可以和数据库的保留关键字冲突。
     |      
     |      :param str name: 数据集名称
     |      :return: 合法的数据集名称
     |      :rtype: str
     |  
     |  get\_count(self)
     |      获取数据集的个数
     |      
     |      :rtype: int
     |  
     |  get\_dataset(self, item)
     |      根据数据集名称或序号,获取数据集对象
     |      
     |      :param item:  数据集的名称或序号
     |      :type: str or int
     |      :return: 数据集对象
     |      :rtype: Dataset
     |  
     |  index\_of(self, name)
     |      返回给定数据集名称对应的数据集在数据集集合中所处的索引值
     |      
     |      :param str name: 数据集名称
     |      :rtype: int
     |  
     |  inner\_point\_to\_dataset(self, source\_dataset, out\_dataset\_name=None)
     |      创建矢量数据集的内点数据集,并把矢量数据集中几何对象的属性复制到相应的点数据集属性表中
     |      
     |      :param source\_dataset: 要计算内点数据集的矢量数据集
     |      :type source\_dataset: DatasetVector or str
     |      :param str out\_dataset\_name: 目标数据集的名称。当名称为空或者不合法时,会自动获取到一个合法的数据集名称
     |      :return: 创建成功返回内点数据集。创建失败返回 None
     |      :rtype: DatasetVector
     |  
     |  is\_available\_dataset\_name(self, name)
     |      判断用户传进来的数据集的名称是否合法。创建数据集时应检查其名称的合法性。
     |      
     |      :param str name: 待检查的数据集名称
     |      :return: 如果数据集名称合法,返回 True,否则返回 False
     |      :rtype: bool
     |  
     |  is\_opened(self)
     |      返回数据源是否打开的状态,如果数据源处于打开状态,返回 true,如果数据源被关闭,则返回 false。
     |      
     |      :return: 数据源是否打开的状态
     |      :rtype: bool
     |  
     |  is\_readonly(self)
     |      返回数据源是否以只读方式打开。对文件型数据源,如果只读方式打开,就是共享的,可以打开多次;如果以非只读方式打开,则只能打开一次。
     |      对于影像数据源(IMAGEPLUGINS 引擎类型)只会以只读方式打开。
     |      
     |      :return: 数据源是否以只读方式打开
     |      :rtype: bool
     |  
     |  label\_to\_text\_dataset(self, source\_dataset, text\_field, text\_style=None, out\_dataset\_name=None)
     |      用于将数据集的属性字段生成一个文本数据集。通过此方法生成的文本数据集中的文本对象,均以其对应的空间对象的内点作为对应的锚点,
     |      对应的空间对象即当前文本对象的内容来源于相应空间对象的属性值。
     |      
     |      :param source\_dataset: 要计算内点数据集的矢量数据集
     |      :type source\_dataset: DatasetVector or str
     |      :param str text\_field: 要转换的属性字段的名称。
     |      :param TextStyle text\_style:  结果文本对象的风格
     |      :param str out\_dataset\_name: 目标数据集的名称。当名称为空或者不合法时,会自动获取到一个合法的数据集名称
     |      :return:  成功返回一个文本数据集,否则返回None
     |      :rtype: DatasetVector
     |  
     |  last\_date\_updated(self)
     |      获取数据源最后更新的时间
     |      
     |      :rtype: datetime.datetime
     |  
     |  refresh(self)
     |      对数据库类型的数据源进行刷新
     |  
     |  set\_description(self, description)
     |      设置用户添加的关于数据源的描述信息。用户可以在描述信息里加入你想加入的任何信息,例如建立数据源的人员、数据的来源、数据的主要内容、
     |      数据的精度、质量等信息,这些信息对于维护数据具有重要的意义
     |      
     |      :param str description: 用户添加的关于数据源的描述信息
     |  
     |  set\_prj\_coordsys(self, prj)
     |      设置数据源的投影信息
     |      
     |      :param PrjCoordSys prj: 投影信息
     |  
     |  to\_json(self)
     |      将数据源返回为 json 格式字符串。具体返回数据源连接信息的 json 字符串,即使用 :py:class:`DatasourceConnectionInfo.to\_json` 。
     |      
     |      :rtype: str
     |  
     |  write\_attr\_values(self, data, out\_dataset\_name=None)
     |      写入属性数据到属性数据集(DatasetType.TABULAR)中。
     |      
     |      :param data: 待写入的数据。data 必须为一个 list 或者 tuple 或者 set, list(或tuple) 中每个元素项,可以为 list or tuple, 此时
     |                   data 相当于一个二维的数组,例如::
     |      
     |                   >>> data = [[1,2.0,'a1'], [2,3.0,'a2'], [3,4.0,'a3']]
     |      
     |                   或者::
     |      
     |                   >>> data = [(1,2.0,'a1'), (2,3.0,'a2'), (3,4.0,'a3')]
     |      
     |                   data 中元素项如果不是 list 和 tuple,将会被当作一个元素对待。例如::
     |      
     |                   >>> data = [1,2,3]
     |      
     |                   或::
     |      
     |                   >>> data = ['test1','test2','test3']
     |      
     |                   则最后结果数据集将会含有1列3行的数据集。而对于data中的元素项为 dict 这种,则会将每个 dict 对象作为一个字符串写入::
     |      
     |                   >>> data = [\{1:'a'\}, \{2:'b'\}, \{3:'c'\}]
     |      
     |                   等价于写入了::
     |      
     |                   >>> data = ["\{1: 'a'\}", "\{2: 'b'\}", "\{3: 'c'\}"]
     |      
     |                   另外,用户需要确保list中每个元素项结构相同。程序内部会自动采样最多20条记录,根据采样的字段值类型计算合理的字段类型,具体对应为:
     |      
     |                    - int: FieldType.INT64
     |                    - str: FieldType.WTEXT
     |                    - float: FieldType.DOUBLE
     |                    - bool: FieldType.BOOLEAN
     |                    - datetime.datetime: FieldType.DATETIME
     |                    - bytes: FieldType.LONGBINARY
     |                    - bytearray: FieldType.LONGBINARY
     |                    - 其他: FieldType.WTEXT
     |      
     |      :type data: list or tuple
     |      :param str out\_dataset\_name: 结果数据集名称。当名称为空或者不合法时,会自动获取到一个合法的数据集名称
     |      :return: 写入数据成功返回 DatasetVector,否则返回 None
     |      :rtype: DatasetVector
     |  
     |  write\_features(self, data, out\_dataset\_name=None)
     |      将要素对象写入到数据集中。
     |      
     |      :param data: 写入的要素对象集合。用户需确保集合中所有要素的结构必须相同,包括几何对象类型和字段信息都相同。
     |      :type data: list[Feature] or tuple[Feature]
     |      :param str out\_dataset\_name: 结果数据集名称。当名称为空或者不合法时,会自动获取到一个合法的数据集名称
     |      :return: 写入成功返回结果数据集对象,否则返回 None
     |      :rtype: DatasetVector
     |  
     |  write\_recordset(self, source, out\_dataset\_name=None)
     |      将一个记录集对象或数据集对象写到当前数据源中。
     |      
     |      :param source: 待写入的记录集或数据集对象
     |      :param str out\_dataset\_name:  结果数据集名称。当名称为空或者不合法时,会自动获取到一个合法的数据集名称
     |      :return:  写入数据成功返回 DatasetVector,否则返回 None
     |      :rtype: DatasetVector
     |  
     |  write\_spatial\_data(self, data, out\_dataset\_name=None, values=None)
     |      将空间数据(Point2D, Point3D, Rectangle, Geometry) 等写入到矢量数据集中。
     |      
     |      :param data:  待写入的数据。data 必须为一个 list 或者 tuple 或者 set, list(或tuple) 中每个元素项,可以为 Point2D, Point, GeoPoint, GeoLine, GeoRegion, Rectangle:
     |                      - 如果 data 中所有的元素都是 Point2D 或 GeoPoint,则会创建一个点数据集
     |                      - 如果 data 中所有的元素都是 Point3D 或 GeoPoint3D,则会创建一个三维点数据集
     |                      - 如果 data 中所有的元素都是 GeoLine,则会创建一个线数据集
     |                      - 如果 data 中所有的元素都是 Rectangle 或 GeoRegion,则会创建一个面数据集
     |                      - 否则,将会创建一个 CAD 数据集。
     |      :type data: list o tuple
     |      :param str out\_dataset\_name:  结果数据集名称。当名称为空或者不合法时,会自动获取到一个合法的数据集名称
     |      :param values:  空间数据要写到数据集中的属性字段值。如果不为 None,必须为 list 或 tuple,且长度必须与 data 长度相同。
     |                      values 中每个元素项可以为 list 或 tuple,此时 values 相当于一个二维的数组,例如::
     |      
     |                      >>> values = [[1,2.0,'a1'], [2,3.0,'a2'], [3,4.0,'a3']]
     |      
     |                      或者
     |      
     |                      >>> values = [(1,2.0,'a1'), (2,3.0,'a2'), (3,4.0,'a3')]
     |      
     |                      values 中元素项如果不是 list 和 tuple,将会被当作一个元素对待。例如::
     |      
     |                      >>> values = [1,2,3]
     |      
     |                      或::
     |      
     |                      >>> values = ['test1','test2','test3']
     |      
     |                      则最后结果数据集将会含有1列3行的数据集。而对于data中的元素项为 dict 这种,则会将每个 dict 对象作为一个字符串写入::
     |      
     |                      >>> data = [\{1:'a'\}, \{2:'b'\}, \{3:'c'\}]
     |      
     |                      等价于写入了::
     |      
     |                      >>> values = ["\{1: 'a'\}", "\{2: 'b'\}", "\{3: 'c'\}"]
     |      
     |                      另外,用户需要确保list中每个元素项结构相同。程序内部会自动采样最多20条记录,根据采样的字段值类型计算合理的字段类型,具体对应为:
     |      
     |                       - int: FieldType.INT64
     |                       - str: FieldType.WTEXT
     |                       - float: FieldType.DOUBLE
     |                       - bool: FieldType.BOOLEAN
     |                       - datetime.datetime: FieldType.DATETIME
     |                       - bytes: FieldType.LONGBINARY
     |                       - bytearray: FieldType.LONGBINARY
     |                       - 其他: FieldType.WTEXT
     |      
     |      
     |      :return: 写入数据成功返回 DatasetVector,否则返回 None
     |      :rtype: DatasetVector
     |  
     |  ----------------------------------------------------------------------
     |  Static methods defined here:
     |  
     |  create(conn\_info)
     |      根据指定的数据源连接信息,创建新的数据源。
     |      
     |      :param conn\_info: 数据源连接信息,具体可以参考 :py:meth:`DatasourceConnectionInfo.make`
     |      :type conn\_info:  str or dict or DatasourceConnectionInfo
     |      :return:  数据源对象
     |      :rtype: Datasource
     |  
     |  from\_json(value)
     |      从 json 格式字符串打开数据源。json 串格式为 DatasourceConnectionInfo 的 json 字符串格式。具体参
     |      :py:meth:`DatasourceConnectionInfo.to\_json` 和 :py:meth:`to\_json`
     |      
     |      :param str value: json 字符串格式
     |      :return: 数据源对象
     |      :rtype: Datasource
     |  
     |  open(conn\_info)
     |      根据数据源连接信息打开数据源。如果设置的连接信息是UDB类型数据源。则会直接返回。不支持直接打开内存数据源,要使用内存数据源,需要使用 :py:meth:`create` 。
     |      
     |      :param conn\_info: 数据源连接信息,具体可以参考 :py:meth:`DatasourceConnectionInfo.make`
     |      :type conn\_info:  str or dict or DatasourceConnectionInfo
     |      :return:  数据源对象
     |      :rtype: Datasource
     |  
     |  ----------------------------------------------------------------------
     |  Data descriptors defined here:
     |  
     |  alias
     |      str: 数据源的别名。别名用于在工作空间中唯一标识数据源,可以通过它访问数据源。数据源的别名在创建数据源或打开数据源时给定,
     |      打开同一个数据源可以使用不同的别名。
     |  
     |  connection\_info
     |      DatasourceConnectionInfo : 数据源连接信息
     |  
     |  datasets
     |      list[Dataset]:  当前数据源中所有的数据集对象
     |  
     |  description
     |      str: 返回用户添加的关于数据源的描述信息
     |  
     |  prj\_coordsys
     |      PrjCoordSys: 获取数据源的投影信息
     |  
     |  type
     |      EngineType: 数据源引擎类型
     |  
     |  workspace
     |      Workspace: 当前数据源所属的工作空间对象
     |  
     |  ----------------------------------------------------------------------
     |  Data descriptors inherited from iobjectspy.\_jsuperpy.data.\_jvm.JVMBase:
     |  
     |  \_\_dict\_\_
     |      dictionary for instance variables (if defined)
     |  
     |  \_\_weakref\_\_
     |      list of weak references to the object (if defined)
    
    class DatasourceConnectionInfo(iobjectspy.\_jsuperpy.data.\_jvm.JVMBase)
     |  数据源连接信息类。包括了进行数据源连接的所有信息,如所要连接的服务器名称,数据库名称、用户名、密码等。当保存工作空间时,工作空间中的数据源的
     |  连接信息都将存储到工作空间文件中。对于不同类型的数据源,其连接信息有所区别。所以在使用该类所包含的成员时,请注意该成员所适用的数据源类型。
     |  
     |  例如::
     |      >>> conn\_info = DatasourceConnectionInfo('E:/data.udb')
     |      >>> print(conn\_info.server)
     |      'E:\textbackslash{}data.udb'
     |      >>> print(conn\_info.type)
     |      EngineType.UDB
     |  
     |  
     |  创建 OraclePlus 数据库连接信息::
     |  
     |      >>> conn\_info = (DatasourceConnectionInfo().
     |      >>>                    set\_type(EngineType.ORACLEPLUS).
     |      >>>                    set\_server('server').
     |      >>>                    set\_database('database').
     |      >>>                    set\_alias('alias').
     |      >>>                    set\_user('user').
     |      >>>                    set\_password('password'))
     |      >>> print(conn\_info.database)
     |      'server'
     |  
     |  Method resolution order:
     |      DatasourceConnectionInfo
     |      iobjectspy.\_jsuperpy.data.\_jvm.JVMBase
     |      builtins.object
     |  
     |  Methods defined here:
     |  
     |  \_\_init\_\_(self, server=None, engine\_type=None, alias=None, is\_readonly=None, database=None, driver=None, user=None, password=None)
     |      构造数据源连接对象
     |      
     |      :param str server: 数据库服务器名、文件名或服务地址:
     |      
     |                         - 对于 MEMORY, 为 ':memory:'
     |                         - 对于 UDB 文件,为其文件的绝对路径。注意:当绝对路径的长度超过 UTF-8 编码格式的260字节长度,该数据源无法打开。
     |                         - 对于 Oracle 数据库,其服务器名为其 TNS 服务名称;
     |                         - 对于 SQL Server 数据库,其服务器名为其系统的 DSN(Database Source Name)名称;
     |                         - 对于 PostgreSQL 数据库,其服务器名为"IP:端口号",默认的端口号是 5432;
     |                         - 对于 DB2 数据库,已经进行了编目,所以不需要进行服务器的设置;
     |                         - 对于 Kingbase 数据库,其服务器名为其 IP 地址;
     |                         - 对于 GoogleMaps 数据源,为其服务地址,默认设置为 "http://maps.google.com",且不可更改;
     |                         - 对于 SuperMapCloud 数据源,为其服务地址;
     |                         - 对于 MAPWORLD 数据源,为其服务地址,默认设置为 "http://www.tianditu.cn",且不可更改;
     |                         - 对于 OGC 和 REST 数据源,为其服务地址。
     |                         - 若用户设置为 IMAGEPLUGINS 时,将此方法的参数设置为地图缓存配置文件(SCI)名称,则用户可以实现对地图缓存的加载
     |      
     |      :param engine\_type: 数据源连接的引擎类型,可以使用 EngineType 枚举值和名称
     |      :type engine\_type: EngineType or str
     |      :param str alias: 数据源别名。别名是数据源的唯一标识。该标识不区分大小写
     |      :param bool is\_readonly: 是否以只读方式打开数据源。如果以只读方式打开数据源,数据源的相关信息以及其中的数据都不可修改。
     |      :param str database: 数据源连接的数据库名
     |      :param str driver: 数据源连接所需的驱动名称:
     |      
     |                             - 对于SQL Server 数据库,它使用 ODBC 连接,返回的驱动程序名为 SQL Server 或 SQL Native Client。
     |                             - 对于 iServer 发布的 WMTS 服务,返回的驱动名称为 WMTS。
     |      
     |      :param str user: 登录数据库的用户名。对于数据库类型数据源适用。
     |      :param str password: 登录数据源连接的数据库或文件的密码。对于 GoogleMaps 数据源,如果打开的是基于早期版本的数据源,则返回的密码为用户在 Google 官网注册后获取的密钥
     |  
     |  \_\_repr\_\_(self)
     |      Return repr(self).
     |  
     |  \_\_str\_\_(self)
     |      Return str(self).
     |  
     |  from\_dict(self, values)
     |      从 dict 对象中读取数据库数据源连接信息。读取后会覆盖当前对象中的值。
     |      
     |      :param dict values: 包含数据源连接信息的 dict.
     |      :return: self
     |      :rtype: DatasourceConnectionInfo
     |  
     |  is\_same(self, other)
     |      判断当前对象与指定的数据库连接信息对象是否是指向同一个数据源对象。
     |      如果两个数据库连接信息指向同一个数据源,则必须:
     |      
     |          - 数据库引擎类型 (type) 相同
     |          - 数据库服务器名、文件名或服务地址 (server) 相同
     |          - 数据库连接的数据库名称 (database) 相同如果需要设置。
     |          - 数据库的用户名 (user) 相同,如果需要设置。
     |          - 数据源连接的数据库或文件的密码 (password) 相同,如果需要设置。
     |          - 是否以只读方式打开 (is\_readonly) 相同,如果需要设置。
     |      
     |      :param DatasourceConnectionInfo other: 需要比较的数据库连接信息对象。
     |      :return: 返回 True 表示与指定的数据库连接信息是指向同一个数据源对象。否则为 False
     |      :rtype: bool
     |  
     |  save\_as\_dcf(self, file\_path)
     |      将当前数据集连接信息对象保存到 dcf 文件中。
     |      
     |      :param str file\_path: dcf 文件路径。
     |      :return: 成功保存返回 True, 否则返回 False
     |      :type: bool
     |  
     |  set\_alias(self, value)
     |      设置数据源别名
     |      
     |      :param str value:  别名是数据源的唯一标识。该标识不区分大小写
     |      :return: self
     |      :rtype: DatasourceConnectionInfo
     |  
     |  set\_database(self, value)
     |      设置数据源连接的数据库名。对于数据库类型数据源适用
     |      
     |      :param str value:  数据源连接的数据库名。
     |      :return: self
     |      :rtype: DatasourceConnectionInfo
     |  
     |  set\_driver(self, value)
     |      设置数据源连接所需的驱动名称。
     |      
     |      :param str value: 数据源连接所需的驱动名称:
     |      
     |                            - 对于SQL Server 数据库,它使用 ODBC 连接,所设置的驱动程序名为 SQL Server 或 SQL Native Client。
     |                            - 对于 iServer 发布的 WMTS 服务,设置的驱动名称为 WMTS,并且该方法必须调用该方法设置其驱动名称。
     |      
     |      :return: self
     |      :rtype: DatasourceConnectionInfo
     |  
     |  set\_password(self, value)
     |      设置登录数据源连接的数据库或文件的密码
     |      
     |      :param str value:  登录数据源连接的数据库或文件的密码。对于 GoogleMaps 数据源,如果打开的是基于早期版本的数据源,则需要输入密码,其密码为用户在 Google 官网注册后获取的密钥。
     |      :return: self
     |      :rtype: DatasourceConnectionInfo
     |  
     |  set\_readonly(self, value)
     |      设置是否以只读方式打开数据源。
     |      
     |      :param bool value:  指定是否以只读方式打开数据源。对于 UDB 数据源,如果其文件属性为只读的,必须设置为只读时才能打开。
     |      :return: self
     |      :rtype: DatasourceConnectionInfo
     |  
     |  set\_server(self, value)
     |      设置数据库服务器名、文件名或服务地址
     |      
     |      :param str value: 数据库服务器名、文件名或服务地址:
     |      
     |                            - 对于 MEMORY, 为 ':memory:'
     |                            - 对于 UDB 文件,为其文件的绝对路径。注意:当绝对路径的长度超过 UTF-8 编码格式的260字节长度,该数据源无法打开。
     |                            - 对于 Oracle 数据库,其服务器名为其 TNS 服务名称;
     |                            - 对于 SQL Server 数据库,其服务器名为其系统的 DSN(Database Source Name)名称;
     |                            - 对于 PostgreSQL 数据库,其服务器名为"IP:端口号",默认的端口号是 5432;
     |                            - 对于 DB2 数据库,已经进行了编目,所以不需要进行服务器的设置;
     |                            - 对于 Kingbase 数据库,其服务器名为其 IP 地址;
     |                            - 对于 GoogleMaps 数据源,为其服务地址,默认设置为"http://maps.google.com",且不可更改;
     |                            - 对于 SuperMapCloud 数据源,为其服务地址;
     |                            - 对于 MAPWORLD 数据源,为其服务地址,默认设置为"http://www.tianditu.cn",且不可更改;
     |                            - 对于 OGC 和 REST 数据源,为其服务地址。
     |                            - 若用户设置为 IMAGEPLUGINS 时,将此方法的参数设置为地图缓存配置文件(SCI)名称,则用户可以实现对地图缓存的加载
     |      
     |      :return: self
     |      :rtype: DatasourceConnectionInfo
     |  
     |  set\_type(self, value)
     |      设置数据源连接的引擎类型。
     |      
     |      :param value: 数据源连接的引擎类型
     |      :type value: EngineType or str
     |      :return: self
     |      :rtype: DatasourceConnectionInfo
     |  
     |  set\_user(self, value)
     |      设置登录数据库的用户名。对于数据库类型数据源适用
     |      
     |      :param str value: 登录数据库的用户名
     |      :return: self
     |      :rtype: DatasourceConnectionInfo
     |  
     |  to\_dict(self)
     |      将当前数据源连接信息输出为 dict 对象。
     |      
     |      :return: 包含数据源连接信息的 dict
     |      :rtype: dict
     |      
     |      示例::
     |          >>> conn\_info = (DatasourceConnectionInfo().
     |          >>>                set\_type(EngineType.ORACLEPLUS).
     |          >>>                set\_server('oracle\_server').
     |          >>>                set\_database('database\_name').
     |          >>>                set\_alias('alias\_name').
     |          >>>                set\_user('user\_name').
     |          >>>                set\_password('password\_123'))
     |          >>>
     |          >>> print(conn\_info.to\_dict())
     |          \{'type': 'ORACLEPLUS', 'alias': 'alias\_name', 'server': 'oracle\_server', 'user': 'user\_name', 'is\_readonly': False, 'password': 'password\_123', 'database': 'database\_name'\}
     |  
     |  to\_json(self)
     |      输出为 json 格式字符串
     |      
     |      :return:  json 格式字符串
     |      :rtype: str
     |  
     |  to\_xml(self)
     |      将当前数据集连接信息输出为 xml 字符串
     |      
     |      :return:  由当前数据源连接信息对象转换而得到的 XML 字符串。
     |      :rtype: str
     |  
     |  ----------------------------------------------------------------------
     |  Static methods defined here:
     |  
     |  from\_json(value)
     |      从 json 字符串构造数据源连接信息对象。
     |      
     |      :param str value:    json 字符串
     |      :return: 数据源连接信息对象
     |      :rtype: DatasourceConnectionInfo
     |  
     |  load\_from\_dcf(file\_path)
     |      从 dcf 文件中加载数据库连接信息,返回一个新的数据库连接信息对象。
     |      
     |      :param str file\_path: dcf 文件路径。
     |      :return: 数据源连接信息对象
     |      :type: DatasourceConnectionInfo
     |  
     |  load\_from\_xml(xml)
     |      从指定的 xml 字符串中加载数据库连接信息,并返回一个新的数据库连接信息对象。
     |      
     |      :param str xml: 导入的数据源的连接信息的 xml 字符串
     |      :return: 数据源连接信息对象
     |      :type: DatasourceConnectionInfo
     |  
     |  make(value)
     |      构造数据库连接信息对象。
     |      
     |      :param value: 包含数据源连接信息的对象:
     |      
     |                        - 如果是 DatasourceConnectionInfo 对象,则直接返回对象。
     |                        - 如果是 dict, 参考 make\_from\_dict
     |                        - 如果是 str,可以是:
     |      
     |                            - ':memory:',返回内存数据源引擎的数据库连接信息
     |                            - udb 或 udd 文件, 返回UDB数据源引擎的数据库连接信息
     |                            - dcf 文件,参考 save\_as\_dcf
     |                            - xml 字符串,参考 to\_xml
     |      
     |      :type value: str or DatasourceConnectionInfo or dict
     |      :return: 数据源连接信息对象
     |      :rtype: DatasourceConnectionInfo
     |  
     |  make\_from\_dict(values)
     |      从 dict 对象中构造数据源连接对象。返回一个新的数据库连接信息对象。
     |      
     |      :param dict values: 包含数据源连接信息的 dict.
     |      :return: 数据源连接信息对象
     |      :rtype: DatasourceConnectionInfo
     |  
     |  ----------------------------------------------------------------------
     |  Data descriptors defined here:
     |  
     |  alias
     |      str: 数据源别名,别名是数据源的唯一标识。该标识不区分大小写
     |  
     |  database
     |      str: 数据源连接的数据库名
     |  
     |  driver
     |      str: 数据源连接所需的驱动名称
     |  
     |  is\_readonly
     |      bool: 是否以只读方式打开数据源
     |  
     |  password
     |      str: 登录数据源连接的数据库或文件的密码
     |  
     |  server
     |      str: 数据库服务器名、文件名或服务地址
     |  
     |  type
     |      EngineType: 数据源类型
     |  
     |  user
     |      str: 登录数据库的用户名
     |  
     |  ----------------------------------------------------------------------
     |  Data descriptors inherited from iobjectspy.\_jsuperpy.data.\_jvm.JVMBase:
     |  
     |  \_\_dict\_\_
     |      dictionary for instance variables (if defined)
     |  
     |  \_\_weakref\_\_
     |      list of weak references to the object (if defined)
    
    class DatasourceCreatedFailedError(builtins.Exception)
     |  数据源创建失败异常
     |  
     |  Method resolution order:
     |      DatasourceCreatedFailedError
     |      builtins.Exception
     |      builtins.BaseException
     |      builtins.object
     |  
     |  Methods defined here:
     |  
     |  \_\_init\_\_(self, message)
     |      Initialize self.  See help(type(self)) for accurate signature.
     |  
     |  \_\_repr\_\_ = \_\_str\_\_(self)
     |  
     |  \_\_str\_\_(self)
     |      Return str(self).
     |  
     |  ----------------------------------------------------------------------
     |  Data descriptors defined here:
     |  
     |  \_\_weakref\_\_
     |      list of weak references to the object (if defined)
     |  
     |  ----------------------------------------------------------------------
     |  Methods inherited from builtins.Exception:
     |  
     |  \_\_new\_\_(*args, **kwargs) from builtins.type
     |      Create and return a new object.  See help(type) for accurate signature.
     |  
     |  ----------------------------------------------------------------------
     |  Methods inherited from builtins.BaseException:
     |  
     |  \_\_delattr\_\_(self, name, /)
     |      Implement delattr(self, name).
     |  
     |  \_\_getattribute\_\_(self, name, /)
     |      Return getattr(self, name).
     |  
     |  \_\_reduce\_\_({\ldots})
     |      helper for pickle
     |  
     |  \_\_setattr\_\_(self, name, value, /)
     |      Implement setattr(self, name, value).
     |  
     |  \_\_setstate\_\_({\ldots})
     |  
     |  with\_traceback({\ldots})
     |      Exception.with\_traceback(tb) --
     |      set self.\_\_traceback\_\_ to tb and return self.
     |  
     |  ----------------------------------------------------------------------
     |  Data descriptors inherited from builtins.BaseException:
     |  
     |  \_\_cause\_\_
     |      exception cause
     |  
     |  \_\_context\_\_
     |      exception context
     |  
     |  \_\_dict\_\_
     |  
     |  \_\_suppress\_context\_\_
     |  
     |  \_\_traceback\_\_
     |  
     |  args
    
    class DatasourceOpenedFailedError(builtins.Exception)
     |  数据源打开失败异常
     |  
     |  Method resolution order:
     |      DatasourceOpenedFailedError
     |      builtins.Exception
     |      builtins.BaseException
     |      builtins.object
     |  
     |  Methods defined here:
     |  
     |  \_\_init\_\_(self, message)
     |      Initialize self.  See help(type(self)) for accurate signature.
     |  
     |  \_\_repr\_\_ = \_\_str\_\_(self)
     |  
     |  \_\_str\_\_(self)
     |      Return str(self).
     |  
     |  ----------------------------------------------------------------------
     |  Data descriptors defined here:
     |  
     |  \_\_weakref\_\_
     |      list of weak references to the object (if defined)
     |  
     |  ----------------------------------------------------------------------
     |  Methods inherited from builtins.Exception:
     |  
     |  \_\_new\_\_(*args, **kwargs) from builtins.type
     |      Create and return a new object.  See help(type) for accurate signature.
     |  
     |  ----------------------------------------------------------------------
     |  Methods inherited from builtins.BaseException:
     |  
     |  \_\_delattr\_\_(self, name, /)
     |      Implement delattr(self, name).
     |  
     |  \_\_getattribute\_\_(self, name, /)
     |      Return getattr(self, name).
     |  
     |  \_\_reduce\_\_({\ldots})
     |      helper for pickle
     |  
     |  \_\_setattr\_\_(self, name, value, /)
     |      Implement setattr(self, name, value).
     |  
     |  \_\_setstate\_\_({\ldots})
     |  
     |  with\_traceback({\ldots})
     |      Exception.with\_traceback(tb) --
     |      set self.\_\_traceback\_\_ to tb and return self.
     |  
     |  ----------------------------------------------------------------------
     |  Data descriptors inherited from builtins.BaseException:
     |  
     |  \_\_cause\_\_
     |      exception cause
     |  
     |  \_\_context\_\_
     |      exception context
     |  
     |  \_\_dict\_\_
     |  
     |  \_\_suppress\_context\_\_
     |  
     |  \_\_traceback\_\_
     |  
     |  args
    
    class DatasourceReadOnlyError(builtins.Exception)
     |  数据源为只读时的异常。在进行某些功能需要写入数据到数据源或修改数据源中数据,而数据源为只读时将会返回该异常信息。
     |  
     |  Method resolution order:
     |      DatasourceReadOnlyError
     |      builtins.Exception
     |      builtins.BaseException
     |      builtins.object
     |  
     |  Methods defined here:
     |  
     |  \_\_init\_\_(self, message)
     |      Initialize self.  See help(type(self)) for accurate signature.
     |  
     |  \_\_repr\_\_ = \_\_str\_\_(self)
     |  
     |  \_\_str\_\_(self)
     |      Return str(self).
     |  
     |  ----------------------------------------------------------------------
     |  Data descriptors defined here:
     |  
     |  \_\_weakref\_\_
     |      list of weak references to the object (if defined)
     |  
     |  ----------------------------------------------------------------------
     |  Methods inherited from builtins.Exception:
     |  
     |  \_\_new\_\_(*args, **kwargs) from builtins.type
     |      Create and return a new object.  See help(type) for accurate signature.
     |  
     |  ----------------------------------------------------------------------
     |  Methods inherited from builtins.BaseException:
     |  
     |  \_\_delattr\_\_(self, name, /)
     |      Implement delattr(self, name).
     |  
     |  \_\_getattribute\_\_(self, name, /)
     |      Return getattr(self, name).
     |  
     |  \_\_reduce\_\_({\ldots})
     |      helper for pickle
     |  
     |  \_\_setattr\_\_(self, name, value, /)
     |      Implement setattr(self, name, value).
     |  
     |  \_\_setstate\_\_({\ldots})
     |  
     |  with\_traceback({\ldots})
     |      Exception.with\_traceback(tb) --
     |      set self.\_\_traceback\_\_ to tb and return self.
     |  
     |  ----------------------------------------------------------------------
     |  Data descriptors inherited from builtins.BaseException:
     |  
     |  \_\_cause\_\_
     |      exception cause
     |  
     |  \_\_context\_\_
     |      exception context
     |  
     |  \_\_dict\_\_
     |  
     |  \_\_suppress\_context\_\_
     |  
     |  \_\_traceback\_\_
     |  
     |  args
    
    class Feature(builtins.object)
     |  特征要素对象,特征要素对象可以用于描述空间信息和属性信息,也可以只用于属性信息的描述。
     |  
     |  Methods defined here:
     |  
     |  \_\_getitem\_\_(self, item)
     |  
     |  \_\_getstate\_\_(self)
     |  
     |  \_\_init\_\_(self, geometry=None, values=None, id\_value='0', field\_infos=None)
     |      :param  Geometry geometry: 几何对象信息
     |      :param values: 特征要素对象的属性字段值。
     |      :type values: list or tuple or dict
     |      :param str id\_value: 要素对象 ID
     |      :param list[FieldInfo] field\_infos: 特征要素对con象的属性字段信息
     |  
     |  \_\_setitem\_\_(self, key, value)
     |  
     |  \_\_setstate\_\_(self, state)
     |  
     |  add\_field\_info(self, field\_info)
     |      增加一个属性字段。增加一个属性字段后,如果没有属性字段没有设置默认值,将会将属性值设为 None
     |      
     |      :param FieldInfo field\_info: 属性字段信息
     |      :return: 添加成功返回True,否则返回False
     |      :rtype: bool
     |  
     |  clone(self)
     |      复制当前对象
     |      
     |      :rtype: Feature
     |  
     |  get\_field\_info(self, item)
     |      获取指定名称和序号的字段信息
     |      
     |      :param item: 字段名称或序号
     |      :type item: str or int
     |      :rtype: FieldInfo
     |  
     |  get\_value(self, item)
     |      获取当前对象中指定的属性字段的字段值
     |      
     |      :param item: 字段名称或序号
     |      :type item: str or int
     |      :rtype: int or float or str or datetime.datetime or bytes or bytearray
     |  
     |  get\_values(self, exclude\_system=True, is\_dict=False)
     |      获取当前对象的属性字段值。
     |      
     |      :param bool exclude\_system: 是否包含系统字段。所有 "Sm" 开头的字段都是系统字段。默认为 True
     |      :param bool is\_dict: 是否以 dict 形式返回,如果返回 dict,则 dict  的 key 为字段名称, value 为属性字段值。否则以 list 形式返回字段值。默认为 False
     |      :return: 属性字段值
     |      :rtype: dict or list
     |  
     |  remove\_field\_info(self, name)
     |      删除指定字段名称或序号的字段。删除字段后,字段值也会被删除
     |      
     |      :param name: 字段名称或序号
     |      :type name: int or str
     |      :return: 删除成功返回 True,否则返回 False
     |      :rtype: bool
     |  
     |  set\_feature\_id(self, fid)
     |      设置 Feature ID
     |      
     |      :param str fid: feature ID 值,一般用于表示要素对象的唯一的ID值
     |      :return:
     |      :rtype:
     |  
     |  set\_field\_infos(self, field\_infos)
     |      设置属性字段信息
     |      
     |      :param list[FieldInfo] field\_infos: 属性字段信息
     |      :return: self
     |      :rtype: Feature
     |  
     |  set\_geometry(self, geo)
     |      设置几何对象
     |      
     |      :param Geometry geo: 几何对象
     |      :return: self
     |      :rtype: Feature
     |  
     |  set\_value(self, item, value)
     |      设置当前对象中指定的属性字段的字段值
     |      
     |      :param item: 字段名称或序号
     |      :type item: str or int
     |      :param value: 字段值
     |      :type value: int or float or str or datetime.datetime or bytes or bytearray
     |      :rtype: bool
     |  
     |  set\_values(self, values)
     |      设置字段值.
     |      
     |      :param dict values: 要写入的属性字段值。必须是 dict,dict 的键值为字段名称,dict 的值为字段值
     |      :return: 返回成功写入的字段数目
     |      :rtype: int
     |  
     |  to\_json(self)
     |      将当前对象输出为 json 字符串
     |      
     |      :rtype: str
     |  
     |  ----------------------------------------------------------------------
     |  Static methods defined here:
     |  
     |  from\_json(value)
     |      从 json 字符串中读取信息构造特征要素对象
     |      
     |      :param dict value: 包含特征要素对象信息的json字符串
     |      :rtype: Feature
     |  
     |  ----------------------------------------------------------------------
     |  Data descriptors defined here:
     |  
     |  \_\_dict\_\_
     |      dictionary for instance variables (if defined)
     |  
     |  \_\_weakref\_\_
     |      list of weak references to the object (if defined)
     |  
     |  bounds
     |      Rectangle: 获取几何对象的地理范围。如果几何对象为空,则返回空
     |  
     |  feature\_id
     |      str: 返回 Feature ID
     |  
     |  field\_infos
     |      list[FieldInfo]: 返回要素对象的所有字段信息
     |  
     |  geometry
     |      Geometry: 返回几何对象
    
    class FieldInfo(builtins.object)
     |  字段信息类。字段信息类用来存储字段的名称、类型、默认值以及长度等相关信息。 每一个字段对应一个 FieldInfo。对于矢量数据集的每一个字段,只有字段的
     |  别名(caption)可以被修改,其他属性的修改需要依据具体引擎是否支持。
     |  
     |  Methods defined here:
     |  
     |  \_\_getstate\_\_(self)
     |  
     |  \_\_init\_\_(self, name=None, field\_type=None, max\_length=None, default\_value=None, caption=None, is\_required=False, is\_zero\_length\_allowed=True)
     |      构造字段信息对象
     |      
     |      :param str name:  字段名称。字段的名称只能由数字、字母和下划线组成,但不能以数字或下划线开头;用户新建字段时,字段名称不能以 SM 作为前
     |                        缀,以 SM 作为前缀的都是 SuperMap 系统字段,另外,字段的名称不能超过30个字符,且字段的名称不区分大小写。名称用于
     |                        唯一标识该字段,所以字段不可重名。
     |      :param field\_type: 字段类型
     |      :type field\_type: FieldType or str
     |      :param int max\_length: 字段值的最大长度,只对文本字段有效
     |      :param default\_value: 字段的默认值
     |      :type default\_value: int or float or datetime.datetime or str or bytes or bytearray
     |      :param str caption: 字段别名
     |      :param bool is\_required: 是否是必填字段
     |      :param bool is\_zero\_length\_allowed: 是否允许零长度。只对文本类型(TEXT,WTEXT,CHAR)字段有效
     |  
     |  \_\_repr\_\_(self)
     |      Return repr(self).
     |  
     |  \_\_setstate\_\_(self, state)
     |  
     |  \_\_str\_\_(self)
     |      Return str(self).
     |  
     |  clone(self)
     |      拷贝一个新的 FieldInfo  对象
     |      
     |      :rtype: FieldInfo
     |  
     |  from\_dict(self, values)
     |      从 dict 对象中读取字段信息
     |      
     |      :param dict values: 包含 FieldInfo 字段信息的 dict 对象
     |      :return: self
     |      :rtype: FieldInfo
     |  
     |  is\_system\_field(self)
     |      判断当前对象是否是系统字段。对于所有以 Sm 开头(不区分大小写)的字段都是系统系统。
     |      
     |      :rtype: bool
     |  
     |  set\_caption(self, value)
     |      设置此字段的别名。别名可以不唯一,即不同的字段可以有相同的别名,而名称是用来唯一标识一个字段的,所以不可重名
     |      
     |      :param str value: 字段别名。
     |      :return: self
     |      :rtype: FieldInfo
     |  
     |  set\_default\_value(self, value)
     |      设置字段的默认值。当添加一条记录时,如果该字段未被赋值,则以该默认值作为该字段的值。
     |      
     |      :param value: 字段的默认值
     |      :type value: bool or int or float or datetime.datetime or str or bytes or bytearray
     |      :return: self
     |      :rtype: FieldInfo
     |  
     |  set\_max\_length(self, value)
     |      返回字段值的最大长度,只对文本字段有效。单位:字节
     |      
     |      :param int value: 字段值的最大长度
     |      :return: self
     |      :rtype: FieldInfo
     |  
     |  set\_name(self, value)
     |      设置字段名称。字段的名称只能由数字、字母和下划线组成,但不能以数字或下划线开头;用户新建字段时,字段名称不能以 SM 作为前缀,以 SM 作为前缀
     |      的都是 SuperMap 系统字段,另外,字段的名称不能超过30个字符,且字段的名称不区分大小写。名称用于唯一标识该字段,所以字段不可重名。
     |      
     |      :param str value: 字段名称
     |      :return: self
     |      :rtype: FieldInfo
     |  
     |  set\_required(self, value)
     |      设置字段是否为必填字段
     |      
     |      :param str value: 字段名称
     |      :return: self
     |      :rtype: FieldInfo
     |  
     |  set\_type(self, value)
     |      设置字段类型
     |      
     |      :param value: 字段类型
     |      :type value: FieldType or str
     |      :return: self
     |      :rtype: FieldInfo
     |  
     |  set\_zero\_length\_allowed(self, value)
     |      设置字段是否允许零长度。只对文本字段有效。
     |      
     |      :param bool value: 字段是否允许零长度。允许字段零长度设置为True,否则为False。默认值为True。
     |      :return: self
     |      :rtype: FieldInfo
     |  
     |  to\_dict(self)
     |      将当前对象输出到 dict 对象中
     |      
     |      :rtype: dict
     |  
     |  to\_json(self)
     |      将当前对象输出为 json 字符串
     |      
     |      :rtype: str
     |  
     |  ----------------------------------------------------------------------
     |  Static methods defined here:
     |  
     |  from\_json(value)
     |      从 json 字符串中构造 FieldInfo 对象
     |      
     |      :param str value: json 字符串
     |      :rtype: FieldInfo
     |  
     |  make\_from\_dict(values)
     |      从 dict 对象中构造一个新的 FieldInfo 对象
     |      
     |      :param dict values: 包含 FieldInfo 字段信息的 dict 对象
     |      :rtype: FieldInfo
     |  
     |  ----------------------------------------------------------------------
     |  Data descriptors defined here:
     |  
     |  \_\_dict\_\_
     |      dictionary for instance variables (if defined)
     |  
     |  \_\_weakref\_\_
     |      list of weak references to the object (if defined)
     |  
     |  caption
     |      str: 字段别名
     |  
     |  default\_value
     |      int or float or datetime.datetime or str or bytes or bytearray:  字段的默认值
     |  
     |  is\_required
     |      bool: 字段是否为必填字段
     |  
     |  is\_zero\_length\_allowed
     |      bool: 是否允许零长度。只对文本类型(TEXT,WTEXT,CHAR)字段有效
     |  
     |  max\_length
     |      int:  字段值的最大长度,只对文本字段有效
     |  
     |  name
     |      str: 字段名称,字段的名称只能由数字、字母和下划线组成,但不能以数字或下划线开头;用户新建字段时,字段名称不能以 SM 作为前缀,以 SM 作为前
     |      缀的都是 SuperMap 系统字段,另外,字段的名称不能超过30个字符,且字段的名称不区分大小写。名称用于唯一标识该字段,所以字段不可重名。
     |  
     |  type
     |      FieldType: 字段类型
    
    class GeoCoordSys(iobjectspy.\_jsuperpy.data.\_jvm.JVMBase)
     |  地理坐标系类。
     |  
     |  地理坐标系由大地参照系、中央子午线、坐标单位组成。在地理坐标系中,单位一般用度来表示,也可以用度分秒表示。东西向(水平方向)的范围为-180度至180
     |  度。南北向(垂直方向)的范围为-90度至90度。
     |  
     |  地理坐标是用经纬度表示地面点位置的球面坐标。在球形系统中,赤道面的平行面同地球椭球面相交所截的圈称为纬圈,也叫纬线,表示东西方向,通过地球旋转轴
     |  的面与椭球面相交所截的圈为子午圈,也称经线,表示南北方向,这些包围着地球的网格称为经纬格网。
     |  
     |  经纬线一般用度来表示(必要时也用度分秒表示)。经度是指地面上某点所在的经线面与本初子午面所成的二面角,规定本初子午线的经度为 0 度,从本初子午线
     |  向东 0 到 180 度为“东经”,以“E”表示,向西 0 到 -180 度为“西经”,以字母“W”表示;纬度是指地面上某点与地球球心的连线和赤道面所成的线面角,规
     |  定赤道的纬度为 0 度,从赤道向北 0 到 90 度为“北纬”,以字母“N”表示,向南 0 到 -90 度为“南纬”,以字母“S”表示。
     |  
     |  Method resolution order:
     |      GeoCoordSys
     |      iobjectspy.\_jsuperpy.data.\_jvm.JVMBase
     |      builtins.object
     |  
     |  Methods defined here:
     |  
     |  \_\_getstate\_\_(self)
     |  
     |  \_\_init\_\_(self)
     |      Initialize self.  See help(type(self)) for accurate signature.
     |  
     |  \_\_setstate\_\_(self, state)
     |  
     |  clone(self)
     |      复制对象
     |      
     |      :rtype: GeoCoordSys
     |  
     |  from\_xml(self, xml)
     |      从指定的 XML 字符串中构建地理坐标系类的对象,成功返回 True
     |      
     |      :param str xml: XML 字符串
     |      :rtype:  bool
     |  
     |  set\_coord\_unit(self, unit)
     |      设置地理坐标系的单位。
     |      
     |      :param unit: 地理坐标系的单位
     |      :type unit: Unit or str
     |      :return: self
     |      :rtype: GeoCoordSys
     |  
     |  set\_geo\_datum(self, datum)
     |      设置大地参照系对象
     |      
     |      :param GeoDatum datum:
     |      :return: self
     |      :rtype: GeoCoordSys
     |  
     |  set\_geo\_prime\_meridian(self, prime\_meridian)
     |      设置中央子午线对象
     |      
     |      :param GeoPrimeMeridian prime\_meridian:
     |      :return: self
     |      :rtype: GeoCoordSys
     |  
     |  set\_geo\_spatial\_ref\_type(self, spatial\_ref\_type)
     |      设置空间坐标系类型。
     |      
     |      :param spatial\_ref\_type: 空间坐标系类型
     |      :type spatial\_ref\_type: GeoSpatialRefType or str
     |      :return: self
     |      :rtype: GeoCoordSys
     |  
     |  set\_name(self, name)
     |      设置地理坐标系对象的名称
     |      
     |      :param str name: 地理坐标系对象的名称
     |      :return: self
     |      :rtype: GeoCoordSys
     |  
     |  set\_type(self, coord\_type)
     |      设置地理坐标系类型
     |      
     |      :param coord\_type: 地理坐标系类型
     |      :type coord\_type: GeoCoordSysType or str
     |      :return: self
     |      :rtype: GeoCoordSys
     |  
     |  to\_xml(self)
     |      将地理坐标系类的对象转换为 XML 格式的字符串。
     |      
     |      :rtype: str
     |  
     |  ----------------------------------------------------------------------
     |  Data descriptors defined here:
     |  
     |  coord\_unit
     |      Unit: 返回地理坐标系的单位。默认值为 DEGREE
     |  
     |  geo\_datum
     |      GeoDatum: 返回大地参照系对象
     |  
     |  geo\_prime\_meridian
     |      GeoPrimeMeridian: 返回中央子午线对象
     |  
     |  geo\_spatial\_ref\_type
     |      GeoSpatialRefType: 返回空间坐标系类型。
     |  
     |  name
     |      str:  返回地理坐标系对象的名称
     |  
     |  type
     |      GeoCoordSysType: 返回地理坐标系类型
     |  
     |  ----------------------------------------------------------------------
     |  Data descriptors inherited from iobjectspy.\_jsuperpy.data.\_jvm.JVMBase:
     |  
     |  \_\_dict\_\_
     |      dictionary for instance variables (if defined)
     |  
     |  \_\_weakref\_\_
     |      list of weak references to the object (if defined)
    
    class GeoDatum(iobjectspy.\_jsuperpy.data.\_jvm.JVMBase)
     |  大地参照系类。
     |  该类包含有地球椭球参数。
     |  地球椭球体仅仅是描述了地球的大小及形状,为了更准确地描述地球上的地物的具体位置,需要引入大地参照系。大地参照系确定了地球椭球体相对于地球球心的位置,为地表地物的测量提供了一个参照框架,确定了地表经纬网线的原点和方向。大地参照系把地球椭球体的球心当作原点。一个地区的大地参照系的地球椭球体或多或少地偏移了真正的地心,地表上的地物坐标都是相对于该椭球体的球心的。目前被广泛利用的是 WGS84,它被当着大地测量的基本框架。不同的大地参照系适用于不同的国家和地区,一个大地参照系并不适合于所有的地区。
     |  
     |  Method resolution order:
     |      GeoDatum
     |      iobjectspy.\_jsuperpy.data.\_jvm.JVMBase
     |      builtins.object
     |  
     |  Methods defined here:
     |  
     |  \_\_getstate\_\_(self)
     |  
     |  \_\_init\_\_(self, datum\_type=None)
     |      构造大地参照系对象
     |      
     |      :param datum\_type: 大地参照系类型
     |      :type datum\_type: GeoDatumType or str
     |  
     |  \_\_setstate\_\_(self, state)
     |  
     |  clone(self)
     |      复制对象
     |      
     |      :rtype: GeoDatum
     |  
     |  from\_xml(self, xml)
     |      据 XML 字符串构建 GeoDatum 对象,成功返回 True。
     |      
     |      :param str xml:
     |      :rtype: bool
     |  
     |  set\_geo\_spheroid(self, geo\_spheroid)
     |      设置地球椭球体对象。只当大地参照系类型为自定义类型时才可以设置。
     |      人们通常用球体或椭球体来描述地球的形状和大小,有时为了计算方便,可以将地球看作一个球体,但更多的时候是把它看作椭球体。一般情况下在地图比例尺
     |      小于1:1,000,000 时,假设地球形状为一球体,因为在这种比例尺下球体和椭球体的差别几乎无法分辨;而在1:1,000,000 甚至更高精度要求的大比例
     |      尺时,则需用椭球体逼近地球。椭球体是以椭圆为基础的,所以用两个轴来表述地球球体的大小,即长轴(赤道半径)和短轴(极地半径)。
     |      
     |      :param GeoSpheroid geo\_spheroid: 地球椭球体对象
     |      :return: self
     |      :rtype: GeoSpheroid
     |  
     |  set\_name(self, name)
     |      设置大地参照系对象的名称
     |      
     |      :param str name: 大地参照系对象的名称
     |      :return: self
     |      :rtype:  GeoDatum
     |  
     |  set\_type(self, datum\_type)
     |      设置大地参照系的类型。
     |      当大地参照系为自定义时,用户需另外指定椭球体参数;其它的值为系统预定义,用户不必指定椭球体参数。参见 :py:class:`GeoDatumType` 。
     |      
     |      :param datum\_type: 大地参照系的类型
     |      :type datum\_type: GeoDatumType or str
     |      :return: self
     |      :rtype: GeoDatum
     |  
     |  to\_xml(self)
     |      将大地参照系类的对象转换为 XML 格式的字符串
     |      
     |      :rtype: str
     |  
     |  ----------------------------------------------------------------------
     |  Data descriptors defined here:
     |  
     |  geo\_spheroid
     |      GeoSpheroid: 地球椭球体对象
     |  
     |  name
     |      str: 大地参照系对象的名称
     |  
     |  type
     |      GeoDatumType: 大地参照系的类型
     |  
     |  ----------------------------------------------------------------------
     |  Data descriptors inherited from iobjectspy.\_jsuperpy.data.\_jvm.JVMBase:
     |  
     |  \_\_dict\_\_
     |      dictionary for instance variables (if defined)
     |  
     |  \_\_weakref\_\_
     |      list of weak references to the object (if defined)
    
    class GeoLine(Geometry)
     |  线几何对象类。
     |  该类用于描述线状地理实体,如河流,道路,等值线等,一般用一个或多个有序坐标点集合来表示。线的方向决定于有序坐标点的顺序,也可以通过调用 reverse
     |  方法来改变线的方向。线对象由一个或多个部分组成,每个部分称为线对象的一个子对象,每个子对象用一个有序坐标点集合来表示。可以对子对象进行添加,删除,
     |  修改等操作。
     |  
     |  Method resolution order:
     |      GeoLine
     |      Geometry
     |      iobjectspy.\_jsuperpy.data.\_jvm.JVMBase
     |      builtins.object
     |  
     |  Methods defined here:
     |  
     |  \_\_getitem\_\_(self, item)
     |  
     |  \_\_getstate\_\_(self)
     |  
     |  \_\_init\_\_(self, points=None)
     |      构造一个线几何对象
     |      
     |      :param points: 包含点串信息的对象,可以为 list[Point2D] 、tuple[Point2D] 、 GeoLine 、GeoRegion 和 Rectangle
     |      :type points: list[Point2D] or tuple[Point2D] or GeoLine or GeoRegion or Rectangle
     |  
     |  \_\_setitem\_\_(self, item, value)
     |  
     |  \_\_setstate\_\_(self, state)
     |  
     |  add\_part(self, points)
     |      向此线几何对象追加一个子对象。成功返回添加的子对象的序号。
     |      
     |      :param list[Point2D] points: 一个有序点集
     |      :rtype: int
     |  
     |  clone(self)
     |      复制对象
     |      
     |      :rtype: GeoLine
     |  
     |  convert\_to\_region(self)
     |      将当前线对象转换为面几何对象
     |      - 对于没有封闭的线对象,转换为面对象时,会把首尾自动连起来
     |      - GeoLine 对象实例的某个子对象的点数少于 3 时将失败
     |      
     |      :rtype: GeoRegion
     |  
     |  create\_buffer(self, distance, prj=None, unit=None)
     |      构建当前线对象的缓冲区对象。将会构建线对象的圆头全缓冲区。
     |      
     |      :param float distance: 缓冲区半径。如果设置了 prj 和 unit,将使用 unit 的单位作为缓冲区半径的单位。
     |      :param PrjCoordSys prj: 描述点几何对象的投影信息
     |      :param unit: 缓冲区半径单位
     |      :type unit: Unit or str
     |      :return: 缓冲区半径
     |      :rtype: GeoRegion
     |  
     |  find\_point\_on\_line\_by\_distance(self, distance)
     |      在线上以指定的距离找点,查找的起始点为线的起始点。
     |      - 当 distance 大于 Length 时,返回线最后一个子对象的终点。
     |      - 当 distance=0 时,返回线几何对象的起始点;
     |      - 当线几何对象具有多个子对象的时候,按照子对象的序号依次查找
     |      
     |      :param float distance: 要找点的距离
     |      :return: 查找成功返回要找的点,否则返回 None
     |      :rtype: Point2D
     |  
     |  get\_part(self, item)
     |      返回此线几何对象中指定序号的子对象,以有序点集合的方式返回该子对象。 当二维线对象是简单线对象时,如果传入参数0,得到的是此线对象的节点的集合。
     |      
     |      :param int item: 子对象的序号。
     |      :return: 子对象的的节点
     |      :rtype: list[Point2D]
     |  
     |  get\_part\_count(self)
     |      获取子对象的个数
     |      
     |      :rtype: int
     |  
     |  get\_parts(self)
     |      获取当前几何对象的所有点坐标。每个子对象使用一个 list 存储
     |      
     |      >>> points = [Point2D(1,2),Point2D(2,3)]
     |      >>> geo = GeoLine(points)
     |      >>> geo.add\_part([Point2D(3,4),Point2D(4,5)])
     |      >>> print(geo.get\_parts())
     |      [[(1.0, 2.0), (2.0, 3.0)], [(3.0, 4.0), (4.0, 5.0)]]
     |      
     |      :return: 包含所有点坐标list
     |      :rtype: list[list[Point2D]]
     |  
     |  insert\_part(self, item, points)
     |      此线几何对象中的指定位置插入一个子对象。成功则返回 True,否则返回 False
     |      
     |      :param int item: 插入的位置
     |      :param list[Point2D] points: 插入的有序点集合
     |      :rtype: bool
     |  
     |  remove\_part(self, item)
     |      删除此线几何对象中的指定序号的子对象。
     |      
     |      :param int item: 指定的子对象的序号
     |      :return: 成功则返回 true,否则返回 false
     |      :rtype: bool
     |  
     |  to\_json(self)
     |      将当前对象输出为 Simple Json 字符串
     |      
     |      >>> points = [Point2D(1,2), Point2D(2,3), Point2D(1,5), Point2D(1,2)]
     |      >>> geo = GeoLine(points)
     |      >>> print(geo.to\_json())
     |      \{"Line": [[[1.0, 2.0], [2.0, 3.0], [1.0, 5.0], [1.0, 2.0]]], "id": 0\}
     |      
     |      :rtype: str
     |  
     |  ----------------------------------------------------------------------
     |  Data descriptors defined here:
     |  
     |  length
     |      float: 返回线对象的长度
     |  
     |  ----------------------------------------------------------------------
     |  Methods inherited from Geometry:
     |  
     |  \_\_str\_\_(self)
     |      Return str(self).
     |  
     |  from\_xml(self, xml)
     |      根据传入的 XML 字符串重新构造几何对象。该 XML 必须符合 GML3.0 规范。
     |      调用该方法时,首先将该几何对象的原始数据清空,然后根据传入的 XML 字符串重新构造该几何对象。
     |      GML (Geography Markup Language)即地理标识语言, GML 能够表示地理空间对象的空间数据和非空间属性数据。GML 是基于 XML 的空间信息编码
     |      标准,由开放式地理信息系统协会 OpenGIS Consortium (OGC) 提出,得到了许多公司的大力支持,如 Oracle、Galdos、MapInfo、CubeWerx 等。
     |      GML 作为一个空间数据编码规范,提供了一套基本的标签、公共的数据模型,以及用户构建应用模式(GML Application Schemas)的机制。
     |      
     |      :param str xml:  XML 格式的字符串
     |      :return: 构造成功返回 True,否则返回 False。
     |      :rtype: bool
     |  
     |  get\_inner\_point(self)
     |      获取几何对象的内点
     |      
     |      :rtype: Point2D
     |  
     |  hit\_test(self, point, tolerance)
     |      测试在指定容限允许的范围内,指定的点是否在几何对象的范围内。即判断以测试点为圆心,以指定的容限为半径的圆是否与该几何对象有交集,若有交集,
     |      则返回 True;否则返回 False。
     |      
     |      :param Point2D point:  测试点
     |      :param float tolerance: 容限值,单位与数据集的单位相同
     |      :rtype: bool
     |  
     |  is\_empty(self)
     |      判断几何对象是否为空值,不同的几何对象的是否为空的条件各异。
     |      
     |      :rtype: bool
     |  
     |  linear\_extrude(self, height=0.0, twist=0.0, scaleX=1.0, scaleY=1.0, bLonLat=False)
     |      线性拉伸,支持二维和三维矢量面,二三维圆,GeoRectangle
     |      :param height: 拉伸高度
     |      :param twist: 旋转角度
     |      :param scaleX: 绕X轴缩放
     |      :param scaleY: 绕Y轴缩放
     |      :param bLonLat: 是否是经纬度
     |      :return: 返回GeoModel3D对象
     |  
     |  offset(self, dx, dy)
     |      将此几何对象偏移指定的量。
     |      
     |      :param float dx: 偏移 X 坐标的量
     |      :param float dy: 偏移 Y 坐标的量
     |      :return: self
     |      :rtype: Geometry
     |  
     |  resize(self, rc)
     |      缩放此几何对象,使其最小外接矩形等于指定的矩形对象。
     |      对于几何点,该方法只改变其位置,将其移动到指定的矩形的中心点;对于文本对象,该方法将缩放文本大小。
     |      
     |      :param Rectangle rc: 调整大小后几何对象的范围。
     |  
     |  rotate(self, base\_point, angle)
     |      以指定点为基点将此几何对象旋转指定角度,逆时针方向为正方向,角度以度为单位。
     |      
     |      :param Point2D base\_point: 旋转的基点。
     |      :param float angle: 旋转的角度,单位为度
     |  
     |  rotate\_extrude(self, angle=0.0)
     |      旋转拉伸,支持二维和三维矢量面,必须在平面坐标系下构建
     |      :param angle: 旋转角度
     |      :return: 返回GeoModel3D对象
     |  
     |  set\_empty(self)
     |      清空几何对象中的空间数据,但几何对象的标识符和几何风格保持不变。
     |  
     |  set\_id(self, value)
     |      设置几何对象的 ID 值
     |      
     |      :param int value: ID 值。
     |      :return: self
     |      :rtype: Geometry
     |  
     |  to\_geojson(self)
     |      将当前对象信息以  geojson 格式返回。只支持点、线和面对象。
     |      
     |      :rtype: str
     |  
     |  to\_xml(self)
     |      根据 GML 3.0 规范,将该几何对象的空间数据输出为 XML 字符串。 注意:几何对象输出的 XML 字符串只含有该几何对象的地理坐标值,不含有该几何对象的风格和 ID 等信息。
     |      
     |      :rtype: str
     |  
     |  ----------------------------------------------------------------------
     |  Static methods inherited from Geometry:
     |  
     |  from\_geojson(geojson)
     |      从 geojson 中读取信息构造一个几何对象
     |      
     |      :param str geojson: geojson 字符串
     |      :return: 几何对象
     |      :rtype: Geometry
     |  
     |  from\_json(value)
     |      从 json 中构造一个几何对象。参考 :py:meth:`to\_json`
     |      
     |      :param str value: json 字符串
     |      :rtype: Geometry
     |  
     |  ----------------------------------------------------------------------
     |  Data descriptors inherited from Geometry:
     |  
     |  bounds
     |      Rectangle: 返回几何对象的最小外接矩形。点对象的最小外接矩形退化为一个点,即矩形的左边界坐标值等于其右边界坐标值,上边界坐标值等于其
     |      下边界的坐标值,分别为该点的 x 坐标和 y 坐标。
     |  
     |  id
     |      int: 返回几何对象的 ID
     |  
     |  type
     |      GeometryType: 返回几何对象类型
     |  
     |  ----------------------------------------------------------------------
     |  Data descriptors inherited from iobjectspy.\_jsuperpy.data.\_jvm.JVMBase:
     |  
     |  \_\_dict\_\_
     |      dictionary for instance variables (if defined)
     |  
     |  \_\_weakref\_\_
     |      list of weak references to the object (if defined)
    
    class GeoPoint(Geometry)
     |  点几何对象类。
     |  该类一般用于描述点状地理实体。Point2D 和 GeoPoint 都可用来表示二维点,所不同的是 GeoPoint 描述的是地物实体,而 Point2D 描述的是一个位
     |  置点;当赋予 GeoPoint 不同的几何风格,即可用于表示不同的地物实体,而 Point2D 则是广泛用于定位的坐标点
     |  
     |  Method resolution order:
     |      GeoPoint
     |      Geometry
     |      iobjectspy.\_jsuperpy.data.\_jvm.JVMBase
     |      builtins.object
     |  
     |  Methods defined here:
     |  
     |  \_\_getstate\_\_(self)
     |  
     |  \_\_init\_\_(self, point=None)
     |      构造一个点几何对象。
     |      
     |      :param point: 点对象
     |      :type point: Point2D  or GeoPoint or tuple[float] or list[float]
     |  
     |  \_\_setstate\_\_(self, state)
     |  
     |  create\_buffer(self, distance, prj=None, unit=None)
     |      在当前位置点构造一个缓冲区对象
     |      
     |      :param float distance: 缓冲区半径。如果设置了 prj 和 unit,将使用 unit 的单位作为缓冲区半径的单位。
     |      :param PrjCoordSys prj: 描述点几何对象的投影信息
     |      :param unit: 缓冲区半径单位
     |      :type unit: Unit or str
     |      :return: 缓冲区半径
     |      :rtype: GeoRegion
     |  
     |  get\_x(self)
     |      获取点几何对象的 X 坐标值
     |      
     |      :rtype: float
     |  
     |  get\_y(self)
     |      获取点几何对象的 Y 坐标值
     |      
     |      :rtype: float
     |  
     |  set\_point(self, point)
     |      设置点几何对象的地理位置
     |      
     |      :param Point2D point: 点几何对象的地理位置
     |      :return: self
     |      :rtype: GeoPoint
     |  
     |  set\_x(self, x)
     |      设置点几何对象的 X 坐标
     |      
     |      :param float x: X 坐标值
     |      :return: self
     |      :rtype: GeoPoint
     |  
     |  set\_y(self, y)
     |      设置点几何对象的 Y 坐标
     |      
     |      :param float y: Y 坐标值
     |      :return: self
     |      :rtype: GeoPoint
     |  
     |  to\_json(self)
     |      将当前点对象以 simple json 格式返回。
     |      
     |      例如::
     |      
     |          >>> geo = GeoPoint((10,20))
     |          >>> print(geo.to\_json())
     |          \{"Point": [10.0, 20.0], "id": 0\}
     |      
     |      :return: simple json 格式字符串
     |      :rtype: str
     |  
     |  ----------------------------------------------------------------------
     |  Data descriptors defined here:
     |  
     |  bounds
     |      Rectangle: 获取点几何对象的地理范围
     |  
     |  point
     |      返回点几何对象的地理位置
     |      
     |      :rtype: Point2D
     |  
     |  ----------------------------------------------------------------------
     |  Methods inherited from Geometry:
     |  
     |  \_\_str\_\_(self)
     |      Return str(self).
     |  
     |  clone(self)
     |      拷贝对象
     |      
     |      :rtype: Geometry
     |  
     |  from\_xml(self, xml)
     |      根据传入的 XML 字符串重新构造几何对象。该 XML 必须符合 GML3.0 规范。
     |      调用该方法时,首先将该几何对象的原始数据清空,然后根据传入的 XML 字符串重新构造该几何对象。
     |      GML (Geography Markup Language)即地理标识语言, GML 能够表示地理空间对象的空间数据和非空间属性数据。GML 是基于 XML 的空间信息编码
     |      标准,由开放式地理信息系统协会 OpenGIS Consortium (OGC) 提出,得到了许多公司的大力支持,如 Oracle、Galdos、MapInfo、CubeWerx 等。
     |      GML 作为一个空间数据编码规范,提供了一套基本的标签、公共的数据模型,以及用户构建应用模式(GML Application Schemas)的机制。
     |      
     |      :param str xml:  XML 格式的字符串
     |      :return: 构造成功返回 True,否则返回 False。
     |      :rtype: bool
     |  
     |  get\_inner\_point(self)
     |      获取几何对象的内点
     |      
     |      :rtype: Point2D
     |  
     |  hit\_test(self, point, tolerance)
     |      测试在指定容限允许的范围内,指定的点是否在几何对象的范围内。即判断以测试点为圆心,以指定的容限为半径的圆是否与该几何对象有交集,若有交集,
     |      则返回 True;否则返回 False。
     |      
     |      :param Point2D point:  测试点
     |      :param float tolerance: 容限值,单位与数据集的单位相同
     |      :rtype: bool
     |  
     |  is\_empty(self)
     |      判断几何对象是否为空值,不同的几何对象的是否为空的条件各异。
     |      
     |      :rtype: bool
     |  
     |  linear\_extrude(self, height=0.0, twist=0.0, scaleX=1.0, scaleY=1.0, bLonLat=False)
     |      线性拉伸,支持二维和三维矢量面,二三维圆,GeoRectangle
     |      :param height: 拉伸高度
     |      :param twist: 旋转角度
     |      :param scaleX: 绕X轴缩放
     |      :param scaleY: 绕Y轴缩放
     |      :param bLonLat: 是否是经纬度
     |      :return: 返回GeoModel3D对象
     |  
     |  offset(self, dx, dy)
     |      将此几何对象偏移指定的量。
     |      
     |      :param float dx: 偏移 X 坐标的量
     |      :param float dy: 偏移 Y 坐标的量
     |      :return: self
     |      :rtype: Geometry
     |  
     |  resize(self, rc)
     |      缩放此几何对象,使其最小外接矩形等于指定的矩形对象。
     |      对于几何点,该方法只改变其位置,将其移动到指定的矩形的中心点;对于文本对象,该方法将缩放文本大小。
     |      
     |      :param Rectangle rc: 调整大小后几何对象的范围。
     |  
     |  rotate(self, base\_point, angle)
     |      以指定点为基点将此几何对象旋转指定角度,逆时针方向为正方向,角度以度为单位。
     |      
     |      :param Point2D base\_point: 旋转的基点。
     |      :param float angle: 旋转的角度,单位为度
     |  
     |  rotate\_extrude(self, angle=0.0)
     |      旋转拉伸,支持二维和三维矢量面,必须在平面坐标系下构建
     |      :param angle: 旋转角度
     |      :return: 返回GeoModel3D对象
     |  
     |  set\_empty(self)
     |      清空几何对象中的空间数据,但几何对象的标识符和几何风格保持不变。
     |  
     |  set\_id(self, value)
     |      设置几何对象的 ID 值
     |      
     |      :param int value: ID 值。
     |      :return: self
     |      :rtype: Geometry
     |  
     |  to\_geojson(self)
     |      将当前对象信息以  geojson 格式返回。只支持点、线和面对象。
     |      
     |      :rtype: str
     |  
     |  to\_xml(self)
     |      根据 GML 3.0 规范,将该几何对象的空间数据输出为 XML 字符串。 注意:几何对象输出的 XML 字符串只含有该几何对象的地理坐标值,不含有该几何对象的风格和 ID 等信息。
     |      
     |      :rtype: str
     |  
     |  ----------------------------------------------------------------------
     |  Static methods inherited from Geometry:
     |  
     |  from\_geojson(geojson)
     |      从 geojson 中读取信息构造一个几何对象
     |      
     |      :param str geojson: geojson 字符串
     |      :return: 几何对象
     |      :rtype: Geometry
     |  
     |  from\_json(value)
     |      从 json 中构造一个几何对象。参考 :py:meth:`to\_json`
     |      
     |      :param str value: json 字符串
     |      :rtype: Geometry
     |  
     |  ----------------------------------------------------------------------
     |  Data descriptors inherited from Geometry:
     |  
     |  id
     |      int: 返回几何对象的 ID
     |  
     |  type
     |      GeometryType: 返回几何对象类型
     |  
     |  ----------------------------------------------------------------------
     |  Data descriptors inherited from iobjectspy.\_jsuperpy.data.\_jvm.JVMBase:
     |  
     |  \_\_dict\_\_
     |      dictionary for instance variables (if defined)
     |  
     |  \_\_weakref\_\_
     |      list of weak references to the object (if defined)
    
    class GeoPoint3D(Geometry)
     |  点几何对象类。
     |  该类一般用于描述点状地理实体。Point3D 和 GeoPoint3D 都可用来表示三维点,所不同的是 GeoPoint3D 描述的是地物实体,而 Point3D 描述的是一个位
     |  置点;当赋予 GeoPoint3D 不同的几何风格,即可用于表示不同的地物实体,而 Point3D 则是广泛用于定位的坐标点
     |  
     |  Method resolution order:
     |      GeoPoint3D
     |      Geometry
     |      iobjectspy.\_jsuperpy.data.\_jvm.JVMBase
     |      builtins.object
     |  
     |  Methods defined here:
     |  
     |  \_\_getstate\_\_(self)
     |  
     |  \_\_init\_\_(self, point=None)
     |      构造一个点几何对象。
     |      
     |      :param point: 点对象
     |      :type point: Point3D  or GeoPoint3D or tuple[float] or list[float]
     |  
     |  \_\_setstate\_\_(self, state)
     |  
     |  get\_x(self)
     |      获取点几何对象的 X 坐标值
     |      
     |      :rtype: float
     |  
     |  get\_y(self)
     |      获取点几何对象的 Y 坐标值
     |      
     |      :rtype: float
     |  
     |  get\_z(self)
     |      获取点几何对象的 Z 坐标值
     |      
     |      :rtype: float
     |  
     |  set\_point(self, point)
     |      设置点几何对象的地理位置
     |      
     |      :param Point3D point: 点几何对象的地理位置
     |      :return: self
     |      :rtype: GeoPoint
     |  
     |  set\_x(self, x)
     |      设置点几何对象的 X 坐标
     |      
     |      :param float x: X 坐标值
     |      :return: self
     |      :rtype: GeoPoint3D
     |  
     |  set\_y(self, y)
     |      设置点几何对象的 Y 坐标
     |      
     |      :param float y: Y 坐标值
     |      :return: self
     |      :rtype: GeoPoint3D
     |  
     |  set\_z(self, z)
     |      设置点几何对象的 z 坐标
     |      
     |      :param float z: Z 坐标值
     |      :return: self
     |      :rtype: GeoPoint3D
     |  
     |  to\_json(self)
     |      将当前点对象以 simple json 格式返回。
     |      
     |      例如::
     |      
     |          >>> geo = GeoPoint3D((10,20,15))
     |          >>> print(geo.to\_json())
     |          \{"Point3D": [10.0, 20.0], "id": 0\}
     |      
     |      :return: simple json 格式字符串
     |      :rtype: str
     |  
     |  ----------------------------------------------------------------------
     |  Data descriptors defined here:
     |  
     |  bounds
     |      Rectangle: 获取点几何对象的地理范围
     |  
     |  point
     |      返回点几何对象的地理位置
     |      
     |      :rtype: Point3D
     |  
     |  ----------------------------------------------------------------------
     |  Methods inherited from Geometry:
     |  
     |  \_\_str\_\_(self)
     |      Return str(self).
     |  
     |  clone(self)
     |      拷贝对象
     |      
     |      :rtype: Geometry
     |  
     |  from\_xml(self, xml)
     |      根据传入的 XML 字符串重新构造几何对象。该 XML 必须符合 GML3.0 规范。
     |      调用该方法时,首先将该几何对象的原始数据清空,然后根据传入的 XML 字符串重新构造该几何对象。
     |      GML (Geography Markup Language)即地理标识语言, GML 能够表示地理空间对象的空间数据和非空间属性数据。GML 是基于 XML 的空间信息编码
     |      标准,由开放式地理信息系统协会 OpenGIS Consortium (OGC) 提出,得到了许多公司的大力支持,如 Oracle、Galdos、MapInfo、CubeWerx 等。
     |      GML 作为一个空间数据编码规范,提供了一套基本的标签、公共的数据模型,以及用户构建应用模式(GML Application Schemas)的机制。
     |      
     |      :param str xml:  XML 格式的字符串
     |      :return: 构造成功返回 True,否则返回 False。
     |      :rtype: bool
     |  
     |  get\_inner\_point(self)
     |      获取几何对象的内点
     |      
     |      :rtype: Point2D
     |  
     |  hit\_test(self, point, tolerance)
     |      测试在指定容限允许的范围内,指定的点是否在几何对象的范围内。即判断以测试点为圆心,以指定的容限为半径的圆是否与该几何对象有交集,若有交集,
     |      则返回 True;否则返回 False。
     |      
     |      :param Point2D point:  测试点
     |      :param float tolerance: 容限值,单位与数据集的单位相同
     |      :rtype: bool
     |  
     |  is\_empty(self)
     |      判断几何对象是否为空值,不同的几何对象的是否为空的条件各异。
     |      
     |      :rtype: bool
     |  
     |  linear\_extrude(self, height=0.0, twist=0.0, scaleX=1.0, scaleY=1.0, bLonLat=False)
     |      线性拉伸,支持二维和三维矢量面,二三维圆,GeoRectangle
     |      :param height: 拉伸高度
     |      :param twist: 旋转角度
     |      :param scaleX: 绕X轴缩放
     |      :param scaleY: 绕Y轴缩放
     |      :param bLonLat: 是否是经纬度
     |      :return: 返回GeoModel3D对象
     |  
     |  offset(self, dx, dy)
     |      将此几何对象偏移指定的量。
     |      
     |      :param float dx: 偏移 X 坐标的量
     |      :param float dy: 偏移 Y 坐标的量
     |      :return: self
     |      :rtype: Geometry
     |  
     |  resize(self, rc)
     |      缩放此几何对象,使其最小外接矩形等于指定的矩形对象。
     |      对于几何点,该方法只改变其位置,将其移动到指定的矩形的中心点;对于文本对象,该方法将缩放文本大小。
     |      
     |      :param Rectangle rc: 调整大小后几何对象的范围。
     |  
     |  rotate(self, base\_point, angle)
     |      以指定点为基点将此几何对象旋转指定角度,逆时针方向为正方向,角度以度为单位。
     |      
     |      :param Point2D base\_point: 旋转的基点。
     |      :param float angle: 旋转的角度,单位为度
     |  
     |  rotate\_extrude(self, angle=0.0)
     |      旋转拉伸,支持二维和三维矢量面,必须在平面坐标系下构建
     |      :param angle: 旋转角度
     |      :return: 返回GeoModel3D对象
     |  
     |  set\_empty(self)
     |      清空几何对象中的空间数据,但几何对象的标识符和几何风格保持不变。
     |  
     |  set\_id(self, value)
     |      设置几何对象的 ID 值
     |      
     |      :param int value: ID 值。
     |      :return: self
     |      :rtype: Geometry
     |  
     |  to\_geojson(self)
     |      将当前对象信息以  geojson 格式返回。只支持点、线和面对象。
     |      
     |      :rtype: str
     |  
     |  to\_xml(self)
     |      根据 GML 3.0 规范,将该几何对象的空间数据输出为 XML 字符串。 注意:几何对象输出的 XML 字符串只含有该几何对象的地理坐标值,不含有该几何对象的风格和 ID 等信息。
     |      
     |      :rtype: str
     |  
     |  ----------------------------------------------------------------------
     |  Static methods inherited from Geometry:
     |  
     |  from\_geojson(geojson)
     |      从 geojson 中读取信息构造一个几何对象
     |      
     |      :param str geojson: geojson 字符串
     |      :return: 几何对象
     |      :rtype: Geometry
     |  
     |  from\_json(value)
     |      从 json 中构造一个几何对象。参考 :py:meth:`to\_json`
     |      
     |      :param str value: json 字符串
     |      :rtype: Geometry
     |  
     |  ----------------------------------------------------------------------
     |  Data descriptors inherited from Geometry:
     |  
     |  id
     |      int: 返回几何对象的 ID
     |  
     |  type
     |      GeometryType: 返回几何对象类型
     |  
     |  ----------------------------------------------------------------------
     |  Data descriptors inherited from iobjectspy.\_jsuperpy.data.\_jvm.JVMBase:
     |  
     |  \_\_dict\_\_
     |      dictionary for instance variables (if defined)
     |  
     |  \_\_weakref\_\_
     |      list of weak references to the object (if defined)
    
    class GeoPrimeMeridian(iobjectspy.\_jsuperpy.data.\_jvm.JVMBase)
     |  中央子午线类。
     |  该对象主要应用于地理坐标系中,地理坐标系由三部分组成:中央子午线、参照系或者大地基准(Datum)和角度单位。
     |  
     |  Method resolution order:
     |      GeoPrimeMeridian
     |      iobjectspy.\_jsuperpy.data.\_jvm.JVMBase
     |      builtins.object
     |  
     |  Methods defined here:
     |  
     |  \_\_getstate\_\_(self)
     |  
     |  \_\_init\_\_(self, meridian\_type=None)
     |      构造中央子午线对象
     |      
     |      :param meridian\_type:  中央经线类型
     |      :type meridian\_type:  GeoPrimeMeridianType or str
     |  
     |  \_\_setstate\_\_(self, state)
     |  
     |  clone(self)
     |      复制对象
     |      
     |      :rtype: GeoPrimeMeridian
     |  
     |  from\_xml(self, xml)
     |      指定的 XML 字符串构建 GeoPrimeMeridian 对象
     |      
     |      :param str xml: XML 字符串
     |      :return:
     |      :rtype: bool
     |  
     |  set\_longitude\_value(self, value)
     |      设置中央经线值,单位为度
     |      
     |      :param float value: 中央经线值,单位为度
     |      :return: self
     |      :rtype: GeoPrimeMeridian
     |  
     |  set\_name(self, name)
     |      设置中央经线对象的名称
     |      
     |      :param str name: 中央经线对象的名称
     |      :return: self
     |      :rtype:  GeoPrimeMeridian
     |  
     |  set\_type(self, meridian\_type)
     |      设置中央经线类型
     |      
     |      :param meridian\_type: 中央经线类型
     |      :type meridian\_type: GeoPrimeMeridianType or str
     |      :return: self
     |      :rtype: GeoPrimeMeridian
     |  
     |  to\_xml(self)
     |      返回表示 GeoPrimeMeridian 对象的 XML 字符串
     |      
     |      :rtype: str
     |  
     |  ----------------------------------------------------------------------
     |  Data descriptors defined here:
     |  
     |  longitude\_value
     |      float: 中央经线值,单位为度
     |  
     |  name
     |      str: 中央经线对象的名称
     |  
     |  type
     |      GeoPrimeMeridianType: 中央经线类型
     |  
     |  ----------------------------------------------------------------------
     |  Data descriptors inherited from iobjectspy.\_jsuperpy.data.\_jvm.JVMBase:
     |  
     |  \_\_dict\_\_
     |      dictionary for instance variables (if defined)
     |  
     |  \_\_weakref\_\_
     |      list of weak references to the object (if defined)
    
    class GeoRegion(Geometry)
     |  面几何对象类,派生于 Geometry 类。
     |  
     |  该类用于描述面状地理实体,如行政区域,湖泊,居民地等,一般用一个或多个有序坐标点集合来表示。面几何对象由一个或多个部分组成,每个部分称为面几何对
     |  象的一个子对象,每个子对象用一个有序坐标点集合来表示,其起始点和终止点重合。可以对子对象进行添加,删除,修改等操作。
     |  
     |  Method resolution order:
     |      GeoRegion
     |      Geometry
     |      iobjectspy.\_jsuperpy.data.\_jvm.JVMBase
     |      builtins.object
     |  
     |  Methods defined here:
     |  
     |  \_\_getitem\_\_(self, item)
     |  
     |  \_\_getstate\_\_(self)
     |  
     |  \_\_init\_\_(self, points=None)
     |      构造一个面几何对象
     |      
     |      :param points: 包含点串信息的对象,可以为 list[Point2D] 、tuple[Point2D] 、 GeoLine 、GeoRegion 和 Rectangle
     |      :type points: list[Point2D] or tuple[Point2D] or GeoLine or GeoRegion or Rectangle
     |  
     |  \_\_setitem\_\_(self, item, value)
     |  
     |  \_\_setstate\_\_(self, state)
     |  
     |  add\_part(self, points)
     |      向此面几何对象追加一个子对象。成功返回添加的子对象的序号。
     |      
     |      :param list[Point2D] points: 一个有序点集
     |      :rtype: int
     |  
     |  contains(self, point)
     |      判断点是否在面内
     |      
     |      :param point: 待判断的二维点对象
     |      :type point: Point2D or GeoPoint
     |      :return: 点在面内返回 True,否则返回 False
     |      :rtype: bool
     |  
     |  convert\_to\_line(self)
     |      将当前面对象转换为线对象
     |      
     |      :rtype: GeoLine
     |  
     |  create\_buffer(self, distance, prj=None, unit=None)
     |      构建当前面对象的缓冲区对象
     |      
     |      :param float distance: 缓冲区半径。如果设置了 prj 和 unit,将使用 unit 的单位作为缓冲区半径的单位。
     |      :param PrjCoordSys prj: 描述点几何对象的投影信息
     |      :param unit: 缓冲区半径单位
     |      :type unit: Unit or str
     |      :return: 缓冲区半径
     |      :rtype: GeoRegion
     |  
     |  get\_part(self, item)
     |      返回此面几何对象中指定序号的子对象,以有序点集合的方式返回该子对象。
     |      
     |      :param int item: 子对象的序号。
     |      :return: 子对象的的节点
     |      :rtype: list[Point2D]
     |  
     |  get\_part\_count(self)
     |      获取子对象的个数
     |      
     |      :rtype: int
     |  
     |  get\_parts(self)
     |      获取当前几何对象的所有点坐标。每个子对象使用一个 list 存储
     |      
     |      
     |      >>> points = [Point2D(1,2), Point2D(2,3), Point2D(1,5), Point2D(1,2)]
     |      >>> geo = GeoRegion(points)
     |      >>> geo.add\_part([Point2D(2,3), Point2D(4,3), Point2D(4,2), Point2D(2,3)])
     |      >>> geo.get\_parts()
     |      [[(1.0, 2.0), (2.0, 3.0), (1.0, 5.0), (1.0, 2.0)],
     |      [(2.0, 3.0), (4.0, 3.0), (4.0, 2.0), (2.0, 3.0)]]
     |      
     |      
     |      :return: 包含所有点坐标list
     |      :rtype: list[list[Point2D]]
     |  
     |  get\_parts\_topology(self)
     |      判断面对象的子对象之间的岛洞关系。 岛洞关系数组是由 1 和 -1 两个数值组成,数组大小与面对象的子对象相同。其中, 1 表示子对象为岛, -1 表示子对象为洞。
     |      
     |      :rtype: list[int]
     |  
     |  get\_precise\_area(self, prj)
     |      精确计算投影参考系下多边形的面积
     |      
     |      :param prj: 指定的投影坐标系
     |      :type prj: PrjCoordSys
     |      :return: 二维面几何对象的面积
     |      :rtype: float
     |  
     |  insert\_part(self, item, points)
     |      此面几何对象中的指定位置插入一个子对象。成功则返回 True,否则返回 False
     |      
     |      :param int item: 插入的位置
     |      :param list[Point2D] points: 插入的有序点集合
     |      :rtype: bool
     |  
     |  is\_counter\_clockwise(self, sub\_index)
     |      判断面对象的子对象的走向。true 表示走向为逆时针,false 表示走向为顺时针。
     |      
     |      :rtype: bool
     |  
     |  protected\_decompose(self)
     |      面对象保护性分解。区别于组合对象将子对象进行简单分解,保护性分解将复杂的具有多层岛洞嵌套关系的面对象分解成只有一层嵌套关系的面对象。 面对象
     |      中有子对象部分交叠的情形不能保证分解的合理性。
     |      
     |      :return: 保护性分解后得到的对象。
     |      :rtype: list[GeoRegion]
     |  
     |  remove\_part(self, item)
     |      删除此面几何对象中的指定序号的子对象。
     |      
     |      :param int item: 指定的子对象的序号
     |      :return: 成功则返回 true,否则返回 false
     |      :rtype: bool
     |  
     |  to\_json(self)
     |      将当前对象输出为 Simple Json 字符串
     |      
     |      >>> points = [Point2D(1,2), Point2D(2,3), Point2D(1,5), Point2D(1,2)]
     |      >>> geo = GeoRegion(points)
     |      >>> print(geo.to\_json())
     |      \{"Region": [[[1.0, 2.0], [2.0, 3.0], [1.0, 5.0], [1.0, 2.0]]], "id": 0\}
     |      
     |      :rtype: str
     |  
     |  ----------------------------------------------------------------------
     |  Data descriptors defined here:
     |  
     |  area
     |      float: 返回面对象的面积
     |  
     |  perimeter
     |      float: 返回面对象的周长
     |  
     |  ----------------------------------------------------------------------
     |  Methods inherited from Geometry:
     |  
     |  \_\_str\_\_(self)
     |      Return str(self).
     |  
     |  clone(self)
     |      拷贝对象
     |      
     |      :rtype: Geometry
     |  
     |  from\_xml(self, xml)
     |      根据传入的 XML 字符串重新构造几何对象。该 XML 必须符合 GML3.0 规范。
     |      调用该方法时,首先将该几何对象的原始数据清空,然后根据传入的 XML 字符串重新构造该几何对象。
     |      GML (Geography Markup Language)即地理标识语言, GML 能够表示地理空间对象的空间数据和非空间属性数据。GML 是基于 XML 的空间信息编码
     |      标准,由开放式地理信息系统协会 OpenGIS Consortium (OGC) 提出,得到了许多公司的大力支持,如 Oracle、Galdos、MapInfo、CubeWerx 等。
     |      GML 作为一个空间数据编码规范,提供了一套基本的标签、公共的数据模型,以及用户构建应用模式(GML Application Schemas)的机制。
     |      
     |      :param str xml:  XML 格式的字符串
     |      :return: 构造成功返回 True,否则返回 False。
     |      :rtype: bool
     |  
     |  get\_inner\_point(self)
     |      获取几何对象的内点
     |      
     |      :rtype: Point2D
     |  
     |  hit\_test(self, point, tolerance)
     |      测试在指定容限允许的范围内,指定的点是否在几何对象的范围内。即判断以测试点为圆心,以指定的容限为半径的圆是否与该几何对象有交集,若有交集,
     |      则返回 True;否则返回 False。
     |      
     |      :param Point2D point:  测试点
     |      :param float tolerance: 容限值,单位与数据集的单位相同
     |      :rtype: bool
     |  
     |  is\_empty(self)
     |      判断几何对象是否为空值,不同的几何对象的是否为空的条件各异。
     |      
     |      :rtype: bool
     |  
     |  linear\_extrude(self, height=0.0, twist=0.0, scaleX=1.0, scaleY=1.0, bLonLat=False)
     |      线性拉伸,支持二维和三维矢量面,二三维圆,GeoRectangle
     |      :param height: 拉伸高度
     |      :param twist: 旋转角度
     |      :param scaleX: 绕X轴缩放
     |      :param scaleY: 绕Y轴缩放
     |      :param bLonLat: 是否是经纬度
     |      :return: 返回GeoModel3D对象
     |  
     |  offset(self, dx, dy)
     |      将此几何对象偏移指定的量。
     |      
     |      :param float dx: 偏移 X 坐标的量
     |      :param float dy: 偏移 Y 坐标的量
     |      :return: self
     |      :rtype: Geometry
     |  
     |  resize(self, rc)
     |      缩放此几何对象,使其最小外接矩形等于指定的矩形对象。
     |      对于几何点,该方法只改变其位置,将其移动到指定的矩形的中心点;对于文本对象,该方法将缩放文本大小。
     |      
     |      :param Rectangle rc: 调整大小后几何对象的范围。
     |  
     |  rotate(self, base\_point, angle)
     |      以指定点为基点将此几何对象旋转指定角度,逆时针方向为正方向,角度以度为单位。
     |      
     |      :param Point2D base\_point: 旋转的基点。
     |      :param float angle: 旋转的角度,单位为度
     |  
     |  rotate\_extrude(self, angle=0.0)
     |      旋转拉伸,支持二维和三维矢量面,必须在平面坐标系下构建
     |      :param angle: 旋转角度
     |      :return: 返回GeoModel3D对象
     |  
     |  set\_empty(self)
     |      清空几何对象中的空间数据,但几何对象的标识符和几何风格保持不变。
     |  
     |  set\_id(self, value)
     |      设置几何对象的 ID 值
     |      
     |      :param int value: ID 值。
     |      :return: self
     |      :rtype: Geometry
     |  
     |  to\_geojson(self)
     |      将当前对象信息以  geojson 格式返回。只支持点、线和面对象。
     |      
     |      :rtype: str
     |  
     |  to\_xml(self)
     |      根据 GML 3.0 规范,将该几何对象的空间数据输出为 XML 字符串。 注意:几何对象输出的 XML 字符串只含有该几何对象的地理坐标值,不含有该几何对象的风格和 ID 等信息。
     |      
     |      :rtype: str
     |  
     |  ----------------------------------------------------------------------
     |  Static methods inherited from Geometry:
     |  
     |  from\_geojson(geojson)
     |      从 geojson 中读取信息构造一个几何对象
     |      
     |      :param str geojson: geojson 字符串
     |      :return: 几何对象
     |      :rtype: Geometry
     |  
     |  from\_json(value)
     |      从 json 中构造一个几何对象。参考 :py:meth:`to\_json`
     |      
     |      :param str value: json 字符串
     |      :rtype: Geometry
     |  
     |  ----------------------------------------------------------------------
     |  Data descriptors inherited from Geometry:
     |  
     |  bounds
     |      Rectangle: 返回几何对象的最小外接矩形。点对象的最小外接矩形退化为一个点,即矩形的左边界坐标值等于其右边界坐标值,上边界坐标值等于其
     |      下边界的坐标值,分别为该点的 x 坐标和 y 坐标。
     |  
     |  id
     |      int: 返回几何对象的 ID
     |  
     |  type
     |      GeometryType: 返回几何对象类型
     |  
     |  ----------------------------------------------------------------------
     |  Data descriptors inherited from iobjectspy.\_jsuperpy.data.\_jvm.JVMBase:
     |  
     |  \_\_dict\_\_
     |      dictionary for instance variables (if defined)
     |  
     |  \_\_weakref\_\_
     |      list of weak references to the object (if defined)
    
    class GeoSpheroid(iobjectspy.\_jsuperpy.data.\_jvm.JVMBase)
     |  地球椭球体参数类。 该类主要用来描述地球的长半径和扁率。
     |  
     |  人们通常用球体或椭球体来描述地球的形状和大小,有时为了计算方便,可以将地球看作一个球体,但更多的时候是把它看作椭球体。一般情况下在地图比例尺小
     |  于1:1,000,000 时,假设地球形状为一球体,因为在这种比例尺下球体和椭球体的差别几乎无法分辨;而在1:1,000,000 甚至更高精度要求的大比例尺时,
     |  则需用椭球体逼近地球。椭球体是以椭圆为基础的,所以用两个轴来表述地球球体的大小,即长轴(赤道半径)和短轴(极地半径)。
     |  
     |  因为同一个投影方法,不同的椭球体参数,相同的数据投影出来的结果可能相差很大,所以需要选择合适的椭球参数。不同年代、不同国家和地区使用的地球椭球参
     |  数有可能不同,中国目前主要用的是克拉索夫斯基椭球参数;北美大陆及英法等主要用的是克拉克椭球参数。
     |  
     |  Method resolution order:
     |      GeoSpheroid
     |      iobjectspy.\_jsuperpy.data.\_jvm.JVMBase
     |      builtins.object
     |  
     |  Methods defined here:
     |  
     |  \_\_getstate\_\_(self)
     |  
     |  \_\_init\_\_(self, spheroid\_type=None)
     |      构造地球椭球体参数类对象
     |      
     |      :param spheroid\_type: 地球椭球体参数对象类型
     |      :type spheroid\_type: GeoSpheroidType or str
     |  
     |  \_\_setstate\_\_(self, state)
     |  
     |  clone(self)
     |      复制对象
     |      
     |      :rtype: GeoSpheroid
     |  
     |  from\_xml(self, xml)
     |      从指定的 XML 字符串中构建地球椭球体参数类的对象。
     |      
     |      :param str xml: XML 字符串
     |      :return:  如果构建成功返回 True,否则返回 False
     |      :rtype: bool
     |  
     |  set\_axis(self, value)
     |      设置地球椭球体的长半径。地球椭球体的长半径也叫地球赤道半径,通过它和地球扁率可以求得地球椭球体的极地半径、第一偏心率、第二偏心率等等。只当
     |      地球椭球体的类型为自定义类型时,长半径才可以被设置。
     |      
     |      :param float value: 地球椭球体的长半径
     |      :return: self
     |      :rtype:  GeoSpheroid
     |  
     |  set\_flatten(self, value)
     |      设置地球椭球体的扁率。只当地球椭球体的类型为自定义类型时,扁率才可以被设置。地球椭球体的扁率反映了地球椭球体的圆扁情况, 一般为地球长短半轴
     |      之差与长半轴之比。
     |      
     |      :param float value:
     |      :return: self
     |      :rtype: GeoSpheroid
     |  
     |  set\_name(self, name)
     |      设置地球椭球体对象的名称
     |      
     |      :param str name: 地球椭球体对象的名称
     |      :return: self
     |      :rtype: GeoSpheroid
     |  
     |  set\_type(self, spheroid\_type)
     |      设置地球椭球体的类型。该地球椭球体类型为自定义类型时,用户需另外指定椭球体的长半径和扁率;其余的值为系统预定义,用户不必指定长半径和扁率。
     |      可参见地球椭球体 :py:class:`GeoSpheroidType` 枚举类。
     |      
     |      :param spheroid\_type:
     |      :type spheroid\_type: GeoSpheroidType or str
     |      :return: self
     |      :rtype: GeoSpheroid
     |  
     |  to\_xml(self)
     |      将地球椭球参数类的对象转换为 XML 格式的字符串。
     |      
     |      :rtype: str
     |  
     |  ----------------------------------------------------------------------
     |  Data descriptors defined here:
     |  
     |  axis
     |      float: 返回地球椭球体的长半径
     |  
     |  flatten
     |      float: 返回地球椭球体的扁率
     |  
     |  name
     |      str: 地球椭球体对象的名称
     |  
     |  type
     |      GeoSpheroidType: 返回地球椭球体的类型
     |  
     |  ----------------------------------------------------------------------
     |  Data descriptors inherited from iobjectspy.\_jsuperpy.data.\_jvm.JVMBase:
     |  
     |  \_\_dict\_\_
     |      dictionary for instance variables (if defined)
     |  
     |  \_\_weakref\_\_
     |      list of weak references to the object (if defined)
    
    class GeoText(Geometry)
     |  文本类,派生于 Geometry 类。该类主要用于对地物要素进行标识和必要的注记说明。文本对象由一个或多个部分组成,每个部分称为文本对象的一个子对象,每
     |  个子对象都是一个 TextPart 的实例。同一个文本对象的所有子对象都使用相同的文本风格,即使用该文本对象的文本风格进行显示。
     |  
     |  Method resolution order:
     |      GeoText
     |      Geometry
     |      iobjectspy.\_jsuperpy.data.\_jvm.JVMBase
     |      builtins.object
     |  
     |  Methods defined here:
     |  
     |  \_\_getstate\_\_(self)
     |  
     |  \_\_init\_\_(self, text\_part=None, text\_style=None)
     |      构造文本对象
     |      
     |      :param TextPart text\_part: 文本子对象。
     |      :param TextStyle text\_style: 文本对象的风格
     |  
     |  \_\_setstate\_\_(self, state)
     |  
     |  \_\_str\_\_(self)
     |      Return str(self).
     |  
     |  add\_part(self, text\_part)
     |      增加一个文本子对象
     |      
     |      :param TextPart text\_part: 文本子对象
     |      :return: 当添加成功则返回子对象序号,失败时返回-1。
     |      :rtype: int
     |  
     |  get\_part(self, index)
     |      获取指定的文本子对象
     |      
     |      :param int index: 文本子对象的序号
     |      :return:
     |      :rtype: int
     |  
     |  get\_part\_count(self)
     |      获取文本子对象的数目
     |      
     |      :rtype: int
     |  
     |  get\_parts(self)
     |      获取当前文本对象的所有文本子对象
     |      
     |      :rtype: list[TextPart]
     |  
     |  get\_text(self)
     |      文本对象的内容。 如果该对象有多个子对象时,其值为子对象字符串之和。
     |      
     |      :rtype: str
     |  
     |  remove\_part(self, index)
     |      删除此文本对象的指定序号的文本子对象。
     |      
     |      :param int index:
     |      :return: 如果删除成功返回 True,否则返回 False。
     |      :rtype: bool
     |  
     |  set\_part(self, index, text\_part)
     |      修改此文本对象的指定序号的子对象,即用新的文本子对象来替换原来的文本子对象。
     |      
     |      :param int index: 文本子对象序号
     |      :param TextPart text\_part: 文本子对象
     |      :return: 设置成功返回 True,否则返回 False。
     |      :rtype: bool
     |  
     |  set\_text\_style(self, text\_style)
     |      设置文本对象的文本风格。文本风格用于指定文本对象显示时的字体、宽度、高度和颜色等。
     |      
     |      :param TextStyle text\_style:  文本对象的文本风格。
     |      :return: self
     |      :rtype:  GeoText
     |  
     |  to\_json(self)
     |      将当前对象输出为 json 字符串
     |      
     |      :rtype: str
     |  
     |  ----------------------------------------------------------------------
     |  Data descriptors defined here:
     |  
     |  text\_style
     |      TextStyle: 文本对象的文本风格。文本风格用于指定文本对象显示时的字体、宽度、高度和颜色等。
     |  
     |  ----------------------------------------------------------------------
     |  Methods inherited from Geometry:
     |  
     |  clone(self)
     |      拷贝对象
     |      
     |      :rtype: Geometry
     |  
     |  from\_xml(self, xml)
     |      根据传入的 XML 字符串重新构造几何对象。该 XML 必须符合 GML3.0 规范。
     |      调用该方法时,首先将该几何对象的原始数据清空,然后根据传入的 XML 字符串重新构造该几何对象。
     |      GML (Geography Markup Language)即地理标识语言, GML 能够表示地理空间对象的空间数据和非空间属性数据。GML 是基于 XML 的空间信息编码
     |      标准,由开放式地理信息系统协会 OpenGIS Consortium (OGC) 提出,得到了许多公司的大力支持,如 Oracle、Galdos、MapInfo、CubeWerx 等。
     |      GML 作为一个空间数据编码规范,提供了一套基本的标签、公共的数据模型,以及用户构建应用模式(GML Application Schemas)的机制。
     |      
     |      :param str xml:  XML 格式的字符串
     |      :return: 构造成功返回 True,否则返回 False。
     |      :rtype: bool
     |  
     |  get\_inner\_point(self)
     |      获取几何对象的内点
     |      
     |      :rtype: Point2D
     |  
     |  hit\_test(self, point, tolerance)
     |      测试在指定容限允许的范围内,指定的点是否在几何对象的范围内。即判断以测试点为圆心,以指定的容限为半径的圆是否与该几何对象有交集,若有交集,
     |      则返回 True;否则返回 False。
     |      
     |      :param Point2D point:  测试点
     |      :param float tolerance: 容限值,单位与数据集的单位相同
     |      :rtype: bool
     |  
     |  is\_empty(self)
     |      判断几何对象是否为空值,不同的几何对象的是否为空的条件各异。
     |      
     |      :rtype: bool
     |  
     |  linear\_extrude(self, height=0.0, twist=0.0, scaleX=1.0, scaleY=1.0, bLonLat=False)
     |      线性拉伸,支持二维和三维矢量面,二三维圆,GeoRectangle
     |      :param height: 拉伸高度
     |      :param twist: 旋转角度
     |      :param scaleX: 绕X轴缩放
     |      :param scaleY: 绕Y轴缩放
     |      :param bLonLat: 是否是经纬度
     |      :return: 返回GeoModel3D对象
     |  
     |  offset(self, dx, dy)
     |      将此几何对象偏移指定的量。
     |      
     |      :param float dx: 偏移 X 坐标的量
     |      :param float dy: 偏移 Y 坐标的量
     |      :return: self
     |      :rtype: Geometry
     |  
     |  resize(self, rc)
     |      缩放此几何对象,使其最小外接矩形等于指定的矩形对象。
     |      对于几何点,该方法只改变其位置,将其移动到指定的矩形的中心点;对于文本对象,该方法将缩放文本大小。
     |      
     |      :param Rectangle rc: 调整大小后几何对象的范围。
     |  
     |  rotate(self, base\_point, angle)
     |      以指定点为基点将此几何对象旋转指定角度,逆时针方向为正方向,角度以度为单位。
     |      
     |      :param Point2D base\_point: 旋转的基点。
     |      :param float angle: 旋转的角度,单位为度
     |  
     |  rotate\_extrude(self, angle=0.0)
     |      旋转拉伸,支持二维和三维矢量面,必须在平面坐标系下构建
     |      :param angle: 旋转角度
     |      :return: 返回GeoModel3D对象
     |  
     |  set\_empty(self)
     |      清空几何对象中的空间数据,但几何对象的标识符和几何风格保持不变。
     |  
     |  set\_id(self, value)
     |      设置几何对象的 ID 值
     |      
     |      :param int value: ID 值。
     |      :return: self
     |      :rtype: Geometry
     |  
     |  to\_geojson(self)
     |      将当前对象信息以  geojson 格式返回。只支持点、线和面对象。
     |      
     |      :rtype: str
     |  
     |  to\_xml(self)
     |      根据 GML 3.0 规范,将该几何对象的空间数据输出为 XML 字符串。 注意:几何对象输出的 XML 字符串只含有该几何对象的地理坐标值,不含有该几何对象的风格和 ID 等信息。
     |      
     |      :rtype: str
     |  
     |  ----------------------------------------------------------------------
     |  Static methods inherited from Geometry:
     |  
     |  from\_geojson(geojson)
     |      从 geojson 中读取信息构造一个几何对象
     |      
     |      :param str geojson: geojson 字符串
     |      :return: 几何对象
     |      :rtype: Geometry
     |  
     |  from\_json(value)
     |      从 json 中构造一个几何对象。参考 :py:meth:`to\_json`
     |      
     |      :param str value: json 字符串
     |      :rtype: Geometry
     |  
     |  ----------------------------------------------------------------------
     |  Data descriptors inherited from Geometry:
     |  
     |  bounds
     |      Rectangle: 返回几何对象的最小外接矩形。点对象的最小外接矩形退化为一个点,即矩形的左边界坐标值等于其右边界坐标值,上边界坐标值等于其
     |      下边界的坐标值,分别为该点的 x 坐标和 y 坐标。
     |  
     |  id
     |      int: 返回几何对象的 ID
     |  
     |  type
     |      GeometryType: 返回几何对象类型
     |  
     |  ----------------------------------------------------------------------
     |  Data descriptors inherited from iobjectspy.\_jsuperpy.data.\_jvm.JVMBase:
     |  
     |  \_\_dict\_\_
     |      dictionary for instance variables (if defined)
     |  
     |  \_\_weakref\_\_
     |      list of weak references to the object (if defined)
    
    class GeometriesRelation(iobjectspy.\_jsuperpy.data.\_jvm.JVMBase)
     |  几何对象关系判断类,区别与空间查询的是,此类用于几何对象的判断,而不是数据集,实现原理上与空间查询相同。
     |  
     |  下面示例代码展示面查询点的功能,通过讲多个面对象(regions)插入到 GeometriesRelation 后,可以判断每个点对象被哪个面对象包含,
     |  自然可以得到每个面对象包含的所有点对象,在需要处理大量的点对象时,此种方式具有比较好的性能::
     |  
     |  >>> geos\_relation = GeometriesRelation()
     |  >>> for index in range(len(regions))
     |  >>>     geos\_relation.insert(regions[index], index)
     |  >>> results = dict()
     |  >>> for point in points:
     |  >>>     region\_values = geos\_relation.matches(point, 'Contain')
     |  >>>     for region\_value in region\_values:
     |  >>>         region = regions[region\_value]
     |  >>>         if region in results:
     |  >>>             results[region].append(point)
     |  >>>         else:
     |  >>>             results[region] = [point]
     |  >>> del geos\_relation
     |  
     |  Method resolution order:
     |      GeometriesRelation
     |      iobjectspy.\_jsuperpy.data.\_jvm.JVMBase
     |      builtins.object
     |  
     |  Methods defined here:
     |  
     |  \_\_init\_\_(self, tolerance=1e-10, gridding\_level='NONE')
     |      :param float tolerance: 节点容限
     |      :param gridding\_level: 面对象格网化等级。
     |      :type gridding\_level: GriddingLevel or str
     |  
     |  get\_bounds(self)
     |      获取 GeometriesRelation 中所有插入的几何对象的地理范围
     |      
     |      :rtype: Rectangle
     |  
     |  get\_gridding(self)
     |      获取面对象格网化等级。默认不做面对象格网化
     |      
     |      :rtype: GriddingLevel
     |  
     |  get\_sources\_count(self)
     |      获取 GeometriesRelation 中所有插入的几何对象数目
     |      
     |      :rtype: int
     |  
     |  get\_tolerance(self)
     |      获取节点容限
     |      
     |      :rtype: float
     |  
     |  insert(self, data, value)
     |      插入一个用于被匹配的几何对象,被匹配对象在空间查询模式中为查询对象,例如,要进行面包含点对象查询,需要插入面对象到
     |      GeometriesRelation 中,然后依次匹配得到与点对象满足包含关系的面对象。
     |      
     |      :param data: 被匹配的几何对象,必须是点线面,或点线面记录集或数据集
     |      :type data: Geometry or Point2D or Rectangle, Recordset, DatasetVector
     |      :param value: 被匹配的值,是一个唯一值,且必须大于等于0,比如几何对象的 ID 等。如果传入的是 Recordset 或 DatasetVector,
     |                    则 value 为有表示对象唯一整型值且值大于等于0的字段名称,如果为 None,则使用对象的 SmID 值。
     |      :type value: int
     |      :return: 插入成功返回 True,否则返回 False。
     |      :rtype: bool
     |  
     |  intersect\_extents(self, rc)
     |      返回与指定矩形范围相交的所有对象,即对象的矩形范围相交。
     |      
     |      :param rc: 指定的矩形范围
     |      :type rc: Rectangle
     |      :return:  与指定的矩形范围相交的对象的值
     |      :rtype: list[int]
     |  
     |  is\_match(self, data, src\_value, mode)
     |      判断对象是否与指定对象满足空间关系
     |      
     |      :param data: 要进行匹配的对象
     |      :type data:  Geometry or Point2D or Rectangle
     |      :param src\_value:  被匹配对象的值
     |      :type src\_value: int
     |      :param mode: 匹配的空间查询模式
     |      :type mode: SpatialQueryMode or str
     |      :return: 如果指定对象与指定对象满足空间关系返回 True,否则为 False。
     |      :rtype: bool
     |  
     |  matches(self, data, mode, excludes=None)
     |      找出与匹配对象满足空间关系的所有被匹配对象的值。
     |      
     |      :param data: 匹配空间对象
     |      :type data:  Geometry or Point2D or Rectangle
     |      :param mode: 匹配的空间查询模式
     |      :type mode:  SpatialQueryMode or str
     |      :param excludes:  排除的值,即不参与匹配运算
     |      :type excludes: list[int]
     |      :return: 被匹配对象的值
     |      :rtype: list[int]
     |  
     |  set\_gridding(self, gridding\_level)
     |      设置面对象格网化等级。默认不做面对象格网化。
     |      
     |      :param gridding\_level: 格网化等级
     |      :type gridding\_level: GriddingLevel or str
     |      :return: self
     |      :rtype: GeometriesRelation
     |  
     |  set\_tolerance(self, tolerance)
     |      设置节点容限
     |      
     |      :param float tolerance: 节点容限
     |      :return: self
     |      :rtype: GeometriesRelation
     |  
     |  ----------------------------------------------------------------------
     |  Data descriptors inherited from iobjectspy.\_jsuperpy.data.\_jvm.JVMBase:
     |  
     |  \_\_dict\_\_
     |      dictionary for instance variables (if defined)
     |  
     |  \_\_weakref\_\_
     |      list of weak references to the object (if defined)
    
    class Geometry(iobjectspy.\_jsuperpy.data.\_jvm.JVMBase)
     |  几何对象基类,用于表示地理实体的空间特征。并提供相关的处理方法。根据地理实体的空间特征不同,分别用点(GeoPoint),线(GeoLine),面(GeoRegion)等加以描述
     |  
     |  Method resolution order:
     |      Geometry
     |      iobjectspy.\_jsuperpy.data.\_jvm.JVMBase
     |      builtins.object
     |  
     |  Methods defined here:
     |  
     |  \_\_init\_\_(self)
     |      Initialize self.  See help(type(self)) for accurate signature.
     |  
     |  \_\_str\_\_(self)
     |      Return str(self).
     |  
     |  clone(self)
     |      拷贝对象
     |      
     |      :rtype: Geometry
     |  
     |  from\_xml(self, xml)
     |      根据传入的 XML 字符串重新构造几何对象。该 XML 必须符合 GML3.0 规范。
     |      调用该方法时,首先将该几何对象的原始数据清空,然后根据传入的 XML 字符串重新构造该几何对象。
     |      GML (Geography Markup Language)即地理标识语言, GML 能够表示地理空间对象的空间数据和非空间属性数据。GML 是基于 XML 的空间信息编码
     |      标准,由开放式地理信息系统协会 OpenGIS Consortium (OGC) 提出,得到了许多公司的大力支持,如 Oracle、Galdos、MapInfo、CubeWerx 等。
     |      GML 作为一个空间数据编码规范,提供了一套基本的标签、公共的数据模型,以及用户构建应用模式(GML Application Schemas)的机制。
     |      
     |      :param str xml:  XML 格式的字符串
     |      :return: 构造成功返回 True,否则返回 False。
     |      :rtype: bool
     |  
     |  get\_inner\_point(self)
     |      获取几何对象的内点
     |      
     |      :rtype: Point2D
     |  
     |  hit\_test(self, point, tolerance)
     |      测试在指定容限允许的范围内,指定的点是否在几何对象的范围内。即判断以测试点为圆心,以指定的容限为半径的圆是否与该几何对象有交集,若有交集,
     |      则返回 True;否则返回 False。
     |      
     |      :param Point2D point:  测试点
     |      :param float tolerance: 容限值,单位与数据集的单位相同
     |      :rtype: bool
     |  
     |  is\_empty(self)
     |      判断几何对象是否为空值,不同的几何对象的是否为空的条件各异。
     |      
     |      :rtype: bool
     |  
     |  linear\_extrude(self, height=0.0, twist=0.0, scaleX=1.0, scaleY=1.0, bLonLat=False)
     |      线性拉伸,支持二维和三维矢量面,二三维圆,GeoRectangle
     |      :param height: 拉伸高度
     |      :param twist: 旋转角度
     |      :param scaleX: 绕X轴缩放
     |      :param scaleY: 绕Y轴缩放
     |      :param bLonLat: 是否是经纬度
     |      :return: 返回GeoModel3D对象
     |  
     |  offset(self, dx, dy)
     |      将此几何对象偏移指定的量。
     |      
     |      :param float dx: 偏移 X 坐标的量
     |      :param float dy: 偏移 Y 坐标的量
     |      :return: self
     |      :rtype: Geometry
     |  
     |  resize(self, rc)
     |      缩放此几何对象,使其最小外接矩形等于指定的矩形对象。
     |      对于几何点,该方法只改变其位置,将其移动到指定的矩形的中心点;对于文本对象,该方法将缩放文本大小。
     |      
     |      :param Rectangle rc: 调整大小后几何对象的范围。
     |  
     |  rotate(self, base\_point, angle)
     |      以指定点为基点将此几何对象旋转指定角度,逆时针方向为正方向,角度以度为单位。
     |      
     |      :param Point2D base\_point: 旋转的基点。
     |      :param float angle: 旋转的角度,单位为度
     |  
     |  rotate\_extrude(self, angle=0.0)
     |      旋转拉伸,支持二维和三维矢量面,必须在平面坐标系下构建
     |      :param angle: 旋转角度
     |      :return: 返回GeoModel3D对象
     |  
     |  set\_empty(self)
     |      清空几何对象中的空间数据,但几何对象的标识符和几何风格保持不变。
     |  
     |  set\_id(self, value)
     |      设置几何对象的 ID 值
     |      
     |      :param int value: ID 值。
     |      :return: self
     |      :rtype: Geometry
     |  
     |  to\_geojson(self)
     |      将当前对象信息以  geojson 格式返回。只支持点、线和面对象。
     |      
     |      :rtype: str
     |  
     |  to\_json(self)
     |      将当前对象输出为 json 字符串
     |      
     |      :rtype: str
     |  
     |  to\_xml(self)
     |      根据 GML 3.0 规范,将该几何对象的空间数据输出为 XML 字符串。 注意:几何对象输出的 XML 字符串只含有该几何对象的地理坐标值,不含有该几何对象的风格和 ID 等信息。
     |      
     |      :rtype: str
     |  
     |  ----------------------------------------------------------------------
     |  Static methods defined here:
     |  
     |  from\_geojson(geojson)
     |      从 geojson 中读取信息构造一个几何对象
     |      
     |      :param str geojson: geojson 字符串
     |      :return: 几何对象
     |      :rtype: Geometry
     |  
     |  from\_json(value)
     |      从 json 中构造一个几何对象。参考 :py:meth:`to\_json`
     |      
     |      :param str value: json 字符串
     |      :rtype: Geometry
     |  
     |  ----------------------------------------------------------------------
     |  Data descriptors defined here:
     |  
     |  bounds
     |      Rectangle: 返回几何对象的最小外接矩形。点对象的最小外接矩形退化为一个点,即矩形的左边界坐标值等于其右边界坐标值,上边界坐标值等于其
     |      下边界的坐标值,分别为该点的 x 坐标和 y 坐标。
     |  
     |  id
     |      int: 返回几何对象的 ID
     |  
     |  type
     |      GeometryType: 返回几何对象类型
     |  
     |  ----------------------------------------------------------------------
     |  Data descriptors inherited from iobjectspy.\_jsuperpy.data.\_jvm.JVMBase:
     |  
     |  \_\_dict\_\_
     |      dictionary for instance variables (if defined)
     |  
     |  \_\_weakref\_\_
     |      list of weak references to the object (if defined)
    
    class JoinItem(builtins.object)
     |  连接信息类。用于矢量数据集与外部表的连接。外部表可以为另一个矢量数据集(其中纯属性数据集中没有空间几何信息)所对应的DBMS表,也可以是用户自建
     |  的业务表。需要注意的是,矢量数据集与外部表必须属于同一数据源。当两个表格之间建立了连接,通过对主表进行操作,可以对外部表进行查询,制作专题图以
     |  及分析等。当两个表格之间是一对一或多对一的关系时,可以使用join连接。当为多对一的关系时,允许指定多个字段之间的关联。该类型的实例可被创建。
     |  
     |  数据集表之间的联系的建立有两种方式,一种是连接(join),一种是关联(link)。连接的相关设置是通过 JoinItem 类实现的,关联的相关设置是通过
     |  LinkItem 类实现的,另外,用于建立连接的两个数据集表必须在同一个数据源下,而用于建立关联关系的两个数据集表可以不在同一个数据源下。
     |  
     |  下面通过查询的例子来说明连接和关联的区别,假设用来进行查询的数据集表为 DatasetTableA,被关联或者连接的表为 DatasetTableB,现通过建立
     |  DatasetTableA 与 DatasetTableB 的连接或关联关系来查询 DatasetTableA 中满足查询条件的记录:
     |  
     |      - 连接(join)
     |          设置将 DatasetTableB 连接 DatasetTableA 的连接信息,即建立 JoinItem 类并设置其属性,当执行 DatasetTableA 的查询操作时,
     |          系统将根据连接条件及查询条件,将满足条件的 DatasetTableA 中的内容与满足条件的 DatasetTableB 中的内容构成一个查询结果表,并且这个
     |          查询表保存在内存中,需要返回结果时,再从内存中取出相应的内容。
     |  
     |      - 关联(link)
     |          设置将 DatasetTableB (副表)关联到 DatasetTableA (主表)的关联信息,即建立 LinkItem 类并设置其属性,DatasetTableA 与
     |          DatasetTableB 是通过主表 DatasetTableA 的外键(LinkItem.foreign\_keys)和副表 DatasetTableB 的主键
     |          (LinkItem.primary\_keys 方法)实现关联的,当执行 DatasetTableA 的查询操作时,系统将根据关联信息中的过滤条件及查询条件,
     |          分别查询 DatasetTableA 与 DatasetTableB 中满足条件的内容,DatasetTableA 的查询结果与 DatasetTableB 的查询结果分别作为独立
     |          的两个结果表保存在内存中,当需要返回结果时,SuperMap 将对两个结果进行拼接并返回,因此,在应用层看来,连接和关联操作很相似。
     |  
     |      - LinkItem 只支持左连接,UDB、PostgreSQL 和 DB2 数据源不支持 LinkItem,即对 UDB、PostgreSQL 和 DB2 类型的数据引擎设置 LinkItem 不起作用;
     |  
     |      - JoinItem 目前支持左连接和内连接,不支持全连接和右连接,UDB 引擎不支持内连接;
     |  
     |      - 使用 LinkItem 的约束条件:空间数据和属性数据必须有关联条件,即主空间数据集与外部属性表之间存在关联字段。主空间数据集:用来与外部表进行关联的数据集。外部属性表:用户通过 Oracle 或者 SQL Server 创建的数据表,或者是另一个矢量数据集所对应的 DBMS 表。
     |  
     |  
     |  示例::
     |  
     |      >>> ds = Workspace().get\_datasource('data')
     |      >>> dataset\_world = ds['World']
     |      >>> dataset\_capital = ds['Capital']
     |      >>> foreign\_table\_name = dataset\_capital.table\_name
     |      >>>
     |      >>> join\_item = JoinItem()
     |      >>> join\_item.set\_foreign\_table(foreign\_table\_name)
     |      >>> join\_item.set\_join\_filter('World.capital=\%s.capital' \% foreign\_table\_name)
     |      >>> join\_item.set\_join\_type(JoinType.LEFTJOIN)
     |      >>> join\_item.set\_name('Connect')
     |  
     |      >>> query\_parameter  = QueryParameter()
     |      >>> query\_parameter.set\_join\_items([join\_item])
     |      >>> recordset = dataset\_world.query(query\_parameter)
     |      >>> print(recordset.get\_record\_count())
     |      >>> recordset.close()
     |  
     |  Methods defined here:
     |  
     |  \_\_init\_\_(self, name=None, foreign\_table=None, join\_filter=None, join\_type=None)
     |      构造 JoinItem 对象
     |      
     |      :param str name: 连接信息对象的名称
     |      :param str foreign\_table: 外部表的名称
     |      :param str join\_filter:  与外部表之间的连接表达式,即设定两个表之间关联的字段
     |      :param str join\_type: 两个表之间连接的类型
     |  
     |  \_\_str\_\_(self)
     |      Return str(self).
     |  
     |  from\_dict(self, values)
     |      从 dict 读取JoinItem 信息
     |      
     |      :param dict values: 含有 JoinItem 信息的 dict,具体查看 to\_dict
     |      :return: self
     |      :rtype: JoinItem
     |  
     |  set\_foreign\_table(self, value)
     |      设置连接信息外部表名称
     |      
     |      :param str value: 外部表名称
     |      :return: self
     |      :rtype: JoinItem
     |  
     |  set\_join\_filter(self, value)
     |      设置与外部表之间的连接表达式,即设定两个表之间关联的字段。例如,将一个房屋的面数据集(Building)的 district 字段与一个房屋拥有者的纯属
     |      性数据集(Owner)的 region 字段相连接,两个数据集对应的表名称分别为 Table\_Building 和 Table\_Owner,则连接表达式为
     |      Table\_Building.district = Table\_Owner.region,当有多个字段相连接时,用 AND 将多个表达式相连。
     |      
     |      :param str value: 与外部表之间的连接表达式,即设定两个表之间关联的字段
     |      :return: self
     |      :rtype: JoinItem
     |  
     |  set\_join\_type(self, value)
     |      设置两个表之间连接的类型。连接类型用于对两个连接的表进行查询时,决定了返回的记录的情况。
     |      
     |      :param value: 两个表之间连接的类型
     |      :type value: JoinType or str
     |      :return: self
     |      :rtype: JoinItem
     |  
     |  set\_name(self, value)
     |      设置连接信息对象的名称
     |      
     |      :param str value:  连接信息名称
     |      :return: self
     |      :rtype: JoinItem
     |  
     |  to\_dict(self)
     |      将当前对象信息输出为 dict
     |      
     |      :rtype: dict
     |  
     |  to\_json(self)
     |      将当前对象输出为 json 字符串
     |      
     |      :rtype: str
     |  
     |  ----------------------------------------------------------------------
     |  Static methods defined here:
     |  
     |  from\_json(value)
     |      从 json 字符串中解析 JoinItem 信息,构造一个新的 JoinItem 对象
     |      
     |      :param str value:  json 字符串
     |      :rtype: JoinItem
     |  
     |  make\_from\_dict(values)
     |      从 dict 读取信息构造 JoinItem 对象。
     |      
     |      :param dict values: 含有 JoinItem 信息的 dict,具体查看 to\_dict
     |      :rtype: JoinItem
     |  
     |  ----------------------------------------------------------------------
     |  Data descriptors defined here:
     |  
     |  \_\_dict\_\_
     |      dictionary for instance variables (if defined)
     |  
     |  \_\_weakref\_\_
     |      list of weak references to the object (if defined)
     |  
     |  foreign\_table
     |      str: 外部表的名称
     |  
     |  join\_filter
     |      str: 与外部表之间的连接表达式,即设定两个表之间关联的字段
     |  
     |  join\_type
     |      JoinType: 两个表之间连接的类型
     |  
     |  name
     |      str:  连接信息对象的名称
    
    class LinkItem(builtins.object)
     |  关联信息类,用于矢量数据集与其它数据集的关联。关联数据集可以为另一个矢量数据集(其中纯属性数据集中没有空间几何信息)所对应的 DBMS 表,用户自
     |  建业务表需要外挂到 SuperMap 数据源中。需要注意的是,矢量数据集与关联数据集可以属于不同的数据源。数据集表之间的联系的建立有两种方式,一种是
     |  连接(join),一种是关联(link)。连接的相关设置是通过 JoinItem 类实现的,关联的相关设置是通过 LinkItem 类实现的,另外,用于建立连接的两
     |  个数据集表必须在同一个数据源下,而用于建立关联关系的两个数据集表可以不在同一个数据源下。
     |  下面通过查询的例子来说明连接和关联的区别,假设用来进行查询的数据集表为 DatasetTableA,被关联或者连接的表为 DatasetTableB,现通过建立
     |  DatasetTableA 与 DatasetTableB 的连接或关联关系来查询 DatasetTableA 中满足查询条件的记录:
     |  
     |      - 连接(join)
     |          设置将 DatasetTableB 连接 DatasetTableA 的连接信息,即建立 JoinItem 类并设置其属性,当执行 DatasetTableA 的查询操作时,
     |          系统将根据连接条件及查询条件,将满足条件的 DatasetTableA 中的内容与满足条件的 DatasetTableB 中的内容构成一个查询结果表,并且这个
     |          查询表保存在内存中,需要返回结果时,再从内存中取出相应的内容。
     |  
     |      - 关联(link)
     |          设置将 DatasetTableB (副表)关联到 DatasetTableA (主表)的关联信息,即建立 LinkItem 类并设置其属性,DatasetTableA 与
     |          DatasetTableB 是通过主表 DatasetTableA 的外键(LinkItem.foreign\_keys)和副表 DatasetTableB 的主键
     |          (LinkItem.primary\_keys 方法)实现关联的,当执行 DatasetTableA 的查询操作时,系统将根据关联信息中的过滤条件及查询条件,
     |          分别查询 DatasetTableA 与 DatasetTableB 中满足条件的内容,DatasetTableA 的查询结果与 DatasetTableB 的查询结果分别作为独立
     |          的两个结果表保存在内存中,当需要返回结果时,SuperMap 将对两个结果进行拼接并返回,因此,在应用层看来,连接和关联操作很相似。
     |  
     |      - LinkItem 只支持左连接,UDB、PostgreSQL 和 DB2 数据源不支持 LinkItem,即对 UDB、PostgreSQL 和 DB2 类型的数据引擎设置 LinkItem 不起作用;
     |  
     |      - JoinItem 目前支持左连接和内连接,不支持全连接和右连接,UDB 引擎不支持内连接;
     |  
     |      - 使用 LinkItem 的约束条件:空间数据和属性数据必须有关联条件,即主空间数据集与外部属性表之间存在关联字段。主空间数据集:用来与外部表进行关联的数据集。外部属性表:用户通过 Oracle 或者 SQL Server 创建的数据表,或者是另一个矢量数据集所对应的 DBMS 表。
     |  
     |  示例:
     |      \# 'source' 数据集为主数据集,source 数据集用于关联的字段为 'LinkID','lind\_dt' 数据集为外表数据集,即被关联的数据集,link\_dt 中用于关联的字段为 'ID'
     |  
     |      >>> ds\_db1 = Workspace().get\_datasource('data\_db\_1')
     |      >>> ds\_db2 = Workspace().get\_datasource('data\_db\_2')
     |      >>> source\_dataset = ds\_db1['source']
     |      >>> linked\_dataset = ds\_db2['link\_dt']
     |      >>> linked\_dataset\_name = linked\_dataset.name
     |      >>> linked\_dataset\_table\_name = linked\_dataset.table\_name
     |      >>>
     |      >>> link\_item = LinkItem()
     |      >>>
     |      >>> link\_item.set\_connection\_info(ds\_db2.connection\_info)
     |      >>> link\_item.set\_foreign\_table(linked\_dataset\_name)
     |      >>> link\_item.set\_foreign\_keys(['LinkID'])
     |      >>> link\_item.set\_primary\_keys(['ID'])
     |      >>> link\_item.set\_link\_fields([linked\_dataset\_table\_name+'.polulation'])
     |      >>> link\_item.set\_link\_filter('ID < 100')
     |      >>> link\_item.set\_name('link\_name')
     |  
     |  Methods defined here:
     |  
     |  \_\_init\_\_(self, foreign\_keys=None, name=None, foreign\_table=None, primary\_keys=None, link\_fields=None, link\_filter=None, connection\_info=None)
     |      构造 LinkItem 对象
     |      
     |      :param list[str] foreign\_keys: 主数据集用于关联外表的字段
     |      :param str name: 关联信息对象的名称
     |      :param str foreign\_table: 外表的数据集名称,即被关联的数据集名称
     |      :param list[str] primary\_keys: 外表数据集中用于关联的字段
     |      :param list[str] link\_fields: 外表数据集中被查询的字段名称
     |      :param str link\_filter: 外表数据集的查询条件
     |      :param DatasourceConnectionInfo connection\_info: 外表数据集所在数据源的连接信息
     |  
     |  from\_dict(self, values)
     |      从 dict 读取信息构造 LinkItem 对象。
     |      
     |      :param dict values:  含有 LinkItem 信息的 dict,具体查看 to\_dict
     |      :return: self
     |      :rtype: LinkItem
     |  
     |  link\_filter(self)
     |      str:  外表数据集的查询条件
     |  
     |  set\_connection\_info(self, value)
     |      设置外表数据集所在数据源的连接信息
     |      
     |      :param DatasourceConnectionInfo value: 外表数据集所在数据源的连接信息
     |      :return: self
     |      :rtype: LinkItem
     |  
     |  set\_foreign\_keys(self, value)
     |      设置主数据集用于关联外表的字段
     |      
     |      :param list[str] value: 主数据集用于关联外表的字段
     |      :return: self
     |      :rtype: LinkItem
     |  
     |  set\_foreign\_table(self, value)
     |      设置外表的数据集名称,即被关联的数据集名称
     |      
     |      :param str value: 外表的数据集名称,即被关联的数据集名称
     |      :return: self
     |      :rtype: LinkItem
     |  
     |  set\_link\_fields(self, value)
     |      设置外表数据集中被查询的字段名称
     |      
     |      :param list[str] value: 外表数据集中被查询的字段名称
     |      :return: self
     |      :rtype: LinkItem
     |  
     |  set\_link\_filter(self, value)
     |      设置外表数据集的查询条件
     |      
     |      :param str value: 外表数据集的查询条件
     |      :return: self
     |      :rtype: LinkItem
     |  
     |  set\_name(self, value)
     |      设置关联信息对象的名称
     |      
     |      :param str value: 关联信息对象的名称
     |      :return: self
     |      :rtype: LinkItem
     |  
     |  set\_primary\_keys(self, value)
     |      设置外表数据集中用于关联的字段
     |      
     |      :param list[str] value: 外表数据集中用于关联的字段
     |      :return: self
     |      :rtype: LinkItem
     |  
     |  to\_dict(self)
     |      将当前对象的信息输出到dict中
     |      
     |      :rtype: dict
     |  
     |  to\_json(self)
     |      将当前对象信息输出到 json 字符串中,具体查看 to\_dict.
     |      
     |      :rtype: str
     |  
     |  ----------------------------------------------------------------------
     |  Static methods defined here:
     |  
     |  from\_json(value)
     |      从 json 字符串中构造 LinkItem 对象
     |      
     |      :param str value:  json 字符串信息
     |      :rtype: LinkItem
     |  
     |  make\_from\_dict(values)
     |      从 dict 读取信息构造 LinkItem 对象,构造一个新的 LinkItem 对象
     |      
     |      :param dict values: 含有 LinkItem 信息的 dict,具体查看 to\_dict
     |      :rtype: LinkItem
     |  
     |  ----------------------------------------------------------------------
     |  Data descriptors defined here:
     |  
     |  \_\_dict\_\_
     |      dictionary for instance variables (if defined)
     |  
     |  \_\_weakref\_\_
     |      list of weak references to the object (if defined)
     |  
     |  connection\_info
     |      DatasourceConnectionInfo: 外表数据集所在数据源的连接信息
     |  
     |  foreign\_keys
     |      list[str]: 主数据集用于关联外表的字段
     |  
     |  foreign\_table
     |      str: 外表的数据集名称,即被关联的数据集名称
     |  
     |  link\_fields
     |      list[str]: 外表数据集中被查询的字段名称
     |  
     |  name
     |      str: 关联信息对象的名称
     |  
     |  primary\_keys
     |      list[str]:  外表数据集中用于关联的字段
    
    class ObjectDisposedError(builtins.RuntimeError)
     |  对象被释放后的异常对象。在检查 Python 实例中绑定的 java 对象被释放后会抛出该异常。
     |  
     |  Method resolution order:
     |      ObjectDisposedError
     |      builtins.RuntimeError
     |      builtins.Exception
     |      builtins.BaseException
     |      builtins.object
     |  
     |  Methods defined here:
     |  
     |  \_\_init\_\_(self, message)
     |      Initialize self.  See help(type(self)) for accurate signature.
     |  
     |  \_\_repr\_\_ = \_\_str\_\_(self)
     |  
     |  \_\_str\_\_(self)
     |      Return str(self).
     |  
     |  ----------------------------------------------------------------------
     |  Data descriptors defined here:
     |  
     |  \_\_weakref\_\_
     |      list of weak references to the object (if defined)
     |  
     |  ----------------------------------------------------------------------
     |  Methods inherited from builtins.RuntimeError:
     |  
     |  \_\_new\_\_(*args, **kwargs) from builtins.type
     |      Create and return a new object.  See help(type) for accurate signature.
     |  
     |  ----------------------------------------------------------------------
     |  Methods inherited from builtins.BaseException:
     |  
     |  \_\_delattr\_\_(self, name, /)
     |      Implement delattr(self, name).
     |  
     |  \_\_getattribute\_\_(self, name, /)
     |      Return getattr(self, name).
     |  
     |  \_\_reduce\_\_({\ldots})
     |      helper for pickle
     |  
     |  \_\_setattr\_\_(self, name, value, /)
     |      Implement setattr(self, name, value).
     |  
     |  \_\_setstate\_\_({\ldots})
     |  
     |  with\_traceback({\ldots})
     |      Exception.with\_traceback(tb) --
     |      set self.\_\_traceback\_\_ to tb and return self.
     |  
     |  ----------------------------------------------------------------------
     |  Data descriptors inherited from builtins.BaseException:
     |  
     |  \_\_cause\_\_
     |      exception cause
     |  
     |  \_\_context\_\_
     |      exception context
     |  
     |  \_\_dict\_\_
     |  
     |  \_\_suppress\_context\_\_
     |  
     |  \_\_traceback\_\_
     |  
     |  args
    
    class Point2D(builtins.object)
     |  二维点对象,使用两个浮点数分别表示 x 和 y 轴的位置。
     |  
     |  Methods defined here:
     |  
     |  \_\_cmp\_\_(self, other)
     |  
     |  \_\_eq\_\_(self, other)
     |      Return self==value.
     |  
     |  \_\_getitem\_\_(self, item)
     |  
     |  \_\_getstate\_\_(self)
     |  
     |  \_\_init\_\_(self, x=None, y=None)
     |      使用 x 和 y 值构造二维点对象。
     |      
     |      :param float x: x 坐标轴值
     |      :param float y: y 坐标值
     |  
     |  \_\_len\_\_(self)
     |  
     |  \_\_lt\_\_(self, other)
     |      Return self<value.
     |  
     |  \_\_repr\_\_(self)
     |      Return repr(self).
     |  
     |  \_\_setstate\_\_(self, state)
     |  
     |  \_\_str\_\_(self)
     |      Return str(self).
     |  
     |  clone(self)
     |      复制当前对象,返回一个新的对象
     |      
     |      :rtype: Point2D
     |  
     |  distance\_to(self, other)
     |      计算当前点与指定点之间的距离
     |      
     |      :param Point2D other: 目标点
     |      :return: 返回两个点之间的几何距离
     |      :rtype: float
     |  
     |  equal(self, other, tolerance=0.0)
     |      判断当前点与指定点在容限范围内是否相等
     |      
     |      :param Point2D other: 待判断的点
     |      :param float tolerance: 容限值
     |      :rtype: bool
     |  
     |  from\_dict(self, value)
     |      从 dict 中读取点信息
     |      
     |      :param dict value: 包含 x 和 y 值的 dict
     |      :return: self
     |      :rtype: Point2D
     |  
     |  to\_dict(self)
     |      输出为 dict 对象
     |      
     |      :rtype: dict
     |  
     |  to\_json(self)
     |      将当前二维点对象输出为 json 字符串
     |      
     |      :rtype: str
     |  
     |  ----------------------------------------------------------------------
     |  Static methods defined here:
     |  
     |  from\_json(value)
     |      从 json 字符串中构造二维点坐标
     |      
     |      :param str value: json 字符串
     |      :rtype: Point2D
     |  
     |  make(p)
     |      构造一个二维点对象
     |      
     |      :param p: x 和 y 值
     |      :type p: tuple[float,float] or list[float,float] or GeoPoint or Point2D or dict
     |      :rtype: Point2D
     |  
     |  make\_from\_dict(value)
     |      从 dict 中读取点信息构建二维点对象
     |      
     |      :param dict  value: 包含 x 和 y 值的 dict
     |      :rtype: Point2D
     |  
     |  ----------------------------------------------------------------------
     |  Data descriptors defined here:
     |  
     |  \_\_dict\_\_
     |      dictionary for instance variables (if defined)
     |  
     |  \_\_weakref\_\_
     |      list of weak references to the object (if defined)
     |  
     |  ----------------------------------------------------------------------
     |  Data and other attributes defined here:
     |  
     |  \_\_hash\_\_ = None
    
    class Point3D(builtins.object)
     |  Methods defined here:
     |  
     |  \_\_cmp\_\_(self, other)
     |  
     |  \_\_eq\_\_(self, other)
     |      Return self==value.
     |  
     |  \_\_getitem\_\_(self, item)
     |  
     |  \_\_getstate\_\_(self)
     |  
     |  \_\_init\_\_(self, x=None, y=None, z=None)
     |      Initialize self.  See help(type(self)) for accurate signature.
     |  
     |  \_\_len\_\_(self)
     |  
     |  \_\_lt\_\_(self, other)
     |      Return self<value.
     |  
     |  \_\_repr\_\_(self)
     |      Return repr(self).
     |  
     |  \_\_setstate\_\_(self, state)
     |  
     |  \_\_str\_\_(self)
     |      Return str(self).
     |  
     |  clone(self)
     |      复制当前对象
     |      
     |      :rtype: Point3D
     |  
     |  from\_dict(self, value)
     |      从 dict 中读取点信息
     |      
     |      :param dict value: 包含 x、y 和z值的 dict
     |      :return: self
     |      :rtype: Point3D
     |  
     |  to\_dict(self)
     |      输出为 dict 对象
     |      
     |      :rtype: dict
     |  
     |  to\_json(self)
     |      将当前三维点对象输出为 json 字符串
     |      
     |      :rtype: str
     |  
     |  ----------------------------------------------------------------------
     |  Static methods defined here:
     |  
     |  from\_json(value)
     |      从 json 字符串中构造三维点坐标
     |      
     |      :param str value: json 字符串
     |      :rtype: Point3D
     |  
     |  make(p)
     |      构造一个三维点对象
     |      
     |      :param p: x,y 和 z 值
     |      :type p: tuple[float,float,float] or list[float,float,float] or GeoPoint3D or Point3D or dict
     |      :rtype: Point2D
     |  
     |  make\_from\_dict(value)
     |      从 dict 中读取点信息构造三维点坐标
     |      
     |      :param dict value: 包含 x、y 和z值的 dict
     |      :return: self
     |      :rtype: Point3D
     |  
     |  ----------------------------------------------------------------------
     |  Data descriptors defined here:
     |  
     |  \_\_dict\_\_
     |      dictionary for instance variables (if defined)
     |  
     |  \_\_weakref\_\_
     |      list of weak references to the object (if defined)
     |  
     |  ----------------------------------------------------------------------
     |  Data and other attributes defined here:
     |  
     |  \_\_hash\_\_ = None
    
    class PrjCoordSys(iobjectspy.\_jsuperpy.data.\_jvm.JVMBase)
     |  投影坐标系类。投影坐标系统由地图投影方式、投影参数、坐标单位和地理坐标系组成。SuperMap Objects Java 中提供了很多预定义的投影系统,用户可以
     |  直接使用,此外,用户还可以定制自己投影系统。投影坐标系是定义在二维平面上的,不同于地理坐标系用经纬度定位地面点,投影坐标系是用 X、Y 坐标来定位
     |  的。每一个投影坐标系都基于一个地理坐标系。
     |  
     |  Method resolution order:
     |      PrjCoordSys
     |      iobjectspy.\_jsuperpy.data.\_jvm.JVMBase
     |      builtins.object
     |  
     |  Methods defined here:
     |  
     |  \_\_getstate\_\_(self)
     |  
     |  \_\_init\_\_(self, prj\_type=None)
     |      构造投影坐标系对象
     |      
     |      :param prj\_type: 投影坐标系类型
     |      :type prj\_type: PrjCoordSysType or str
     |  
     |  \_\_setstate\_\_(self, state)
     |  
     |  clone(self)
     |      拷贝一个对象
     |      
     |      :rtype: PrjCoordSys
     |  
     |  from\_file(self, file\_path)
     |      从 xml 文件或 prj 文件中读取投影坐标信息
     |      
     |      :param str file\_path: 文件路径
     |      :return: 构建成功返回 True,否则返回 False
     |      :rtype: bool
     |  
     |  from\_xml(self, xml)
     |      从 xml 字符串中读取投影信息
     |      
     |      :param str xml: xml 字符串
     |      :return: 如果构建成功返回 True,否则返回 False。
     |      :rtype: bool
     |  
     |  set\_geo\_coordsys(self, geo\_coordsys)
     |      设置投影坐标系的地理坐标系统对象。每个投影系都要依赖于一个地理坐标系。该方法仅在坐标系类型为自定义投影坐标系和自定义地理坐标系时有效。
     |      
     |      :param GeoCoordSys geo\_coordsys:
     |      :return: self
     |      :rtype: PrjCoordSys
     |  
     |  set\_name(self, name)
     |      设置投影坐标系对象的名称
     |      
     |      :param str name: 投影坐标系对象的名称
     |      :return: self
     |      :rtype: PrjCoordSys
     |  
     |  set\_prj\_parameter(self, parameter)
     |      设置投影坐标系统对象的投影参数。
     |      
     |      :param PrjParameter parameter: 投影坐标系统对象的投影参数
     |      :return: self
     |      :rtype: PrjCoordSys
     |  
     |  set\_projection(self, projection)
     |      设置投影坐标系统的投影方式。投影方式如等角圆锥投影、等距方位投影等等。
     |      
     |      :param Projection projection:
     |      :return: self
     |      :rtype: PrjCoordSys
     |  
     |  set\_type(self, prj\_type)
     |      设置投影坐标系类型
     |      
     |      :param prj\_type: 投影坐标系类型
     |      :type prj\_type: PrjCoordSysType or str
     |      :return: self
     |      :rtype: PrjCoordSys
     |  
     |  to\_epsg\_code(self)
     |      返回当前对象的 EPSG 编码
     |      
     |      :rtype: int
     |  
     |  to\_file(self, file\_path)
     |      将投影坐标信息输出到文件中。只支持输出为 xml 文件。
     |      
     |      :param str file\_path:  XML 文件的全路径。
     |      :return: 导出成功返回 True,否则返回 False。
     |      :rtype: bool
     |  
     |  to\_wkt(self)
     |      将当前投影信息输出为 WKT 字符串
     |      
     |      :rtype: str
     |  
     |  to\_xml(self)
     |      将投影坐标系类的对象转换为 XML 格式的字符串。
     |      
     |      :return: 表示投影坐标系类的对象的 XML 字符串
     |      :rtype: str
     |  
     |  ----------------------------------------------------------------------
     |  Static methods defined here:
     |  
     |  from\_epsg\_code(code)
     |      由 EPSG 编码构造投影坐标系对象
     |      
     |      :param int code: EPSG 编码
     |      :rtype: PrjCoordSys
     |  
     |  from\_wkt(wkt)
     |      从 WKT 字符串中构建投影坐标系对象
     |      
     |      :param str wkt: WKT 字符串
     |      :rtype: PrjCoordSys
     |  
     |  make(prj)
     |      构造 PrjCoordSys 对象,支持从 epsg 编码,PrjCoordSysType 类型,xml 或者 wkt,或者投影信息文件中构造。注意,如果传入整型值,
     |      必须是 epsg 编码,不能是 PrjCoordSysType 类型的整型值。
     |      
     |      :param prj: 投影信息
     |      :type prj: int or str or PrjCoordSysType
     |      :return: 投影对象
     |      :rtype: PrjCoordSys
     |  
     |  ----------------------------------------------------------------------
     |  Data descriptors defined here:
     |  
     |  coord\_unit
     |      Unit: 返回投影系统坐标单位。投影系统的坐标单位与距离单位(distance\_unit)可以不同,例如经纬度坐标下的坐标单位是度,距离单位可以是米、
     |      公里等;即使是普通平面坐标或者投影坐标,这两个单位同样可不同。
     |  
     |  distance\_unit
     |      Unit: 距离(长度)单位
     |  
     |  geo\_coordsys
     |      GeoCoordSys: 投影坐标系的地理坐标系统对象
     |  
     |  name
     |      str: 投影坐标系对象的名称
     |  
     |  prj\_parameter
     |      PrjParameter: 投影坐标系统对象的投影参数
     |  
     |  projection
     |      Projection: 投影坐标系统的投影方式。投影方式如等角圆锥投影、等距方位投影等等。
     |  
     |  type
     |      PrjCoordSysType: 投影坐标系类型
     |  
     |  ----------------------------------------------------------------------
     |  Data descriptors inherited from iobjectspy.\_jsuperpy.data.\_jvm.JVMBase:
     |  
     |  \_\_dict\_\_
     |      dictionary for instance variables (if defined)
     |  
     |  \_\_weakref\_\_
     |      list of weak references to the object (if defined)
    
    class PrjParameter(iobjectspy.\_jsuperpy.data.\_jvm.JVMBase)
     |  地图投影参数类。 地图投影的参数,比如中央经线、原点纬度、双标准纬线的第一和第二条纬线等
     |  
     |  Method resolution order:
     |      PrjParameter
     |      iobjectspy.\_jsuperpy.data.\_jvm.JVMBase
     |      builtins.object
     |  
     |  Methods defined here:
     |  
     |  \_\_getstate\_\_(self)
     |  
     |  \_\_init\_\_(self)
     |      Initialize self.  See help(type(self)) for accurate signature.
     |  
     |  \_\_setstate\_\_(self, state)
     |  
     |  clone(self)
     |      复制对象
     |      
     |      :rtype: PrjParameter
     |  
     |  from\_xml(self, xml)
     |      根据传入的 XML 字符串构建 PrjParameter 对象
     |      
     |      :param str xml:
     |      :return:  如果构建成功返回 True,否则返回 False
     |      :rtype: bool
     |  
     |  set\_azimuth(self, value)
     |      设置方位角。主要用于斜轴投影。单位:度
     |      
     |      :param float value: 方位角
     |      :return: self
     |      :rtype: PrjParameter
     |  
     |  set\_central\_meridian(self, value)
     |      设置中央经线角度值。单位:度。 取值范围为-180度至180度。
     |      
     |      :param float value: 中央经线角度值。单位:度
     |      :return: self
     |      :rtype: PrjParameter
     |  
     |  set\_central\_parallel(self, value)
     |      设置坐标原点对应纬度值。单位:度。 取值范围为-90度至90度,在圆锥投影中通常就是投影区域最南端的纬度值。
     |      
     |      :param float value:  坐标原点对应纬度值
     |      :return: self
     |      :rtype: PrjParameter
     |  
     |  set\_false\_easting(self, value)
     |      设置坐标水平偏移量。单位:米。 此方法的参数值是为了避免系统坐标出现负值而加上的一个偏移量。通常用于高斯--克吕格、UTM 和墨卡托投影中。一般的值为500000米。
     |      
     |      :param float value: 坐标水平偏移量。单位:米。
     |      :return: self
     |      :rtype: PrjParameter
     |  
     |  set\_false\_northing(self, value)
     |      设置坐标垂直偏移量。单位:米。此方法的参数值是为了避免系统坐标出现负值而加上的一个偏移量。通常用于高斯--克吕格、UTM 和墨卡托投影中。一般的值为1000000米。
     |      
     |      :param float value: 坐标垂直偏移量。单位:米
     |      :return: self
     |      :rtype: PrjParameter
     |  
     |  set\_first\_point\_longitude(self, value)
     |      设置第一个点的经度。用于方位投影或斜投影。单位:度
     |      
     |      :param float value: 第一个点的经度。单位:度
     |      :return: self
     |      :rtype: PrjParameter
     |  
     |  set\_rectified\_angle(self, value)
     |      设置改良斜正射投影(ProjectionType.RectifiedSkewedOrthomorphic)参数中的纠正角,单位为弧度。
     |      
     |      :param float value:  改良斜正射投影(ProjectionType.RectifiedSkewedOrthomorphic)参数中的纠正角,单位为弧度
     |      :return: self
     |      :rtype: PrjParameter
     |  
     |  set\_scale\_factor(self, value)
     |      设置投影转换的比例因子。 用于减少投影变换的误差。墨卡托、高斯--克吕格和 UTM 投影的值一般为0.9996。
     |      
     |      :param float value: 投影转换的比例因
     |      :return: self
     |      :rtype: PrjParameter
     |  
     |  set\_second\_point\_longitude(self, value)
     |      设置第二个点的经度。用于方位投影或斜投影。单位:度。
     |      
     |      :param float value: 第二个点的经度。单位:度
     |      :return: self
     |      :rtype: PrjParameter
     |  
     |  set\_standard\_parallel1(self, value)
     |      设置第一标准纬线的纬度值。单位:度。主要应用于圆锥投影中。如果是单标准纬线,则第一标准纬线与第二标准纬线的纬度值相同。
     |      
     |      :param float value:  第一标准纬线的纬度值
     |      :return: self
     |      :rtype: PrjParameter
     |  
     |  set\_standard\_parallel2(self, value)
     |      设置第二标准纬线的纬度值。单位:度。 主要应用于圆锥投影中。如果是单标准纬线,则第一标准纬线与第二标准纬线的纬度值相同;如果是双标准纬线,则
     |      其值不能与第一标准纬线的值相同。
     |      
     |      :param float value: 第二标准纬线的纬度值。单位:度。
     |      :return: self
     |      :rtype: PrjParameter
     |  
     |  to\_xml(self)
     |      返回 PrjParameter 对象的 XML 字符串表示
     |      
     |      :rtype: str
     |  
     |  ----------------------------------------------------------------------
     |  Data descriptors defined here:
     |  
     |  azimuth
     |      float: 方位角
     |  
     |  central\_meridian
     |      float: 中央经线角度值。单位:度。 取值范围为-180度至180度
     |  
     |  central\_parallel
     |      float: 返回坐标原点对应纬度值。单位:度。 取值范围为-90度至90度,在圆锥投影中通常就是投影区域最南端的纬度值。
     |  
     |  false\_easting
     |      float: 坐标水平偏移量。单位:米
     |  
     |  false\_northing
     |      float: 坐标垂直偏移量
     |  
     |  first\_point\_longitude
     |      float: 返回第一个点的经度。用于方位投影或斜投影。单位:度
     |  
     |  rectified\_angle
     |      float: 返回改良斜正射投影(ProjectionType.RectifiedSkewedOrthomorphic)参数中的纠正角,单位为弧度
     |  
     |  scale\_factor
     |      float: 返回投影转换的比例因子。 用于减少投影变换的误差。墨卡托、高斯--克吕格和 UTM 投影的值一般为0.9996
     |  
     |  second\_point\_longitude
     |      float: 返回第二个点的经度。用于方位投影或斜投影。单位:度
     |  
     |  standard\_parallel1
     |      float: 返回第一标准纬线的纬度值。单位:度。主要应用于圆锥投影中。如果是单标准纬线,则第一标准纬线与第二标准纬线的纬度值相同。
     |  
     |  standard\_parallel2
     |      float: 返回第二标准纬线的纬度值。单位:度。主要应用于圆锥投影中。如果是单标准纬线,则第一标准纬线与第二标准纬线的纬度值相同;如果是双
     |      标准纬线,则其值不能与第一标准纬线的值相同。
     |  
     |  ----------------------------------------------------------------------
     |  Data descriptors inherited from iobjectspy.\_jsuperpy.data.\_jvm.JVMBase:
     |  
     |  \_\_dict\_\_
     |      dictionary for instance variables (if defined)
     |  
     |  \_\_weakref\_\_
     |      list of weak references to the object (if defined)
    
    class Projection(iobjectspy.\_jsuperpy.data.\_jvm.JVMBase)
     |  投影坐标系地图投影类。 地图投影就是将球面坐标转化为平面坐标的过程。
     |  
     |  一般来说,地图投影按变形性质可以分为等角投影、等距投影和等积投影,适于不同的用途,如果是航海图,等角投影是很常用。还有一种是各类变形介于这几种之
     |  间的任意投影,一般用作参考用途和教学地图。地图投影也可以按照构成方法分成两大类,分别为几何投影和非几何投影。几何投影是把椭球面上的经纬线网投影到
     |  几何面上,然后将几何面展为平面而得的,包括方位投影、圆柱投影和圆锥投影;非几何投影不借助几何面,根据某些条件有数学解析法确定球面与平面之间点与点
     |  的函数关系,包括伪方位投影、伪圆柱投影、伪圆锥投影和多圆锥投影。有关投影方式类型的详细信息请参考 :py:class:`ProjectionType`
     |  
     |  Method resolution order:
     |      Projection
     |      iobjectspy.\_jsuperpy.data.\_jvm.JVMBase
     |      builtins.object
     |  
     |  Methods defined here:
     |  
     |  \_\_getstate\_\_(self)
     |  
     |  \_\_init\_\_(self, projection\_type=None)
     |      :param projection\_type:
     |      :type projection\_type: ProjectionType or str
     |  
     |  \_\_setstate\_\_(self, state)
     |  
     |  clone(self)
     |      复制对象
     |      
     |      :rtype: Projection
     |  
     |  from\_xml(self, xml)
     |      根据 XML 字符串构建投影坐标方式对象,成功返回 True。
     |      
     |      :param str xml:  指定的 XML 字符串
     |      :rtype: bool
     |  
     |  set\_name(self, name)
     |      为您的自定义投影设置的名称
     |      
     |      :param str name: 自定义投影的名称
     |      :return: self
     |      :rtype: Projection
     |  
     |  set\_type(self, projection\_type)
     |      设置投影坐标系统的投影方式的类型。
     |      
     |      :param projection\_type:  投影坐标系统的投影方式的类型
     |      :type projection\_type:  ProjectionType or str
     |      :return: self
     |      :rtype: Projection
     |  
     |  to\_xml(self)
     |      返回投影方式对象的 XML 字符串表示。
     |      
     |      :rtype: str
     |  
     |  ----------------------------------------------------------------------
     |  Data descriptors defined here:
     |  
     |  name
     |      str: 投影方式对象的名称
     |  
     |  type
     |      ProjectionType: 投影坐标系统的投影方式的类型
     |  
     |  ----------------------------------------------------------------------
     |  Data descriptors inherited from iobjectspy.\_jsuperpy.data.\_jvm.JVMBase:
     |  
     |  \_\_dict\_\_
     |      dictionary for instance variables (if defined)
     |  
     |  \_\_weakref\_\_
     |      list of weak references to the object (if defined)
    
    class QueryParameter(builtins.object)
     |  查询参数类。 用于描述一个条件查询的限制条件,如所包含的 SQL 语句,游标方式,空间数据的位置关系条件设定等。条件查询,是查询满足一定条件的所
     |  有要素的记录,其查询得到的结果是记录集。查询参数类是用来设置条件查询的查询条件从而得到记录集。条件查询包括两种最主要的查询方式,一种为 SQL
     |  查询,又称属性查询,即通过构建包含属性字段、运算符号和数值的 SQL 条件语句来选择记录,从而得到记录集;另一种为空间查询,即根据要素间地理或空间
     |  的关系来查询记录来得到记录集。
     |  
     |  QueryParameter 包含以下参数:
     |  
     |  - attribute\_filter : str
     |      查询所构建的 SQL 条件语句,即 SQL WHERE clause 语句。SQL 查询又称为属性查询,是通过一个或多个 SQL 条件语句来查询记录。
     |      SQL 语句是包含属性字段、运算符号和数值的条件语句。例如,你希望查询一个商业区内去年的年销售额超过30万的服装店,则构建的 SQL 查询语句为::
     |  
     |      >>> attribute\_filter = 'Sales > 30,0000 AND SellingType = ‘Garment’'
     |  
     |      对于不同引擎的数据源,不同函数的适用情况及函数用法有所不同,对于数据库型数据源(Oracle Plus、SQL Server Plus、PostgreSQL 和 DB2 数据源),函数的用法请参见数据库相关文档。
     |  
     |  - cursor\_type : CursorType
     |      查询所采用的游标类型。SuperMap 支持两种类型的游标,分别为动态游标和静态游标。使用动态游标查询时,记录集会动态的刷新,耗费很多的资源,
     |      而当使用静态游标时,查询的为记录集的静态副本,效率较高。推荐在查询时使用静态游标,使用静态游标获得的记录集是不可编辑的。
     |      详细信息请参见 CursorType 类型。
     |      默认使用 DYNAMIC 类型。
     |  
     |  - has\_geometry : bool
     |      查询结果是否包含几何对象字段。 若查询时不取空间数据,即只查询属性信息,则在返回的 Recordset 中,凡是对记录集的空间对象进行操作的方法,
     |      都将无效,例如,调用 :py:meth:`Recordset.get\_geometry` 将返回空。
     |  
     |  - result\_fields : list of str
     |      设置查询结果字段集合。对于查询结果的记录集中,可以设置其中所包含的字段,如果为空,则查询所有字段。
     |  
     |  - order\_by : list of str
     |      SQL 查询排序的字段。 对于 SQL 查询得到的记录集中的各记录,可以根据指定的字段进行排序,并可以指定为升序排列或是降序排列,其中 asc 表示升序,desc 表示降序。用于排序的字段必须为数值型。例如按 SmID 降序排序,可以设置为::
     |  
     |      >>> query\_paramater.set\_order\_by(['SmID desc'])
     |  
     |  - group\_by : list of str
     |      SQL 查询分组条件的字段。对于 SQL 查询得到的记录集中的各字段,可以根据指定的字段进行分组,指定的字段值相同的记录将被放置在一起。
     |      注意:
     |  
     |          - 空间查询不支持 group\_by ,否则可能导致空间查询的结果不正确
     |          - 只有 cursor\_type 为 STATIC 时, group\_by 才有效
     |  
     |  - spatial\_query\_mode : SpatialQueryMode
     |      空间查询模式
     |  
     |  - spatial\_query\_object : DatasetVector or Recordset or Geometry or Rectangle or Point2D
     |      空间查询的搜索对象。
     |      若搜索对象是数据集或是记录集类型,则必须同被搜索图层对应的数据集的地理坐标系一致。
     |      当搜索数据集/记录集中存在对象重叠的情况时,空间查询的结果可能不正确,建议采用遍历搜索数据集/记录集,逐个使用单对象查询的方式进行空间查询。
     |  
     |  - time\_conditions : list of TimeCondition
     |      时空模型查询条件。具体查看 :py:class:`TimeCondition` 说明。
     |  
     |  - link\_items : list of LinkItem
     |      关联查询的信息。当被查询的矢量数据集有相关联的外部表时,查询得到的结果中会包含相关联的外部表中满足条件的记录。具体查看 :py:class:`LinkItem` 说明
     |  
     |  -  join\_items: list of JoinItem
     |      连接查询的信息。当被查询的矢量数据集有相连接的外部表时,查询得到的结果中会包含相连接的外部表中满足条件的记录。具体查看 :py:class:`JoinItem` 说明
     |  
     |  
     |  示例::
     |      \# 进行 SQL 查询
     |  
     |      >>> parameter = QueryParameter('SmID < 100', 'STATIC', False)
     |      >>> ds = Datasource.open('E:/data.udb')
     |      >>> dt = ds['point']
     |      >>> rd = dt.query(parameter)
     |      >>> print(rd.get\_record\_count())
     |      99
     |      >>> rd.close()
     |  
     |  
     |      \# 进行空间查询
     |  
     |      >>> geo = dt.get\_geometries('SmID = 1')[0]
     |      >>> query\_geo = geo.create\_buffer(10, dt.prj\_coordsys, 'meter')
     |      >>> parameter.set\_spatial\_query\_mode('contain').set\_spatial\_query\_object(query\_geo)
     |      >>> rd2 = dt.query(parameter)
     |      >>> print(rd2.get\_record\_count())
     |      10
     |      >>> rd2.close()
     |  
     |  Methods defined here:
     |  
     |  \_\_init\_\_(self, attr\_filter=None, cursor\_type=CursorType.DYNAMIC, has\_geometry=True, result\_fields=None, order\_by=None, group\_by=None)
     |      Initialize self.  See help(type(self)) for accurate signature.
     |  
     |  from\_dict(self, values)
     |      从 dict 中读取查询参数
     |      
     |      :param dict values: 包含查询参数的 dict 对象。具体查看 :py:meth:`to\_dict`
     |      :return: self
     |      :rtype: QueryParameter
     |  
     |  set\_attribute\_filter(self, value)
     |      设置属性查询条件
     |      
     |      :param str value: 属性查询条件
     |      :return: self
     |      :rtype: QueryParameter
     |  
     |  set\_cursor\_type(self, value)
     |      设置查询所采用的游标类型。默认为 DYNAMIC
     |      
     |      :param value: 游标类型
     |      :type value: CursorType or str
     |      :return: self
     |      :rtype: QueryParameter
     |  
     |  set\_group\_by(self, value)
     |      设置 SQL 查询分组条件的字段。
     |      
     |      :param list[str] value:  SQL 查询分组条件的字段
     |      :return: self
     |      :rtype: QueryParameter
     |  
     |  set\_has\_geometry(self, value)
     |      设置是否查询几何对象。如果设置为 False,将不会返回几何对象,默认为 True
     |      
     |      :param bool value: 查询是否包含几何对象。
     |      :return: self
     |      :rtype: QueryParameter
     |  
     |  set\_join\_items(self, value)
     |      设置连接查询的查询条件
     |      
     |      :param list[JoinItem] value:  接查询的查询条件
     |      :return: self
     |      :rtype: QueryParameter
     |  
     |  set\_link\_items(self, value)
     |      设置关联查询的查询条件
     |      
     |      :param list[LinkItem] value:  关联查询的查询条件
     |      :return: self
     |      :rtype: QueryParameter
     |  
     |  set\_order\_by(self, value)
     |      设置 SQL 查询排序的字段
     |      
     |      :param list[str] value: SQL查询排序的字段
     |      :return: self
     |      :rtype: QueryParameter
     |  
     |  set\_result\_fields(self, value)
     |      设置查询结果字段
     |      
     |      :param list[str] value:  查询结果字段
     |      :return: self
     |      :rtype: QueryParameter
     |  
     |  set\_spatial\_query\_mode(self, value)
     |      设置空间查询模式,具体查看 :py:class:`SpatialQueryMode` 说明
     |      
     |      :param  value: 空间查询的查询模式。
     |      :type value: SpatialQueryMode or str
     |      :return: self
     |      :rtype: QueryParameter
     |  
     |  set\_spatial\_query\_object(self, value)
     |      设置空间查询的搜索对象
     |      
     |      :param value: 空间查询的搜索对象
     |      :type value: DatasetVector or Recordset or Geometry or Rectangle or Point2D
     |      :return: self
     |      :rtype: QueryParameter
     |  
     |  set\_time\_conditions(self, value)
     |      设置时间字段进行时空查询的查询条件
     |      
     |      :param list[TimeCondition] value: 时空查询的查询条件
     |      :return: self
     |      :rtype: QueryParameter
     |  
     |  to\_dict(self)
     |      将数据集查询参数信息输出到 dict 中
     |      
     |      :rtype: dict
     |  
     |  to\_json(self)
     |      将查询参数输出为 json 字符串
     |      
     |      :rtype: str
     |  
     |  ----------------------------------------------------------------------
     |  Static methods defined here:
     |  
     |  from\_json(value)
     |      从 json 字符串中构造数据集查询参数对象
     |      
     |      :param str value:  json 字符串
     |      :rtype: QueryParameter
     |  
     |  make\_from\_dict(values)
     |      从 dict 中构造数据集查询参数对象
     |      
     |      :param dict values: 包含查询参数的 dict 对象。具体查看 :py:meth:`to\_dict`
     |      :rtype: QueryParameter
     |  
     |  ----------------------------------------------------------------------
     |  Data descriptors defined here:
     |  
     |  \_\_dict\_\_
     |      dictionary for instance variables (if defined)
     |  
     |  \_\_weakref\_\_
     |      list of weak references to the object (if defined)
     |  
     |  attribute\_filter
     |      str : 查询所构建的 SQL 条件语句,即 SQL WHERE clause 语句。
     |  
     |  cursor\_type
     |      CursorType: 查询所采用的游标类型
     |  
     |  group\_by
     |      list[str]: SQL 查询分组条件的字段
     |  
     |  has\_geometry
     |      bool: 查询结果是否包含几何对象字段
     |  
     |  join\_items
     |      list[JoinItem]: 连接查询的信息。当被查询的矢量数据集有相连接的外部表时,查询得到的结果中会包含相连接的外部表中满足条件的记录。具体查看 :py:class:`JoinItem` 说明
     |  
     |  link\_items
     |      list[LinkItem]: 关联查询的信息。当被查询的矢量数据集有相关联的外部表时,查询得到的结果中会包含相关联的外部表中满足条件的记录。具体查看 :py:class:`LinkItem` 说明
     |  
     |  order\_by
     |      list[str]:  SQL 查询排序的字段。
     |  
     |  result\_fields
     |      list[str]: 查询结果字段集合。对于查询结果的记录集中,可以设置其中所包含的字段,如果为空,则查询所有字段。
     |  
     |  spatial\_query\_mode
     |      SpatialQueryMode: 空间查询模式。
     |  
     |  spatial\_query\_object
     |      DatasetVector or Recordset or Geometry or Rectangle or Point2D: 空间查询的搜索对象。
     |  
     |  time\_conditions
     |      list[TimeCondition]:  时空模型查询条件。具体查看 :py:class:`TimeCondition` 说明。
    
    class Recordset(iobjectspy.\_jsuperpy.data.\_jvm.JVMBase)
     |  记录集类。 通过此类,可以实现对矢量数据集中的数据进行操作。 数据源有文件型和数据库型,数据库型数据中空间几何信息和属性信息一体化存储,一个矢量数
     |  据集对应一个 DBMS 表,其几何形状以及属性信息都一体化存储其中,表中的几何字段存储要素的空间几何信息。对于矢量数据集中的纯属性数据集,其中没有几
     |  何字段,记录集为 DBMS 表的一个子集;而在文件型数据中空间几何信息和属性信息是分别存储的,记录集的应用可能比较让人费解,实际上,操作时是屏蔽掉文
     |  件型和数据库型数据的区别,将数据都看成是一个空间信息和属性信息一体化存储的表,而记录集是从其中取出的用来操作的一个子集。记录集中的一条记录,即一
     |  行,对应着一个要素,包含该要素的空间几何信息和属性信息。记录集中的一列对应一个字段的信息。
     |  
     |  记录集可以直接从矢量数据集中获得一个记录集,有两种方法:用户可以通过 :py:meth:`DatasetVector.get\_recordset` 方法直接从矢量数据集中返回
     |  记录集,也可以通过查询语句返回记录集,所不同的是前者得到的记录集包含该类集合的全部空间几何信息和属性信息,而后者得到的是经过查询语句条件过滤的记录集。
     |  
     |  以下代码演示从记录集中读取数据以及批量写入数据到新的记录集中::
     |  
     |  >>> dt = Datasource.open('E:/data.udb')['point']
     |  >>> rd = dt.query\_with\_filter('SmID < 100 or SmID > 1000', 'STATIC')
     |  >>> all\_points = []
     |  >>> while rd.has\_next():
     |  >>>     geo = rd.get\_geometry()
     |  >>>     all\_points.append(geo.point)
     |  >>>     rd.move\_next()
     |  >>> rd.close()
     |  >>>
     |  >>> new\_dt = dt.datasource.create\_vector\_dataset('new\_point', 'Point', adjust\_name=True)
     |  >>> new\_dt.create\_field(FieldInfo('object\_time', FieldType.DATETIME))
     |  >>> new\_rd = new\_dt.get\_recordset(True)
     |  >>> new\_rd.batch\_edit()
     |  >>> for point in all\_points:
     |  >>>     new\_rd.add(point, \{'object\_time': datetime.datetime.now()\} )
     |  >>> new\_rd.batch\_update()
     |  >>> print(new\_rd.get\_record\_count())
     |  >>> new\_rd.close()
     |  
     |  Method resolution order:
     |      Recordset
     |      iobjectspy.\_jsuperpy.data.\_jvm.JVMBase
     |      builtins.object
     |  
     |  Methods defined here:
     |  
     |  \_\_del\_\_(self)
     |  
     |  \_\_enter\_\_(self)
     |  
     |  \_\_exit\_\_(self, exc\_type, exc\_val, exc\_tb)
     |  
     |  \_\_init\_\_(self)
     |      Initialize self.  See help(type(self)) for accurate signature.
     |  
     |  \_\_iter\_\_(self)
     |  
     |  \_\_next\_\_(self)
     |  
     |  add(self, data, values=None)
     |      向记录集中新增一条记录。记录集必须开启编辑模式,具体查看 :py:meth:`edit` 和 :py:meth:`batch\_edit`
     |      
     |      :param data: 被写入的空间对象,如果记录集的数据集为属性表,则传入 None。如果 data 不为空,几何对象的类型必须与数据集的类型想匹配才能写入成功。
     |                   例如:
     |      
     |                   - :py:class:`Point2D` 和 :py:class:`GeoPoint` 支持写入到点数据集和 CAD 数据集
     |                   - :py:class:`GeoLine` 支持写入到线数据集和CAD数据集
     |                   - :py:class:`Rectangle` 和 :py:class:`GeoRegion` 支持写入到面数据集和 CAD 数据集
     |                   - :py:class:`GeoText` 支持写入到文本数据集和 CAD 数据集
     |      
     |      :type data: Point2D or Rectangle or Geometry or Feature
     |      :param dict values: 要写入的属性字段值。必须是 dict,dict 的键值为字段名称,dict 的值为字段值。如果 data 为 Feature,此参数无效,因为 Feature 已经包含有属性字段值。
     |      :return: 写入成功返回 True,否则返回 False
     |      :rtype: bool
     |  
     |  batch\_edit(self)
     |      批量更新操作开始。批量更新操作完成后,需要使用 :py:meth:`batch\_update` 来提交修改的记录。可以使用 :py:meth:`set\_batch\_record\_max` 修改
     |      批量更新操作结果提交的最大记录数,具体查看 :py:meth:`set\_batch\_record\_max`
     |  
     |  batch\_update(self)
     |      批量更新操作的统一提交。调用该方法后,之前进行的批量更新操作才会生效,同时更新状态将变为单条更新,如果需要之后的操作批量进行,还需再次调用 :py:meth:`batch\_edit` 方法。
     |  
     |  close(self)
     |      释放记录集,记录集完成操作不再使用后必须释放记录集
     |  
     |  delete(self)
     |      用于删除数据集中的当前记录,成功则返回 true。
     |      
     |      :rtype: bool
     |  
     |  delete\_all(self)
     |      物理性删除指定记录集中的所有记录,即把记录从计算机的物理存储介质上删除,无法恢复。
     |      
     |      :rtype: bool
     |  
     |  dispose(self)
     |      释放记录集,记录集完成操作不再使用后必须释放记录集, 与 close 功能相同
     |  
     |  edit(self)
     |      锁定并编辑记录集的当前记录,成功则返回 True。用该方法编辑后,一定要用 :py:meth:`update` 方法更新记录集,而且在 :py:meth:`update` 之
     |      前不能移动当前记录的位置,否则编辑失败,记录集也可能被损坏。
     |      
     |      :rtype: bool
     |  
     |  get\_batch\_record\_max(self)
     |      int: 返回批量更新操作结果自动提交的最大记录数
     |  
     |  get\_feature(self)
     |      获取当前记录的要素对象,如果获取失败返回 None
     |      
     |      :rtype: Feature
     |  
     |  get\_features(self)
     |      获取记录集的所有要素对象。调用该方法后,记录集的位置会移动的最开始的位置。
     |      
     |      :rtype: list[Feature]
     |  
     |  get\_field\_count(self)
     |      获取字段数目
     |      
     |      :rtype: int
     |  
     |  get\_field\_info(self, value)
     |      根据字段名称或序号获取字段信息
     |      
     |      :param value: 字段名称或序号
     |      :type value: str or int
     |      :return: 字段信息
     |      :rtype: FieldInfo
     |  
     |  get\_geometries(self)
     |      获取记录集的所有几何对象。调用该方法后,记录集的位置会移动的最开始的位置。
     |      
     |      :rtype: list[Geometry]
     |  
     |  get\_geometry(self)
     |      获取当前记录的几何对象,如果记录集没有几何对象或获取失败,返回 None
     |      
     |      :rtype: Geometry
     |  
     |  get\_id(self)
     |      返回数据集的属性表中当前记录对应的几何对象的 ID 号(即 SmID 字段的值)。
     |      
     |      :rtype: int
     |  
     |  get\_query\_parameter(self)
     |      获取当前记录集对应的查询参数
     |      
     |      :rtype: QueryParameter
     |  
     |  get\_record\_count(self)
     |      返回记录集中记录数目
     |      
     |      :rtype: int
     |  
     |  get\_value(self, item)
     |      获取当前记录中指定的属性字段的字段值
     |      
     |      :param item: 字段名称或序号
     |      :type item: str or int
     |      :rtype: int or float or str or datetime.datetime or bytes or bytearray
     |  
     |  get\_values(self, exclude\_system=True, is\_dict=False)
     |      获取当前记录的属性字段值。
     |      
     |      :param bool exclude\_system: 是否包含系统字段。所有 "Sm" 开头的字段都是系统字段。默认为 True
     |      :param bool is\_dict: 是否以 dict 形式返回,如果返回 dict,则 dict  的 key 为字段名称, value 为属性字段值。否则以 list 形式返回字段值。默认为 False
     |      :return: 属性字段值
     |      :rtype: dict or list
     |  
     |  has\_next(self)
     |      记录集是否还有下一条记录可以读取,如果有返回 True,否则返回 False
     |      
     |      :rtype: bool
     |  
     |  index\_of\_field(self, name)
     |      获取指定字段名称序号
     |      
     |      :param str name: 字段名称
     |      :return: 如果字段存在返回字段的序号,否则返回 -1
     |      :rtype: int
     |  
     |  is\_bof(self)
     |      判断当前记录的位置是否在记录集中第一条记录的前面(当然第一条记录的前面是没有数据的),如果是返回 True;否则返回 False。
     |      
     |      :rtype: bool
     |  
     |  is\_close(self)
     |      判断记录集是否已经被关闭。被关闭返回 True,否则返回 False。
     |      
     |      :rtype: bool
     |  
     |  is\_empty(self)
     |      判断记录集中是否有记录。True 表示该记录集中无数据
     |      
     |      :rtype: bool
     |  
     |  is\_eof(self)
     |      记录集是否到达末尾,如果到达末尾返回 True,否则返回 False
     |      
     |      :rtype: bool
     |  
     |  is\_readonly(self)
     |      判断记录集是否是只读的,只读返回 True,否则返回 False
     |      
     |      :rtype: bool
     |  
     |  move(self, count)
     |      将当前记录位置移动 count 行,将该位置的记录设置为当前记录。成功返回 True。count 小于0表示向前移,大于0表示向后移动,等于0时不移动。如
     |      果移动的行数太多,超出了 Recordset 的范围,将会返回 False,当前记录不移动。
     |      
     |      :param int count: 移动的记录数
     |      :rtype: bool
     |  
     |  move\_first(self)
     |      用于移动当前记录位置到第一条记录,使第一条记录成为当前记录。成功则返回 True。
     |      
     |      :rtype: bool
     |  
     |  move\_last(self)
     |      用于移动当前记录位置到最后一条记录,使最后一条记录成为当前记录。成功则返回 True
     |      
     |      :rtype: bool
     |  
     |  move\_next(self)
     |      动当前记录位置到下一条记录,使该记录成为当前记录。成功则返回 True,否则返回 False
     |      
     |      :rtype: bool
     |  
     |  move\_prev(self)
     |      移动当前记录位置到上一条记录,使该记录成为当前记录。成功则返回 True。
     |      
     |      :rtype: bool
     |  
     |  move\_to(self, position)
     |      用于移动当前记录位置到指定的位置,将该指定位置的记录作为当前记录。成功则返回 True。
     |      
     |      :param int position: 移动到的位置,即第几条记录
     |      :rtype: bool
     |  
     |  refresh(self)
     |      刷新当前记录集,用来反映数据集中的变化。如果成功返回 True,否则返回 False。此方法与 :py:meth:`update` 的区别在于 update 方法是提交
     |      修改结果,而 refresh 方法是动态刷新记录集,在多用户并发操作时,为了动态显示数据集中的变化,经常用到 refresh 方法。
     |      
     |      :rtype: bool
     |  
     |  seek\_id(self, value)
     |      在记录中搜索指定 ID 号的记录,并定位该记录为当前记录。成功则返回 true,否则返回 false
     |      
     |      :param int value: 要搜索的 ID 号
     |      :rtype: bool
     |  
     |  set(self, data, values=None)
     |      修改当前记录。记录集必须开启编辑模式,具体查看 :py:meth:`edit` 和 :py:meth:`batch\_edit`
     |      
     |      :param data: 被写入的空间对象。如果记录集的数据集为属性表,则传入 None。如果 data 不为空,几何对象的类型必须与数据集的类型想匹配才能写入成功。
     |                   例如:
     |      
     |                   - :py:class:`Point2D` 和 :py:class:`GeoPoint` 支持写入到点数据集和 CAD 数据集
     |                   - :py:class:`GeoLine` 支持写入到线数据集和CAD数据集
     |                   - :py:class:`Rectangle` 和 :py:class:`GeoRegion` 支持写入到面数据集和 CAD 数据集
     |                   - :py:class:`GeoText` 支持写入到文本数据集和 CAD 数据集
     |      
     |      :type data:  Point2D or Rectangle or Geometry or Feature
     |      :param values: 要写入的属性字段值。必须是 dict,dict 的键值为字段名称,dict 的值为字段值。如果 data 为 Feature,此参数无效,因为
     |                     Feature 已经包含有属性字段值。如果 data 为空,将只写入属性字段值。
     |      :return: 写入成功返回 True,否则返回 False
     |      :rtype: bool
     |  
     |  set\_batch\_record\_max(self, count)
     |      设置批量更新操作结果提交的最大记录数,当所有需要更新的记录批量更新完成后,在提交更新结果时,如果更新的记录数超过了这个最大记录数时,系统将分
     |      批提交更新的结果,即每次提交最大记录数目个记录,直到所有的更新记录都提交完毕。例如,如果设定提交的最大记录数为1000,而需要更新的记录数为3800,
     |      那么批量更新记录后,在提交结果时,系统将分四次提交更新结果,即第一次提交1000条记录,第二次1000条,第三次1000条,第四次800条。
     |      
     |      :param int count: 批量更新操作结果提交的最大记录数。
     |  
     |  set\_value(self, item, value)
     |      写入字段值到指定的字段中。记录集必须开启编辑模式,具体查看 :py:meth:`edit` 和 :py:meth:`batch\_edit`
     |      
     |      :param item: 字段名称或序号,不能为系统字段。
     |      :type item: str or int
     |      :param value: 待写入的字段值。字段类型与值类型对应关系为:
     |      
     |                    - BOOLEAN: bool
     |                    - BYTE: int
     |                    - INT16: int
     |                    - INT32: int
     |                    - INT64: int
     |                    - SINGLE: float
     |                    - DOUBLE: float
     |                    - DATETIME: datetime.datetime 或 int(时间戳,单位为秒)或满足 ”\%Y-\%m-\%d \%H:\%M:\%S“ 格式的 字符串
     |                    - LONGBINARY: bytearray or bytes
     |                    - TEXT: str
     |                    - CHAR: str
     |                    - WTEXT: str
     |                    - JSONB: str
     |      
     |      :type value: bool or int or float or datetime.datetime or bytes or bytearray or str
     |      :return: 成功返回 True,否则返回 False
     |      :rtype: bool
     |  
     |  set\_values(self, values)
     |      设置字段值。记录集必须开启编辑模式,具体查看 :py:meth:`edit` 和 :py:meth:`batch\_edit`
     |      
     |      :param dict values: 要写入的属性字段值。必须是 dict,dict 的键值为字段名称,dict 的值为字段值
     |      :return: 返回成功写入的字段数目
     |      :rtype: int
     |  
     |  statistic(self, item, stat\_mode)
     |      通过字段名称或序号,对指定字段进行诸如最大值、最小值、平均值,总和,标准差和方差等方式的统计。
     |      
     |      :param item: 字段名称或序号
     |      :type item: str or int
     |      :param stat\_mode: 统计方式
     |      :type stat\_mode:  StatisticMode or str
     |      :return: 统计结果。
     |      :rtype: float
     |  
     |  to\_json(self)
     |      将当前记录集的数据集和查询参数输出为 json 字符串。注意,使用 Recordset 的 to\_json 只保存数据集信息的查询参数,只适用于使用 DatasetVector 入口查询得到的结果记录集,包括
     |      :py:meth:`DatasetVector.get\_recordset` , :py:meth:`DatasetVector.query`, :py:meth:`DatasetVector.query\_with\_bounds`,
     |      :py:meth:`DatasetVector.query\_with\_distance`, :py:meth:`DatasetVector.query\_with\_filter` 和 :py:meth:`DatasetVector.query\_with\_ids`。
     |      如果是由其他功能内部查询得到的记录集,可能无法完全确保查询参数的是否与查询时输入的查询参数是否一致。
     |      
     |      :rtype: str
     |  
     |  update(self)
     |      用于提交对记录集的修改,包括添加、编辑记录、修改字段值的操作。使用 :py:meth:`edit`  对记录集做修改之后,都需要使用 update 来提交修改。每对一条记录做完修改就
     |      需要调用一次 update 来提交修改。
     |      
     |      :rtype: bool
     |  
     |  ----------------------------------------------------------------------
     |  Static methods defined here:
     |  
     |  from\_json(value)
     |      从 json 字符串中解析获取记录集
     |      
     |      :param str value: json 字符串
     |      :rtype: Recordset
     |  
     |  ----------------------------------------------------------------------
     |  Data descriptors defined here:
     |  
     |  bounds
     |      Rectangle: 返回记录集的属性数据表中所有记录对应的几何对象的外接矩形。
     |  
     |  dataset
     |      DatasetVector: 记录集所在的数据集
     |  
     |  datasource
     |      Datasource: 记录集所在的数据源
     |  
     |  field\_infos
     |      list[FieldInfo]: 数据集的所有字段信息
     |  
     |  ----------------------------------------------------------------------
     |  Data descriptors inherited from iobjectspy.\_jsuperpy.data.\_jvm.JVMBase:
     |  
     |  \_\_dict\_\_
     |      dictionary for instance variables (if defined)
     |  
     |  \_\_weakref\_\_
     |      list of weak references to the object (if defined)
    
    class Rectangle(builtins.object)
     |  矩形对象使用四个浮点数表示一个矩形的范围。其中 left 代表 x 方向的最小值,top 代表 y 方向的最大值,
     |  right 代表 x 方向的最大值,bottom 代表 y 方向的最小值。当使用矩形表示一个地理范围时,通常 left 表示经度的最小值,
     |  right 表示经度的最大值,top 表示纬度的最大值,bottom表示维度的最小值
     |  该类的对象通常用于确定范围,可用来表示几何对象的最小外接矩形、地图窗口的可视范围,数据集的范围等,另外在进行矩形选择,矩形查询等时也会用到此类的对象。
     |  
     |  Methods defined here:
     |  
     |  \_\_eq\_\_(self, other)
     |      判断当前矩形对象与指定矩形对象是否相同。只有当上下左右边界完全相同时才能判断为相同。
     |      
     |      :param Rectangle other:  待判断的矩形对象。
     |      :return:  如果当前对象与矩形对象相同,返回True,否则返回False
     |      :rtype: bool
     |  
     |  \_\_getitem\_\_(self, item)
     |      当矩形对象使用四个二维坐标点描述具体的坐标位置时,返回点坐标值。
     |      
     |      :param int item: 0 ,1,2,3值
     |      :return: 根据 item 的值,分别返回左上点,右上点,右下点,左下点
     |      :rtype: float
     |  
     |  \_\_getstate\_\_(self)
     |  
     |  \_\_init\_\_(self, left=None, top=None, right=None, bottom=None)
     |      Initialize self.  See help(type(self)) for accurate signature.
     |  
     |  \_\_len\_\_(self)
     |      :return: 当矩形对象使用四个二维坐标点描述具体的坐标位置时,返回点的数目。固定为4.
     |      :rtype: int
     |  
     |  \_\_repr\_\_(self)
     |      Return repr(self).
     |  
     |  \_\_setstate\_\_(self, state)
     |  
     |  \_\_str\_\_(self)
     |      Return str(self).
     |  
     |  clone(self)
     |      复制当前对象
     |      
     |      :rtype: Rectangle
     |  
     |  contains(self, item)
     |      判断一个点对象或矩形矩形是否在当前矩形对象内部
     |      
     |      :param Point2D item: 二维点对象(含有x和y两个属性)或矩形对象。矩形对象要求非空(矩形对象是否为空,可以参考 @Rectangle.is\_empty)
     |      :return: 待判断的对象在当前矩形内部返回 True,否则为 False
     |      :rtype: bool
     |      
     |      >>> rect = Rectangle(1.0, 20, 2.0, 3)
     |      >>> rect.contains(Point2D(1.1,10))
     |      True
     |      >>> rect.contains(Point2D(0,0))
     |      False
     |      >>> rect.contains(Rectangle(1.0,10,1.5,5))
     |      True
     |  
     |  from\_dict(self, value)
     |      从一个字典对象中读取矩形对象的边界值。读取成功后将会覆盖矩形对象现有的值。
     |      
     |      :param dict value: 字典对象,字典对象的keys 必须有 'left', 'top', 'right', 'bottom'
     |      :return: 返回当前对象,self
     |      :rtype: Rectangle
     |  
     |  has\_intersection(self, item)
     |      判断一个二维点、矩形对象或空间几何对象是否与当前矩形对象相交。待判断的对象只要与当前矩形对象有相交区域或者接触都会判定为相交。
     |      
     |      :param Point2D  item: 待判断的二维点对象、矩形对象和空间几何对象,空间几何对象支持点线面和文本对象。
     |      :return: 判断相交返回True,否则返回False
     |      :rtype: bool
     |      
     |      >>> rc = Rectangle(1,2,2,1)
     |      >>> rc.has\_intersection(Rectangle(0,1.5,1.5,0))
     |      True
     |      >>> rc.has\_intersection(GeoLine([Point2D(0,0),Point2D(3,3)]))
     |      True
     |  
     |  inflate(self, dx, dy)
     |      对当前矩形对象在垂直(y方向)和水平(x方向)进行缩放。缩放完成后将会改变当前对象上下或左右值,但中心点不变。
     |      
     |      :param float dx: 水平方向缩放量
     |      :param float dy: 垂直方向缩放量
     |      :return: self
     |      :rtype: Rectangle
     |      
     |      >>> rc = Rectangle(1,2,2,1)
     |      >>> rc.inflate(3,None)
     |      (-2.0, 2.0, 5.0, 1.0)
     |      >>> rc.left == -2
     |      True
     |      >>> rc.right == 5
     |      True
     |      >>> rc.top == 2
     |      True
     |      >>> rc.inflate(0, 2)
     |      (-2.0, 4.0, 5.0, -1.0)
     |      >>> rc.left == -2
     |      True
     |      >>> rc.top == 4
     |      True
     |  
     |  intersect(self, rc)
     |      指定矩形对象与当前对象求交集,并改变当前矩形对象。
     |      
     |      :param Rectangle rc: 用于进行求交操作的矩形
     |      :return: 当前对象,self
     |      :rtype: Rectangle
     |      
     |      >>> rc = Rectangle(1,2,2,1)
     |      >>> rc.intersect(Rectangle(0,1.5,1.5,0))
     |      (1.0, 1.5, 1.5, 1.0)
     |  
     |  is\_empty(self)
     |      判断矩形对象是否为空,当矩形的上下左右边界值有一个为 None 时,矩形为空。当矩形的上下左右边界值有
     |      一个为 -1.7976931348623157e+308 时矩形为空
     |      
     |      :return: 矩形为空,返回True,否则返回False
     |      :rtype: bool
     |  
     |  offset(self, dx, dy)
     |      将此矩形在 x 方向平移 dx,在 y 方向平移 dy,此方法将改变当前对象。
     |      
     |      :param float dx: 水平偏移该位置的量。
     |      :param float dy: 垂直偏移该位置的量。
     |      :return: self
     |      :rtype: Rectangle
     |      
     |      >>> rc = Rectangle(1,2,2,1)
     |      >>> rc.offset(2,3)
     |      (3.0, 5.0, 4.0, 4.0)
     |  
     |  set\_bottom(self, value)
     |      设置当前矩形对象的下边界值。如果上下边界值都有效,当下边界值大于上边界值,会将上下边界值互换
     |      
     |      :param float value:  下边界值
     |      :return: self
     |      :rtype: Rectangle
     |  
     |  set\_left(self, value)
     |      设置当前矩形对象的左边界值。如果左右边界值都有效,当左边界值大于右边界值,会将左右边界值互换
     |      
     |      :param float value:  左边界值
     |      :return: self
     |      :rtype: Rectangle
     |  
     |  set\_right(self, value)
     |      设置当前矩形对象的右边界值。如果左右边界值都有效,当左边界值大于右边界值,会将左右边界值互换
     |      
     |      :param float value:  右边界值
     |      :return: self
     |      :rtype: Rectangle
     |      
     |      >>> rc = Rectangle(left=10).set\_right(5.0)
     |      >>> rc.right, rc.left
     |      (10.0, 5.0)
     |  
     |  set\_top(self, value)
     |      设置当前矩形对象的上边界值。如果上下边界值都有效,当下边界值大于上边界值,会将上下边界值互换
     |      
     |      :param float value:  上边界值
     |      :return: self
     |      :rtype: Rectangle
     |      
     |      >>> rc = Rectangle(bottom=10).set\_top(5.0)
     |      >>> rc.top, rc.bottom
     |      (10.0, 5.0)
     |  
     |  to\_dict(self)
     |      将矩形对象返回为一个字典对象
     |      
     |      :rtype: dict
     |  
     |  to\_json(self)
     |      得到矩形对象的 json 字符串形式
     |      
     |      :rtype: str
     |      
     |      >>> Rectangle(1,2,2,1).to\_json()
     |      '\{"rectangle": [1.0, 2.0, 2.0, 1.0]\}'
     |  
     |  to\_region(self)
     |      用一个几何面对象表示矩形对象所表示的范围。返回的面对象的点坐标顺序为:第一个点表示左上点,第二个点表示右上点,第三个点表示右下点,
     |      第四个点表示左下点,第五个点与第一个点坐标相同。
     |      
     |      :return: 返回由矩形范围所表示的几何面对象
     |      :rtype: GeoRegion
     |      
     |      >>> rc = Rectangle(2.0, 20, 3.0, 10)
     |      >>> geoRegion = rc.to\_region()
     |      >>> print(geoRegion.area)
     |      10.0
     |  
     |  to\_tuple(self)
     |      得到一个元组对象,元组对象的元素分别为矩形的左、上、右、下
     |      
     |      :rtype: tuple
     |  
     |  union(self, rc)
     |      当前矩形对象合并指定的矩形对象,合并完成后,矩形的范围将会是合并前的矩形与指定的矩形对象的并集。
     |      
     |      :param Rectangle rc: 指定的用于合并的矩形对象
     |      :return: self
     |      :rtype: Rectangle
     |      
     |      >>> rc = Rectangle(1,2,2,1)
     |      >>> rc.union(Rectangle(0,0,1,-2))
     |      (0.0, 2.0, 2.0, -2.0)
     |  
     |  ----------------------------------------------------------------------
     |  Static methods defined here:
     |  
     |  from\_json(value)
     |      从 json 字符串中构造一个矩形对象。
     |      
     |      :param str value: json 字符串
     |      :return: 矩形对象
     |      :rtype: Rectangle
     |      
     |      >>> s = '\{"rectangle": [1.0, 2.0, 2.0, 1.0]\}'
     |      >>> Rectangle.from\_json(s)
     |      (1.0, 2.0, 2.0, 1.0)
     |  
     |  make(value)
     |      构造一个二维矩形对象
     |      
     |      :param value: 含有二维矩形对象左、上、右、下的信息
     |      :type value: Rectangle or list or str or dict
     |      :return:
     |      :rtype:
     |  
     |  make\_from\_dict(value)
     |      从一个字典对象中构造出矩形对象。
     |      
     |      :param value: 字典对象,字典对象的keys 必须有 'left', 'top', 'right', 'bottom'
     |      :rtype: Rectangle
     |  
     |  ----------------------------------------------------------------------
     |  Data descriptors defined here:
     |  
     |  \_\_dict\_\_
     |      dictionary for instance variables (if defined)
     |  
     |  \_\_weakref\_\_
     |      list of weak references to the object (if defined)
     |  
     |  bottom
     |      float: 返回当前矩形对象下边界的坐标值
     |  
     |  center
     |      Point2D: 返回当前矩形对象的中心点
     |  
     |  height
     |      float: 返回当前矩形对象的高度值
     |  
     |  left
     |      float: 返回当前矩形对象左边界的坐标值
     |  
     |  points
     |      tuple[Point2D]: 获取矩形四个顶点坐标值,返回一个4个二维点(Point2D)的元组。其中第一个点表示左上点,第二个点表示右上点,第三个点表示右下点,第四个点表示左下点。
     |      
     |      >>> rect = Rectangle(1.0, 20, 2.0, 3)
     |      >>> points = rect.points
     |      >>> len(points)
     |      4
     |      >>> points[0] == Point2D(1.0,20)
     |      True
     |      >>> points[2] == Point2D(2.0,3)
     |      True
     |  
     |  right
     |      float: 返回当前矩形对象右边界的坐标值
     |  
     |  top
     |      float: 返回当前矩形对象上边界的坐标值
     |  
     |  width
     |      float: 返回当前矩形对象的宽度值
     |  
     |  ----------------------------------------------------------------------
     |  Data and other attributes defined here:
     |  
     |  \_\_hash\_\_ = None
    
    class SpatialIndexInfo(builtins.object)
     |  空间索引信息类。该类提供了创建空间索引的所需信息,包括空间索引的类型、叶结点个数、图幅字段、图幅宽高和多级网格的大小等信息。
     |  
     |  Methods defined here:
     |  
     |  \_\_init\_\_(self, index\_type=None)
     |      构造数据集空间索引信息类。
     |      
     |      :param index\_type: 数据集空间索引类型
     |      :type index\_type:  SpatialIndexType or str
     |  
     |  from\_dict(self, values)
     |       从 dict 中读取 SpatialIndexInfo 信息
     |      
     |      :param dict values:
     |      :return: self
     |      :rtype: SpatialIndexInfo
     |  
     |  set\_grid\_center(self, value)
     |      设置网格索引的中心点。一般为数据集的中心点。
     |      
     |      :param Point2D value: 网格索引的中心点
     |      :return: self
     |      :rtype: SpatialIndexInfo
     |  
     |  set\_grid\_size0(self, value)
     |      设置多级格网索引中第一级格网的大小。
     |      
     |      :param float value:  多级格网索引中第一级格网的大小。
     |      :return: self
     |      :rtype: SpatialIndexInfo
     |  
     |  set\_grid\_size1(self, value)
     |      设置多级网格索引的第二级索引网格的大小。单位与数据集的单位一致
     |      
     |      :param float value: 多级网格索引的第二级索引网格的大小
     |      :return: self
     |      :rtype: SpatialIndexInfo
     |  
     |  set\_grid\_size2(self, value)
     |      设置多级格网索引中第三级格网的大小。
     |      
     |      :param float value:  三级网格的大小。单位与数据集同
     |      :return: self
     |      :rtype: SpatialIndexInfo
     |  
     |  set\_leaf\_object\_count(self, value)
     |      设置 R 树空间索引中叶结点的个数。
     |      
     |      :param int value: R 树空间索引中叶结点的个数
     |      :return: self
     |      :rtype: SpatialIndexInfo
     |  
     |  set\_quad\_level(self, value)
     |      设置四叉树索引的层级,最大值为13
     |      
     |      :param int value: 四叉树索引的层级
     |      :return: self
     |      :rtype: SpatialIndexInfo
     |  
     |  set\_tile\_height(self, value)
     |      设置空间索引的图幅高度。单位与数据集范围的单位一致
     |      
     |      :param float value: 空间索引的图幅高度
     |      :return: self
     |      :rtype: SpatialIndexInfo
     |  
     |  set\_tile\_width(self, value)
     |      设置空间索引的图幅宽度。单位与数据集范围的单位一致。
     |      
     |      :param float value: 空间索引的图幅宽度
     |      :return: self
     |      :rtype: SpatialIndexInfo
     |  
     |  set\_type(self, value)
     |      设置空间索引类型
     |      
     |      :param value: 空间索引类型
     |      :type value: SpatialIndexType or str
     |      :return: self
     |      :rtype: SpatialIndexInfo
     |  
     |  to\_dict(self)
     |      将当前对象输出到 dict 中
     |      
     |      :rtype: dict
     |  
     |  ----------------------------------------------------------------------
     |  Static methods defined here:
     |  
     |  make\_from\_dict(values)
     |      从 dict 中读取信息构造 SpatialIndexInfo 对象。
     |      
     |      :param dict values:
     |      :rtype: SpatialIndexInfo
     |  
     |  make\_mgrid(center, grid\_size0, grid\_size1, grid\_size2)
     |      构建多级网格索引
     |      
     |      多级网格索引,又叫动态索引。
     |      多级网格索引结合了 R 树索引与四叉树索引的优点,提供非常好的并发编辑支持,具有很好的普适性。若不能确定数据适用于哪种空间索引,可为其建立多级
     |      网格索引。采用划分多层网格的方式来组织管理数据。网格索引的基本方法是将数据集按照一定的规则划分成相等或不相等的网格,记录每一个地理对象所占的
     |      网格位置。在 GIS 中常用的是规则网格。当用户进行空间查询时,首先计算出用户查询对象所在的网格,通过该网格快速查询所选地理对象,可以优化查询操作。
     |      
     |      :param Point2D center: 指定的网格中心点
     |      :param float grid\_size0: 一级网格的大小。单位与数据集同
     |      :param float grid\_size1: 二级网格的大小。单位与数据集同
     |      :param float grid\_size2: 三级网格的大小。单位与数据集同
     |      :return: 多级网格索引信息
     |      :rtype: SpatialIndexInfo
     |  
     |  make\_qtree(level)
     |      构建四叉树索引信息。
     |      四叉树索引。四叉树是一种重要的层次化数据集结构,主要用来表达二维坐标下空间层次关系,实际上它是一维二叉树在二维空间的扩展。那么,四叉树索引
     |      就是将一张地图四等分,然后再每一个格子中再四等分,逐层细分,直至不能再分。现在在 SuperMap 中四叉树最多允许分成13层。基于希尔伯特
     |      (Hilbert)编码的排序规则,从四叉树中可确定索引类中每个对象实例的被索引属性值是属于哪个最小范围。从而提高了检索效率
     |      
     |      :param int level: 四叉树的层级,最大为13级
     |      :return: 四叉树索引信息
     |      :rtype: SpatialIndexInfo
     |  
     |  make\_rtree(leaf\_object\_count)
     |      构建 R 树索引信息。
     |      R 树索引是基于磁盘的索引结构,是 B 树(一维)在高维空间的自然扩展,易于与现有数据库系统集成,能够支持各种类型的空间查询处理操作,在实践中得
     |      到了广泛的应用,是目前最流行的空间索引方法之一。R 树空间索引方法是设计一些包含空间对象的矩形,将一些空间位置相近的目标对象,包含在这个矩形
     |      内,把这些矩形作为空间索引,它含有所包含的空间对象的指针。
     |      
     |      在进行空间检索的时候,首先判断哪些矩形落在检索窗口内,再进一步判断哪些目标是被检索的内容。这样可以提高检索速度。
     |      
     |      :param int leaf\_object\_count: R 树空间索引中叶结点的个数
     |      :return: R 树索引信息
     |      :rtype: SpatialIndexInfo
     |  
     |  make\_tile(tile\_width, tile\_height)
     |      构建图幅索引信息。
     |      在 SuperMap 中根据数据集的某一属性字段或根据给定的一个范围,将空间对象进行分类,通过索引进行管理已分类的空间对象,以此提高查询检索速度
     |      
     |      :param float tile\_width: 图幅宽度
     |      :param float tile\_height: 图幅高度
     |      :return: 图幅索引信息
     |      :rtype: SpatialIndexInfo
     |  
     |  ----------------------------------------------------------------------
     |  Data descriptors defined here:
     |  
     |  \_\_dict\_\_
     |      dictionary for instance variables (if defined)
     |  
     |  \_\_weakref\_\_
     |      list of weak references to the object (if defined)
     |  
     |  grid\_center
     |      Point2D:  网格索引的中心点。一般为数据集的中心点。
     |  
     |  grid\_size0
     |      float:  多级网格索引的第一层网格的大小
     |  
     |  grid\_size1
     |      float: 多级网格索引的第二层网格的大小
     |  
     |  grid\_size2
     |      float: 多级网格索引的第三层网格的大小
     |  
     |  leaf\_object\_count
     |      int: R 树空间索引中叶结点的个数
     |  
     |  quad\_level
     |      int: 四叉树索引的层级
     |  
     |  tile\_height
     |      float: 空间索引的图幅高度
     |  
     |  tile\_width
     |      float: 空间索引的图幅宽度
     |  
     |  type
     |      SpatialIndexType: 空间索引的类型
    
    class StepEvent(builtins.object)
     |  指示进度条的事件。当监听器的目标进度发生变化时触发该事件。
     |  某些功能能返回当前任务执行的进度信息,进度信息通过 StepEvent 返回,用户可以从 StepEvent 中获取当前任务进行的状态。
     |  
     |  例如,用户可以定义一个函数来显示缓冲区分析的进度信息。
     |  
     |  >>> def progress\_function(step\_event):
     |          print('\%s-\%s' \% (step\_event.title, step\_event.message))
     |  >>>
     |  >>> ds = Workspace().open\_datasource('E:/data.udb')
     |  >>> dt = ds['point'].query('SmID < 1000')
     |  >>> buffer\_dt = create\_buffer(dt, 10, 10, progress=progress\_function)
     |  
     |  Methods defined here:
     |  
     |  \_\_init\_\_(self, title=None, message=None, percent=None, remain\_time=None, cancel=None)
     |      Initialize self.  See help(type(self)) for accurate signature.
     |  
     |  \_\_repr\_\_(self)
     |      Return repr(self).
     |  
     |  set\_cancel(self, value)
     |      设置事件的取消状态。操作进行时如果设置为取消,则任务会中断执行。
     |      
     |      :param bool value: 事件取消的状态,如果为true,则中断执行
     |  
     |  ----------------------------------------------------------------------
     |  Data descriptors defined here:
     |  
     |  \_\_dict\_\_
     |      dictionary for instance variables (if defined)
     |  
     |  \_\_weakref\_\_
     |      list of weak references to the object (if defined)
     |  
     |  is\_cancel
     |      bool: 事件的取消状态
     |  
     |  message
     |      str: 正在进行操作的信息
     |  
     |  percent
     |      int: 当前操作完成的百分比
     |  
     |  remain\_time
     |      int: 完成当前操作预计的剩余时间,单位为秒
     |  
     |  title
     |      str: 进度信息的标题
    
    class TextPart(builtins.object)
     |  文本子对象类。 用于表示文本对象 :py:class:`GeoText` 的子对象,其存储子对象的文本,旋转角度,锚点等信息并提供对子对象进行处理的相关方法。
     |  
     |  Methods defined here:
     |  
     |  \_\_getstate\_\_(self)
     |  
     |  \_\_init\_\_(self, text=None, anchor\_point=None, rotation=None)
     |      构造文本子对象。
     |      
     |      :param str text:  文本子对象实例的文本内容。
     |      :param Point2D anchor\_point: 文本子对象实例的锚点。
     |      :param float rotation: 文本子对象的旋转角度,以度为单位,逆时针为正方向。
     |  
     |  \_\_repr\_\_(self)
     |      Return repr(self).
     |  
     |  \_\_setstate\_\_(self, state)
     |  
     |  \_\_str\_\_(self)
     |      Return str(self).
     |  
     |  clone(self)
     |      复制对象
     |      
     |      :rtype: TextPart
     |  
     |  from\_dict(self, values)
     |      从 dict 中读取文本子对象的信息
     |      
     |      :param dict values: 文本子对象
     |      :return: self
     |      :rtype: TextPart
     |  
     |  set\_anchor\_point(self, value)
     |      设置此文本子对象的锚点。该锚点与文本的对齐方式共同决定该文本子对象的显示位置。关于锚点与文本的对齐方式如何确定文本子对象的显示位置,请参见 :py:class:`TextAlignment` 类。
     |      
     |      :param Point2D value: 文本子对象的锚点
     |      :return: self
     |      :rtype: TextPart
     |  
     |  set\_rotation(self, value)
     |      设置此文本子对象的旋转角度。逆时针为正方向,单位为度。
     |      文本子对象通过数据引擎存储后返回的旋转角度,精度为 0.1 度;通过构造函数直接构造的文本子对象,返回的旋转角度精度不变。
     |      
     |      :param float value:  文本子对象的旋转角度
     |      :return: self
     |      :rtype: TextPart
     |  
     |  set\_text(self, value)
     |      设置文本子对象的文本子内容
     |      
     |      :param str value: 文本子对象的文本子内容
     |      :return: self
     |      :rtype: TextPart
     |  
     |  set\_x(self, value)
     |      设置此文本子对象锚点的横坐标
     |      
     |      :param float value: 此文本子对象锚点的横坐标
     |      :return: self
     |      :rtype: TextPart
     |  
     |  set\_y(self, value)
     |      设置文本子对象锚点的纵坐标
     |      
     |      :param float value: 文本对象锚点的纵坐标
     |      :return: self
     |      :rtype: TextPart
     |  
     |  to\_dict(self)
     |      将当前子对象输出为 dict
     |      
     |      :rtype: dict
     |  
     |  to\_json(self)
     |      将当前子对象输出为 json 字符串
     |      
     |      :rtype: str
     |  
     |  ----------------------------------------------------------------------
     |  Static methods defined here:
     |  
     |  from\_json(value)
     |      从 json 字符串中读取信息构造文本子对象
     |      
     |      :param str value: json 字符串
     |      :rtype: TextPart
     |  
     |  make\_from\_dict(values)
     |      从 dict 中读取信息构造文本子对象
     |      
     |      :param dict values: 文本子对象
     |      :rtype: TextPart
     |  
     |  ----------------------------------------------------------------------
     |  Data descriptors defined here:
     |  
     |  \_\_dict\_\_
     |      dictionary for instance variables (if defined)
     |  
     |  \_\_weakref\_\_
     |      list of weak references to the object (if defined)
     |  
     |  anchor\_point
     |      文本子对象实例的锚点。该锚点与文本的对齐方式共同决定该文本子对象的显示位置。关于锚点与文本的对齐方式如何确定文本子对象的显示位置,
     |      请参见 :py:class:`TextAlignment` 类。
     |  
     |  rotation
     |      文本子对象的旋转角度,以度为单位,逆时针为正方向。
     |  
     |  text
     |      str: 此文本子对象的文本内容
     |  
     |  x
     |      float: 文本子对象锚点的横坐标,默认值为0
     |  
     |  y
     |      float: 文本子对象锚点的纵坐标,默认值为0
    
    class TextStyle(builtins.object)
     |  文本风格类。 用于设置 :py:class:`GeoText` 类对象的风格
     |  
     |  Methods defined here:
     |  
     |  \_\_getstate\_\_(self)
     |  
     |  \_\_init\_\_(self)
     |      Initialize self.  See help(type(self)) for accurate signature.
     |  
     |  \_\_setstate\_\_(self, state)
     |  
     |  \_\_str\_\_(self)
     |      Return str(self).
     |  
     |  clone(self)
     |      拷贝对象
     |      
     |      :rtype: TextStyle
     |  
     |  from\_dict(self, values)
     |      从 dict 中读取文本风格信息
     |      
     |      :param dict values: 包含文本风格信息的 dict
     |      :return: self
     |      :rtype: TextStyle
     |  
     |  set\_alignment(self, value)
     |      设置文本的对齐方式
     |      
     |      :param value:  文本的对齐方式
     |      :type value: TextAlignment or str
     |      :return: self
     |      :rtype: TextStyle
     |  
     |  set\_back\_color(self, value)
     |      设置文本的背景色。
     |      
     |      :param value: 文本的背景色
     |      :type value: int or tuple
     |      :return: self
     |      :rtype: TextStyle
     |  
     |  set\_back\_opaque(self, value)
     |      设置文本背景是否不透明,True 表示文本背景不透明
     |      
     |      :param bool value: 文本背景是否不透明
     |      :return: self
     |      :rtype: TextStyle
     |  
     |  set\_bold(self, value)
     |      设置文本是否为粗体字,True 表示为粗体
     |      
     |      :param bool value: 文本是否为粗体字
     |      :return: self
     |      :rtype: TextStyle
     |  
     |  set\_border\_spacing\_width(self, value)
     |      设置文字背景矩形框边缘与文字边缘的间隔,单位为:像素。
     |      
     |      :param int value:
     |      :return: self
     |      :rtype: TextStyle
     |  
     |  set\_font\_height(self, value)
     |      设置文本字体的高度。在固定大小时单位为1毫米,否则使用地理坐标单位。
     |      
     |      :param float value: 文本字体的高度
     |      :return: self
     |      :rtype: TextStyle
     |  
     |  set\_font\_name(self, value)
     |      设置文本字体的名称。 如果在Windows平台下对地图中的文本图层指定了某种字体,并且该地图数据需要在Linux平台下进行应用,那么请确保您的Linux
     |      平台下也存在同样的字体,否则,文本图层的字体显示效果会有问题。
     |      
     |      :param str value: 文本字体的名称。文本字体的名称的默认值为 "Times New Roman"。
     |      :return: self
     |      :rtype: TextStyle
     |  
     |  set\_font\_scale(self, value)
     |      设置注记字体的缩放比例
     |      
     |      :param float value: 注记字体的缩放比例
     |      :return: self
     |      :rtype: TextStyle
     |  
     |  set\_font\_width(self, value)
     |      设置文本的宽度。字体的宽度以英文字符为标准,由于一个中文字符相当于两个英文字符。在固定大小时单位为1毫米,否则使用地理坐标单位。
     |      
     |      :param float value: 文本的宽度
     |      :return: self
     |      :rtype: TextStyle
     |  
     |  set\_fore\_color(self, value)
     |      设置文本的前景色
     |      
     |      :param value: 文本的前景色
     |      :type value: int or tuple
     |      :return: self
     |      :rtype: TextStyle
     |  
     |  set\_italic(self, value)
     |      设置文本是否采用斜体,true 表示采用斜体。
     |      
     |      :param bool value: 文本是否采用斜体
     |      :return: self
     |      :rtype: TextStyle
     |  
     |  set\_italic\_angle(self, value)
     |      设置字体倾斜角度,正负度之间,以度为单位,精确到0.1度。当倾斜角度为0度,为系统默认的字体倾斜样式。
     |      正负度是指以纵轴为起始零度线,其纵轴左侧为正,右侧为负。允许的最大角度为60,最小-60。大于60按照60处理,小于-60按照-60处理。
     |      
     |      :param float value: 字体倾斜角度,正负度之间,以度为单位,精确到0.1度
     |      :return: self
     |      :rtype: TextStyle
     |  
     |  set\_opaque\_rate(self, value)
     |      设置注记文字的不透明度。不透明度的范围为0-100。
     |      
     |      :param int value: 注记文字的不透明度
     |      :return: self
     |      :rtype: TextStyle
     |  
     |  set\_outline(self, value)
     |      设置是否以轮廓的方式来显示文本的背景。false,表示不以轮廓的方式来显示文本的背景。
     |      
     |      :param bool value: 是否以轮廓的方式来显示文本的背景
     |      :return: self
     |      :rtype: TextStyle
     |  
     |  set\_outline\_width(self, value)
     |      设置文本轮廓的宽度,数值的单位为:像素,数值范围是从0到5之间的任意整数,其中设置为0值时表示没有轮廓。 必须通过方法 :py:meth:`is\_outline` 为
     |      True 时,文本轮廓的宽度设置才有效。
     |      
     |      :param int value: 文本轮廓的宽度,数值的单位为:像素,数值范围是从0到5之间的任意整数,其中设置为0值时表示没有轮廓。
     |      :return: self
     |      :rtype: TextStyle
     |  
     |  set\_rotation(self, value)
     |      设置文本旋转的角度。逆时针方向为正方向,单位为度。
     |      
     |      :param float value: 文本旋转的角度
     |      :return: self
     |      :rtype: TextStyle
     |  
     |  set\_shadow(self, value)
     |      设置文本是否有阴影。True 表示给文本增加阴影
     |      
     |      :param bool value: 文本是否有阴影
     |      :return: self
     |      :rtype: TextStyle
     |  
     |  set\_size\_fixed(self, value)
     |      设置文本大小是否固定。False,表示文本为非固定尺寸的文本。
     |      
     |      :param bool value: 文本大小是否固定。False,表示文本为非固定尺寸的文本。
     |      :return: self
     |      :rtype: TextStyle
     |  
     |  set\_strikeout(self, value)
     |      设置文本字体是否加删除线。
     |      
     |      :param bool value:  文本字体是否加删除线。
     |      :return: self
     |      :rtype: TextStyle
     |  
     |  set\_string\_alignment(self, value)
     |      设置文本的排版方式,可以对多行文本设置左对齐、右对齐、居中对齐、两端对齐
     |      
     |      :param value: 文本的排版方式
     |      :type value: StringAlignment
     |      :return: self
     |      :rtype: TextStyle
     |  
     |  set\_underline(self, value)
     |      设置文本字体是否加下划线。True 表示加下划线。
     |      
     |      :param bool value: 文本字体是否加下划线。True 表示加下划线
     |      :return: self
     |      :rtype: TextStyle
     |  
     |  set\_weight(self, value)
     |      设置文本字体的磅数,表示粗体的具体数值。取值范围为从0-900之间的整百数,如400表示正常显示,700表示为粗体,可参见微软 MSDN 帮助中关于
     |      LOGFONT 类的介绍
     |      
     |      :param int value: 文本字体的磅数。
     |      :return: self
     |      :rtype: TextStyle
     |  
     |  to\_dict(self)
     |      将当前对象输出为 dict
     |      
     |      :rtype: dict
     |  
     |  to\_json(self)
     |      输出为 json 字符串
     |      
     |      :rtype: str
     |  
     |  ----------------------------------------------------------------------
     |  Static methods defined here:
     |  
     |  from\_json(value)
     |      从 json 字符串中构建 TextStyle 对象
     |      
     |      :param str value:
     |      :rtype: TextStyle
     |  
     |  make\_from\_dict(values)
     |      从 dict 中读取文本风格信息构造 TextStyle
     |      
     |      :param dict values: 包含文本风格信息的 dict
     |      :rtype: TextStyle
     |  
     |  ----------------------------------------------------------------------
     |  Data descriptors defined here:
     |  
     |  \_\_dict\_\_
     |      dictionary for instance variables (if defined)
     |  
     |  \_\_weakref\_\_
     |      list of weak references to the object (if defined)
     |  
     |  alignment
     |      TextAlignment: 文本的对齐方式。
     |  
     |  back\_color
     |      tuple: 文本的背景色,默认颜色为黑色
     |  
     |  border\_spacing\_width
     |      int: 返回文字背景矩形框边缘与文字边缘的间隔,单位为:像素
     |  
     |  font\_height
     |      float: 文本字体的高度。在固定大小时单位为1毫米,否则使用地理坐标单位。默认值为 6。
     |  
     |  font\_name
     |      str: 返回文本字体的名称。如果在Windows平台下对地图中的文本图层指定了某种字体,并且该地图数据需要在Linux平台下进行应用,那么请确保您
     |      的Linux平台下也存在同样的字体,否则,文本图层的字体显示效果会有问题。文本字体的名称的默认值为 "Times New Roman"。
     |  
     |  font\_scale
     |      float: 注记字体的缩放比例
     |  
     |  font\_width
     |      float: 文本的宽度。字体的宽度以英文字符为标准,由于一个中文字符相当于两个英文字符。在固定大小时单位为1毫米,否则使用地理坐标单位。
     |  
     |  fore\_color
     |      tuple: 文本的前景色,默认色为黑色。
     |  
     |  is\_back\_opaque
     |      bool: 文本背景是否不透明,True 表示文本背景不透明。 默认不透明
     |  
     |  is\_bold
     |      bool: 返回文本是否为粗体字,True 表示为粗体
     |  
     |  is\_italic
     |      bool: 文本是否采用斜体,True 表示采用斜体
     |  
     |  is\_outline
     |      bool: 返回是否以轮廓的方式来显示文本的背景
     |  
     |  is\_shadow
     |      bool: 文本是否有阴影。True 表示给文本增加阴影
     |  
     |  is\_size\_fixed
     |      bool: 文本大小是否固定。False,表示文本为非固定尺寸的文本
     |  
     |  is\_strikeout
     |      bool: 文本字体是否加删除线。True 表示加删除线。
     |  
     |  is\_underline
     |      bool: 文本字体是否加下划线。True 表示加下划线。
     |  
     |  italic\_angle
     |      float: 返回字体倾斜角度,正负度之间,以度为单位,精确到0.1度。当倾斜角度为0度,为系统默认的字体倾斜样式。
     |      正负度是指以纵轴为起始零度线,其纵轴左侧为正,右侧为负。允许的最大角度为60,最小-60。大于60按照60处理,小于-60按照-60处理。
     |  
     |  opaque\_rate
     |      int: 设置注记文字的不透明度。不透明度的范围为0-100。
     |  
     |  outline\_width
     |      float: 文本轮廓的宽度,数值的单位为:像素,数值范围是从0到5之间的任意整数。
     |  
     |  rotation
     |      float: 文本旋转的角度。逆时针方向为正方向,单位为度。
     |  
     |  string\_alignment
     |      StringAlignment: 文本的排版方式
     |  
     |  weight
     |      float: 文本字体的磅数,表示粗体的具体数值。取值范围为从0-900之间的整百数,如400表示正常显示,700表示为粗体,可参见微软 MSDN 帮助中
     |      关于 LOGFONT 类的介绍。默认值为400
    
    class TimeCondition(builtins.object)
     |  定义了单时间字段时空模型管理查询功能接口
     |  
     |  Methods defined here:
     |  
     |  \_\_init\_\_(self, field\_name=None, time=None, condition=None, back\_condition=None)
     |      构造时空模型查询条件对象
     |      
     |      :param field\_name: 字段名称
     |      :type field\_name: str
     |      :param time: 查询条件的时间
     |      :type time: datetime.datetime
     |      :param condition: 查询时间的条件操作符。例如:>、<、>=、<=、=
     |      :type condition: str
     |      :param back\_condition: 指定的后一个条件操作符,例如:and、or
     |      :type back\_condition: str
     |  
     |  \_\_str\_\_(self)
     |      Return str(self).
     |  
     |  from\_dict(self, values)
     |      从 dict 对象中读取 TimeCondition 信息。
     |      
     |      :param dict values:  包含 TimeCondition 信息的 dict,具体参考 to\_dict
     |      :rtype: TimeCondition
     |      :rtype: TimeCondition
     |  
     |  set\_back\_condition(self, value)
     |      查询条件的后一个条件操作符。例如:and、or
     |      
     |      :param str value: 查询条件的后一个条件操作符
     |      :return: self
     |      :rtype: TimeCondition
     |  
     |  set\_condition(self, value)
     |      设置查询条件的条件操作符。
     |      
     |      :param str value: 查询条件的条件操作符,>、<、>=、<=、=
     |      :return: self
     |      :rtype: TimeCondition
     |  
     |  set\_field\_name(self, value)
     |      设置查询条件的字段名
     |      
     |      :param str value: 查询条件的字段名
     |      :return: self
     |      :rtype: TimeCondition
     |  
     |  set\_time(self, value)
     |      设置查询条件的时间值
     |      
     |      :param datetime.datetime value: 时间值
     |      :return: self
     |      :rtype: TimeCondition
     |  
     |  to\_dict(self)
     |      将当前对象输出为 dict 对象
     |      
     |      :rtype: dict
     |  
     |  to\_json(self)
     |      将当前对象输出为 json 字符串
     |      
     |      :rtype: str
     |  
     |  ----------------------------------------------------------------------
     |  Static methods defined here:
     |  
     |  from\_json(value)
     |      从 json 字符串中构造 TimeCondition。
     |      
     |      :param str value: json 字符串
     |      :rtype: TimeCondition
     |  
     |  make\_from\_dict(values)
     |      从 dict 对象中构造 TimeCondition 对象
     |      
     |      :param dict values:  包含 TimeCondition 信息的 dict,具体参考 to\_dict
     |      :rtype: TimeCondition
     |  
     |  ----------------------------------------------------------------------
     |  Data descriptors defined here:
     |  
     |  \_\_dict\_\_
     |      dictionary for instance variables (if defined)
     |  
     |  \_\_weakref\_\_
     |      list of weak references to the object (if defined)
     |  
     |  back\_condition
     |      指定的后一个条件操作符,例如:and、or
     |  
     |  condition
     |      str: 查询时间的条件操作符。例如:>、<、>=、<=、=
     |  
     |  field\_name
     |      str: 字段名称
     |  
     |  time
     |      datetime.datetime: 作为查询条件的时间
    
    class Workspace(iobjectspy.\_jsuperpy.data.\_jvm.JVMBase)
     |  工作空间是用户的工作环境,主要完成数据的组织和管理,包括打开、关闭、创建、保存工作空间文件。工作空间(Workspace)是 SuperMap 中的一个重要的
     |  概念,工作空间存储了一个工程项目(同一个事务过程)中所有的数据源,地图的组织关系。通过工作空间对象可以管理数据源和地图。工作空间中只存储数据源的
     |  连接信息和位置等,实际的数据源都是存储在数据库或者 UDB 中。工作空间只存储地图的一些配置信息,如地图包含图层的个数,图层引用的数据集,地图范围,
     |  背景风格等。在当前版本中,一个程序只能存在一个工作空间对象,如果用户没有打开特定的工作空间,程序将默认创建一个工作空间对象。用户如果需要打开新的
     |  工作空间对象,需要先将当前工作空间保存和关闭,否则,存储在工作空间中的一些信息可能会丢失。
     |  
     |  例如,创建数据源对象::
     |  
     |      >>> ws = Workspace()
     |      >>> ws.create\_datasource(':memory:')
     |      >>> print(len(ws.datasources))
     |      1
     |      >>> ws\_a = Workspace()
     |      >>> ws\_a.create\_datasource(':memory:')
     |      >>> ws == ws\_a
     |      True
     |      >>> print(len(ws\_a.datasources))
     |      2
     |      >>> ws.close()
     |  
     |  Method resolution order:
     |      Workspace
     |      iobjectspy.\_jsuperpy.data.\_jvm.JVMBase
     |      builtins.object
     |  
     |  Methods defined here:
     |  
     |  \_\_init\_\_(self)
     |      Initialize self.  See help(type(self)) for accurate signature.
     |  
     |  close\_all\_datasources(self)
     |      关闭所有的数据源
     |  
     |  close\_datasource(self, item)
     |      关闭指定的数据源。
     |      
     |      :param item: 数据源的别名或序号
     |      :type item: str or int
     |      :return: 关闭成功返回 True,否则返回 False
     |      :rtype: bool
     |  
     |  create\_datasource(self, conn\_info)
     |      根据指定的数据源连接信息,创建新的数据源。
     |      
     |      :param conn\_info: udb文件路径或数据源连接信息:
     |                        - 数据源连接信息。具体可以参考 :py:meth:`DatasourceConnectionInfo.make`
     |                        - 如果 conn\_info 为 str 时,可以为 ':memory:', udb 文件路径,udd 文件路径,dcf 文件路径,数据源连接信息的 xml 字符串
     |                        - 如果 conn\_info 为 dict,为  :py:meth:`DatasourceConnectionInfo.to\_dict` 的返回结果。
     |      :type conn\_info: str or dict or DatasourceConnectionInfo
     |      :return: 数据源对象
     |      :rtype: Datasource
     |  
     |  get\_datasource(self, item)
     |      获取指定的数据源对象。
     |      
     |      :param item: 数据源的别名或序号
     |      :type item:  str or int
     |      :return: 数据源对象
     |      :rtype: Datasource
     |  
     |  index\_of\_datasource(self, alias)
     |      查找指定的数据源别名所在序号。不存在将抛出异常。
     |      
     |      :param str alias: 数据源别名
     |      :return: 数据源所在的序号
     |      :rtype: int
     |      :raise ValueError: 不存在指定的数据源别名时抛出异常。
     |  
     |  is\_contains\_datasource(self, item)
     |      是否存在指定序号或者数据源别名的数据源
     |      
     |      :param item: 数据源的别名或序号
     |      :type item:  str or int
     |      :return:  存在返回 True,否则返回 False
     |      :rtype: bool
     |  
     |  is\_modified(self)
     |      返回工作空间的内容是否有改动,如果对工作空间的内容进行了一些修改,则返回 True,否则返回 False。工作空间负责管理数据源、地图,其中任何
     |      一项内容发生变动,此属性都会返回 True,在关闭整个应用程序时,先用此属性判断工作空间是否已有改动,可用于提示用户是否需要存盘。
     |      
     |      :return: 对工作空间的内容进行了一些修改,则返回 True,否则返回 False
     |      :rtype: bool
     |  
     |  modify\_datasource\_alias(self, old\_alias, new\_alias)
     |      修改数据源的别名。数据源别名不区分大小写
     |      
     |      :param str old\_alias: 待修改的数据源别名
     |      :param str new\_alias: 数据源的新别名
     |      :return: 如果对数据源修改别名成功,则返回 True,否则返回 False
     |      :rtype: bool
     |  
     |  open\_datasource(self, conn\_info, is\_get\_existed=True)
     |      根据数据源连接信息打开数据源。如果设置的连接信息是UDB类型数据源,或者 is\_get\_existed 为 True,如果工作空间中已经存在对应的数据源,则
     |      会直接返回。不支持直接打开内存数据源,要使用内存数据源,需要使用 :py:meth:`create\_datasource` 创建内存数据源。
     |      
     |      :param conn\_info: udb文件路径或数据源连接信息:
     |                        - 数据源连接信息。具体可以参考 :py:meth:`DatasourceConnectionInfo.make`
     |                        - 如果 conn\_info 为 str 时,可以为 ':memory:', udb 文件路径,udd 文件路径,dcf 文件路径,数据源连接信息的 xml 字符串
     |                        - 如果 conn\_info 为 dict,为  :py:meth:`DatasourceConnectionInfo.to\_dict` 的返回结果。
     |      :type conn\_info: str or dict or DatasourceConnectionInfo
     |      :param bool is\_get\_existed: is\_get\_existed 为 True,如果工作空间中已经存在对应的数据源,则会直接返回。为 false 时,则会打开新的数据源。对于 UDB 类型数据源,无论 is\_get\_existed 为 True 还是 False,都会优先返回工作空间中的数据源。判断 DatasourceConnectionInfo 是否与工作空间中的数据源是同一个数据源,可以查看 :py:meth:`DatasourceConnectionInfo.is\_same`
     |      :return: 数据源对象
     |      :rtype: Datasource
     |      
     |      
     |      >>> ws = Workspace()
     |      >>> ds = ws.open\_datasource('E:/data.udb')
     |      >>> print(ds.type)
     |      EngineType.UDB
     |  
     |  set\_caption(self, caption)
     |      设置工作空间显示名称。
     |      
     |      :param str caption: 工作空间显示名称
     |  
     |  set\_description(self, description)
     |      设置用户加入的对当前工作空间的描述或说明性信息
     |      
     |      :param str description: 用户加入的对当前工作空间的描述或说明性信息
     |  
     |  ----------------------------------------------------------------------
     |  Class methods defined here:
     |  
     |  close() from builtins.type
     |      关闭工作空间,关闭工作空间将会销毁当前工作空间的实例对象。工作空间的关闭之前确保使用的该工作空间的地图等内容关闭或断开链接。
     |      如果工作空间是在 Java 端注册的,将不会实际关闭工作空间对象,只会解除对 Java 工作空间对象的绑定关系,后续将不能继续操作 Java
     |      的工作空间对象,除非使用 Workspace() 构造新的实例。
     |  
     |  create(conn\_info, save\_existed=True, saved\_connection\_info=None) from builtins.type
     |      创建一个新的工作空间对象。在创建新的工作空间前,用户可以通过设定 save\_existed 为 True 先保存当前工作空间对象,也可以设定
     |      saved\_connection\_info 将当前工作空间另存为指定的位置。
     |      
     |      :param WorkspaceConnectionInfo conn\_info: 工作空间的连接信息
     |      :param bool save\_existed:  是否保存当前的工作空间工作。如果设置为 True,则会先将当前工作空间保存然后再关闭当前工作空间,否则会直接关闭当前工作空间,然后再打开新的工作空间对象。save\_existed 只适合用于当前工作空间不在内存中的情形。默认为 True。
     |      :param WorkspaceConnectionInfo saved\_connection\_info: 选择将当前工作另存到 saved\_connection\_info 指定的工作空间中。 默认为 None。
     |      :return:  新的工作空间对象
     |      :rtype: Workspace
     |  
     |  open(conn\_info, save\_existed=True, saved\_connection\_info=None) from builtins.type
     |      打开一个新的工作空间对象。在打开新的工作空间前,用户可以通过设定 save\_existed 为 True 先保存当前工作空间对象,也可以设定
     |      saved\_connection\_info 将当前工作空间另存为指定的位置。
     |      
     |      :param WorkspaceConnectionInfo conn\_info: 工作空间的连接信息
     |      :param bool save\_existed:  是否保存当前的工作空间工作。如果设置为 True,则会先将当前工作空间保存然后再关闭当前工作空间,否则会直接关闭当前工作空间,然后再打开新的工作空间对象。save\_existed 只适合用于当前工作空间不在内存中的情形。默认为 True。
     |      :param WorkspaceConnectionInfo saved\_connection\_info: 选择将当前工作另存到 saved\_connection\_info 指定的工作空间中。 默认为 None。
     |      :return:  新的工作空间对象
     |      :rtype: Workspace
     |  
     |  save() from builtins.type
     |      用于将现存的工作空间存盘,不改变原有的名称
     |      
     |      :return:  保存成功返回 True,否则返回 False
     |      :rtype: bool
     |  
     |  save\_as(conn\_info) from builtins.type
     |      用指定的工作空间连接信息对象来保存工作空间文件。
     |      
     |      :param WorkspaceConnectionInfo conn\_info: 工作空间连接信息对象
     |      :return: 另存成功返回 True,否则返回 False
     |      :rtype: bool
     |  
     |  ----------------------------------------------------------------------
     |  Static methods defined here:
     |  
     |  \_\_new\_\_(cls)
     |      Create and return a new object.  See help(type) for accurate signature.
     |  
     |  ----------------------------------------------------------------------
     |  Data descriptors defined here:
     |  
     |  caption
     |      str: 工作空间显示名称,便于用户做一些标识。
     |  
     |  connection\_info
     |      WorkspaceConnectionInfo: 工作空间的连接信息
     |  
     |  datasources
     |      list[Datasource]: 当前工作空间下的所有数据源对象。
     |  
     |  description
     |      str: 用户加入的对当前工作空间的描述或说明性信息
     |  
     |  ----------------------------------------------------------------------
     |  Data descriptors inherited from iobjectspy.\_jsuperpy.data.\_jvm.JVMBase:
     |  
     |  \_\_dict\_\_
     |      dictionary for instance variables (if defined)
     |  
     |  \_\_weakref\_\_
     |      list of weak references to the object (if defined)
    
    class WorkspaceConnectionInfo(iobjectspy.\_jsuperpy.data.\_jvm.JVMBase)
     |  工作空间连接信息类。包括了进行工作空间连接的所有信息,如所要连接的服务器名称,数据库名称,用户名,密码等。对不同类型的工作空间,所以在使用该类所
     |  包含的成员时,请注意该成员所适用的工作空间类型。
     |  
     |  Method resolution order:
     |      WorkspaceConnectionInfo
     |      iobjectspy.\_jsuperpy.data.\_jvm.JVMBase
     |      builtins.object
     |  
     |  Methods defined here:
     |  
     |  \_\_init\_\_(self, server=None, workspace\_type=None, version=None, driver=None, database=None, name=None, user=None, password=None)
     |      初始化工作空间链接信息对象。
     |      
     |      :param str server:  数据库服务器名或文件名
     |      :param workspace\_type:  工作空间的类型
     |      :type workspace\_type: WorkspaceType or str
     |      :param version: 工作空间的版本
     |      :type version: WorkspaceVersion or str
     |      :param str driver: 设置使用 ODBC 连接的数据库的驱动程序名,对目前支持的数据库工作空间中,SQL Server 数据库使用 ODBC 连接,SQL Server 数据库的驱动程序名如为 SQL Server 或 SQL Native Client
     |      :param str database: 工作空间连接的数据库名
     |      :param str name: 工作空间在数据库中的名
     |      :param str user: 登录数据库的用户名
     |      :param str password: 登录工作空间连接的数据库或文件的密码。
     |  
     |  set\_database(self, value)
     |      设置工作空间连接的数据库名。对数据库类型工作空间适用
     |      
     |      :param str value:  工作空间连接的数据库名
     |      :return: self
     |      :rtype: WorkspaceConnectionInfo
     |  
     |  set\_driver(self, value)
     |      设置 使用 ODBC 连接的数据库的驱动程序名。对目前支持的数据库工作空间中,SQL Server 数据库使用 ODBC 连接,SQL Server 数据库的驱动
     |      程序名如为 SQL Server 或 SQL Native Client。
     |      
     |      :param str value:   使用 ODBC 连接的数据库的驱动程序名
     |      :return: self
     |      :rtype: WorkspaceConnectionInfo
     |  
     |  set\_name(self, value)
     |      设置工作空间在数据库中的名称。
     |      
     |      :param str value: 工作空间在数据库中的名称,对文件型的工作空间,此名称设为空
     |      :return: self
     |      :rtype: WorkspaceConnectionInfo
     |  
     |  set\_password(self, value)
     |      设置登录工作空间连接的数据库或文件的密码。此密码的设置只对 Oracle 和 SQL 数据源有效,对本地(UDB)数据源无效。
     |      
     |      :param str value:  登录工作空间连接的数据库或文件的密码
     |      :return: self
     |      :rtype: WorkspaceConnectionInfo
     |  
     |  set\_server(self, value)
     |      设置数据库服务器名或文件名。
     |      
     |      :param str value: 对于 Oracle 数据库,其服务器名为其 TNS 服务名称; 对于 SQL Server 数据库,其服务器名为其系统的 DNS(Database Source Name)名称;对于 SXWU 和 SMWU 文件,其服务器名称为其文件名称,其中包括路径名称和文件的后缀名。特别地,此处的路径为绝对路径。
     |      :return: self
     |      :rtype: WorkspaceConnectionInfo
     |  
     |  set\_type(self, value)
     |      设置工作空间的类型。
     |      
     |      :param value:  工作空间的类型
     |      :type value: WorkspaceType or str
     |      :return: self
     |      :rtype: WorkspaceConnectionInfo
     |  
     |  set\_user(self, value)
     |      设置登录数据库的用户名。对数据库类型工作空间适用。
     |      
     |      :param str value:  登录数据库的用户名
     |      :return: self
     |      :rtype: WorkspaceConnectionInfo
     |  
     |  set\_version(self, value)
     |      设置工作空间版本。
     |      
     |      :param value:  工作空间的版本
     |      :type value: WorkspaceVersion or str
     |      :return: self
     |      :rtype: WorkspaceConnectionInfo
     |      
     |      
     |      例如,设置工作空间版本为 UGC60::
     |      
     |          >>> conn\_info = WorkspaceConnectionInfo()
     |          >>> conn\_info.set\_version('UGC60')
     |          >>> print(conn\_info.version)
     |          WorkspaceVersion.UGC60
     |  
     |  ----------------------------------------------------------------------
     |  Data descriptors defined here:
     |  
     |  database
     |      str: 工作空间连接的数据库名。对数据库类型工作空间适用
     |  
     |  driver
     |      str: 使用 ODBC 连接的数据库的驱动程序名
     |  
     |  name
     |      str: 工作空间在数据库中的名称,对文件型的工作空间,此名称为空
     |  
     |  password
     |      str: 登录工作空间连接的数据库或文件的密码
     |  
     |  server
     |      str: 数据库服务器名或文件名
     |  
     |  type
     |      WorkspaceType: 工作空间的类型。工作空间可以存储在文件中,也可以存储在数据库中。目前支持的文件型的工作空间的类型为 SXWU 格式和 SMWU 格式的工作空间;
     |      数据库型工作空间为 ORACLE 格式和 SQL 格式的工作空间;默认的工作空间类型为未存储的工作空间
     |  
     |  user
     |      str: 登录数据库的用户名
     |  
     |  version
     |      WorkspaceVersion: 工作空间的版本。默认为 UGC70.
     |  
     |  ----------------------------------------------------------------------
     |  Data descriptors inherited from iobjectspy.\_jsuperpy.data.\_jvm.JVMBase:
     |  
     |  \_\_dict\_\_
     |      dictionary for instance variables (if defined)
     |  
     |  \_\_weakref\_\_
     |      list of weak references to the object (if defined)

FUNCTIONS
    aggregate\_points\_geo(points, min\_pile\_point\_count, distance, unit='Meter', prj=None, as\_region=False)
        对点集合进行密度聚类。密度聚类算法介绍参考 :py:meth:`iobjectspy.aggregate\_points`
        
        :param points: 输入的点集合
        :type points: list[Point2D] or tuple[Point2D]
        :param int min\_pile\_point\_count: 密度聚类点数目阈值,必须大于等于2。阈值越大表示能聚类为一簇的条件越苛刻。推荐值为4。
        :param float distance: 密度聚类半径。
        :param unit: 密度聚类半径的单位。如果空间参考坐标系prjCoordSys无效,此参数也无效
        :type unit: Unit or str
        :param prj:  点集合的空间参考坐标系
        :type prj: PrjCoordSys
        :param bool as\_region: 是否返回聚类后的面对象
        :return: 当 as\_region 为 False 时,返回一个list,list中每个值代表点对象的聚类类别,聚类类别从1开始,0表示为无效聚类。
                 当 as\_region 为 True 时,将返回每一簇点集聚集成的多边形对象
        :rtype: list[int] or list[GeoRegion]
    
    can\_contain(geo\_search, geo\_target)
        判断搜索几何对象是否包含被搜索几何对象。包含则返回 True。
        注意,如果存在包含关系,则:
        
            * 搜索几何对象的外部和被搜索几何对象的内部的交集为空;
            * 两个几何对象的内部交集不为空或者搜索几何对象的边界与被搜索几何对象的内部交集不为空;
            * 点查线,点查面,线查面,不存在包含情况;
            * 与 :py:meth:`is\_within` 是逆运算;
            * 该关系适合的几何对象类型:
        
                * 搜索几何对象:点、线、面;
                * 被搜索几何对象:点、线、面。
        
        .. image:: ../image/Geometrist\_CanContain.png
        
        :param Geometry geo\_search: 搜索几何对象,支持点、线、面类型。
        :param Geometry geo\_target: 被搜索几何对象,支持点、线、面类型。
        :return: 搜索几何对象包含被搜索几何对象返回 True;否则返回 False。
        :rtype: bool
    
    clip(geometry, clip\_geometry)
        生成被操作对象经过操作对象裁剪后的几何对象。
        注意:
        
        * 被操作几何对象只有落在操作几何对象内的那部分才会被输出为结果几何对象;
        * clip 与 intersect 在空间处理上是一致的,不同在于对结果几何对象属性的处理,clip 分析只是用来做裁剪,结果几何对象只保留被操作几何对象的非系统字段,而 Intersect 求交分析的结果则可以根据字段设置情况来保留两个几何对象的字段。
        * 该操作适合的几何对象类型:
        
            * 操作几何对象:面;
            * 被操作几何对象:线、面。
        
        .. image:: ../image/Geometrist\_Clip.png
        
        
        :param geometry:  被操作几何对象,支持线和面类型。
        :type geometry: GeoLine or GeoRegion
        :param clip\_geometry: 操作几何对象,必须是面对象。
        :type clip\_geometry:  GeoRegion or Rectangle
        :return: 裁剪结果对象
        :rtype: Geometry
    
    close\_datasource(item)
        关闭指定的数据源。
        
        具体参考  :py:meth:`Workspace.close\_datasource`
        
        :param item: 数据源的别名或序号
        :type item: str or int
        :return: 关闭成功返回 True,否则返回 False
        :rtype: bool
    
    combine\_band(red\_dataset, green\_dataset, blue\_dataset, out\_data=None, out\_dataset\_name=None)
        三个单波段数据集合成RGB数据集
        
        :param red\_dataset:  单波段数据集R。
        :type red\_dataset: Dataset or str
        :param green\_dataset: 单波段数据集G
        :type green\_dataset: Dataset or str
        :param blue\_dataset: 单波段数据集B
        :type blue\_dataset: Dataset or str
        :param out\_data: 结果数据集所在的数据源,为空时使用 red\_dataset 数据集所在的数据源
        :type out\_data: Datasource or DatasourceConnectionInfo or str
        :param str out\_dataset\_name:  合成RGB数据集的名称。
        :return: 合成成功返回数据集对象或数据集名称,失败返回 None
        :rtype: Dataset
    
    compute\_concave\_hull(points, angle=45.0)
        计算点集的凹闭包。
        
        :param points:  指定的点集。
        :type points: list[Point2D] or tuple[Point2D]
        :param angle: 凹包内最小角度。 推荐值为 45度到75度,角度越大,凹包会更解决凸包的形状,角度越小,产生的凹多边形相邻顶点之间的夹角可能比较尖锐。
        :type angle: float
        :return: 返回可以包含指定点集中所有点的凹多边形。
        :rtype: GeoRegion
    
    compute\_convex\_hull(points)
        计算几何对象的凸闭包,即最小外接多边形。返回一个简单凸多边形。
        
        :param points:  点集
        :type points: list[Point2D] or tuple[Point2D] or Geometry
        :return: 最小外接多边形。
        :rtype: GeoRegion
    
    compute\_default\_tolerance(prj)
        计算坐标系默认容限。
        
        :param prj:  指定的投影坐标系类型对象。
        :type prj: PrjCoordSys
        :return: 返回指定的投影坐标系类型对象的默认容限值。
        :rtype: float
    
    compute\_distance(geometry1, geometry2)
        求两个几何对象之间的距离。
        注意:几何对象的类型只能是点、线和面。这里的距离指的是两个几何对象边线间最短距离。例如:点到线的最短距离就是点到该线的垂直距离。
        
        :param geometry1: 第一个几何对象
        :type geometry1: Geometry or Point2D or Rectangle
        :param geometry2: 第二个几何对象
        :type geometry2: Geometry or Point2D or Rectangle
        :return: 两个几何对象之间的距离
        :rtype: float
    
    compute\_geodesic\_area(geometry, prj)
        计算经纬度面积。
        
        注意:
        
        * 使用该方法计算经纬度面积,在通过 prj 参数指定投影坐标系类型对象(PrjCoordSys)时,必须通过该对象的 set\_type 方法设置投影坐
          标系类型为地理经纬坐标系( PrjCoordSysType.PCS\_EARTH\_LONGITUDE\_LATITUDE),否则计算结果错误。
        
        :param geometry: 指定的需要计算经纬度面积的面对象。
        :type geometry: GeoRegion
        :param prj: 指定的投影坐标系类型
        :type prj: PrjCoordSys
        :return: 经纬度面积
        :rtype: float
    
    compute\_geodesic\_distance(points, major\_axis, flatten)
        计算测地线的长度。
        曲面上两点之间的短程线称为测地线。球面上的测地线即是大圆。
        测地线又称“大地线”或“短程线”,是地球椭球面上两点间的最短曲线。在大地线上,各点的主曲率方向均与该点上曲面法线相合。它在圆球面上为
        大圆弧, 在平面上就是直线。在大地测量中,通常用大地线来代替法截线,作为研究和计算椭球面上各种问题。
        
        测地线是在一个曲面上,每一点处测地曲率均为零的曲线。
        
        :param points:  构成测地线的经纬度坐标点串。
        :type points: list[Point2D] or tuple[Point2D]
        :param major\_axis: 测地线所在椭球体的长轴。
        :type major\_axis: float
        :param flatten:  测地线所在椭球体的扁率。
        :type flatten: float
        :return: 测地线的长度。
        :rtype: float
    
    compute\_geodesic\_line(start\_point, end\_point, prj, segment=18000)
        根据指定起始终止点计算测地线,返回结果线对象。
        
        :param Point2D start\_point: 输入的测地线起始点。
        :param Point2D end\_point: 输入的测地线终止点。
        :param prj:  空间参考坐标系。
        :type prj: PrjCoordSys
        :param int segment: 用来拟合半圆的弧段个数
        :return: 构造测地线成功,返回测地线对象,否则返回 None
        :rtype: GeoLine
    
    compute\_geodesic\_line2(start\_point, angle, distance, prj, segment=18000)
        根据指定起始点、方位角度以及距离计算测地线,返回结果线对象。
        
        :param Point2D start\_point:  输入的测地线起始点。
        :param float angle: 输入的测地线方位角。正负均可。
        :param float distance: 输入的测地线长度。单位为米。
        :param prj:  空间参考坐标系。
        :type prj: PrjCoordSys
        :param int segment: 用来拟合半圆的弧段个数
        :return: 构造测地线成功,返回测地线对象,否则返回 None
        :rtype: GeoLine
    
    compute\_parallel(geo\_line, distance)
        根据距离求已知折线的平行线,返回平行线。
        
        :param GeoLine geo\_line: 已知折线对象。
        :param float distance: 所求平行线间的距离。
        :return: 平行线。
        :rtype: GeoLine
    
    compute\_parallel2(point, start\_point, end\_point)
        求经过指定点与已知直线平行的直线。
        
        :param Point2D point: 直线外的任意一点。
        :param Point2D start\_point: 直线上的一点。
        :param Point2D end\_point:  直线上的另一点。
        :return: 平行线
        :rtype: GeoLine
    
    compute\_perpendicular(point, start\_point, end\_point)
        计算已知点到已知线的垂线。
        
        :param Point2D point:  已知一点。
        :param Point2D start\_point: 直线上的一点。
        :param Point2D end\_point: 直线上的另一点。
        :return: 点到直线的垂线
        :rtype: GeoLine
    
    compute\_perpendicular\_position(point, start\_point, end\_point)
        计算已知点到已知线的垂足。
        
        :param Point2D point: 已知一点。
        :param Point2D start\_point: 直线上的一点。
        :param Point2D end\_point: 直线上的另一点。
        :return: 点在直线上的垂足
        :rtype: GeoLine
    
    create\_datasource(conn\_info)
        根据指定的数据源连接信息,创建新的数据源。
        
        :param conn\_info: udb文件路径或数据源连接信息:
                          - 数据源连接信息。具体可以参考 :py:meth:`DatasourceConnectionInfo.make`
                          - 如果 conn\_info 为 str 时,可以为 ':memory:', udb 文件路径,udd 文件路径,dcf 文件路径,数据源连接信息的 xml 字符串
                          - 如果 conn\_info 为 dict,为  :py:meth:`DatasourceConnectionInfo.to\_dict` 的返回结果。
        :type conn\_info: str or dict or DatasourceConnectionInfo
        :return: 数据源对象
        :rtype: Datasource
    
    dataset\_dim2\_to\_dim3(source, z\_field\_or\_value, line\_to\_z\_field=None, saved\_fields=None, out\_data=None, out\_dataset\_name=None)
        将二维数据集转换为三维数据集,二维的点、线和面数据集将会分别转换为三维的点、线和面数据集。
        
        :param source: 二维数据集,支持点、线和面数据集
        :type source: DatasetVector or str
        :param z\_field\_or\_value: z 值的来源字段名称或指定的 z 值,如果为字段,则必须是数值型字段。
        :type z\_field\_or\_value: str or float
        :param line\_to\_z\_field: 当输入的是二维线数据集时,用于指定终止 z 值的字段名称,则 z\_field\_or\_value 则作为起始 z 值的字段的名称。
                                line\_to\_z\_field 必须为字段名称,不支持指定的 z 值。
        :type line\_to\_z\_field: str
        :param saved\_fields:  将要保留的字段名称
        :type saved\_fields: list[str] or str
        :param out\_data: 结果数据集所在的数据源
        :type out\_data: Datasource or DatasourceConnectionInfo or str
        :param str out\_dataset\_name: 结果数据集名称
        :return: 结果三数据集或数据集名称
        :rtype: DatasetVector or str
    
    dataset\_dim3\_to\_dim2(source, out\_z\_field='Z', saved\_fields=None, out\_data=None, out\_dataset\_name=None)
        将三维的点、线和面数据集转换为二维的点、线和面数据集。
        
        :param source: 三维数据集,支持三维点、线和面数据集
        :type source: DatasetVector or str
        :param out\_z\_field: 保留 Z 值的字段,如果为 None 或不合法,将会获取到一个有效的字段用于存储对象的 Z 值
        :type out\_z\_field: str
        :param saved\_fields: 需要保留的字段名称
        :type saved\_fields: list[str] or str
        :param out\_data: 结果数据集所在的数据源
        :type out\_data: Datasource or DatasourceConnectionInfo or str
        :param str out\_dataset\_name: 结果数据集名称
        :return: 结果二维数据集或数据集名称
        :rtype: DatasetVector or str
    
    dataset\_field\_to\_point(source, x\_field, y\_field, z\_field=None, saved\_fields=None, out\_data=None, out\_dataset\_name=None)
        根据数据集中字段,构造二维点数据集或三维点数据集。如果指定了有效的 z\_field将会得到三维点数据集,否则将会得到二维点数据集
        
        :param source: 提供数据的数据集,可以为属性表或点、线、面等数据集
        :type source: DatasetVector or str
        :param x\_field: x 坐标值的来源字段,必须有效。
        :type x\_field: str
        :param y\_field: y 坐标值的来源字段,必须有效。
        :type y\_field: str
        :param z\_field: z 坐标值的来源字段,可选。
        :type z\_field: str
        :param saved\_fields: 需要保留的字段名称。
        :type saved\_fields: list[str] or str
        :param out\_data: 结果数据集所在的数据源
        :type out\_data: Datasource or DatasourceConnectionInfo or str
        :param str out\_dataset\_name: 结果数据集名称
        :return: 二维点或三维点数据集或数据集名称
        :rtype: DatasetVector or str
    
    dataset\_field\_to\_text(source, field, saved\_fields=None, out\_data=None, out\_dataset\_name=None)
        将点数据集转换为文本数据集
        
        :param source: 输入的二维点数据集
        :type source: DatasetVector or str
        :param str field: 包含文本信息的字段,用于构造文本几何对象的文本信息。
        :param saved\_fields: 需要保留的字段名称
        :type saved\_fields: list[str] or str
        :param out\_data: 结果数据集所在的数据源
        :type out\_data: Datasource or DatasourceConnectionInfo or str
        :param str out\_dataset\_name: 结果数据集名称
        :return: 结果文本数据集或数据集名称
        :rtype: DatasetVector or str
    
    dataset\_line\_to\_point(source, mode='VERTEX', saved\_fields=None, out\_data=None, out\_dataset\_name=None)
        将线数据集转换为点数据集
        
        :param source: 二维线数据集
        :type source: DatasetVector or str
        :param mode: 线对象转换为点对象的方式
        :type mode: LineToPointMode or str
        :param saved\_fields: 需要保留的字段名称
        :type saved\_fields: list[str] or str
        :param out\_data: 结果数据集所在的数据源
        :type out\_data: Datasource or DatasourceConnectionInfo or str
        :param str out\_dataset\_name: 结果数据集名称
        :return: 结果二维点数据集或数据集名称
        :rtype: DatasetVector or str
    
    dataset\_line\_to\_region(source, saved\_fields=None, out\_data=None, out\_dataset\_name=None)
        将线数据集转换为面数据集。此方法将线对象直接转换为面对象,如果线对象不是首位相接,可能会转换失败。如果要将线数据集转换为面数据集,更
        可靠的方式的是拓扑构面 :py:func:`.topology\_build\_regions`
        
        :param source: 二维线数据集
        :type source: DatasetVector or str
        :param saved\_fields: 需要保留的字段名称
        :type saved\_fields: list[str] or str
        :param out\_data: 结果数据集所在的数据源信息
        :type out\_data: Datasource or DatasourceConnectionInfo or str
        :param str out\_dataset\_name: 结果数据集名称
        :return: 结果二维面数据集或数据集名称
        :rtype: DatasetVector or str
    
    dataset\_network\_to\_line(source, saved\_fields=None, out\_data=None, out\_dataset\_name=None)
        将二维网络数据集的转换为线数据集,网络数据集的 SmEdgeID、SmFNode和SmTNode 字段值将会存储在结果数据集的 EdgeID、FNode和TNode 字段
        中,如果  EdgeID、FNode 或 TNode 已被占用,会获取到一个有效的字段。
        
        :param source: 被转换的二维网络数据集
        :type source: DatasetVector or str
        :param saved\_fields: 需要保存的字段名称。
        :type saved\_fields: list[str] or str
        :param out\_data: 结果数据集所在的数据源信息
        :type out\_data: Datasource or DatasourceConnectionInfo or str
        :param str out\_dataset\_name: 结果数据集名称
        :return: 结果二维线数据集或数据集名称
        :rtype: DatasetVector or str
    
    dataset\_network\_to\_point(source, saved\_fields=None, out\_data=None, out\_dataset\_name=None)
        将二维网络数据集的点子数据集转换为点数据集,网络数据集的 SmNodeID 字段值将会存储在结果数据集的 NodeID 字段中,如果 NodeID 已被占用,会获取到一个有效的字段。
        
        :param source: 被转换的二维网络数据集
        :type source: DatasetVector or str
        :param saved\_fields: 需要保存的字段名称。
        :type saved\_fields: list[str] or str
        :param out\_data: 结果数据集所在的数据源信息
        :type out\_data: Datasource or DatasourceConnectionInfo or str
        :param str out\_dataset\_name: 结果数据集名称
        :return: 结果二维点数据集或数据集名称
        :rtype: DatasetVector or str
    
    dataset\_point\_to\_line(source, group\_fields=None, order\_fields=None, field\_stats=None, out\_data=None, out\_dataset\_name=None)
        将二维点数据集中点对象,根据分组字段进行分组构造线对象,返回一个二维线数据集
        
        :param source: 二维点数据集
        :type source: DatasetVector or str
        :param group\_fields: 二维点数据集中用于分组的字段名称,只有分组字段名称的字段值都相等时才会将点连接成线。
        :type group\_fields: list[str]
        :param order\_fields: 排序字段,同一分组内的点,按照排序字段的字段值的升序进行排序,再连接成线。如果为 None,则默认使用 SmID 字段进行排序。
        :type order\_fields: list[str] or str
        :param field\_stats: 字段统计信息,同一分组内的点属性进行字段统计。为一个list,list中每个元素为一个 tuple,tuple的大小为2,tuple的第一个元素为被统计的字段名称,tuple的第二个元素为统计类型。
                            注意,不支持 :py:attr:`AttributeStatisticsMode.MAXINTERSECTAREA`
        :type field\_stats: list[tuple(str,AttributeStatisticsMode)] or list[tuple(str,str)] or str
        :param out\_data: 结果数据集所在的数据源
        :type out\_data: Datasource or DatasourceConnectionInfo or str
        :param str out\_dataset\_name: 结果数据集名称
        :return: 二维点数据集或数据集名称
        :rtype: DatasetVector or str
    
    dataset\_region\_to\_line(source, saved\_fields=None, out\_data=None, out\_dataset\_name=None)
        将二维面对象转换为线数据集。此方法会将每个点对象直接转换为线对象,如果需要提取不包含重复线的线数据集,可以使用 :py:func:`.pickup\_border`
        
        :param source: 二维面数据集
        :type source: DatasetVector or str
        :param saved\_fields: 需要保留的字段名称
        :type saved\_fields: list[str] or str
        :param out\_data: 结果数据集所在的数据源
        :type out\_data: Datasource or DatasourceConnectionInfo or str
        :param str out\_dataset\_name: 结果数据集名称
        :return: 结果线数据集或数据集名称
        :rtype: DatasetVector or str
    
    dataset\_region\_to\_point(source, mode='INNER\_POINT', saved\_fields=None, out\_data=None, out\_dataset\_name=None)
        将二维面数据集转换为点数据集
        
        :param source: 二维面数据集
        :type source: DatasetVector or str
        :param mode: 面对象转换为点对象的方式
        :type mode: RegionToPointMode or str
        :param saved\_fields: 需要保留的字段名称
        :type saved\_fields: list[str] or str
        :param out\_data: 结果数据集所在的数据源信息
        :type out\_data: Datasource or DatasourceConnectionInfo or str
        :param str out\_dataset\_name: 结果数据集名称
        :return: 结果二维点数据集或数据集名称
        :rtype: DatasetVector or str
    
    dataset\_text\_to\_field(source, out\_field='Text')
        将二维文本数据集的文本信息存储到字段中。文本对象的文本信息将会存储在指定的 out\_field 字段中。
        
        :param source: 输入的二维文本数据集。
        :type source: DatasetVector or str
        :param out\_field: 存储文本信息的字段名称。如果 out\_field 指定的字段名称已经存在,必须为文本型字段。如果不存在,将会新建一个文本型字段。
        :type out\_field: str
        :return: 成功返回 True,否则返回  False。
        :rtype: bool
    
    dataset\_text\_to\_point(source, out\_field='Text', saved\_fields=None, out\_data=None, out\_dataset\_name=None)
        将二维文本数据集转换为点数据集,文本对象的文本信息将会存储在指定的 out\_field 字段中
        
        :param source: 输入的二维文本数据集
        :type source: DatasetVector or str
        :param str out\_field: 存储文本信息的字段名称。如果 out\_field 指定的字段名称已经存在,必须为文本型字段。如果不存在,将会新建一个文本型字段。
        :param saved\_fields:  需要保留的字段名称。
        :type saved\_fields: list[str] or str
        :param out\_data: 结果数据集所在的数据源
        :type out\_data: Datasource or DatasourceConnectionInfo or str
        :param str out\_dataset\_name: 结果数据集名称
        :return: 二维点数据集或数据集名称
        :rtype: DatasetVector or str
    
    erase(geometry, erase\_geometry)
        在被操作对象上擦除掉与操作对象相重合的部分。
        注意:
        
        * 如果对象全部被擦除了,则返回 None;
        * 操作几何对象定义了擦除区域,凡是落在操作几何对象区域内的被操作几何对象都将被去除,而落在区域外的特征要素都将被输出为结果几何对象,与 Clip 运算相反;
        * 该操作适合的几何对象类型:
        
            * 操作几何对象:面;
            * 被操作几何对象:点、线、面。
        
        .. image:: ../image/Geometrist\_Erase.png
        
        :param geometry:  被操作几何对象,支持点、线、面对象类型
        :type geometry: GeoPoint or GeoLine or GeoRegion
        :param erase\_geometry: 操作几何对象,必须为面对象类型。
        :type erase\_geometry: GeoRegion or Rectangle
        :return: 擦除操作后的几何对象。
        :rtype: Geometry
    
    georegion\_to\_center\_line(source\_region, pnt\_from=None, pnt\_to=None)
        提取面对象的中心线,一般用于提取河流的中心线。
        该方法用于提取面对象的中心线。如果面包含岛洞,提取时会绕过岛洞,采用最短路径绕过。如下图。
        
        .. image:: ../image/RegionToCenterLine\_1.png
        
        如果面对象不是简单的长条形,而是具有分叉结构,则提取的中心线是最长的一段。如下图所示。
        
        .. image:: ../image/RegionToCenterLine\_2.png
        
        如果提取的不是期望的中心线,可以通过指定起点和终点,提取面对象的中心线,一般用于提取河流的中心线。尤其是河流干流的中心线,
        并且可以指定提取的起点和终点。如果面包含岛洞,提取时会绕过岛洞,采用的是最短路径绕过。如下图。
        
        .. image:: ../image/RegionToCenterLine\_3.png
        
        pnt\_from 参数和 pnt\_to 参数所指定起点和终点,是作为提取的参考点,也就是说,系统提取的中心线可能不会严格从指定的起点出发,到指定的终点结束。系统一般会在指定的起点和终点的附近,找到一个较近的点作为提取的起点或终点。
        同时需要注意:
        
            * 如果将起点和终点指定为相同的点,即等同于不指定提取的起点和终点,则提取的是面对象的最长的一条中心线。
            * 如果指定的起点或终点在面对象的外面,则提取失败。
        
        :param GeoRegion source\_region: 指定的待提取中心线的面对象。
        :param Point2D pnt\_from: 指定的提取中心线的起点。
        :param Point2D pnt\_to: 指定的提取中心线的终点。
        :return: 提取的中心线,是一个二维线对象
        :rtype: GeoLine
    
    get\_datasource(item)
        获取指定的数据源对象。
        
        具体参考  :py:meth:`Workspace.get\_datasource`
        
        :param item: 数据源的别名或序号
        :type item:  str or int
        :return: 数据源对象
        :rtype: Datasource
    
    has\_area\_intersection(geo\_search, geo\_target, tolerance=None)
        判断对象是否面积相交,查询对象和目标对象至少有一个对象是面对象,相交的结果不包括仅接触的情形。支持点、线、面和文本对象。
        
        :param Geometry geo\_search: 查询对象
        :param Geometry geo\_target: 目标对象
        :param float tolerance: 节点容限
        :return: 两对象面积相交返回 True,否则为 False
        :rtype: bool
    
    has\_common\_line(geo\_search, geo\_target)
        判断搜索几何对象是否与被搜索几何对象有公共线段。有公共线段返回 True。
        
        .. image:: ../image/Geometrist\_HasCommonLine.png
        
        :param geo\_search: 搜索几何对象,只支持线、面类型。
        :type geo\_search: GeoLine or GeoRegion
        :param geo\_target: 被搜索几何对象,只支持线、面类型。
        :type geo\_target:  GeoLine or GeoRegion
        :return: 搜索几何对象与被搜索几何对象有公共线段返回 True;否则返回 False。
        :rtype: bool
    
    has\_common\_point(geo\_search, geo\_target)
        判断搜索几何对象是否与被搜索几何对象有共同节点。有共同节点返回 true。
        
        .. image:: ../image/Geometrist\_HasCommonPoint.png
        
        :param geo\_search: 搜索几何对象,支持点、线、面类型。
        :type geo\_search: Geometry
        :param geo\_target: 被搜索几何对象,支持点、线、面类型。
        :type geo\_target: Geometry
        :return: 搜索几何对象与被搜索几何对象有共同节点返回 true;否则返回 false。
        :rtype: bool
    
    has\_cross(geo\_search, geo\_target)
        判断搜索几何对象是否穿越被搜索几何对象。穿越则返回 True。
        注意,如果两个几何对象存在穿越关系则:
        
        * 搜索几何对象内部与被搜索几何对象的内部的交集不为空且搜索几何对象的内部与被搜索几何对象的外部的交集不为空。
        * 被搜索几何对象为线时,搜索几何对象内部与被搜索几何对象的内部的交集不为空但是边界交集为空;
        * 该关系适合的几何对象类型:
        
            * 搜索几何对象:线;
            * 被搜索几何对象:线、面。
        
        .. image:: ../image/Geometrist\_HasCross.png
        
        :param GeoLine geo\_search: 搜索几何对象,只支持线类型。
        :param geo\_target: 被搜索几何对象,支持线、面类型。
        :type geo\_target: GeoLine or GeoRegion or Rectangle
        :return: 搜索几何对象穿越被搜索对象返回 True;否则返回 False。
        :rtype: bool
    
    has\_hollow(geometry)
        判断指定的面对象是否包含有洞类型的子对象
        
        :param geometry: 待判断的面对象,目前只支持二维面对象
        :type geometry: GeoRegion
        :return: 面对象是否含有洞类型的子对象,包含则返回 True,否则返回 False
        :rtype: bool
    
    has\_intersection(geo\_search, geo\_target)
        判断被搜索几何对象与搜索几何对象是否有面积相交。相交返回 true。
        注意:
        
        * 被搜索几何对象和搜索几何对象必须有一个为面对象;
        * 该关系适合的几何对象类型:
        
            * 搜索几何对象:点、线、面;
            * 被搜索几何对象:点、线、面。
        
        .. image:: ../image/Geometrist\_HasIntersection.png
        
        :param Geometry geo\_search: 查询对象
        :param Geometry geo\_target: 目标对象
        :return: 两对象面积相交返回 True,否则为 False
        :rtype: bool
    
    has\_overlap(geo\_search, geo\_target)
        判断被搜索几何对象是否与搜索几何对象部分重叠。有部分重叠则返回 true。
        注意:
        
        * 点与任何一种几何对象都不存在部分重叠的情况;
        * 被搜索几何对象与搜索几何对象的维数要求相同,即只可以是线查询线或者面查询面;
        * 该关系适合的几何对象类型:
        
            * 搜索几何对象:线、面;
            * 被搜索几何对象:线、面。
        
        .. image:: ../image/Geometrist\_HasOverlap.png
        
        :param geo\_search: 搜索几何对象,只支持线、面类型。
        :type geo\_search: GeoLine or GeoRegion or Rectangle
        :param geo\_target: 被搜索几何对象,只支持线、面类型
        :type geo\_target: GeoLine or GeoRegion or Rectangle
        :return: 被搜索几何对象与搜索几何对象部分重叠返回 True;否则返回 False
        :rtype: bool
    
    has\_touch(geo\_search, geo\_target)
        判断被搜索几何对象的边界是否与搜索几何对象的边界相触。相触时搜索几何对象和被搜索几何对象的内部交集为空。
        注意:
        
        * 点与点不存在边界接触的情况;
        * 该关系适合的几何对象类型:
        
            * 搜索几何对象:点、线、面;
            * 被搜索几何对象:点、线、面。
        
        .. image:: ../image/Geometrist\_HasTouch.png
        
        :param geo\_search: 搜索几何对象。
        :type geo\_search: Geometry
        :param geo\_target: 被搜索几何对象。
        :type geo\_target: Geometry
        :return: 被搜索几何对象的边界与搜索几何对象边界相触返回 True;否则返回 False。
        :rtype: bool
    
    identity(geometry, identity\_geometry)
        对被操作对象进行同一操作。即操作执行后,被操作几何对象包含来自操作几何对象的几何形状。
        注意:
        
        * 同一运算就是操作几何对象与被操作几何对象先求交,然后求交结果再与被操作几何对象求并的运算。
        
            * 如果被操作几何对象为点类型,则结果几何对象为被操作几何对象;
            * 如果被操作几何对象为线类型,则结果几何对象为被操作几何对象,但是与操作几何对象相交的部分将被打断;
            * 如果被操作几何对象为面类型,则结果几何对象保留以被操作几何对象为控制边界之内的所有多边形,并且把与操作几何对象相交的地方分割成多个对象。
        
        * 该操作适合的几何对象类型:
            操作几何对象:面;
            被操作几何对象:点、线、面。
        
        .. image:: ../image/Geometrist\_Identity.png
        
        :param geometry: 被操作几何对象,支持点、线、面对象。
        :type geometry: GeoPoint or GeoLine or GeoRegion
        :param identity\_geometry: 操作几何对象,必须为面对象。
        :type identity\_geometry: GeoRegion or Rectangle
        :return: 同一操作后的几何对象
        :rtype: Geometry
    
    intersect(geometry1, geometry2, tolerance=None)
        对两个几何对象求交,返回两个几何对象的交集。目前仅支持线线求交、面面求交。
        目前仅支持面面求交和线线求交,如下图示所示:
        
        .. image:: ../image/Geometrist\_Intersect.png
        
        注意,如果两对象有多个相离的公共部分,求交的结果将是一个复杂对象。
        
        :param geometry1: 进行求交运算的第一个几何对象,支持线、面类型。
        :type geometry1: GeoLine or GeoRegion
        :param geometry2:  进行求交运算的第二个几何对象,支持线、面类型。
        :type geometry2: GeoLine or GeoRegion
        :param tolerance: 节点容限,目前仅支持线线求交。
        :type tolerance: float
        :return: 求交操作后的几何对象。
        :rtype: Geometry
    
    intersect\_line(start\_point1, end\_point1, start\_point2, end\_point2, is\_extended)
        返回两条线段(直线)的交点。
        
        :param Point2D start\_point1:  第一条线的起点。
        :param Point2D end\_point1: 第一条线的终点。
        :param Point2D start\_point2: 第二条线的起点。
        :param Point2D end\_point2: 第二条线的终点。
        :param bool is\_extended: 是否将线段进行延长计算,如果为 True,就按直线计算,否则按线段计算。
        :return: 两条线段(直线)的交点。
        :rtype: Point2D
    
    intersect\_polyline(points1, points2)
        返回两条折线的交点。
        
        :param points1:  构成第一条折线的点串。
        :type points1: list[Point2D] or tuple[Point2D]
        :param points2: 构成第二条折线的点串。
        :type points2: list[Point2D] or tuple[Point2D]
        :return: 点串构成的折线的交点。
        :rtype: list[Point2D]
    
    is\_disjointed(geo\_search, geo\_target)
        判断被搜索几何对象是否与搜索几何对象分离。分离返回 true。
        注意:
        
        * 搜索几何对象和被搜索几何对象分离,即无任何交集;
        * 该关系适合的几何对象类型:
        
             * 搜索几何对象:点、线、面;
             * 被搜索几何对象:点、线、面。
        
        .. image:: ../image/Geometrist\_IsDisjointed.png
        
        :param geo\_search:  搜索几何对象,支持点、线、面类型。
        :type geo\_search: Geometry
        :param geo\_target: 被搜索几何对象,支持点、线、面类型。
        :type geo\_target: Geometry
        :return: 两个几何对象分离返回 True;否则返回 False
        :rtype: bool
    
    is\_identical(geo\_search, geo\_target, tolerance=None)
        判断被搜索几何对象是否与搜索几何对象完全相等。即几何对象完全重合、对象节点数目相等,正序或逆序对应的坐标值相等。
        注意:
        
        * 被搜索几何对象与搜索几何对象的类型必须相同;
        * 该关系适合的几何对象类型:
        
             * 搜索几何对象:点、线、面;
             * 被搜索几何对象:点、线、面。
        
        .. image:: ../image/Geometrist\_IsIdentical.png
        
        :param geo\_search: 搜索几何对象,支持点、线、面类型
        :type geo\_search: Geometry
        :param geo\_target: 被搜索几何对象,支持点、线、面类型。
        :type geo\_target: Geometry
        :param float tolerance: 节点容限
        :return: 两个对象完全相等返回 True;否则返回 False
        :rtype: bool
    
    is\_left(point, start\_point, end\_point)
        判断点是否在线的左侧。
        
        :param Point2D point: 指定的待判断的点
        :param Point2D start\_point: 指定的直线上的一点
        :param Point2D end\_point: 指定的直线上的另一点。
        :return: 如果点在线的左侧,返回 True,否则返回 False
        :rtype: bool
    
    is\_on\_same\_side(point1, point2, start\_point, end\_point)
        判断两点是否在线的同一侧。
        
        :param Point2D point1: 指定的待判断的一个点
        :param Point2D point2: 指定的待判断的另一个点
        :param Point2D start\_point: 指定的直线上的一点。
        :param Point2D end\_point: 指定的直线上的另一点。
        :return: 如果点在线的同一侧,返回 True,否则返回 False
        :rtype: bool
    
    is\_parallel(start\_point1, end\_point1, start\_point2, end\_point2)
        判断两条线是否平行。
        
        :param Point2D start\_point1: 第一条线的起点。
        :param Point2D end\_point1: 第一条线的终点。
        :param Point2D start\_point2:  第二条线的起点。
        :param Point2D end\_point2:  第二条线的终点。
        :return: 平行返回 True;否则返回 False
        :rtype: bool
    
    is\_perpendicular(start\_point1, end\_point1, start\_point2, end\_point2)
        判断两条直线是否垂直。
        
        :param Point2D start\_point1: 第一条线的起点。
        :param Point2D end\_point1: 第一条线的终点。
        :param Point2D start\_point2:  第二条线的起点。
        :param Point2D end\_point2:  第二条线的终点。
        :return: 垂直返回 True;否则返回 False。
        :rtype: bool
    
    is\_point\_on\_line(point, start\_point, end\_point, is\_extended=True)
        判断已知点是否在已知线段(直线)上,点在线上返回 True, 否则返回 False。
        
        :param Point2D point: 已知点
        :param Point2D start\_point: 已知线段的起点
        :param Point2D end\_point:  已知线段的终点
        :param bool is\_extended: 是否将线段进行延长计算,如果为 True,就按直线计算,否则按线段计算
        :return: 点在线上返回 True;否则返回 False
        :rtype: bool
    
    is\_project\_on\_line\_segment(point, start\_point, end\_point)
        判断已知点到已知线段的垂足是否在该线段上,如果在,返回 True, 否则返回 False。
        
        :param Point2D point: 已知点
        :param Point2D start\_point: 已知线段的起点
        :param Point2D end\_point:  已知线段的终点
        :return: 点与线段的垂足是否在线段上。如果在,返回 True,否则返回 False。
        :rtype: bool
    
    is\_right(point, start\_point, end\_point)
        判断点是否在线的右侧。
        
        :param Point2D point: 指定的待判断的点
        :param Point2D start\_point: 指定的直线上的一点
        :param Point2D end\_point: 指定的直线上的另一点。
        :return: 如果点在线的右侧,返回 True,否则返回 False
        :rtype: bool
    
    is\_within(geo\_search, geo\_target, tolerance=None)
        判断搜索几何对象是否在被搜索几何对象内。如果在则返回 True。
        注意:
        
        * 线查询点,面查询线或面查询点都不存在 Within 情况;
        * 与 can\_contain 是逆运算;
        *  该关系适合的几何对象类型:
        
            * 搜索几何对象:点、线、面;
            * 被搜索几何对象:点、线、面。
        
        .. image:: ../image/Geometrist\_IsWithin.png
        
        :param geo\_search:  搜索几何对象,支持点、线、面类型。
        :type geo\_search: Geometry
        :param geo\_target: 被搜索几何对象,支持点、线、面类型
        :type geo\_target: Geometry
        :param float tolerance: 节点容限
        :return: 搜索几何对象在被搜索几何对象内返回 True;否则返回 False
        :rtype: bool
    
    list\_datasources()
        返回当前工作空间下的所有数据源对象。
        
        具体参考  :py:attr:`Workspace.datasources`
        
        :return: 当前工作空间下的所有数据源对象
        :rtype: list[Datasource]
    
    nearest\_point\_to\_vertex(vertex, geometry)
        从几何对象上找一点与给定的点距离最近。
        
        :param vertex: 指定的点
        :type vertex: Point2D
        :param geometry: 指定的几何对象
        :type geometry: Rectangle or GeoLine or GeoRegion
        :return: 几何对象上与指定点距离最近的一点。
        :rtype: Point2D
    
    open\_datasource(conn\_info, is\_get\_existed=True)
        根据数据源连接信息打开数据源。如果设置的连接信息是UDB类型数据源,或者 is\_get\_existed 为 True,如果工作空间中已经存在对应的数据源,则
        会直接返回。不支持直接打开内存数据源,要使用内存数据源,需要使用 :py:meth:`create\_datasource` 创建内存数据源。
        
        具体参考  :py:meth:`Workspace.open\_datasource`
        
        :param conn\_info: udb文件路径或数据源连接信息:
                              - 数据源连接信息。具体可以参考 :py:meth:`DatasourceConnectionInfo.make`
                              - 如果 conn\_info 为 str 时,可以为 ':memory:', udb 文件路径,udd 文件路径,dcf 文件路径,数据源连接信息的 xml 字符串
                              - 如果 conn\_info 为 dict,为  :py:meth:`DatasourceConnectionInfo.to\_dict` 的返回结果。
        :type conn\_info: str or dict or DatasourceConnectionInfo
        :param bool is\_get\_existed: is\_get\_existed 为 True,如果工作空间中已经存在对应的数据源,则会直接返回。为 false 时,则会打开新的数据源。对于 UDB 类型数据源,无论 is\_get\_existed 为 True 还是 False,都会优先返回工作空间中的数据源。判断 DatasourceConnectionInfo 是否与工作空间中的数据源是同一个数据源,可以查看 :py:meth:`DatasourceConnectionInfo.is\_same`
        :return: 数据源对象
        :rtype: Datasource
    
    orthogonal\_polygon\_fitting(geometry, width\_threshold, height\_threshold)
        面对象的直角多边形拟合
        如果一串连续的节点到最小面积外接矩形的下界的距离大于 height\_threshold,
        且节点的总宽度大于 width\_threshold,则对连续节点进行拟合。
        
        :param geometry: 待直角化的多边形对象,只能是简单面对象
        :type geometry: GeoRegion or Rectangle
        :param float width\_threshold: 点到最小面积外接矩形的左右边界的阈值
        :param float height\_threshold: 点到最小面积外接矩形的上下边界的阈值
        :return: 进行直角化的多边形对象,如果失败,返回 None
        :rtype: GeoRegion
    
    point\_to\_segment\_distance(point, start\_point, end\_point)
        计算已知点到已知线段的距离。
        
        :param Point2D point:  已知点。
        :param Point2D start\_point: 已知线段的起点。
        :param Point2D end\_point: 已知线段的终点。
        :return: 点到线段的距离。如果点到线段的垂足不在线段上,则返回点到线段较近的端点的距离。
        :rtype: float
    
    resample(geometry, distance, resample\_type='RTBEND')
        对几何对象进行重采样。
        对几何对象重采样是按照一定规则剔除一些节点,以达到对数据进行简化的目的(如下图所示),其结果可能由于使用不同的重采样方法而不同。
        SuperMap 提供两种方法对几何对象进行重采样,分别为光栏法和道格拉斯-普克法。有关这两种方法的详细介绍,请参见 :py:class:`VectorResampleType` 类。
        
        .. image:: ../image/VectorResample.png
        
        
        :param geometry: 指定的要进行重采样的几何对象。支持线对象和面对象。
        :type geometry: GeoLine or GeoRegion
        :param distance:  指定的重采样容限。
        :type distance: float
        :param resample\_type: 指定的重采样方法。
        :type resample\_type: VectorResampleType or str
        :return: 重采样后的几何对象。
        :rtype: GeoLine or GeoRegion
    
    smooth(points, smoothness)
        对指定的点串对象进行光滑处理
        
        有关光滑的更多内容,可以参考 :py:meth:`jsupepry.analyst.smooth` 方法的介绍。
        
        :param points: 需要进行光滑处理的点串。
        :type points: list[Point2D] or tuple[Point2D] or GeoLine or GeoRegion
        :param smoothness: 光滑系数。有效范围为大于等于2,设置为小于2的值会抛出异常。光滑系数越大,线对象或面对象边界的节点数越多,也就越光滑。 建议取值范围为[2,10]。
        :type smoothness: int
        :return: 光滑处理结果点串。
        :rtype: list[Point2D] or GeoLine or GeoRegion
    
    split\_line(source\_line, split\_geometry, tolerance=1e-10)
        使用点、线或面对象对线对象进行分割(打断)。
        该方法可用于使用点、线、面对象对线对象进行打断或分割。下面以一个简单线对象对这三种情况进行说明:
        
        * 点对象打断线对象。使用点对象对线对象进行打断,原线对象在点对象位置打断为两个线对象。如下图所示,使用点(黑色)对线(蓝色)进行打断,结果为两个线对象(红色线和绿色线 )。
        
        .. image:: ../image/PointSplitLine.png
        
        * 使用点、线或面对象对线对象进行分割(打断)。 该方法可用于使用点、线、面对象对线对象进行打断或分割。下面以一个简单线对象
          对这三种情况进行说明
        
          * 当分割线为线段时,操作线将会在其与分割线的交点处被分割为两个线对象。如下图所示,图中黑色线为分割线,分割后原线对象被分为两个线对象(红色线和绿色线 )。
        
          .. image:: ../image/LineSplitLine\_1.png
        
          * 当分割线为折线时,可能与操作线有多个交点,此时会在所有交点处将操作线打断,然后按顺序将位于奇数和偶数次序的线段分别合并,产生两个线对象。也就是说, 使用
            折线分割线时,可能会产生复杂线对象。下图展示的就是这种情况,分割后,红色的线和绿色的线分别为一个复杂线对象。
        
          .. image:: ../image/LineSplitLine\_2.png
        
        
        * 面对象分割线对象。面对象分割线对象与线分割线类似,会在分割面和操作线的所有交点处将操作线打断,然后分别将位于奇数和偶数位置的线合并,产生两个线对象。
          这种情况会产生至少一个复杂线对象。下图中,面对象(浅橙色)将线对象分割为红色和绿色两个复杂线对象。
        
        .. image:: ../image/RegionSplitLine.png
        
        
        注意:
        
        1. 如果被分割的线对象为复杂对象,那么如果分割线经过子对象,则会将该子对象分割为两个线对象,因此,分割复杂线对象可能产生多个线对象。
        2. 用于分割的线对象或者面对象如果有自相交,分割不会失败,但分割的结果可能不正确。因此,应尽量使用没有自相交的线或面对象来分割线。
        
        
        :param source\_line: 待分割(打断)的线对象
        :type source\_line: GeoLine
        :param split\_geometry: 用于分割(打断)线对象的对象,支持点、线、面对象。
        :type split\_geometry: GeoPoint or GeoRegion or GeoLine or Rectangle or Point2D
        :param tolerance: 指定的容限,用于判断点对象是否在线上,若点到线的垂足距离大于该容限值,则认为用于打断的点对象无效,从而不执行打断。
        :type tolerance: float
        :return: 分割后的线对象数组。
        :rtype: list[GeoLine]
    
    split\_region(source\_region, split\_geometry)
        用线或面几何对象分割面几何对象。
        注意:参数中的分割对象与被分割对象必须至少有两个交点,否则的话会分割失败。
        
        :param source\_region: 被分割的面对象。
        :type source\_region: GeoRegion or Rectangle
        :param split\_geometry:  用于分割的几何对象,可以是线或面几何对象。
        :type split\_geometry: GeoLine or GeoRegion or Rectangle
        :return:  返回分割后的面对象,正确分隔后会得到两个面对象。
        :rtype: tuple[GeoRegion]
    
    union(geometry1, geometry2)
        对两个对象进行合并操作。进行合并后,两个面对象在相交处被多边形分割。
        注意:
        
        * 进行求并运算的两个几何对象必须是同类型的,目前版本只支持面、线类型的合并。
        * 该操作适合的几何对象类型:
        
            * 操作几何对象:面、线;
            * 被操作几何对象:面、线。
        
        .. image:: ../image/Geometrist\_Union.png
        
        :param geometry1:  被操作几何对象。
        :type geometry1: GeoLine or GeoRegion
        :param geometry2: 操作几何对象。
        :type geometry2: GeoLine or GeoRegion
        :return: 合并操作后的几何对象。只支持生成简单线对象。
        :rtype: Geometry
    
    update(geometry, update\_geometry)
        对被操作对象进行更新操作。用操作几何对象替换与被操作几何对象的重合部分,是一个先擦除后粘贴的过程。操作对象和被操作对象必须都是面对象。
        
        .. image:: ../image/Geometrist\_Update.png
        
        :param geometry: 被操作几何对象,即被更新的几何对象,必须为面对象。
        :type geometry: GeoRegion  or Rectangle
        :param update\_geometry: 操作几何对象,用于进行更新运算的几何对象,必须为面对象。
        :type update\_geometry:  GeoRegion  or Rectangle
        :return: 更新操作后的几何对象。
        :rtype: GeoRegion
    
    xor(geometry1, geometry2)
        对两个对象进行异或运算。即对于每一个被操作几何对象,去掉其与操作几何对象相交的部分,而保留剩下的部分。
        进行异或运算的两个几何对象必须是同类型的,只支持面面。
        
        .. image:: ../image/Geometrist\_XOR.png
        
        :param geometry1:  被操作几何对象,只支持面类型。
        :type geometry1: GeoRegion  or Rectangle
        :param geometry2:  操作几何对象,只支持面类型。
        :type geometry2: GeoRegion or Rectangle
        :return: 进行异或运算的结果几何对象。
        :rtype: GeoRegion

DATA
    \_\_all\_\_ = ['DatasourceConnectionInfo', 'Datasource', 'Dataset', 'Datas{\ldots}

FILE
    /opt/conda/lib/python3.6/site-packages/iobjectspy/data.py



    \end{Verbatim}


    % Add a bibliography block to the postdoc
    
    
    
    \end{document}
